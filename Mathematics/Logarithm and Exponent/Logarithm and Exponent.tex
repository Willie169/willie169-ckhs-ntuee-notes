\documentclass[a4paper,12pt]{article}
\setcounter{secnumdepth}{5}
\setcounter{tocdepth}{3}
\input{/usr/share/latex-toolkit/template.tex}
\begin{document}
\title{Logarithm and Exponent}
\author{沈威宇}
\date{\temtoday}
\titletocdoc
\section{Logarithm (對數) and Exponent (指數)}
\ssc{Principal branch (主分支)}
In the following definitions, $\arg(x\in\mathbb{C}_{\neq 0})$ represents the argument (輻角) of $x$. If we replace $\arg(x)$ with the principal argument (輻角主值) of $x$, we get the definition of the principal branch of them.
\sssc{Positive integer exponent (正整數指數)}
The exponent of $a\in\mathbb{R}$ to the power of $n\in\mathbb{N}$, denoted as $a^n$, is defined as
\[a^n\coloneq\prod_{k=1}^na.\]
\sssc{Natural logarithm (自然對數) in the real numbers}
The natural logarithm of $x$, denoted as $\ln(x)$ or $\log_e(x)$, is defined in the real numbers as:
\[\ln(x)\coloneq\int_1^x\frac{1}{t}\,\mathrm{d}t,\quad x\in\mathbb{R}_{>0}.\]
\sssc{Natural logarithm in the complex numbers}
The natural logarithm of $x$, denoted as $\ln(x)$ or $\log_e(x)$, is defined in the complex numbers as:
\[\ln|x|+i\arg(x),\quad x\in\mathbb{C}_{\neq 0}.\]
\sssc{Logarithm (對數) in real numbers}
The logarithm of $x$, called argument (真數), to the $a$, called base (底數), denoted as $\log_a(x)$, is defined in real numbers as
\[\log_a(x)\coloneq\frac{\ln(x)}{\ln(a)},\quad a,x\in\mathbb{R}_{\neq 0}\land a\neq 1.\]
\sssc{Logarithm in complex numbers}
The logarithm of $x$, called argument, to the $a$, called base, denoted as $\log_a(x)$, is defined in complex numbers as
\[\log_a(x)\coloneq\frac{\ln(x)}{\ln(a)},\quad a,x\in\mathbb{C}_{\neq 0}\land a\neq 1.\]
\sssc{Common logarithm (常用對數)}
$\log_{10}(x)$ is called the common logarithm of $x$, also denoted as $\log(x)$.
\sssc{Exponent (指數) in real numbers}
The exponent of $a$, called the base (底數), to the exponent (指數) or power (冪次) of $n$, denoted as $a^n$, is defined in real numbers as
\[a^n\coloneq\begin{cases}
e^{n\ln(a)},\quad & a\in\mathbb{R}_{>0}\land n\in\mathbb{R},\\
-|a|^n,\quad & a\in\mathbb{R}_{<0}\land \exists p,q\in\mathbb{N}\tx{\ s.t.\ } n=\frac{2p-1}{2q-1},\\
|a|^n,\quad & a\in\mathbb{R}_{<0}\land \exists p,q\in\mathbb{N}\tx{\ s.t.\ } n=\frac{2p-2}{2q-1},\\
0,\quad & a=0\land n\in\mathbb{R}_{>0}.
\end{cases}\]
\sssc{Exponent in complex numbers} 
The exponent of $a$, called the base, to the exponent or power of $n$, denoted as $a^n$, is defined in complex numbers as
\[a^n\coloneq\begin{cases}
e^{n\ln(a)},\quad & a\in\mathbb{C}_{\neq 0}\land n\in\mathbb{C},\\
0,\quad & a=0\land n\in\mathbb{R}_{>0}.
\end{cases}\]
\sssc{Euler's number (歐拉數/尤拉數)}
The Euler's number, denoted as $e$, is defined as
\[e=\lim_{n\to\infty}\left(1+\frac{1}{n}\right)^n.\]

Some other equivalent definitions include:
\[e=\sum_{n=0}^{\infty}\frac{1}{n!}.\]
\begin{proof}
\[\ba
\left(1+\frac{1}{n}\right)^n&=\sum_{k=0}^n\binom{n}{k}\frac{1}{n^k}\\
&=\sum_{k=0}^n\frac{\prod_{j=0}^{k-1}(n-j)}{n^kk!}\\
    &=\sum_{k=0}^n\prod_{j=0}^{k-1}\qty(1-\frac{j}{n})\frac{1}{k!}
\ea\]
For fixed $k$:
\[\lim_{n\to\infty}\prod_{j=0}^{k-1}\qty(1-\frac{j}{n})=1\]
\[\lim_{n\to\infty}\sum_{k=0}^n\prod_{j=0}^{k-1}\qty(1-\frac{j}{n})\frac{1}{k!}=\sum_{k=0}^\infty \frac{1}{k!}.\]
\end{proof}
\[e^x\coloneq \text{\ Solution of\ }f'(x)=f(x), \quad f(0)=1.\]
\begin{proof}
    By separation of variables:
    \[\dv{y}{x}=y\]
    \[\frac{\mathrm{d}y}{y}=\dd{x}.\]
    \[\ln|y|=x+C\]
    \[y(x)=Ce^x\]
    \[C=1\]
    \[y(x)=e^x.\]
\end{proof}
\[\int_1^e\frac{\mathrm{d}x}{x}=1.\]
\begin{proof}
\[f(x)\coloneq\sum_{n=0}^{\infty}\frac{x^n}{n!}.\]
    \[f'(x)=\sum_{n=1}^{\infty}\frac{x^{n-1}}{(n-1)!}=f(x).\]
    By uniqueness of solution of $f'(x)=f(x)$ and $f(0)=1$,
    \[f(x)=e^x.\]
Threrefore,
    \[e=f(1).\]
    By the definition of natural logarithm:
    \[\ln y=\int_1^y\frac{\mathrm{d}x}{x}.\]
    \[\int_1^e\frac{\mathrm{d}x}{x}=1.\]
\end{proof}
\sssc{Root (根號)}
The $y$th root ($y$次方根) of $w\in\mathbb{C}$ and $y$ such that $w^{\frac{1}{y}}$ is defined, denoted as $\sqrt[y]{w}$ is defined as the principal branch of $w^{\frac{1}{y}}$, where $\sqrt{\phantom{w}}$ is called the radical symbol, radical sign, root symbol, or surd (根號).
\sssc{Scientific notation (科學記號)}
Scientific notation refers to the representation of a real number in the form of $a\times 10^n$, where $1\leq |a|<10\land n\in\mathbb{Z}$.
\sssc{Characteristic (首數) and mantissa (尾數)}
Given a common logarithm $\log(x)$, the characteristic of it is defined as $\lfloor\log(x)\rfloor$, and the mantissa of it is defined as $\log(x)-\lfloor\log(x)\rfloor$.
\sssc{Logarithmic function (對數函數)}
$f(x)=k\log_a(x)$ where $a>0\land a\neq 1\land k\in\mathbb{R}_{\neq 0}$ is called a logarithmic function with base $a$, of which the domain is $\mathbb{R}_{>0}$ and the range is $\mathbb{R}$.
\sssc{Exponential function (指數函數)}
$f(x)=ka^x$ where $a\neq 0\land k\in\mathbb{R}_{\neq 0}$ is called an exponential function with base $a$, of which the domain is $\mathbb{R}$ and the range is $\mathbb{R}_{>0}$ when $k>0$ and $\mathbb{R}_{<0}$ when $k<0$.
\sssc{Natural exponential function (自然指數函數)}
$e^x$ is called the natural exponential function.
\sssc{Exponential growth (指數成長)}
Exponential growth usually refers to a functional that satisfies $f'(x)=kf(x)$ where $k\in\mathbb{R}$, that is, $f(x)=e^{kx}+c$ where $k,c\in\mathbb{R}$.
\ssc{Laws}
\sssc{Logarithmic laws (對數律)}
\[\begin{aligned}
& \log_a(r)+\log_a(s)=\log_a(rs)\\
& \log_a(r)-\log_a(s)=\log_a\left(\frac{r}{s}\right)\\
& \log_{a^m}(r^n)=\frac{n}{m}\log_a(r)\\
& \log_a(b)=\frac{\log_c(b)}{\log_c{a}}\\
& a^{\log_c(b)}=b^{\log_c(a)}
\end{aligned}\]
\sssc{Exponential laws (指數律)}
\[\begin{aligned}
& a^r\cdot a^s=a^{r+s},\\
& (a^r)^s=a^{rs},\\
& (a\cdot b)^r=a^r\cdot b^r
\end{aligned}\]
\sssc{Inverse}
For any $a>0\land a\neq 1\land b\neq 0$, the logarithmic function $k\log_a\left(\frac{x}{b}\right)$ and the exponential function $ba^{\frac{x}{k}}$ are inverses of each other, that is, they are symmetric about $y=x$.
\sssc{Tangent}
In the $xy$ plane, $a=e^{\qty(e^{-1})}$ if and only if any two of $y=a^x$, $y=x$, and $y=\log_ax$ are tangent to each other. If $a=e^{\qty(e^{-1})}$, $y=a^x$, $y=x$, and $y=\log_ax$ are tangent at $(e,e)$, and any two of them do not meet at any point other than $(e,e)$.
\begin{proof}
\[\dv{a^x}{x}=\ln(a)a^x=1\]
\[x=a^x=\frac{1}{\ln(a)}\]
\[a^{\frac{1}{\ln(a)}}=\frac{1}{\ln(a)}\]
\[\ln(a)\frac{1}{\ln(a)}=1=-\ln\qty(\ln(a))\]
\[a=e^{\qty(e^{-1})}\]
\end{proof}
\sssc{Self-power function}
\[\arg\min(x^x)=\frac{1}{e},\quad\min(x^x)=\qty(e^{-1})^{\qty(e^{-1})}\]
\ssc{Values}
\[\log 2\approx 0.3010,\quad\log e\approx 0.4343,\quad\log 3\approx 0.4771,\quad\log\pi\approx 0.4971,\quad\log 4\approx 0.6021,\quad\log 5\approx 0.6990,\]
\[\log 6\approx 0.7782,\quad\log 7\approx 0.8451,\quad\log 8\approx 0.9031,\quad\log 9\approx 0.9542,\quad\log 11\approx 1.0414,\quad\log 12\approx 1.0792.\]
\[\ln 2\approx 0.6931,\quad\ln 3\approx 1.0986,\quad\ln\pi\approx 1.1447,\quad\ln 4\approx 1.3863,\quad\ln 5\approx 1.6094,\quad\ln 6\approx 1.7918,\]
\[\ln 7\approx 1.9459,\quad\ln 10\approx 2.3026,\quad\ln 11\approx 2.3980,\quad\ln 12\approx 2.4849,\quad\frac{1}{\ln 2}\approx 1.4427.\]
\[\sqrt{2}\approx 1.4142,\quad\sqrt{e}=1.6487,\quad\sqrt{3}\approx 1.7321,\quad\sqrt{\pi}\approx 1.7725,\quad\sqrt{5}\approx 2.2361,\]
\[\sqrt{6}\approx 2.4495,\quad\sqrt{7}\approx 2.6458,\quad\sqrt{10}\approx 3.1623,\quad\sqrt{11}\approx 3.3166,\quad\sqrt{13}\approx 3.6056,\]
\[\sqrt{14}\approx 3.7417,\quad\sqrt{15}\approx 3.8730,\quad\sqrt{17}\approx 4.1231,\quad\sqrt{19}\approx 4.3590,\quad\sqrt{21}\approx 4.5826,\]
\[\frac{\sqrt{5}+1}{2}\approx 1.6180,\quad\frac{\sqrt{5}-1}{2}\approx 0.6180,\quad\frac{\sqrt{6}+\sqrt{2}}{4}\approx 0.9659,\quad\frac{\sqrt{6}-\sqrt{2}}{4}\approx 0.2588,\]
\[\frac{\sqrt{10+2\sqrt{5}}}{4}\approx 0.9511,\quad\frac{\sqrt{10-2\sqrt{5}}}{4}\approx 0.5878.\]
\[\langle a_n=n^2\rangle_{n=1}^{20}=1,4,9,16,25,36,49,64,81,100,121,144,169,196,225,256,289,324,361,400.\]
\[\langle a_n=n^2\rangle_{n=21}^{30}=441,484,529,576,625,676,729,784,841,900.\]
\[\langle a_n=n^2\rangle_{n=31}^{40}=961,1024,1089,1156,1225,1296,1369,1444,1521,1600.\]
\[\langle a_n=n^{-1}\rangle_{n=1}^{15}=1,0.5,0.\ol{3},0.25,0.2,0.1\ol{6},0.\ol{142857},0.125,0.\ol{1},0.1,0.\ol{09},0.08\ol{3},0.\ol{076923},0.0\ol{714285},0.0\ol{6}.\]
\[\frac{1}{\sqrt{2}}\approx 0.7071,\quad\frac{1}{\sqrt{3}}\approx 0.5774,\quad\frac{1}{\sqrt{5}}\approx 0.4772,\quad\frac{1}{2\sqrt{2}}\approx 0.3536,\quad\frac{1}{2\sqrt{3}}\approx 0.2887,\quad\frac{1}{2\sqrt{5}}\approx 0.2236.\]
\[e\approx 2.7183,\quad\frac{1}{e}\approx 0.3680,\quad e^e\approx 15.1543,\quad e^{\qty(e^{-1})}\approx 1.4447,\quad \qty(e^{-1})^e\approx 0.0660,\quad\qty(e^{-1})^{\qty(e^{-1})}\approx 0.6922.\]
\end{document}
