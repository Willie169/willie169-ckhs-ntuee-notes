\documentclass[a4paper,12pt]{report}
\setcounter{secnumdepth}{5}
\setcounter{tocdepth}{3}
\input{/usr/share/LaTeX-ToolKit/template.tex}
\begin{document}
\title{Ordinary Differential Equations and Dynamical Systems}
\author{沈威宇}
\date{\temtoday}
\titletocdoc
\ch{Ordinary Differential Equations (常微分方程) and Dynamical Systems (動態系統 or 動力系統)}
\sct{Ordinary Differential Equations}
\ssc{Introduction}
\sssc{Differential form of first-order ODEs}
For a first-order ODE of a dependent variable $y$ with respect to an independent variable $x$,
\[M(x,y)=-N(x,y)\dv{y}{x},\]
where $M$ and $N$ are functions of two variables, $x$ and $y$, the differential form of it is
\[M(x,y)\dd{x}+N(x,y)\dd{y}=0.\]
\sssc{Normal form of ODEs}
The normal form of an $n$th-order ODE of a dependent variable $y$ in codomain $Y$ with respect to an independent variable $x$ in domain $X$ is
\[\dv[n]{y}{x}=F(x,y,y',y'',\dots y^{(n-1)}),\]
in which $F$ is a function of $(n+1)$ variables, $x,y,y',y'',\dots y^{(n-1)}$, with domain being a subset of $X\times Y^n$ and codomain being $Y$.
\sssc{Linear ODE}
An $n$th order ODE of a dependent variable $y$ with respect to an independent variable $x$ is linear if and only if it can be written in the form
\[\sum_{i=0}^na_n(x)y^{(i)}=g(x),\]
where $a_0(x),\dots a_n(x)$, called the coefficients, and $g(x)$, called the forcing term, are given functions of one variable, $x$, and $a_n(x)\neq 0$ on the domain of interest.

An ODE is nonlinear if it is not linear.

A linear ODE is called a homgeneous linear ODE if $g(x)=0$; otherwise, it is called a nonhomogeneous linear ODE.

Let $L[y]=\sum_{i=0}^na_n(x)y^{(i)}$ denotes the left-hand side, then $L[\alpha y_1+\beta y_2]=\alpha L[y_1]+\beta L[y_2]$. That's why it's called to be linear.
\sssc{Solution or flow of an ODE}
Any function $f$ that is defined on an interval $I$, called interval of definition, the interval of existence, the interval of validity, or the domain of the solution, and of class $C^n$ on $I$, and when substituted into an $n$th-order ordinary differential equation reduces the equation to an identity, is said to be a solution (aka flow) of the equation on $I$.

An ODE does not necessarily have to possess a solution.
\sssc{Solution curve or trajectory}
The graph of a solution $f$ on its interval of definition of an ODE is called a solution curve or a trajectory.
\sssc{Explicit solution}
An explicit solution of an ODE of a dependent variable $y$ with respect to an independent variable $x$ is in the form $y=f(x)$, where $f(x)$ is a function of $x$.
\sssc{Implicit solution}
A relation $G(x, y) = 0$ is said to be an implicit solution of an ODE on an interval $I$ if there exists at least one function $f$ that satisfes the relation $G(x, y) = 0$ as well as the ODE on $I$.
\sssc{Type of solutions}
An $n$-parameter family of solutions of an $(\geq n)$th-order ODE of a dependent variable $y$ with respect to an independent variable $x$ is in the form $y=f(x,c_1,c_2,\dots c_n)$ (explicit) or $G(x,y,c_1,c_2,\dots c_n)=0$ (implicit), in which $c_1,c_2,\dots c_n$ are parameters that are arbitrary given that the solution obtained is a solution of the ODE.

If every solution of an $(\geq n)$th-order ODE on an interval $I$ can be obtained from an $n$-parameter family of solutions by appropriate choices of the parameters, we then say that that family of solutions is the general solution (通解) of the ODE.

A solution of an ODE that is free of parameters is called a particular solution (特解).

Sometimes a differential equation possesses a solution that is not a member of a family of solutions of the equation, that is, a solution that cannot be obtained by specializing any of the parameters in the family of solutions. Such an extra solution is called a singular solution (奇異解).
\sssc{System of ODEs (常微分方程組)}
A system of ODEs is two or more equations involving the derivatives of two or more unknown functions or dependent variables of a single independent variable.

A solution of a system of ODEs involving the derivatives of $n$ unknown functions or dependent variables, $y_1,y_2,\dots y_n$ of a single independent variable $x$ is a $n$-tuple of sufficiently smooth functions of $x$ defined on a common interval of definition that satisfy all equations in the system.

A system of ODEs does not necessarily have to possess a solution.
\sssc{Slope function or rate function}
In a first-order ODE in the form $\dv{y}{x}=f(x,y)$, $f$ is called the slope function or rate function.
\sssc{Direction field or slope field (斜率場)}
In a first-order ODE in the form $\dv{y}{x}=f(x,y)$, if we evaluate $f$ over a rectangle grid of points and draw a lineal element at each point $(x,y)$ of the grid with slope $f(x,y)$, then the collection of all these lineal elements is called a direction field or a slope field of the ODE.
\sssc{Isocline}
In a first-order ODE in the form $\dv{y}{x}=f(x,y)$, the isocline for slope $m$ is defined as the family of points $\{(x,y)\mid f(x,y)=m\}$. In the method of isocline, we draw a lineal element at each point in the isocline for slope $m$ of $f$ with slope $m$.
\sssc{Autonomous system (自治系統) (of ODEs)}
A system of first-order ODE is called autonomous if it can be written in the form
\[\dv{x}{t}=f(x),\quad x\in D\subseteq\mathbb{R}^n\land f\colon D\to\mathbb{R}^n.\]
\sssc{Critical point (臨界點), equilibrium, equilibrium point (平衡點), or stationary point (駐點) of autonomous systems of ODEs}
For an autonomous systems of ODEs $\dv{y}{x}=f(y)$, we say that a point $c$ in the domain of $f$ is a critical point, equilibrium, equilibrium point, or stationary point if $f(c)=0$.

If $c$ is a critical point, then $y(x)=c$ is a constant solution of the system of autonomous first-order ODEs, also called an equilibrium solution (平衡解).

If $c$ is a critical point and $f$ is locally Lipschitz at $c$, then $y(x)=c$ is the unique solution of the system of autonomous first-order ODEs through $c$.
\sssc{Translation property or shift property of solutions of autonomous systems of ODEs}
Let function $\Phi(t,y)$ denotes the solution of the autonomous system of ODEs
\[\dv{x}{t}=f(x),\quad x\in D\subseteq\mathbb{R}^n\land f\colon D\to\mathbb{R}^n,\]
with IC $x(0)=y$.

Then the translation property or shift property states that
\[\forall s\in\mathbb{R}\colon\Phi(t+s,y)=\Phi(t,\Phi(s,y)).\]
\ssc{Initial-value problem (IVP) (初值問題)}
\sssc{Initial-value problem (IVP)}
An $n$th-order initial-value problem is an $n$th-order ODE with $n$ conditions for $y$ and $(<n)$th derivative functions of $y$ at the same point, that is, $\{y^{(k)}(x_0)=y_k\}_{k\in[0,n-1]\cap\mathbb{Z}}$, called initial conditions (IC) (初始條件).

A solution of an IVP is a solution of the ODE that complies with all the ICs.

The interval of existence of an IVP is the largest open interval containing $x_0$ where a solution of the IVP exists; the interval of existence of an IVP is the largest open interval containing $x_0$ where a unique solution of the IVP exists.
\sssc{Rectangle}
A rectangle in $\mathbb{R}\times\mathbb{R}^n$ is a subset $D$ of it in the form:
\[D=\{(t,y)\mid|t-T|\leq a\land\forall\tx{\ integer\ }1\leq i\leq n\colon |y_i-Y_i|\leq b_i\}\]
where $t,T\in\mathbb{R}$, $y=(y_1, y_2,\dots  y_n)\in\mathbb{R}^n$, $Y=(Y_1, Y_2,\dots  Y_n)\in\mathbb{R}^n$, and constant $a,b_i\in\mathbb{R}_{>0}$.
\sssc{(Golbal/regular) Lipschitz continuity (利普希茨連續) of functions between metric spaces}
Let $(X,d_1)$ and $(Y,d_2)$ be metric spaces, and $f\colon D\subseteq X\to Y$ be a function. $f$ is called to be Lipschitz continuous if there exists a real constant $K$ such that $\forall x,y\in D$, $d_{2}(f(x),f(y))\leq Kd_{1}(x,y)$.

Let $(X,d_1)$ and $(Y,d_2)$ be metric spaces, and $f\colon O\subseteq X\to Y$ be a function. $f$ is called to be Lipschitz continuous on $D\subseteq O$ if there exists a real constant $K$ such that $\forall x,y\in D$, $d_{2}(f(x),f(y))\leq Kd_{1}(x,y)$.

Lipschitz continuity implies absolute continuity.

If a function $f\colon D\subseteq X\to Y$ between metric spaces $(X,d_1)$ and $(Y,d_2)$ is such that there exists a real constant $K$ such that $\forall x,y\in D$, $d_{2}(f(x),f(y))\leq Kd_{1}(x,y)$, then, any such $K$ is sometimes called a Lipschitz constant (利普希茨常數) of the function $f$, $f$ is sometimes called to be $K$-Lipschitz, and the smallest such $K$ is sometimes called the best Lipschitz constant, the Lipschitz constant, or the dilation of $f$.
\sssc{Short map}
A short map is a function between metric spaces that is $1$-Lipschitz.
\sssc{Contraction mapping, contraction map, contraction, contractive mapping, contractive map, or contractor (壓縮映射)}
A contraction mapping is a function $f$ from a metric space to itself such that there exists a real number $0\leq K<1$ such that $f$ is $K$-Lipschitz, in which $K$ is sometimes called a contraction constant and the smallest such $K$ is sometimes called the best contraction constant or the contraction constant.
\sssc{Banach fixed-point theorem (巴拿赫不動點定理), contraction mapping theorem (壓縮映射定理), contraction principle, or Banach–Caccioppoli theorem}
For any contraction mapping $T$ over a non-empty complete metric space $X$ with best contraction constant $K$, there must exist a unique fixed-point $x^*$ of $T$ in $X$, and for any $x\in X$, for any sequence $\langle x_n\rangle_{n\in \mathbb {N}_0}$ defined as
\[\begin{cases}
&x_0=x\\
&x_n=T\qty(x_{n-1}),n\in\mathbb{N}
\end{cases},\]
\[\lim_{n\to \infty }x_{n}=x^{*}\]
and
\[d(x_n,x^*)\leq\frac{K^n}{1-K}d(x_1,x_0).\]
\sssc{Locally Lipschitz of functions between metric spaces}
Let $(X,d_1)$ and $(Y,d_2)$ be metric spaces, and $f\colon D\subseteq X\to Y$ be a function. $f$ is called to be locally Lipschitz if $\forall x\in D$, there exists a real constant $K$ and an open set $V\subseteq D$ with $x\in V$ such that $d_{2}(f(y),f(z))\leq Kd_{1}(y,z)$ for all $y,z\in V$.

Let $(X,d_1)$ and $(Y,d_2)$ be metric spaces, and $f\colon O\subseteq X\to Y$ be a function. $f$ is called to be locally Lipschitz on $D\subseteq O$ if $\forall x\in D$, there exists a real constant $K$ and an open set $V\subseteq D$ with $x\in V$ such that $d_{2}(f(y),f(z))\leq Kd_{1}(y,z)$ for all $y,z\in V$.


{{{ below Cwait for change to "Ordinary Differential Equations and Dynamical Systems" version
\sssc{Peano existence theorem (皮亞諾存在性定理), Peano theorem (皮亞諾定理), or Cauchy–Peano theorem (柯西-皮亞諾定理)}
Let $D$ be an open subset of $\mathbb{R}\times\mathbb{R}^n$ and $f\colon D\to \mathbb {R}$ be a continuous function, then every initial value problem given by explicit first-order ODE $y'(t)=f\left(t,y(t)\right)$ defined on $D$ and initial condition $y\left(t_{0}\right)=y_{0}$ with $(t_{0},y_{0})\in D$ has a local solution $z\colon I\to \mathbb {R}^n$ where $I$ is a neighbourhood of $t_0$ in $\mathbb {R} $.
\sssc{Carathéodory's existence theorem}
Let
\[D=\{(t,y)\mid|t-T|\leq a\land\forall\tx{\ integer\ }1\leq i\leq n\colon |y_i-Y_i|\leq b_i\}\]
be a rectangle in $\mathbb{R}\times\mathbb{R}^n$ where $t,T\in\mathbb{R}$, $y=(y_1, y_2,\dots  y_n)\in\mathbb{R}^n$, $Y=(Y_1, Y_2,\dots  Y_n)\in\mathbb{R}^n$, and constant $a,b_i\in\mathbb{R}_{>0}$, and $f(t,y)\colon D\to \mathbb {R}$ be a function that is:
\bit
\item continuous in $y$ for each fixed $t$,
\item Lebesgue-measurable in $t$ for each fixed $y$, and
\item such that there is a Lebesgue-integrable function $m\colon [T-a,T+a]\to [0,\infty )$ such that $|f(t,y)|\leq m(t)$ for all $(t,y)\in D$,
\end{itemize}
then every initial value problem given by explicit first-order ODE $y'(t)=f\left(t,y(t)\right)$ defined on $D$ and initial condition $y\left(t_{0}\right)=y_{0}$ with $(t_{0},y_{0})\in D$ has a local solution $z\colon I\to \mathbb {R}^n$ where $I$ is a neighbourhood of $t_0$ in $\mathbb{R}$.

Let $I$ be an open interval of $\mathbb{R}$ and $f(t,y)\colon I\times\mathbb{R}^n\to\mathbb{R}$ where $t\in\mathbb{R}$, $y=(y_1, y_2,\dots  y_n)\in\mathbb{R}^n$ be a function that is:
\bit
\item continuous in $y$ for each fixed $t$,
\item Lebesgue-measurable in $t$ for each fixed $y$, and
\item such that there is a Lebesgue-integrable function $m\colon I\to [0,\infty )$ such that $|f(t,y)|\leq m(t)$ for all $(t,y)\in I\times\mathbb{R}^n$,
\end{itemize}
then every initial value problem given by explicit first-order ODE $y'(t)=f\left(t,y(t)\right)$ defined on $I\times\mathbb{R}^n$ and initial condition $y\left(t_{0}\right)=y_{0}$ with $(t_{0},y_{0})\in I\times\mathbb{R}^n$ has a local solution $z\colon I\to \mathbb {R}^n$ where $I$ is a neighbourhood of $t_0$ in $\mathbb{R}$.
\sssc{Picard–Lindelöf theorem (皮卡-林德勒夫定理) or Cauchy–Lipschitz theorem (柯西-利普希茨定理) local version}
Let
\[D=\{(t,y)\mid|t-T|\leq a\land\forall\tx{\ integer\ }1\leq i\leq n\colon |y_i-Y_i|\leq b_i\}\]
be a rectangle in $\mathbb{R}\times\mathbb{R}^n$ where $t,T\in\mathbb{R}$, $y=(y_1, y_2,\dots  y_n)\in\mathbb{R}^n$, $Y=(Y_1, Y_2,\dots  Y_n)\in\mathbb{R}^n$, and constant $a,b_i\in\mathbb{R}_{>0}$, and $f(t,y)\colon D\to \mathbb {R}$ be a function that is continuous in $t$ andi locally Lipschitz on $D$ in all $y_i$, then every initial value problem given by explicit first-order ODE $y'(t)=f\left(t,y(t)\right)$ defined on $D$ and initial condition $y\left(t_{0}\right)=y_{0}$ with $(t_{0},y_{0})\in D$ has a unique local solution $z\colon I\to \mathbb {R}^n$ where $I$ is a neighbourhood of $t_0$ in $\mathbb {R} $, that is, let $I_1,I_2$ be two neighbourhoods of $t_0$ in $\mathbb{R}$ and $z_i\colon I_i\to\mathbb{R}^n$ for $i\in\{1,2\}$ be two differentiable functions such that $z_i'(t)=f\left(t,z_i(t)\right)$ for all $t\in I_i$, for $i\in\{1,2\}$, then $\exists t_0\in I_1\cap I_2\text{\ s.t.\ }z_1(t_0)=z_2(t_0)\implies\forall t\in I_1\cap I_2\colon z_1(t)=z_2(t)$.

Let $I$ be an open interval of $\mathbb{R}$ and $f(t,y)\colon I\times\mathbb{R}^n\to\mathbb{R}$ where $t\in\mathbb{R}$, $y=(y_1, y_2,\dots  y_n)\in\mathbb{R}^n$ be a function that is continuous in $t$ and locally Lipschitz in all $y_i$, then every initial value problem given by explicit first-order ODE $y'(t)=f\left(t,y(t)\right)$ defined on $I\times\mathbb{R}^n$ and initial condition $y\left(t_{0}\right)=y_{0}$ with $(t_{0},y_{0})\in I\times\mathbb{R}^n$ has a unique solution $z\colon I\to\mathbb {R}^n$, that is, let $I_1,I_2\subseteq\mathbb{R}$ and $z_i\colon I_i\to\mathbb{R}^n$ for $i\in\{1,2\}$ be two differentiable functions such that $z_i'(t)=f\left(t,z_i(t)\right)$ for all $t\in I_i$, for $i\in\{1,2\}$, then $\exists t_0\in I_1\cap I_2\text{\ s.t.\ }z_1(t_0)=z_2(t_0)\implies\forall t\in I_1\cap I_2\colon z_1(t)=z_2(t)$.
\ssc{Boundary-value problem (BVP) (邊值問題)}
\sssc{Boundary-value problem (BVP)}
An $n$th-order boundary-value problem is an $n$th-order ODE with some conditions called boundary conditions (BC) (邊界條件).

A solution of an BVP is a solution of the ODE that complies with all the BCs.
\sct{Solving an ODE}
\ssc{Solving a separable ODE}
{{{


\ssc{Dynamical system (動態系統 or 動力系統)}
\sssc{Dynamical system}
A dynamical system is a tuple $(T, X, \Phi)$ where $T$ is a monoid, written additively and with the independent variable $t$ in it called evolution parameter (演化參數) or time, $X$ is a non-empty set, called the phase space (相空間) or state space (狀態空間) with the independent variable $x$ in it called (system) phase (相) or state (狀態), and $\Phi$ is a function, called the evolution function (演化函數) or flow,
\[\Phi(t,x)\colon U\subseteq(T\times X)\to X\]
where $U$ is such that the second projection map $\pi_2$ for $T\times X$ satisfies
\[\pi_2(U)=X,\]
and for any $x \in X$, it satisfies
\bit
\item identity:
\[\Phi (0,x)=x,\]
\item semigroup property:
\[\Phi (t_{2},\Phi (t_{1},x))=\Phi (t_{2}+t_{1},x)\]
for any $t_{1},t_{2}+t_{1}\in I(x)$ and $t_{2}\in I(\Phi (t_{1},x))$i, in which
\[I(x)\coloneq\{t\in T\mid(t,x)\in U\}\]
for any $x\in X$.
\eit
\sssc{Orbit (軌道)}
Given a dynamical system $(T, X, \Phi)$ with
\[\Phi(t,x)\colon U\subseteq(T\times X)\to X\]
and
\[I(x)\coloneq\{t\in T\mid(t,x)\in U\}.\]
The set
\[\gamma _{x}\coloneq\{\Phi (t,x)\colon t\in I(x)\}\subset X\]
is called the orbit through $x$.

An orbit which consists of a single point is called constant orbit; a non-constant orbit is called closed or periodic if there exists $t\neq 0\in I(x)$ such that $\Phi (t,x)=x$.
\sssc{Phase portrait (相圖)}
Given a dynamical system $(T, X, \Phi)$, the phase portrait is a geometric representation of the collection of all orbits $\gamma$ through any $x\in X$.
\sssc{Attractor (吸引子)}
For a dynamical system $(\mathbb{R}, X, \Phi)$ with $D\subseteq X$ and $\Phi(t,x)\colon \mathbb{R}_{\geq 0}\times D\to X$, a non-empty subset $A$ of $D$ is called an attractor if and only if it satisfies the following three conditions:
\bit
\item Forward invariance:
    \[\forall a\in A\colon\forall t>0\colon\Phi(t,a)\in A.\]
\item Attraction: There exists a neighborhood of $A$, called the basin of attraction for $A$ and denoted $B(A)$, such that for any $b\in B(A)$ and any open neighborhood $N$ of $A$, there is a positive constant $T$ such that $\forall t>T\colon\Phi(t,b)\in N$.
\item Minimality: There is no non-empty proper subset of $A$ that satisfies the above two conditions.
\eit

A point $a$ in $D$ is called an attractor if and only if $\{a\}$ is an attractor.
\sssc{Repeller}
For a dynamical system $(\mathbb{R}, X, \Phi)$ with $D\subseteq X$ and $\Phi(t,x)\colon \mathbb{R}_{\geq 0}\times D\to X$, a non-empty subset $R$ of $X$ is called a repeller if and only if it satisfies the following three conditions:
\bit
\item Forward invariance:
    \[\forall r\in R\colon\forall t>0\colon\Phi(t,r)\in R.\]
\item Repulsion: There exists a neighborhood of $R$, called the basin of repulsion for $R$ and denoted $B(R)$, such that for any $b\in B(R)\setminus R$ and any open neighborhood $N$ of $R$, there is a positive constant $T$ such that $\forall t>T\colon\Phi(t,b)\notin N$.
\item Minimality: There is no non-empty proper subset of $R$ that satisfies the above two conditions.
\eit

A point $r$ in $D$ is called a repeller if and only if $\{r\}$ is a repeller.
\sssc{Continuous-time dynamical system}
A dynamical system $(\mathbb{R}, \mathbb{R}^n, \Phi)$ with $A=\mathbb{R}_{\geq 0}$ (or $\mathbb{R}$), $B\subseteq\mathbb{R}^n$, and $\Phi(t,x)\colon A\times B\to\mathbb{R}^n$ is called a continuous-time dynamical system if $\pdv{\Phi}{t}\big\vert_{t=0}$ exists for all $X\in B$.
\sssc{Autonomous or time-invariant continuous-time dynamical system}
A continuous-time dynamical system $(\mathbb{R}, \mathbb{R}^n, \Phi)$ is called autonomous, aka time-invariant, if $\pdv{\Phi(t,x)}{t}\big\vert_{t=0}$ is only dependent on $x$.
\sssc{Linear (autonomous) continuous-time dynamical system}
An autonomous continuous-time dynamical system $(\mathbb{R}, \mathbb{R}^n, \Phi(t,x))$ with evolution rule $\dv{x}{t}=f(x)$ is called linear if $f(x)=Ax$ where $A$ is a constant matrix in $\mathbb{R}^{n\times n}$.

The eigenvalues and eigen vectors of $A$ are called the eigenvalues and eigen vectors of the linear autonomous continuous-time dynamical system.
\sssc{Phase portrait}
For an autonomous continuous-time dynamical system, one usually plots the space $X$, marks equilibria, and draw arrows at each point $x$ in a grid indicating $f(x)$ at it.

Specifically, a phase portrait for an autonomous continuous-time dynamical system in which $X=\mathbb{R}$ is also called a phase line.
\sssc{Equilibria, fixed points, or fixpoints of autonomous continuous-time dynamical systems}
An equilibrium, fixed point, or fixpoint of an autonomous continuous-time dynamical system $(T, X, \Phi)$ is a point $x^*\in X$ such that the vector field $f(x)=\pdv{\Phi}{t}\big\vert_{t=0}$ vanishes there, that is, $f(x^*)=0$.
\sssc{Systems of first-order ODEs as continuous-time dynamical systems}
Consider a system of first-order ODEs:
\[\dv{x}{t}=f(t,x), \quad x\in\mathbb{R}^n\land f\colon\mathbb{R}\times\mathbb{R}^n\to\mathbb{R}^n.\]

Let $U\subseteq\mathbb{R}\times\mathbb{R}^n$ be the set such that for any $(t_0,x_0)$ in $U$, $t_0$ is in the interval of definition of the solution of it with the initial condition $x(0)=x_0$ and that the solution is unique, and $x(t;x_0)$ be the solution of it with the initial condition $x(0)=x_0$.

Then the dynamical system represented by it has phase space $\mathbb{R}^n$, evolution parameter $t$, phase or state $x$, and evolution function $\Phi\colon U\to\mathbb{R}^n;\Phi(t,x_0)=x(t;x_0)$. The system of ODEs is called the evolution rule of the dynamical system, and the dynamical system is called to be defined by the system of ODEs.
\sssc{Autonomous system of ODEs as autonomous continuous-time dynamical systems}
Consider an autonomous system of ODEs:
\[\dv{x}{t}=f(x), \quad x\in\mathbb{R}^n\land f\colon\mathbb{R}^n\to\mathbb{R}^n.\]

Let $U\subseteq\mathbb{R}\times\mathbb{R}^n$ be the set such that for any $(t_0,x_0)$ in $U$, $t_0$ is in the interval of definition of the solution of it with the initial condition $x(0)=x_0$ and that the solution is unique, and $x(t;x_0)$ be the solution of it with the initial condition $x(0)=x_0$.

Then the dynamical system represented by it has phase space $\mathbb{R}^n$, evolution parameter $t$, phase or state $x$, and evolution function $\Phi\colon U\to\mathbb{R}^n;\Phi(t,x_0)=x(t;x_0)$. The system of ODEs is called the evolution rule of the dynamical system, and the dynamical system is called to be defined by the system of ODEs. The equilibria of the dynamical system are the critical points of the system of ODEs.
\sssc{Stability of equilibria}
Let $x^*$ be an equilibrium of an autonomous continuous-time dynamical system defined by the evolution rule $\dv{x}{t}=f(x)$ and $A=\mathbb{R}_{\geq 0}$ or $\mathbb{R}$ be the domain of the evolution function. Then
\bit
\item $x^*$ is called to be (Lyapunov) stable ((李亞普諾夫)穩定) if and only if
    \[\forall\varepsilon>0\colon\exists\delta>0\text{\ s.t.\ }\|x(0)-x^*\|<\delta\implies\forall t\in A\colon\|x(t)-x^*\|<\varepsilon.\]
\item $x^*$ is called to be asymptotically stable if and only if
    \[\exists\delta>0\text{\ s.t.\ }\|x(0)-x^*\|<\delta\implies\lim_{t\to\infty}x(t)=x^*.\]
An asymptotically stable (漸近穩定) equilibrium is necessarily (Lyapunov) stable and an attractor.
\item $x^*$ is called to be exponentially (指數穩定) stable if and only if
    \[\exists M>0,\alpha>0,\delta>0\text{\ s.t.\ }\|x(0)-x^*\|<\delta\implies\forall t\in A\|x(t)-x^*\|\leq\|x(0)-x^*\|e^{-\alpha t}.\]
    An exponentially stable equilibrium is necessarily asymptotically stable.
\item for $n=1$, $x^*$ is called to be semi-stable or semistable if and only if it is (Lyapunov) stable and
    \[(\exists\delta>0\text{\ s.t.\ }x(0)-x^*<\delta\implies\lim_{t\to\infty}x(t)=x^*)\lor (\exists\delta>0\text{\ s.t.\ }x^*-x(0)<\delta\implies\lim_{t\to\infty}x(t)=x^*).\]
\item for linear autonomous continuous-time dynamical system, $x^*$ is called to be marginally stable (臨界穩定) if it is (Lyapunov) stable but not asymptotically stable,
\item $x^*$ is called to be unstable (不穩定) if and only if it is not (Lyapunov) stable.

An equilibrium that is in a repeller is necessarily unstable.
\eit
\end{document}

