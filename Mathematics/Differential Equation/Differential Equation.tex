\documentclass[a4paper,12pt]{report}
\setcounter{secnumdepth}{5}
\setcounter{tocdepth}{3}
\input{/usr/share/LaTeX-ToolKit/template.tex}
\begin{document}
\title{Differential Equation}
\author{沈威宇}
\date{\temtoday}
\titletocdoc
\chapter{Differential Equation}
\sct{Introduction}
\ssc{Definition}
An equation containing the derivatives of one or more unknown functions (or
dependent variables), with respect to one or more independent variables, is
said to be a differential equation (DE).
\ssc{Classification by type}
If a differential equation contains only ordinary
derivatives of one or more unknown functions with respect to a single independent
variable, it is said to be an ordinary differential equation (ODE). An equation
involving partial derivatives of one or more unknown functions of two or more inde-
pendent variables is called a partial differential equation (PDE). The independent variables of DEs are usually real numbers.
\ssc{Classification by order}
The order of a differential equation is the order of the highest derivative in the equation.
\ssc{Differential form of first-order ODEs}
For a first-order ODE containing independent variable $x$ and dependent variable $y$,
\[M(x,y)=-N(x,y)\dv{y}{x},\]
in which $M$ and $N$ are functions of two variables, $x$ and $y$, the differential form of it is
\[M(x,y)\dd{x}+N(x,y)\dd{y}=0.\]
\ssc{Normal form of ODEs}
The normal form of an $n$th-order ODE containing independent variable $x\in\mathbb{R}$ and dependent variable $y$ in codomain $Y$, is
\[\dv[n]{y}{x}=F(x,y,y',y'',\ldots,y^{(n-1)}),\]
in which $F$ is a function of $(n+1)$ variables, $x,y,y',y'',\ldots,y^{(n-1)}$, with domain being a subset of $X\times Y^n$ and codomain being $Y$.
\ssc{Classification by linearity}
An $n$-th order ODE
