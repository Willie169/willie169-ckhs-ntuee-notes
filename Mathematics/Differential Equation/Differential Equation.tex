\documentclass[a4paper,12pt]{report}
\setcounter{secnumdepth}{5}
\setcounter{tocdepth}{3}
\input{/usr/share/LaTeX-ToolKit/template.tex}
\begin{document}
\title{Differential Equation}
\author{沈威宇}
\date{\temtoday}
\titletocdoc
\chapter{Differential Equation}
\sct{Introduction}
\ssc{Definition}
An equation containing the derivatives of one or more unknown functions (or
dependent variables), with respect to one or more independent variables, is
said to be a differential equation (DE).
\ssc{Classification by type}
If a differential equation contains only ordinary
derivatives of one or more unknown functions with respect to a single independent
variable, it is said to be an ordinary differential equation (ODE). An equation
involving partial derivatives of one or more unknown functions of two or more inde-
pendent variables is called a partial differential equation (PDE). The independent variables of DEs are usually real numbers.
\ssc{Classification by order}
The order of a differential equation is the order of the highest derivative in the equation.
\ssc{Differential form of first-order ODEs}
For a first-order ODE of a dependent variable $y$ with respect to an independent variable $x$,
\[M(x,y)=-N(x,y)\dv{y}{x},\]
where $M$ and $N$ are functions of two variables, $x$ and $y$, the differential form of it is
\[M(x,y)\dd{x}+N(x,y)\dd{y}=0.\]
\ssc{Normal form of ODEs}
The normal form of an $n$th-order ODE of a dependent variable $y$ in codomain $Y$ with respect to an independent variable $x$ in domain $X$ is
\[\dv[n]{y}{x}=F(x,y,y',y'',\ldots,y^{(n-1)}),\]
in which $F$ is a function of $(n+1)$ variables, $x,y,y',y'',\ldots,y^{(n-1)}$, with domain being a subset of $X\times Y^n$ and codomain being $Y$.
\ssc{Classification by linearity}
An $n$-th order ODE of a dependent variable $y$ with respect to an independent variable $x$ is linear if and only if it can be written in the form
\[\sum_{i=0}^na_n(x)y^{(i)}=g(x),\]
where $a_0(x),\ldots,a_n(x)$, called the coefficients, and $g(x)$, called the forcing term, are given functions of one variable, $x$, and $a_n(x)\neq 0$ on the domain of interest.

An ODE is nonlinear if it is not linear.

A linear ODE is called a homgeneous linear ODE if $g(x)=0$; otherwise, it is called a nonhomogeneous linear ODE.

Let $L[y]=\sum_{i=0}^na_n(x)y^{(i)}$ denotes the left-hand side, then $L[\alpha y_1+\beta y_2]=\alpha L[y_1]+\beta L[y_2]$. That's why it's called to be linear.
\ssc{Solution if an ODE}
Any function $f$ that is defined on an continuous $I$, called interval of definition, the interval of existence, the interval of validity, or the domain of the solution, and of class $C^k$ on $I$, and when substituted into an $n$th-order ordinary differential equation reduces the equation to an identity, is said to be a solution of the equation on $I$.

