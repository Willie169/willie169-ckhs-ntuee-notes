\documentclass[a4paper,12pt]{report}
\setcounter{secnumdepth}{5}
\setcounter{tocdepth}{3}
\input{/usr/share/LaTeX-ToolKit/template.tex}
\begin{document}
\title{Differential Equation}
\author{沈威宇}
\date{\temtoday}
\titletocdoc
\section{Differential Equation}
\ssc{Introduction}
\sssc{Definition}
An equation containing the derivatives of one or more unknown functions or dependent variables, with respect to one or more independent variables, is said to be a differential equation (DE). The independent variables of DEs are usually real numbers.
\sssc{Ordinary differential equation (ODE) and Partial differential equation (PDE)}
If a differential equation contains only ordinary derivatives of one or more unknown functions with respect to a single independent variable, it is said to be an ordinary differential equation (ODE). An equation involving partial derivatives of one or more unknown functions of two or more independent variables is called a partial differential equation (PDE).
\sssc{Classification by order}
The order of a differential equation is the order of the highest derivative in the equation.
\ssc{Ordinary differential equation (ODE)}
\sssc{Differential form of first-order ODEs}
For a first-order ODE of a dependent variable $y$ with respect to an independent variable $x$,
\[M(x,y)=-N(x,y)\dv{y}{x},\]
where $M$ and $N$ are functions of two variables, $x$ and $y$, the differential form of it is
\[M(x,y)\dd{x}+N(x,y)\dd{y}=0.\]
\sssc{Normal form of ODEs}
The normal form of an $n$th-order ODE of a dependent variable $y$ in codomain $Y$ with respect to an independent variable $x$ in domain $X$ is
\[\dv[n]{y}{x}=F(x,y,y',y'',\ldots,y^{(n-1)}),\]
in which $F$ is a function of $(n+1)$ variables, $x,y,y',y'',\ldots,y^{(n-1)}$, with domain being a subset of $X\times Y^n$ and codomain being $Y$.
\sssc{Linear ODE}
An $n$-th order ODE of a dependent variable $y$ with respect to an independent variable $x$ is linear if and only if it can be written in the form
\[\sum_{i=0}^na_n(x)y^{(i)}=g(x),\]
where $a_0(x),\ldots,a_n(x)$, called the coefficients, and $g(x)$, called the forcing term, are given functions of one variable, $x$, and $a_n(x)\neq 0$ on the domain of interest.

An ODE is nonlinear if it is not linear.

A linear ODE is called a homgeneous linear ODE if $g(x)=0$; otherwise, it is called a nonhomogeneous linear ODE.

Let $L[y]=\sum_{i=0}^na_n(x)y^{(i)}$ denotes the left-hand side, then $L[\alpha y_1+\beta y_2]=\alpha L[y_1]+\beta L[y_2]$. That's why it's called to be linear.
\sssc{Solution of an ODE}
Any function $f$ that is defined on an interval $I$, called interval of definition, the interval of existence, the interval of validity, or the domain of the solution, and of class $C^k$ on $I$, and when substituted into an $n$th-order ordinary differential equation reduces the equation to an identity, is said to be a solution of the equation on $I$.

An ODE does not necessarily have to possess a solution.
\sssc{Solution curve}
The graph of a solution $f$ on its interval of definition of an ODE is called a solution curve.
\sssc{Explicit solution}
An explicit solution of an ODE of a dependent variable $y$ with respect to an independent variable $x$ is in the form $y=f(x)$, where $f(x)$ is a function of $x$.
\sssc{Implicit solution}
A relation $G(x, y) = 0$ is said to be an implicit solution of an ODE on an interval $I$ if there exists at least one function $f$ that satisfes the relation $G(x, y) = 0$ as well as the ODE on $I$.
\sssc{Type of solutions}
An $n$-parameter family of solutions of an $(\geq n)$th-order ODE of a dependent variable $y$ with respect to an independent variable $x$ is in the form $y=f(x,c_1,c_2,\ldots,c_n)$ (explicit) or $G(x,y,c_1,c_2,\ldots,c_n)=0$ (implicit), in which $c_1,c_2,\ldots,c_n$ are parameters that are arbitrary given that the solution obtained is a solution of the ODE.

If every solution of an $(\geq n)$th-order ODE on an interval $I$ can be obtained from an $n$-parameter family of solutions by appropriate choices of the parameters, we then say that that family of solutions is the general solution (通解) of the ODE.

A solution of an ODE that is free of parameters is called a particular solution (特解).

Sometimes a differential equation possesses a solution that is not a member of a family of solutions of the equation, that is, a solution that cannot be obtained by specializing any of the parameters in the family of solutions. Such an extra solution is called a singular solution (奇異解).
\sssc{System of ODEs (常微分方程組)}
A system of ODEs is two or more equations involving the derivatives of two or more unknown functions or dependent variables of a single independent variable.

A solution of a system of ODEs involving the derivatives of $n$ unknown functions or dependent variables, $y_1,y_2,\ldots,y_n$ of a single independent variable $x$ is a $n$-tuple of sufficiently smooth functions of $x$ defined on a common interval of definition that satisfy all equations in the system.

A system of ODEs does not necessarily have to possess a solution.

