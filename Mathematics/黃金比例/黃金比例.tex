\documentclass[a4paper,12pt]{article}
\setcounter{secnumdepth}{5}
\setcounter{tocdepth}{3}
\newcounter{ZhRenew}
\setcounter{ZhRenew}{1}
\newcounter{SectionLanguage}
\setcounter{SectionLanguage}{1}
\input{/usr/share/latex-toolkit/template.tex}
\begin{document}
\title{黃金比例}
\author{沈威宇}
\date{\temtoday}
\titletocdoc
\section{黃金比例(Golden Ratio)}
\subsection{黃金比例(Golden ratio)}
黃金比例\(\Phi=\frac{\sqrt{5}+1}{2}\approx1.618\)。\newline
黃金比例之倒數\(\frac{1}{\phi}=\frac{\sqrt{5}-1}{2}\approx0.618\)。
\subsection{黃金分割點}
定義:\(\overline{AB}\)上取一點\(C\),使\(\frac{\overline{AB}}{\overline{AC}}=\frac{\overline{AC}}{\overline{CB}}\),此時C點稱線段\(\overline{AB}\)之黃金分割點,且\(\frac{\overline{AB}}{\overline{AC}}=\frac{\overline{AC}}{\overline{CB}}=\Phi\)。
\subsection{黃金矩形(Golden rectangle)}
定義:矩形\(ABCD\)使\(\frac{\overline{AB}}{\overline{AD}}=\Phi\)。\newline
分割:將黃金矩形\(ABCD\)分成二塊,其一\(ABFE\)為正方形,則其二\(DEFC\)為黃金矩形。
\subsection{黃金三角形(Golden triangle)}
定義:腰長與底長之比值為黃金比例或其倒數之等腰三角形,即頂角\(\frac{\pi}{5}\)或\(\frac{3\cdot\pi}{5}\)之等腰三角形。

頂角\(\frac{\pi}{5}\)之等腰三角形之分割:取一腰,其上取一點,使底邊及底邊二端點與該點之連線為一頂角\(\frac{\pi}{5}\)之等腰三角形;取該三角形不與原三角形之邊重合之腰,其上取一點,使底邊及底邊二端點與該點之連線為一頂角\(\frac{\pi}{5}\)之等腰三角形;後不斷取新三角形不與前一三角形之邊重合之腰,其上取一點,使底邊及底邊二端點與該點之連線為一頂角\(\frac{\pi}{5}\)之等腰三角形。依此法,每次分割均產生一頂角\(\frac{\pi}{5}\)之等腰三角形及一頂角\(\frac{3\cdot\pi}{5}\)之等腰三角形。
\subsection{直角黃金三角形}
定義:三邊長比例為\(1:\sqrt{\Phi}:\Phi\)之直角三角形。
\subsection{正五角星形(Pentagram)}
性質:令正五角星形\(AFBGCHDIEJ\)外接五邊形\(ABCDE\)及內接五邊形\(FGHIJ\),則點\(J\)為線段\(\overline{BE}\)的黃金分割點、點F為線段\(\overline{BJ}\)的黃金分割點、內接五邊形邊長為外接五邊形邊長的\(\frac{1}{\Phi^2}=\frac{3-\sqrt{5}}{2}\)倍、\(\overline{AF}\)為外接五邊形邊長的\(\frac{\sqrt{5}-1}{2}\)倍。
\subsection{黃金螺線(Golden spiral)}
定義:自原點向外展開的螺線以極座標方程表示為\(r=a\cdot\Phi^{\frac{2\cdot\theta}{\pi}}\),其中\(a\)為常數。\(a\)為正則螺線逆時針向外展開,為負則螺線順時針向外展開,為零則圖形僅一點;\(|a|\)愈大,螺線相鄰二層間距愈大。\newline
以黃金矩形近似:將黃金矩形分割之各正方形依序作圓心角\(\frac{\pi}{2}\)之扇形,半徑為該正方形之邊長,第一個扇形之圓心為靠近剩餘黃金矩形的二頂點任一者,其餘扇形之圓心為最靠近前一扇形圓心者。\newline
以頂角\(\frac{\pi}{5}\)之等腰三角形近似:取其分割之各頂角\(\frac{3\cdot\pi}{5}\)之等腰三角形,以頂角為圓心,腰為半徑,作圓心角\(\frac{3\cdot\pi}{5}\)之扇形。
\subsection{黃金角(Golden angle)}
定義:將圓周長依\(1:\Phi\)分割成二段,較小之弧長對應之圓心角稱之。其值為\(2\cdot\pi-\frac{2\cdot\pi}{\Phi}\approx137.5^\circ\)。
\subsection{費波納契數列(Successione di Fibonacci)}
費波納契數列第\(n\)項\(a_n=\frac{1}{\sqrt{5}}\cdot(\frac{1+\sqrt{5}}{2})^n-\frac{1}{\sqrt{5}}\cdot(\frac{1-\sqrt{5}}{2})^n\)。\newline
\(lim_{n\to\infty}\frac{a_{n+1}}{a_n}=\frac{1+\sqrt{5}}{2}\)
\subsection{生成螺線(Generative spiral)}
若用顯微鏡觀察新芽頂端,可見所有植物的主要徵貌,包含葉子、花瓣、萼片、小花(floret)等,之生長過程,在頂端的中央有一圓形的組織,稱「頂尖」(apex);而在頂尖的周圍有微小隆起物一個接一個的形成,這些隆起稱「原基」(primordium)。成長時,每一個原基自頂尖移開(頂尖從隆起處向外生長,新的原基在原地),最後長成葉子、花瓣、萼片等,每一原基並希望其生成之器官能夠獲得最大的生長空間,故原基與原基隔得相當開,且較早產生的原基移開得較遠,與頂尖之距離較長。若依照原基的生成時間順序描出原基的位置,可畫出一條捲繞得非常緊的螺線,稱「生成螺線」(generative spiral)。相鄰兩原基之間的角度,稱「發散角」(divergence angle)。 在發散角固定的假設下,使原基盡可能緊密排列的發散角為黃金角。當發散角為黃金角時,同時可見左右旋螺線,即一組順時針、一組逆時針,稱「斜列線」(parastichy)。兩組螺線的數目是相鄰的費波納契數。若發散角非黃金角,則原基非盡可能緊密排列,且僅可見一組斜列線。
\end{document}