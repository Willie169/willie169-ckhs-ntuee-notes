\documentclass[a4paper,12pt]{article}
\setcounter{secnumdepth}{5}
\setcounter{tocdepth}{3}
\input{/usr/share/LaTeX-ToolKit/template.tex}
\begin{document}
\title{Sequences and Series}
\author{沈威宇}
\date{\temtoday}
\titletocdoc
\sct{Sequences (數列) and Series (級數)}
\ssc{Sequences}
\begin{itemize}
\item\textbf{Sequence}: A sequence in $X$ is a function of which the domain is a set $\{x\in\mathbb{Z}\mid l\leq x\leq m\}$ or $\{x\in\mathbb{Z}\mid l\leq x\}$ and the codomain is a topological space $X$, in which $l$ is an integer, usually $0$ or $1$, and $m$ is a integer, denoted as $\langle a_n\rangle$, $\{a_n\}$, $(a_n)$, $\langle a_n\rangle_{n=l}^m$, $\{a_n\}_{n=l}^m$, or $(a_n)_{n=l}^m$, with $m=\infty$ when its domain is $\{x\in\mathbb{Z}\mid l\leq x\}$, where the subscript $n$ refers to the $n$th element of the sequence, that is, the function value when the  independent variable is $n$.
\item\textbf{Finite Sequence (有限數列)}: A finite sequence is a sequence with finite terms, i.e. $\langle a_n\rangle_{n=l}^m,\quad m\in\mathbb{Z}$.
\item\textbf{Infinite sequence (無窮數列)}: An infinite sequence is a sequence with infinite terms, i.e. $\langle a_n\rangle_{n=l}^\infty$. Unless otherwise specified, sequences refer to infinite sequences.
\end{itemize}
\ssc{Series}
\begin{itemize}
\item\textbf{Series}: The sum of the terms of a sequence.
\item\textbf{Finite Series (有限級數)}: The sum of the terms of a finite sequence.
\item\textbf{Infinite Series (無窮級數)}: The sum of the terms of an infinite sequence.
\end{itemize}
\ssc{List of Sequences}
\sssc{Arithmetic progression or sequence (等差數列)}
An arithmetic sequence is a sequence $\langle a_n\rangle=\langle a_1+(n-1)d\rangle$. 

Given $a$ and $b$, $\frac{a+b}{2}$ is called the median of an arithmetic sequence (等差中項).

\[\nexists\lim_{n\to\infty}a_n,\quad d\neq 0\]
\[\lim_{n\to\infty}a_n=a_1,\quad d=0\]
\sssc{Geometric progression or sequence (等比 or 幾何數列)}
A geometric sequence is a sequence $\langle a_n\rangle=\langle a_1\cdot r^{n-1}\rangle$, where $a_1r\neq 0$.

Given $a$ and $b$, $\pm\sqrt{ab}$ is called the median of an geometric sequence (等比中項).
\[\nexists\lim_{n\to\infty}a_n,\quad |d|\geq 1\land d\neq 1\]
\[\lim_{n\to\infty}a_n=a_1,\quad d=1\]
\[\lim_{n\to\infty}a_n=0,\quad |d|<1\]
\ssc{List of Series}
\sssc{Arithmetic series (等差級數)}
An arithmetic series is a series $S_n=\sum_{i=1}^na_i$, where $\langle a_n\rangle$ is an arithmetic sequence.
\[S_n=\frac{n}{2}\qty(a_1+a_n)=\frac{n}{2}\qty(2a_1+(n-1)d)=na_1+\frac{n(n-1)d}{2}\]
\[\nexists\lim_{n\to\infty}S_n,\quad a_1\neq 0\lor d\neq 0\]
\[\lim_{n\to\infty}S_n=0,\quad a_1=0\land d=0\]
\sssc{Geometric series (等比 or 幾何級數)}
A geometric series is a series $S_n=\sum_{i=1}^na_i$, where $\langle a_n\rangle$ is a geometric sequence.

\[S_n=\frac{a_1\qty(1-r^n)}{1-r},\quad r\neq 1\]
\[S_n=na_1,\quad r=1\]
\[\lim_{n\to\infty}S_n=\frac{a_1}{1-r},\quad \abs{r}<1\]
\[\nexists\lim_{n\to\infty}S_n,\quad \abs{r}\geq 1\]
\sssc{Riemann zeta function (黎曼 zeta 函數)}
\[\begin{aligned}
\zeta(s) &= \sum_{n=1}^\infty\frac{1}{n^s}\\
&= \frac{1}{\Gamma (s)}\int _0^\infty \frac {x^{s-1}}{e^x-1}\,\mathrm {d} x
\eam

Harmonic Series (調和級數):
\[S_n=\sum_{n=1}^n\frac{1}{n}\]
\[\nexists\sum_{n=1}^\infty\frac{1}{n}\]

Basel Problem (巴塞爾問題):
\[\zeta(2)=\frac{\pi^2}{6}\]

Other Even Positive Integers:
\[\zeta(4)=\frac{\pi^4}{90}\]
\[\zeta(6)=\frac{\pi^6}{945}\]
\[\zeta(8)=\frac{\pi^8}{9450}\]
\[\zeta(10)=\frac{\pi^{10}}{93555}\]
\[\zeta(12)=\frac{691\pi^{12}}{638512875}\]
\[\zeta(14)=\frac{2\pi^{14}}{18243225}\]

Infinity:
\[\lim_{n\to\infty}\zeta(n)=1\]
\sssc{Euler–Mascheroni constant (歐拉–馬斯克若尼常數)}
\[\begin{aligned}
\gamma &= \lim _{n\to \infty }\left(\left(\sum _{k=1}^n\frac {1}{k}\right)-\ln(n)\right)\\
&= \int _1^\infty \left(\frac{1}{\lfloor x\rfloor}-\frac{1}{x}\right)\,\mathrm{d}x
\end{aligned}\]
\sssc{Power series (冪級數)}
\[\begin{aligned}
\sum_{i=1}^ni &= \frac{n\qty(n+1)}{2}\\
\sum_{i=1}^ni^2 &= \frac{n\qty(n+1)\qty(2n+1)}{6}\\
\sum_{i=1}^ni^3 &= \qty(\frac{n(n+1)}{2})^2\\
\sum_{i=1}^ni^r &= n + \sum_{k=1}^{n-1} (n-k)((k+1)^r - k^r)\\
&= n + \sum_{k=1}^{n-1} (n-k)\sum_{j=0}^{r-1}\binom{r}{j}k^{j}
\end{aligned}\]
\end{document}