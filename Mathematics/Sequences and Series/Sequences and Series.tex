\documentclass[a4paper,12pt]{article}
\setcounter{secnumdepth}{5}
\setcounter{tocdepth}{3}
\input{/usr/share/latex-toolkit/template.tex}
\begin{document}
\title{Sequences and Series}
\author{沈威宇}
\date{\temtoday}
\titletocdoc
\section{Sequences and Series (數列與級數)}
\ssc{Sequence (數列)}
\sssc{Sequence (數列)}
A sequence is a function whose domain is an interval of integers, usually denoted as $\langle a_n\rangle$, $\{a_n\}$, or $(a_n)$, sometimes with domain as $\langle a_n\rangle_{n=1}^m$, $\{a_n\}_{n=1}^m$, or $(a_n)_{n=1}^m$, where the subscript $n$ refers to the nth element of the sequence, that is, the function value when the independent variable is $n$.
\sssc{Finite sequence (有限數列)}
A finite sequence is a sequence with finite terms, e.g. $\langle a_n\rangle_{n=1}^m=\langle a_1,a_2,\ldots,a_m\rangle,m\geq 1$ and $m$ is finite.
\sssc{Infinite sequence (無窮數列)}
An infinite sequence is a sequence with infinite terms, e.g. $\langle a_n\rangle_{n=1}^\infty=\langle a_1,a_2,\ldots\rangle$. Unless otherwise specified, the sequences referred to below are infinite sequences.
\ssc{Series (級數)}
\sssc{Series (級數)}
The sum of the terms of a sequence.
\sssc{Finite Series (有限級數)}
The sum of the terms of a finite sequence.
\sssc{Infinite Series (無窮級數)}
The sum of the terms of an infinite sequence.
\subsection{Arithmetic progression/sequence (等差數列) and arithmetic series (等差級數)}
An arithmetic sequence is a sequence $\langle a_n\rangle=\langle a_1+(n-1)d\rangle$. 

Given $a$ and $b$, $\frac{a+b}{2}$ is called the median of an arithmetic sequence (等差中項).

An arithmetic series is a series $S_n=\sum_{i=1}^na_i$, where $\langle a_n\rangle$ is an arithmetic sequence.

\[\nexists\lim_{n\to\infty}a_n,\quad d\neq 0\]
\[\lim_{n\to\infty}a_n=a_1,\quad d=0\]
\[S_n=\frac{n}{2}\qty(a_1+a_n)=\frac{n}{2}\qty(2a_1+(n-1)d)=na_1+\frac{n(n-1)d}{2}\]
\[\nexists\lim_{n\to\infty}S_n,\quad a_1\neq 0\lor d\neq 0\]
\[\lim_{n\to\infty}S_n=0,\quad a_1=0\land d=0\]
\subsection{Geometric progression/sequence (等比/幾何數列) and geometric series (等比/幾何級數)}
A geometric sequence is a sequence $\langle a_n\rangle=\langle a_1\cdot r^{n-1}\rangle$, where $a_1r\neq 0$.

Given $a$ and $b$, $\pm\sqrt{ab}$ is called the median of an geometric sequence (等比中項).

A geometric series is a series $S_n=\sum_{i=1}^na_i$, where $\langle a_n\rangle$ is a geometric sequence.

\[S_n=\frac{a_1\qty(1-r^n)}{1-r},\quad r\neq 1\]
\[S_n=na_1,\quad r=1\]
\[\lim_{n\to\infty}S_n=\frac{a_1}{1-r},\quad \abs{r}<1\]
\[\nexists\lim_{n\to\infty}S_n,\quad \abs{r}\geq 1\]
\subsection{Riemann zeta function (黎曼 zeta 函數)}
\[\begin{aligned}
\zeta(s) &= \sum_{n=1}^\infty\frac{1}{n^s}\\
&= \frac{1}{\Gamma (s)}\int _0^\infty \frac {x^{s-1}}{e^x-1}\,\mathrm {d} x
\eam
\sssc{Harmonic series (調和級數)}
\[S_n=\sum_{n=1}^n\frac{1}{n}\]
\[\nexists\sum_{n=1}^\infty\frac{1}{n}\]
\sssc{Basel problem (巴塞爾問題)}
\[\zeta(2)=\frac{\pi^2}{6}\]
\sssc{Other even positive integers}
\[\zeta(4)=\frac{\pi^4}{90}\]
\[\zeta(6)=\frac{\pi^6}{945}\]
\[\zeta(8)=\frac{\pi^8}{9450}\]
\[\zeta(10)=\frac{\pi^{10}}{93555}\]
\[\zeta(12)=\frac{691\pi^{12}}{638512875}\]
\[\zeta(14)=\frac{2\pi^{14}}{18243225}\]
\sssc{Infinity}
\[\lim_{n\to\infty}\zeta(n)=1\]
\subsection{Euler–Mascheroni constant (歐拉–馬斯克若尼常數)}
\[\begin{aligned}
\gamma &= \lim _{n\to \infty }\left(\left(\sum _{k=1}^n\frac {1}{k}\right)-\ln(n)\right)\\
&= \int _1^\infty \left(\frac{1}{\lfloor x\rfloor}-\frac{1}{x}\right)\,\mathrm{d}x
\end{aligned}\]
\subsection{冪級數(Power series)}
\[\begin{aligned}
\sum_{i=1}^ni &= \frac{n\qty(n+1)}{2}\\
\sum_{i=1}^ni^2 &= \frac{n\qty(n+1)\qty(2n+1)}{6}\\
\sum_{i=1}^ni^3 &= \qty(\frac{n(n+1)}{2})^2\\
\sum_{i=1}^ni^r &= n + \sum_{k=1}^{n-1} (n-k)((k+1)^r - k^r)\\
&= n + \sum_{k=1}^{n-1} (n-k)\sum_{j=0}^{r-1}\binom{r}{j}k^{j}
\end{aligned}\]
\end{document}