\documentclass[a4paper,12pt]{article}
\setcounter{secnumdepth}{5}
\setcounter{tocdepth}{3}
\input{/usr/share/latex-toolkit/template.tex}
\begin{document}
\title{Complex Analysis}
\author{沈威宇}
\date{\temtoday}
\titletocdoc
\sct{Complex Analysis (複分析 or 複變分析)}
\ssc{Derivative}
Let $W$ be a topological vector space. The ordinary derivative or derivative of a function $f(u)\colon U\subseteq\mathbb{C}\to W$ (with respect to $u$) at $x\in U$, denoted as $f'(x)$, is defined as
\[f'(x)=\lim_{z\to x}\frac{f(z)-f(x)}{z-x}\]
if the limit exists. If such limit exists, we say $f$ is (complex) differentiable at $x$.

We define the (first(-order)) ordinary derivative (function) or derivative (function) of $f$ (with respect to $u$) as a function $f'$ with codomain $W$ such that for any $x\in U$ at which $f$ is (complex) differentiable, $f'$ maps $x$ to the derivative of $f$ at $x$.

The derivative of the $k$th(-order) ($k\in\mathbb{N}$) derivative function of $f$ at $x\in U$ is called the $(k+1)$th(-order) derivative of $f$ (with respect to $u$) at $x\in U$. The derivative function of the $k$th(-order) ($k\in\mathbb{N}$) derivative function of $f$ is called the $k+1$th(-order) derivative function of $f$ (with respect to $u$).

If for any $n\in\mathbb{N}$, the $n$th(-order) derivative of a function $f$ at a point $x$ in its domain exists, we say $f$ is infinitely (complex) differentiable at $x$.

If $f$ is (complex) differentiable at all point in $I\subseteq U$, we say $f$ is (complex) differentiable on $I$; if $f$ is (complex) differentiable on $U$, we say $f$ is (complex) differentiable. If $f^{(n-1)}$ exists and is (complex) differentiable at all point in $I\subseteq U$, we say $f$ is $n$-times (complex) differentiable on $I$; if $f^{(n-1)}$ exists and is (complex) differentiable on $U$, we say $f$ is $n$-times (complex) differentiable.

The operation of finding the derivative or derivative function is called ordinary differentiation or differentiation.

Specifically, the $0$th(-order) derivative of $f$ is $f$ itself.
\ssc{Cauchy–Riemann equations}
A function $f(x+iy)=u(x,y)+iv(x,y)$ for some real-valued functions $u,v$ with $D_u,D_v\subseteq\bbR_2$ is holomorphic if $u$ and $v$ satisfy the Cauchy–Riemann equations:
\[\pdv{u}{x}=\pdv{v}{y},\]
\[\pdv{u}{y}=-\pdv{v}{x}.\]
The converse is true if $f$ is continuous.
\bpr
PLACEHOLDER
\epr
\ssc{Holomorphic or analytic function}
A function $f$ is holomorphic or analytic on $U\subseteq D_f$ if it is complex differentiable at every $z\in U$.

A function $f$ is holomorphic or analytic if it is holomorphic on $D_f$.

A function $f$ is holomorphic or analytic at $z\in D_f$ if it is holomorphic on some open neighborhood of $z$.
\ssc{Biholomorphic function}
A function $f$ is biholomorphic on $U\subseteq D_f$ iff $f_U$ is bijective and holomorphic and $(f_U)^{-1}$ is holomorphic.

A function is biholomorphic iff it is bijective and holomorphic and its inverse is holomorphic.
\ssc{Poles}
An isolated singularity $z_0$ of a complex function $f(z)$ is a pole of it iff there exists $n\in\bbN$, called the order, multiplicity, or degree of $z_0$, such that
\[(z-z_0)^nf(z)\]
is holomorphic on an open neighborhood of $z_0$ and
\[\lim_{z\to z_0}(z-z_0)^nf(z)\neq 0.\]
A simple pole is a pole of order 1.
\ssc{Zeros}
A point is a zero of a complex function $f$ iff it is a pole of $\frac{1}{f}$.

A point is a zero of order $-n$ of a complex function $f$ iff it is a pole of order $n$ of $\frac{1}{f}$.

A simple zero is a zero of order -1.
\sssc{Meromorphic function}
A complex function $f$ is meromorphic on an open set $U$ if for all $z\in U$, either $f$ is holomorphic at $z$ or $z$ is a pole of $f$.

A complex function $f$ is meromorphic if it is meromorphic on $D_f$.
\ssc{Essential singularity}
An isolated singularity of a complex function is an essential singularity iff it is neither a removable singularity nor a pole.
\ssc{Annulus}
An annulus $\ann(a; r, R)$ with $R\leq\infty$ in the complex plane is a region defined as 
\[r<|z-a|<R.\]
If $r=0$, the region is also called punctured disk.
\ssc{Contour}
A contour is a parameterized curve $\gamma(t)\colon[a,b]\to\bbC$ that is piecewise $C^1$.
\ssc{Contour integral}
The contour integral of a complex function $f$ defined on the image of a contour $\gamma(t)\colon[a,b]\to\bbC$ over $\gamma$ is defined as the Riemann–Stieltjes integral:
\[\int_{\gamma}f(z)\dd{z}\coloneq\int_{t=a}^bf(\gamma(t))\dd{\gamma(t)}.\]
\sssc{Cauchy integral theorem or Cauchy–Goursat theorem}
PLACEHOLDER
\sssc{Cauchy's integral formula}
PLACEHOLDER
\sssc{Laurent series}
Let $f$ be holomorphic on an annulus $A=\ann(z_0; r; R)$. The Laurent series of $f$ at $z_0$ is
\[f(z)=\sum_{n=0}^{\infty}a_n(z-z_0)^n+\sum_{n=1}^{\infty}a_{-n}(z-z_0)^{-n},\quad z\in A,\]
where the first term is called regular or analytic part and the second term is called principal or singular part, and $a_n$ is given by
\[a_n=\frac{1}{2\pi i}\oint_{\gamma}\frac{f(z)}{(z-z_0)^{n+1}}\dd{z}\]
for any closed contour $\gamma\in A$ enclosing $z_0$.
\bit
\item $a_{-n}=0\forall n\in\bbN$: either $f$ is holomorphic at $z_0$ or $z_0$ is a removable singularity of $f$.
\item $N\in\bbN\land a_{-n}=0$\forall n\in\bbN_{>N}\land a_{-N}\neq 0$: $z_0$ is a pole of order $N$ of $f$.
\item Otherwise: $z_0$ is a essential singularity of $f$.
\eit
$a_{-1}$ is called residue of $f$ at $z_0$, denoted as $\Res(f,z_0)$ or $\Res_{z_0}(f)$, that is,
\[\Res(f,z_0)=\frac{1}{2\pi i}\oint_{\gamma}f(z)\dd{z}.\]
\end{document}
