\documentclass[a4paper,12pt]{report}
\setcounter{secnumdepth}{5}
\setcounter{tocdepth}{3}
\input{/usr/share/latex-toolkit/template.tex}
\begin{document}
\title{Complex Analysis}
\author{沈威宇}
\date{\temtoday}
\titletocdoc
\ch{Complex Analysis}
\sct{Complex Number}
\ssc{Complex number (複數)}
\sssc{Complex Number}
The set of complex numbers $\bbC$ is defined as
\[\bbC\coloneq\{a+bi\mid a,b\in\bbR\},\]
where $i$ is defined as the only complex number such that $i^2=-1$ and is called the imaginary unit.

For a complex number $z=a+bi$ with $a,b\in\bbR$, $a$ is called its real part (實部) and denoted as $\Re(z)$, $\mathrm{Re}(z)$, or $\mathcal{Re}(z)$, $b$ is called its imaginary part (虛部) and denoted as $\Im(z)$, $\mathrm{Im}(z)$, or $\mathcal{Im}(z)$, $z$ is called an imaginary number (虛數) if $b\neq 0$, and $z$ is called a pure imaginary number (純虛數) if $a=0$ and $b\neq 0$.
\sssc{Addition, subtraction, and multiplication}
For two complex numbers $y=a+bi$ and $z=c+di$ with $a,b,c,d\in\bbR$:
\[y+z=(a+c)+(b+d)i.\]
\[y-z=(a-c)+(b-d)i.\]
\[y\cdot z=(ac-bd)+(ad+bc)i.\]
\sssc{Absolute value (絕對值), magnitude, radial distance (向徑), or modulus (模/模長)}
The absolute value, magnitude, radial distance, or modulus of a complex number $z=a+bi$ with $a,b\in\bbR$ is $\abs{z}=\sqrt{a^2+b^2}$.
\sssc{Conjugate (共軛)}
The conjugate of a complex number $z=a+bi$ with $a,b\in\bbR$, denoted as $\ol{z}$ and called z bar, is defined as $a-bi$.
\sssc{Division}
\[\frac{1}{z}=\frac{\bar{z}}{|z|^2},z\neq 0\]
\sssc{Properties}
For complex numbers $z,z_1,z_2$:
\[|z|=\sqrt{z\cdot\ol{z}}\]
\[\ol{z_1}+\ol{z_2}=\ol{z_1+z_2}.\]
\[\ol{z_1}-\ol{z_2}=\ol{z_1-z_2}.\]
\[\ol{z_1\cdot z_2}=\ol{z_1}\cdot\ol{z_2}.\]
\[\Re(\ol{z_1}\cdot z_2)=\Re(z_1\cdot\ol{z_2}).\]
\[\frac{\ol{z_1}}{\ol{z_2}}=\ol{\qty(\frac{z_1}{z_2})},\quad z_2\neq 0.\]
\[\ol{z^n}=\qty(\ol{z})^n,\quad zn\neq 0\land n\in\bbZ.\]
\ssc{Polynomials}
\sssc{Fundamental theorem of algebra (代數基本定理), d'Alembert's theorem, or the d'Alembert–Gauss theorem}
Every non-constant single-variable polynomial with complex coefficients has at least one complex root.

Every non-zero, single-variable, degree n polynomial with complex coefficients has, counted with multiplicity, exactly n complex roots.
\sssc{Equality of polynomials with real coefficients of complex number and its conjugate}
For all polynomials with real coefficients $f$, for all complex numbers $z$:
\[\ol{f(z)}=f\qty(\ol{z}).\]
Pair of imaginary roots (虛根成對定理): If a polynomial with real coefficients has an imaginary root, then its conjugate is also a root. Thus, an odd-degree polynomial with real coefficients must have at least one real root.
\subsection{Argument}
\subsubsection{Argument (輻角)}
For a complex number $z\neq 0$, the argument of $z$, denoted as $\arg(z)$, is definited as any real number $\varphi $ such that
\[z=|z|e^{i\varphi}.\]
\subsubsection{Principal argument (輻角主值/主輻角)}
For a complex number $z=\neq 0$, the principal argument of $z$, denoted as $\Arg(z)$, is definited as the argument of $z$ with in $(-\pi,\pi]$. (Some sources define it as the argument of $z$ with in $[0,2\pi)$.)
\sssc{Polar form (極式)}
The polar form of a complex number $z$ is
\[z=|z|(\cos\theta+i\sin\theta),\quad\theta\in\bbR.\]
\sssc{Properties}
For any real number $\theta,\varphi,a,b$:
\[\ol{e^{i\theta}}=e^{-i\theta}.\]
\[(ae^{i\theta})(be^{i\varphi})=(ab)e^{i(\theta+\varphi)}.\]
\[\frac{ae^{i\theta}}{be^{i\varphi}}=\frac{a}{b}e^{i(\theta-\varphi)}.\]
\ssc{Complex plane (複數平面), Argand plane (阿爾岡平面), or Gaussian plane (高斯平面)}
\sssc{Complex plane, Argand plane, or Gaussian plane}
The complex plane is the plane formed by the complex numbers, with a Cartesian coordinate system such that the horizontal x-axis, called the real axis, is formed by the real numbers, and the vertical y-axis, called the imaginary axis, is formed by the imaginary numbers.
\sssc{De Moivre's formula (隸美弗公式)}
\[(r(\cos\theta+i\sin\theta))^n=r^n(\cos(n\theta)+i\sin(n\theta)),\quad r\neq 0\land n\in\mathbb{Z}.\]
\sssc{Hyperbolic De Moivre's formula}
\[(r(\cosh\theta+i\sinh\theta))^n=r^n(\cosh(n\theta)+i\sinh(n\theta)),\quad r\neq 0\land n\in\mathbb{Z}.\]
\sssc{Dot product}
The dot product of two vectors in $\bbR^2$ corresponding to the two complex numbers $z_1$ and $z_2$ on the complex plane is $\Re(z_1\cdot\ol{z_2})$.
\ssc{Geometric meaning of exponentiations of complex numbers on complex plane}
\sssc{Raciporial of positive number exponentiation of 1}
For $n\in\bbN$, $1^{\frac{1}{n}}$, that are, the $n$ roots of $x^n=1$, that are, $\omega=\cos\frac{2\pi}{n}+i\sin\frac{2\pi}{n}$ to the power of $0$ to $(n-1)$, that are,
\[1^{\frac{1}{n}}=e^{i\frac{2\pi k}{n}},\quad k\in\mathbb{N}_0\land k<n,\]
are the $n$ vertices of a regular $n$-gon inscribed in a unit circle centered at the origin including $1$ on the complex plane.
\sssc{Raciporial of positive number exponentiation of nonzero complex number}
For $n\in\bbN$ and $z\in\bbC_{\neq 0}$, $z^{\frac{1}{n}}$, that are, the $n$ roots of $x^n=z$, that are, $\sqrt[n]{|z|}$ multiplied by $\omega=\cos\frac{2\pi}{n}+i\sin\frac{2\pi}{n}$ to the power of $0$ to $(n-1)$, that are,
\[z^{\frac{1}{n}}=\sqrt[n]{|z|}e^{i\frac{\Arg(z)+2\pi k}{n}},\quad k\in\mathbb{N}_0\land k<n,\]
are the $n$ vertices of a regular $n$-gon inscribed in a circle of radius $\sqrt[n]{|z|}$ centered at the origin including $\sqrt[n]{|z|}e^{i\frac{\Arg(z)}{n}}$ on the complex plane.
\sssc{Nonzero complex number exponentiation of nonzero complex number}
For $w,z\in\bbC_{\neq 0}$, $z^w$, that are,
\[\begin{aligned}
z^w&=e^{w\ln(z)}\\
&=\left(|z|e^{i\qty(\Arg(z)+2\pi k)}\right)^w\\
&=|z|^{\Re(w)}|z|^{\Im(w)i}e^{i\Re(w)\qty(\Arg(z)+2\pi k)}e^{-\Im(w)\qty(\Arg(z)+2\pi k)}\\
&=|z|^{\Re(w)}e^{-\Im(w)\qty(\Arg(z)+2\pi k)+i(\Re(w)\qty(\Arg(z)+2\pi k)+\Im(w)\ln|z|)},\\
&\quad k\in\mathbb{Z},
\end{aligned}\]
are points with equal angle difference $\Re(w)2\pi$ on the spiral
\[|z|^{\Re(w)}e^{-\Im(w)\theta}\left(\cos(\Re(w)\theta+\Im(w)\ln|z|)+i\sin(\Re(w)\theta+\Im(w)\ln|z|)\right)\]
of parameter $\theta\in\bbR$ including that of $\theta=\Arg(z)$ on the complex plane.

\tb{Modulus:}
\[\abs{z^w}=|z|^{\Re(w)}e^{-\Im(w)\qty(\Arg(z)+2\pi k)}.\]
\begin{itemize}
\item When $\Im(w) > 0$: The modulus decays exponentially with respect to $k$, approaching zero as $k$ increases and diverging to infinity as $k$ decreases.
\item When $\Im(w) < 0$: The modulus increases exponentially with respect to $k$, diverging to infinity as $k$ increases and approaching zero as $k$ decreases.
\item When $\Im(w) = 0$: The modulus is always $|z|^{\Re(w)}$, lying on a circle with radius $|z|^{\Re(w)}$ centered at the origin.
\end{itemize}
\tb{Angle (parameter $\theta$):}
\[\theta(k)=\Re(w)\qty(\Arg(z)+2\pi k)+\Im(w)\ln|z|.\]
\begin{itemize}
\item When $\Re(w) > 0$: The angle increases linearly with respect to $k$, rotating counterclockwise as $k$ increases and clockwise as $k$ decreases.
\item.When $\Re(w) < 0$: The angle decreases linearly with respect to $k$, rotating clockwise as $k$ increases and counterclockwise as $k$ decreases.
\item When $\Re(w) \in \mathbb{Z}$: The angle is always $\left(\Re(w)\Arg(z)+\Im(w)\ln|z|\right)\mod (2\pi)$, lying one the ray with angle $\Re(w)\Arg(z)+\Im(w)\ln|z|$.
\end{itemize}
\tb{Spiral direction:}
\begin{itemize}
\item When $\Re(w)\Im(w)>0$: The root lies in a clockwise spiral centered at the origin on the complex plane.
\item When $\Re(w)\Im(w)<0$: The root lies in a counterclockwise spiral centered at the origin on the complex plane.
\item When $\Im(w)=0$: The root lies in a circle centered at the origin on the complex plane.
\end{itemize}
\tb{Number of distinct roots:}
\begin{itemize}
\item When $\Im(w)=0\land\Re(w)\in\mathbb{Q}$: Suppose $\abs{\Re(w)}=\frac{m}{n}$ with $n\in\mathbb{N}$ and $\gcd(m,n)=1$, there are $n$ distinct roots, which are the $n$ vertices of a regular $n$-gon inscribed in a circle of radius $|z|^{\Re(w)}$ centered at the origin including $|z|^{\Re(w)}e^{i\Re(w)\Arg(z)}$ on the complex plane.
\item Otherwise: There are countably infinitely many distinct roots.
\end{itemize}
\ssc{Phasor or complex amplitude}
\sssc{Definition)
The phasor or complex amplitude of a sinusoid $A\cos(\omega t+\theta)$ for some real nonnegaative constants $A,\omega,\theta$ is a complex number
\[Ae^{i\theta},\quad i=\sqrt{-1},\]
also denoted as $A\angle\theta$ or a two-dimensional vector
\[(A\cos\theta,A\sin\theta),\]
which satisfies
\[\Re(Ae^{i\theta}e^{i\omega t})=A\cos(\omega t+\theta).\]
\sssc{Linearity}
The phasor of
\[aA\cos(\omega t+\theta)+bB\cos(\omega t+\varphi)\]
is
\[aAe^{i\theta}+bBe^{i\varphi}.\]
\sssc{Differentiation}
The phasor of $\dv{}{t}\qty(A\cos(\omega t+\theta))$ is $i\omega Ae^{i\theta}$.
\sssc{Integration}
The phasor of the sinusoid part of $\int A\cos(\omega\tau+\theta)\dd{\tau}$ is $-\frac{i}{\omega}Ae^{i\theta}$.
\sct{Integral Transform}
PLACEHOLDER
Integral Transform
Linear transform
\ssc{(Unilateral or One-sided) Laplace transform ((單邊)拉普拉斯變換)}
\sssc{Introduction}
The (unilateral or one-sided) Laplace transform is an integral transform that converts a function of a real variable $t$, called time domain, to a complex function of a complex variable $s$, called (complex-valued) frequency domain, $s$-domain, $s$-plane, or Laplace domain. The functions are often denoted in lowercase for the time-domain and uppercase for the frequency-domain.
\sssc{Definition as Lebesgue integral}
For any set $S$, for any complex-valued functions $g_j(t)$ defined on $D_j\superseteq\mathbb{R}_{\geq 0}$, for any indexed family $A_j\colon I_j\to\mathcal{A}_j\subseteq D_j;\;i\mapsto A_{j_i}$ with $I_j=J_j\cap\bbN$ for some interval $J_j$, the Laplace transform of the distribution (often called function)
\[f(t)=\sum_{j\in S}g_j(t)\sum_{i\in I_j}\delta(A_{j_i}),\]
denoted as $\mathcal{L}\qty\{f(t)\}(s)$ or $F(s)$, is defined as
\[\mathcal{L}\qty\{f(t)\}(s)=F(s)=\int_0^{\infty}f(t)e^{-st}\dd{t},\]
with integrals involving Dirac detla functions defined via Dirac measures.

The convergence of a Laplace transform at a value of $s$ is defined as the existence of such Lebesgue integral.
\sssc{Definition as improper Riemann–Stieltjes integral}
For any set $S$, for any complex-valued functions $g_j(t)$ defined on $D_j\superseteq\mathbb{R}_{\geq 0}$, for any indexed family $A_j\colon I_j\to\mathcal{A}_j\subseteq D_j;\;i\mapsto A_{j_i}$ with $I_j=J_j\cap\bbN$ for some interval $J_j$, the Laplace transform of the distribution (often called function)
\[f(t)=\sum_{j\in S}g_j(t)\sum_{i\in I_j}\delta(A_{j_i}),\]
denoted as $\mathcal{L}\qty\{f(t)\}(s)$ or $F(s)$, is defined, with $\mathcal{f}(t)$, typically denoted simply as $f(t)$, defined as any function such that
\bit
\item its domain is a super set of an interval $[a,\infty)$ for some $a\in\bbR_{<0}$,
\item $\mathcal{f}(t)=f(t)$ for all $t\in\bbR_{\geq 0}$, and
\item it's continuous on some interval $[a,0)$ for some $a\in\bbR_{<0}$,
\eit
as,
\[\mathcal{L}\qty\{f(t)\}(s)=F(s)=\lim_{\varepsilon\to 0^+}\int_{-\varepsilon}^{\infty}\mathcal{f}(t)e^{-st}\dd{t},\]
with integrals involving Dirac detla functions defined via improper Riemann–Stieltjes integral, where $\lim_{\varepsilon\to 0^+}\int_{-\varepsilon}^{\infty}$ is typically denoted as $\int_{0^-}^{\infty}$.

The convergence of a Laplace transform at a value of $s$ is defined as the absolute convergence of such improper Riemann–Stieltjes integral.
\sssc{Definition as distribution}
PLACEHOLDER

PLACEHOLDER bilateral move here, below all add for it too
\sssc{Region of convergence (ROC)}
The set of values of $s$ for which $F(s)$ converges, called region of convergence (ROC), is a class of either $\Re(s) > a$ or $\Re(s) \geq a$ for some extended real constant $a$, called abscissa of convergence.
\sssc{Laplace transformability}
A such $f(t)$ is called Laplace-transformable iff there exists some real number $a$ such that $F(a)$ converges.
\sssc{Laplace transformability theorem}
A complex-valued function $f(t)$ defined on $[0,\infty)$ is Laplace-transformable iff it is piecewise continuous on all closed subintervals of $[0,\infty)$ and of exponential order.
\begin{proof}
Assume that a $f(t)$ is piecewise continuous on $[0,\infty)$ and there exists $M>0$ and $T>0$ such that:
\[|f(t)|\leq Me^{\alpha t}\quad \forall t>T.\]

For $s>\alpha$, since $f(t)$ is piecewise continuous on $[0,\infty)$, there exist a locally finite indexed family $A=\{[a_i,b_i]\mid i\in I\}$ such that for each $i\in I\setminus\{j\}$, the integral
\[\int_{a_i}^{b_i} |f(t)| e^{-st}\dd{t}\]
exists and is finite.

Split the integral:
\[\int_0^{\infty}f(t)e^{-st}\dd{t}=\int_0^Tf(t)e^{-st}\dd{t}+\int_T^{\infty}f(t)e^{-st}\dd{t}.\]
For the first integral, by local finiteness, there exists a finite cover of $[0,T]$ that is a subset of $A$. The sum over finitely many finite integral is finite.

For the second integral, since $|f(t)| \le M e^{\alpha t}$, we have
\[\int_T^\infty |f(t)| e^{-st}\dd{t}\le \int_T^\infty M e^{\alpha t} e^{-st}\dd{t} = M \int_T^\infty e^{-(s-\alpha)t}\dd{t},\]
which converges since $\Re(s)>\alpha$.
\end{proof}
For any complex function $g(t)$ defined on $D\superseteq\mathbb{R}_{\geq 0}$, for any indexed family $A\colon I\to\mathcal{A}\subseteq D;\;i\mapsto A_i$ with $I=J\cap\bbN$ for some interval $J\ni 0$, the function
\[f(t)=g(t)\sum_{i\in I}\delta(A_i)\]
is Laplace-transformable iff $g(t)$ is Laplace-transformable and that $\sum_{a\in\mathcal{A}}a$ converges.

If $f(t)$ and $g(t)$ are Laplace-transformable, then $f(t)+g(t)$ is Laplace-transformable.
\sssc{Unit impulses and abscissa of convergence theorem}
For any indexed family $A\colon I\to\mathcal{A}\subseteq D;\;i\mapsto A_i$ with $I=J\cap\bbN$ for some interval $J\ni 1$ and $A_i\leq A_{i+1}$ for all $i\in I\land (i+1)\in I$,
\[\mathcal{L}\qty\{\sum_{i\in I}\delta(t-A_i)\}=\sum_{i\in I}e^{-sA_i},\quad \mathcal{A}\subseteq\bbR_{\geq 0}.\]
The ROC is $\sum_{i\in I}e^{-\Re(s)A_i}<\infty$, which, for $|I|<\aleph_0$, is $\bbC$, for $\abs{\{i\in I\mid A_i=0\}}=\aleph_0$, is $\varnothing$, and otherwise, with
\[N(x)\coloneq\abs{\{i\in I\mid A_i\leq x\}},\]
is,
\[\int_0^{\infty}e^{-\Re(s)x}\dd{N(x)}<\infty,\]
that is,
\[\Re(s)>\limsup_{x\to\infty}\frac{\ln(N(x))}{x},\]
that is,
\bit
\item for $\limsup_{i\to\infty}\frac{\ln(i)}{A_i}=\infty$, $\varnothing$, and
\item for $L\coloneq\limsup_{i\to\infty}\frac{\ln(i)}{A_i}\in\bbR$, $\Re(s)>L$.
\eit
The ROC of
\[\mathcal{L}\qty\{f(t)\sum_{i\in I}\delta(t-A_i)\},\]
where $f(t)$ is a complex Laplace-transformable function, is $\sum_{i\in I}\abs{f\qty(A_i)}e^{-\Re(s)A_i}<\infty$, which, for $|I|<\aleph_0$, is $\bbC$, for $\abs{\{i\in I\mid A_i=0\}}=\aleph_0\land f(0)\neq 0$, is $\varnothing$, and otherwise, with
\[N(x)\coloneq\abs{\{i\in I\mid A_i\leq x\}},\]
is,
\[\int_0^{\infty}f(x)e^{-\Re(s)x}\dd{N(x)}<\infty,\]
that is,
\[\Re(s)>\limsup_{x\to\infty}\frac{\ln(f(x)N(x))}{x},\]
that is,
\bit
\item for $\limsup_{i\to\infty}\frac{\ln(f\qty(A_i)i)}{A_i}=\infty$, $\varnothing$,
\item for $L\coloneq\limsup_{i\to\infty}\frac{\ln(f\qty(A_i)i)}{A_i}\in\bbR$, $\Re(s)>L$, and
\item for $\limsup_{i\to\infty}\frac{\ln(f\qty(A_i)i)}{A_i}=\infty$, $\bbC$.
\eit
\sssc{Linearity}
\[\mathcal{L}\qty\{af(t)+bg(t)\}=a\mathcal{L}\qty\{f(t)\}+b\mathcal{L}\qty\{g(t)\},\quad a,b\in\bbC.\]
The ROC is the intersection of the ROC of $\mathcal{L}\qty\{f(t)\}$ and $\mathcal{L}\qty\{g(t)\}$.
\sssc{Exponential function}
\[\mathcal{L}\qty\{e^{at}\}(s) = \frac{1}{s-a},\quad a\in\bbC.\]
\sssc{First shifting theorem or Frequency shift}
\[\mathcal{L}\qty\{e^{at} f(t)\}(s) = F(s-a),\quad a\in\bbC.\]
The ROC shifts right by $\Re(a)$.
\begin{proof}
\[\mathcal{L}\qty\{e^{at} f(t)\}(s)=\int_0^{\infty}e^{-(s-a)t}f(t)=F(s-a)\]
\end{proof}
\sssc{Unit step function}
\[\mathcal{L}\qty\{u(t-a)\}(s) = \frac{e^{-as}}{s},\quad a\in\bbR_{\geq 0}.\]
\sssc{Second shifting theorem, Time delay, or Time shift}
\[\mathcal{L}\qty\{f(t-a)u(t-a)\}=e^{-as}F(s), \quad a\in\bbR_{\geq 0}.\]
The ROC doesn't change.
\begin{proof}
\[\ba
\mathcal{L}\qty\{f(t-a)u(t-a)\}(s)&=\int_a^{\infty} e^{-st}f(t-a)\dd{t}\\
&=\int_0^{\infty}e^{-s\tau}e^{-as}f(\tau)\dd{\tau}\\
&=e^{-as}F(s)
\ea\]
\end{proof}
\sssc{Differentiation in Time Domain}
\[\mathcal{L}\qty\{f'(t)\}(s) = sF(s) - f(0^-).\]
\[\mathcal{L}\qty\{f^{(n)}(t)\}(s) = s^n F(s) - \sum_{i=0}^{n-1}s^{n-1-i}f^{(i)}(0^-).\]
\begin{proof}
\[u = e^{-st}, \quad \dd{u} = -s e^{-st} \dd{t}.\]
\[\dd{v} = f'(t)\dd{t}, \quad v = f(t).\]
\[\ba
\mathcal{L}\qty\{f'(t)\}(s)&=\int_0^\infty e^{-st} f'(t)\dd{t}\\
&=\qty(e^{-st}f(t))\big\vert_0^\infty+\int_0^\infty f(t)se^{-st}\dd{t}\\
&=s\mathcal{L}\qty\{f(t)\}(s)-f(0^-)
\ea\]
Nota that $0^-$ is required instead of $0$ in the cases where $f'(t)$ contains $\delta(0)$.
\end{proof}
\sssc{Integration in Time Domain}
\[\mathcal{L}\left\{\int_0^t f(\tau)\dd{\tau}\right\}(s) = \frac{1}{s} F(s).\]
\begin{proof}
\bma
\mathcal{L}\qty\{\int_0^t f(\tau)\dd{\tau}\}(s) &= \int_0^{\infty} e^{-st} \int_0^t f(\tau)\dd{\tau}\dd{t}\\
&=\qty(-\frac{1}{s}e^{-st}\int_0^t f(\tau)\dd{\tau})\big\vert_0^\infty+\int_0^{\infty} \frac{1}{s}e^{-st}f(t)\dd{t}\\
&=\frac{1}{s}\int_0^{\infty} e^{-st}f(t)\dd{t}.
\eam
\end{proof}
\sssc{Differentiation in Frequency Domain}
\[\mathcal{L}\qty\{t f(t)\}(s) = -\dv{}{s} F(s).\]
\[\mathcal{L}\qty\{t^n f(t)\}(s) = (-1)^n \dv[n]{}{s} F(s).\]
\begin{proof}
By Leibniz integral rule,
\bma
-\dv{}{s} F(s)&=-\dv{}{s}\int_0^{\infty} e^{-st} f(t)\,\mathrm{d}t\\
&=-\int_0^{\infty} \pdv{}{s}\qty(e^{-st} f(t))\,\mathrm{d}t\\
&=-\int_0^{\infty} -te^{-st} f(t)\,\mathrm{d}t\\
&=\int_0^{\infty} te^{-st} f(t)\,\mathrm{d}t\\
&=\mathcal{L}\qty\{t f(t)\}(s)
\eam
\end{proof}
\sssc{Scaling in Time Domain}
\[\mathcal{L}\qty\{f(at)\}(s) = \frac{1}{a} F\left(\frac{s}{a}\right), \quad a>0.\]
\sssc{Convolution (卷積)}
The convolution of two functions $f(t)$ and $g(t)$ defined on $\bbR_{\geq 0}$ is a function denoted as $(f * g)(t)$ and defined as
\[(f * g)(t) = \int_0^t f(\tau)g(t-\tau)\dd{\tau}.\]
The convolution of two functions $f(t)$ and $g(t)$ defined on $\bbR$ is a function denoted as $(f * g)(t)$ and defined as
\[(f * g)(t) = \int_{-\infty}^{\infty} f(\tau)g(t-\tau)\dd{\tau}.\]
PLACEHOLDER
\sssc{Convolution Theorem}
If $h(t) = (f * g)(t)$, then
\[\mathcal{L}\qty\{h(t)\}(s) = F(s)G(s).\]
\begin{proof}
\[\mathcal{L}\qty\{h(t)\}(s)=\int_0^{\infty} e^{-st} \qty(\int_0^t f(\tau)g(t-\tau)\dd{\tau})\dd{t}\]
By Fubini's theorem,
\[\mathcal{L}\qty\{h(t)\}(s)=\int_0^\infty\int_\tau^\infty e^{-st} f(\tau)g(t-\tau)\dd{t}\dd{\tau}\]
Let $u=t-\tau$. $\dd{t}=\dd{u}$.
\[\begin{aligned}
\mathcal{L}\qty\{h(t)\}(s)&=\int_0^\infty f(\tau)\int_0^\infty e^{-s(u+\tau)} g(u)\dd{u}\dd{\tau}\\
&=\int_0^\infty f(\tau)e^{-s\tau}\int_0^\infty e^{-su} g(u)\dd{u}\dd{\tau}\\
&=F(s)G(s)
\end{aligned}\]
\end{proof}
Note that $\int_0^tf(\tau)\dd{\tau}=(f*u)(t)$.
PLACEHOLDER bilateral
\sssc{Initial Value Theorem}
If $f(t)$ and $f'(t)$ are Laplace-transformable and $f(t)$ contains no unit impulses at origin:
PLACEHOLDER: condition
\[\lim_{t\to 0^+}f(t)=\lim_{s\to\infty}sF(s).\]
\begin{proof}
\[sF(s)=\int_0^\infty sf(t)e^{-st}\dd{t}.\]
Let $u=st$, $\dd{t}=\frac{\dd{u}}{s}$.
\[sF(s)=\int_0^\infty f\qty(\frac{u}{s})e^{-u}\dd{u}.\]
\[\lim_{s\to\infty}sF(s)=\int_0^\infty f\qty(\frac{u}{s})e^{-u}\dd{u}.\]
We define a net of functions $\langle f_s(u)=f\qty(\frac{u}{s})\rangle_{s\in\mathbb{R}_{>s_0}}$, where $s_0$ is such that $F(s)$ converges for all $\Re(s)>s_0$.

For every fixed $u\in\mathbb{R}_{>0}$, $\lim_{s\to\infty}\frac{u}{s}\to 0^+$, so $f_s(u)$ pointwise converges to $f(0^+)$.

For dominated convergence theorem, we require an integrable function $g(u)$ such that
\[\abs{f\qty(\frac{u}{s})e^{-u}}\le g(u),\quad\forall s>0.\]
Since $f(t)$ is Laplace-transformable, it is of exponential order, that is, there exists $\alpha>0$, $M>0$, and $T>0$ such that:
\[|f(t)|\le Me^{\alpha t}\quad\forall t>T.\]
\[Me^{\alpha\frac{u}{s}}e^{-u}=Me^{-u\qty(1-\frac{\alpha}{s})}\le Me^{-\frac{u}{2}},\quad\forall s>2\alpha.\]
By dominated convergence theorem, we obtain:
\[\lim_{s\to\infty}sF(s)=\lim_{n\to\infty}\int_0^\infty f(0^+)e^{-u}\dd{u}=f(0^+)\int_0^\infty e^{-u}\dd{u}=f(0^+).\]
\end{proof}
\sssc{Final Value Theorem (FVT)}
PLACEHOLDER
\sssc{Cosine and sine functions}
\[\mathcal{L}\{\cos(\omega t)\}=\frac{s}{s^2+\omega^2}.\]
\begin{proof}
\[\ba
\mathcal{L}\{\cos(\omega t)\}&=\frac{1}{2}\qty(\mathcal{L}\{e^{i\omega t}\}+\mathcal{L}\{e^{-i\omega t}\})\\
&=\frac{1}{2}\qty(\frac{1}{s-i\omega}+\frac{1}{s+i\omega})\\
&=\frac{1}{2}\frac{2s}{s^2+\omega^2}\\
&=\frac{s}{s^2+\omega^2}
\ea\]
\end{proof}
\[\mathcal{L}\{\sin(\omega t)\}=\frac{\omega}{s^2+\omega^2}.\]
\begin{proof}
\[\ba
\mathcal{L}\{\sin(\omega t)\}&=\frac{1}{2i}\qty(\mathcal{L}\{e^{i\omega t}\}-\mathcal{L}\{e^{-i\omega t}\})\\
&=\frac{1}{2i}\qty(\frac{1}{s-i\omega}-\frac{1}{s+i\omega})\\
&=\frac{1}{2i}\frac{2i\omega}{s^2+\omega^2}\\
&=\frac{\omega}{s^2+\omega^2}
\ea\]
\end{proof}
\[\mathcal{L}\{\cos(\omega t-\varphi)\}=\frac{\cos\varphi s+\sin\varphi\omega}{s^2+\omega^2}.\]
\begin{proof}
\[\ba
\mathcal{L}\{\cos(\omega t-\varphi)\}&=\mathcal{L}\{\cos\varphi\cos(\omega t)+\sin\varphi\sin(\omega t)\}\\
&=\frac{\cos\varphi s+\sin\varphi\omega}{s^2+\omega^2}
\ea\]
\end{proof}
\[\mathcal{L}\{\sin(\omega t-\varphi)\}=\frac{\cos\varphi\omega-\sin\varphi s}{s^2+\omega^2}.\]
\begin{proof}
\[\ba
\mathcal{L}\{\sin(\omega t-\varphi)\}&=\mathcal{L}\{\cos\varphi\sin(\omega t)-\sin\varphi\cos(\omega t)\}\\
&=\frac{\cos\varphi\omega-\sin\varphi s}{s^2+\omega^2}
\ea\]
\end{proof}
\sssc{Power function}
\[\mathcal{L}\{t^q\}=\frac{q!}{s^{q+1}},\quad q\in\bbN_0.\]
\[\mathcal{L}\{t^q\}=\frac{\Gamma(q+1)}{s^{q+1}},\quad\Re(q)>-1.\]
The ROC is $\Re(s)>0$.
\begin{proof}
Let $u=st,\quad\dd{u}=s\dd{t}$.
\[\ba
\mathcal{L}\{t^q\}&=\frac{1}{s}\int_0^{\infty}e^{-u}\qty(\frac{u}{s})^q\dd{u}\\
&=\frac{1}{s^{q+1}}\int_0^{\infty}u^qe^{-u}\dd{u}\\
&=\frac{\Gamma(q+1)}{s^{q+1}}
\ea\]
\end{proof}
PLACEHOLDER: pole of order $\Re(q+1)$ at $s=0$. branches for $q\notin\bbZ$.
\ssc{Mellin's inverse formula, Bromwich integral, or Fourier–Mellin integral of Inverse Laplace transform (反拉普拉斯變換)}
The inverse Laplace transform of a complex function $F(s)$, denoted as $\mathcal{L}^{-1}\{F(s)\}(t)$ or $f(t)$, is a defined as
\[\mathcal{L}^{-1}\{F(s)\}(t)=f(t)=\frac{1}{2\pi i}\lim_{T\to\infty}\int_{\gamma-iT}^{\gamma+iT}e^{st}F(s)\dd{s},\]
called Mellin's inverse formula, Bromwich integral, or Fourier–Mellin integral, where $\gamma$ is any real number such that it is greater than the real part of all singularities of $F$ and that $F$ is bounded on the line $s=\gamma$, and is such that
\[\mathcal{L}\qty\{f(t)\}(s) = F(s).\]
\begin{proof}
PLACEHOLDER
\end{proof}
\ssc{(Bilateral or Two-sided) Laplace Transform ((雙邊)拉普拉斯變換)}
PLACEHOLDER: move up and theorems add both uni and bi versions.

The (bilateral or two-sided) Laplace transform is an integral transform that converts a function of a real variable (usually $t$, in the time domain) to a function of a complex variable (usually $s$, in the complex-valued angular frequency domain, also known as $s$-domain or $s$-plane). The functions are often denoted in lowercase for the time-domain representation and uppercase for the frequency-domain.

The Laplace transform of a real function $f(t)$, denoted as $\mathcal{B}\{f(t)\}(s)$, is defined by the improper integral
\[\mathcal{B}\{f(t)\}(s) = \int_{-\infty}^{\infty} e^{-st} f(t)\,\mathrm{d}t.\]
\ssc{Fourier Transform (FT) (傅立葉變換)}
PLACEHOLDER: as Laplace tranform special case
\sssc{Fourier Transform (FT)}
Fourier transform is an integral transform that converts a function of a real variable (usually $t$, in the time domain) to a function of another real variable (usually $\omega$, in the real-valued (angular) frequency or Fourier domain). The functions are often denoted in lowercase for the time-domain representation and uppercase for the frequency-domain.

The Fourier transform, denoted as $\mathcal{F}\{f(t)\}(\omega)$ or $F(\omega)$, is defined by the improper integral
\[\mathcal{F}\{f(t)\}(\omega) = F(\omega) = \int_{-\infty}^{\infty}e^{-i\omega t}f(t)\dd{t}.\]
\sssc{Inverse Fourier Transform (反傅立葉變換)}
The inverse Fourier transform of a complex function $F(s)$, denoted as $\mathcal{F}^{-1}\{F(s)\}(t)$ or $f(t)$, is defined as a real function such that
\[\mathcal{F}\{f(t)\}(s) = F(s),\]
where $\mathcal {F}$ denotes the Fourier transform.

The inverse Laplace transform of a complex function $F(s)$ is given by the line integral:
\[f(t) = \frac{1}{2\pi}\int_{-\infty}^{\infty} F(\omega) e^{i\omega t}\dd{\omega} .\]

\end{document}