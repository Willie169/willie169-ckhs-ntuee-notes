\documentclass[a4paper,12pt]{article}
\setcounter{secnumdepth}{5}
\setcounter{tocdepth}{3}
\input{/usr/share/LaTeX-ToolKit/template.tex}
\begin{document}
\title{Function}
\author{沈威宇}
\date{\temtoday}
\titletocdoc
\section{Function (函數)}
\subsection{Function, map, or mapping (映射)}
A function, a map, or a mapping is formed by three sets, the domain (定義域) $X$, the codomain (對應域) $Y$, and the graph $R$ that satisfy the three following conditions:
\[R\subseteq \{(x,y)\mid x\in X,y\in Y\},\]
\[\forall x\in X,\exists y\in Y,\left(x,y\right)\in R ,\]
\[ (x,y)\in R\land (x,z)\in R\implies y=z.\]

We say a function is from, over, or on its domain (or sometimes a superset of its domain) and into, to, or in its codomain. We also say a function is between its domain (or sometimes a superset of its domain) and its codomain.

A function $f$ that satisfies the above is denoted as:
\[f\colon X\to Y.\]
in which the domain $X$ is also denoted as $D_f$.

The range (值域) or image (像), denoted as $R_f$ or $f(X)$, is defined as:
\[\{ y \mid \exists x \in X, (x, y) \in R \}.\]

If $x\in X$ and $(x, y) \in R$, we write $y = f(x)$, in which $f(x)$ is called the image of $x$ under $f$, $x$ is called the independent variable (自變數 or 獨立變數) or input, and $y$ is called the dependent variable (應變數 or 依賴變數) or output; and the function $f$ is also denoted as:
\[f \colon X \to Y;\, x \mapsto y.\]

If $I$ is a subset of the domain of $f$, $f(I)$ is defined as
\[f(I)\coloneq\{y\mid\exists x\in I \tx{\ s.t.\ }f(x)=y\}.\]

Let $f\colon X\to Y$ and $I\subeteq X$. Then $f\vert_I$ is defined as a function $f\vert_I\colon I\to Y$ such that for every $x\in I$, $f\vert_I(x)=f(x)$.

Let $f\colon X\subseteq\mathbb{R}\to Y$ and $(a,b)\subeteq X$. Then $f\vert_a^b$ is defined as a function $f\vert_a^b\colon (a,b)\to Y$ such that for every $x\in (a,b)$, $f\vert_a^b(x)=f(x)$.
\ssc{Type of Functions}
Consider a function $f$:
\[f\colon X\to Y;\,x\mapsto y.\]
\sssc{Injection (單射) or Injective function or One-to-one (一對一) function}
\[\forall a,b\in X\text{\ s.t.\ } f(a)=f(b)\colon a=b\]
\sssc{Many-to-one (多對一) function}
\[\exists a\neq b\in X\colon f(a)=f(b)\]
\sssc{Surjection (滿射 or 蓋射) or Surjective function or Onto function}
\[f(X)=Y\]
\sssc{Bijection (對射) or Bijective function or One-to-one (一對一) function or One-to-one correspondence (一一對應)}
Injective and surjective function.
\ssc{Tests of real functions}
\sssc{Vertical line test (垂線測試)}
For a graph on $xy$ plane, if any vertical line intersects it, $y$ is not a function of $x$.
\sssc{Horizontal line test (水平線測試)}
Given a graph of a real function $y=f(x)$ on $xy$ plane, if any horizontal line intersects it more than once, $f(x)$ is not bijective.
\ssc{Fixed point, fixpoint, or invariant point (不動點 or 定點)}
A fixed point, fixpoint, or invariant point of a function $f\colon X\subseteq Y\to Y$, is a point $x\in X$ such that $f(x)=x$.
\ssc{Property in a variable}
Unless otherwise specified, given a property $P$ defined for a function $f(w_1,\dots w_n,x,y_1,\dots y_n)$ in which $w_1,\dots w_n,x,y_1,\dots y_n$ are independent variables, we say $f$ is with property $P$ in $x$ if the function $g(x)$ that is defined to map $x$ to $f(w_1,\dots w_n,x,y_1,\dots y_n)$ for any $x$ such that $w_1,\dots w_n,x,y_1,\dots y_n$ is in the domain of $f$ for any fixed $w_1,\dots w_n,y_1,\dots y_n$ is with property $P$.

Unless otherwise specified, given a property $P$ defined for a function $f(w_1,\dots w_n,x,y_1,\dots y_n)$ in which $w_1,\dots w_n,x,y_1,\dots y_n$ are independent variables, we say $f$ is with property $P$ in $x$ for a fixed $w_1,\dots w_n,y_1,\dots y_n$ if the function $g(x)$ that is defined to map $x$ to $f(w_1,\dots w_n,x,y_1,\dots y_n)$ for any $x$ such that $w_1,\dots w_n,x,y_1,\dots y_n$ is in the domain of $f$ is with property $P$.

Unless otherwise specified, given a property $P$ defined for a function $f(w_1,\dots w_n,x,y_1,\dots y_n)$ in which $w_1,\dots w_n,x,y_1,\dots y_n$ are independent variables, we say $f$ is with property $P$ in $x$ at a point $w_1,\dots w_n,x_0,y_1,\dots y_n$ in the domain of $f$ if there exists a neighbourhood $X$ of $x_0$ such that the function $g(x)$ that is defined to map $x$ to $f(w_1,\dots w_n,x,y_1,\dots y_n)$ for any $x\in X$ such that $w_1,\dots w_n,x,y_1,\dots y_n$ is in the domain of $f$ is with property $P$.
\ssc{Increasing and Decreasing Real Functions}
Consider a function $f$
\[f\colon X\subseteq Y\to Z\]
with $Y,Z$ being posets.
\sssc{Non-decreasing (非遞減) or (Monotone) increasing ((單調)遞增) function}
$f$ is non-decreasing on $I\subseteq X$ if and only if
\[\forall a,b\in I\colon a<b\implies f(a)\leq f(b).\]
$f$ is non-decreasing if and only if
\[\forall a,b\in X\colon a<b\implies f(a)\leq f(b).\]
\sssc{Strictly increasing (嚴格遞增) function}
$f$ is strictly increasing on $I\subseteq X$ if and only if
\[\forall a,b\in I\colon a<b\implies f(a)<f(b).\]
$f$ is strictly increasing if and only if
\[\forall a,b\in X\colon a<b\implies f(a)<f(b).\]
\sssc{Non-increasing (非遞增) or (Monotone) decreasing ((單調)遞減) function}
$f$ is non-increasing on $I\subseteq X$ if and only if
\[\forall a,b\in I\colon a<b\implies f(a)\geq f(b).\]
$f$ is non-increasing if and only if
\[\forall a,b\in X\colon a<b\implies f(a)\geq f(b).\]
\sssc{Strictly decreasing (嚴格遞減) function}
$f$ is strictly decreasing on $I\subseteq X$ if and only if
\[\forall a,b\in I\colon a<b\implies f(a)>f(b).\]
$f$ is strictly decreasing if and only if
\[\forall a,b\in X\colon a<b\implies f(a)>f(b).\]
\sssc{Monotone (單調) function}
$f$ is monotone on $I\subseteq X$ if and only if it is either non-increasing or non-decreasing on $I$.

$f$ is monotone if and only if it is either non-increasing or non-decreasing.
\ssc{Homogeneous}
A function $f$ with domain $\Omega\subseteq\bbR^n$ is called a homogeneous function of degree $\alpha\in\bbR$ if and only if for any nonzero real number $t$ and $\mathbf{x},t\mathbf{x}\in\Omega$, $f(t\mathbf{x})=t^{\alpha}f(\mb{x})$.
\ssc{Of Exponential Order}
A real or complex-valued net $f$ on a subset of $\mathbb{R}$ is considered of exponential order $\alpha\in\mathbb{R}_{>0}$ if there exists $M>0$ and $T>0$ such that:
\[|f(t)|\leq Me^{\alpha t}\quad \forall t>T.\]
\ssc{Symmetry}
Given a function $f\colon X\to Y$ such that $\forall x\in X\colon -x\in X$.
\sssc{Even Function (偶函數)}
$f$ is an even function if for all $x\in X$, $f(x)=f(-x)$.
\sssc{Odd Function (奇函數)}
$f$ is an odd function if for all $x\in X$, $-f(x)=f(-x)$.
\ssc{Transformation (變換)}
\sssc{Translation (平移)}
For any function $f\colon\mathbb{R}\to\mathbb{R}$, shifting $y=f(x)$ right by $h$ units and up by $k$ units on the $xy$ coordinate plane yields $y=f(x-h)+k$.
\sssc{Scaling (伸縮 or 縮放 or 拉伸)}
For any function $f\colon\mathbb{R}\to\mathbb{R}$, on the $xy$ coordinate plane, expand $y=f(x)$ vertically by $a$ times the original value with the $x$ axis as the reference line, and expand $y=af\qty(\frac{x}{b})$ horizontally by $b$ times the original value with the $y$ axis as the reference line, to obtain $y=af\qty(\frac{x}{b})$.
\ssc{Function composition (函數合成)}
For two functions $f\colon X\to Y$ and $g\colon V\to W$ such that $g(V)\subseteq X$, the composition of them, denoted as $(f\circ g)$, is defined as:
\[(f \circ g)\colon V\to Y;\,x\mapsto = f(g(x))\]
\ssc{Inverse function (反函數)}
For a bijective function $f\colon X\to Y$, the inverse of it, denoted as $f^{-1}$, is defined as:
\[f^{-1}\colon Y\to X;\,f(x)\mapsto x\]
\ssc{Preimage (像原) or Inverse image}
Given an open subset $B$ of the codomain of a function $f$ the preimage or inverse image of $f$ on $B$, denoted as $f^{-1}(B)$ is defined as
\[f^{-1}(B)=\{x\in X|f(x)\in B\}.\]
\ssc{Extension (延拓) of a function}
We say a function $f\colon X\to Y$ is an extension of another function $g\colon U\subseteq X\to Y$ if $\forall u\in U\colon f(u)=g(u)$.
\ssc{Piecewise function (分段函數)}
A piecewise function is a function defined in the form:
\[f(x) =
\begin{cases}
f_1(x), & \quad x\in A_1, \\
f_2(x), & \quad x\in A_2, \\
\dots  \\
f_n(x), & \quad x \in A_n
\end{cases},\]
where
\[\bigcup_{i=1}^nA_i=D_f\land\forall i\neq j\land i,j\in\mathbb{N}\land i,j\leq n\colon A_i\cap A_j=\varnothing.\]
\subsection{Indicator function (指示函數或示性函數) or characteristic function (特徵函數)}
An indicator function or a characteristic function of a subset $A$ of a set $X$ is a function that maps elements of the subset to one, and all other elements to zero, often denoted as $1_A$.
\ssc{Elementary functions}
A constants, a power function, a logarithmic function, an exponential function, a trigonometric function, an inverse trigonometric function, or any function that can be built up with finite defined operations of addition, subtraction, multiplication, division, and composition of functions on any finite set of elementary functions is called an elementary function.

An integral of which the integrand is an elementary function is called an elementary integral.
\ssc{Real Algebraic Function (實代數函數)}
\sssc{Power Function (冪函數)}
A power function if a function $f(x)=kx^a$ where $k\in\mathbb{C}$, called coefficient, and $a\in\mathbb{R}$, called exponent, are constants (some requires $k\in\mathbb{R}$ or $k=1$).
\sssc{Root Function (根式函數)}
A root function is a power function of which the exponent is $\frac{1}{n}$ where $n\in\mathbb{N}$.
\sssc{Reciprocal Function}
The reciprocal function is the function $f(x)=\frac{1}{x}$.
\sssc{Polynomial (多項式)}
A polynomial is a function $P(x)=\sum_{k=0}^na_kx^k$ where $a_k\in\mathbb{R}$ or $\mathbb{C}$ are constants called coefficients and $n$ is a nonnegative integer called the degree of $P$.
\sssc{Rational Function (有理函數)}
A function $f$ is called a rational function if there exists two polynomials $P(x)$ and $Q(x)$ such that $f(x)=\frac{P(x)}{Q(x)}$ for all $x$ in the domain.
\sssc{Algebraic Function (代數函數)}
A function $f$ is called an algebraic function if there exists a polynomial $P(y,x)$ in two variables such that $P(f(x),x)=0$ for all $x$ in the domain.
\ssc{Piecewise functions}
\sssc{Sign function (符號函數)}
The sign function $\sgn\colon\bbR\to\bbR$ is defined as:
\[\sgn(x)=\bcs
-1,\quad & x<0\\
0,\quad & x=0\\
1,\quad & x>0
\ecs\]
\sssc{Floor function (下取整函數), integral part, integer part (整數部份), greatest integer,or entier}
The floor function $\lfloor\cdot\rfloor\colon\bbR\to\bbR$ or $[\cdot]\colon\bbR\to\bbR$ is defined as:
\[\lfloor x\rfloor=[x]=\max\{m\in\bbZ\mid m\leq x\},\]
in which $[\cdot]$ is called the Gauss sign (高斯符號).
\sssc{Ceiling function (上取整函數)}
The ceiling function $\lceil\cdot\rceil\colon\bbR\to\bbR$ is defined as:
\[\lceil x\rceil=\min\{m\in\bbZ\mid m\geq x\}.\]
\sssc{Heaviside step function (黑維塞階躍函數) or Unit step function (單位階躍函數)}
\[H(x)=u(x)=
\begin{cases}0,\quad &x<0\\
1,\quad &x\geq 0
\end{cases},\]
where $H(0)=u(0)$ is sometimes defined to be $\frac{1}{2}$ or other values instead.
\[H'(x)=u'(x)=\delta(x).\]
\sssc{Dirac delta function (狄拉克 δ 函數)}
\[
\delta(x) =
\begin{cases}
\infty, & x = 0, \\
0, & x \neq 0.
\end{cases}
\]
\[
\forall \epsilon > 0:\,\int_{a-\epsilon}^{a+\epsilon} f(x) \delta(x-a) \, \mathrm{d}x = f(a).
\]
\[
\int _{-\infty }^{\infty }\delta(x) \, \mathrm{d}x = 1.
\]
\[\int_0^x\delta(t)\,\mathrm{d}t=H(x).\]
\ssc{Error function}
\sssc{Error funcction (誤差函數) or Gauss error function (高斯誤差函數)}
The error function or Gauss error function $\erf\colon\bbC\to\bbC$ or $\erf\colon\bbR\to\bbR$ is defined as:
\[\erf(z)=\frac{2}{\sqrt{\pi}}\int_0^ze^{-t^2}\dd{t}.\]
\sssc{Complementary error function (互補誤差函數)}
The complementary error function $\erfc\colon\bbC\to\bbC$ or $\erfc\colon\bbR\to\bbR$ is defined as:
\[\erfc(z)=1-\erf(z)=\int_z^{\infty}e^{-t^2}\dd{t}.\]
\sssc{Imaginary error function (虛誤差函數)}
The imaginary error function $\erfi\colon\bbC\to\bbC$ or $\erfi\colon\bbR\to\bbR$ is defined as:
\[\erfi(z)=-i\erf(iz).\]
\ssc{Sequences (數列)}
\begin{itemize}
\item\textbf{Sequence}: A sequence in $X$ is a function of which the domain is a set $\{x\in\mathbb{Z}\mid l\leq x\leq m\}$ or $\{x\in\mathbb{Z}\mid l\leq x\}$ and the codomain is a topological space $X$, in which $l$ is an integer, usually $0$ or $1$, and $m$ is a integer, denoted as $\langle a_n\rangle$, $\{a_n\}$, $(a_n)$, $\langle a_n\rangle_{n=l}^m$, $\{a_n\}_{n=l}^m$, or $(a_n)_{n=l}^m$, with $m=\infty$ when its domain is $\{x\in\mathbb{Z}\mid l\leq x\}$, where the subscript $n$ refers to the $n$th element of the sequence, that is, the function value when the  independent variable is $n$.
\item\textbf{Finite Sequence (有限數列)}: A finite sequence is a sequence with finite terms, i.e. $\langle a_n\rangle_{n=l}^m,\quad m\in\mathbb{Z}$.
\item\textbf{Infinite sequence (無窮數列)}: An infinite sequence is a sequence with infinite terms, i.e. $\langle a_n\rangle_{n=l}^\infty$. Unless otherwise specified, sequences refer to infinite sequences.
\end{itemize}
\ssc{Series (級數)}
\begin{itemize}
\item\textbf{Series}: The sum of the terms of a sequence.
\item\textbf{Finite Series (有限級數)}: The sum of the terms of a finite sequence.
\item\textbf{Infinite Series (無窮級數)}: The sum of the terms of an infinite sequence.
\end{itemize}
\ssc{List of Sequences}
\sssc{Arithmetic progression or sequence (等差數列)}
An arithmetic sequence is a sequence $\langle a_n\rangle=\langle a_1+(n-1)d\rangle$. 

Given $a$ and $b$, $\frac{a+b}{2}$ is called the median of an arithmetic sequence (等差中項).

\[\nexists\lim_{n\to\infty}a_n,\quad d\neq 0\]
\[\lim_{n\to\infty}a_n=a_1,\quad d=0\]
\sssc{Geometric progression or sequence (等比 or 幾何數列)}
A geometric sequence is a sequence $\langle a_n\rangle=\langle a_1\cdot r^{n-1}\rangle$, where $a_1r\neq 0$.

Given $a$ and $b$, $\pm\sqrt{ab}$ is called the median of an geometric sequence (等比中項).
\[\nexists\lim_{n\to\infty}a_n,\quad |d|\geq 1\land d\neq 1\]
\[\lim_{n\to\infty}a_n=a_1,\quad d=1\]
\[\lim_{n\to\infty}a_n=0,\quad |d|<1\]
\ssc{List of Series}
\sssc{Arithmetic series (等差級數)}
An arithmetic series is a series $S_n=\sum_{i=1}^na_i$, where $\langle a_n\rangle$ is an arithmetic sequence.
\[S_n=\frac{n}{2}\qty(a_1+a_n)=\frac{n}{2}\qty(2a_1+(n-1)d)=na_1+\frac{n(n-1)d}{2}\]
\[\nexists\lim_{n\to\infty}S_n,\quad a_1\neq 0\lor d\neq 0\]
\[\lim_{n\to\infty}S_n=0,\quad a_1=0\land d=0\]
\sssc{Geometric series (等比 or 幾何級數)}
A geometric series is a series $S_n=\sum_{i=1}^na_i$, where $\langle a_n\rangle$ is a geometric sequence.

\[S_n=\frac{a_1\qty(1-r^n)}{1-r},\quad r\neq 1\]
\[S_n=na_1,\quad r=1\]
\[\lim_{n\to\infty}S_n=\frac{a_1}{1-r},\quad \abs{r}<1\]
\[\nexists\lim_{n\to\infty}S_n,\quad \abs{r}\geq 1\]
\sssc{Riemann zeta function (黎曼 zeta 函數)}
\[\begin{aligned}
\zeta(s) &= \sum_{n=1}^\infty\frac{1}{n^s}\\
&= \frac{1}{\Gamma (s)}\int _0^\infty \frac {x^{s-1}}{e^x-1}\,\mathrm {d} x
\eam

Harmonic Series (調和級數):
\[S_n=\sum_{n=1}^n\frac{1}{n}\]
\[\nexists\sum_{n=1}^\infty\frac{1}{n}\]

Basel Problem (巴塞爾問題):
\[\zeta(2)=\frac{\pi^2}{6}\]

Other Even Positive Integers:
\[\zeta(4)=\frac{\pi^4}{90}\]
\[\zeta(6)=\frac{\pi^6}{945}\]
\[\zeta(8)=\frac{\pi^8}{9450}\]
\[\zeta(10)=\frac{\pi^{10}}{93555}\]
\[\zeta(12)=\frac{691\pi^{12}}{638512875}\]
\[\zeta(14)=\frac{2\pi^{14}}{18243225}\]

Infinity:
\[\lim_{n\to\infty}\zeta(n)=1\]
\sssc{Euler–Mascheroni constant (歐拉–馬斯克若尼常數)}
\[\begin{aligned}
\gamma &= \lim _{n\to \infty }\left(\left(\sum _{k=1}^n\frac {1}{k}\right)-\ln(n)\right)\\
&= \int _1^\infty \left(\frac{1}{\lfloor x\rfloor}-\frac{1}{x}\right)\,\mathrm{d}x
\end{aligned}\]
\sssc{Power series (冪級數)}
\[\begin{aligned}
\sum_{i=1}^ni &= \frac{n\qty(n+1)}{2}\\
\sum_{i=1}^ni^2 &= \frac{n\qty(n+1)\qty(2n+1)}{6}\\
\sum_{i=1}^ni^3 &= \qty(\frac{n(n+1)}{2})^2\\
\sum_{i=1}^ni^r &= n + \sum_{k=1}^{n-1} (n-k)((k+1)^r - k^r)\\
&= n + \sum_{k=1}^{n-1} (n-k)\sum_{j=0}^{r-1}\binom{r}{j}k^{j}
\end{aligned}\]
\end{document}
