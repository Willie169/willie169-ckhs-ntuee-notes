\documentclass[a4paper,12pt]{article}
\setcounter{secnumdepth}{5}
\setcounter{tocdepth}{3}
\input{/usr/share/LaTeX-ToolKit/template.tex}
\begin{document}
\title{Function}
\author{沈威宇}
\date{\temtoday}
\titletocdoc






\sct{Constants and Functions Defined by Integrals}
\ssc{Euler's constant or Euler–Mascheroni constant}
PLACEHOLDER
\ssc{Error function}
\sssc{Error funcction (誤差函數) or Gauss error function (高斯誤差函數)}
The error function or Gauss error function $\erf\colon\bbC\to\bbC$ or $\erf\colon\bbR\to\bbR$ is defined as:
\[\erf(z)=\frac{2}{\sqrt{\pi}}\int_0^ze^{-t^2}\dd{t}.\]
\sssc{Complementary error function (互補誤差函數)}
The complementary error function $\erfc\colon\bbC\to\bbC$ or $\erfc\colon\bbR\to\bbR$ is defined as:
\[\erfc(z)=1-\erf(z)=\int_z^{\infty}e^{-t^2}\dd{t}.\]
\sssc{Imaginary error function (虛誤差函數)}
The imaginary error function $\erfi\colon\bbC\to\bbC$ or $\erfi\colon\bbR\to\bbR$ is defined as:
\[\erfi(z)=-i\erf(iz).\]
\ssc{Lambert W function, omega function, or product logarithm}
Lambert W function, omega function, or product logarithm $W(z)$ is defined implicitly by
\[W(z)e^{W(z)}=z,\quad z\in\bbC.\]
Each branch $k$, starting from $0$ as the principal branch, is denoted as $W_k(z)$.

In the reals, there are two branches: $W_0(x)\colon[-\frac{1}{e},\infty)\to\bbR$ is the inverse of
\[xe^x,\quad & x\geq-1.\]
$W_{-1}(x)\colon[-\frac{1}{e},0)\to\bbR$ is the inverse of
\[xe^x,\quad & x\leq-1.\]
\ssc{Fresnel Integrals}
Fresnel integrals $S(x)$ and $C(x)$, and their auxiliary functions $F(x)$ and $G(x)$ are defined as
\[S(x)=\int_0^x\sin(t^2)\dd{t},\]
\[C(x)=\int_0^x\cos(t^2)\dd{t},\]
\[F(x)=\qty(\frac{1}{2}-S(x))\cos(x^2)-\qty(\frac{1}{2}-C(x))\sin(x^2),\]
\[F(x)=\qty(\frac{1}{2}-S(x))\sin(x^2)+\qty(\frac{1}{2}-C(x))\cos(x^2),\]
The parametric curve $(S(t),C(t))$ is the Euler spiral or clothoid, a curve whose curvature varies linearly with arclength.

The Auxiliary functions $F(x)$ and $G(x)$ provide monotonic bounds for the Fresnel Integrals:
\[\frac{1}{2}-F(x)-G(x)\leq S(x)\leq\frac{1}{2}+F(x)+G(x),\]
\[\frac{1}{2}-F(x)-G(x)\leq C(x)\leq\frac{1}{2}+F(x)+G(x).\]
\begin{proof}
PLACEHOLDER
\end{proof}
\ssc{Trigonometric and Hyperbolic Integrals}
\sssc{Sine Integral}
The two different sine integral definitions are
\[\operatorname{Si}(x)=\int_0^x\frac{\sin(t)}{t}\dd{t},\]
\[\operatorname{si}(x)=-\int_x^\infty\frac{\sin(t)}{t}\dd{t}.\]
Their difference is
\[\operatorname{Si}(x)-\operatorname{si}(x)=\int_0^\infty\frac{\sin(t)}{t}\dd{t}=\frac{\pi}{2}.\]
\begin{proof}
PLACEHOLDER
\end{proof}
\sssc{Cosine Integral}
The two different cosine integral definitions are
\[\operatorname{Cin}(x)=\int_0^x\frac{1-\cos(t)}{t}\dd{t},\]
\[\operatorname{Ci}(x)=-\int_x^\infty\frac{\cos(t)}{t}\dd{t}.\]
Their sum is
\[\operatorname{Cin}(x)+\operatorname{Ci}(x)=\ln(x)+\gamma.\]
where $\gamma$ is the Euler–Mascheroni constant.
\begin{proof}
PLACEHOLDER
\end{proof}


\end{document}
