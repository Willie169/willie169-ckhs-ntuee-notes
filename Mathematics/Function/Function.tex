\documentclass[a4paper,12pt]{article}
\setcounter{secnumdepth}{5}
\setcounter{tocdepth}{3}
\input{/usr/share/LaTeX-ToolKit/template.tex}
\begin{document}
\title{Function}
\author{沈威宇}
\date{\temtoday}
\titletocdoc
\section{Function (函數)}
\ssc{Variable (變數) or indeterminate}
A variable or an indeterminate is a symbol, typically a letter, that refers to an unspecified mathematical object.
\ssc{Function, map, or mapping (映射)}
A function, a map, or a mapping is formed by three sets, the domain (定義域) $X$, the codomain (對應域) $Y$, and the graph $R$ that satisfy the three following conditions:
\[R\subseteq \{(x,y)\mid x\in X,y\in Y\},\]
\[\forall x\in X,\exists y\in Y,\left(x,y\right)\in R ,\]
\[ (x,y)\in R\land (x,z)\in R\implies y=z.\]

We say a function is from, over, or on its domain (or sometimes a superset of its domain) and into, to, or in its codomain. We also say a function is between its domain (or sometimes a superset of its domain) and its codomain.

A function $f$ that satisfies the above is denoted as:
\[f\colon X\to Y.\]
in which the domain $X$ is also denoted as $D_f$.

The range (值域) or image (像), denoted as $R_f$ or $f(X)$, is defined as:
\[\{ y \mid \exists x \in X, (x, y) \in R \}.\]

If $x\in X$ and $(x, y) \in R$, we write $y = f(x)$, in which $f(x)$ is called the image of $x$ under $f$, $x$ is called the independent variable (自變數 or 獨立變數) or input, and $y$ is called the dependent variable (應變數 or 依賴變數) or output; and the function $f$ is also denoted as:
\[f \colon X \to Y;\, x \mapsto y.\]

If $I$ is a subset of the domain of $f$, $f(I)$ is defined as
\[f(I)\coloneq\{y\mid\exists x\in I \tx{\ s.t.\ }f(x)=y\}.\]

A function whose domain is a subset of $\bbR$ is called a real-domain function. A function whose range is a subset of $\bbR$ is called a real-valued function. A function whose domain and range are both subsets of $\bbR$ is called a real function.
\ssc{Types of Functions}
Consider a function $f$:
\[f\colon X\to Y;\,x\mapsto y.\]
\sssc{Injection (單射) or Injective function or One-to-one (一對一) function}
\[\forall a,b\in X\text{\ s.t.\ } f(a)=f(b)\colon a=b\]
\sssc{Many-to-one (多對一) function}
\[\exists a\neq b\in X\colon f(a)=f(b)\]
\sssc{Surjection (滿射 or 蓋射) or Surjective function or Onto function}
\[f(X)=Y\]
\sssc{Bijection (對射) or Bijective function or One-to-one (一對一) function or One-to-one correspondence (一一對應)}
Injective and surjective function.
\ssc{Partial Function (偏函數)}
A partial function from $X$ to $Y$ is a function from a subset of $X$ to $Y$. The notation of a partial function is the same as a function except the fact that it is a partial function should be specified.
\ssc{Arity (元數), valency, or valence}
The number of arguments taken by a function. A function with arity of zero is called nullary, a function with arity of one is called unary, and a function with arity of two is called binary.
\ssc{Restriction and extension (延拓)}
We say that a function $f\colon X\to Y$ is an extension of another function $g\colon U\subseteq X\to Y$ and that $g$ is a restriction of $f$ iff $\forall u\in U\colon f(u)=g(u)$.

Let $f\colon X\to Y$ be a function and $I$ be a set. $f\vert_I$ or $f\upharpoonright I$ is defined as a function $f\vert_I\colon I\cap X\to Y$ such that for every $x\in I\cap X$, $f\vert_I(x)=f(x)$.
\ssc{Property in a variable}
Unless otherwise specified, given a property $P$ defined for a function $f(w_1,\dots w_n,x,y_1,\dots y_n)$ in which $w_1,\dots w_n,x,y_1,\dots y_n$ are independent variables, we say $f$ is with property $P$ in $x$ if the function $g(x)$ that is defined to map $x$ to $f(w_1,\dots w_n,x,y_1,\dots y_n)$ for any $x$ such that $w_1,\dots w_n,x,y_1,\dots y_n$ is in the domain of $f$ for any fixed $w_1,\dots w_n,y_1,\dots y_n$ is with property $P$.

Unless otherwise specified, given a property $P$ defined for a function $f(w_1,\dots w_n,x,y_1,\dots y_n)$ in which $w_1,\dots w_n,x,y_1,\dots y_n$ are independent variables, we say $f$ is with property $P$ in $x$ for a fixed $w_1,\dots w_n,y_1,\dots y_n$ if the function $g(x)$ that is defined to map $x$ to $f(w_1,\dots w_n,x,y_1,\dots y_n)$ for any $x$ such that $w_1,\dots w_n,x,y_1,\dots y_n$ is in the domain of $f$ is with property $P$.

Unless otherwise specified, given a property $P$ defined for a function $f(w_1,\dots w_n,x,y_1,\dots y_n)$ in which $w_1,\dots w_n,x,y_1,\dots y_n$ are independent variables, we say $f$ is with property $P$ in $x$ at a point $w_1,\dots w_n,x_0,y_1,\dots y_n$ in the domain of $f$ if there exists a neighbourhood $X$ of $x_0$ such that the function $g(x)$ that is defined to map $x$ to $f(w_1,\dots w_n,x,y_1,\dots y_n)$ for any $x\in X$ such that $w_1,\dots w_n,x,y_1,\dots y_n$ is in the domain of $f$ is with property $P$.
\ssc{Indicator function (指示函數或示性函數) or characteristic function (特徵函數)}
An indicator function or a characteristic function of a subset $A$ of a set $X$ is a function that maps elements of the subset to one, and all other elements to zero, often denoted as $1_A$.
\ssc{(Indexed) family}
An indexed family is a function $f$ from a set $I$ called index set to a set. $f(i)$ for any $i\in I$ is denoted as $x_i$. The image of $f$ is denoted as $(x_i)_{i\in I}$ or $\{x_i\}_{i\in I}$, or $(x_i)$ or $\{x_i\}$ if $I$ is assumed to be known, often $\bbN_0$ or $\bbN$.
\ssc{Sequences (數列)}
\sssc{Sequence}
A sequence in $X$ is a function of which the domain is a set $\{x\in\mathbb{Z}\mid l\leq x\leq m\}$ or $\{x\in\mathbb{Z}\mid l\leq x\}$ and the codomain is a topological space $X$, in which $l$ is an integer, usually $0$ or $1$, and $m$ is a integer, denoted as $\langle a_n\rangle$, $\{a_n\}$, $(a_n)$, $\langle a_n\rangle_{n=l}^m$, $\{a_n\}_{n=l}^m$, or $(a_n)_{n=l}^m$, with $m=\infty$ when its domain is $\{x\in\mathbb{Z}\mid l\leq x\}$, where the subscript $n$ refers to the $n$th element of the sequence, that is, the function value when the  independent variable is $n$.
\sssc{Finite Sequence (有限數列)}
A finite sequence is a sequence with finite terms, i.e. $\langle a_n\rangle_{n=l}^m,\quad m\in\mathbb{Z}$.
\sssc{Infinite sequence (無窮數列)}
An infinite sequence is a sequence with infinite terms, i.e. $\langle a_n\rangle_{n=l}^\infty$. Unless otherwise specified, sequences refer to infinite sequences.
\sssc{Sequence of partial sums}
The sequence of partial sums of a sequence $\langle a_n\rangle_{n=l}^{\infty}$ is the sequence $\left\langle s_n=\sum_{i=l-1}^na_i\right\rangle_{n=l-1}^{\infty}$ where $s_{l-1}=0$ or $\left\langle s_n=\sum_{i=l}^na_i\right\rangle_{n=l}^{\infty}$.
\ssc{Series (級數)}
\sssc{Series}
The sequence of partial sums of a sequence.
\sssc{Finite Series (有限級數)}
The sequence of partial sums of a finite sequence.
\sssc{Infinite Series (無窮級數)}
The sequence of partial sums of an infinite sequence.
\sssc{Cauchy product}
\[\qty(\sum_{i=0}^{\infty}a_i)\qty(\sum_{j=0}^{\infty}b_j)=\sum_{k=0}^{\infty}\sum_{l=0}^ka_lb_{k-l}.\]
\sssc{Power series}
A power series in $x^n$ is a series in the form
\[\sum_{n=0}^{\infty}a_nx^n.\]
\ssc{Evaluation}
Let $f$ be a function. We denote $f(b)-f(a)$ as $\evlv{f(x)}_a^b$, $\evlv[f(x)]_a^b$, $\evlv(f(x))_a^b$, $\evlp{f(x)}_a^b$, $\evlp[f(x)]_a^b$, $\evlp(f(x))_a^b$, $\evlb{f(x)}_a^b$, $\evlb[f(x)]_a^b$, or $\evlb(f(x))_a^b$.
\ssc{Tests of real functions}
\sssc{Vertical line test (垂線測試)}
For a graph on $xy$ plane, if any vertical line intersects it, $y$ is not a function of $x$.
\sssc{Horizontal line test (水平線測試)}
Given a graph of a real function $y=f(x)$ on $xy$ plane, if any horizontal line intersects it more than once, $f(x)$ is not bijective.
\ssc{Fixed point, fixpoint, or invariant point (不動點 or 定點)}
A fixed point, fixpoint, or invariant point of a function $f\colon X\subseteq Y\to Y$, is a point $x\in X$ such that $f(x)=x$.
\sssc{Increasing (遞增) Decreasing (遞減), and Monotone (單調)}
Let $f\colon X\subseteq\bbR\to\bbR$ be a function.
\bit
\item $f$ is called strictly increasing on an interval $I\subseteq X$ if
\[\forall a,b\in I\colon a<b\implies f(a)<f(b).\]
\item $f$ is called strictly increasing if
\[\forall a,b\in X\colon a<b\implies f(a)<f(b).\]
\item $f$ is called strictly decreasing on an interval $I\subseteq X$ if
\[\forall a,b\in I\colon a<b\implies f(a)>f(b).\]
\item $f$ is called strictly decreasing if
\[\forall a,b\in X\colon a<b\implies f(a)>f(b).\]
\item $f$ is called non-decreasing or monotone increasing on an interval $I\subseteq X$ if
\[\forall a,b\in I\colon a<b\implies f(a)\leq f(b).\]
\item $f$ is called non-decreasing or monotone increasing if
\[\forall a,b\in X\colon a<b\implies f(a)\leq f(b).\]
\item $f$ is called non-increasing or monotone decreasing on an interval $I\subseteq X$ if
\[\forall a,b\in I\colon a<b\implies f(a)\geq f(b).\]
\item $f$ is called non-increasing or monotone decreasing if
\[\forall a,b\in X\colon a<b\implies f(a)\geq f(b).\]
\item $f$ is called monotone or monotonic on an interval $I\subseteq X$ if it is either monotone increasing or monotone decreasing on $I$.
\item $f$ is called monotone or monotonic if it is either monotone increasing or monotone decreasing.
\eit
The terms increasing and decreasing are define to be either monotone or strictly definition in different sources. Here, we use the monotone definition.
\ssc{Homogeneous function}
\sssc{General homogeneity}
A function $F$ with domain $\Omega\subseteq\bbR^n$ is called a homogeneous function of degree $\alpha\in\bbR$ if and only if for any nonzero real number $t$ and $\mathbf{x},t\mathbf{x}\in\Omega$, $F(t\mathbf{x})=t^{\alpha}F(\mb{x})$.
\sssc{Positive homogeneity}
A function $F$ with domain $\Omega\subseteq\bbR^n$ is called a positively homogeneous function of degree $\alpha\in\bbR$ if and only if for any positive real number $t$ and $\mathbf{x},t\mathbf{x}\in\Omega$, $F(t\mathbf{x})=t^{\alpha}F(\mb{x})$.
\ssc{Symmetry}
Given a function $f\colon X\to Y$ such that $\forall x\in X\colon -x\in X$.
\sssc{Even Function (偶函數)}
$f$ is an even function if for all $x\in X$, $f(x)=f(-x)$.
\sssc{Odd Function (奇函數)}
$f$ is an odd function if for all $x\in X$, $-f(x)=f(-x)$.
\ssc{Transformation (變換)}
\sssc{Translation (平移)}
For any function $f\colon\mathbb{R}\to\mathbb{R}$, shifting $y=f(x)$ right by $h$ units and up by $k$ units on the $xy$ coordinate plane yields $y=f(x-h)+k$.

$f(x-a)$ is also denoted as $(\tau_af)(x)$, where $\tau_a$ is translation operator.
\sssc{Scaling (伸縮 or 縮放 or 拉伸)}
For any function $f\colon\mathbb{R}\to\mathbb{R}$, on the $xy$ coordinate plane, expand $y=f(x)$ vertically by $a$ times the original value with the $x$ axis as the reference line, and expand $y=af\qty(\frac{x}{b})$ horizontally by $b$ times the original value with the $y$ axis as the reference line, to obtain $y=af\qty(\frac{x}{b})$.

$f(-x)$, called reflection of $f$, is also denoted as $\tilde{f}$ or $\hat{R}f$, where $\tilde$ or $\hat{R}$ is reflection operator.
\ssc{Function composition (函數合成)}
For two functions $f\colon X\to Y$ and $g\colon V\to W$ such that $g(V)\subseteq X$, the composition of them, denoted as $(f\circ g)$, is defined as:
\[(f \circ g)\colon V\to Y;\,x\mapsto = f(g(x))\]
\ssc{Inverse function (反函數)}
For a bijective function $f\colon X\to Y$, the inverse of it, denoted as $f^{-1}$, is defined as:
\[f^{-1}\colon Y\to X;\,f(x)\mapsto x.\]
\ssc{Preimage (像原) or Inverse image}
Given an open subset $B$ of the codomain of a function $f$ the preimage or inverse image of $f$ on $B$, denoted as $f^{-1}(B)$ is defined as
\[f^{-1}(B)=\{x\in X|f(x)\in B\}.\]
\ssc{Translation invariance}
For a map $M$ defined for some functions, translation invariance is defined as: for all $a$ such that $M(f(x))$ and $f(x+a)$ are both defined,
\[M(f(x+a))=M(f(x)).\]
\ssc{Piecewise function (分段函數)}
A piecewise function is a function defined in the form:
\[f(x) =
\begin{cases}
f_1(x), & \quad x\in A_1, \\
f_2(x), & \quad x\in A_2, \\
\dots  \\
f_n(x), & \quad x \in A_n
\end{cases},\]
where
\[\bigcup_{i=1}^nA_i=D_f\land\forall i\neq j\land i,j\in\mathbb{N}\land i,j\leq n\colon A_i\cap A_j=\varnothing.\]
\ssc{Elementary functions}
A constants, a power function, a logarithmic function, an exponential function, a trigonometric function, an inverse trigonometric function, or any function that can be built up with finite defined operations of addition, subtraction, multiplication, division, and composition of functions on any finite set of elementary functions is called an elementary function.

An integral of which the integrand is an elementary function is called an elementary integral.
\ssc{Some Types of Real Functions}
\sssc{Power Function (冪函數)}
A power function if a function $f(x)=kx^a$ where $k\in\mathbb{C}$, called coefficient, and $a\in\mathbb{R}$, called exponent, are constants (some requires $k\in\mathbb{R}$ or $k=1$).
\sssc{Root Function (根式函數)}
A root function is a power function of which the exponent is $\frac{1}{n}$ where $n\in\mathbb{N}$.
\sssc{Reciprocal Function}
The reciprocal function is the function $f(x)=\frac{1}{x}$.
\sssc{Polynomial (多項式)}
A polynomial is a function $P(x)=\sum_{k=0}^na_kx^k$ where $a_k\in\mathbb{R}$ or $\mathbb{C}$ are constants called coefficients and $n$ is a nonnegative integer called the degree of $P$.
\sssc{Rational Function (有理函數)}
A function $f$ is called a rational function if there exists two polynomials $P(x)$ and $Q(x)$ such that $f(x)=\frac{P(x)}{Q(x)}$ for all $x$ in the domain.

If $\deg(P)\geq\deg(Q)$, $f(x)$ is improper; otherwise $f(x)$ is proper.
\sssc{Algebraic Function (代數函數)}
A function $f$ is called an algebraic function if there exists a polynomial $P(y,x)$ in two variables such that $P(f(x),x)=0$ for all $x$ in the domain.
\sssc{Transcendental Function (超越函數)}
A real function is called a transcendental function if it is not an algebraic function.
\ssc{Periodic function (週期函數)}
\sssc{Periodic function (週期函數)}
Let $f$ be a function with $D_f\subseteq I$. If there exists an interval $I$ and positive real number $T$ such that
\bit
\item Either $I$ is unbound or $|I|=nT$ for some $n\in\bbN$,
\item
\[t\in D_f\land t+T\in I\implies t+T\in D_f,\]
\[t\in D_f\land t-T\in I \implies t-T\in D_f,\]
and
\item For any $t\in D_f$ such that $t+T\in D_f$
\[f(t+T)=f(t),\]
\eit
then, we say that $f$ is periodic and that any such $T$ is a period (週期) of $f$, and in contexts where its period is referred to as a single value, it's the smallest positive $T$.
\sssc{Translation-invariant map property}
If $f$ is a periodic function with period $T$ and $M$ is a map defined for $f$ that satifies translation invariance, then for any interval $I\subseteq D_f$ that is either unbounded or of length $nT$ for some $n\in\bbN$ such that $M(f_I)$ is defined, $M(f)=M(f_I)$.
\sssc{Phase (相位) and phase angle (相位角)}
The phase $\Phi(t)$ and phase angle $\phi(t)$ of a periodic function $f(t)$ with period $T$ given a reference start of a period $t_0$ are defined as
\[\Phi(t)=\qty(\frac{t-t_0}{T}-\left\lfloor\frac{t-t_0}{T}\right\rfloor),\]
\[\phi(t)=2\pi\Phi(t),\]
which are both functions of period $T$ and are zero at $t_0+nT$ for all $n\in\bbN$.

Some sources define phase as $\phi$ and/or phase angle as $\Phi$, which is not used here.

For two periodic functions $f(t)$ and $g(t)$ such that $g(t)=Af(t+\tau)$ for some real number $A,\tau$, the phase shift (相位移) $\Delta\Phi$ and phase angle shift (相位角移) $\Delta\phi$ of $g(t)$ relative to $f(t)$ are defined as
\[\Delta\Phi=\frac{\tau}{T}-\left\lfloor\frac{\tau}{T}\right\rfloor\]
\[\Delta\phi=2\pi(\Delta\Phi).\]
The phase difference (相位差) and phase angle difference (相位角差) between them are defined as the magnitude of $\Delta\Phi$ and $\Delta\phi$ respectively.

If the phase difference of $f(t)$ and $g(t)$ is 0, they are called in phase (同相); otherwise they are called out of phase (異相). If the phase difference of $f(t)$ and $g(t)$ is 0.5 and $\alpha f(t)+g(t)=0$, they are called completely out of phase (反相).
\end{document}
