\documentclass[a4paper,12pt]{article}
\setcounter{secnumdepth}{5}
\setcounter{tocdepth}{3}
\input{/usr/share/LaTeX-ToolKit/template.tex}
\begin{document}
\title{Function}
\author{沈威宇}
\date{\temtoday}
\titletocdoc
\section{Function (函數)}
\ssc{Variable (變數) or indeterminate}
A variable or an indeterminate is a symbol, typically a letter, that refers to an unspecified mathematical object.
\ssc{Function, map, or mapping (映射)}
A function, a map, or a mapping is formed by three sets, the domain (定義域) $X$, the codomain (對應域) $Y$, and the graph $R$ that satisfy the three following conditions:
\[R\subseteq \{(x,y)\mid x\in X,y\in Y\},\]
\[\forall x\in X,\exists y\in Y,\left(x,y\right)\in R ,\]
\[ (x,y)\in R\land (x,z)\in R\implies y=z.\]

We say a function is from, over, or on its domain (or sometimes a superset of its domain) and into, to, or in its codomain. We also say a function is between its domain (or sometimes a superset of its domain) and its codomain.

A function $f$ that satisfies the above is denoted as:
\[f\colon X\to Y.\]
in which the domain $X$ is also denoted as $D_f$.

The range (值域) or image (像), denoted as $R_f$ or $f(X)$, is defined as:
\[\{ y \mid \exists x \in X, (x, y) \in R \}.\]

If $x\in X$ and $(x, y) \in R$, we write $y = f(x)$, in which $f(x)$ is called the image of $x$ under $f$, $x$ is called the independent variable (自變數 or 獨立變數) or input, and $y$ is called the dependent variable (應變數 or 依賴變數) or output; and the function $f$ is also denoted as:
\[f \colon X \to Y;\, x \mapsto y.\]

If $I$ is a subset of the domain of $f$, $f(I)$ is defined as
\[f(I)\coloneq\{y\mid\exists x\in I \tx{\ s.t.\ }f(x)=y\}.\]

A function whose domain is a subset of $\bbR$ is called a real-domain function. A function whose range is a subset of $\bbR$ is called a real-valued function. A function whose domain and range are both subsets of $\bbR$ is called a real function.
\ssc{Types of Functions}
Consider a function $f$:
\[f\colon X\to Y;\,x\mapsto y.\]
\sssc{Injection (單射) or Injective function or One-to-one (一對一) function}
\[\forall a,b\in X\text{\ s.t.\ } f(a)=f(b)\colon a=b\]
\sssc{Many-to-one (多對一) function}
\[\exists a\neq b\in X\colon f(a)=f(b)\]
\sssc{Surjection (滿射 or 蓋射) or Surjective function or Onto function}
\[f(X)=Y\]
\sssc{Bijection (對射) or Bijective function or One-to-one (一對一) function or One-to-one correspondence (一一對應)}
Injective and surjective function.
\ssc{Partial Function (偏函數)}
A partial function from $X$ to $Y$ is a function from a subset of $X$ to $Y$. The notation of a partial function is the same as a function except the fact that it is a partial function should be specified.
\ssc{Arity (元數), valency, or valence}
The number of arguments taken by a function. A function with arity of zero is called nullary, a function with arity of one is called unary, and a function with arity of two is called binary.
\ssc(Restriction and extension (延拓)}
We say that a function $f\colon X\to Y$ is an extension of another function $g\colon U\subseteq X\to Y$ and that $g$ is a restriction of $f$ iff $\forall u\in U\colon f(u)=g(u)$.

Let $f\colon X\to Y$ and $I\subeteq X$. Then $f\vert_I$ is defined as a function $f\vert_I\colon I\to Y$ such that for every $x\in I$, $f\vert_I(x)=f(x)$.
\ssc{Property in a variable}
Unless otherwise specified, given a property $P$ defined for a function $f(w_1,\dots w_n,x,y_1,\dots y_n)$ in which $w_1,\dots w_n,x,y_1,\dots y_n$ are independent variables, we say $f$ is with property $P$ in $x$ if the function $g(x)$ that is defined to map $x$ to $f(w_1,\dots w_n,x,y_1,\dots y_n)$ for any $x$ such that $w_1,\dots w_n,x,y_1,\dots y_n$ is in the domain of $f$ for any fixed $w_1,\dots w_n,y_1,\dots y_n$ is with property $P$.

Unless otherwise specified, given a property $P$ defined for a function $f(w_1,\dots w_n,x,y_1,\dots y_n)$ in which $w_1,\dots w_n,x,y_1,\dots y_n$ are independent variables, we say $f$ is with property $P$ in $x$ for a fixed $w_1,\dots w_n,y_1,\dots y_n$ if the function $g(x)$ that is defined to map $x$ to $f(w_1,\dots w_n,x,y_1,\dots y_n)$ for any $x$ such that $w_1,\dots w_n,x,y_1,\dots y_n$ is in the domain of $f$ is with property $P$.

Unless otherwise specified, given a property $P$ defined for a function $f(w_1,\dots w_n,x,y_1,\dots y_n)$ in which $w_1,\dots w_n,x,y_1,\dots y_n$ are independent variables, we say $f$ is with property $P$ in $x$ at a point $w_1,\dots w_n,x_0,y_1,\dots y_n$ in the domain of $f$ if there exists a neighbourhood $X$ of $x_0$ such that the function $g(x)$ that is defined to map $x$ to $f(w_1,\dots w_n,x,y_1,\dots y_n)$ for any $x\in X$ such that $w_1,\dots w_n,x,y_1,\dots y_n$ is in the domain of $f$ is with property $P$.
\ssc{Indicator function (指示函數或示性函數) or characteristic function (特徵函數)}
An indicator function or a characteristic function of a subset $A$ of a set $X$ is a function that maps elements of the subset to one, and all other elements to zero, often denoted as $1_A$.
\ssc{Binary operation}
A binary operation $*$ on a set $S$ is a function with domain $S^2$ and codomain $S$, and the output is denoted as $a*b$ for inputs $(a,b)\in S^2$.
\ssc{Sequences (數列)}
\sssc{Sequence}
A sequence in $X$ is a function of which the domain is a set $\{x\in\mathbb{Z}\mid l\leq x\leq m\}$ or $\{x\in\mathbb{Z}\mid l\leq x\}$ and the codomain is a topological space $X$, in which $l$ is an integer, usually $0$ or $1$, and $m$ is a integer, denoted as $\langle a_n\rangle$, $\{a_n\}$, $(a_n)$, $\langle a_n\rangle_{n=l}^m$, $\{a_n\}_{n=l}^m$, or $(a_n)_{n=l}^m$, with $m=\infty$ when its domain is $\{x\in\mathbb{Z}\mid l\leq x\}$, where the subscript $n$ refers to the $n$th element of the sequence, that is, the function value when the  independent variable is $n$.
\sssc{Finite Sequence (有限數列)}
A finite sequence is a sequence with finite terms, i.e. $\langle a_n\rangle_{n=l}^m,\quad m\in\mathbb{Z}$.
\sssc{Infinite sequence (無窮數列)}
An infinite sequence is a sequence with infinite terms, i.e. $\langle a_n\rangle_{n=l}^\infty$. Unless otherwise specified, sequences refer to infinite sequences.
\sssc{Sequence of partial sums}
The sequence of partial sums of a sequence $\langle a_n\rangle_{n=l}^{\infty}$ is the sequence $\left\langle s_n=\sum_{i=l-1}^na_i\right\rangle_{n=l-1}^{\infty}$ where $s_{l-1}=0$ or $\left\langle s_n=\sum_{i=l}^na_i\right\rangle_{n=l}^{\infty}$.
\ssc{Series (級數)}
\sssc{Series}
The sequence of partial sums of a sequence.
\sssc{Finite Series (有限級數)}
The sequence of partial sums of a finite sequence.
\sssc{Infinite Series (無窮級數)}
The sequence of partial sums of an infinite sequence.
\sssc{Cauchy product}
\[\qty(\sum_{i=0}^{\infty}a_i)\qty(\sum_{j=0}^{\infty}b_j)=\sum_{k=0}^{\infty}\sum_{l=0}^ka_lb_{k-l}.\]
\sssc{Power series}
A power series in $x^n$ is a series in the form
\[\sum_{n=0}^{\infty}a_nx^n.\]
\end{document}