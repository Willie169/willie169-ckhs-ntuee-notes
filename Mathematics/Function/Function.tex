\documentclass[a4paper,12pt]{article}
\setcounter{secnumdepth}{5}
\setcounter{tocdepth}{3}
\input{/usr/share/LaTeX-ToolKit/template.tex}
\begin{document}
\title{Function}
\author{沈威宇}
\date{\temtoday}
\titletocdoc
\section{Function (函數)}
\subsection{Function, map, or mapping (映射)}
A function, a map, or a mapping is formed by three sets, the domain (定義域) $X$, the codomain (對應域) $Y$, and the graph $R$ that satisfy the three following conditions:
\[R\subseteq \{(x,y)\mid x\in X,y\in Y\},\]
\[\forall x\in X,\exists y\in Y,\left(x,y\right)\in R ,\]
\[ (x,y)\in R\land (x,z)\in R\implies y=z.\]

We say a function is from, over, or on its domain (or sometimes a superset of its domain) and into, to, or in its codomain. We also say a function is between its domain (or sometimes a superset of its domain) and its codomain.

A function $f$ that satisfies the above is denoted as:
\[f\colon X\to Y.\]
in which the domain $X$ is also denoted as $D_f$.

The range (值域) or image (像), denoted as $R_f$ or $f(X)$, is defined as:
\[\{ y \mid \exists x \in X, (x, y) \in R \}.\]

If $x\in X$ and $(x, y) \in R$, we write $y = f(x)$, in which $f(x)$ is called the image of $x$ under $f$, $x$ is called the independent variable (自變數 or 獨立變數) or input, and $y$ is called the dependent variable (應變數 or 依賴變數) or output; and the function $f$ is also denoted as:
\[f \colon X \to Y;\, x \mapsto y.\]

If $I$ is a subset of the domain of $f$, $f(I)$ is defined as
\[f(I)\coloneq\{y\mid\exists x\in I \tx{\ s.t.\ }f(x)=y\}.\]

Let $f\colon X\to Y$ and $I\subeteq X$. Then $f\vert_I$ is defined as a function $f\vert_I\colon I\to Y$ such that for every $x\in I$, $f\vert_I(x)=f(x)$.

Let $f\colon X\subseteq\mathbb{R}\to Y$ and $(a,b)\subeteq X$. Then $f\vert_a^b$ is defined as a function $f\vert_a^b\colon (a,b)\to Y$ such that for every $x\in (a,b)$, $f\vert_a^b(x)=f(x)$.
\ssc{Type of Functions}
Consider a function $f$:
\[f\colon X\to Y;\,x\mapsto y.\]
\sssc{Injection (單射) or Injective function or One-to-one (一對一) function}
\[\forall a,b\in X\text{\ s.t.\ } f(a)=f(b)\colon a=b\]
\sssc{Many-to-one (多對一) function}
\[\exists a\neq b\in X\colon f(a)=f(b)\]
\sssc{Surjection (滿射 or 蓋射) or Surjective function or Onto function}
\[f(X)=Y\]
\sssc{Bijection (對射) or Bijective function or One-to-one (一對一) function or One-to-one correspondence (一一對應)}
Injective and surjective function.
\ssc{Tests of real functions}
\sssc{Vertical line test (垂線測試)}
For a graph on $xy$ plane, if any vertical line intersects it, $y$ is not a function of $x$.
\sssc{Horizontal line test (水平線測試)}
Given a graph of a real function $y=f(x)$ on $xy$ plane, if any horizontal line intersects it more than once, $f(x)$ is not bijective.
\ssc{Fixed point, fixpoint, or invariant point (不動點 or 定點)}
A fixed point, fixpoint, or invariant point of a function $f\colon X\subseteq Y\to Y$, is a point $x\in X$ such that $f(x)=x$.
\ssc{Property in a variable}
Unless otherwise specified, given a property $P$ defined for a function $f(w_1,\dots w_n,x,y_1,\dots y_n)$ in which $w_1,\dots w_n,x,y_1,\dots y_n$ are independent variables, we say $f$ is with property $P$ in $x$ if the function $g(x)$ that is defined to map $x$ to $f(w_1,\dots w_n,x,y_1,\dots y_n)$ for any $x$ such that $w_1,\dots w_n,x,y_1,\dots y_n$ is in the domain of $f$ for any fixed $w_1,\dots w_n,y_1,\dots y_n$ is with property $P$.

Unless otherwise specified, given a property $P$ defined for a function $f(w_1,\dots w_n,x,y_1,\dots y_n)$ in which $w_1,\dots w_n,x,y_1,\dots y_n$ are independent variables, we say $f$ is with property $P$ in $x$ for a fixed $w_1,\dots w_n,y_1,\dots y_n$ if the function $g(x)$ that is defined to map $x$ to $f(w_1,\dots w_n,x,y_1,\dots y_n)$ for any $x$ such that $w_1,\dots w_n,x,y_1,\dots y_n$ is in the domain of $f$ is with property $P$.

Unless otherwise specified, given a property $P$ defined for a function $f(w_1,\dots w_n,x,y_1,\dots y_n)$ in which $w_1,\dots w_n,x,y_1,\dots y_n$ are independent variables, we say $f$ is with property $P$ in $x$ at a point $w_1,\dots w_n,x_0,y_1,\dots y_n$ in the domain of $f$ if there exists a neighbourhood $X$ of $x_0$ such that the function $g(x)$ that is defined to map $x$ to $f(w_1,\dots w_n,x,y_1,\dots y_n)$ for any $x\in X$ such that $w_1,\dots w_n,x,y_1,\dots y_n$ is in the domain of $f$ is with property $P$.
\ssc{Homogeneous}
A function $F$ with domain $\Omega\subseteq\bbR^n$ is called a homogeneous function of degree $\alpha\in\bbR$ if and only if for any nonzero real number $t$ and $\mathbf{x},t\mathbf{x}\in\Omega$, $F(t\mathbf{x})=t^{\alpha}F(\mb{x})$.

For any homogeneous function $F(x_1,\mb{x})$ of $n$ variables where $\mb{x}$ represents the second to the $n$th variable, there exists a function $f$ of $n-1$ variables such that $F(x_1,\mb{x})=f\qty(\frac{\mb{x}}{x_1})$ for all $x_1\neq 0$ such that $(x_1,\mb{x})$ is in the domain of $F$.
\ssc{Periodic function (週期函數)}
Let $f$ be a function with domain $\bbR$ or $\bbR_{\geq 0}$. If there exists $T>0$ such that
\[f(t+T)=f(t)\]
for any $t$ in the domain of $f$, we say that $f$ is periodic and that $T$ is the period (週期) of $f$.
\ssc{Of Exponential Order}
A real or complex-valued net $f$ on a subset of $\mathbb{R}$ is considered of exponential order $\alpha\in\mathbb{R}_{>0}$ if there exists $M>0$ and $T>0$ such that:
\[|f(t)|\leq Me^{\alpha t}\quad \forall t>T.\]
\ssc{Symmetry}
Given a function $f\colon X\to Y$ such that $\forall x\in X\colon -x\in X$.
\sssc{Even Function (偶函數)}
$f$ is an even function if for all $x\in X$, $f(x)=f(-x)$.
\sssc{Odd Function (奇函數)}
$f$ is an odd function if for all $x\in X$, $-f(x)=f(-x)$.
\ssc{Transformation (變換)}
\sssc{Translation (平移)}
For any function $f\colon\mathbb{R}\to\mathbb{R}$, shifting $y=f(x)$ right by $h$ units and up by $k$ units on the $xy$ coordinate plane yields $y=f(x-h)+k$.
\sssc{Scaling (伸縮 or 縮放 or 拉伸)}
For any function $f\colon\mathbb{R}\to\mathbb{R}$, on the $xy$ coordinate plane, expand $y=f(x)$ vertically by $a$ times the original value with the $x$ axis as the reference line, and expand $y=af\qty(\frac{x}{b})$ horizontally by $b$ times the original value with the $y$ axis as the reference line, to obtain $y=af\qty(\frac{x}{b})$.
\ssc{Function composition (函數合成)}
For two functions $f\colon X\to Y$ and $g\colon V\to W$ such that $g(V)\subseteq X$, the composition of them, denoted as $(f\circ g)$, is defined as:
\[(f \circ g)\colon V\to Y;\,x\mapsto = f(g(x))\]
\ssc{Inverse function (反函數)}
For a bijective function $f\colon X\to Y$, the inverse of it, denoted as $f^{-1}$, is defined as:
\[f^{-1}\colon Y\to X;\,f(x)\mapsto x\]
\ssc{Preimage (像原) or Inverse image}
Given an open subset $B$ of the codomain of a function $f$ the preimage or inverse image of $f$ on $B$, denoted as $f^{-1}(B)$ is defined as
\[f^{-1}(B)=\{x\in X|f(x)\in B\}.\]
\ssc{Extension (延拓) of a function}
We say a function $f\colon X\to Y$ is an extension of another function $g\colon U\subseteq X\to Y$ if $\forall u\in U\colon f(u)=g(u)$.
\ssc{Piecewise function (分段函數)}
A piecewise function is a function defined in the form:
\[f(x) =
\begin{cases}
f_1(x), & \quad x\in A_1, \\
f_2(x), & \quad x\in A_2, \\
\dots  \\
f_n(x), & \quad x \in A_n
\end{cases},\]
where
\[\bigcup_{i=1}^nA_i=D_f\land\forall i\neq j\land i,j\in\mathbb{N}\land i,j\leq n\colon A_i\cap A_j=\varnothing.\]
\subsection{Indicator function (指示函數或示性函數) or characteristic function (特徵函數)}
An indicator function or a characteristic function of a subset $A$ of a set $X$ is a function that maps elements of the subset to one, and all other elements to zero, often denoted as $1_A$.
\ssc{Elementary functions}
A constants, a power function, a logarithmic function, an exponential function, a trigonometric function, an inverse trigonometric function, or any function that can be built up with finite defined operations of addition, subtraction, multiplication, division, and composition of functions on any finite set of elementary functions is called an elementary function.

An integral of which the integrand is an elementary function is called an elementary integral.
\ssc{Real Algebraic Function (實代數函數)}
\sssc{Power Function (冪函數)}
A power function if a function $f(x)=kx^a$ where $k\in\mathbb{C}$, called coefficient, and $a\in\mathbb{R}$, called exponent, are constants (some requires $k\in\mathbb{R}$ or $k=1$).
\sssc{Root Function (根式函數)}
A root function is a power function of which the exponent is $\frac{1}{n}$ where $n\in\mathbb{N}$.
\sssc{Reciprocal Function}
The reciprocal function is the function $f(x)=\frac{1}{x}$.
\sssc{Polynomial (多項式)}
A polynomial is a function $P(x)=\sum_{k=0}^na_kx^k$ where $a_k\in\mathbb{R}$ or $\mathbb{C}$ are constants called coefficients and $n$ is a nonnegative integer called the degree of $P$.
\sssc{Rational Function (有理函數)}
A function $f$ is called a rational function if there exists two polynomials $P(x)$ and $Q(x)$ such that $f(x)=\frac{P(x)}{Q(x)}$ for all $x$ in the domain.
\sssc{Algebraic Function (代數函數)}
A function $f$ is called an algebraic function if there exists a polynomial $P(y,x)$ in two variables such that $P(f(x),x)=0$ for all $x$ in the domain.
\ssc{Piecewise functions}
\sssc{Sign function or signum function (符號函數)}
The sign function $\sgn\colon\bbR\to\bbR$ is defined as:
\[\sgn(x)=\frac{x}{|x|}.\]
\sssc{Floor function (下取整函數), integral part, integer part (整數部份), greatest integer, entier, rounding down (無條件捨去), or rounding toward negative infinity}
The floor function $\lfloor\cdot\rfloor\colon\bbR\to\bbR$ or $[\cdot]\colon\bbR\to\bbR$ is defined as:
\[\lfloor x\rfloor=[x]=\max\{m\in\bbZ\mid m\leq x\},\]
in which $[\cdot]$ is called the Gauss sign (高斯符號).
\sssc{Ceiling function (上取整函數), rounding up (無條件進位), or rounding toward positive infinity}
The ceiling function $\lceil\cdot\rceil\colon\bbR\to\bbR$ is defined as:
\[\lceil x\rceil=\min\{m\in\bbZ\mid m\geq x\}.\]
\sssc{Truncation, rounding toward zero, or rounding away from infinity}
The truncation $\operatorname{truncate}\colon\bb{R}\to\bb{R}$ is defined as:
\[\operatorname{truncate}(x)=
\bcs
\lfloor x\rfloor,\quad & x\geq 0\\
\lceil x\rceil,\quad & x<0
\ecs\]
\sssc{Rounding half up (四捨五入) or rounding half toward positive infinity}
The rounding half up $\operatorname{roundhalfup}\colon\bbR\to\bbR$ is defined as:
\[\operatorname{roundhalfup}(x)=\left\lfloor x+\frac{1}{2}\right\rfloor.\]
\sssc{Rounding half down (五捨六入) or rounding half toward negative infinity}
The rounding half down $\operatorname{roundhalfdown}\colon\bbR\to\bbR$ is defined as:
\[\operatorname{roundhalfdown}(x)=\left\lceil x-\frac{1}{2}\right\rceil.\]
\sssc{Rounding half toward zero or rounding half away from infinity}
The rounding half toward zero $\operatorname{roundhalftowardzero}\colon\bbR\to\bbR$ is defined as:
\[\operatorname{roundhalftowardzero}(x)=\sgn(x)\left\lceil |x|-\frac{1}{2}\right\rceil.\]
\sssc{Rounding half away from zero or rounding half toward infinity}
The rounding half away from zero $\operatorname{roundhalfawayfromzero}\colon\bbR\to\bbR$ is defined as:
\[\operatorname{roundhalfawayfromzero}(x)=\sgn(x)\left\lfloor |x|+\frac{1}{2}\right\rfloor.\]
\sssc{Rounding half to even (四捨六入五成雙)}
The rounding half to even $\operatorname{roundhalftoeven}\colon\bbR\to\bbR$ is defined as:
\[\operatorname{roundhalftoeven}(x)=\bcs
\left\lfloor x+\frac{1}{2}\right\rfloor,\quad & \frac{1}{2}\left\lfloor x+\frac{1}{2}\right\rfloor\in\mathbb{Z}\\
\left\lceil x-\frac{1}{2}\right\rceil,\quad & \frac{1}{2}\left\lfloor x+\frac{1}{2}\right\rfloor\notin\mathbb{Z}
\ecs.\]
\sssc{Rounding half to odd (四捨六入五成單)}
The rounding half to odd $\operatorname{roundhalftoodd}\colon\bbR\to\bbR$ is defined as:
\[\operatorname{roundhalftoodd}(x)=\bcs
\left\lfloor x+\frac{1}{2}\right\rfloor,\quad & \frac{1}{2}\left\lfloor x+\frac{1}{2}\right\rfloor\notin\mathbb{Z}\\
\left\lceil x-\frac{1}{2}\right\rceil,\quad & \frac{1}{2}\left\lfloor x+\frac{1}{2}\right\rfloor\in\mathbb{Z}
\ecs.\]
\sssc{Heaviside step function (黑維塞階躍函數) or Unit step function (單位階躍函數)}
\[H(x)=u(x)=
\begin{cases}0,\quad &x<0\\
1,\quad &x\geq 0
\end{cases},\]
where $H(0)=u(0)$ is sometimes defined to be $\frac{1}{2}$ or other values instead.
\[H'(x)=u'(x)=\delta(x).\]
\sssc{Dirac delta function (狄拉克 δ 函數)}
\[
\delta(x) =
\begin{cases}
\infty, & x = 0, \\
0, & x \neq 0.
\end{cases}
\]
\[
\forall \epsilon > 0:\,\int_{a-\epsilon}^{a+\epsilon} f(x) \delta(x-a) \, \mathrm{d}x = f(a).
\]
\[
\int _{-\infty }^{\infty }\delta(x) \, \mathrm{d}x = 1.
\]
\[\int_0^x\delta(t)\,\mathrm{d}t=H(x).\]
\ssc{Error function}
\sssc{Error funcction (誤差函數) or Gauss error function (高斯誤差函數)}
The error function or Gauss error function $\erf\colon\bbC\to\bbC$ or $\erf\colon\bbR\to\bbR$ is defined as:
\[\erf(z)=\frac{2}{\sqrt{\pi}}\int_0^ze^{-t^2}\dd{t}.\]
\sssc{Complementary error function (互補誤差函數)}
The complementary error function $\erfc\colon\bbC\to\bbC$ or $\erfc\colon\bbR\to\bbR$ is defined as:
\[\erfc(z)=1-\erf(z)=\int_z^{\infty}e^{-t^2}\dd{t}.\]
\sssc{Imaginary error function (虛誤差函數)}
The imaginary error function $\erfi\colon\bbC\to\bbC$ or $\erfi\colon\bbR\to\bbR$ is defined as:
\[\erfi(z)=-i\erf(iz).\]
\end{document}
