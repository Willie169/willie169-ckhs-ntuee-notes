\documentclass[a4paper,12pt]{report}
\setcounter{secnumdepth}{5}
\setcounter{tocdepth}{3}
\input{/usr/share/latex-toolkit/template.tex}
\begin{document}
\title{Distribution Theory}
\author{沈威宇}
\date{\temtoday}
\titletocdoc
\section{Distribution Theory}
\ssc{Support of functional}
For a function $f\colon U\subseteq\bbR^n\to\bbR$ (or $\bbC$), the support of it, denoted as $\supp(f)$, is defined as
\[\supp(f)=\closure(\{x\in U\mid f(x)\neq 0\}).\]
If $\supp(f)$ is a compact set, $f$ is called of compact support. If $\supp(f)$ is a compact subset of a set $V\subseteq U$, $f$ is called of compact support on $V$. $C^k_c(U)$ for $k\in\bbN\cup\{\infty\}$ denotes the vector space of $C^k(U)$-functions with compact support. For any compact subset $K$ of $U$, $C^k(K;U)$ denotes the vector space of $C^k(U)$-functions with compact support on $K$.
\ssc{Test function}
A test function on $U$ is a function in $C^{\infty}_c(U)$. The space of test functions on $U$ is $C^{\infty}_c(U)$, also denoted as $\mathcal{D}(U)$.
\ssc{Set of compact subsets as directed set}
The set of compact subsets of $U$ form a directed set $\bbK$ with order by inclusion.
\ssc{Topology of $C^k(K;U)$ and $C^k(U)$}
For any subset $U$ of $\bbR^n$, for any $k\in\bbN\cup\{\infty\}$, for any $f\in C^k(U)$, for any integer $i$ such that $0\leq i\leq k$, for any nonempty compact subset $K$ of $U$, with the set $\mathcal{P}$ of all multi-indices $p$ such that $|p|\leq i$, define,
\[q_{i,K}(f)\coloneq\sup_{p\in\mathcal{P}}\qty(\sup_{x\in K}\abs{D^pf(x)}),\]
and define $q_{i,\varnothing}(f)=0$.

Define the topology of $C^k(K;U)$ as the topology generated by the family of seminorms $q_{i,K}$ for any integer $i$ such that $0\leq i\leq k$, thus $C^k(K;U)$ is a locally convex topological vector space.

Define the topology of $C^k(U)$ as the topology generated by the family of seminorms $q_{i,K}$ for any compact subset $K$ of $U$, for any integer $i$ such that $0\leq i\leq k$, thus $C^k(U)$ is a locally convex topological vector space.
\ssc{Canonical LF topology}
$C^k_c(U)$ is equipped with the canonical LF topology defined as the finest topology such that for any compact subset $K$ of $U$, the inclusion map from $C^k(K;U)$ to $C^k_c(U)$ is continuous.
\ssc{Distribution or Schwartz distribution}
A distribution or Schwartz distribution on $U$ is a continuous linear functional on $C^{\infty}_c(U)$. The space of distributions on $U$ is the strong dual space of $C^{\infty}_c(U)$, denoted as $\mathcal{D}'(U)$ and equipped with the strong dual topology.
\ssc{Localization of distributions}
\ssc{Extension and restriction}
Let $V\subseteq U\subseteq\bbR^n$. For every function $f\in\mathcal{D}(V)$, the trivial extension of $f$ to $U$ is defined as a function $E_{VU}(f)\in\mathcal{D}(U)$ such that for all $v\in V$,
\[E_{VU}(f)(v)=f(v),\]
and for all $u\in U\setminus V$,
\[E_{VU}(f)(u)=0.\]
This defines $E_{VU}\colon\mathcal{D}(V)\to\mathcal{D}(U)$ called trivial extension operator to $U$ in $V$, which is a continuous injective linear map.

The transpose of $E_{VU}$ is $\rho_{VU}\colon\mathcal{D}'(U)\to\mathcal{D}'(V)$ called restriction operator to $V$ in $U$, and for all distribution $T\in\mathcal{D}'(U)$, $\rho_{VU}(T)$ is called restriction of $T$ to $V$. A distribution in the image of $\rho_{VU}$ is called extendable to $U$. A distribution in the image of $\rho_{V\bbR^n}$ is called extendable.
\ssc{Extension theorem}
The image of $E_{VU}$ is a vector space. For all $V\subset U$, $E_{VU}$ is not a topological embedding and that the subspace topology of the image of $E_{VU}$ is strictly finer than the topology on the image of $E_{VU}$ that makes $E_{VU}$ a homeomorphism.
\ssc{Vanish}
$T\in\mathcal{D}'(U)$ is said to vanish in a subset $V$ of $U$ if for all $f\in\mathcal{D}(U)$, $\langle T,f\rangle=0$.
\ssc{Gluing theorem}
Let $(U_i)_{i\in I}$ be an indexed family of open subsets of $\bbR^n$. For each $i\in I$, let $T_i\in\mathcal{D}'(U_i)$ and suppose that for all $i,j\in I$, the restriction of $T_i$ to $U_i\cap U_j$ is equal to the restriction of $T_j$ to $U_i\cap U_j$. Then there exists unique $T\in\mathcal{D}'\qty(\bigcup_{i\in I}U_i)$ such that for all $i\in I$, the restriction of $T$ to $U_i$ is equal to $T_i$.
\ssc{Support of distribution}
For any $T\in\mathcal{D}'(U)$, the support of $T$ is defined as
\[\supp(T)=U\setminus\bigcup\{V\subseteq U\mid\rho_{VU}(T)=0\}.\]
If $\supp(T)$ is a compact set, $T$ is called of compact support. If $\supp(T)$ is a compact subset of a set $V\subseteq U$, $T$ is called of compact support on $V$.
\ssc{Distribution induced by measure}
For a possibly signed Radon measure $\mu$, the distribution $T_{\mu}\in\mathcal{D}'(U)$ induced by $\mu$ is defined by, for all $\varphi\in\mathcal{D}(U)$,
\[\langle T_{\mu},\varphi\rangle=\int_U\varphi\dd{\mu}.\]
Distribution induced by a measure is often denoted with the same symbol of the measure.
\ssc{Derivative of distribution}
For any open subset $U\subseteq\bbR^n$, for any $\varphi\in\mathcal{D}'(U)$, for any $f\in\mathcal{D}(U)$, for any multi-index $\alpha=(\alpha_1,\alpha_2,\ldots,\alpha_n)$, the distributional derivative $D^{\alpha}T$ is defined by,
\[\langle D^{\alpha}T,\varphi\rangle=(-1)^{\abs{\alpha}}\langle T,D^{\alpha}\varphi\rangle.\]
\ssc{Convolution of test function and distribution}
For any open subset $U\subseteq\bbR^n$, for any function $f,g\in\mathcal{D}(U)$, for any $T\in\mathcal{D}'(U)$,
\[\langle f*T,g\rangle\coloneq\langle T,\tilde{f}*g\rangle.\]
The convolution with $T$ defines a linear map $C_T\colon\mathcal{D}(U)\to\mathcal{D}'(U);f\mapsto T*f\coloneq f*T$.
\ssc{Convolution of distributions}
For any open subset $U\subseteq\bbR^n$, for any distribution $S,T\in\mathcal{D}'(U)$, for any $f\in\mathcal{D}(U)$, with $g(x)\coloneq\tau_{-x}f$,
\[\langle S*T,f\rangle\coloneq\langle S,\langle T,g\rangle\rangle.\]
