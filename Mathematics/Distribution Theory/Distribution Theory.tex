\documentclass[a4paper,12pt]{report}
\setcounter{secnumdepth}{5}
\setcounter{tocdepth}{3}
\input{/usr/share/latex-toolkit/template.tex}
\begin{document}
\title{Distribution Theory}
\author{沈威宇}
\date{\temtoday}
\titletocdoc
\section{Distribution Theory}
\ssc{Test function and distribution}
\sssc{Support of functional}
For a function $f\colon U\subseteq\bbR^n\to\bbR$ (or $\bbC$), the support of it, denoted as $\supp(f)$, is defined as
\[\supp(f)=\closure(\{x\in U\mid f(x)\neq 0\}).\]
If $\supp(f)$ is a compact set, $f$ is called with compact support. If $\supp(f)$ is a compact subset of a set $V\subseteq U$, $f$ is called with compact support on $V$. $C^k_c(U)$ for $k\in\bbN\cup\{\infty\}$ denotes the vector space of $C^k(U)$-functions with compact support. For any compact subset $K$ of $U$, $C^k(K;U)$ denotes the vector space of $C^k(U)$-functions with compact support on $K$.
\sssc{Test function}
A test function on $U$ is a function in $C^{\infty}_c(U)$. The space of test functions on $U$ is $C^{\infty}_c(U)$, also denoted as $\mathcal{D}(U)$.
\sssc{Set of compact subsets as directed set}
The set of compact subsets of $U$ form a directed set $\bbK$ with order by inclusion.
\sssc{Topology of $C^k(K;U)$ and $C^k(U)$}
For any subset $U$ of $\bbR^n$, for any $k\in\bbN\cup\{\infty\}$, for any $f\in C^k(U)$, for any integer $i$ such that $0\leq i\leq k$, for any nonempty compact subset $K$ of $U$, with the set $\mathcal{P}$ of all multi-indices $p$ such that $|p|\leq i$, define,
\[q_{i,K}(f)\coloneq\sup_{p\in\mathcal{P}}\qty(\sup_{x\in K}\abs{D^pf(x)}),\]
and define $q_{i,\varnothing}(f)=0$.

Define the topology of $C^k(K;U)$ as the topology generated by the family of seminorms $q_{i,K}$ for any integer $i$ such that $0\leq i\leq k$, thus $C^k(K;U)$ is a locally convex topological vector space.

Define the topology of $C^k(U)$ as the topology generated by the family of seminorms $q_{i,K}$ for any compact subset $K$ of $U$, for any integer $i$ such that $0\leq i\leq k$, thus $C^k(U)$ is a locally convex topological vector space.
\sssc{Canonical LF topology}
$C^k_c(U)$ is equipped with the canonical LF topology defined as the finest topology such that for any compact subset $K$ of $U$, the inclusion map from $C^k(K;U)$ to $C^k_c(U)$ is continuous.
\sssc{Distribution or Schwartz distribution}
A distribution or Schwartz distribution on $U$ is a continuous linear functional on $C^{\infty}_c(U)$. The space of distributions on $U$ is the strong dual space of $C^{\infty}_c(U)$, denoted as $\mathcal{D}'(U)$ and equipped with the strong dual topology.
\sssc{Support of distribution}
For any $T\in\mathcal{D}'(U)$, the support of $T$ is defined as
\[\supp(T)=U\setminus\bigcup\{V\subseteq U\mid\rho_{VU}(T)=0\}.\]
If $\supp(T)$ is a compact set, $T$ is called with compact support. If $\supp(T)$ is a compact subset of a set $V\subseteq U$, $T$ is called with compact support on $V$.

If $T\in\mathcal{D}'(U)$ is with compact support, for any $f\in C^{\infty}(U)$, $\langle T,f\rangle$ is defined as
\[\langle T,f\rangle\coloneq\lange T,f\chi\rangle,\]
where $\chi\in\mathcal{D}(U)$ is an arbitrary function such that there exists a neighborhood $N$ of $\supp(T)$ such that $\chi(N\cap U)=\{1\}$ and $f\chi\in\mathcal{D}(U)$. It's obvious that $\lange T,f\chi\rangle$ doesn't depend on the choice of $\chi$.
\ssc{Induced distribution}
\sssc{Distribution induced by measure}
For a possibly signed Radon measure $\mu$, the distribution $T_{\mu}\in\mathcal{D}'(U)$ induced by $\mu$ is defined by, for all $\varphi\in\mathcal{D}(U)$,
\[\langle T_{\mu},\varphi\rangle=\int_U\varphi\dd{\mu}.\]
\sssc{Distribution induced by $L^p_{\tx{loc}}$ function}
For a function $f\in L^p_{\tx{loc}}(U)$, the distribution $T_f\in\mathcal{D}'(U)$ induced by $f$ is defined by, for all $\varphi\in\mathcal{D}(U)$,
\[\langle T_f,\varphi\rangle=\int_Uf(x)\varphi(x)\dd{x}.\]
\ssc{Extension and restriction}
\sssc{Extension and restriction}
Let $V\subseteq U\subseteq\bbR^n$. For every function $f\in\mathcal{D}(V)$, the trivial extension of $f$ to $U$ is defined as a function $E_{VU}(f)\in\mathcal{D}(U)$ such that for all $v\in V$,
\[E_{VU}(f)(v)=f(v),\]
and for all $u\in U\setminus V$,
\[E_{VU}(f)(u)=0.\]
This defines $E_{VU}\colon\mathcal{D}(V)\to\mathcal{D}(U)$ called trivial extension operator to $U$ in $V$, which is a continuous injective linear map.

The transpose of $E_{VU}$ is $\rho_{VU}\colon\mathcal{D}'(U)\to\mathcal{D}'(V)$ called restriction operator to $V$ in $U$, and for all distribution $T\in\mathcal{D}'(U)$, $\rho_{VU}(T)$ is called restriction of $T$ to $V$. A distribution in the image of $\rho_{VU}$ is called extendable to $U$. A distribution in the image of $\rho_{V\bbR^n}$ is called extendable.
\sssc{Extension theorem}
The image of $E_{VU}$ is a vector space. However, for all $V\subset U$, $E_{VU}$ is not a topological embedding and that the subspace topology of the image of $E_{VU}$ is strictly finer than the topology on the image of $E_{VU}$ that makes $E_{VU}$ a homeomorphism.
\sssc{Vanish}
$T\in\mathcal{D}'(U)$ is said to vanish in a subset $V$ of $U$ if for all $f\in\mathcal{D}(U)$, $\langle T,f\rangle=0$.
\sssc{Gluing theorem}
Let $(U_i)_{i\in I}$ be an indexed family of open subsets of $\bbR^n$. For each $i\in I$, let $T_i\in\mathcal{D}'(U_i)$ and suppose that for all $i,j\in I$, the restriction of $T_i$ to $U_i\cap U_j$ is equal to the restriction of $T_j$ to $U_i\cap U_j$. Then there exists unique $T\in\mathcal{D}'\qty(\bigcup_{i\in I}U_i)$ such that for all $i\in I$, the restriction of $T$ to $U_i$ is equal to $T_i$.
\ssc{Operation}
\sssc{Multiplication by smooth function}
For any open subset $U\subseteq\bbR^n$, for any $T\in\mathcal{D}'(U)$, for any $f\in C^{\infty}(U)$, the distribution $fT$ is defined by, for any $g\in\mathcal{D}(U)$,
\[\langle fT,g\rangle=\langle T,fg\rangle.\]
\sssc{Scaling operator}
For any open subset $U\subseteq\bbR^n$, for any $T\in\mathcal{D}'(U)$, $T$ scaled by $a\in\bbR$, $S_aT$ or $T(a\cdot)$, is defined by, for any $f\in\mathcal{D}(U)$,
\[\langle S_aT,f\rangle=\frac{1}{|a|^n}\langle T,f\qty(\frac{\cdot}{a})\rangle.\]
\sssc{Translation operator}
For any open subset $U\subseteq\bbR^n$, for any $T\in\mathcal{D}'(U)$, $T$ translated by $a\in\bbR^n$, $\tau_aT$ or $T(\cdot-a)$, as a member of $\mathcal{D}'(\{u+a\mid u\in U\})$, is defined by, for any $f\in\mathcal{D}(\{u+a\mid u\in U\})$,
\[\langle\tau_aT,f\rangle=\langle T,\tau_{-a}f\rangle.\]
\sssc{Derivative of distribution}
For any open subset $U\subseteq\bbR^n$, for any $T\in\mathcal{D}'(U)$, for any $\varphi\in\mathcal{D}(U)$, for any multi-index $\alpha=(\alpha_1,\alpha_2,\ldots,\alpha_n)$, the distributional derivative $D^{\alpha}T$ of $T$ is defined by,
\[\langle D^{\alpha}T,\varphi\rangle=(-1)^{\abs{\alpha}}\langle T,D^{\alpha}\varphi\rangle.\]
\sssc{Convolution of functions}
For any open subset $U\subseteq\bbR^n$, for any functionals $f$ defined on $U$ and $g$ defined on $\{u-v\mid u,v\in U\}$, the convolution of $f$ and $g$, which is a functional defined on $U$, denoted as $f*g$, is defined as
\[f*g(x)=\int_Uf(y)g(x-y)\dd{y}.\]
Commutativity:
\[f*g=g*f.\]
\bpr
PLACEHOLDER
\epr
\sssc{Convolution of test function and distribution}
For any function $f\in C^{\infty}(\bbR^n)$, for any $T\in\mathcal{D}'(\bbR^n)$, with the constraint that either $f$ or $T$ is with compact support, the convolution of $f$ and $T$, which is an element of $C^{\infty}(\bbR^n)$, denoted as $f*T$, is defined as
\[f*T\coloneq x\mapsto\langle T,f(x-\cdot)\rangle.\]
\sssc{Convolution of distributions}
For any open subset $U\subseteq\bbR^n$, for any distributions $S,T\in\mathcal{D}'(\bbR^n)$ with the constraint that at least one of them is with compact support, for any $f\in\mathcal{D}(\bbR^n)$, the convolution of $S$ and $T$, which is an element of $\mathcal{D}'(U)$, denoted as $S*T$, is defined as
\[\langle S*T,f\rangle\coloneq\langle S,x\mapsto\langle T,\tau_{-x}f\rangle\rangle.\]
Commutativity:
\[S*T=T*S.\]
\bpr
PLACEHOLDER
\epr
\ssc{Constant one distribution}
For all $\varphi\in\mathcal{D}(U)$,
\[\langle 1,\varphi\rangle=\int_U\varphi(x)\dd{x}.\]
\ssc{Dirac delta function or unit impulse}
\sssc{Fake function}
The Dirac delta function $\delta(x)$ is often denoted as
\[\delta(x)
\begin{cases}
\infty, & x = 0, \\
0, & x \neq 0.
\end{cases}\]
With the derivative of unit step function $H'(x)$ defined as $\delta(x)$.
\sssc{Stieltjes integral}
For any function $f$ defined on $U\subseteq\bbR$, for any indexed family with $(A_i)_{i\in I}\subseteq U$,
\[\int_Uf(x)\sum_{i\in I}\delta(x-A_i)\dd{x}=\int_Uf(x)\dd{\sum_{i\in I}H(x-A_i)}=\sum_{i\in I}f(A_i).\]
\sssc{Dirac measure}
Dirac (delta) measure at $a$ is a Radon measure defined, for any set $A$, as
\[\delta_a(A)\coloneq\bcs 1,\quad&a\in A\\0,\quad a\notin A\ecs.\]
For any function $f$ defined on $U\subseteq\bbR^n$, for any indexed family with $(A_i)_{i\in I}\subseteq U$,
\[\int_Uf(x)\sum_{i\in I}\delta_{A_i}(x)\dd{x}=\int_Uf(x)\dd{\sum_{i\in I}\delta_{A_i}(x)}=\sum_{i\in I}f(A_i).\]
\sssc{Dirac distribution}
Dirac (delta) distribution is the distribution induced by Dirac measure, that is, for any set $U\subseteq\bbR^n$, for any $a\in U$, for any $\varphi\in\mathcal{D}(U)$,
\[\langle\delta_a,\varphi\rangle=\varphi(a).\]
\sssc{Convolution}
For all $a\in U\subseteq\bbR^n$, for all $f\in\mathcal{D}(U)$, for all $T\in\mathcal{D}'(U)$,
\[f*\delta_0=f,\]
\[T*\delta_0=T,\]
\[f*\delta_a=\tau_af,\]
\[T*\delta_a=\tau_aT.\]
\sssc{Sinc function as nascent delta distribution}
For any complex-valued or real-valued $\varphi\in L^1(\bbR)$:
\[\lim_{a\to 0}\int_{-\infty}^{\infty}\qty(\frac{\sin\qty(\frac{\pi x}{a})}{\pi x})\varphi(y-x)\dd{x}=\varphi(y).\]
\bpr
PLACEHOLDER
\epr
\ssc{Heaviside (step) distribution or unit step distribution}
Heaviside (step) distribution or unit step distribution at $a$, denoted as $H_a$ or $u_a$, is defined, for all $U\subseteq\bbR$, for all $\varphi\in\mathcal{D}(U)$, as
\[\langle H_a,\varphi\rangle=\int_{U\cap\bbR_{\geq a}}\varphi(x)\dd{x}.\]
The derivative of it is
\[H_a'=\delta_a.\]
\end{document}
