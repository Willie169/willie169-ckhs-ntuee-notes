\documentclass[a4paper,12pt]{article}
\setcounter{secnumdepth}{5}
\setcounter{tocdepth}{3}
\input{/usr/share/latex-toolkit/template.tex}
\begin{document}
\title{Integral Transform}
\author{沈威宇}
\date{\temtoday}
\titletocdoc
\sct{Integral Transform}
\ssc{Laplace Transform (拉普拉斯變換)}
\sssc{Introduction}
The one-sided or unilateral Laplace transform and two-sided or bilateral Laplace transform, where Laplace tranform may refer to either of them or both of them depending on context and often refer to the former when not clear, are two integral transform that convert a complex-valued signal, often denoted in lowercase such as $f$ or $f(t)$, of a real variable $t$, called time domain, to a complex-valued function, often denoted in uppercase such as $F$ or $F(s)$, as $\mathcal{L}\{f\}$, $\mathcal{L}\{f(t)\}$, $\mathcal{L}\{f\}(s)$, or $\mathcal{L}\{f(t)\}(s)$ for both one-sided and two-sided, or $\mathcal{B}\{f\}$, $\mathcal{B}\{f(t)\}$, $\mathcal{B}\{f\}(s)$, or $\mathcal{B}\{f(t)\}(s)$ for two-sided, of a complex variable $s$, called (complex) frequency domain, (complex) frequency plane, $s$-domain, $s$-plane, Laplace domain, or Laplace plane.
\sssc{Definition}
Let $R$ be $\bbR_{\geq0}$ for one-sided Laplace transform and $\bbR$ for two-sided Laplace transform.

For any complex-valued (or real-valued) functionals $f$ defined on $R$, the one-sided Laplace transform is defined as
\[\int_0^{\infty}e^{-st}f(t)\dd{t},\]
and the two-sided Laplace transform is defined as
\[\int_{-\infty}^{\infty}e^{-st}f(t)\dd{t},\]
where convergence of Laplace transform is defined as the finiteness of integral.

For any complex-valued (or real-valued) distribution $T\in\mathcal{D}(R)$ with compact support, the Laplace transform is defined as
\[\langle T,e^{-st}\rangle,\]
where Laplace transform is always convergent.

Distributions are often denoted in function notation but should be interpreted as distribution.

Laplace transform and derivative of sum of distribution and functional should be interpreted as the sum of the Laplace transform and derivative of them.

For one-sided Laplace transform, $\int_0^{\infty}$ is also denoted as $\int_{0^-}^{\infty}$ or $\lim_{\varepsilon\to 0^+}\int_{-\varepsilon}^{\infty}$, and $0$ is also denoted as $0^-$. This is for compatibility of Riemann–Stieltjes integral interpretation of Dirac delta and does not imply that the signal needs to be defined on $[a,0)$ for some $a<0$.
\sssc{Region of convergence (ROC)}
The set of values of $s$ for which $F(s)$ converges is called region of convergence (ROC). A function is called Laplace-transformable iff the ROC of its Laplace transform is nonempty.

For one-sided Laplace tranform, ROC is a class of the form of either $\Re(s) > a$ or $\Re(s) \geq a$ for some extended real constant $a$, called abscissa of convergence.

For two-sided Laplace tranform, ROC is a class of the form of either $\Re(s) > a$ or $\Re(s) \geq a$ and either $\Re(s) <  $ or $\Re(s) \leq b$ for some extended real constants $a$ and $b$, called abscissas of convergence.
\sssc{Of exponential order}
A complex-valued (or real-valued) functional $f$ defined on $\mathbb{R}_{\geq a}$ for some $a\in\bbR$ is of exponential order $\alpha\in\mathbb{R}_{>0}$ iff there exists $M>0$ and $T>0$ such that:
\[|f(t)|\leq Me^{\alpha t}\quad\forall t>T.\]
A complex-valued (or real-valued) functional $f$ defined on $\mathbb{R}_{\geq a}$ for some $a\in\bbR$ is of strict exponential order $\alpha\in\mathbb{R}_{>0}$ iff there exists $M>0$ and $T>0$ such that:
\[|f(t)|<Me^{\alpha t}\quad\forall t>T.\]
A complex-valued (or real-valued) functional $f$ defined on $\mathbb{R}_{\leq a}$ for some $a\in\bbR$ is of exponential order $\alpha\in\mathbb{R}_{<0}$ iff there exists $M>0$ and $T<0$ such that:
\[|f(t)|\leq Me^{\alpha t}\quad\forall t<T.\]
A complex-valued (or real-valued) functional $f$ defined on $\mathbb{R}_{\leq a}$ for some $a\in\bbR$ is of strict exponential order $\alpha\in\mathbb{R}_{<0}$ iff there exists $M>0$ and $T<0$ such that:
\[|f(t)|<Me^{\alpha t}\quad\forall t<T.\]
A complex-valued (or real-valued) functional $f$ defined on $\mathbb{R}$ is of exponential order iff both $f_{\bbR_{\geq0}}$ and $f_{\bbR_{\leq0}}$ are of exponential order.
\sssc{Laplace transformability theorem}
A complex-valued (or real-valued) functional $f$ defined on $R$ is one-sided Laplace-transformable if it is piecewise continuous on all closed subintervals of $R$ and of exponential order.
\begin{proof}
We will show the proof for one-sided Laplace transform, the proof for two-sided Laplace transforme is similar.

Assume that $f(t)$ is piecewise continuous on $[0,\infty)$ and there exists $M>0$ and $T>0$ such that:
\[|f(t)|\leq Me^{\alpha t}\quad \forall t>T.\]
For $\Re(s)>\alpha$, since $f(t)$ is piecewise continuous on $[0,\infty)$, there exist a locally finite indexed family $A=\{[a_i,b_i]\mid i\in I\}$ such that for each $i\in I\setminus\{j\}$, the integral
\[\int_{a_i}^{b_i} |f(t)| e^{-st}\dd{t}\]
exists and is finite.

Split the integral:
\[\int_0^{\infty}f(t)e^{-st}\dd{t}=\int_0^Tf(t)e^{-st}\dd{t}+\int_T^{\infty}f(t)e^{-st}\dd{t}.\]
For the first integral, by local finiteness, there exists a finite cover of $[0,T]$ that is a subset of $A$. The sum over finitely many finite integral is finite.

For the second integral, since $|f(t)| \le M e^{\alpha t}$, we have
\[\int_T^\infty |f(t)| e^{-st}\dd{t}\le \int_T^\infty M e^{\alpha t} e^{-st}\dd{t} = M \int_T^\infty e^{-(s-\alpha)t}\dd{t},\]
which converges since $\Re(s)>\alpha$.
\end{proof}
For one-sided Laplace transform, the ROC is
\bit
\item $\Re(s)\geq\alpha$ if $f$ is of strict exponential order $\alpha$ and not of exponential order $\beta$ for all $0<\beta<\alpha$, or
\item $\Re(s)>\alpha$ if $f$ is of exponential order $\alpha$, not of strict exponential order $\alpha$, and not of exponential order $\beta$ for all $0<\beta<\alpha$.
\eit
For two-sided Laplace transform, the ROC is the intersection of the following two regions:
\bit
\item First region:
\bit
\item $\Re(s)\geq\alpha_1$ if $f_{\bbR_{\geq0}}$ is of strict exponential order $\alpha_1$ and not of exponential order $\beta$ for all $0<\beta<\alpha_1$, or
\item $\Re(s)>\alpha_1$ if $f_{\bbR_{\geq0}}$ is of exponential order $\alpha_1$, not of strict exponential order $\alpha_1$, and not of exponential order $\beta$ for all $0<\beta<\alpha_1$.
\eit
\item Second region:
\bit
\item $\Re(s)\leq\alpha_2$ if $f_{\bbR_{\leq0}}$ is of strict exponential order $\alpha_2$ and not of exponential order $\beta$ for all $0<\beta<\alpha_2$, or
\item $\Re(s)<\alpha_2$ if $f_{\bbR_{\leq0}}$ is of exponential order $\alpha_2$, not of strict exponential order $\alpha_2$, and not of exponential order $\beta$ for all $0<\beta<\alpha_2$.
\eit
\eit
\sssc{Linearity}
\[\mathcal{L}\qty\{af(t)+bg(t)\}=a\mathcal{L}\qty\{f(t)\}+b\mathcal{L}\qty\{g(t)\},\quad a,b\in\bbC_{\neq0}.\]
The ROC is the intersection of the ROC of $\mathcal{L}\qty\{f(t)\}$ and $\mathcal{L}\qty\{g(t)\}$.
\sssc{Unit step function}
\[\mathcal{L}\qty\{u(t-a)\} = \frac{e^{-as}}{s},\quad a\in R.\]
The ROC is $\Re(s)>0$.
\sssc{Delta distribution}
\[\mathcal{L}\{\dv[n]{}{t}\delta(t-a)\}=s^ne^{-sa},\quad n\in\bbN_0\land a\in R.\]
The ROC is $\bbC$.
\sssc{Exponential function}
\[\mathcal{L}\qty\{e^{at}\} = \frac{1}{s-a},\quad a\in\bbC_{\neq0}.\]
For one-sided Laplace transformer, the ROC is $\Re(s)>\Re(a)$.

For two-sided Laplace transformer, the ROC is $\varnothing$.
\sssc{First shifting theorem or frequency shift}
\[\mathcal{L}\qty\{e^{at}f(t)\}=F(s-a),\quad a\in\bbC_{\neq0}.\]
For one-sided Laplace transformer, the ROC shifts right by $\Re(a)$.
\sssc{Second shifting theorem, time delay, or time shift}
\[\mathcal{L}\qty\{f(t-a)u(t-a)\}=e^{-as}F(s), \quad a\in\bbR_{\geq 0}.\]
The ROC doesn't change.
\begin{proof}
\[\ba
\mathcal{L}\qty\{f(t-a)u(t-a)\}&=\int_a^{\infty}e^{-st}f(t-a)\dd{t}\\
&=e^{-as}F(s)
\ea\]
\end{proof}
\sssc{Differentiation in time domain}
For one-sided Laplace transform of functional:
\[\mathcal{L}\qty\{f'(t)\}=sF(s)-f(0)+\lim_{t\to\infty}e^{-st}f(t).\]
\bpr
\[u=e^{-st},\quad\dd{u}=-se^{-st}\dd{t}.\]
\[\dd{v}=f'(t)\dd{t},\quad v=f(t).\]
\[\ba
\mathcal{L}\qty\{f'(t)\}&=\int_0^\infty e^{-st}f'(t)\dd{t}\\
&=\qty(e^{-st}f(t))\big\vert_0^\infty+\int_0^\infty f(t)se^{-st}\dd{t}\\
&=sF(s)-f(0)+\lim_{t\to\infty}e^{-st}f(t)
\ea\]
\epr
The ROC is the same as that of $F(s)$.

If $f$ is of exponential order, $\lim_{t\to\infty}e^{-st}f(t)=0$, and
\[\mathcal{L}\qty\{f'(t)\}=sF(s)-f(0).\]
For two-sided Laplace transform of functional:
\[\mathcal{L}\qty\{f'(t)\}=sF(s)-\lim_{t\to-\infty}e^{-st}f(t)+\lim_{t\to\infty}e^{-st}f(t).\]
\bpr
\[u=e^{-st},\quad\dd{u}=-se^{-st}\dd{t}.\]
\[\dd{v}=f'(t)\dd{t},\quad v=f(t).\]
\[\ba
\mathcal{L}\qty\{f'(t)\}&=\int_{-\infty}^\infty e^{-st}f'(t)\dd{t}\\
&=\qty(e^{-st}f(t))\big\vert_{-\infty}^\infty+\int_{-\infty}^\infty f(t)se^{-st}\dd{t}\\
&=sF(s)-\lim_{t\to-\infty}e^{-st}f(t)+\lim_{t\to\infty}e^{-st}f(t)
\ea\]
\epr
The ROC is the same as that of $F(s)$.

If $f$ is of exponential order, $\lim_{t\to-\infty}e^{-st}f(t)=\lim_{t\to\infty}e^{-st}f(t)=0$, and
\[\mathcal{L}\qty\{f'(t)\}=sF(s).\]
For Laplace transform of distribution:
\[\mathcal{L}\{T'\}=s\mathcal{L}\{T\}.\]
\bpr
\[\ba
\langle T',e^{-st}\rangle&=-\langle T,-se^{-st}\rangle\\
&=s\langle T,e^{-st}\rangle\\
&=s\mathcal{L}\{T\}
\ea\]
\epr
\sssc{Integration in time domain}
For one-sided Laplace transform of functional:
\[\mathcal{L}\qty\{\int_0^tf(\tau)\dd{\tau}\}=\frac{1}{s}F(s)-\frac{1}{s}\lim_{t\to\infty}e^{-st}\int_0^tf(\tau)\dd{\tau}.\]
\bpr
\[\ba
\mathcal{L}\qty\{\int_0^tf(\tau)\dd{\tau}\}&=\int_0^{\infty}e^{-st}\int_0^tf(\tau)\dd{\tau}\dd{t}\\
&=\qty(-\frac{1}{s}e^{-st}\int_0^tf(\tau)\dd{\tau})\big\vert_0^\infty+\int_0^{\infty}\frac{1}{s}e^{-st}f(t)\dd{t}\\
&=\frac{1}{s}F(s)-\frac{1}{s}\lim_{t\to\infty}e^{-st}\int_0^tf(\tau)\dd{\tau}
\ea\]
\epr
If $f$ is of exponential order, $\lim_{t\to\infty}e^{-st}\int_0^tf(\tau)\dd{\tau}=0$ for all $s$ in its ROC,
\[\mathcal{L}\qty\{\int_0^tf(\tau)\dd{\tau}\}=\frac{1}{s}F(s),\]
and the ROC is the same as that of $F(s)$.

For two-sided Laplace transform of functional:
\[\mathcal{L}\qty\{\int_{-\infty}^tf(\tau)\dd{\tau}\}=\frac{1}{s}F(s)-\frac{1}{s}\lim_{t\to\infty}e^{-st}\int_{-\infty}^tf(\tau)\dd{\tau}.\]
\bpr
\[\ba
\mathcal{L}\qty\{\int_{-\infty}^tf(\tau)\dd{\tau}\}&=\int_{-\infty}^{\infty}e^{-st}\int_{-\infty}^tf(\tau)\dd{\tau}\dd{t}\\
&=\qty(-\frac{1}{s}e^{-st}\int_{-\infty}^tf(\tau)\dd{\tau})\big\vert_{-\infty}^\infty+\int_{-\infty}^{\infty}\frac{1}{s}e^{-st}f(t)\dd{t}\\
&=\frac{1}{s}F(s)-\frac{1}{s}\lim_{t\to\infty}e^{-st}\int_{-\infty}^tf(\tau)\dd{\tau}
\ea\]
\epr
If $f$ is of exponential order, $\lim_{t\to\infty}e^{-st}\int_{-\infty}^tf(\tau)\dd{\tau}=0$ for all $s$ in its ROC,
\[\mathcal{L}\qty\{\int_{-\infty}^tf(\tau)\dd{\tau}\}=\frac{1}{s}F(s),\]
and the ROC is the same as that of $F(s)$.
\[\mathcal{L}\qty\{\int_t^{\infty}f(\tau)\dd{\tau}\}=\frac{1}{s}F(s)-\frac{1}{s}\lim_{t\to-\infty}e^{-st}\int_t^{\infty}f(\tau)\dd{\tau}.\]
\bpr
\[\ba
\mathcal{L}\qty\{\int_t^{\infty}f(\tau)\dd{\tau}\}&=\int_{-\infty}^{\infty}e^{-st}\int_t^{\infty}f(\tau)\dd{\tau}\dd{t}\\
&=\qty(-\frac{1}{s}e^{-st}\int_t^{\infty}f(\tau)\dd{\tau})\big\vert_{-\infty}^\infty+\int_{-\infty}^{\infty}\frac{1}{s}e^{-st}f(t)\dd{t}\\
&=\frac{1}{s}F(s)-\frac{1}{s}\lim_{t\to-\infty}e^{-st}\int_t^{\infty}f(\tau)\dd{\tau}
\ea\]
\epr
If $f$ is of exponential order, $\lim_{t\to-\infty}e^{-st}\int_t^{\infty}f(\tau)\dd{\tau}=0$ for all $s$ in its ROC,
\[\mathcal{L}\qty\{\int_t^{\infty}f(\tau)\dd{\tau}\}=\frac{1}{s}F(s),\]
and the ROC is the same as that of $F(s)$.

PLACEHOLDER 以下

\sssc{Differentiation in frequency domain}
\[\mathcal{L}\qty\{t f(t)\} = -\dv{}{s} F(s).\]
\[\mathcal{L}\qty\{t^n f(t)\} = (-1)^n \dv[n]{}{s} F(s).\]
\begin{proof}
One-sided: By Leibniz integral rule,
\bma
-\dv{}{s} F(s)&=-\dv{}{s}\int_0^{\infty} e^{-st} f(t)\dd{t}\\
&=-\int_0^{\infty} \pdv{}{s}\qty(e^{-st} f(t))\dd{t}\\
&=-\int_0^{\infty} -te^{-st} f(t)\dd{t}\\
&=\mathcal{L}\qty\{t f(t)\}
\eam
Two-sided: By Leibniz integral rule,
\bma
-\dv{}{s} F(s)&=-\dv{}{s}\int_{-\infty}^{\infty} e^{-st} f(t)\dd{t}\\
&=-\int_{-\infty}^{\infty} \pdv{}{s}\qty(e^{-st} f(t))\dd{t}\\
&=-\int_{-\infty}^{\infty} -te^{-st} f(t)\dd{t}\\
&=\mathcal{L}\qty\{t f(t)\}
\eam
\end{proof}
\sssc{Scaling in time domain for one-sided Laplace transform}
\[\mathcal{L}\qty\{f(at)\} = \frac{1}{a} F\left(\frac{s}{a}\right), \quad a>0.\]
\sssc{Scaling in time domain for two-sided Laplace transform}
\[\mathcal{L}\qty\{f(at)\} = \frac{1}{a} F\left(\frac{s}{a}\right), \quad a\neq 0.\]
\sssc{Convolution theorem}
\[\mathcal{L}\qty\{(f*g)(t)\} = F(s)G(s).\]
\begin{proof}
One-sided:
\[\mathcal{L}\qty\{(f*g)(t)\}=\int_0^{\infty} e^{-st} \qty(\int_0^t f(\tau)g(t-\tau)\dd{\tau})\dd{t}\]
By Fubini's theorem,
\[\mathcal{L}\qty\{(f*g)(t)\}=\int_0^\infty\int_\tau^\infty e^{-st} f(\tau)g(t-\tau)\dd{t}\dd{\tau}\]
Let $u=t-\tau$. $\dd{t}=\dd{u}$.
\[\begin{aligned}
\mathcal{L}\qty\{(f*g)(t)\}&=\int_0^\infty f(\tau)\int_0^\infty e^{-s(u+\tau)} g(u)\dd{u}\dd{\tau}\\
&=\int_0^\infty f(\tau)e^{-s\tau}\int_0^\infty e^{-su} g(u)\dd{u}\dd{\tau}\\
&=F(s)G(s)
\end{aligned}\]
Two-sided:
\[\mathcal{L}\qty\{(f*g)(t)\}=\int_{-\infty}^{\infty} e^{-st} \qty(\int_{-\infty}^t f(\tau)g(t-\tau)\dd{\tau})\dd{t}\]
By Fubini's theorem,
\[\mathcal{L}\qty\{(f*g)(t)\}=\int_{-\infty}^\infty\int_\tau^\infty e^{-st} f(\tau)g(t-\tau)\dd{t}\dd{\tau}\]
Let $u=t-\tau$. $\dd{t}=\dd{u}$.
\[\begin{aligned}
\mathcal{L}\qty\{(f*g)(t)\}&=\int_{-\infty}^\infty f(\tau)\int_{-\infty}^\infty e^{-s(u+\tau)} g(u)\dd{u}\dd{\tau}\\
&=\int_{-\infty}^\infty f(\tau)e^{-s\tau}\int_0
{-\infty}^\infty e^{-su} g(u)\dd{u}\dd{\tau}\\
&=F(s)G(s)
\end{aligned}\]
\end{proof}
\sssc{Initial Value Theorem (IVT) for one-sided Laplace tranform}
If $f$ has not unit impulse at $0$, then
\[\lim_{t\to 0^+}f(t)=\lim_{s\to\infty}sF(s).\]
\begin{proof}
\[sF(s)=\int_0^\infty sf(t)e^{-st}\dd{t}.\]
Let $u=st$, $\dd{t}=\frac{\dd{u}}{s}$.
\[sF(s)=\int_0^\infty f\qty(\frac{u}{s})e^{-u}\dd{u}.\]
\[\lim_{s\to\infty}sF(s)=\int_0^\infty f\qty(\frac{u}{s})e^{-u}\dd{u}.\]
We define a net of functions $\langle f_s(u)=f\qty(\frac{u}{s})\rangle_{s\in\mathbb{R}_{>s_0}}$, where $s_0$ is such that $F(s)$ converges for all $\Re(s)>s_0$.

For every fixed $u\in\mathbb{R}_{>0}$, $\lim_{s\to\infty}\frac{u}{s}\to 0^+$, so $f_s(u)$ pointwise converges to $f(0^+)$.

For dominated convergence theorem, we require an integrable function $g(u)$ such that
\[\abs{f\qty(\frac{u}{s})e^{-u}}\le g(u),\quad\forall s>0.\]
Since $f(t)$ is Laplace-transformable, it is of exponential order, that is, there exists $\alpha>0$, $M>0$, and $T>0$ such that:
\[|f(t)|\le Me^{\alpha t}\quad\forall t>T.\]
\[Me^{\alpha\frac{u}{s}}e^{-u}=Me^{-u\qty(1-\frac{\alpha}{s})}\le Me^{-\frac{u}{2}},\quad\forall s>2\alpha.\]
By dominated convergence theorem, we obtain:
\[\lim_{s\to\infty}sF(s)=\lim_{n\to\infty}\int_0^\infty f(0^+)e^{-u}\dd{u}=f(0^+)\int_0^\infty e^{-u}\dd{u}=f(0^+).\]
\end{proof}
\sssc{Final Value Theorem (FVT)}
PLACEHOLDER
\sssc{Cosine and sine functions}
\[\mathcal{L}\{\cos(\omega t)\}=\frac{s}{s^2+\omega^2}.\]
\begin{proof}
\[\ba
\mathcal{L}\{\cos(\omega t)\}&=\frac{1}{2}\qty(\mathcal{L}\{e^{i\omega t}\}+\mathcal{L}\{e^{-i\omega t}\})\\
&=\frac{1}{2}\qty(\frac{1}{s-i\omega}+\frac{1}{s+i\omega})\\
&=\frac{1}{2}\frac{2s}{s^2+\omega^2}\\
&=\frac{s}{s^2+\omega^2}
\ea\]
\end{proof}
\[\mathcal{L}\{\sin(\omega t)\}=\frac{\omega}{s^2+\omega^2}.\]
\begin{proof}
\[\ba
\mathcal{L}\{\sin(\omega t)\}&=\frac{1}{2i}\qty(\mathcal{L}\{e^{i\omega t}\}-\mathcal{L}\{e^{-i\omega t}\})\\
&=\frac{1}{2i}\qty(\frac{1}{s-i\omega}-\frac{1}{s+i\omega})\\
&=\frac{1}{2i}\frac{2i\omega}{s^2+\omega^2}\\
&=\frac{\omega}{s^2+\omega^2}
\ea\]
\end{proof}
\[\mathcal{L}\{\cos(\omega t-\varphi)\}=\frac{\cos\varphi s+\sin\varphi\omega}{s^2+\omega^2}.\]
\begin{proof}
\[\ba
\mathcal{L}\{\cos(\omega t-\varphi)\}&=\mathcal{L}\{\cos\varphi\cos(\omega t)+\sin\varphi\sin(\omega t)\}\\
&=\frac{\cos\varphi s+\sin\varphi\omega}{s^2+\omega^2}
\ea\]
\end{proof}
\[\mathcal{L}\{\sin(\omega t-\varphi)\}=\frac{\cos\varphi\omega-\sin\varphi s}{s^2+\omega^2}.\]
\begin{proof}
\[\ba
\mathcal{L}\{\sin(\omega t-\varphi)\}&=\mathcal{L}\{\cos\varphi\sin(\omega t)-\sin\varphi\cos(\omega t)\}\\
&=\frac{\cos\varphi\omega-\sin\varphi s}{s^2+\omega^2}
\ea\]
\end{proof}
\sssc{Hyperbolic cosine and sine functions}
\[\mathcal{L}\{\cosh(\omega t)\}=\frac{s}{s^2-\omega^2}.\]
\begin{proof}
\[\ba
\mathcal{L}\{\cosh(\omega t)\}&=\frac{1}{2}\qty(\mathcal{L}\{e^{\omega t}\}+\mathcal{L}\{e^{-\omega t}\})\\
&=\frac{1}{2}\qty(\frac{1}{s-\omega}+\frac{1}{s+\omega})\\
&=\frac{1}{2}\frac{2s}{s^2-\omega^2}\\
&=\frac{s}{s^2-\omega^2}
\ea\]
\end{proof}
\[\mathcal{L}\{\sinh(\omega t)\}=\frac{\omega}{s^2-\omega^2}.\]
\begin{proof}
\[\ba
\mathcal{L}\{\sinh(\omega t)\}&=\frac{1}{2}\qty(\mathcal{L}\{e^{\omega t}\}-\mathcal{L}\{e^{-\omega t}\})\\
&=\frac{1}{2}\qty(\frac{1}{s-\omega}-\frac{1}{s+\omega})\\
&=\frac{1}{2}\frac{2\omega}{s^2-\omega^2}\\
&=\frac{\omega}{s^2-\omega^2}
\ea\]
\end{proof}
\[\mathcal{L}\{\cosh(\omega t-\varphi)\}=\frac{\cosh\varphi s-\sinh\varphi\omega}{s^2-\omega^2}.\]
\begin{proof}
\[\ba
\mathcal{L}\{\cosh(\omega t-\varphi)\}&=\mathcal{L}\{\cosh\varphi\cosh(\omega t)-\sinh\varphi\sinh(\omega t)\}\\
&=\frac{\cosh\varphi s-\sinh\varphi\omega}{s^2-\omega^2}
\ea\]
\end{proof}
\[\mathcal{L}\{\sinh(\omega t-\varphi)\}=\frac{\cosh\varphi\omega-\sinh\varphi s}{s^2-\omega^2}.\]
\begin{proof}
\[\ba
\mathcal{L}\{\sinh(\omega t-\varphi)\}&=\mathcal{L}\{\cosh\varphi\sinh(\omega t)-\sinh\varphi\cosh(\omega t)\}\\
&=\frac{\cosh\varphi\omega-\sinh\varphi s}{s^2-\omega^2}
\ea\]
\end{proof}
\sssc{Power function}
\[\mathcal{L}\{t^q\}=\frac{q!}{s^{q+1}},\quad q\in\bbN_0.\]
\[\mathcal{L}\{t^q\}=\frac{\Gamma(q+1)}{s^{q+1}},\quad\Re(q)>-1.\]
The ROC is $\Re(s)>0$.
\begin{proof}
Let $u=st,\quad\dd{u}=s\dd{t}$.
\[\ba
\mathcal{L}\{t^q\}&=\frac{1}{s}\int_0^{\infty}e^{-u}\qty(\frac{u}{s})^q\dd{u}\\
&=\frac{1}{s^{q+1}}\int_0^{\infty}u^qe^{-u}\dd{u}\\
&=\frac{\Gamma(q+1)}{s^{q+1}}
\ea\]
\end{proof}
PLACEHOLDER: pole of order $\Re(q+1)$ at $s=0$. branches for $q\notin\bbZ$.
\ssc{Mellin's inverse formula, Bromwich integral, or Fourier–Mellin integral of Inverse Laplace transform (反拉普拉斯變換)}
PLACEHOLDER: mb move up
The inverse Laplace transform of a complex function $F(s)$, denoted as $\mathcal{L}^{-1}\{F(s)\}(t)$ or $f(t)$, is a defined as
\[\mathcal{L}^{-1}\{F(s)\}(t)=f(t)=\frac{1}{2\pi i}\lim_{T\to\infty}\int_{\gamma-iT}^{\gamma+iT}F(s)e^{st}\dd{s},\]
called Mellin's inverse formula, Bromwich integral, or Fourier–Mellin integral, where $\gamma$ is any real number such that it is greater than the real part of all singularities of $F$ and that $F$ is bounded on the line $s=\gamma$, and is such that
\[\mathcal{L}\qty\{f(t)\} = F(s).\]
\begin{proof}
We want to show that for all Laplace-transformable complex function $f$ without removable discontinuity and with $f(t)=0$ for all $\Re(t)<0$:
\[f(t)=\frac{1}{2\pi i}\lim_{T\to\infty}\int_{\gamma-iT}^{\gamma+iT}e^{st}\int_0^{\infty}f(u)e^{-su}\dd{u}\dd{s}.\]
By Fubini's theorem:
\[\frac{1}{2\pi i}\lim_{T\to\infty}\int_{\gamma-iT}^{\gamma+iT}e^{st}\int_0^{\infty}f(u)e^{-su}\dd{u}\dd{s}=\lim_{T\to\infty}\int_0^{\infty}f(u)\frac{1}{2\pi i}\int_{\gamma-iT}^{\gamma+iT}e^{s(t-u)}\dd{s}\dd{u}\]
Let
\[K_T(t-u)=\frac{1}{2\pi i}\int_{\gamma-iT}^{\gamma+iT}e^{s(t-u)}\dd{s}.\]
Let
\[s=\gamma+i\omega.\]
\[K_T(\tau)=\frac{e^{\gamma\tau}}{2\pi}\int_{-T}^Te^{i\omega\tau}\dd{\omega}=e^{\gamma\tau}\frac{\sin(T\tau)}{\pi\tau}.\]
Let
\[g(u)=f(u)e^{-\gamma u}.\]
Let
\[v=t-u.\]
\[\ba
\lim_{T\to\infty}\int_0^{\infty}f(u)\frac{1}{2\pi i}\int_{\gamma-iT}^{\gamma+iT}e^{s(t-u)}\dd{s}\dd{u}&=\lim_{T\to\infty}\int_0^{\infty}f(u)e^{\gamma(t-u)}\frac{\sin(T(t-u))}{\pi(t-u)}\dd{u}\\
&=e^{\gamma t}\lim_{T\to\infty}\int_0^{\infty}g(u)\frac{\sin(T(t-u))}{\pi(t-u)}\dd{u}\\
&=e^{\gamma t}\lim_{T\to\infty}\int_0^{\infty}g(t-v)\frac{\sin(Tv)}{\pi v}\dd{v}\\
&=e^{\gamma t}g(t)\\
&=f(t)
\ea\]
\end{proof}
\ssc{Fourier Transform (FT) (傅立葉變換)}
PLACEHOLDER: as Laplace tranform special case
\sssc{Fourier Transform (FT)}
Fourier transform is an integral transform that converts a function of a real variable (usually $t$, in the time domain) to a function of another real variable (usually $\omega$, in the frequency or Fourier domain). The functions are often denoted in lowercase for the time-domain representation and uppercase for the frequency-domain.

The Fourier transform, denoted as $\mathcal{F}\{f(t)\}(\omega)$ or $F(\omega)$, is defined by the improper integral
\[\mathcal{F}\{f(t)\}(\omega) = F(\omega) = \int_{-\infty}^{\infty}e^{-i\omega t}f(t)\dd{t}.\]
\sssc{Inverse Fourier Transform (反傅立葉變換)}
PLACEHOLDER
\ssc{Dirichlet kernel}
PLACEHOLDER
\end{document}