\documentclass[a4paper,12pt]{article}
\setcounter{secnumdepth}{5}
\setcounter{tocdepth}{3}
\input{/usr/share/latex-toolkit/template.tex}
\begin{document}
\title{Integral Transform}
\author{沈威宇}
\date{\temtoday}
\titletocdoc
\sct{Integral Transform}
\ssc{Laplace Transform (拉普拉斯變換)}
\sssc{Introduction}
The one-sided or unilateral Laplace transform and two-sided or bilateral Laplace transform, where Laplace tranform may refer to either of them or both of them depending on context and often refer to the former when not clear, are two integral transform that convert a complex-valued signal, often denoted in lowercase such as $f$ or $f(t)$, of a real variable $t$, called time domain, to a complex-valued function, often denoted in uppercase such as $F$ or $F(s)$, as $\mathcal{L}\{f\}$, $\mathcal{L}\{f(t)\}$, $\mathcal{L}\{f\}(s)$, or $\mathcal{L}\{f(t)\}(s)$ for both one-sided and two-sided, or $\mathcal{B}\{f\}$, $\mathcal{B}\{f(t)\}$, $\mathcal{B}\{f\}(s)$, or $\mathcal{B}\{f(t)\}(s)$ for two-sided, of a complex variable $s$, called (complex) frequency domain, (complex) frequency plane, $s$-domain, $s$-plane, Laplace domain, or Laplace plane.
\sssc{Definition}
Let $R$ be $\bbR_{\geq0}$ for one-sided Laplace transform and $\bbR$ for two-sided Laplace transform.

For any complex-valued (or real-valued) functional $f$ defined on $R$, the one-sided Laplace transform is defined as
\[\int_0^{\infty}e^{-st}f(t)\dd{t},\]
and the two-sided Laplace transform is defined as
\[\int_{-\infty}^{\infty}e^{-st}f(t)\dd{t},\]
where convergence of Laplace transform is defined as the finiteness of integral.

For any complex-valued (or real-valued) distribution $T\in\mathcal{D}(R)$ with compact support, the Laplace transform is defined as
\[\langle T,e^{-st}\rangle,\]
where Laplace transform is always convergent.

Distributions are often denoted in function inside integral notation, notably such as Dirac distribution, but should be interpreted as distribution.

Laplace transform and derivative of a sum of functional and distribution in function inside integral notation should be interpreted as the sum of the Laplace transform and derivative of it. Function value at a point of a functional and distribution in function inside integral notation should be interpreted as function value of the functional.

For one-sided Laplace transform, $\int_0^{\infty}$ is also denoted as $\int_{0^-}^{\infty}$ or $\lim_{\varepsilon\to 0^+}\int_{-\varepsilon}^{\infty}$, and $0$ is also denoted as $0^-$. This is for compatibility of Riemann–Stieltjes integral interpretation of Dirac delta and does not imply that the signal needs to be defined on $[a,0)$ for some $a<0$.

For one-sided transform, functional and distribution are implicitly assigned $0$ for all $t<0$.
\sssc{Region of convergence (ROC)}
The set of values of $s$ for which $F(s)$ converges is called region of convergence (ROC). A function is called Laplace-transformable iff the ROC of its Laplace transform is nonempty.

For one-sided Laplace tranform, ROC is a class of the form of either $\Re(s) > a$ or $\Re(s) \geq a$ for some extended real constant $a$, called abscissa of convergence.

For two-sided Laplace tranform, ROC is a class of the form of either $\Re(s) > a$ or $\Re(s) \geq a$ and either $\Re(s) <  $ or $\Re(s) \leq b$ for some extended real constants $a$ and $b$, called abscissas of convergence.
\sssc{Of exponential order}
A complex-valued (or real-valued) functional $f$ defined on $\mathbb{R}_{\geq a}$ for some $a\in\bbR$ is of exponential order $\alpha\in\mathbb{R}_{>0}$ iff there exists $M>0$ and $T>0$ such that:
\[|f(t)|\leq Me^{\alpha t}\quad\forall t>T.\]
A complex-valued (or real-valued) functional $f$ defined on $\mathbb{R}_{\geq a}$ for some $a\in\bbR$ is of strict exponential order $\alpha\in\mathbb{R}_{>0}$ iff there exists $M>0$ and $T>0$ such that:
\[|f(t)|<Me^{\alpha t}\quad\forall t>T.\]
A complex-valued (or real-valued) functional $f$ defined on $\mathbb{R}_{\leq a}$ for some $a\in\bbR$ is of exponential order $\alpha\in\mathbb{R}_{<0}$ iff there exists $M>0$ and $T<0$ such that:
\[|f(t)|\leq Me^{\alpha t}\quad\forall t<T.\]
A complex-valued (or real-valued) functional $f$ defined on $\mathbb{R}_{\leq a}$ for some $a\in\bbR$ is of strict exponential order $\alpha\in\mathbb{R}_{<0}$ iff there exists $M>0$ and $T<0$ such that:
\[|f(t)|<Me^{\alpha t}\quad\forall t<T.\]
A complex-valued (or real-valued) functional $f$ defined on $\mathbb{R}$ is of exponential order iff both $f_{\bbR_{\geq0}}$ and $f_{\bbR_{\leq0}}$ are of exponential order.
\sssc{Laplace transformability theorem}
A complex-valued (or real-valued) functional $f$ defined on $R$ is one-sided Laplace-transformable if it is piecewise continuous on all closed subintervals of $R$ and of exponential order.
\begin{proof}
We will show the proof for one-sided Laplace transform, the proof for two-sided Laplace transforme is similar.

Assume that $f(t)$ is piecewise continuous on $[0,\infty)$ and there exists $M>0$ and $T>0$ such that:
\[|f(t)|\leq Me^{\alpha t}\quad \forall t>T.\]
For $\Re(s)>\alpha$, since $f(t)$ is piecewise continuous on $[0,\infty)$, there exist a locally finite indexed family $A=\{[a_i,b_i]\mid i\in I\}$ such that for each $i\in I\setminus\{j\}$, the integral
\[\int_{a_i}^{b_i} |f(t)| e^{-st}\dd{t}\]
exists and is finite.

Split the integral:
\[\int_0^{\infty}f(t)e^{-st}\dd{t}=\int_0^Tf(t)e^{-st}\dd{t}+\int_T^{\infty}f(t)e^{-st}\dd{t}.\]
For the first integral, by local finiteness, there exists a finite cover of $[0,T]$ that is a subset of $A$. The sum over finitely many finite integral is finite.

For the second integral, since $|f(t)| \le M e^{\alpha t}$, we have
\[\int_T^\infty |f(t)| e^{-st}\dd{t}\le \int_T^\infty M e^{\alpha t} e^{-st}\dd{t} = M \int_T^\infty e^{-(s-\alpha)t}\dd{t},\]
which converges since $\Re(s)>\alpha$.
\end{proof}
For one-sided Laplace transform, the ROC is
\bit
\item $\Re(s)\geq\alpha$ if $f$ is of strict exponential order $\alpha$ and not of exponential order $\beta$ for all $0<\beta<\alpha$, or
\item $\Re(s)>\alpha$ if $f$ is of exponential order $\alpha$, not of strict exponential order $\alpha$, and not of exponential order $\beta$ for all $0<\beta<\alpha$.
\eit
For two-sided Laplace transform, the ROC is the intersection of the following two regions:
\bit
\item First region:
\bit
\item $\Re(s)\geq\alpha_1$ if $f_{\bbR_{\geq0}}$ is of strict exponential order $\alpha_1$ and not of exponential order $\beta$ for all $0<\beta<\alpha_1$, or
\item $\Re(s)>\alpha_1$ if $f_{\bbR_{\geq0}}$ is of exponential order $\alpha_1$, not of strict exponential order $\alpha_1$, and not of exponential order $\beta$ for all $0<\beta<\alpha_1$.
\eit
\item Second region:
\bit
\item $\Re(s)\leq\alpha_2$ if $f_{\bbR_{\leq0}}$ is of strict exponential order $\alpha_2$ and not of exponential order $\beta$ for all $0<\beta<\alpha_2$, or
\item $\Re(s)<\alpha_2$ if $f_{\bbR_{\leq0}}$ is of exponential order $\alpha_2$, not of strict exponential order $\alpha_2$, and not of exponential order $\beta$ for all $0<\beta<\alpha_2$.
\eit
\eit
\sssc{Linearity}
\[\mathcal{L}\qty\{af(t)+bg(t)\}=a\mathcal{L}\qty\{f(t)\}+b\mathcal{L}\qty\{g(t)\},\quad a,b\in\bbC_{\neq0}.\]
The ROC is the intersection of the ROC of $\mathcal{L}\qty\{f(t)\}$ and $\mathcal{L}\qty\{g(t)\}$.
\sssc{Unit step function}
\[\mathcal{L}\qty\{u(t-a)\} = \frac{e^{-as}}{s},\quad a\in R.\]
The ROC is $\Re(s)>0$.
\sssc{Delta distribution}
\[\mathcal{L}\{\dv[n]{}{t}\delta(t-a)\}=s^ne^{-sa},\quad n\in\bbN_0\land a\in R.\]
The ROC is $\bbC$.
\sssc{Exponential function}
\[\mathcal{L}\qty\{e^{at}\} = \frac{1}{s-a},\quad a\in\bbC_{\neq0}.\]
For one-sided Laplace transformer, the ROC is $\Re(s)>\Re(a)$.

For two-sided Laplace transformer, the ROC is $\varnothing$.
\sssc{First shifting theorem or frequency shift}
\[\mathcal{L}\qty\{e^{at}f(t)\}=F(s-a),\quad a\in\bbC_{\neq0}.\]
For one-sided Laplace transformer, the ROC shifts right by $\Re(a)$.

For two-sided Laplace transformer, the ROC change depends on $f$.
\sssc{Second shifting theorem, time delay, or time shift}
For one-sided Laplace transform:
\[\mathcal{L}\qty\{f(t-a)\}=e^{-as}F(s),\quad a\in\bbR_{\geq 0}.\]
The ROC doesn't change.
\begin{proof}
For functional:
\[\ba
\mathcal{L}\qty\{f(t-a)\}&=\int_a^{\infty}e^{-st}f(t-a)\dd{t}\\
&=\int_0^{\infty}e^{-s(u+a)}f(u)\dd{u}\\
&=e^{-as}F(s)
\ea\]
For distribution:
\[\langle\tau_aT,e^{-st}\rangle=\langle T,e^{-s(t+a)}\rangle=e^{-as}\langle T,e^{-st}.\]
\end{proof}
For two-sided Laplace transform:
\[\mathcal{L}\qty\{f(t-a)\}=e^{-as}F(s),\quad a\in\bbR_{\neq 0}.\]
The ROC doesn't change.
\begin{proof}
For functional:
\[\ba
\mathcal{L}\qty\{f(t-a)\}&=\int_{-\infty}^{\infty}e^{-st}f(t-a)\dd{t}\\
&=\int_{-\infty}^{\infty}e^{-s(u+a)}f(u)\dd{u}\\
&=e^{-as}F(s)
\ea\]
For distribution:
\[\langle\tau_aT,e^{-st}\rangle=\langle T,e^{-s(t+a)}\rangle=e^{-as}\langle T,e^{-st}.\]
\end{proof}
\sssc{Differentiation in time domain}
For one-sided Laplace transform of functional:
\[\mathcal{L}\qty\{f'(t)\}=sF(s)+\lim_{t\to\infty}e^{-st}f(t)-f(0),\]
\[\mathcal{L}\qty\{\dv[n]{f}{t}(t)\}=s^nF(s)+\sum_{i=0}^{n-1}s^{n-1-i}\qty(\lim_{t\to\infty}e^{-st}\dv[i]{f}{t}(t)-\dv[i]{f}{t}(0)).\]
\bpr
\[u=e^{-st},\quad\dd{u}=-se^{-st}\dd{t}.\]
\[\dd{v}=f'(t)\dd{t},\quad v=f(t).\]
\[\ba
\mathcal{L}\qty\{f'(t)\}&=\int_0^\infty e^{-st}f'(t)\dd{t}\\
&=\qty(e^{-st}f(t))\big\vert_0^\infty+\int_0^\infty f(t)se^{-st}\dd{t}\\
&=sF(s)+\lim_{t\to\infty}e^{-st}f(t)-f(0)
\ea\]
\epr
If $f$ is of exponential order, $\lim_{t\to\infty}e^{-st}f(t)=0$,
\[\mathcal{L}\qty\{f'(t)\}=sF(s)-f(0),\]
\[\mathcal{L}\qty\{\dv[n]{f}{t}(t)\}=s^nF(s)-\sum_{i=0}^{n-1}s^{n-1-i}\dv[i]{f}{t}(0),\]
and the ROC is a superset of that of $F(s)$.

For two-sided Laplace transform of functional:
\[\mathcal{L}\qty\{f'(t)\}=sF(s)+\lim_{t\to\infty}e^{-st}f(t)-\lim_{t\to-\infty}e^{-st}f(t),\]
\[\mathcal{L}\qty\{\dv[n]{f}{t}(t)\}=s^nF(s)+\sum_{i=0}^{n-1}s^{n-1-i}\qty(\lim_{t\to\infty}e^{-st}\dv[i]{f}{t}(t)-\lim_{t\to-\infty}e^{-st}\dv[i]{f}{t}(t)).\]
\bpr
\[u=e^{-st},\quad\dd{u}=-se^{-st}\dd{t}.\]
\[\dd{v}=f'(t)\dd{t},\quad v=f(t).\]
\[\ba
\mathcal{L}\qty\{f'(t)\}&=\int_{-\infty}^\infty e^{-st}f'(t)\dd{t}\\
&=\qty(e^{-st}f(t))\big\vert_{-\infty}^\infty+\int_{-\infty}^\infty f(t)se^{-st}\dd{t}\\
&=sF(s)+\lim_{t\to\infty}e^{-st}f(t)-\lim_{t\to-\infty}e^{-st}f(t)
\ea\]
\epr
If $f$ is of exponential order, $\lim_{t\to-\infty}e^{-st}f(t)=\lim_{t\to\infty}e^{-st}f(t)=0$,
\[\mathcal{L}\qty\{f'(t)\}=sF(s),\]
\[\mathcal{L}\qty\{\dv[n]{f}{t}(t)\}=s^nF(s),\]
and the ROC is a superset of that of $F(s)$.

For Laplace transform of distribution:
\[\mathcal{L}\{T'\}=s\mathcal{L}\{T\},\]
\[\mathcal{L}\{\dv[n]{T}{t}\}=s^n\mathcal{L}\{T\}.\]
\bpr
\[\ba
\langle T',e^{-st}\rangle&=-\langle T,-se^{-st}\rangle\\
&=s\langle T,e^{-st}\rangle\\
&=s\mathcal{L}\{T\}
\ea\]
\epr
\sssc{Integration in time domain}
For one-sided Laplace transform of functional:
\[\mathcal{L}\qty\{\int_0^tf(\tau)\dd{\tau}\}=\frac{1}{s}F(s)-\frac{1}{s}\lim_{t\to\infty}e^{-st}\int_0^tf(\tau)\dd{\tau}.\]
\bpr
\[\ba
\mathcal{L}\qty\{\int_0^tf(\tau)\dd{\tau}\}&=\int_0^{\infty}e^{-st}\int_0^tf(\tau)\dd{\tau}\dd{t}\\
&=\qty(-\frac{1}{s}e^{-st}\int_0^tf(\tau)\dd{\tau})\big\vert_0^\infty+\int_0^{\infty}\frac{1}{s}e^{-st}f(t)\dd{t}\\
&=\frac{1}{s}F(s)-\frac{1}{s}\lim_{t\to\infty}e^{-st}\int_0^tf(\tau)\dd{\tau}
\ea\]
\epr
If $f$ is of exponential order, $\lim_{t\to\infty}e^{-st}\int_0^tf(\tau)\dd{\tau}=0$ for all $s$ in its ROC,
\[\mathcal{L}\qty\{\int_0^tf(\tau)\dd{\tau}\}=\frac{1}{s}F(s),\]
and the ROC is $F(s)$ if there is a zero at $0$ of $F(s)$ and $F(s)\cap\{s\in\bbC\mid\Re(s)>0\}$ otherwise.

For two-sided Laplace transform of functional:
\[\mathcal{L}\qty\{\int_{-\infty}^tf(\tau)\dd{\tau}\}=\frac{1}{s}F(s)-\frac{1}{s}\lim_{t\to\infty}e^{-st}\int_{-\infty}^tf(\tau)\dd{\tau}.\]
\bpr
\[\ba
\mathcal{L}\qty\{\int_{-\infty}^tf(\tau)\dd{\tau}\}&=\int_{-\infty}^{\infty}e^{-st}\int_{-\infty}^tf(\tau)\dd{\tau}\dd{t}\\
&=\qty(-\frac{1}{s}e^{-st}\int_{-\infty}^tf(\tau)\dd{\tau})\big\vert_{-\infty}^\infty+\int_{-\infty}^{\infty}\frac{1}{s}e^{-st}f(t)\dd{t}\\
&=\frac{1}{s}F(s)-\frac{1}{s}\lim_{t\to\infty}e^{-st}\int_{-\infty}^tf(\tau)\dd{\tau}
\ea\]
\epr
If $f$ is of exponential order, $\lim_{t\to\infty}e^{-st}\int_{-\infty}^tf(\tau)\dd{\tau}=0$ for all $s$ in its ROC,
\[\mathcal{L}\qty\{\int_{-\infty}^tf(\tau)\dd{\tau}\}=\frac{1}{s}F(s),\]
and the ROC is $F(s)$ if there is a zero at $0$ of $F(s)$ and $F(s)\cap\{s\in\bbC\mid\Re(s)>0\}$$ otherwise.
\[\mathcal{L}\qty\{\int_t^{\infty}f(\tau)\dd{\tau}\}=-\frac{1}{s}F(s)-\frac{1}{s}\lim_{t\to-\infty}e^{-st}\int_t^{\infty}f(\tau)\dd{\tau}.\]
\bpr
\[\ba
\mathcal{L}\qty\{\int_t^{\infty}f(\tau)\dd{\tau}\}&=\int_{-\infty}^{\infty}e^{-st}\int_t^{\infty}f(\tau)\dd{\tau}\dd{t}\\
&=\qty(-\frac{1}{s}e^{-st}\int_t^{\infty}f(\tau)\dd{\tau})\big\vert_{-\infty}^\infty-\int_{-\infty}^{\infty}\frac{1}{s}e^{-st}f(t)\dd{t}\\
&=-\frac{1}{s}F(s)-\frac{1}{s}\lim_{t\to-\infty}e^{-st}\int_t^{\infty}f(\tau)\dd{\tau}
\ea\]
\epr
If $f$ is of exponential order, $\lim_{t\to-\infty}e^{-st}\int_t^{\infty}f(\tau)\dd{\tau}=0$ for all $s$ in its ROC,
\[\mathcal{L}\qty\{\int_t^{\infty}f(\tau)\dd{\tau}\}=-\frac{1}{s}F(s),\]
and the ROC is $F(s)$ if there is a zero at $0$ of $F(s)$ and $F(s)\cap\{s\in\bbC\mid\Re(s)<0\}$ otherwise.
\sssc{Differentiation in frequency domain}
\[\mathcal{L}\{tf(t)\}=-\dv{}{s}F(s),\]
\[\mathcal{L}\qty\{t^nf(t)\}=(-1)^n\dv[n]{}{s}F(s).\]
\begin{proof}
For functional, by Leibniz integral rule,
\[\ba
-\dv{}{s}F(s)&=-\dv{}{s}\int_Re^{-st}f(t)\dd{t}\\
&=-\int_R\pdv{}{s}\qty(e^{-st}f(t))\dd{t}\\
&=-\int_R-te^{-st}f(t)\dd{t}\\
&=\mathcal{L}\{tf(t)\}
\ea\]
For distribution:
\[\dv{}{s}\mathcal{L}\{T\}=\langle T,-te^{-st}\rangle=-\mathcal{L}\{tT\}.\]
\end{proof}
\sssc{Scaling in time domain}
For one-sided Laplace transform:
\[\mathcal{L}\qty\{f(at)\} = \frac{1}{a} F\left(\frac{s}{a}\right), \quad a>0.\]
For two-sided Laplace transform:
\[\mathcal{L}\qty\{f(at)\} = \frac{1}{a} F\left(\frac{s}{a}\right), \quad a\neq 0.\]
\bpr
For functional:
\[u=at,\quad\dd{t}=\frac{1}{a}\dd{u}\]
\[\int_Re^{-st}f(at)\dd{t}=\frac{1}{a}\int_Re^{-\frac{s}{a}u}f(u)\dd{u}\]
For distribution:
\[\langle S_aT,e^{-st}\rangle=\frac{1}{a}\langle T,e^{-\frac{st}{a}}\rangle.\]
\epr
\sssc{Complex conjugate theorem}
\[\mathcal{L}\{\ol{f{t)}\}=\ol{F(\ol{s})}.\]
The ROC is the same.
\bpr
\[\int_R\ol{f(t)}e^{-st}\dd{t}=\int_R\ol{f(t)e^{-\ol{s}t}}\dd{t}=\ol{\int_Rf(t)e^{-\ol{s}t}\dd{t}}.\]
\epr
\sssc{Convolution theorem for functionals}
For functionals:
\[\mathcal{L}\qty\{(f*g)(t)\}=F(s)G(s).\]
\bpr
By Fubini's theorem,
\[\ba
\mathcal{L}\qty\{(f*g)(t)\}&=\int_Re^{-st}\int_Rf(\tau)g(t-\tau)\dd{\tau}\dd{t},\quad u=t-\tau\\
&=\int_Re^{-s\tau}f(\tau)\dd{\tau}\int_Re^{-su}g(u)\dd{u}\\
&=F(s)G(s)
\ea\]
\epr
\sssc{Cross-correlation theorem for two-sided Laplace transform of functionals}
For two-sided Laplace transform of functionals:
\[\mathcal{L}\qty\{(f\star g)(t)\}=\ol{F(s)}G(s).\]
\bpr
By Fubini's theorem,
\[\ba
\mathcal{L}\qty\{(f\star g)(t)\}&=\int_{-\infty}^{\infty}e^{-st}\int_{-\infty}^{\infty}\ol{f(-\tau)}g(t-\tau)\dd{\tau}\dd{t},\quad u=t-\tau\\
&=\int_{-\infty}^{\infty}e^{-s\tau}\ol{f(-\tau)}\dd{\tau}\int_{-\infty}^{\infty}e^{-su}g(u)\dd{u}\\
&=\ol{F(s)}G(s)
\ea\]
\epr
\sssc{Initial Value Theorem (IVT) for one-sided Laplace tranform of functional}
For functional $f$ of expertial order and its one-sided Laplace transform $F(s)$, for $s\in\bbR$,
\[\lim_{t\to 0^+}f(t)=\lim_{s\to\infty}sF(s).\]
\begin{proof}
For $s\in\bbR$:
\[sF(s)=\int_0^\infty sf(t)e^{-st}\dd{t}.\]
Let $u=st$, $\dd{t}=\frac{\dd{u}}{s}$.
\[sF(s)=\int_0^\infty f\qty(\frac{u}{s})e^{-u}\dd{u}.\]
\[\lim_{s\to\infty}sF(s)=\lim_{s\to\infty}\int_0^\infty f\qty(\frac{u}{s})e^{-u}\dd{u}.\]
Define a net of functions $\langle f_s(u)=f\qty(\frac{u}{s})\rangle$ where $s$ is in the ROC of $F(s)$.

Since $f(t)$ is in $L^1(\bbR_{\geq0})$, $f(0^+)$ exists. For any fixed $u\in\mathbb{R}_{>0}$, $\lim_{s\to\infty}\frac{u}{s}\to 0^+$, so $f_s(u)$ pointwise converges to $f(0^+)$, and $f_s(u)e^{-u}$ pointwise converges to $f(0^+)e^{-u}$.

For dominated convergence theorem, we require an integrable function $g(u)$ such that
\[\abs{f\qty(\frac{u}{s})e^{-u}}\le g(u)\]
a.e. for all $s>0$.

Since $f(t)$ is in $L^1(\bbR_{\geq0})$, there exists $C\in\bbR$ such that
\[|f(t)|\le C\quad\forall 0\leq t\leq T.\]
Since $f(t)$ is of exponential order, there exists $\alpha>0$, $M>0$, and $T>0$ such that:
\[|f(t)|\le Me^{\alpha t}\quad\forall t>T.\]
\[f\qty(\frac{u}{s})e^{-u}\leq Ce^{-u}+Me^{\alpha\frac{u}{s}}e^{-u}=Ce^{-u}+Me^{-u\qty(1-\frac{\alpha}{s})}\leq Ce^{-u}+Me^{-\frac{u}{2}},\quad\forall s>2\alpha.\]
By dominated convergence theorem, we obtain:
\[\ba
\lim_{s\to\infty}\int_0^\infty f_s(u)e^{-u}\dd{u}&=\lim_{n\to\infty}\int_0^\infty f(0^+)e^{-u}\dd{u}\\
&=f(0^+)\int_0^\infty e^{-u}\dd{u}\\
&=f(0^+).
\ea\]
\end{proof}
\sssc{Final Value Theorem (FVT) for one-sided Laplace tranform of functional}
For functional $f(t)$ with $\lim_{t\to\infty}f(t)\in\bbC$ and its one-sided Laplace transform $F(s)$ with no pole in $\{s\in\bbC\mid \Re(s)\geq 0\}\setminus\{0\}$, for $\{s\in\bbC\mid \Re(s)\geq 0\}$, with $\lim_{s\to 0}sF(s)\in\bbC$,
\[\lim_{t\to\infty}f(t)=\lim_{s\to 0}sF(s).\]
\bpr
Since $\lim_{t\to\infty}f(t)$, there exists $M$ such that $\abs{f(t)}\leq M\in\bbR$ a.e. on $(a,\infty)$ for some $a\geq0$. Let
\[L=\lim_{t\to\infty}f(t).\]
For all $\varepsilon>0$, there exists $\delta>0$ such that $t>\delta$ implies $\abs{f(t)-L}\leq\varepsilon$.

Since
\[s\int_0^{\infty}e^{-st}\dd{t}=1,\]
for every $s$,
\[sF(s)-L=s\int_0^{\infty}(f(t)-L)e^{-st}\dd{t}.\]
Therefore,
\[\ba
\abs{sF(s)-L}&\leq s\int_0^{\infty}\abs{f(t)-L}e^{-st}\dd{t}\\
&\leq 2Ms\int_0^{\delta}e^{-st}\dd{t}+\varepsilon s\int_{\delta}^{\infty}e^{-st}\\
&\leq 2M\qty(1-e^{-s\delta})+\varepsilon s\int_0^{\infty}e^{-st}\\
&\leq 2Ms\delta+\varepsilon
\ea\]
\epr
\sssc{Cosine and sine functions}
\[\mathcal{L}\{\cos(\omega t)\}=\frac{s}{s^2+\omega^2}.\]
\begin{proof}
\[\ba
\mathcal{L}\{\cos(\omega t)\}&=\frac{1}{2}\qty(\mathcal{L}\{e^{i\omega t}\}+\mathcal{L}\{e^{-i\omega t}\})\\
&=\frac{1}{2}\qty(\frac{1}{s-i\omega}+\frac{1}{s+i\omega})\\
&=\frac{1}{2}\frac{2s}{s^2+\omega^2}\\
&=\frac{s}{s^2+\omega^2}
\ea\]
\end{proof}
\[\mathcal{L}\{\sin(\omega t)\}=\frac{\omega}{s^2+\omega^2}.\]
\begin{proof}
\[\ba
\mathcal{L}\{\sin(\omega t)\}&=\frac{1}{2i}\qty(\mathcal{L}\{e^{i\omega t}\}-\mathcal{L}\{e^{-i\omega t}\})\\
&=\frac{1}{2i}\qty(\frac{1}{s-i\omega}-\frac{1}{s+i\omega})\\
&=\frac{1}{2i}\frac{2i\omega}{s^2+\omega^2}\\
&=\frac{\omega}{s^2+\omega^2}
\ea\]
\end{proof}
\[\mathcal{L}\{\cos(\omega t-\varphi)\}=\frac{\cos\varphi s+\sin\varphi\omega}{s^2+\omega^2}.\]
\begin{proof}
\[\ba
\mathcal{L}\{\cos(\omega t-\varphi)\}&=\mathcal{L}\{\cos\varphi\cos(\omega t)+\sin\varphi\sin(\omega t)\}\\
&=\frac{\cos\varphi s+\sin\varphi\omega}{s^2+\omega^2}
\ea\]
\end{proof}
\[\mathcal{L}\{\sin(\omega t-\varphi)\}=\frac{\cos\varphi\omega-\sin\varphi s}{s^2+\omega^2}.\]
\begin{proof}
\[\ba
\mathcal{L}\{\sin(\omega t-\varphi)\}&=\mathcal{L}\{\cos\varphi\sin(\omega t)-\sin\varphi\cos(\omega t)\}\\
&=\frac{\cos\varphi\omega-\sin\varphi s}{s^2+\omega^2}
\ea\]
\end{proof}
\sssc{Hyperbolic cosine and sine functions}
\[\mathcal{L}\{\cosh(\omega t)\}=\frac{s}{s^2-\omega^2}.\]
\begin{proof}
\[\ba
\mathcal{L}\{\cosh(\omega t)\}&=\frac{1}{2}\qty(\mathcal{L}\{e^{\omega t}\}+\mathcal{L}\{e^{-\omega t}\})\\
&=\frac{1}{2}\qty(\frac{1}{s-\omega}+\frac{1}{s+\omega})\\
&=\frac{1}{2}\frac{2s}{s^2-\omega^2}\\
&=\frac{s}{s^2-\omega^2}
\ea\]
\end{proof}
\[\mathcal{L}\{\sinh(\omega t)\}=\frac{\omega}{s^2-\omega^2}.\]
\begin{proof}
\[\ba
\mathcal{L}\{\sinh(\omega t)\}&=\frac{1}{2}\qty(\mathcal{L}\{e^{\omega t}\}-\mathcal{L}\{e^{-\omega t}\})\\
&=\frac{1}{2}\qty(\frac{1}{s-\omega}-\frac{1}{s+\omega})\\
&=\frac{1}{2}\frac{2\omega}{s^2-\omega^2}\\
&=\frac{\omega}{s^2-\omega^2}
\ea\]
\end{proof}
\[\mathcal{L}\{\cosh(\omega t-\varphi)\}=\frac{\cosh\varphi s-\sinh\varphi\omega}{s^2-\omega^2}.\]
\begin{proof}
\[\ba
\mathcal{L}\{\cosh(\omega t-\varphi)\}&=\mathcal{L}\{\cosh\varphi\cosh(\omega t)-\sinh\varphi\sinh(\omega t)\}\\
&=\frac{\cosh\varphi s-\sinh\varphi\omega}{s^2-\omega^2}
\ea\]
\end{proof}
\[\mathcal{L}\{\sinh(\omega t-\varphi)\}=\frac{\cosh\varphi\omega-\sinh\varphi s}{s^2-\omega^2}.\]
\begin{proof}
\[\ba
\mathcal{L}\{\sinh(\omega t-\varphi)\}&=\mathcal{L}\{\cosh\varphi\sinh(\omega t)-\sinh\varphi\cosh(\omega t)\}\\
&=\frac{\cosh\varphi\omega-\sinh\varphi s}{s^2-\omega^2}
\ea\]
\end{proof}
\sssc{Power function}
\[\mathcal{L}\{t^q\}=\frac{q!}{s^{q+1}},\quad q\in\bbN_0.\]
\[\mathcal{L}\{t^q\}=\frac{\Gamma(q+1)}{s^{q+1}},\quad\Re(q)>-1.\]
The ROC is $\Re(s)>0$.
\begin{proof}
Let $u=st,\quad\dd{u}=s\dd{t}$.
\[\ba
\mathcal{L}\{t^q\}&=\frac{1}{s}\int_0^{\infty}e^{-u}\qty(\frac{u}{s})^q\dd{u}\\
&=\frac{1}{s^{q+1}}\int_0^{\infty}u^qe^{-u}\dd{u}\\
&=\frac{\Gamma(q+1)}{s^{q+1}}
\ea\]
\end{proof}
\sssc{Mellin's inverse formula, Bromwich integral, or (Fourier–Mellin integral of) inverse Laplace transform}
The inverse Laplace transform of a complex function $F(s)$, denoted as $\mathcal{L}^{-1}\{F(s)\}(t)$ or $f(t)$, is a complex-valued functional over the reals defined as
\[\mathcal{L}^{-1}\{F(s)\}(t)=f(t)=\frac{1}{2\pi i}\lim_{T\to\infty}\int_{\gamma-iT}^{\gamma+iT}F(s)e^{st}\dd{s},\]
where $\gamma$ is any real number such that it is greater than the real parts of all singularities of $F$ and that $F$ is bounded on the line $s=\gamma$.

For any complex-valued functional $f$ defined on $\bbR$ such that $f(t)e^{-\gamma t}\in L^1(\bbR)$, the inverse Laplace transform of the two-sided Laplace transform of $f(t)$, $F(s)$, is equal to $f(t)$ a.e.
\begin{proof}
We want to show that:
\[f(t)=\frac{1}{2\pi i}\lim_{T\to\infty}\int_{\gamma-iT}^{\gamma+iT}e^{st}\int_{-\infty}^{\infty}f(u)e^{-su}\dd{u}\dd{s}\]
a.e.

By Fubini's theorem:
\[\frac{1}{2\pi i}\lim_{T\to\infty}\int_{\gamma-iT}^{\gamma+iT}e^{st}\int_{-\infty}^{\infty}f(u)e^{-su}\dd{u}\dd{s}=\lim_{T\to\infty}\int_{-\infty}^{\infty}f(u)\frac{1}{2\pi i}\int_{\gamma-iT}^{\gamma+iT}e^{s(t-u)}\dd{s}\dd{u}.\]
Let
\[K_T(t-u)=\frac{1}{2\pi i}\int_{\gamma-iT}^{\gamma+iT}e^{s(t-u)}\dd{s}.\]
Let
\[s=\gamma+i\omega.\]
\[K_T(\tau)=\frac{e^{\gamma\tau}}{2\pi}\int_{-T}^Te^{i\omega\tau}\dd{\omega}=e^{\gamma\tau}\frac{\sin(T\tau)}{\pi\tau}.\]
Let
\[g(u)=f(u)e^{-\gamma u}.\]
Let
\[v=t-u.\]
\[\ba
\lim_{T\to\infty}\int_{-\infty}^{\infty}f(u)\frac{1}{2\pi i}\int_{\gamma-iT}^{\gamma+iT}e^{s(t-u)}\dd{s}\dd{u}&=\lim_{T\to\infty}\int_{-\infty}^{\infty}f(u)e^{\gamma(t-u)}\frac{\sin(T(t-u))}{\pi(t-u)}\dd{u}\\
&=e^{\gamma t}\lim_{T\to\infty}\int_{-\infty}^{\infty}g(u)\frac{\sin(T(t-u))}{\pi(t-u)}\dd{u}\\
&=e^{\gamma t}\lim_{T\to\infty}\int_{-\infty}^{\infty}g(t-v)\frac{\sin(Tv)}{\pi v}\dd{v}\\
&=e^{\gamma t}g(t)\\
&=f(t)
\ea\]
\end{proof}
For any complex-valued functional $f$ defined on $\bbR$ such that $f(t)=0$ for all $\Re(t)<0$ and $f(t)e^{-\gamma t}\in L^1(\bbR_{\geq0})$, the inverse Laplace transform of the one-sided Laplace transform of $f(t)$, $F(s)$, is equal to $f(t)$ a.e.




PLACEHOLDER
\ssc{Fourier Transform (FT) (傅立葉變換)}
PLACEHOLDER: mb as Laplace tranform special case?
\sssc{Fourier Transform (FT)}
Fourier transform is an integral transform that converts a function of a real variable (usually $t$, in the time domain) to a function of another real variable (usually $\omega$, in the frequency or Fourier domain). The functions are often denoted in lowercase for the time-domain representation and uppercase for the frequency-domain.

The Fourier transform, denoted as $\mathcal{F}\{f(t)\}(\omega)$ or $F(\omega)$, is defined by the improper integral
\[\mathcal{F}\{f(t)\}(\omega) = F(\omega) = \int_{-\infty}^{\infty}e^{-i\omega t}f(t)\dd{t}.\]
\sssc{Inverse Fourier Transform (反傅立葉變換)}
PLACEHOLDER
\ssc{Dirichlet kernel}
PLACEHOLDER
\end{document}