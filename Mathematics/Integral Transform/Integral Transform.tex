\documentclass[a4paper,12pt]{article}
\setcounter{secnumdepth}{5}
\setcounter{tocdepth}{3}
\input{/usr/share/latex-toolkit/template.tex}
\begin{document}
\title{Integral Transform}
\author{沈威宇}
\date{\temtoday}
\titletocdoc
\sct{Integral Transform}
\ssc{Laplace Transform (拉普拉斯變換)}
\sssc{Introduction}
The one-sided or unilateral Laplace transform and two-sided or bilateral Laplace transform, where Laplace tranform may refer to either of them or both of them depending on context and often refer to the former when not clear, are two integral transform that convert a function, often denoted in lowercase such as $f$ or $f(t)$, of a real variable $t$, called time domain, to a complex function, often denoted in uppercase such as $F$ or $F(s)$, as $\mathcal{L}\{f\}$, $\mathcal{L}\{f(t)\}$, $\mathcal{L}\{f\}(s)$, or $\mathcal{L}\{f(t)\}(s)$ for both one-sided and two-sided, or $\mathcal{B}\{f\}$, $\mathcal{B}\{f(t)\}$, $\mathcal{B}\{f\}(s)$, or $\mathcal{B}\{f(t)\}(s)$ for two-sided, of a complex variable $s$, called (complex) frequency domain, (complex) frequency plane, $s$-domain, $s$-plane, Laplace domain, or Laplace plane.
\sssc{Definition}
Given the distribution, often called function, $f$,
\[f(t)=\sum_{j\in S}g_j(t)\sum_{i\in I_j}\delta(A_{j_i}),\]
for any set $S$, for any complex-valued functions $g_j(t)$ defined on $D_j\superseteq\mathbb{R}_{\geq 0}$ for one-sided Laplace tranform and $D_j=\mathbb{R}$ for two-sided Laplace tranform, for any indexed family $A_j\colon I_j\to\mathcal{A}_j\subseteq D_j;\;i\mapsto A_{j_i}$ with $I_j=J_j\cap\bbN$ for some interval $J_j$.

As Lebesgue integrals with integrals involving Dirac detla functions defined via Dirac measures and convergence of a Laplace transform at a value $z$ of $s$ defined as the existence of such Lebesgue integral as $s=z$, the one-sided Laplace transform of it is defined  as
\[\int_0^{\infty}f(t)e^{-st}\dd{t},\]
and the two-sided Laplace transform of it is defined as
\[\int_{-\infty}^{\infty}f(t)e^{-st}\dd{t}.\]
As improper Riemann–Stieltjes integrals with integrals involving Dirac detla functions defined via Riemann–Stieltjes integrals and convergence of a Laplace transform at a value $z$ of $s$ defined as the absolute convergence of such improper Riemann–Stieltjes integral as $s=z$, the one-sided Laplace transform of it is defined, with $\mathcal{f}(t)$, typically denoted simply as $f(t)$, defined as any function such that
\bit
\item its domain is a super set of an interval $[a,\infty)$ for some $a\in\bbR_{<0}$,
\item $\mathcal{f}(t)=f(t)$ for all $t\in\bbR_{\geq 0}$, and
\item it's continuous on some interval $[b,0)$ for some $b\in\bbR_{<0}$,
\eit
as
\[\lim_{\varepsilon\to 0^+}\int_{-\varepsilon}^{\infty}\mathcal{f}(t)e^{-st}\dd{t},\]
where $\lim_{\varepsilon\to 0^+}\int_{-\varepsilon}^{\infty}$ is typically denoted as $\int_{0^-}^{\infty}$, and the two-sided Laplace transform of it is defined as
\[\int_{-\infty}^{\infty}f(t)e^{-st}\dd{t}.\]
As distribution
PLACEHOLDER
\sssc{Region of convergence (ROC)}
The set of values of $s$ for which $F(s)$ converges is called region of convergence (ROC). A function is called Laplace-transformable iff the ROC of its Laplace transform is nonempty.

For one-sided Laplace tranform, ROC is a class of the form of either $\Re(s) > a$ or $\Re(s) \geq a$ for some extended real constant $a$, called abscissa of convergence.

For two-sided Laplace tranform, ROC is a class of the form of either $\Re(s) > a$ or $\Re(s) \geq a$ and either $\Re(s) <  $ or $\Re(s) \leq b$ for some extended real constants $a$ and $b$, called abscissas of convergence.
\sssc{Laplace transformability theorem}
A complex-valued function $f(t)$ defined on $[0,\infty)$ is one-sided Laplace-transformable if it is piecewise continuous on all closed subintervals of $[0,\infty)$ and of exponential order.
\begin{proof}
Assume that a $f(t)$ is piecewise continuous on $[0,\infty)$ and there exists $M>0$ and $T>0$ such that:
\[|f(t)|\leq Me^{\alpha t}\quad \forall t>T.\]

For $s>\alpha$, since $f(t)$ is piecewise continuous on $[0,\infty)$, there exist a locally finite indexed family $A=\{[a_i,b_i]\mid i\in I\}$ such that for each $i\in I\setminus\{j\}$, the integral
\[\int_{a_i}^{b_i} |f(t)| e^{-st}\dd{t}\]
exists and is finite.

Split the integral:
\[\int_0^{\infty}f(t)e^{-st}\dd{t}=\int_0^Tf(t)e^{-st}\dd{t}+\int_T^{\infty}f(t)e^{-st}\dd{t}.\]
For the first integral, by local finiteness, there exists a finite cover of $[0,T]$ that is a subset of $A$. The sum over finitely many finite integral is finite.

For the second integral, since $|f(t)| \le M e^{\alpha t}$, we have
\[\int_T^\infty |f(t)| e^{-st}\dd{t}\le \int_T^\infty M e^{\alpha t} e^{-st}\dd{t} = M \int_T^\infty e^{-(s-\alpha)t}\dd{t},\]
which converges since $\Re(s)>\alpha$.
\end{proof}
A complex-valued function $f(t)$ defined on $\bbR$ is two-sided Laplace-transformable if it is piecewise continuous on all closed intervals and of exponential order and that $f(-t)$ is of exponential order.
\sssc{Unit impulses abscissa of convergence theorem for one-sided Laplace tranform}
For any indexed family $A\colon I\to\mathcal{A}\subseteq\bbR_{\geq 0};\;i\mapsto A_i$ with $I=J\cap\bbN$ for some interval $J\ni 1$ and $A_i\leq A_{i+1}$ for all $i\in I\land (i+1)\in I$,
\[\mathcal{L}\qty\{\sum_{i\in I}\delta(t-A_i)\}=\sum_{i\in I}e^{-sA_i}.\]
The ROC is $\sum_{i\in I}e^{-\Re(s)A_i}<\infty$, which, for $|I|<\aleph_0$, is $\bbC$, for $\abs{\{i\in I\mid A_i=0\}}=\aleph_0$, is $\varnothing$, and otherwise, with
\[N(x)\coloneq\abs{\{i\in I\mid A_i\leq x\}},\]
is
\[\int_0^{\infty}e^{-\Re(s)x}\dd{N(x)}<\infty,\]
that is,
\[\limsup_{x\to-\infty}\frac{\ln(N(x))}{x}>\Re(s)>\limsup_{x\to\infty}\frac{\ln(N(x))}{x},\]
that is,
\bit
\item for $\limsup_{i\to\infty}\frac{\ln(i)}{A_i}=\infty$, $\varnothing$, and
\item for $L\coloneq\limsup_{i\to\infty}\frac{\ln(i)}{A_i}\in\bbR$, $\Re(s)>L$.
\eit
For any complex-valued function $f(t)$, the ROC of
\[\mathcal{L}\qty\{f(t)\sum_{i\in I}\delta(t-A_i)\}\]
is $\sum_{i\in I}\abs{f\qty(A_i)}e^{-\Re(s)A_i}<\infty$, which, for $|I|<\aleph_0$, is $\bbC$, for $\abs{\{i\in I\mid A_i=0\}}=\aleph_0\land f(0)\neq 0$, is $\varnothing$, and otherwise, with
\[N(x)\coloneq\abs{\{i\in I\mid A_i\leq x\}},\]
is,
\[\int_0^{\infty}f(x)e^{-\Re(s)x}\dd{N(x)}<\infty,\]
that is,
\[\Re(s)>\limsup_{x\to\infty}\frac{\ln(f(x)N(x))}{x},\]
that is,
\bit
\item for $\limsup_{i\to\infty}\frac{\ln(\abs{f\qty(A_i)}i)}{A_i}=\infty$, $\varnothing$, and
\item for $L\coloneq\limsup_{i\to\infty}\frac{\ln(\abs{f\qty(A_i)}i)}{A_i}\in\bbR$, $\Re(s)>L$, and
\item for $\limsup_{i\to\infty}\frac{\ln(\abs{f\qty(A_i)}i)}{A_i}=\infty$, $\bbC$.
\eit
\sssc{Unit impulses abscissa of convergence theorem for two-sided Laplace tranform}
For any two indexed families $A\colon I\to\mathcal{A}\subseteq\bbR_{>0};\;i\mapsto A_i$ with $I=J\cap\bbN$ for some interval $J$ and $A_i\leq A_{i+1}$ for all $i\in I\land (i+1)\in I$ and $B\colon K\to\mathcal{B}\subseteq\bbR_{<0};\;i\mapsto B_i$ with $K=L\cap\bbN$ for some interval $L$ and $B_i\geq B_{i+1}$ for all $i\in I\land (i+1)\in K$, and $C\in\bbN_0\cup\{\aleph_0\}$, with either $1\in J\cup L$ or $C>0$,
\[\mathcal{L}\qty\{\sum_{i\in I}\delta(t-A_i)+\sum_{i\in K}\delta(t-B_i)+C\delta(t)\}=\sum_{i\in I}e^{-sA_i}+\sum_{i\in K}e^{-sB_i}+C.\]
The ROC is $\sum_{i\in I}e^{-sA_i}+\sum_{i\in K}e^{-sB_i}+C<\infty$, which is the intersection of the following two regions given by:
\bit
\item First region: for $|I|+C<\aleph_0$, is $\bbC$, for $C=\aleph_0$, is $\varnothing$, and otherwise, with
\[N_+(x)\coloneq\abs{\{i\in I\mid A_i\leq x\}}+C,\]
is
\[\int_0^{\infty}e^{-\Re(s)x}\dd{N_+(x)}<\infty,\]
that is,
\[\Re(s)>\limsup_{x\to\infty}\frac{\ln(N_+(x))}{x},\]
that is,
\bit
\item for $\limsup_{i\to\infty}\frac{\ln(i)}{A_i}=\infty$, $\varnothing$, and
\item for $L\coloneq\limsup_{i\to\infty}\frac{\ln(i)}{A_i}\in\bbR$, $\Re(s)>L$.
\eit
\item Second region: for $|K|+C<\aleph_0$, is $\bbC$, for $C=\aleph_0$, is $\varnothing$, and otherwise, with
\[N_-(x)\coloneq\abs{\{i\in I\mid B_i\geq -x\}}+C,\]
is
\[\int_0^{\infty}e^{\Re(s)x}\dd{N_-(x)}<\infty,\]
that is,
\[\Re(s)<-\limsup_{x\to-\infty}\frac{\ln(N_-(x))}{-x},\]
that is,
\bit
\item for $\limsup_{i\to\infty}\frac{\ln(i)}{-B_i}=\infty$, $\varnothing$, and
\item for $L\coloneq\limsup_{i\to\infty}\frac{\ln(i)}{-B_i}\in\bbR$, $\Re(s)<L$.
\eit
\eit
For any complex-valued function $f(t)$, the ROC of
\[\mathcal{L}\qty\{f(t)\sum_{i\in I}\delta(t-A_i)\}\]
is the intersection of the following two regions given by:
\bit
\item First region: for $|I|+C<\aleph_0$, is $\bbC$, for $C=\aleph_0\land f(0)\neq 0$, is $\varnothing$, and otherwise, with
\[N_+(x)\coloneq\abs{\{i\in I\mid A_i\leq x\}}+C,\]
is
\[\int_0^{\infty}f(x)e^{-\Re(s)x}\dd{N_+(x)}<\infty,\]
that is,
\[\Re(s)>\limsup_{x\to\infty}\frac{\ln(f(x)N_+(x))}{x},\]
that is,
\bit
\item for $\limsup_{i\to\infty}\frac{\ln(\abs{f\qty(A_i)}i)}{A_i}=\infty$, $\varnothing$, and
\item for $L\coloneq\limsup_{i\to\infty}\frac{\ln(\abs{f\qty(A_i)}i)}{A_i}\in\bbR$, $\Re(s)>L$, and
\item for $\limsup_{i\to\infty}\frac{\ln(\abs{f\qty(A_i)}i)}{A_i}=\infty$, $\bbC$.
\eit
\item Second region: for $|K|+C<\aleph_0$, is $\bbC$, for $C=\aleph_0\land f(0)\neq 0$, is $\varnothing$, and otherwise, with
\[N_-(x)\coloneq\abs{\{i\in I\mid B_i\geq -x\}}+C,\]
is
\[\int_0^{\infty}f(x)e^{\Re(s)x}\dd{N_-(x)}<\infty,\]
that is,
\[\Re(s)<-\limsup_{x\to-\infty}\frac{\ln(N_-(x))}{-x},\]
that is,
\bit
\item for $\limsup_{i\to\infty}\frac{\ln(\abs{f\qty(B_i)}i)}{-B_i}=\infty$, $\varnothing$, and
\item for $L\coloneq\limsup_{i\to\infty}\frac{\ln(\abs{f\qty(B_i)}i)}{-B_i}\in\bbR$, $\Re(s)<L$, and
\item for $\limsup_{i\to\infty}\frac{\ln(\abs{f\qty(B_i)}i)}{-B_i}=\infty$, $\bbC$.
\eit
\eit
\sssc{Linearity}
\[\mathcal{L}\qty\{af(t)+bg(t)\}=a\mathcal{L}\qty\{f(t)\}+b\mathcal{L}\qty\{g(t)\},\quad a,b\in\bbC.\]
For $a,b\neq 0$, the ROC is the intersection of the ROC of $\mathcal{L}\qty\{f(t)\}$ and $\mathcal{L}\qty\{g(t)\}$.
\sssc{Exponential function}
\[\mathcal{L}\qty\{e^{at}\}(s) = \frac{1}{s-a},\quad a\in\bbC.\]
\sssc{First shifting theorem or Frequency shift}
\[\mathcal{L}\qty\{e^{at} f(t)\}(s) = F(s-a),\quad a\in\bbC.\]
The ROC shifts right by $\Re(a)$.
\begin{proof}
\[\mathcal{L}\qty\{e^{at} f(t)\}(s)=\int_0^{\infty}e^{-(s-a)t}f(t)=F(s-a)\]
\end{proof}
\sssc{Unit step function}
\[\mathcal{L}\qty\{u(t-a)\}(s) = \frac{e^{-as}}{s},\quad a\in\bbR_{\geq 0}.\]
\sssc{Second shifting theorem, Time delay, or Time shift}
\[\mathcal{L}\qty\{f(t-a)u(t-a)\}=e^{-as}F(s), \quad a\in\bbR_{\geq 0}.\]
The ROC doesn't change.
\begin{proof}
\[\ba
\mathcal{L}\qty\{f(t-a)u(t-a)\}(s)&=\int_a^{\infty} e^{-st}f(t-a)\dd{t}\\
&=\int_0^{\infty}e^{-s\tau}e^{-as}f(\tau)\dd{\tau}\\
&=e^{-as}F(s)
\ea\]
\end{proof}
\sssc{Differentiation in time domain for one-sided Laplace transform}
\[\mathcal{L}\qty\{f'(t)\}(s) = sF(s) - f(0^-).\]
\[\mathcal{L}\qty\{f^{(n)}(t)\}(s) = s^n F(s) - \sum_{i=0}^{n-1}s^{n-1-i}f^{(i)}(0^-).\]
\begin{proof}
\[u = e^{-st}, \quad \dd{u} = -s e^{-st} \dd{t}.\]
\[\dd{v} = f'(t)\dd{t}, \quad v = f(t).\]
\[\ba
\mathcal{L}\qty\{f'(t)\}(s)&=\int_0^\infty e^{-st} f'(t)\dd{t}\\
&=\qty(e^{-st}f(t))\big\vert_0^\infty+\int_0^\infty f(t)se^{-st}\dd{t}\\
&=s\mathcal{L}\qty\{f(t)\}(s)-f(0^-)
\ea\]
\end{proof}
The ROC is the same as that of $F(s)$.
\sssc{Differentiation in time domain for two-sided Laplace transform}
\[\mathcal{L}\qty\{f'(t)\}(s) = sF(s).\]
\[\mathcal{L}\qty\{f^{(n)}(t)\}(s) = s^n F(s).\]
\begin{proof}
\[u = e^{-st}, \quad \dd{u} = -s e^{-st} \dd{t}.\]
\[\dd{v} = f'(t)\dd{t}, \quad v = f(t).\]
\[\ba
\mathcal{L}\qty\{f'(t)\}(s)&=\int_{-\infty}^\infty e^{-st} f'(t)\dd{t}\\
&=\qty(e^{-st}f(t))\big\vert_{-\infty}^\infty+\int_{-\infty}^\infty f(t)se^{-st}\dd{t}\\
&=s\mathcal{L}\qty\{f(t)\}(s)
\ea\]
\end{proof}
The ROC is the same as that of $F(s)$.


\sssc{Integration in Time Domain}
\[\mathcal{L}\left\{\int_0^t f(\tau)\dd{\tau}\right\}(s) = \frac{1}{s} F(s).\]
\begin{proof}
\bma
\mathcal{L}\qty\{\int_0^t f(\tau)\dd{\tau}\}(s) &= \int_0^{\infty} e^{-st} \int_0^t f(\tau)\dd{\tau}\dd{t}\\
&=\qty(-\frac{1}{s}e^{-st}\int_0^t f(\tau)\dd{\tau})\big\vert_0^\infty+\int_0^{\infty} \frac{1}{s}e^{-st}f(t)\dd{t}\\
&=\frac{1}{s}\int_0^{\infty} e^{-st}f(t)\dd{t}.
\eam
\end{proof}
\sssc{Differentiation in Frequency Domain}
\[\mathcal{L}\qty\{t f(t)\}(s) = -\dv{}{s} F(s).\]
\[\mathcal{L}\qty\{t^n f(t)\}(s) = (-1)^n \dv[n]{}{s} F(s).\]
\begin{proof}
By Leibniz integral rule,
\bma
-\dv{}{s} F(s)&=-\dv{}{s}\int_0^{\infty} e^{-st} f(t)\,\mathrm{d}t\\
&=-\int_0^{\infty} \pdv{}{s}\qty(e^{-st} f(t))\,\mathrm{d}t\\
&=-\int_0^{\infty} -te^{-st} f(t)\,\mathrm{d}t\\
&=\int_0^{\infty} te^{-st} f(t)\,\mathrm{d}t\\
&=\mathcal{L}\qty\{t f(t)\}(s)
\eam
\end{proof}
\sssc{Scaling in Time Domain}
\[\mathcal{L}\qty\{f(at)\}(s) = \frac{1}{a} F\left(\frac{s}{a}\right), \quad a>0.\]
\sssc{Convolution (卷積)}
The convolution of two functions $f(t)$ and $g(t)$ defined on $\bbR_{\geq 0}$ is a function denoted as $(f * g)(t)$ and defined as
\[(f * g)(t) = \int_0^t f(\tau)g(t-\tau)\dd{\tau}.\]
The convolution of two functions $f(t)$ and $g(t)$ defined on $\bbR$ is a function denoted as $(f * g)(t)$ and defined as
\[(f * g)(t) = \int_{-\infty}^{\infty} f(\tau)g(t-\tau)\dd{\tau}.\]
PLACEHOLDER
\sssc{Convolution Theorem}
If $h(t) = (f * g)(t)$, then
\[\mathcal{L}\qty\{h(t)\}(s) = F(s)G(s).\]
\begin{proof}
\[\mathcal{L}\qty\{h(t)\}(s)=\int_0^{\infty} e^{-st} \qty(\int_0^t f(\tau)g(t-\tau)\dd{\tau})\dd{t}\]
By Fubini's theorem,
\[\mathcal{L}\qty\{h(t)\}(s)=\int_0^\infty\int_\tau^\infty e^{-st} f(\tau)g(t-\tau)\dd{t}\dd{\tau}\]
Let $u=t-\tau$. $\dd{t}=\dd{u}$.
\[\begin{aligned}
\mathcal{L}\qty\{h(t)\}(s)&=\int_0^\infty f(\tau)\int_0^\infty e^{-s(u+\tau)} g(u)\dd{u}\dd{\tau}\\
&=\int_0^\infty f(\tau)e^{-s\tau}\int_0^\infty e^{-su} g(u)\dd{u}\dd{\tau}\\
&=F(s)G(s)
\end{aligned}\]
\end{proof}
Note that $\int_0^tf(\tau)\dd{\tau}=(f*u)(t)$.
PLACEHOLDER two-sided
\sssc{Initial Value Theorem}
If $f(t)$ and $f'(t)$ are Laplace-transformable and $f(t)$ contains no unit impulses at origin:
PLACEHOLDER: condition
\[\lim_{t\to 0^+}f(t)=\lim_{s\to\infty}sF(s).\]
\begin{proof}
\[sF(s)=\int_0^\infty sf(t)e^{-st}\dd{t}.\]
Let $u=st$, $\dd{t}=\frac{\dd{u}}{s}$.
\[sF(s)=\int_0^\infty f\qty(\frac{u}{s})e^{-u}\dd{u}.\]
\[\lim_{s\to\infty}sF(s)=\int_0^\infty f\qty(\frac{u}{s})e^{-u}\dd{u}.\]
We define a net of functions $\langle f_s(u)=f\qty(\frac{u}{s})\rangle_{s\in\mathbb{R}_{>s_0}}$, where $s_0$ is such that $F(s)$ converges for all $\Re(s)>s_0$.

For every fixed $u\in\mathbb{R}_{>0}$, $\lim_{s\to\infty}\frac{u}{s}\to 0^+$, so $f_s(u)$ pointwise converges to $f(0^+)$.

For dominated convergence theorem, we require an integrable function $g(u)$ such that
\[\abs{f\qty(\frac{u}{s})e^{-u}}\le g(u),\quad\forall s>0.\]
Since $f(t)$ is Laplace-transformable, it is of exponential order, that is, there exists $\alpha>0$, $M>0$, and $T>0$ such that:
\[|f(t)|\le Me^{\alpha t}\quad\forall t>T.\]
\[Me^{\alpha\frac{u}{s}}e^{-u}=Me^{-u\qty(1-\frac{\alpha}{s})}\le Me^{-\frac{u}{2}},\quad\forall s>2\alpha.\]
By dominated convergence theorem, we obtain:
\[\lim_{s\to\infty}sF(s)=\lim_{n\to\infty}\int_0^\infty f(0^+)e^{-u}\dd{u}=f(0^+)\int_0^\infty e^{-u}\dd{u}=f(0^+).\]
\end{proof}
\sssc{Final Value Theorem (FVT)}
PLACEHOLDER
\sssc{Cosine and sine functions}
\[\mathcal{L}\{\cos(\omega t)\}=\frac{s}{s^2+\omega^2}.\]
\begin{proof}
\[\ba
\mathcal{L}\{\cos(\omega t)\}&=\frac{1}{2}\qty(\mathcal{L}\{e^{i\omega t}\}+\mathcal{L}\{e^{-i\omega t}\})\\
&=\frac{1}{2}\qty(\frac{1}{s-i\omega}+\frac{1}{s+i\omega})\\
&=\frac{1}{2}\frac{2s}{s^2+\omega^2}\\
&=\frac{s}{s^2+\omega^2}
\ea\]
\end{proof}
\[\mathcal{L}\{\sin(\omega t)\}=\frac{\omega}{s^2+\omega^2}.\]
\begin{proof}
\[\ba
\mathcal{L}\{\sin(\omega t)\}&=\frac{1}{2i}\qty(\mathcal{L}\{e^{i\omega t}\}-\mathcal{L}\{e^{-i\omega t}\})\\
&=\frac{1}{2i}\qty(\frac{1}{s-i\omega}-\frac{1}{s+i\omega})\\
&=\frac{1}{2i}\frac{2i\omega}{s^2+\omega^2}\\
&=\frac{\omega}{s^2+\omega^2}
\ea\]
\end{proof}
\[\mathcal{L}\{\cos(\omega t-\varphi)\}=\frac{\cos\varphi s+\sin\varphi\omega}{s^2+\omega^2}.\]
\begin{proof}
\[\ba
\mathcal{L}\{\cos(\omega t-\varphi)\}&=\mathcal{L}\{\cos\varphi\cos(\omega t)+\sin\varphi\sin(\omega t)\}\\
&=\frac{\cos\varphi s+\sin\varphi\omega}{s^2+\omega^2}
\ea\]
\end{proof}
\[\mathcal{L}\{\sin(\omega t-\varphi)\}=\frac{\cos\varphi\omega-\sin\varphi s}{s^2+\omega^2}.\]
\begin{proof}
\[\ba
\mathcal{L}\{\sin(\omega t-\varphi)\}&=\mathcal{L}\{\cos\varphi\sin(\omega t)-\sin\varphi\cos(\omega t)\}\\
&=\frac{\cos\varphi\omega-\sin\varphi s}{s^2+\omega^2}
\ea\]
\end{proof}
\sssc{Hyperbolic cosine and sine functions}
\[\mathcal{L}\{\cosh(\omega t)\}=\frac{s}{s^2-\omega^2}.\]
\begin{proof}
\[\ba
\mathcal{L}\{\cosh(\omega t)\}&=\frac{1}{2}\qty(\mathcal{L}\{e^{\omega t}\}+\mathcal{L}\{e^{-\omega t}\})\\
&=\frac{1}{2}\qty(\frac{1}{s-\omega}+\frac{1}{s+\omega})\\
&=\frac{1}{2}\frac{2s}{s^2-\omega^2}\\
&=\frac{s}{s^2-\omega^2}
\ea\]
\end{proof}
\[\mathcal{L}\{\sinh(\omega t)\}=\frac{\omega}{s^2-\omega^2}.\]
\begin{proof}
\[\ba
\mathcal{L}\{\sinh(\omega t)\}&=\frac{1}{2}\qty(\mathcal{L}\{e^{\omega t}\}-\mathcal{L}\{e^{-\omega t}\})\\
&=\frac{1}{2}\qty(\frac{1}{s-\omega}-\frac{1}{s+\omega})\\
&=\frac{1}{2}\frac{2\omega}{s^2-\omega^2}\\
&=\frac{\omega}{s^2-\omega^2}
\ea\]
\end{proof}
\[\mathcal{L}\{\cosh(\omega t-\varphi)\}=\frac{\cosh\varphi s-\sinh\varphi\omega}{s^2-\omega^2}.\]
\begin{proof}
\[\ba
\mathcal{L}\{\cosh(\omega t-\varphi)\}&=\mathcal{L}\{\cosh\varphi\cosh(\omega t)-\sinh\varphi\sinh(\omega t)\}\\
&=\frac{\cosh\varphi s-\sinh\varphi\omega}{s^2-\omega^2}
\ea\]
\end{proof}
\[\mathcal{L}\{\sinh(\omega t-\varphi)\}=\frac{\cosh\varphi\omega-\sinh\varphi s}{s^2-\omega^2}.\]
\begin{proof}
\[\ba
\mathcal{L}\{\sinh(\omega t-\varphi)\}&=\mathcal{L}\{\cosh\varphi\sinh(\omega t)-\sinh\varphi\cosh(\omega t)\}\\
&=\frac{\cosh\varphi\omega-\sinh\varphi s}{s^2-\omega^2}
\ea\]
\end{proof}
\sssc{Power function}
\[\mathcal{L}\{t^q\}=\frac{q!}{s^{q+1}},\quad q\in\bbN_0.\]
\[\mathcal{L}\{t^q\}=\frac{\Gamma(q+1)}{s^{q+1}},\quad\Re(q)>-1.\]
The ROC is $\Re(s)>0$.
\begin{proof}
Let $u=st,\quad\dd{u}=s\dd{t}$.
\[\ba
\mathcal{L}\{t^q\}&=\frac{1}{s}\int_0^{\infty}e^{-u}\qty(\frac{u}{s})^q\dd{u}\\
&=\frac{1}{s^{q+1}}\int_0^{\infty}u^qe^{-u}\dd{u}\\
&=\frac{\Gamma(q+1)}{s^{q+1}}
\ea\]
\end{proof}
PLACEHOLDER: pole of order $\Re(q+1)$ at $s=0$. branches for $q\notin\bbZ$.
\ssc{Mellin's inverse formula, Bromwich integral, or Fourier–Mellin integral of Inverse Laplace transform (反拉普拉斯變換)}
PLACEHOLDER: mb move up
The inverse Laplace transform of a complex function $F(s)$, denoted as $\mathcal{L}^{-1}\{F(s)\}(t)$ or $f(t)$, is a defined as
\[\mathcal{L}^{-1}\{F(s)\}(t)=f(t)=\frac{1}{2\pi i}\lim_{T\to\infty}\int_{\gamma-iT}^{\gamma+iT}F(s)e^{st}\dd{s},\]
called Mellin's inverse formula, Bromwich integral, or Fourier–Mellin integral, where $\gamma$ is any real number such that it is greater than the real part of all singularities of $F$ and that $F$ is bounded on the line $s=\gamma$, and is such that
\[\mathcal{L}\qty\{f(t)\}(s) = F(s).\]
\begin{proof}
We want to show that for all Laplace-transformable complex function $f$ without removable discontinuity and with $f(t)=0$ for all $\Re(t)<0$:
\[f(t)=\frac{1}{2\pi i}\lim_{T\to\infty}\int_{\gamma-iT}^{\gamma+iT}e^{st}\int_0^{\infty}f(u)e^{-su}\dd{u}\dd{s}.\]
By Fubini's theorem:
\[\frac{1}{2\pi i}\lim_{T\to\infty}\int_{\gamma-iT}^{\gamma+iT}e^{st}\int_0^{\infty}f(u)e^{-su}\dd{u}\dd{s}=\lim_{T\to\infty}\int_0^{\infty}f(u)\frac{1}{2\pi i}\int_{\gamma-iT}^{\gamma+iT}e^{s(t-u)}\dd{s}\dd{u}\]
Let
\[K_T(t-u)=\frac{1}{2\pi i}\int_{\gamma-iT}^{\gamma+iT}e^{s(t-u)}\dd{s}.\]
Let
\[s=\gamma+i\omega.\]
\[K_T(\tau)=\frac{e^{\gamma\tau}}{2\pi}\int_{-T}^Te^{i\omega\tau}\dd{\omega}=e^{\gamma\tau}\frac{\sin(T\tau)}{\pi\tau}.\]
Let
\[g(u)=f(u)e^{-\gamma u}.\]
Let
\[v=t-u.\]
\[\ba
\lim_{T\to\infty}\int_0^{\infty}f(u)\frac{1}{2\pi i}\int_{\gamma-iT}^{\gamma+iT}e^{s(t-u)}\dd{s}\dd{u}&=\lim_{T\to\infty}\int_0^{\infty}f(u)e^{\gamma(t-u)}\frac{\sin(T(t-u))}{\pi(t-u)}\dd{u}\\
&=e^{\gamma t}\lim_{T\to\infty}\int_0^{\infty}g(u)\frac{\sin(T(t-u))}{\pi(t-u)}\dd{u}\\
&=e^{\gamma t}\lim_{T\to\infty}\int_0^{\infty}g(t-v)\frac{\sin(Tv)}{\pi v}\dd{v}\\
&=e^{\gamma t}g(t)\\
&=f(t)
\ea\]
\end{proof}
\ssc{Fourier Transform (FT) (傅立葉變換)}
PLACEHOLDER: as Laplace tranform special case
\sssc{Fourier Transform (FT)}
Fourier transform is an integral transform that converts a function of a real variable (usually $t$, in the time domain) to a function of another real variable (usually $\omega$, in the frequency or Fourier domain). The functions are often denoted in lowercase for the time-domain representation and uppercase for the frequency-domain.

The Fourier transform, denoted as $\mathcal{F}\{f(t)\}(\omega)$ or $F(\omega)$, is defined by the improper integral
\[\mathcal{F}\{f(t)\}(\omega) = F(\omega) = \int_{-\infty}^{\infty}e^{-i\omega t}f(t)\dd{t}.\]
\sssc{Inverse Fourier Transform (反傅立葉變換)}
The inverse Fourier transform of a complex function $F(s)$, denoted as $\mathcal{F}^{-1}\{F(s)\}(t)$ or $f(t)$, is defined as a real function such that
\[\mathcal{F}\{f(t)\}(s) = F(s),\]
where $\mathcal {F}$ denotes the Fourier transform.

The inverse Laplace transform of a complex function $F(s)$ is given by the line integral:
\[f(t) = \frac{1}{2\pi}\int_{-\infty}^{\infty} F(\omega) e^{i\omega t}\dd{\omega} .\]
\ssc{Dirichlet kernel}
PLACEHOLDER
\end{document}