\documentclass[a4paper,12pt]{article}
\setcounter{secnumdepth}{5}
\setcounter{tocdepth}{3}
\input{/usr/share/latex-toolkit/template.tex}
\begin{document}
\title{Logic}
\author{沈威宇}
\date{\temtoday}
\titletocdoc
\section{Logic (邏輯)}
\ssc{Propositional logic (PL) (命題邏輯), statement logic (陳述邏輯), sentential logic (句子邏輯), propositional calculus (命題演算), statement calculus (陳述演算), sentential calculus (句子演算), or zeroth-order logic (零階邏輯)}
\sssc{Statement (陳述) or Proposition (命題)}
A meaningful declarative sentence that is either true or false but not true and false at the same time.
\ssc{First-order logic (一階邏輯)}
\sssc{Expression (表達式)}
An arrangement of symbols following the context-dependent, syntactic conventions of mathematical notation.
\sssc{Conditional sentence (條件句)}
A statement that declare that a statement $P$ must be true for another statement $Q$ to be true, denoted as $P\implies Q$, in which $P$ is called the antecedent or hypothesis and $Q$ is called the consequent or conclusion, and $P$ and $Q$ are called conditions.
\sssc{(Logical) equivalence ((邏輯)等價)}
Statements $P$ and $Q$ are said to be logical equivalent or logical equivalence of each other, denoted as $P\iff Q$, if $Q$ is true whenever $P$ is true and false whenever $P$ is false, and $P$ is true whenever $Q$ is true and false whenever $Q$ is false.
\sssc{Negation, (logical) not, or (logical) complement ((邏輯)非 or (邏輯)補)}
The negation of a proposition $P$, denoted as $P'$, $\neg P$, $\mathord{\sim}P$, or $\ol{P}$, is a proposition that is true whenever $P$ is false and false whenever $P$ is true.
\sssc{Inverse (否命題)}
The inverse of a conditional sentence $P \implies Q$ is $\neg P \implies \neg Q$.
\sssc{Converse (逆命題)}
The converse of a conditional sentence $P \implies Q$ is $Q \implies \neg P$.
\sssc{Contraposition, contrapositive, or transposition (否逆命題, 逆否命題, or 對偶命題)}
The contraposition, contrapositive, or transposition of a conditional sentence $P \implies Q$ is $\neg Q \implies \neg P$.

The law of contraposition says that a conditional statement is true if and only if its contraposition is true.
\sssc{De Morgan's laws for logic (邏輯的笛摩根定律)}
\[\neg (P\land Q)\iff (\neg P)\lor (\neg Q)\]
\[\neg (P\lor Q)\iff (\neg P)\land (\neg Q)\]
\sssc{Sufficient condition (充分條件)}
If \( A \implies B \), we say $B$ is a sufficient condition of $A$.
\sssc{Sufficient and necessary condition (充要條件 or 充分必要條件)}
If \( A \iff B \), we say $A$ is a sufficient and necessary condition of $B$.

\end{document}
