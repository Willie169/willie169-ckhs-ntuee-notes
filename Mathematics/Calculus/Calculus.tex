\documentclass[a4paper,12pt]{report}
\setcounter{secnumdepth}{5}
\setcounter{tocdepth}{3}
\input{/usr/share/LaTeX-ToolKit/template.tex}
\begin{document}
\title{Calculus}
\author{沈威宇}
\date{\temtoday}
\titletocdoc
\chapter{Calculus (微積分)}
\section{Function (函數)}
\subsection{Definition and notation}
A function is formed by three sets, the domain (定義域) $X$, the codomain (對應域) $Y$, and the graph $R$ that satisfy the three following conditions:
\[R\subseteq \{(x,y)\mid x\in X,y\in Y\}\]
\[\forall x\in X,\exists y\in Y,\left(x,y\right)\in R \]
\[ (x,y)\in R\land (x,z)\in R\implies y=z\]

A function $f$ that satisfies the above is denoted as:
\[f\colon X\to Y.\]
in which the domain $X$ is also denoted as $D_f$.

The range (值域) or image (像), denoted as $R_f$ or $f(X)$, is defined as:
\[\{ y \mid \exists x \in X, (x, y) \in R \}.\]

If $x\in X$ and $(x, y) \in R$, we write $y = f(x)$, in which $f(x)$ is called the image of $x$ under $f$, $x$ is called the independent variable (自變數 or 獨立變數) or input, and $y$ is called the dependent variable (應變數 or 依賴變數) or output; and the function $f$ is also denoted as:
\[f \colon X \to Y;\, x \mapsto y.\]

If $I$ is a subset of the domain of $f$, $f(I)$ is defined as
\[f(I)\coloneq\{y\mid\exists x\in I \tx{\ s.t.\ }f(x)=y\}.\]
\ssc{Type of Functions}
Consider a function $f$:
\[f\colon X\to Y;\,x\mapsto y.\]
\sssc{Injection (單射) or Injective function or One-to-one (一對一) function}
\[\forall a,b\in X\text{\ s.t.\ } f(a)=f(b)\colon a=b\]
\sssc{Many-to-one (多對一) function}
\[\exists a\neq b\in X\colon f(a)=f(b)\]
\sssc{Surjection (滿射 or 蓋射) or Surjective function or Onto function}
\[f(X)=Y\]
\sssc{Bijection (對射) or Bijective function or One-to-one (一對一) function or One-to-one correspondence (一一對應)}
Injective and surjective function.
\ssc{Increasing and Decreasing}
Consider a function $f$:
\[f\colon X\to Y;\,x\mapsto y,\]
in which $Y$ is a preordered set.
\sssc{Non-Decreasing (非遞減) or Monotone Increasing (單調遞增) function}
$f$ is non-decreasing on $I\subseteq X$ if and only if
\[\forall a,b\in I\colon a<b\implies f(a)\leq f(b).\]
$f$ is non-decreasing if and only if
\[\forall a,b\in X\colon a<b\implies f(a)\leq f(b).\]
\sssc{Strictly increasing (嚴格遞增) function}
$f$ is strictly increasing on $I\subseteq X$ if and only if
\[\forall a,b\in I\colon a<b\implies f(a)<f(b).\]
$f$ is strictly increasing if and only if
\[\forall a,b\in X\colon a<b\implies f(a)<f(b).\]
\sssc{Non-Increasing (非遞增) or Monotone Decreasing (單調遞減) function}
$f$ is non-increasing on $I\subseteq X$ if and only if
\[\forall a,b\in I\colon a<b\implies f(a)\geq f(b).\]
$f$ is non-increasing if and only if
\[\forall a,b\in X\colon a<b\implies f(a)\geq f(b).\]
\sssc{Strictly decreasing (嚴格遞減) function}
$f$ is strictly decreasing on $I\subseteq X$ if and only if
\[\forall a,b\in I\colon a<b\implies f(a)>f(b).\]
$f$ is strictly decreasing if and only if
\[\forall a,b\in X\colon a<b\implies f(a)>f(b).\]
\sssc{Monotone (單調) or Order-preserving function or Order homomorphism}
$f$ is monotone on $I\subseteq X$ if and only if it is either non-increasing or non-decreasing on $I$.

$f$ is monotone if and only if it is either non-increasing or non-decreasing.
\ssc{Of Exponential Order}
A real or complex-valued function $f$ with domain being a subset of $\mathbb{R}$ that is a directed set is considered of exponential order $\alpha\in\mathbb{R}_{>0}$ if there exists $M>0$ and $T>0$ such that:
\[|f(t)|\leq Me^{\alpha t}\quad \forall t>T.\]
\ssc{Symmetry}
Given a function $f\colon X\to Y$ such that $\forall x\in X\colon -x\in X$.
\sssc{Even Function (偶函數)}
$f$ is an even function if for all $x\in X$, $f(x)=f(-x)$.
\sssc{Odd Function (奇函數)}
$f$ is an odd function if for all $x\in X$, $-f(x)=f(-x)$.
\ssc{Transformation (變換)}
\sssc{Translation (平移)}
For any function $f\colon\mathbb{R}\to\mathbb{R}$, shifting $y=f(x)$ right by $h$ units and up by $k$ units on the $xy$ coordinate plane yields $y=f(x-h)+k$.
\sssc{Scaling (伸縮 or 縮放 or 拉伸)}
For any function $f\colon\mathbb{R}\to\mathbb{R}$, on the $xy$ coordinate plane, expand $y=f(x)$ vertically by $a$ times the original value with the $x$ axis as the reference line, and expand $y=af\qty(\frac{x}{b})$ horizontally by $b$ times the original value with the $y$ axis as the reference line, to obtain $y=af\qty(\frac{x}{b})$.
\ssc{Function composition (函數合成)}
For two functions $f\colon X\to Y$ and $g\colon V\to W$ such that $g(V)\subseteq X$, the composition of them, denoted as $(f\circ g)$, is defined as:
\[(f \circ g)\colon V\to Y;\,x\mapsto = f(g(x))\]
\ssc{Inverse function (反函數)}
For a bijective function $f\colon X\to Y$, the inverse of it, denoted as $f^{-1}$, is defined as:
\[f^{-1}\colon Y\to X;\,f(x)\mapsto x\]
\sssc{Preimage or Inverse image}
Given an open subset $B$ of the codomain of a function $f$ the preimage or inverse image of $f$ on $B$, denoted as $f^{-1}(B)$ is defined as
\[f^{-1}(B)=\{x\in X|f(x)\in B\}.\]
\ssc{Piecewise function (分段函數)}
A piecewise function is a function defined in the form:
\[f(x) =
\begin{cases}
f_1(x), & \quad x\in A_1, \\
f_2(x), & \quad x\in A_2, \\
\vdots \\
f_n(x), & \quad x \in A_n
\end{cases},\]
where
\[\bigcup_{i=1}^nA_i=D_f\land\forall i\neq j\land i,j\in\mathbb{N}\land i,j\leq n\colon A_i\cap A_j=\varnothing.\]
\subsection{Indicator function (指示函數或示性函數) or characteristic function (特徵函數)}
An indicator function or a characteristic function of a subset $A$ of a set $X$ is a function that maps elements of the subset to one, and all other elements to zero, often denoted as $1_A$.
\ssc{Common Functions}
\sssc{Power Function}
A power function if a function $f(x)=kx^a$ where $k\in\mathbb{C}$, called coefficient, and $a\in\mathbb{R}$, called exponent, are constants (some requires $k\in\mathbb{R}$ or $k=1$).
\sssc{Root Function}
A root function is a power function of which the exponent is $\frac{1}{n}$ where $n\in\mathbb{N}$.
\sssc{Reciprocal Function}
A reciprocal function is function $f(x)=\frac{1}{x}$.
\sssc{Polynomial (多項式)}
A polynomial is a function $P(x)=\sum_{k=0}^na_kx^k$ where $a_k\in\mathbb{R}$ or $\mathbb{C}$ are constants called coefficients and $n$ is a nonnegative integer called the degree of $P$.
\sssc{Rational Function (有理函數)}
A function $f$ is called a rational function if there exists two polynomials $P(x)$ and $Q(x)$ such that $f(x)=\frac{P(x)}{Q(x)}$ for all $x$ in the domain.
\sssc{Algebraic Function (代數函數)}
A function $f$ is called an algebraic function if there exists a polynomial $P(y,x)$ in two variables such that $P(f(x),x)=0$ for all $x$ in the domain.
\sssc{Heaviside step function (黑維塞階躍函數) or Unit step function (單位階躍函數)}
\[H(x)=u(x)=
\begin{cases}0,\quad &x<0\\
1,\quad &x\geq 0
\end{cases},\]
where $H(0)=u(0)$ is sometimes defined to be $\frac{1}{2}$ or other values instead.
\[H'(x)=u'(x)=\delta(x).\]
\sssc{Dirac delta function (狄拉克 δ 函數)}
\[
\delta(x) =
\begin{cases}
\infty, & x = 0, \\
0, & x \neq 0.
\end{cases}
\]
\[
\forall \epsilon > 0:\,\int_{a-\epsilon}^{a+\epsilon} f(x) \delta(x-a) \, \mathrm{d}x = f(a).
\]
\[
\int _{-\infty }^{\infty }\delta(x) \, \mathrm{d}x = 1.
\]
\[\int_0^x\delta(t)\,\mathrm{d}t=H(x).\]



\section{Sequences and Series (數列與級數)}
\ssc{Sequence}
\begin{itemize}
\item\textbf{Sequence}: A sequence in $X$ is a function of which the domain is a set $\{x\in\mathbb{Z}\mid l\leq x\leq m\}$ or $\{x\in\mathbb{Z}\mid l\leq x\}$ and the codomain is a topological space $X$, in which $l$ is an integer, usually $0$ or $1$, and $m$ is a integer, denoted as $\langle a_n\rangle$, $\{a_n\}$, $(a_n)$, $\langle a_n\rangle_{n=l}^m$, $\{a_n\}_{n=l}^m$, or $(a_n)_{n=l}^m$, with $m=\infty$ when its domain is $\{x\in\mathbb{Z}\mid l\leq x\}$, where the subscript $n$ refers to the $n$th element of the sequence, that is, the function value when the  independent variable is $n$.
\item\textbf{Finite Sequence (有限數列)}: A finite sequence is a sequence with finite terms, i.e. $\langle a_n\rangle_{n=l}^m,\quad m\in\mathbb{Z}$.
\item\textbf{Infinite sequence (無窮數列)}: An infinite sequence is a sequence with infinite terms, i.e. $\langle a_n\rangle_{n=l}^\infty$. Unless otherwise specified, sequences refer to infinite sequences.
\end{itemize}
\ssc{Series (級數)}
\begin{itemize}
\item\textbf{Series}: The sum of the terms of a sequence.
\item\textbf{Finite Series (有限級數)}: The sum of the terms of a finite sequence.
\item\textbf{Infinite Series (無窮級數)}: The sum of the terms of an infinite sequence.
\end{itemize}



\section{Limit (極限)}
\subsection{Limits of Real Sequences}
Below, we are discussing limits of sequences with domain $\mathbb{N}$ and codomain $\mathbb{R}$. For sequences $\langle a_n\rangle_{i=l}^\infty$ in $\mathbb{R}$, its limit is the same as the limit of another sequence $\langle b_n=a_{n+l-1}\rangle_{i=1}^\infty$.
\subsubsection{Definition}
For a real sequence \(\langle a_n\rangle\), the limit of \(\langle a_n\rangle\) (as $n$ approaches infinity), denoted as $\lim_{n \to \infty} a_n$ or $\lim_n a_n$, is defined as follows:
\[\lim_{n \to \infty} a_n = L \equiv \forall \epsilon > 0:\, \exists M \in\mathbb{N}\text{\ s.t.\ } n\in\mathbb{N}\land n \geq M\implies |a_n - L| < \epsilon.\]
In other words, as \(n\) becomes arbitrarily large, \(a_n\) gets arbitrarily close to \(L\).

If such $M$ exists, we say the limit exists or the sequence converges (收斂) to $L$; otherwise, we say the limit doesn't exist or the sequence diverge (發散).
\subsubsection{Infinite Limits}
\[\lim_{n\to \infty}a_n=\infty \equiv \forall M > 0, \exists M \in\mathbb{N} \text{\ s.t.\ } n\in\mathbb{N}\land n \geq M \implies a_n > M.\]
\[\lim_{n\to \infty}a_n=-\infty \equiv \forall M < 0, \exists M \in\mathbb{N} \text{\ s.t.\ } n\in\mathbb{N}\land n \geq M \implies a_n < M.\]
Note that if a limit equals to $\infty$ or $-\infty$, the limit doesn't exist.
\sssc{Preservation of equal to}
Given real sequences $\langle a_n\rangle$ and $\langle b_n\rangle$ which for all $k$ that is greater than or equal to a specific $j\in \mathbb{N}$, $a_k=b_k$ and that $\lim_{n\to\infty}a_n$ exists, then $\lim_{n\to\infty}b_n$ exists and that:
\[\lim_{n\to\infty}a_n=\lim_{n\to\infty}b_n.\]
\sssc{Preservation of less than or equal to}
Given real sequences $\langle a_n\rangle$ and $\langle b_n\rangle$ which for all $k$ that is greater than or equal to a specific $j\in \mathbb{N}$, $a_k\leq b_k$ and that both $\lim_{n\to\infty}a_n$ and $\lim_{n\to\infty}b_n$ exist, then:
\[\lim_{n\to\infty}a_n\leq\lim_{n\to\infty}b_n.\]
\sssc{Squeeze (夾擠) theorem or Sandwich (三明治) theorem}
Given real sequences $\langle a_n\rangle$, $\langle b_n\rangle$, and $\langle c_n\rangle$ which for all $k$ that is greater than or equal to a specific $j\in \mathbb{N}$:
\[a_k\leq c_k\leq b_k\]
and
\[\lim_{n\to\infty}a_n=\lim_{n\to\infty}b_n=L,\]
then: 
\[\lim_{n\to\infty}c_n=L.\]
\subsubsection{Monotone Convergence Theorem (單調收斂定理) or Completeness of the Real Number (實數的完備性)}
\textit{Statement.}
\begin{enumerate}[label=(\Alph*)]
\item For a non-decreasing and bounded-above sequence of real numbers $\langle a_n\rangle_{n\in\mathbb {N}}$:
\[\lim_{n\to\infty}a_n=\sup_n a_n\]
\item For a non-increasing and bounded-below sequence of real numbers $\langle a_n\rangle_{n\in\mathbb {N}}$:
\[\lim_{n\to\infty}a_n=\inf_n a_n\]
\end{enumerate}
\begin{proof}\mbox{}\\
Let $\{a_{n}\}$ be the set of values of $\langle a_n\rangle_{n\in\mathbb {N}}$. By assumption, $\{a_n\}$ is non-empty and bounded-above by $\sup_n a_n$. Let $c=\sup_n a_n$.
\[\forall\epsilon>0:\,\exists M\in\mathbb {N}\text{\ s.t.\ }c\geq a_M>c-\epsilon,\]
since otherwise $c-\epsilon$ is a strictly smaller upper bound of $\langle a_n\rangle$, contradicting the definition of the supremum. 

Then since $\langle a_n\rangle$ is non-decreasing, and $c$ is an upper bound:
\[\forall\epsilon>0:\,\exists M\in\mathbb {N}\text{\ s.t.\ }\forall n\geq M:\,|c-a_n|=c-a_n\leq c-a_M=|c-a_M|<\epsilon.\]
The proof of the (B) part is analogous or follows from (A) by considering $\langle -a_{n}\rangle_{n\in \mathbb{N}}$.
\end{proof}
\textit{Statement.}

If $\langle a_n\rangle_{n\in\mathbb {N}}$ is a monotone sequence of real numbers, i.e., if 
$a_n\leq a_{n+1}$ for every $n\geq 1$ or $a_n\geq a_{n+1}$ for every $n\geq 1$, then this sequence has a finite limit if and only if the sequence is bounded.
\begin{proof}
"If"-direction: The proof follows directly from the proposition.

"Only If"-direction: By $(\epsilon,\delta)$-definition of limit, every sequence $\langle a_n\rangle_{n\in\mathbb {N}}$ with a finite limit $L$ is necessarily bounded.
\end{proof}
\sssc{Arithmetic progression or sequence (等差數列)}
An arithmetic sequence is a sequence $\langle a_n\rangle=\langle a_1+(n-1)d\rangle$. 

Given $a$ and $b$, $\frac{a+b}{2}$ is called the median of an arithmetic sequence (等差中項).

\[\nexists\lim_{n\to\infty}a_n,\quad d\neq 0\]
\[\lim_{n\to\infty}a_n=a_1,\quad d=0\]
\sssc{Geometric progression or sequence (等比 or 幾何數列)}
A geometric sequence is a sequence $\langle a_n\rangle=\langle a_1\cdot r^{n-1}\rangle$, where $a_1r\neq 0$.

Given $a$ and $b$, $\pm\sqrt{ab}$ is called the median of an geometric sequence (等比中項).
\[\nexists\lim_{n\to\infty}a_n,\quad |d|\geq 1\land d\neq 1\]
\[\lim_{n\to\infty}a_n=a_1,\quad d=1\]
\[\lim_{n\to\infty}a_n=0,\quad |d|<1\]
\subsection{Limits of Real Series}
\subsubsection{Definition}
Let:
\[S_n = \sum_{i=1}^n a_i,\]
where \(a_i\) are terms of a real sequence. The limit of \(S_n\), denoted as \(\lim_{n\to\infty}S_n\) or \(\sum_{i=1}^{\infty}a_i\), is defined as the following:
\[\sum_{i=1}^{\infty}a_i = L \equiv \forall \epsilon > 0:\, \exists M \in\mathbb{N}\text{\ s.t.\ } n \geq M\implies |S_n - L| < \epsilon.\]

If such $M$ exists, we say the limit exists or the series converges to $L$; otherwise, we say the limit doesn't exist or the series diverge.
\subsubsection{Absolute convergence and conditional convergence}
A series $S_n=\sum_{i=1}^{\infty}a_i$ converges absolutely to $L$ if $\exists\lim_{n\to\infty}\sum_{i=1}^{\infty}\abs{a_1}\land\lim_{n\to\infty}\sum_{i=1}^{\infty}\abs{a_1}=L$. If $S_n$ is convergent but not convergent absolutely, we say $S_n$ converges conditionally.
\sssc{Preservation of equal to}
Given real sequences $\langle a_n\rangle$ and $\langle b_n\rangle$ which for all $k$ that is greater than or equal to a specific $j\in \mathbb{N}$:
\[\sum_{i=1}^ka_i=\sum_{i=1}^kb_i\]
and that $\sum_{i=1}^{\infty}a_i$ exists, then $\sum_{i=1}^{\infty}b_i$ exists and that
\[\sum_{i=1}^{\infty}a_i=\sum_{i=1}^{\infty}b_i.\]
\sssc{Preservation of less than or equal to}
Given real sequences $\langle a_n\rangle$ and $\langle b_n\rangle$ which for all $k$ that is greater than or equal to a specific $j\in \mathbb{N}$:
\[\sum_{i=1}^ka_i\leq\sum_{i=1}^kb_i\]
and that both $\sum_{i=1}^{\infty}a_i$ and $\sum_{i=1}^{\infty}b_i$ exist, then
\[\sum_{i=1}^{\infty}a_i\leq\sum_{i=1}^{\infty}b_i.\]
\sssc{Squeeze theorem or Sandwich theorem}
Given real sequences $\langle a_n\rangle$, $\langle b_n\rangle$, and $\langle c_n\rangle$ which for all $k$ that is greater than or equal to a specific $j\in \mathbb{N}$:
\[\sum_{i=1}^ka_i\leq\sum_{i=1}^kc_i\leq\sum_{i=1}^kb_i\]
and
\[\sum_{i=1}^{\infty}a_i=\sum_{i=1}^{\infty}b_i=L,\]
then: 
\[\sum_{i=1}^{\infty}c_i=L.\]
\sssc{Arithmetic series (等差級數)}
An arithmetic series is a series $S_n=\sum_{i=1}^na_i$, where $\langle a_n\rangle$ is an arithmetic sequence.
\[S_n=\frac{n}{2}\qty(a_1+a_n)=\frac{n}{2}\qty(2a_1+(n-1)d)=na_1+\frac{n(n-1)d}{2}\]
\[\nexists\lim_{n\to\infty}S_n,\quad a_1\neq 0\lor d\neq 0\]
\[\lim_{n\to\infty}S_n=0,\quad a_1=0\land d=0\]
\sssc{Geometric series (等比 or 幾何級數)}
A geometric series is a series $S_n=\sum_{i=1}^na_i$, where $\langle a_n\rangle$ is a geometric sequence.

\[S_n=\frac{a_1\qty(1-r^n)}{1-r},\quad r\neq 1\]
\[S_n=na_1,\quad r=1\]
\[\lim_{n\to\infty}S_n=\frac{a_1}{1-r},\quad \abs{r}<1\]
\[\nexists\lim_{n\to\infty}S_n,\quad \abs{r}\geq 1\]
\sssc{Riemann zeta function (黎曼 zeta 函數)}
\[\begin{aligned}
\zeta(s) &= \sum_{n=1}^\infty\frac{1}{n^s}\\
&= \frac{1}{\Gamma (s)}\int _0^\infty \frac {x^{s-1}}{e^x-1}\,\mathrm {d} x
\eam

Harmonic Series (調和級數):
\[S_n=\sum_{n=1}^n\frac{1}{n}\]
\[\nexists\sum_{n=1}^\infty\frac{1}{n}\]

Basel Problem (巴塞爾問題):
\[\zeta(2)=\frac{\pi^2}{6}\]

Other Even Positive Integers:
\[\zeta(4)=\frac{\pi^4}{90}\]
\[\zeta(6)=\frac{\pi^6}{945}\]
\[\zeta(8)=\frac{\pi^8}{9450}\]
\[\zeta(10)=\frac{\pi^{10}}{93555}\]
\[\zeta(12)=\frac{691\pi^{12}}{638512875}\]
\[\zeta(14)=\frac{2\pi^{14}}{18243225}\]

Infinity:
\[\lim_{n\to\infty}\zeta(n)=1\]
\sssc{Euler–Mascheroni constant (歐拉–馬斯克若尼常數)}
\[\begin{aligned}
\gamma &= \lim _{n\to \infty }\left(\left(\sum _{k=1}^n\frac {1}{k}\right)-\ln(n)\right)\\
&= \int _1^\infty \left(\frac{1}{\lfloor x\rfloor}-\frac{1}{x}\right)\,\mathrm{d}x
\end{aligned}\]
\sssc{Power series (冪級數)}
\[\begin{aligned}
\sum_{i=1}^ni &= \frac{n\qty(n+1)}{2}\\
\sum_{i=1}^ni^2 &= \frac{n\qty(n+1)\qty(2n+1)}{6}\\
\sum_{i=1}^ni^3 &= \qty(\frac{n(n+1)}{2})^2\\
\sum_{i=1}^ni^r &= n + \sum_{k=1}^{n-1} (n-k)((k+1)^r - k^r)\\
&= n + \sum_{k=1}^{n-1} (n-k)\sum_{j=0}^{r-1}\binom{r}{j}k^{j}
\end{aligned}\]
\subsection{Limits of Real Nets}
Below, we are discussing limits of nets in the set of real number.
\subsubsection{Definition}
For a real net $\langle x_a\rangle_{a\in A}$ in which $A$ is a directed set, the limit of $\langle x_a\rangle_{a\in A}$, denoted as $\lim_{a\in A} x_a$ or $\lim_a x_a$, is defined as follows:
\[\lim_ax_a = L \equiv \forall \epsilon > 0:\, \exists a_0 \in A\text{\ s.t.\ } a\in A\land a\ge a_0\implies |x_a - L| < \epsilon.\]

If such $a_0$ exists, we say the limit exists or the net converges to $L$; otherwise, we say the limit doesn't exist.
\subsubsection{Infinite Limits}
\[\lim_{a}x_a=\infty \equiv \forall M > 0, \exists a_0 \in A \text{\ s.t.\ } a\in A\land a\geq a_0 \implies x_a > M.\]
\[\lim_{a}x_a=-\infty \equiv \forall M < 0, \exists a_0 \in A \text{\ s.t.\ } a\in A\land a\geq a_0 \implies x_a < M.\]
Note that if a limit equals to $\infty$ or $-\infty$, the limit doesn't exist.
\sssc{Preservation of equal to}
Let $A$ be a directed set. Given real nets $\langle x_a\rangle_{a\in A}$ and $\langle b_n\rangle_{a\in A}$ which for all $k$ that is greater than or equal to a specific $j\in A$, $a_k=b_k$ and that $\lim_{a}x_a$ exists, then $\lim_{a}b_n$ exists and that:
\[\lim_{a}x_a=\lim_{a}b_n.\]
\sssc{Preservation of less than or equal to}
Let $A$ be a directed set. Given real nets $\langle x_a\rangle_{a\in A}$ and $\langle b_n\rangle_{a\in A}$ which for all $k$ that is greater than or equal to a specific $j\in A$, $a_k\leq b_k$ and that both $\lim_{a}x_a$ and $\lim_{a}b_n$ exist, then:
\[\lim_{a}x_a\leq\lim_{a}b_n.\]
\sssc{Squeeze theorem or Sandwich theorem}
Let $A$ be a directed set. Given real nets $\langle x_a\rangle_{a\in A}$, $\langle b_n\rangle_{a\in A}$, and $\langle c_n\rangle_{a\in A}$ which for all $k$ that is greater than or equal to a specific $j\in A$:
\[a_k\leq c_k\leq b_k\]
and
\[\lim_{a}x_a=\lim_{a}b_n=L,\]
then: 
\[\lim_{a}c_n=L.\]
\subsection{Limits of Functions with Real Domains}
\subsubsection{Definition at Finity}
Let \(I\) be an interval containing the point \(a\). Let \( f(x) \) be a function defined on \(I\), except possibly at \(a\) itself. The limit of \( f(x) \) as \( x \) approaches \( a \) is defined as follows:
\[\lim_{x \to a} f(x) = L \equiv \forall \epsilon > 0:\, \exists \delta > 0 \text{\ s.t.\ } 0 < |x - a| < \delta \implies |f(x) - L| < \epsilon.\]
In other words, as \(x\) becomes arbitrarily close to \(a\), \(f(x)\) gets arbitrarily close to \(L\).

If such $\detla$s exist, we say the limit exists; otherwise, we say the limit doesn't exist.
\subsubsection{Definition at Infinity}
Let \(I\) be a left-bounded, right-unbounded interval with the point \(a\) being its endpoint on the left. Let \( f(x) \) be a function defined on \(I\). The limit of \( f(x) \) as \( x \) approaches \( \infty \) is defined as follows:
\[\lim_{x \to \infty} f(x) = L \equiv \forall \epsilon > 0: \, \exists M > a \text{\ s.t.\ } x > M \implies |f(x) - L| < \epsilon.\]
In other words, as \(x\) becomes arbitrarily large, \(f(x)\) gets arbitrarily close to \(L\). We say $f(x)$ converge to $L$ as $x\to\infty$ if $\lim_{x \to \infty} f(x) = L$.

Let \(I\) be a right-bounded, left-unbounded interval with the point \(a\) being its endpoint on the right. Let \( f(x) \) be a function defined on \(I\). The limit of \( f(x) \) as \( x \) approaches \( -\infty \) is defined as follows:
\[\lim_{x \to -\infty} f(x) = L \equiv \forall \epsilon > 0: \, \exists M < a \text{\ s.t.\ } x < M \implies |f(x) - L| < \epsilon.\]
In other words, as \(x\) becomes arbitrarily small, \(f(x)\) gets arbitrarily close to \(L\). We say $f(x)$ converge to $L$ as $x\to-\infty$ if $\lim_{x \to -\infty} f(x) = L$.
\end{itemize}
\subsubsection{Horizontal asymptote (水平漸近線)}
\[ \qty(\lim_{x \to \infty} f(x)=L \lor\lim_{x \to -\infty} f(x)=L) \iff \qty(y=L\tx{ is a horizontal asymptote of $y=f(x)$}).\]
\subsubsection{Slant asymptote (斜漸近線)}
\[ \qty(\lim_{x \to \infty} f(x)-(mx+b)=0 \lor\lim_{x \to -\infty} f(x)-(mx+b)=0) \iff \qty(y=mx+b\tx{\ is a slant asymptote of $y=f(x)$}).\]
\subsubsection{One-side Limits}
\tb{Right-hand Limit (右極限)}: Let \(I\) be a left-open interval with the point \(a\) being its endpoint on the left. Let \( f(x) \) be a function defined on \(I\). The right-hand limit of \( f(x) \) as \( x \) approaches \( a \) is defined as follows:
\[\lim_{x \to a^+} f(x) = L \equiv \forall \epsilon > 0 :\,\exists \delta > 0 \text{\ s.t.\ } 0 < x - a < \delta \implies |f(x) - L| < \epsilon.\]
In other words, as \(x\) becomes arbitrarily close to \(a\) and is greater than \(a\), \(f(x)\) gets arbitrarily close to \(L\). For a function $f(x)$, $\lim_{x\to a^+}f(x)$ can also be denoted as $f(a^+)$.

\tb{Left-hand Limit (左極限)}: Let \(I\) be a right-open interval with the point \(a\) being its endpoint on the right. Let \( f(x) \) be a function defined on \(I\). The left-hand limit of \( f(x) \) as \( x \) approaches \( a \) is defined as follows:
\[\lim_{x \to a^-} f(x) = L \equiv \forall \epsilon > 0 :\,\exists \delta > 0 \text{\ s.t.\ } 0 < a - x < \delta \implies |f(x) - L| < \epsilon.\]
In other words, as \(x\) becomes arbitrarily close to \(a\) and is less than \(a\), \(f(x)\) gets arbitrarily close to \(L\). For a function $f(x)$, $\lim_{x\to a^-}f(x)$ can also be denoted as $f(a^-)$.
\subsubsection{Infinite Limits}
\[\lim_{x\to a}f(x)=\infty \equiv \forall N > 0:\, \exists \delta > 0 \text{\ s.t.\ } 0 < |x - a| < \delta \implies f(x) > N.\]
\[\lim_{x\to a^+}f(x)=\infty \equiv \forall N > 0:\, \exists \delta > 0 \text{\ s.t.\ } 0 < x - a < \delta \implies f(x) > N.\]
\[\lim_{x\to a^-}f(x)=\infty \equiv \forall N > 0:\, \exists \delta > 0 \text{\ s.t.\ } 0 < a - x < \delta \implies f(x) > N.\]
\[\lim_{x\to a}f(x)=-\infty \equiv \forall N < 0:\, \exists \delta > 0 \text{\ s.t.\ } 0 < |x - a| < \delta \implies f(x) < N.\]
\[\lim_{x\to a^+}f(x)=-\infty \equiv \forall N < 0:\, \exists \delta > 0 \text{\ s.t.\ } 0 < x - a < \delta \implies f(x) < N.\]
\[\lim_{x\to a^-}f(x)=-\infty \equiv \forall N < 0:\, \exists \delta > 0 \text{\ s.t.\ } 0 < a - x < \delta \implies f(x) < N.\]
\[\lim_{x\to\infty}f(x)=\infty \equiv \forall N > 0:\, \exists M > 0 \text{\ s.t.\ } x > M \implies f(x) > N.\]
\[\lim_{x\to\infty}f(x)=-\infty \equiv \forall N > 0:\, \exists M > 0 \text{\ s.t.\ } x > M \implies f(x) < N.\]
\[\lim_{x\to-\infty}f(x)=\infty \equiv \forall N > 0:\, \exists M < 0 \text{\ s.t.\ } x < M \implies f(x) > N.\]
\[\lim_{x\to-\infty}f(x)=-\infty \equiv \forall N > 0:\, \exists M < 0 \text{\ s.t.\ } x < M \implies f(x) < N.\]
Note that if a limit equals to $\infty$ or $-\infty$, the limit doesn't exist.
\sssc{Vertical asymptote (鉛直漸近線)}
\[\qty(\exists a\in\mathbb{R}\colon\abs{\lim_{x \to a^+} f(x)}=\infty\lor\abs{\lim_{x \to a^-} f(x)}=\infty)\iff \qty(x=a\tx{\ is a vertical asymptote of $y=f(x)$}).\]
\sssc{Preservation of equal to at finity}
Let \(I\) be an interval containing the point \(a\). Let $f(x)$ and $g(x)$ be functions defined on \(I\), except possibly at \(a\) itself, which for all $x\in I\land x\neq a$, $f(x)=g(x)$, and that $\lim_{x\to a}f(x)$ exists, then $\lim_{x\to a}g(x)$ exists and that
\[\lim_{x\to a}f(x)=\lim_{x\to a}g(x).\]
\sssc{Preservation of less than or equal to at finity}
Let \(I\) be an interval containing the point \(a\). Let $f(x)$ and $g(x)$ be functions defined on \(I\), except possibly at \(a\) itself, which for all $x\in I\land x\neq a$, $f(x)\leq g(x)$, and that both $\lim_{x\to a}f(x)$ and $\lim_{x\to a}g(x)$ exist, then
\[\lim_{x\to a}f(x)\leq\lim_{x\to a}g(x).\]
\subsubsection{Squeeze theorem or Sandwich theorem at finity}
Let \(I\) be an interval containing the point \(a\). Let $f(x)$, $g(x)$, and $h(x)$ be functions defined on \(I\), except possibly at \(a\) itself, which for all $x\in I\land x\neq a$:
\[f(x)\leq h(x)\leq g(x)\]
and
\[\lim_{x\to a}f(x)=\lim_{x\to a}g(x)=L,\]
then: 
\[\lim_{x\to a}h(x)=L.\]
\sssc{Preservation of equal to at finity of one-sided limit}
Let \(I\) be an interval \((a,b)\) in which $a<b$. Let $f(x)$ and $g(x)$ be functions defined on \(I\), which for all $x\in I$, $f(x)=g(x)$, and that $\lim_{x\to a^+}f(x)$ exists, then $\lim_{x\to a^+}g(x)$ exists and that
\[\lim_{x\to a^+}f(x)=\lim_{x\to a^+}g(x).\]

Let \(I\) be an interval \((b,a)\) in which $a>b$. Let $f(x)$ and $g(x)$ be functions defined on \(I\), which for all $x\in I$, $f(x)=g(x)$, and that $\lim_{x\to a^-}f(x)$ exists, then $\lim_{x\to a^-}g(x)$ exists and that
\[\lim_{x\to a^-}f(x)=\lim_{x\to a^-}g(x).\]
\sssc{Preservation of less than or equal to at finity of one-sided limit}
Let \(I\) be an interval \((a,b)\) in which $a<b$. Let $f(x)$ and $g(x)$ be functions defined on \(I\), which for all $x\in I$, $f(x)\leq g(x)$, and that both $\lim_{x\to a^+}f(x)$ and $\lim_{x\to a^+}g(x)$ exist, then
\[\lim_{x\to a^+}f(x)\leq\lim_{x\to a^+}g(x).\]

Let \(I\) be an interval \((b,a)\) in which $a>b$. Let $f(x)$ and $g(x)$ be functions defined on \(I\), which for all $x\in I$, $f(x)\leq g(x)$, and that both $\lim_{x\to a^-}f(x)$ and $\lim_{x\to a^-}g(x)$ exist, then
\[\lim_{x\to a^-}f(x)\leq\lim_{x\to a^-}g(x).\]
\subsubsection{Squeeze theorem or Sandwich theorem at finity of one-sided limit}
Let \(I\) be an interval \((a,b)\) in which $a<b$. Let $f(x)$, $g(x)$, and $h(x)$ be functions defined on \(I\), which for all $x\in I$:
\[f(x)\leq h(x)\leq g(x)\]
and
\[\lim_{x\to a^+}f(x)=\lim_{x\to a^+}g(x)=L,\]
then: 
\[\lim_{x\to a^+}h(x)=L.\]

Let \(I\) be an interval \((b,a)\) in which $b>a$. Let $f(x)$, $g(x)$, and $h(x)$ be functions defined on \(I\), which for all $x\in I$:
\[f(x)\leq h(x)\leq g(x)\]
and
\[\lim_{x\to a^-}f(x)=\lim_{x\to a^-}g(x)=L,\]
then: 
\[\lim_{x\to a^-}h(x)=L.\]
\sssc{Preservation of less than or equal to at infinity}
Let \(I\) be an interval $(a,\infty)$. Let $f(x)$ and $g(x)$ be functions defined on \(I\), which for all $x\in I$, $f(x)\leq g(x)$, and that both $\lim_{x\to\infty}f(x)$ and $\lim_{x\to\infty}g(x)$ exist, then
\[\lim_{x\to\infty}f(x)\leq\lim_{x\to\infty}g(x).\]

Let \(I\) be an interval $(-\infty,a)$. Let $f(x)$ and $g(x)$ be functions defined on \(I\), which for all $x\in I\land x\neq a$, $f(x)\leq g(x)$, and that both $\lim_{x\to-\infty}f(x)$ and $\lim_{x\to-\infty}g(x)$ exist, then
\[\lim_{x\to-\infty}f(x)\leq\lim_{x\to-\infty}g(x).\]
\subsubsection{Squeeze theorem or Sandwich theorem at infinity}
Let \(I\) be an interval $(a,\infty)$. Let $f(x)$, $g(x)$, and $h(x)$ be functions defined on \(I\), which for all $x\in I$:
\[f(x)\leq h(x)\leq g(x)\]
and
\[\lim_{x\to\infty}f(x)=\lim_{x\to\infty}g(x)=L,\]
then: 
\[\lim_{x\to\infty}h(x)=L.\]

Let \(I\) be an interval $(-\infty,a)$. Let $f(x)$, $g(x)$, and $h(x)$ be functions defined on \(I\), which for all $x\in I$:
\[f(x)\leq h(x)\leq g(x)\]
and
\[\lim_{x\to-\infty}f(x)=\lim_{x\to-\infty}g(x)=L,\]
then: 
\[\lim_{x\to-\infty}h(x)=L.\]
\sssc{Direct substitution property of rational functions}
If $f$ is a rational function and $a$ is in the domain of $f$, then
\[\lim_{x\to a}f(x)=f(a).\]
\sssc{Limits involving quotient functions}
Let \( a \) and \( b \) be real numbers, set
\[A=\left\{f\colon U\subseteq\mathbb{R}\to\mathbb{R} \middle | f(x) = x \lor \ln(f(x)) \in A \lor e^{f\left(x\right)}  \in A \right\},\]
and function $f\in A$. Then:
\[\lim_{x \to \infty} \frac{\left(f\left(x\right)\right)^a}{\left(f\left(x\right)\right)^b} = \infty, \quad a > b \]
\[ \lim_{x \to \infty} \frac{af\left(x\right)}{bf\left(x\right)} = \frac{a}{b}, \quad b \neq 0 \]
\[ \lim_{x \to \infty}\frac{n^{af\left(x\right)}}{bf\left(x\right)} = \infty, \quad a,b > 0 \land  n > 1 \]
\[ \lim_{x \to \infty}\frac{n^{af\left(x\right)}}{bf\left(x\right)} = 0, \quad a,b > 0 \land  0\leq n<1 \]
\[ \lim_{x \to \infty}\frac{af\left(x\right)}{b\log_n f\left(x\right)} = \infty, \quad a,b > 0 \land  n > 1 \]
\ssc{Limit Laws}
Given the limits of the functions involved exist,
\sssc{Sum Law}
The limit of a sum of functions is the sum of the limits of the functions.
\sssc{Difference Law}
The limit of a difference of functions is the difference of the limits of the functions.
\sssc{Constant Multiple Law}
The limit of a constant times a function is the constant times the limit of the function.
\sssc{Product Law}
The limit of a product of functions is the product of the limits of the functions.
\sssc{Quotient Law}
The limit of a quotient of functions is the quotient of the limits of the functions, provided that the limit of the denominator is not 0.
\sssc{Power Law}
The limit of the $n$th power of a function, in which $n$ is a positive integer, is the $n$th power of the limit of the function.
\sssc{Root Law}
The limit of the $n$th root of a function, in which $n$ is a positive integer, is the $n$th root of the limit of the function.
\sssc{The Uniqueness of Limits}
If a limit exists, it is unique.
\ssc{Pointwise Convergence (逐點收斂)}
Let $X$ be a set and $Y$ be topological space. A net of functions $f_n$ all having the same domain $X$ and codomain $Y$ is said to converge pointwise to a given function $f\colon X\to Y$, denoted as 
\[\lim _{n\to \infty }f_{n}=f\tx{\ pointwise},\]
if and only if the limit of the sequence $f_{n}(x)$ evaluated at each point $x\in X$ is equal to $f(x)$, that is
\[\forall x\in X,\lim _{n\to \infty }f_{n}(x)=f(x).\]
The function $f$ is said to be the pointwise limit function of $f_n$.
\subsection{Continuity (連續性)}
\sssc{Definition for real functions}
\begin{itemize}
\item For a point $a$ in the domain of a function $f$, if and only if $\exists\lim_{x\to a}f(x)$ and $\lim_{x\to a}f(x)=f(a)$, we say $f(x)$ is continuous (連續的) at $a$; otherwise, we say $f(x)$ is discontinuous (不連續的) at $a$.
\item For a point $a$ in the domain of a function $f$, if and only if $\exists\lim_{x\to a^+}f(x)$ and $\lim_{x\to a^+}f(x)=f(a)$, we say $f(x)$ is continuous from the right at $a$; if and only if $\exists \lim_{x\to a^-} f(x)$ and $\lim_{x\to a^-} f(x)=f(a)$, we say $f(x)$ is continuous from the left at $a$.
\item For an open interval $I$ that is a subset of the domain of a function $f$, if and only if $f(x)$ is continuous at all points in $I$, we say $f(x)$ is continuous on $I$; otherwise, we say $f(x)$ is discontinuous on $I$.
\item For an right-open interval $[a,b)$ that is a subset of the domain of a function $f$, if and only if $f(x)$ is continuous on $(a,b)$ and continuous from the right at $a$, we say $f(x)$ is continuous on $[a,b)$; otherwise, we say $f(x)$ is discontinuous on $[a,b).
\item For an left-open interval $(a,b]$ that is a subset of the domain of a function $f$, if and only if $f(x)$ is continuous on $(a,b)$ and continuous from the left at $b$, we say $f(x)$ is continuous on $(a,b]$; otherwise, we say $f(x)$ is discontinuous on $(a,b].
\item For an closed interval $[a,b]$ that is a subset of the domain of a function $f$, if and only if $f(x)$ is continuous on both $[a,b)$ and $(a,b]$, we say $f(x)$ is continuous on $[a,b]$; otherwise, we say $f(x)$ is discontinuous on $[a,b].
\item if and only if $f(x)$ is continuous at all points in its domain, we say $f(x)$ is continuous.
\eit
\sssc{Singularity or singular point (奇點) of real functions}
In real analysis, a point is usually called a singularity or a singular point of a real function $f\colon I\subseteq\mathbb{R}\to\mathbb{R}$ if it is a cluster point of $I$ but not in $I$, or it is a discontinuity of $f$.
\sssc{Definition for functions between topological spaces}
\begin{itemize}
\item A function $f\colon I\subeteq X\to Y$ where $X$ and $Y$ are topological spaces is continuous at a point $x\in I$ if and only if for any neighborhood $V$ of $f(x)$ in $Y$, there is a neighborhood $U$ of $x$ such that $f(U)\subseteq V$.
\item A function $f\colon I\subeteq X\to Y$ where $X$ and $Y$ are topological spaces is continuous if and only if for any open subset $V$ of $Y$, the preimage of $f$ on $V$ is an open subset of $X$.
\item A function $f\colon I\subeteq X\to Y$ where $X$ and $Y$ are topological spaces is continuous on a subset $J$ of $I$ if and only if for any open subset $V$ of $Y$, the joint set of the preimage of $f$ on $V$ and $J$ is open in the subspace topology of $X$ in $J$.
\eit
\sssc{Type of discontinuity for real functions}
\bit
\item \tb{Removable discontinuity}: If $\exists\lim_{x\to a}f(x)\land\lim_{x\to a}f(x)\neq f(a)$, we call $f(a)$ removable discontinuity.
\item \tb{Jump discontinuity}: If $\exists\lim_{x\to a^-}f(x)\land\exists\lim_{x\to a^+}f(x)\land\lim_{x\to a^-}f(x)\neq\lim_{x\to a^+}f(x)$, we call $f(a)$ jump discontinuity.
\item \tb{Type I discontinuity}: Removable discontinuity and jump discontinuity are collectively called type I discontinuity.
\item \tb{Essential discontinuity or type II discontinuity}: At least one of $\lim_{x\to a^-}f(x)$ and $\lim_{x\to a^+}f(x)$ doesn't exist in $\mathbb{R}$, we call $f(a)$ essential discontinuity or type II discontinuity.
\item\tb{Infinite discontinuity}: If at least one of $\lim_{x\to a^-}f(x)$ and $\lim_{x\to a^+}f(x)$ does not exist and those in $\lim_{x\to a^-}f(x)$ and $\lim_{x\to a^+}f(x)$ that do not exist are either $\infty$ or $-\infty$, we call $f(a)$ infinite discontinuity.
\item\tb{Essential singularity}: An essential discontinuity that is not an infinite discontinuity is called an essential singularity.
\eit
\sssc{Piecewise continuity for real functions}
For a real function $f$ defined on an interval $I$, if there is a finite set $J$ (i.e. $n(J)\in\mathbb{N}$) of subintervals of $I$ such that $n(I\setminus\bigcup_{j\in J}j)\in\mathbb{N}_0$ and that $f$ is continuous on each $j\in J$, we say $f$ is piecewise-continuous.
\sssc{Piecewise continuity for functions between topological spaces}
For a function $f\colon I\subeteq X\to Y$ where $X$ and $Y$ are topological spaces, if there is a finite set $J$ (i.e. $n(J)\in\mathbb{N}$) of subsets of $I$ that are connected, such that $n(I\setminus\bigcup_{j\in J}j)\in\mathbb{N}_0$ and that $f$ is continuous on each $j\in J$, we say $f$ is piecewise-continuous.
\sssc{Arithmetic Laws}
If $f$ and $g$ are continuous at $a$ and $c$ is a constant, then the following functions are also continuous at $a$:
\[f+g;\quad f-g;\quad cf;\quad fg;\]
\[\frac{f}{g}\text{\ if\ }g(a)\neq 0.\]
\sssc{Composte Laws}
If $g$ is continuous at $a$ and $f$ is continuous at $g(a)$, then the composite function $f\circ g$ is continuous at $a$.
\sssc{Examples of continuous real functions}
The following types of functions are continuous at every number in their domains: algebraic functions, trigonometric functions, inverse trigonometric functions, exponential functions, logarithmic functions.
\sssc{Intermediate Value Theorem (IVT) (中間值定理) for real functions}
Let $f\colon I\subseteq\mathbb{R}\to\mathbb{R}$ be continuous on $J\subseteq$, then for any interval $[a,b]\subseteq J$ such that $f(a)\neq f(b)$,
\[k\in (f(a),f(b)) \implies \qty(\exists c\in (a,b) \text{\ s.t.\ }f(c)=k).\]
\sssc{Intermediate Value Theorem (IVT) for functions between topological spaces}
Let $X$ and $Y$ be topological spaces and function $f\colon I\subseteq X\to Y$ be continuous on a subset $J$ of $I$, then for any connected subset $K$ of $J$, $f(K)$ is a connected subset of $Y$.

Let $X$ and $Y$ be topological spaces and function $f\colon I\subseteq X\to Y$ be continuous on a subset $J$ of $I$, then for any path-connected subset $K$ of $J$, $f(K)$ is a path-connected subset of $Y$.



\section{Differentiation (微分)}
\ssc{Notation}
\sssc{Leibniz's notation (萊布尼茲符號) for differentiation}
Suppose a dependent variable $y$ represents a function $f$ of an independent variable $x$, that is,
\[y=f(x).\]
Then, the derivative of the function $f$ can be written as
\[\frac{\mathrm{d}y}{\mathrm{d}x},\quad\frac{\mathrm{d}}{\mathrm{d}x}y,\quad\frac{\mathrm{d}\qty(f(x))}{\mathrm{d}x},\quad\text{or\ }\frac{\mathrm{d}}{\mathrm{d}x}\qty(f(x)),\]
in which $\frac{\mathrm{d}}{\mathrm{d}x}$ is called a differential operator (微分運算子) or a derivative operator (導數運算子);

the $n$th derivative of the function $f$ can be written as
\[\frac{\mathrm{d}^ny}{\mathrm{d}x^n},\quad\frac{\mathrm{d}^n}{\mathrm{d}x^n}y,\quad\frac{\mathrm{d}^n\qty(f(x))}{\mathrm{d}x^n},\quad\tx{or\ }\frac{\mathrm{d}^n}{\mathrm{d}x^n}\qty(f(x)),\]
in which $\frac{\mathrm{d}^n}{\mathrm{d}x^n}$ is called a differential operator or a derivative operator.
\sssc{Leibniz's notation for partial differentiation}
Suppose a dependent variable $y$ represents a function $f$ of an independent variable vector $\mb{x}=(x_1,x_2,\ldots,x_n)$, that is,
\[y=f(\mb{x}).\]
Then, the partial derivative of the function $f$ with respect to $x_i$ can be written as
\[\frac{\partial y}{\partial x_i},\quad\frac{\partial }{\partial x_i}y,\quad\frac{\partial \qty(f(x))}{\partial x_i},\quad\text{or\ }\frac{\partial }{\partial x_i}\qty(f(x)),\]
in which $\frac{\partial}{\partial x_i}$ is called a partial differential operator (偏微分運算子) or a partial derivative operator (偏導數運算子);

the $n$th partial derivative of the function $f$ with respect to $x_i$ can be written as
\[\frac{\partial^ny}{\partial x_i^{\phantom{i}n}},\quad\frac{\partial^n}{\partial x_i^{\phantom{i}n}}y,\quad\frac{\partial^n\qty(f(x))}{\partial x_i^{\phantom{i}n}}\text{or\ }\frac{\partial^n}{\partial x_i^{\phantom{i}n}}\qty(f(x)),\]
in which $\frac{\partial^n}{\partial x_i^{\phantom{i}n}}$ is called a partial differential operator or a partial derivative operator;

the $n$th mixed partial derivative of the function $f$ $m_{i_1}$ times with respect to $x_{i_1}$, $m_{i_1}$ times with respect to $x_{i_2}$, $\ldots$, $m_{i_k}$ times with respect to $x_{i_k}$, in which $\sum_{j=1}^km_{i_j}=n$, can be written as
\[\frac{\partial^ny}{\partial^{m_{i_1}}x_{i_1}^{\phantom{i_1}m_{i_1}}\partial^{m_{i_2}}x_{i_2}^{\phantom{i_2}m_{i_2}}\ldots\partial^{m_{i_k}}x_{i_k}^{\phantom{i_k}m_{i_k}}},\]
\[\frac{\partial^n}{\partial^{m_{i_1}}x_{i_1}^{\phantom{i_1}m_{i_1}}\partial^{m_{i_2}}x_{i_2}^{\phantom{i_2}m_{i_2}}\ldots\partial^{m_{i_k}}x_{i_k}^{\phantom{i_k}m_{i_k}}}y,\]
\[\frac{\partial^nf\qty(\mb{x})}{\partial^{m_{i_1}}x_{i_1}^{\phantom{i_1}m_{i_1}}\partial^{m_{i_2}}x_{i_2}^{\phantom{i_2}m_{i_2}}\ldots\partial^{m_{i_k}}x_{i_k}^{\phantom{i_k}m_{i_k}}},\quad \tx{or}\]
\[\frac{\partial^n}{\partial^{m_{i_1}}x_{i_1}^{\phantom{i_1}m_{i_1}}\partial^{m_{i_2}}x_{i_2}^{\phantom{i_2}m_{i_2}}\ldots\partial^{m_{i_k}}x_{i_k}^{\phantom{i_k}m_{i_k}}}f(\mb{x}),\]
in which $\frac{\partial^n}{\partial^{m_{i_1}}x_{i_1}^{\phantom{i_1}m_{i_1}}\partial^{m_{i_2}}x_{i_2}^{\phantom{i_2}m_{i_2}}\ldots\partial^{m_{i_k}}x_{i_k}^{\phantom{i_k}m_{i_k}}}$ is called a partial differential operator or a partial derivative operator.
\sssc{Lagrange's notation (拉格朗日符號) or Prime notation for differentiation}
Suppose a dependent variable $y$ represents a function $f$ of an independent variable $x$, that is,
\[y=f(x).\]
Then, the derivative of the function $f$ can be written as
\[y',\quad\tx{or\ }f'(x);\]
the $n$th derivative of the function $f$ can also be written as
\[y^{(n)},\quad\tx{or\ }f^{(n)}(x),\]
in which $^{(n)}$ can be replaced with $n$ primes ($'$) for usually $n<4$ (For example, $y''$ is equivalent to $y^{(2)}$, and $f''$ is equivalent to $f^{(2)}$.).
\sssc{Newton's notation (牛頓符號), dot notation, flyspeck notation, or fluxions for differentiation}
Suppose a dependent variable $y$ represents a function $f$ of an independent variable $t$, that is,
\[y=f(t),\]
where $t$ usually represents time.

Then, the derivative of the function $f$ can be written as
\[\dot{y},\]
the second derivative of the function $f$ can be written as
\[\ddot{y},\]
and so on.
\sssc{Subscript notation for partial differentiation}
Suppose a dependent variable $y$ represents a function $f$ of an independent variable vector $\mb{x}=(x_1,x_2,\ldots,x_n)$, that is,
\[y=f(\mb{x}).\]
Then, the partial derivative of the function $f$ with respect to $x_i$ can be written as
\[y_{x_i},\quad y'_{x_i},\quad f_{x_i},\quad \tx{or\ }f'_{x_i};\]
the $n$th partial derivative of the function $f$ with respect to $x_i$ can be written as below, in which subscript are $n$ $x_i$s
\[y_{x_ix_i\ldots x_i},\quad y^{(n)}_{\pht{(n)}x_ix_i\ldots x_i},\quad f_{x_ix_i\ldots x_i},\tx{or\ }\quad f^{(n)}_{\pht{(n)}x_ix_i\ldots x_i},\]
in which $^{(n)}$ can be replaced with $n$ primes ($'$) for usually $n<4$ (For example, $y''$ is equivalent to $y^{(2)}$, and $f''$ is equivalent to $f^{(2)}$.);

the $n$th mixed partial derivative of the function $f$ $m_{i_1}$ times with respect to $x_{i_1}$, $m_{i_1}$ times with respect to $x_{i_2}$, $\ldots$, $m_{i_k}$ times with respect to $x_{i_k}$, in which $\sum_{j=1}^km_{i_j}=n$, can be written as below, in which subscript are $m_{i_j}$ $x_{i_j}$s for all $x_{i_j}$
\[y_{x_{i_1}x_{i_1}\ldots x_{i_2}x_{i_2}x_{i_2}\ldots x_{i_2}\ldots x_{i_k}x_{i_k}\ldots x_{i_k}},\quad y^{(n)}_{\pht{(n)}x_{i_1}x_{i_1}\ldots x_{i_2}x_{i_2}x_{i_2}\ldots x_{i_2}\ldots x_{i_k}x_{i_k}\ldots x_{i_k}},\]
\[f_{x_{i_1}x_{i_1}\ldots x_{i_2}x_{i_2}x_{i_2}\ldots x_{i_2}\ldots x_{i_k}x_{i_k}\ldots x_{i_k}},\quad \tx{or\ }f^{(n)}_{\pht{(n)}x_{i_1}x_{i_1}\ldots x_{i_2}x_{i_2}x_{i_2}\ldots x_{i_2}\ldots x_{i_k}x_{i_k}\ldots x_{i_k}}.\]
\sssc{Euler’s notation (歐拉符號) for differentiation}
Given a function $f$ of an independent variable $x$, that is,
\[f(x).\]
Then, the derivative of the function $f$ can be written as
\[Df(x)\quad\tx{or\ }(Df)(x)\]
in which $Df$ is called a differential operator or a derivative operator;

the $n$th derivative of the function $f$ can be written as
\[D^nf(x)\quad\tx{or\ }(D^nf)(x)\]
in which $D^nf$ is called a differential operator or a derivative operator.
\sssc{Euler’s notation for partial differentiation}
Given a function $f$ of an independent variable vector $\mb{x}=(x_1,x_2,\ldots,x_n)$, that is,
\[f(\mb{x}).\]
Then, the partial derivative of the function $f$ with respect to $x_i$ can be written as
\[\partial_{x_i}f,\quad \tx{or\ }D_{x_i}f,\]
in which $\partial_{x_i}$ and $D_{x_i}$ are called partial differential operators or partial derivative operators;

the $n$th partial derivative of the function $f$ with respect to $x_i$ can be written as
\[\partial_{x_ix_i\ldots x_i}f,\quad\partial^{(n)}_{\pht{(n)}x_ix_i\ldots x_i}f,\quad D_{x_ix_i\ldots x_i}f,\quad\tx{or\ }D^{(n)}_{\pht{(n)}x_ix_i\ldots x_i}f,\]
in which subscript are $n$ $x_i$s, $^{(n)}$ can be replaced with $n$ primes ($'$) for usually $n<4$ (For example, $\partial''$ is equivalent to $\partial^{(2)}$, and $D''$ is equivalent to $D^{(2)}$.), and $\partial_{x_ix_i\ldots x_i}$, $\partial^{(n)}_{\pht{(n)}x_ix_i\ldots x_i}$, $D_{x_ix_i\ldots x_i}$, and $D^{(n)}_{\pht{(n)}x_ix_i\ldots x_i}$ are called partial differential operators or partial derivative operators;

the $n$th mixed partial derivative of the function $f$ $m_{i_1}$ times with respect to $x_{i_1}$, $m_{i_1}$ times with respect to $x_{i_2}$, $\ldots$, $m_{i_k}$ times with respect to $x_{i_k}$, in which $\sum_{j=1}^km_{i_j}=n$, can be written as
\[\partial_{x_{i_1}x_{i_1}\ldots x_{i_2}x_{i_2}x_{i_2}\ldots x_{i_2}\ldots x_{i_k}x_{i_k}\ldots x_{i_k}}f,\quad\partial^{(n)}_{\pht{(n)}x_{i_1}x_{i_1}\ldots x_{i_2}x_{i_2}x_{i_2}\ldots x_{i_2}\ldots x_{i_k}x_{i_k}\ldots x_{i_k}}f,\]
\[ D_{x_{i_1}x_{i_1}\ldots x_{i_2}x_{i_2}x_{i_2}\ldots x_{i_2}\ldots x_{i_k}x_{i_k}\ldots x_{i_k}}f,\quad \tx{or\ }D^{(n)}_{\pht{(n)}x_{i_1}x_{i_1}\ldots x_{i_2}x_{i_2}x_{i_2}\ldots x_{i_2}\ldots x_{i_k}x_{i_k}\ldots x_{i_k}}f,\]
in which subscript are $m_{i_j}$ $x_{i_j}$s for all $x_{i_j}$, $^{(n)}$ can be replaced with $n$ primes ($'$) for usually $n<4$ (For example, $\partial''$ is equivalent to $\partial^{(2)}$, and $D''$ is equivalent to $D^{(2)}$.), and $\partial_{x_ix_i\ldots x_i}$, $\partial^{(n)}_{\pht{(n)}x_ix_i\ldots x_i}$, $D_{x_ix_i\ldots x_i}$, and $D^{(n)}_{\pht{(n)}x_ix_i\ldots x_i}$ are called partial differential operators or partial derivative operators.
\ssc{Ordinary differentiation (常微分) or differentiation (微分) of Function with Real Domain}
Let $W$ be a topological vector space. The derivative of a function $f\colon U\subeteq\mathbb{R}\to W$ at $x\in U$ is defined as
\[\lim_{h\to 0}\frac{f(x+h)-f(x)}{h}\]
if the limit exists. If such limit exists, we say $f$ is differentiable at $x$.

We define the (first(-order)) derivative function (導函數) of $f$ as a function $Df$ with codomain $W$ such that for any $x\in U$ at which $f$ is differentiable, $Df$ maps $x$ to the derivative of $f$ at $x$.

The derivative of the $k$th(-order) ($k\in\mathbb{N}$) derivative function of $f$ at $x\in U$ is called the $k+1$th(-order) derivative of $f$ at $x\in U$. The derivative function of the $k$th(-order) ($k\in\mathbb{N}$) derivative function of $f$ is called the $k+1$th(-order) derivative function of $f$.

If for any $n\in\mathbb{N}$, the $n$th(-order) derivative of a function $f$ at a point $x$ in its domain exists, we say $f$ is infinitely differentiable at $x$.

If $f$ is differentiable at all point in $I\subeteq U$, we say $f$ is differentiable on $I$; if $f$ is differentiable on $U$, we say $f$ is differentiable. If $f^{(n-1)}$ exists and is differentiable at all point in $I\subeteq U$, we say $f$ is $n$-times differentiable on $I$; if $f^{(n-1)}$ exists and is differentiable on $U$, we say $f$ is $n$-times differentiable.

The operation of finding the derivative or derivative function is called ordinary differentiation (常微分) or differentiation (微分).

Specifically, the $0$th(-order) derivative of $f$ is $f$ itself.
\ssc{Partial differentiation (偏微分) of Function with Real Vector Domain}
Let $W$ be a topological vector space and $f\colon U\subeteq\mathbb{R}^n\to W$ be a function, $\mb{x}$ be the independent variable vector of $f$, and $X$ be the set of all independent variables of $f$.

The (first(-order)) partial derivative (偏導數) $\pdv{f}{x_i}$ of $f$ with respect to $x_i\in X$ at $u\in U$ is defined as
\bma
\pdv{f}{x_i}&=\lim_{h\to 0}\frac{f(u+h\mb{e}_i)-f(u)}{h}\]
&=\frac{\mathrm{d}}{\mathrm{d}h}f(u+h\mb{e}_i)\big\vert_{h=0}
\end{aligned},\]
in which $\mb{e}_i\in\mathbb{R}^n$ is the unit vector in the direction of $x_i$.

The partial derivative of the $k$th(-order) ($k\in\mathbb{N}$) partial derivative function of $f$ with respect to $ x \in X$ with respect to $ x \in X$ at $u\in U$ is called the $k+1$th(-order) partial derivative of $f$ with respect to $ x$ at $u\in U$. The partial derivative function of the $k$th(-order) ($k\in\mathbb{N}$) partial derivative function of $f$ with respect to $ x \in X$ with respect to $ x \in X$ is called the $k+1$th(-order) partial derivative function of $f$ with respect to $ x$.

The partial derivative of the $k$th(-order) partial derivative function of $f$ with respect to $ x_1\in X$ with respect to $ x_2\in X$ at $u\in U$ is called the $(k+1)$th(-order) mixed partial derivative of $f$ with respect to $ x_1, x_1,\ldots, x_1$ ($k$ times) and $ x_2$ at $u\in U$. The partial derivative function of the $k$th(-order) partial derivative function of $f$ with respect to $ x_1\in X$ with respect to $ x_2\in X$ at $u\in U$ is called the $(k+1)$th(-order) mixed partial derivative function of $f$ with respect to $ x_1, x_1,\ldots, x_1$ ($k$ times) and $ x_2$.

The partial derivative of the $k$th(-order) mixed partial derivative function of $f$ with respect to $ x_1, x_2,\ldots, x_k\in X$ with respect to $ x_{k+1}\in X$ at $u\in U$ is called the $(k+1)$th(-order) mixed partial derivative of $f$ with respect to $ x_1, x_2,\ldots, x_k, x_{k+1}$ at $u\in U$. The partial derivative function of the $k$th(-order) mixed partial derivative function of $f$ with respect to $ x_1, x_2,\ldots, x_k\in X$ with respect to $ x_{k+1}\in X$ is called the $(k+1)$th(-order) mixed partial derivative function of $f$ with respect to $ x_1, x_2,\ldots, x_k, x_{k+1}$.

The operation of finding the partial derivative or partial derivative function is called partial differentiation.

Specifically, the $0$th(-order) partial derivative of $f$ is $f$ itself.
\ssc{Fréchet Differentiation (弗蘭歇微分)}
\sssc{Fréchet derivative (弗蘭歇導數), ordinary derivative, or derivative}
Let $V$ and $W$ be normed vector spaces and $U$ be an open subset of $V$. A function $f\colon U\to W$ is Fréchet differentiable (弗蘭歇可微的) or differentiable at $x\in U$ if there exists a bounded linear operator $A\colon V\to W$ such that
\[\lim_{\|h\|_V\to 0}\frac{\|f(x+h)-f(x)-A(h)\|_W}{\|h\|_V}=0.\]
If there exists such an operator $A$, it is unique, so we define the (first(-order)) Fréchet derivative, ordinary derivative, or derivative of $f$ at $x$, denoted as $Df(x)$, as $A$.

We define the (first(-order)) Fréchet derivative function (弗蘭歇導函數) or derivative function of $f$ as a function $Df$ with codomain $B(V,W)$, in which $B(V,W)$ is the space of all bounded linear operators from $V$ to $W$, such that for any $x\in U$ at which $f$ is differentiable, $Df$ maps $x$ to the derivative of $f$ at $x$.

The derivative of the $k$th(-order) ($k\in\mathbb{N}$) derivative function of $f$ at $x\in U$ is called the $k+1$th(-order) derivative of $f$ at $x\in U$. The derivative function of the $k$th(-order) ($k\in\mathbb{N}$) derivative function of $f$ is called the $k+1$th(-order) derivative function of $f$.

If for any $n\in\mathbb{N}$, the $n$th(-order) derivative of a function $f$ at a point $x$ in its domain exists, we say $f$ is infinitely differentiable at $x$.

If $f$ is differentiable at all point in $I\subeteq U$, we say $f$ is differentiable on $I$; if $f$ is differentiable on $U$, we say $f$ is differentiable. If $f^{(n-1)}$ exists and is differentiable at all point in $I\subeteq U$, we say $f$ is $n$-times differentiable on $I$; if $f^{(n-1)}$ exists and is differentiable on $U$, we say $f$ is $n$-times differentiable.

The operation of finding the derivative or derivative function is called Fréchet differentiation (弗蘭歇微分), ordinary differentiation, or differentiation.

Specifically, the $0$th(-order) derivative of $f$ is $f$ itself.
\sssc{Smoothness (光滑性 or 平滑性)}
A function $f$ that has a $k$th derivative that is continuous on its domain is said to be of class $C^k$, denoted as $f\in C^k$, or be a $C^k$-function.

A function $f$ that has a $k$th derivative that is continuous on a subset $I$ of its domain is said to be of class $C^k$ on $I$ or of class $C^k(I)$, denoted as $f\in C^k(I)$.

Generally, the term smooth function refers to a $C^{\infty}$-function. However, it may also mean "sufficiently differentiable" for the problem under consideration.
\ssc{Gateaux Differentiation (加托微分)}
\sssc{Gateaux derivative (加托導數) and partial derivatives}
Let $V$ be a locally convex topological vector spaces (LCTVS), $W$ be a topological vector space, $U$ be an open subset of $V$, and $f\colon U\to W$ be a function.

The (first(-order)) Gateaux derivative $df(u;\,\psi)$ of $f$ at $u\in U$ in the direction $\psi \in V$ is defined to be
\[\begin{aligned}
df(u;\,\psi) &= \lim_{\tau\to 0}\frac{f(u+\tau \psi)-f(u)}{\tau}\\
&= \frac{\mathrm{d}}{\mathrm{d}\tau}f(u+\tau \psi)\big\vert_{\tau =0}
\end{aligned}\]
if the limit exists. If for any $\psi \in V$, the Gateaux derivative exists, then it is said that $f$ is Gateaux differentiable (加托可微的) at $u$.

We define the (first(-order)) Gateaux derivative function (加托導函數) of $f$ in the direction $\psi \in V$, denoted as $df$, as a function $df\colon U\to W$, such that for any $u\in U$ at which $f$ is Gateaux differentiable, $df$ maps $u$ to the Gateaux derivative of $f$ at $u$ in the direction $\psi \in V$.

The Gateaux derivative of the $k$th(-order) ($k\in\mathbb{N}$) Gateaux derivative function of $f$ in the direction $\psi \in V$ in the direction $\psi \in V$ at $u\in U$ is called the $k+1$th(-order) Gateaux derivative of $f$ in the direction $\psi$ at $u\in U$. The Gateaux derivative function of the $k$th(-order) ($k\in\mathbb{N}$) Gateaux derivative function of $f$ in the direction $\psi \in V$ in the direction $\psi \in V$ is called the $k+1$th(-order) Gateaux derivative function of $f$ in the direction $\psi$.

The Gateaux derivative of the $k$th(-order) Gateaux derivative function of $f$ in the direction $\psi_1\in V$ in the direction $\psi_2\in V$ at $u\in U$ is called the $(k+1)$th(-order) mixed Gateaux derivative of $f$ in the direction $\psi_1,\psi_1,\ldots,\psi_1$ ($k$ times) and $\psi_2$ at $u\in U$. The Gateaux derivative function of the $k$th(-order) Gateaux derivative function of $f$ in the direction $\psi_1\in V$ in the direction $\psi_2\in V$ is called the $(k+1)$th(-order) mixed Gateaux derivative function of $f$ in the direction $\psi_1,\psi_1,\ldots,\psi_1$ ($k$ times) and $\psi_2$.

The Gateaux derivative of the $k$th(-order) mixed Gateaux derivative function of $f$ in the direction $\psi_1,\psi_2,\ldots,\psi_k\in V$ in the direction $\psi_{k+1}\in V$ at $u\in U$ is called the $(k+1)$th(-order) mixed Gateaux derivative of $f$ in the direction $\psi_1,\psi_2,\ldots,\psi_k,\psi_{k+1}$ at $u\in U$. The Gateaux derivative function of the $k$th(-order) mixed Gateaux derivative function of $f$ in the direction $\psi_1,\psi_2,\ldots,\psi_k\in V$ in the direction $\psi_{k+1}\in V$ is called the $(k+1)$th(-order) mixed Gateaux derivative function of $f$ in the direction $\psi_1,\psi_2,\ldots,\psi_k,\psi_{k+1}$.

The operation of finding the Gateaux derivative or Gateaux derivative function is called Gateaux differentiation (加托微分).

Specifically, the $0$th(-order) Gateaux derivative of $f$ is $f$ itself.
\sssc{Partial derivative}
Let $V$ be a locally convex topological vector spaces (LCTVS), $W$ be a topological vector space, $U$ be an open subset of $V$, and $f\colon U\to W$ be a function, $\mb{x}$ be the independent variable vector of $f$, and $X$ be the set of all independent variables of $f$.

The (first(-order)) partial derivative of $f$ with respect to $x_i\in X$ at $u\in U$ is defined as the Gateaux derivative of $f$ in the direction of $x_i$ at $u$ if it exists.

The partial derivative of the $k$th(-order) ($k\in\mathbb{N}$) partial derivative function of $f$ with respect to $ x \in X$ with respect to $ x \in X$ at $u\in U$ is called the $k+1$th(-order) partial derivative of $f$ with respect to $ x$ at $u\in U$. The partial derivative function of the $k$th(-order) ($k\in\mathbb{N}$) partial derivative function of $f$ with respect to $ x \in X$ with respect to $ x \in X$ is called the $k+1$th(-order) partial derivative function of $f$ with respect to $ x$.

The partial derivative of the $k$th(-order) partial derivative function of $f$ with respect to $ x_1\in X$ with respect to $ x_2\in X$ at $u\in U$ is called the $(k+1)$th(-order) mixed partial derivative of $f$ with respect to $ x_1, x_1,\ldots, x_1$ ($k$ times) and $ x_2$ at $u\in U$. The partial derivative function of the $k$th(-order) partial derivative function of $f$ with respect to $ x_1\in X$ with respect to $ x_2\in X$ at $u\in U$ is called the $(k+1)$th(-order) mixed partial derivative function of $f$ with respect to $ x_1, x_1,\ldots, x_1$ ($k$ times) and $ x_2$.

The partial derivative of the $k$th(-order) mixed partial derivative function of $f$ with respect to $ x_1, x_2,\ldots, x_k\in X$ with respect to $ x_{k+1}\in X$ at $u\in U$ is called the $(k+1)$th(-order) mixed partial derivative of $f$ with respect to $ x_1, x_2,\ldots, x_k, x_{k+1}$ at $u\in U$. The partial derivative function of the $k$th(-order) mixed partial derivative function of $f$ with respect to $ x_1, x_2,\ldots, x_k\in X$ with respect to $ x_{k+1}\in X$ is called the $(k+1)$th(-order) mixed partial derivative function of $f$ with respect to $ x_1, x_2,\ldots, x_k, x_{k+1}$.

The operation of finding the partial derivative or partial derivative function is called partial differentiation.

Specifically, the $0$th(-order) partial derivative of $f$ is $f$ itself.
\subsection{Taylor series (泰勒級數) or Taylor expansion (泰勒展開)}
Assume that $F:\,\mathbb{R}\to\mathbb{R}$ is an infinitely differentiable function, and its derivatives of every order exist on $\mathbb{R}$, then the Taylor series of $F$ at $a$ is
\[F(x) = \sum_{n\in\mathbb{N}_0} \frac{F^{(n)}(a)}{n!}(x-a)^n,\]
that is,
\[F(x) = \sum^k_{n=0} \frac{F^{(n)}(a)}{n!}(x-a)^n+\int_0^1\frac{(1-t)^k}{k!}F^{(k+1)}(a+t(x-a))(x-a)^{k+1}\,\mathrm{d}t.\]
Also, the $k$th-order approximation of $f$ near $a$ is
\[F(x) \approx \sum^k_{n=0} \frac{F^{(n)}(a)}{n!}(x-a)^n,\]
and the first-order approximation near $a$ is
\[F(x) \approx F(0)+F'(a)(x-a).\]
The Taylor series of $F$ at $0$ is called Maclaurin series (馬克勞林級數) or Maclaurin expansion (馬克勞林展開).
\ssc{Real analyticity (實解析性)}
A real function $f$ is real analytic at a point $x_0$ in its domain, if it is infinitely differentiable at $x_0$ and that the Taylor expansion of $f$ at $x_0$ converges to $f(x)$ pointwise for any $x$ in a neighborhood of $x_0$.

A real function is real analytic on an interval $I$ that is a subset of its domain if it is real analytic at any point in $I$.

A real function is real analytic if it is real analytic at any point in its domain.
\subsection{Critical point (臨界點)}
Let $f$ be a real function and \( c \) be a point in \( D_f \) , if \( f'(c) = 0 \) or \( f' \) does not exist at \( c \), then \( c \) is a critical point of \( f \).
\subsection{Relative extremum (相對極值), local extremum (局部極值), or extremum (極值)}
Let $f$ be a real function and \( c \) be a point in \( D_f \), a relative maximum or maximum \( f(c) \) of \(f\) is said to occur at $c$ if there exists an open interval $I$ where \( c\in I \subseteq D_f \) such that \( \forall x\in I:\, f(c) \geq f(x) \).

Let $f$ be a real function and \( c \) be a point in \( D_f \), a relative minimum or minimum \( f(c) \) of \(f\) is said to occur at $c$ if there exists an open interval $I$ where \( c\in I \subseteq D_f \) such that \( \forall x\in I:\, f(c) \leq f(x) \).

The relative maximum and relative minimum are collectively called the relative extreme.
\subsection{Absolute extremum (絕對極值 or 最值) or global extremum (全域極值)}
Let $f$ be a real function and \( c \) be a point in \( D_f \), an absolute maximum (絕對極大值 or 最大值) \( f(c) \) of \(f\) is said to occur at $c$ if \( \forall x\in D_f:\, f(c) \geq f(x) \).

Let $f$ be a real function and \( c \) be a point in \( D_f \), an absolute minimum (絕對極小值 or 最小值) \( f(c) \) of \(f\) is said to occur at $c$ if \( \forall x\in D_f:\, f(c) \leq f(x) \).

The absolute maximum and absolute minimum are collectively called the absolute extreme.
\subsection{Concavity (凹性)}
Let $f:\,J\subseteq\mathbb{R}\to\mathbb{R}$ be differentiable on the open interval $I\subseteq J$. If $f'$ is strictly increasing on $I$, the graph of $f$ is said to concave upward on $I$; if $f'$ is strictly decreasing on $I$, the graph of $f$ is said to concave downward on $I$; if $f'$ is a constant on $I$, the graph of $f$ is said to be neither upward nor downward (or both upward and downward or undefined in some contexts) on $I$.
\subsection{Point of inflection or inflection point (反曲點 or 拐點)}
Let $f:\,I\subseteq\mathbb{R}\to\mathbb{R}$ be a continuous function and be differentiable on an open interval $J\subseteq I$, and let $a<b<c\land a,b,c\in J$. If the graph of $f$ is concave upward on interval $(a,b)$ and concave downward on interval $(b,c)$, or concave downward on interval $(a,b)$ and concave upward on interval $(b,c)$, then $(b,f(b))$ is called an inflection point of the graph of $f$.



\section{Integration (積分)}
\ssc{Terms}
\bit
\item \tb{Integral (積分)} noun: A definite integral or antiderivative of a function.
\item \tb{Integrate (積分)} verb: Find the integral definite integral or antiderivative of a function.
\item \tb{Integration (積分)} noun: The operation of finding the integral definite integral or antiderivative of a function.
\item \tb{Integrand (被積函數)} noun: The function that is to be integrated.
\item \tb{Domain of integration (積分域)} noun: The domain of integration of the definite integral with $a,b\in\mathbb{R}\cup\{-\infty,\infty}$:
\[\int_a^bf\dd{x}\]
is $(a,b)$.

The domain of integration of the definite integral:
\[\int_{\Omega}f\dd{\omega}\]
is $\Omega$.
\item \tb{Integration variable (積分變數)} noun: The integration variable of the definite integral with $a,b\in\mathbb{R}\cup\{-\infty,\infty}$:
\[\int_a^bf\dd{x}\]
is $x$.

The integration variable of the definite integral:
\[\int_{\Omega}f\dd{\omega}\]
is $\omega$.
\item \tb{Limits (極限)}: The limits of the definite integral with $a,b\in\mathbb{R}\cup\{-\infty,\infty}$:
\[\int_a^bf\dd{x}\]
are $a$ and $b$.

The limits of the definite integral:
\[\int_{\Omega}f\dd{\omega}\]
are the limit points of $\Omega$.
\eit
\subsection{Riemann integral (黎曼積分) and Darboux integral (達布積分)}
Two equivalent definitions of definite integral (定積分).
\subsubsection{Partition of an interval}
A partition $P(x, n)$ of an integral interval $(a,b)$ is a finite sequence of numbers of the form
\[P(x, n):=\{x_i\colon x_0=a^+\land x_n=b^-\land\forall 1\leq i<j\leq n\colon x_i<x_j\}_{i=0}^n,\]
in which when $a^+$ is in an expression, we define the expression as 
\[\lim_{x\to a^+}\qty(\tx{the expression with $a^+$ replaced with $x$});\]
when $b^-$ is in an expression, we define the expression as 
\[\lim_{y\to b^-}\qty(\tx{the expression with $b^+$ replaced with $y$}).\]

Each $[x_i, x_{i+1}]$ is called a sub-interval of the partition, in which $[a^+,x_1]$ is defined as $(a,x_1)$, and $[x_{n-1},b^-]$ is defined as $[x_{n-1},b)$. The mesh or norm of a partition is defined to be the length of the longest sub-interval, that is,
\[\max\left(x_{i+1}-x_{i}\right),\quad i\in [0,n-1].\]
A tagged partition $P(x,n,\xi)$ of an integral interval $(a,b)$ is a partition together with a choice of a sample point within each of all $n$ sub-intervals, that is, numbers $\{\xi_i\}_{i=0}^{n-1}$ with $\xi_i\in [x_i,x_{i+1}]$ for each $i\in [0,n-1]$. The mesh of a tagged partition is the same as that of an ordinary partition.

Suppose that two tagged partitions $P(x,n,\xi)$ and $Q(y,m,\zeta)$ are both partitions of the integral interval $(a,b)$. We say that $Q(y,m,\zeta)$ is a refinement of $P(x,n,\xi)$ if for each integer $i\in [0,n]$, there exists an integer $r(i)\in [0,m]$ such that $x_i = y_{r(i)}$ and that $\forall i\in [0,n-1]:\,\exists j\in [r(i),r(i + 1)] \text{\ s.t.\ }\xi_i = \zeta_j$. That is, a tagged partition breaks up some of the sub-intervals and adds sample points where necessary, "refining" the accuracy of the partition.

We can turn the set of all tagged partitions into a directed set by saying that one tagged partition is greater than or equal to another if the former is a refinement of the latter.
\subsubsection{Riemann sum}
Let $f$ be a real-valued function defined on the integral interval $(a,b)$. The Riemann sum of $f$ with respect to the tagged partition $P(x,n,\xi)$ is defined to be
\[R(f,P):=\sum_{i=0}^{n-1}f(\xi_i)\left(x_{i+1}-x_i\right).\]
Each term in the sum is the product of the value of the function at a given point and the length of an interval. Consequently, each term represents the (signed) area of a rectangle with height $f(\xi_i)$ and width $x_{i + 1} − x_i$. The Riemann sum is the (signed) area of all the rectangles.
\subsubsection{Darboux sum}
Lower and upper Darboux sums of $f$ with respect to the partition $P(x,n)$ are two specific Riemann sums of which the tags are chosen to be the infimum and supremum (respectively) of $f$ on each sub-interval:
\[\begin{aligned}
L(f,P)&:=\sum_{i=0}^{n-1}\inf_{\xi\in [x_i,x_{i+1}]}f(\xi)(x_{i+1}-x_i),\\
U(f,P)&:=\sum_{i=0}^{n-1}\sup_{\xi\in [x_i,x_{i+1}]}f(\xi)(x_{i+1}-x_i).
\end{aligned}\]
\subsubsection{Riemann integral}
The Riemann integral of $f$ exists and equals $s$ if for all $\varepsilon > 0$, there exists $\delta > 0$ such that for any tagged partition $P(x,n,\xi)$ whose mesh is less than $\delta$,
\[\abs{R(f,P)-s}<\varepsilon .\]
\subsubsection{Darboux integral (達布積分)}
The Darboux integral of $f$ exists and equals $s$ if for all $\varepsilon > 0$, there exists $\delta > 0$ such that for any partition $P$ whose mesh is less than $\delta$,
\[\abs{U(f,P)-s}<\varepsilon \land \abs{L(f,P)-s}<\varepsilon.\]
\subsubsection{Integrability}
A function is Riemann-integrable if and only if it is Darboux-integrable.
\subsubsection{Improper integral (瑕積分)}
In the simplest case of a real-valued function of a single variable integrated in the sense of Riemann or Darboux over a single interval, $\int _{a}^{b}f(x)\dd{x}$ is an improper integrals if one or more of below conditions occur: 
\ben
\item$a=-\infty$
\item$b=\infty$
\item$f(x)$ is undefined or discontinuous somewhere on $(a,b)$
\een

The improper integrals are defined with:
\bit
\item $a=-\infty$:
\[\int _{-\infty }^bf(x)\dd{x}=\lim _{a\to -\infty }\int _{a}^{b}f(x)\dd{x}.\]
\item $b=\infty$:
\[\int _{a}^{\infty }f(x)\dd{x}=\lim _{b\to \infty }\int _{a}^{b}f(x)\dd{x}.\]
\item $f(x)$ is undefined or discontinuous at $c$, in which $a<c<b$:
\[\int _{a}^{b}f(x)\dd{x}=\lim_{t\to c^-}\int _{a}^{t}f(x)\dd{x}+\lim_{t\to c^+}\int _{t}^{b}f(x)\dd{x}.\]
\item If any of the terms diverge, the improper integral is undefined.
\eit

$\lim _{a\to -\infty }\int _{a}^{b}f(x)\dd{x}$ converges absolutely to $L$ if $\exists\lim _{a\to -\infty }\int _{a}^{b}\abs{f(x)}\dd{x}\land\lim _{a\to -\infty }\int _{a}^{b}\abs{f(x)}\dd{x}=L$.

$\lim _{b\to \infty }\int _{a}^{b}f(x)\dd{x}$ converges absolutely to $L$ if $\exists\lim _{b\to \infty }\int _{a}^{b}f(x)\dd{x}\dd{x}\land\lim _{b\to \infty }\int _{a}^{b}f(x)\dd{x}=L$.

If an improper integral is convergent but not convergent absolutely, we say it converges conditionally.
\sssc{Cauchy principal value or PV integral}
The Cauchy principal value or PV integral, denoted as p.v., is a weaker notion of convergence, defined by taking symmetric limits. 

The p.v. of $\int _{-\infty}^{\infty}f(x)\dd{x}$, denoted as $\text{p.v.\ }\int_{-\infty}^{\infty} f(x)\dd{x}$, is defined with:
\[\text{p.v.\ }\int_{-\infty}^{\infty} f(x)\dd{x}\coloneq\lim_{R\to\infty} \int_{-R}^{R} f(x)\dd{x},\]
if the limit exists.

The p.v. of $\int _{a}^{b}f(x)\dd{x}$, in which $f(x)$ is undefined or discontinuous at $c$ and $a<c<b$, denoted as $\text{p.v.\ }\int_{a}^{b} f(x)\dd{x}$, is defined with:
\[\text{p.v.\ }\int_{a}^{b} f(x)\dd{x}\coloneq\lim_{\epsilon\to 0}\qty(\int_a^{c-\epsilon}f(x)\dd{x}+\int_{c+\epsilon}^bf(x)\dd{x}),\]
if the limit exists.
\subsection{Lebesgue integral (勒貝格積分)}
A definition of definite integral.

Below, we will define the Lebesgue integral of measurable functions from a measure space $(E,\Sigma,\mu)$ into $\mathbb{R}\cup\{-\infty,\infty\}$.
\subsubsection{Of an Indicator functions}
The integral of an indicator function of a measurable set $S$ is defined to be
\[\int 1_{S}\,\mathrm{d}{\mu} =\mu (S).\]
\paragraph{Simple functions}
Simple functions are finite real linear combinations of indicator functions. A simple function $s$ of the form
\[s:=\sum_ka_k1_{S_k},\]
where the coefficients $a_k$ are real numbers and $S_k$ are disjoint measurable sets, is called a measurable simple function. When the coefficients $a_k$ positive real numbers, $s$ is called a non-negative measurable simple function. The integral of a non-negative measurable simple function $\sum_ka_k1_{S_k}$ is defined to be
\[\int\left(\sum_ka_k1_{S_k}\right)\,\mathrm{d}\mu=\sum_ka_k\,\int 1_{S_k}\,\mathrm{d}\mu=\sum_ka_k\,\mu(S_k).\]
whether this sum is finite or $+\infty$.

If $B$ is a measurable subset of $E$ and $s:=\sum_ka_k1_{S_k}$ is a non-negative measurable simple function, one defines
\[\int_Bs\,\mathrm{d}\mu=\int 1_{B}\,s\,\mathrm {d} \mu =\sum _{k}a_{k}\,\mu (S_{k}\cap B).\]
\subsubsection{Non-negative measurable functions}
Let $f$ be a non-negative measurable function on some measurable subset $B$ of $E$. We define
\[\int_Bf\,\mathrm{d}\mu=\sup\left\{\int_Bs\,\mathrm{d}\mu:\,\forall x\in B:\,0\leq s(x)\leq f(x)\land s\text{ is a measurable simple function}\right\}.\]
\subsubsection{Signed functions}
Let $f$ be a measurable function from a measure set $E$ into $\mathbb{R}\cup\{-\infty,\infty\}$. We define
\[\begin{aligned}
f^{+}(x)&=
\begin{cases}
f(x)\hphantom{-}&\text{if }f(x)>0,\\
0&\text{otherwise},
\end{cases}\\
f^{-}(x)&=
\begin{cases}
-f(x&{\text{if }}f(x)<0,\\
0&\text{otherwise}.
\end{cases}
\end{aligned}\]
Note that both $f^+$ and $f^-$ are non-negative and that
\[f=f^+-f^-,\quad |f|=f^++f^-.\]
We say that the Lebesgue integral of $f$ exists, if
\[ \min \left(\int f^{+}\,\mathrm{d}\mu ,\int f^{-}\,\mathrm{d}\mu \right)<\infty .\]
In this case we define
\[ \int f\,\mathrm{d}\mu =\int f^{+}\,\mathrm{d}\mu -\int f^{-}\,\mathrm{d}\mu.\]
If
\[\int |f|\,\mathrm {d} \mu <\infty ,\]
we say that $f$ is Lebesgue integrable.
\sssc{$L^p(\mu)$ space}
Let $(S,\Sigma,\mu)$ be a measure space. $L^p(\mu)$ or $L^p(S,\Sigma,\mu)$ ($p\in\mathbb{N}$) denotes the space of all measurable functions $f$ on $S$ such that
\[\int_S|f|^p\dd{\mu}<\infty,\]
with norm $\|f\|_{L^p}$ defined as
\[\|f\|_{L^p} = \left(\int_S |f|^p \dd{\mu} \right)^{1/p}.\]
\ssc{Antiderivative (反導函數), inverse derivative, primitive function, primitive integral or indefinite integral (不定積分)}
An antiderivative of a continuous function $f$ is a differentiable function $F$ whose derivative is equal to the original function $f$, that is,
\[F'=f.\]
Suppose $f$ is a function of an independent variable $x$, then its antiderivative $F$ is denoted as
\[F=\int f\,\mathrm{d}x.\]
The process of solving for antiderivatives is called antidifferentiation (反微分) or indefinite integration (不定積分).



\section{Fundamental Theorem of Calculus (FTC) (微積分基本定理)}
\subsection{The First Theorem}
Let $F(x)$ be a differentiable function defined on $[a,b]$. Then:
\[\int_a^bF'(x)\dd{x}=F(b)-F(a).\]
\begin{proof}\mbox{}\\
By the definition of the Riemann integral:
\[
\int_a^b F'(x)\, \dd{x} = \lim_{n \to \infty} \sum_{i=1}^n F'(x_i^*) \Delta x_i,
\]
where \( \{x_i^*\} \) are sample points in the subintervals of a partition \( P = \{x_0, x_1, \dots, x_n\} \) of \([a, b]\), and \( \Delta x_i = x_i - x_{i-1} \).
By the Mean Value Theorem for derivatives, since \( F(x) \) is differentiable, there exists an adequately refined partition \( P = \{x_0, x_1, \dots, x_n\} \) such that on each subinterval \([x_{i-1}, x_i]\) there exists a point \( x_i^* \in [x_{i-1}, x_i] \) such that:
\[
F'(x_i^*) \cdot \Delta x_i = F(x_i) - F(x_{i-1}).
\]
Thus, the Riemann sum becomes:
\[
\sum_{i=1}^n F'(x_i^*) \Delta x_i = \sum_{i=1}^n \left(F(x_i) - F(x_{i-1})\right) = F(b) - F(a).
\]
\end{proof}
\subsection{The Second Theorem}
Let $f(x)$ be a continuous function defined on $[a,b]$. Then:
\[\dv{x}\int_a^xf(t)\dd{t}=f(x).\]
\begin{proof}
\[\dv{x}\qty(\int_a^xf(t)\,\dd{t})=\lim_{h\to 0}\frac{\int_a^{x+h}f(t)\,\dd{t}-\int_a^xf(t)\,\dd{t}}{h}\]
Using the additivity property of integrals:
\[\int_a^{x+h}f(t)\,\dd{t}-\int_a^xf(t)\,\dd{t}=\int_x^{x+h}f(t)\,\dd{t}\]
By the Mean Value Theorem for integrals, since \(f(t)\) is continuous on \([x,x+h]\), there exists a point \(c\in [x,x+h]\) such that:
\[\int_x^{x+h} f(t)\, \dd{t} = f(c) \cdot h.\]
Substituting into the difference quotient:
\[\frac{\int_x^{x+h}f(t)\,\dd{t}}{h} = f(c).\]
\[\lim_{h\to 0}\frac{\int_x^{x+h}f(t)\,\dd{t}}{h} = f(x).\]
\end{proof}



\section{Multivariable (多變數 or 多變量 or 多元) differentiation}
\subsection{Operators}
\begin{itemize}
\item Dot product (點積) operator: $\cdot$
\item Cross product (叉積) operator: $\times$
\item Gradient (梯度) operator: $\nabla$
\item Divergence (散度) operator: $\nabla \cdot$
\item Curl (旋度) operator: $\nabla \times$
\item Directional derivative (方向導數) operator: $\cdot\nabla$
\item Laplace (拉普拉斯) operator: $\nabla^2$或$\Delta$
\item Line or Path integral operator: $\int$
\item Surface integral operator: $\iint$
\item Volume integration operator: $\iiint$
\item Closed line integral operator: $\oint$
\item Closed surface Integral Operator: $\oiint$
\item $\int\mathbf{F}\cdot\mathrm{d}\mathbf{S}$ is used as a shorthand for $\int(\mathbf{F}\cdot\mathbf{\hat{n}})\,\mathrm{d}S$, where $\hat{n}$ is the outward pointing unit normal at almost each point on $S$.
\end{itemize}
\subsection{Convention}
If not otherwise specified:
\begin{itemize}
\item The domain of the funcitons or maps below are subsets of a Euclidean vector space. If not otherwise specified, the coordinates are the Cartesian coordinates, the norms are the Euclidean norms, and the measures are the Lebesgue measures.
\item $\mathbf{0}$ or $0$ refers to the zero tensor (零張量) in the interested Euclidean tensor space $V$, that is, it satisfies 
\[\forall\mathbf{v}\in V:\,\mathbf{v}+\mathbf{0}=\mathbf{v}.\]
\item Unit vector (單位向量): $\mathbf{e}_i$ is the unit vector in the $i$th direction, i.e., a vector with zero norm.
\item Independent variable vector: $\mathbf{x}=(x_1,x_2,\dots,x_n)$
\item Vector fields: $\mathbf{F}(\mathbf{x}) = \sum_{i=1}^n F_i(\mathbf{x}) \mathbf{e}_i$、$\mathbf{G}$
\item Scalar fields: $A(\mathbf{x})$、$B(\mathbf{x})$
\item Tensor fields: $f(\mathbf{x})$、$g(\mathbf{x})$
\item Three-dimensional tensor space field: $\mathbf{T}(\mathbf{x})$
\item The $i$-th component of the map $f$: $f_i$
\end{itemize}
\subsection{Gradient}
\[
\nabla f = \begin{pmatrix}\qty(\pdv{f}{x_1})^T & \qty(\pdv{f}{x_2})^T & \cdots & \qty(\pdv{f}{x_n})^T\end{pmatrix}
\]
The gradient of a scalar field is a vector field, the gradient of a vector field is a second-order tensor (matrix) field, and the gradient of a $k$-order tensor field is a $k+1$-order tensor field. 

In particular, the gradient of a scalar field $A$ is
\[
\nabla A = \sum_{i=1}^n \pdv{A}{x_i}e_i.
\]
And the gradient of a vector field $\mathbf{F}$ is also called the Jacobian matrix (雅可比矩陣) of it and also denoted as $\mathbf{J}(\mathbf{F})$, $J(\mathbf{F})$, or $J_{\mathbf{F}}$, of which the $( i,j )$th entry is
\[\mathbf{J}_{ij}=\frac{\partial F_i}{\partial x_j}.\]
The determinant $\det\left(J_{\mathbf{F}}\right)$ of a Jacobian matrix is called a Jacobian determinant, or Jacobian for short.
\subsection{Divergence}
\[
\nabla \cdot f = \sum_{i=1}^n\frac{\partial f_i}{\partial x_i}
\]
The divergence of a vector field is a scalar field, the gradient of a second-order tensor (i.e. matrix) field is a vector field, and the divergence of a $k+1$-order tensor field is a $k$-order tensor field.
\subsection{Curl}
The curl is only defined on three-dimensional vector field.
\[
\nabla \times \mathbf{T} = 
\begin{pmatrix}
\mathbf{e}_1 & \mathbf{e}_2 & \mathbf{e}_3 \\
\frac{\partial}{\partial x_1} & \frac{\partial}{\partial x_2} & \frac{\partial}{\partial x_3} \\
T_1 & T_2 & T_3 \\
\end{pmatrix}
\]
The curl of a three-dimensional vector field is a three-dimensional vector field.
\subsection{Directional derivative}
\[(\mathbf{f}\cdot\nabla)\mathbf{g}=\sum_{i=1}^n f_i\pdv{g}{x_i}\]
\subsection{Laplace operator}
\[
\nabla^2 f = \nabla \cdot (\nabla f) = \sum_{i=1}^n\frac{\partial^2 f}{\partial x_i^{\phantom{i}2}}
\]
The Laplace operator applied to a tensor field is a tensor field of the same order and same dimension (but not necessarily the same field).
\subsection{Poisson's equation (卜瓦松 or 帕松 or 泊松方程)}
\[
\nabla^2 A = B(\mathbf{x})
\]
\subsection{Laplace's equation (拉普拉斯方程)}
\[
\nabla^2 A = 0
\]
A real function $A$ with real independent variables that is second-order differentiable for all independent variables is called a harmonic function if $A$ satisfies Laplace's equation.
\subsection{Multi-index notation (多重指標記號)}
Suppose there are \( n \) variables \( x_1, x_2, \dots, x_n \), then a multi-index $\alpha$ is a vector of \( n \) non-negative integers: 
\[
\alpha = (\alpha_1, \alpha_2, \dots, \alpha_n), \quad \text{where } \alpha_i \in \mathbb{N}_0.
\]
Define: 
\begin{itemize}
\item Norm \( \|\alpha\| \): 
\[
\|\alpha\| = \alpha_1 + \alpha_2 + \cdots + \alpha_n.
\]
\item Factorial \( \alpha! \): 
\[
\alpha! = \alpha_1! \cdot \alpha_2! \cdot \cdots \cdot \alpha_n!.
\]
\item Power \( \mathbf{x}^\alpha \): 
If \( \mathbf{x} = (x_1, x_2, \dots, x_n) \), then
\[
\mathbf{x}^\alpha = x_1^{\alpha_1} \cdot x_2^{\alpha_2} \cdots x_n^{\alpha_n}.
\]
\item Higher-order mixed partial derivatives $D^\alpha f$: 
\[
D^\alpha f = \frac{\partial^{\|\alpha\|} f}{\partial x_1^{\alpha_1} \partial x_2^{\alpha_2} \cdots \partial x_n^{\alpha_n}}.
\]
\end{itemize}
\subsection{Higher-order derivative}
The $k$th order derivative of $\mathbf{F}:\,\mathbb{R}^n\to\mathbb{R}^m$, denoted as $\mathbf{F}^{(k)}(\mathbf{x})$ or $D^{k}\mathbf{F}(\mathbf{x})$, is a $(\mathbb{R}^n)^k\to\mathbb{R}$ function, where $(\mathbb{R}^n)^k$ is a Cartesian product of $k$ copies of $\mathbb{R}^n$ vector, that is,
\[D^{k}\mathbf{F}(\mathbf{x})=\sum_{\|\alpha\|=k} \left(D^\alpha \mathbf{F}(\mathbf{a})\right).\]
In particular, the first-order derivative of $\mathbf{F}$ is the gradient of it.
\subsection{Taylor expansion}
Assume that $\mathbf{F}:\,\mathbb{R}^n\to\mathbb{R}^m$ is an infinitely differentiable function, and its partial derivatives of every order exist on $\mathbb{R}^n$, then the Taylor expansion of $\mathbf{F}$ at $\mathbf{a}$ is
\[\mathbf{F}(\mathbf{x}) = \sum_{\|\alpha\|\in\mathbb{N}_0} \frac{D^\alpha \mathbf{F}(\mathbf{a})}{\alpha!} (\mathbf{x} - \mathbf{a})^\alpha,\]
that is,
\[\mathbf{F}(\mathbf{x}) = \sum_{\|\alpha\|\leq k} \frac{D^\alpha \mathbf{F}(\mathbf{a})}{\alpha!} (\mathbf{x} - \mathbf{a})^\alpha+\int_0^1\frac{(1-t)^k}{k!} D^{k+1}\mathbf{F}(\mathbf{a} + t(\mathbf{x} - \mathbf{a})) (\mathbf{x} - \mathbf{a})^{k+1} \, \mathrm{d}t.\]
Also, the $k$th-order approximation of $\mathbf{F}$ near $\mathbf{a}$ is
\[\mathbf{F}(\mathbf{x}) \approx \sum_{\|\alpha\|\leq k} \frac{D^\alpha \mathbf{F}(\mathbf{a})}{\alpha!} (\mathbf{x} - \mathbf{a})^\alpha,\]
and the first-order approximation near $\mathbf{a}$ is
\[\mathbf{F}(\mathbf{x}) \approx \mathbf{F}(\mathbf{a}) + \nabla \mathbf{F}(\mathbf{a}) \cdot (\mathbf{x} - \mathbf{a}).\]



\section{Line integral (線積分) or Path integral (路徑積分)}
\subsection{Scalar field line or path integral}
For a scalar field $A : \,U\subseteq \mathbb {R} ^{n}\to \mathbb {R}$ and the path $C \in U$, the line integral of $A$ is: 
\[\int _{C}A\,\mathrm {d} s=\int _{a}^{b}A(\mathbf{r}(t))\|\dv{t}\mathbf {r} (t)\|\,\mathrm {d} t,\]
where $\mathbf{r}:\, [a, b] \to C$ is a one-to-one parametric function with $\mathbf{r}(a)$ and $\mathbf{r}(b)$ being the two endpoints of the path $C$. 

$A$ is called the integral function, $C$ is called the integral path, and the result of the line integration does not depend on the parametric function $r$.
\subsection{Vector field line or path integral}
For a scalar field $\mathbf{F}: \,U\subseteq \mathbb {R} ^{n}\to \mathbb {R}^n$ and the path $C \in U$, the line integral of $\mathbf{F}$ is: 
\[\int _{C}\mathbf {F} (\mathbf {r} )\cdot \,\mathrm {d} \mathbf{r}=\int _{a}^{b}\mathbf {F} (\mathbf {r} (t))\cdot \dv{t}\mathbf {r} (t)\,\mathrm {d} t\]
where $\mathbf{r}:\, [a, b] \to C$ is a one-to-one parametric function with $\mathbf{r}(a)$ and $\mathbf{r}(b)$ being the two endpoints of the path $C$. 

$\mathbf{F}$ is called the integral function, $C$ is called the integral path, and the result of the line integration does not depend on the parametric function $\mathbf{r}$.
\subsection{Conservative field (保守場)}
A field $f$ whose domain is a subset $U$ of a Euclidean tensor space is called a conservative field if for all paths $C$ between point $a$ and $b$, the integral 
\[\int_Cf(\mathbf{x})\cdot\mathrm{d}\mathbf{x}\]
are the same.

This implies 
\begin{itemize}
\item For any closed path $C$, 
\[\int_Cf(\mathbf{x})\cdot\mathrm{d}\mathbf{x}=0.\]
\item If $\operatorname{dim}(U)=3$, then for any subset of $U$ where $f$ is smooth,
\[\nabla\times f=0.\]
\end{itemize}



\section{Fundamental theorem of multivariable calculus (多變數微積分基本定理)}
\subsection{Gradient theorem (梯度定理)}
Suppose $r$ is a oriented differentiable curve that starts at a point $\mathbf{p}$ and ends at a point $\mathbf{q}$. If $\mathbf{F}$ is a differentiable tensor field defined on a neighborhood of $\mathbf{F}$, then,
\[\int_r(\nabla\mathbf{F})\cdot\mathrm{d}\mathbf{r}=\mathbf{F}\left(\mathbf{q}\right)-\mathbf{F}\left(\mathbf{p}\right).\]
Gradient theorem is a special case of generalized Stokes theorem.
\subsection{Divergence theorem, Gauss's theorem, or Ostrogradsky's theorem (高斯散度定理)}
Suppose $V\subseteq\mathbb{R}^n$ is compact and has a piecewise smooth boundary $S$ (also indicated with $\partial V=S$). The closed, measurable set $\partial V$ is oriented by outward-pointing normals. If $F$ is a continuously differentiable vector field defined on a neighborhood of $V$, then,
\[\iiint_V\left(\nabla\cdot\mathbf {F}\right)\,\mathrm{d}V=\oiint_S\mathbf{F}\cdot\mathrm{d}\mathbf{S}\]
Divergence theorem is a special case of generalized Stokes theorem.
\subsection{Stokes' theorem (斯托克斯定理) or Kelvin–Stokes theorem}
Let $S$ be a positively oriented, piecewise smooth surface in $\mathbb{R}^3$ with boundary $\partial S\equiv L$. If a vector field $\mathbf{F}:\,\mathbb{R}^3\rightarrow\mathbb{R}^3$ is defined and has continuous first order partial derivatives in a region containing $S$, then,
\[\iint_S(\nabla\times\mathbf{F})\cdot \mathrm{d}\mathbf{S}=\oint_{L}\mathbf{F}\cdot\mathrm{d}\mathbf{L}\]
Stokes' theorem is a special case of generalized Stokes theorem.
\subsection{Green's theorem (格林定理或綠定理)}
Let $S$ be a positively oriented, piecewise smooth surface in $\mathbb{R}^2$ with boundary $\partial S\equiv L$. If scalar function $P,(x,y)\,Q(x,y)$ are defined and has continuous first order partial derivatives in a region containing $S$, then,
\[\oint_L (P\mathrm{d}x+Q\mathrm{d}y)=\iint_S\left(\frac{\partial Q}{\partial x}-\frac{\partial P}{\partial y}\right)\mathrm{d}x\mathrm{d}y\]
where the path of integration along C is counterclockwise.

Green's theorem is a special case of Stokes' theorem.
\subsection{Generalized Stokes theorem, Stokes–Cartan theorem, or fundamental theorem of multivariable calculus}
The generalized Stokes theorem says that the integral of a differential form $\omega$ over the boundary $\partial\Omega$ of some orientable manifold $\Omega$ is equal to the integral of its exterior derivative $\mathrm{d}\boldsymbol{\omega}$ over the whole of $\Omega$, i.e.,
\[\int _{\partial\Omega}\omega=\int_{\Omega}\mathrm{d}\boldsymbol{\omega}\]



\section{Differential theorems}
\ssc{Distributive over addition and subtraction}
Differentiation is distributive over addition and subtraction.
\ssc{Product rule (乘法定則)}
\[\dv{x}\qty(f(x)g(x))=f'(x)g(x)+f(x)g'(x).\]
\ssc{General Leibniz rule}
If $f$ and $g$ are $n$-times differentiable functions, then the product $fg$ is also n-times differentiable and its $n$-th derivative is given by 
\[(fg)^{(n)}=\sum _{k=0}^{n}{n \choose k}f^{(n-k)}g^{(k)}.\] 
\ssc{Quotient rule (除法定則)}
\[\dv{x}\qty(\frac{f(x)}{g(x)})=\frac{f'(x)g(x)-f(x)g'(x)}{\qty(g(x))^2}.\]
\ssc{Chain rule (連鎖律)}
\[\dv{x}\qty((f\circ g)(x))=f'\qty(g(x))g'(x).\]
\ssc{Faà di Bruno's formula}
\bma
&\frac{\mathrm{d}^n}{\mathrm{d}x}f\left(g(x)\right)\\
=&\sum_{\sum_{i=1}^nim_i=n,\quad m_i\in\mathbb{N}_0}\frac{n!}{\prod_{j=1}^nm_j!}\cdot\\
&f^{\qty(\sum_{j=1}^nm_j)}\left(g(x)\right)\cdot\\
&\prod_{j=1}^n\left(\frac {g^{(j)}(x)}{j!}\right)^{m_j}.
\eam
\subsection{Mean Value Theorem, MVT (均值定理)}
Let $f:\, I\subseteq\mathbb{R}\to\mathbb{R}$ be a continuous function, and $f$ is differentiable on an interval $(a, b)\subseteq I$, then
\[\exists c\in (a, b)\text{\ s.t.\ }f'(c)=\frac{f(b)-f(a)}{b-a}\]
\subsection{Extreme Value Theorem, EVT (極值定理)}
Let function $f:\, I\subseteq\mathbb{R}\to\mathbb{R}$ be continuous on an interval $[a,b]$, then
\[\exists c,d \in [a, b] \text{\ s.t.\ }\qty(\forall x \in [a, b]\colon f(c)\geq f(x)\geq f(d))\]
\subsection{L'Hôpital's rule (羅必達法則) or Bernoulli's rule}
Let \(I\) be an interval containing the point \(a\). Let \( f(x) \) and \( g(x) \) be functions defined on \(I\), except possibly at \(a\) itself. Let \( f(x) \) and \( g(x) \) be differentiable at all points except $a$ in $I$. If $\lim_{x\to a}f(x)=\lim_{x\to a}g(x)\in\{0,\infty,-\infty\}$, and $\exists \lim_{x\to a}\frac{f'(x)}{g'(x)}$, then:
\[\lim_{x\to a}\frac{f(x)}{g(x)}=\lim_{x\to a}\frac{f'(x)}{g'(x)}\]
\subsection{Relative extreme theorem}
A point in $I$ where a relative extreme of $f:\,I\subseteq\mathbb{R}\to\mathbb{R}$ occurs must be a critical point.
\subsection{Concavity theorem}
Let $f:\,J\subseteq\mathbb{R}\to\mathbb{R}$ be differentiable on the open interval $I\subseteq J$. If $\forall x\in I:\,f''>0$, then the graph of $f$ is concave upward on $I$; if $\forall x\in I:\,f''<0$, then the graph of $f$ is concave downward on $I$.
\subsection{Inflection point theorem}
If $(c,f(c))$ is an inflection point of $f$, then $f''(c)=0$ or $f''$ does not exist at $c$.
\subsection{Lagrange multiplier (method) (拉格朗日乘數 or 乘子(法))}
The Lagrange multiplier method is a method for finding the points where extremes occur of a differentiable function under constraints.
\sssc{Univariate form}
Let $f:\,\mathbb{R} \rightarrow \mathbb{R}$ and $g:\,\mathbb{R} \rightarrow \mathbb{R}$ . We want to find the points where extremes of \( f(x) \) under the constraint \( g(x) = \mathbf{0} \) occur. 

First, construct the Lagrangian function \( \mathcal{L}(x,\lambda) \):
\[\mathcal{L}(x,\lambda) = f(x) - \lambda \cdot g(x),\]
where \( \lambda\in\mathbb{R} \) is the Lagrange multiplier (拉格朗日乘數 or 乘子).

Find the derivative of $\mathcal{L}$ and set it to zero:
\[
\dv{\mathcal{L}}{x} = 0
\]
Solve the equation to find \( x \) and \( \lambda \) . 

\text{Statement:} Let $A$ be the set of all solutions for \( x \) such that $\dv{\mathcal{L}}{x} = 0$, and $B$ be the set of all points where extremes of \( f(x) \) under the constraint \( g(x) = \mathbf{0} \) occur. We claim that $B\subseteq A$.
\sssc{Multivariate form}
Let \( \mathbf{x} = (x_1, x_2, \dots, x_n) \) be the independent variable vector and $\mathbf{0}$ be the zero vector. Now we have $f:\,\mathbb{R}^n \rightarrow \mathbb{R}$ and $g:\,\mathbb{R}^n \rightarrow \mathbb{R}^c$. We want to find the points where extremes of \( f(\mathbf{x}) \) under the constraint \( g(\mathbf{x}) = \mathbf{0}\) occur. 

First, construct the Lagrangian function \( \mathcal{L}(\mathbf{x},\lambda) \):
\[
\mathcal{L}(\mathbf{x},\lambda) = f(\mathbf{x}) - \lambda \cdot g(\mathbf{x}),
\]
where \( \lambda\in\mathbb{R}^c \) is the Lagrange multiplier vector.

Find the gradient of $\mathcal{L}$ and set it to zero:
\[
\nabla \mathcal{L} = \mathbf{0}
\]
Solve the equation to find \( \mathbf{x} \) and \( \lambda \) . 

\text{Statement:} Let $A$ be the set of all solutions for \( \mathbf{x} \) such that $\nabla \mathcal{L} = \mathbf{0}$, and $B$ be the set of all points where extremes of \( f(\mathbf{x}) \) under the constraint \( g(\mathbf{x}) = \mathbf{0}\) occur. We claim that $B\subseteq A$.
\begin{proof}\mbox{}\\
Consider $\mathbf{x}^*\in B$. It must satisfy the constraint:
\[g(\mathbf{x}^*) = \mathbf{0}.\]
Any feasible point $\mathbf{x}$ near $\mathbf{x}^*$ must satisfy the constraint. We can represent $\mathbf{x}$ as:
\[\mathbf{x} = \mathbf{x}^* + \delta\mathbf{x},\]
where $\delta\mathbf{x}$ is a differential change tangent to the manifold defined by $g(\mathbf{x})$, that is, it lies in the kernel of $\nabla g(\mathbf{x}^*)$, that is,
\[g(\mathbf{x}^* + \delta\mathbf{x}) = \mathbf{0}.\]
Find the first-order approximation of $f$ at $\mathbf{x}^*$:
\[f(\mathbf{x}^*+ \delta\mathbf{x}) \approx f(\mathbf{x}^*) + \nabla f(\mathbf{x}^*) \cdot \delta\mathbf{x} + O(\|\delta\mathbf{x}\|^2)\]
Find the first-order approximation of $g$ at $\mathbf{x}^*$:
\[g(\mathbf{x}^*+ \delta\mathbf{x}) \approx g(\mathbf{x}^*) + \nabla g(\mathbf{x}^*) \cdot \delta\mathbf{x} + O(\|\delta\mathbf{x}\|^2)\]
Since $\mathbf{x}^*\in B$, for any feasible $\delta\mathbf{x}$ we must have:
\[\nabla f(\mathbf{x}^*) \cdot \delta\mathbf{x} = 0.\]
Since $g(\mathbf{x}^*) = \mathbf{0}$, we have
\[\nabla g(\mathbf{x}^*) \cdot \delta\mathbf{x} = O(\|\delta\mathbf{x}\|^2).\]
Because \( \delta\mathbf{x} \in \ker(\nabla g(\mathbf{x}^)) \), \( \nabla f(\mathbf{x}^) \) can be expressed as a linear combination of \( \nabla g(\mathbf{x}^) \) . This means that there exists a vector to $\lambda$ such that:
\[\nabla\mathcal{L} = \nabla \qty(f(\mathbf{x}) - \lambda \cdot g(\mathbf{x})) = \mathbf{0} \]
\end{proof}
\ssc{Generalized Schwarz's theorem or Clairaut's theorem on equality of mixed partials}
For a function $f\colon\Omega\subseteq\mathbb{R}^n\to\mathbb{R}$, if $\mathbf{p}\in\Omega$ such that $f$ has continuous $m$th partial derivative with respect to $x_h$ for any $h\in\{1,2,\ldots,n\}$ on some neighborhood $N\subseteq\Omega$ of $\mathbf{p}$, then all $r$th mixed partial derivatives \[\frac{\partial^r f(\mathbf{p}}{\partial^{m_{i_1}}x_{i_1}^{\phantom{i_1}m_{i_1}}\partial^{m_{i_2}}x_{i_2}^{\phantom{i_2}m_{i_2}}\ldots\partial^{m_{i_k}}x_{i_k}^{\phantom{i_k}m_{i_k}}}\]
with given constants $t_q$ for any $q\in\{1,2,\ldots,n\}$ in which $0<r<m$, $\sum_{j=1}^km_{i_j}=r$, and
\[\forall q\in\{1,2,\ldots,n\}\colon\sum_{i_j=q}m_{i_j}=t_q,\]
are equivalent.
\ssc{Recurrent helper function}
If a function $y=f(x)$ satisfy that $y'=g\qty(f(x))$ and that $f$ and $g$ are sufficiently smooth such that all needed derivatives exist. We can define functions $F_n(y)$ as
\[\begin{cases}
&F_0(y)=y\\
&F_n(y)=g(y)F_{n-1}'(y).
\end{cases}\]
Then the $n$th derivative of $y=f(x)$ is $F_n(y)$, where $n\in\mathbb{N}.



\section{Integral theorems}
\ssc{Distributive over addition and subtraction}
Integration is distributive over addition and subtraction.
\subsection{Integration by substitution (代換積分法), integration by change of variables (換元積分法), u-substitution (u-代換), reverse chain rule, substitution theroem (代換定理), change of variables theorem (換元定理), or transformation theorem (變換定理)}
\subsubsection{Univariate form}
Let $g:\,I\subseteq\mathbb{R}\to\mathbb{R}$ be injective and differentiable on $[a,b]\subseteq I$, with $g'$ being integrable on $[a,b]$, and $f:\,K\supseteq g([a,b])\to\mathbb{R}$ be integrable on $g([a,b])$. Then:
\[\int_a^bf\circ g(x)\cdot g'(x)\,\mathrm{d}x=\int_{g(a)}^{g(b)}f(u)\,\mathrm{d}u,\]
and for $K=g([a,b])$:
\[\int f\circ g(x)\cdot g'(x)\,\mathrm{d}x=\int f(u)\,\mathrm{d}u.\]
\begin{proof}\mbox{}\\
Consider the interval \([a, b]\) partitioned as
\[P = \{a = x_0 < x_1 < \cdots < x_n = b\}.\]
For each subinterval \([x_{i-1}, x_i]\), let \(\xi_i \in [x_{i-1}, x_i]\) be a sample point. The Riemann sum for the left-hand side of the integral is:
\[S_P = \sum_{i=1}^n f(g(\xi_i)) g'(\xi_i) (x_i - x_{i-1}).\]
Since \(g\) is injective and differentiable, it is either strictly increasing or strictly decreasing on \([a, b]\). Assume \(g\) is strictly increasing (the proof for \(g\) strictly decreasing follows similarly).

Under this assumption, \(g\) maps \([a, b]\) to \([g(a), g(b)]\) (with \(g(a) < g(b)\)). Let \([g(a), g(b)]\) be partitioned as
\[Q = \{g(a) = u_0 < u_1 < \cdots < u_m = g(b)\},\]
where \(u_i = g(x_i)\). For \(g\) increasing, \((u_i - u_{i-1}) = g(x_i) - g(x_{i-1})\).

The Riemann sum for the right-hand side of the integral is:
\[T_Q = \sum_{i=1}^n f(u_i) (u_i - u_{i-1}).\]
Since \(u_i = g(x_i)\) and \(g'(\xi_i) \approx \frac{g(x_i) - g(x_{i-1})}{x_i - x_{i-1}}\), we rewrite:
\[ g'(\xi_i) (x_i - x_{i-1}) \approx g(x_i) - g(x_{i-1}) = u_i - u_{i-1}. \]
Thus, the left-hand Riemann sum \(S_P\) becomes:
\[ S_P = \sum_{i=1}^n f(g(\xi_i)) (u_i - u_{i-1}),\]
which matches the structure of the right-hand Riemann sum \(T_Q\) if we let \(\xi_i = g^{-1}(u_i)\).

As the partition \(P\) of \([a, b]\) gets finer, the Riemann sum \(S_P\) converges to \(\int_a^b f(g(x)) g'(x) \, \mathrm{d}x\). Similarly, as the partition \(Q\) of \([g(a), g(b)]\) gets finer, the Riemann sum \(T_Q\) converges to \(\int_{g(a)}^{g(b)} f(u) \, \mathrm{d}u\).
\end{proof}
\subsubsection{Multivariate form}
Let $\mathbf{T}:\,I\subseteq\mathbb{R}^n\to\mathbb{R}^n$ be injective and differentiable on $D\subseteq I$, with all elements of its gradient (i.e. Jacobian matrix) $\nabla\mathbf{T}$ being continuous on $D$, and $f:\,K\supseteq\mathbf{T}(D)\to\mathbb{R}$ be integrable on $\mathbf{T}(D)$. Then:
\[\int_{\mathbf{T}(D)}f(x_1\,x_2\,\dots\,x_n)\cdot\,\mathrm{d}x_1\,\mathrm{d}x_2\,\dots\,\mathrm{d}x_n=\int_D f(u_1\,u_2\,\dots\,u_n)\abs{\det\left(\nabla\mathbf{T}\right)}\,\mathrm{d}u_1\,\mathrm{d}u_2\,\dots\,\mathrm{d}u_n,\]
and for $K=\mathbf{T}(D)$:
\[\int f(x_1\,x_2\,\dots\,x_n)\cdot\,\mathrm{d}x_1\,\mathrm{d}x_2\,\dots\,\mathrm{d}x_n=\int f(u_1\,u_2\,\dots\,u_n)\abs{\det\left(\nabla\mathbf{T}\right)}\,\mathrm{d}u_1\,\mathrm{d}u_2\,\dots\,\mathrm{d}u_n.\]
\subsubsection{Measure theory form}
Let $X$ be a locally compact Hausdorff space equipped with a finite Radon measure $μ$, and let $Y$ be a \text{\textsigma}-compact Hausdorff space with a \text{\textsigma}-finite Radon measure $\rho$. Let $\phi:\,X\to Y$ be an absolutely continuous function, (which implies that $\mu(E)=0\implies\rho(\phi(E))=0$). Then there exists a real-valued Borel measurable function $w$ on $X$ such that for every Lebesgue integrable function $f:\,Y\to\mathbb{R}$, the function $(f\circ\phi)\cdot w$ is Lebesgue integrable on $X$, and for every open subset $U$ of $X$
\[\int_{\phi(U)}f(y)\,\mathrm{d}\rho(y)=\int_U(f\circ\phi)(x)\cdot w(x)\,\mathrm{d}\mu(x).\]
Furthermore, there exists some Borel measurable function $g$ such that 
\[w(x)=(g\circ\phi)(x).\]
\subsection{Integration by parts (分部積分法) or partial integration (部分積分法)}
\subsubsection{Theorem}
\[\frac{\mathrm{d}}{\mathrm{d}x}\prod_{i=1}^nf_i(x)=\sum_{j=1}^n\left(\frac{\mathrm{d}f_j(x)}{\mathrm{d}x}\frac{\prod_{\substack{i=1\\i\neq j}}^n f_i(x)}{f_j(x)}\right)\]
For example,
\[\int_a^b u\dd{v} = \qty(u v)\big\vert_a^b - \int_a^b v\dd{u}.\]
\subsubsection{Application}
Integration by parts is a heuristic rather than a purely mechanical process for solving integrals; given a single function to integrate, the typical strategy is to carefully separate this single function into a product of two functions such that the residual integral from the integration by parts formula is easier to evaluate than the single function.

The DETAIL rule or the LIATE rule is a rule of thumb for integration by parts. It involves choosing as u the function that comes first in the following list:
\begin{itemize}
\item L: Logarithmic function
\item I: Inverse trigonometric function
\item A: Algebraic function
\item T: Trigonometric function
\item E: Exponential function
\end{itemize}
\ssc{Leibniz integral rule (for differentiation under the integral sign) ((積分符號內取微分的)萊布尼茲積分法則)}
\sssc{Basic form for constant limits}
Let $a,b\in\mathbb{R}$, $f(x,t)$ be a function with domain $\mathbb{R}^2$, and the following integral exists. Then:
\[\begin{aligned}
&\dv{}{x}\qty(\int_a^bf(x,t)\dd{t})\\
=&\int_a^b\pdv{}{x}f(x,t)\dd{t}
\end{aligned}\]
\sssc{Basic form for variable limits}
Let $a(x),b(x)\in\mathbb{R}$, $f(x,t)$ be a function with domain $\mathbb{R}^2$, and the following integral exists. Then:
\[\begin{aligned}
&\dv{}{x}\qty(\int_{a(x)}^{b(x)}f(x,t)\dd{t})\\
=&f\qty(x,b(x))\cdot\dv{b(x)}{x}-f\qty(x,a(x))\cdot\dv{a(x)}{x}+\int_{a(x)}^{b(x)}\pdv{}{x}f(x,t)\dd{t}
\end{aligned}\]
\sssc{Measure theory form for const limits}
Let $X$ be an open subset of $\mathbb{R}$, and $\Omega$ be a measure space. Suppose $f\colon X\times\Omega\to\mathbb{R}$ satisfies the following conditions:
\ben
\item $f(x,\omega)$ is a Lebesgue-integrable function of $\omega$ for each $x\in X$.
\item For almost all $\omega\in\Omega$, the partial derivative $\pdv{f}{x}$ exists for all $x\in X$.
\item There is an integrable function $\theta\colon\Omega\to\mathbb{R}$ such that $\abs{\pdv{f}{x}\qty(x,\omega)}\leq\theta(\omega)$ for all $x\in X$ and almost all $\omegq\in\Omega$.
\een
Then, for all $x\in X$,
\[\dv{}{x}\int_{\Omega}f(x,\omega)\dd{\omega}=\int_{\Omega}\pdv{f}{x}\qty(x,\omega)\dd{\omega}.\]
\ssc{Fubini's theorem (富比尼定理)}
If a function is Lebesgue integrable on $X\times Y$, then:
\[\iint_{X\times Y}f(x,y)\dd{(x,y)}=\int_X\qty(\int_Yf(x,y)\dd{y))\dd{x}=\int_Y\qty(\int_Xf(x,y)\dd{x})\dd{y}.\]
\ssc{(Lebesgue’s) Dominated convergence theorem (DCT) ((勒貝格)控制/受制收斂定理)}
Let $\langle f_n\rangle_{n\in I}$ be a net of real or complex-valued measurable functions on a measure space $(S,\Sigma,\mu)$. If $f_n$ is almost everywhere pointwise convergent to a function $f$, and there is a Lebesgue-integrable function $g$ such that
\[\abs{f_n}\leq g\]
almost everywhere for all $n\in I$.

Then $f_n$ for all $n\in I$ and $f$ are in $L^1(\mu)$ and
\[\lim_n\int_Sf_n\dd{\mu}=\int_S\lim_nf_n\dd{\mu}=\int_Sf\dd{\mu},\]
and
\[\lim_n\int_S\abs{f_n-f}\dd{\mu}=0.\]



\sct{Integral Transform}
\ssc{Laplace Transform (拉普拉斯變換)}
\sssc{(Unilateral or One-sided) Laplace Transform ((單邊)拉普拉斯變換)}
The (unilateral or one-sided) Laplace transform is an integral transform that converts a function of a real variable (usually $t$, in the time domain) to a function of a complex variable (usually $s$, in the complex-valued frequency domain, also known as $s$-domain, or $s$-plane). The functions are often denoted in lowercase for the time-domain representation and uppercase for the frequency-domain.

The Laplace transform of a real function $f(t)$, denoted as $\mathcal{L}\{f(t)\}(s)$ or $F(s)$ is defined by the improper integral
\[\mathcal{L}\{f(t)\}(s) = F(s) = \int_0^{\infty} e^{-st} f(t)\dd{t}.\]

In the context of Laplace transform, convergence refers to absolute convergence, that is, we say $F(s)$ converges or $f(t)$ is Laplace-transformable if the improper integral
\[\int_0^{\infty} \abs{e^{-st} f(t)}\dd{t}\]
converges absolutely for some real $s$.

The set of values for which $F(s)$ converges, called region of convergence (ROC), is either of the form $\Re(s) > a$ or $\Re(s) \geq a$, where $a$ is an extended real constant, i.e. $-\infty\leq a\leq\infty$.
\sssc{Laplace transformability for improper integral definition}
A function $f\colon\mathbb{R}\to\mathbb{R}$ is Laplace-transformable (for improper integral definition of Laplace transform) if and only if it is piecewise continuous and of exponential order.
\begin{proof}
"if" Direction:

Assume that a function $f\colon\mathbb{R}\to\mathbb{R}$ is piecewise continuous and that there exists $M>0$ and $T>0$ such that:
\[|f(t)|\leq Me^{\alpha t}\quad \forall t>T.\]

For $s > \alpha$, we take $t_0\in\mathbb{R}$:
\[\int_0^\infty |f(t)| e^{-st}\dd{t} = \int_0^{t_0} |f(t)| e^{-st}\dd{t} + \int_{t_0}^\infty |f(t)| e^{-st}\dd{t}.\]

The first integral is over a finite interval. Since $f(t)$ is piecewise continuous, it is bounded on $[0,t_0]$, so the first integral is finite.

For the second integral, since $|f(t)| \le M e^{\alpha t}$, we have
\[\int_{t_0}^\infty |f(t)| e^{-st}\dd{t}\le \int_{t_0}^\infty M e^{\alpha t} e^{-st}\dd{t} = M \int_{t_0}^\infty e^{-(s-\alpha)t}\dd{t}.\]

Since $s > \alpha$, the integral converges, so the second integral is finite.

"Only if" Direction:

Assume that a function $f\colon\mathbb{R}\to\mathbb{R}$ is such that there exists some $s_0$ such that
\[\int_0^\infty |f(t)| e^{-s_0 t} \dd{t} < \infty.\]

Piecewise continuity:

Prove by contraction. If $f(t)$ is not piecewise continuous on some interval $[0,T]$, it must have uncountably many discontinuities, which would make the improper integral $\int_0^T |f(t)| e^{-s_0 t}\dd{t}$ diverge.

Exponential order:

For the Laplace transform of $f(t)$ to exist, the integral $\int_0^\infty |f(t)| e^{-s_0 t} \dd{t}$ must converges, so $|f(t)| e^{-s_0 t}$ is bounded-above for sufficiently large $t$, that is, there exists $M > 0$ and $T>0$ such that
\[|f(t)| e^{-s_0 t} \leq M, \quad \forall t \ge T.\]

Multiply both sides by $e^{s_0 t}$:
\[|f(t)| \le M e^{s_0 t}, \quad \forall t \ge T.\]

Hence, $f(t)$ is of exponential order $\alpha = s_0$.
\end{proof}
\sssc{Linearity}
\[\mathcal{L}\{af(t)+bg(t)\}=a\mathcal{L}\{f(t)\}+b\mathcal{L}\{g(t)\}.\]
\sssc{First shifting theorem or Time shift}
\[\mathcal{L}\{f(t-a)u(t-a)\}=e^{-as}F(s), \quad a > 0,\]
where $u(t)$ is the unit step function.
\sssc{Second shifting theorem or Frequency shift}
\[\mathcal{L}\{e^{at} f(t)\}(s) = F(s-a).\]
\sssc{Differentiation in Time Domain}
\[\mathcal{L}\{f'(t)\}(s) = sF(s) - f\qty(0^+).\]
\[\mathcal{L}\{f^{(n)}(t)\}(s) = s^n F(s) - \sum_{i=0}^{n-1}s^{n-1-i}f^{(i)}\qty(0^+).\]
\begin{proof}
\[\mathcal{L}\{f'(t)\}(s) = \int_0^\infty e^{-st} f'(t)\dd{t}.\]
\[u = e^{-st} \quad \Rightarrow \quad \dd{u} = -s e^{-st} \dd{t}.\]
\[dv = f'(t)\dd{t} \quad \Rightarrow \quad v = f(t).\]
Integral by parts.
\[\int_0^\infty u\dd{v} = (u v)\big\vert_0^\infty - \int_0^\infty v\dd{u}.\]
\bma
\int_0^\infty e^{-st} f'(t)\dd{t}&=\qty(e^{-st}f(t))\big\vert_0^\infty+\int_0^\infty f(t)se^{-st}\dd{t}\\
&=-f\qty(0^+)+\int_0^\infty f(t)se^{-st}\dd{t}\\
&=s\mathcal{L}\{f(t)\}(s)-f\qty(0^+)
\end{proof}
\sssc{Integration in Time Domain}
\[\mathcal{L}\left\{\int_0^t f(\tau)\dd{\tau}\right\}(s) = \frac{1}{s} F(s).\]
\begin{proof}
\bma
\mathcal{L}\{\int_0^t f(\tau)\dd{\tau}\}(s) &= \int_0^{\infty} e^{-st} \int_0^t f(\tau)\dd{\tau}\dd{t}\\
&=\qty(-\frac{1}{s}e^{-st}\int_0^t f(\tau)\dd{\tau})\big\vert_0^\infty+\int_0^{\infty} \frac{1}{s}e^{-st}f(t)\dd{t}\\
&=\frac{1}{s}\int_0^{\infty} e^{-st}f(t)\dd{t}.
\eam
\end{proof}
\sssc{Differentiation in Frequency Domain}
\[\mathcal{L}\{t f(t)\}(s) = -\dv{}{s} F(s).\]
\[\mathcal{L}\{t^n f(t)\}(s) = (-1)^n \dv[n]{}{s} F(s).\]
\begin{proof}
By Leibniz integral rule,
\bma
-\dv{}{s} F(s)&=-\dv{}{s}\int_0^{\infty} e^{-st} f(t)\,\mathrm{d}t\\
&=-\int_0^{\infty} \pdv{}{s}\qty(e^{-st} f(t))\,\mathrm{d}t\\
&=-\int_0^{\infty} -te^{-st} f(t)\,\mathrm{d}t\\
&=\int_0^{\infty} te^{-st} f(t)\,\mathrm{d}t\\
&=\mathcal{L}\{t f(t)\}(s)
\eam
\end{proof}
\sssc{Scaling in Time Domain}
\[\mathcal{L}\{f(at)\}(s) = \frac{1}{a} F\left(\frac{s}{a}\right), \quad a>0.\]
\sssc{Convolution Theorem}
If $h(t) = (f * g)(t) = \int_0^t f(\tau)g(t-\tau)\dd{\tau}$, then
\[\mathcal{L}\{h(t)\}(s) = F(s)G(s).\]
\begin{proof}
\[\mathcal{L}\{h(t)\}(s)=\int_0^{\infty} e^{-st} \qty(\int_0^t f(\tau)g(t-\tau)\dd{\tau})\dd{t}\]
By Fubini's theorem,
\[\mathcal{L}\{h(t)\}(s)=\int_0^\infty\int_\tau^\infty e^{-st} f(\tau)g(t-\tau)\dd{t}\dd{\tau}\]
Let $u=t-\tau$. $\dd{t}=\dd{u}$.
\bma
\mathcal{L}\{h(t)\}(s)&=\int_0^\infty f(\tau)\int_0^\infty e^{-s(u+\tau)} g(u)\dd{u}\dd{\tau}\\
&=\int_0^\infty f(\tau)e^{-s\tau}\int_0^\infty e^{-su} g(u)\dd{u}\dd{\tau}\\
&=F(s)G(s)
\eam
\sssc{Initial Value Theorem}
If $f(t)$ and $f'(t)$ are Laplace-transformable:
\[\lim_{t \to 0^+} f(t) = \lim_{s \to \infty} s F(s).\]
\begin{proof}
\[s F(s) = \int_0^\infty s f(t) e^{-st} \dd{t}.\]
Let $u = st$. $\dd{t) = \frac{\dd{u}}{s}$.
\[s F(s) = \int_0^\infty f\qty(\frac{u}{s}) e^{-u} \dd{u}.\]
\[\lim_{s \to \infty}s F(s)=\int_0^\infty f\qty(\frac{u}{s}) e^{-u} \dd{u}.\]

We define a net of functions $\langle f_s\rangle_{s\in\mathbb{R}_{>s_0}}$, where $s_0$ is such that $F(s)$ converges for all $\Re(s)>s_0$.

For fixed $u \in\mathbb{R}_{>0}$, $\lim_{s\to\infty}\frac{u}{s} \to 0^+$, so $f_n\qty(\frac{u}{s})$ pointwise converges to $f(0^+)$.

For dominated convergence theorem, we require an integrable function $g(u)$ such that
\[\abs{f\qty(\frac{u}{s}) e^{-u}} \le g(u), \quad \forall s > 0.\]

Since $f(t)$ is Laplace-transformable, it is of exponential order, that is, there exists $\alpha>0$, $M>0$, and $T>0$ such that:
\[|f(t)|\leq Me^{\alpha t}\quad \forall t>T.\]
\[M e^{\alpha \frac{u}{s}} e^{-u} = M e^{-u\qty(1 - \frac{\alpha}{s})}\le M e^{-\frac{u}{2}}, \quad \forall s > 2\alpha.\]

By dominated convergence theorem, we obtain:
\[\lim_{s \to \infty}s F(s)=\lim_{n\to\infty
\int_0^\infty f(0^+) e^{-u} \dd{u}.\]

Evaluate the integral:
\[\int_0^\infty f(0^+) e^{-u} \dd{u} = f(0^+) \int_0^\infty e^{-u}\dd{u} = f(0^+).\]
\end{proof}
\sssc{Mellin's inverse formula, Bromwich integral, or Fourier–Mellin integral of Inverse Laplace transform (反拉普拉斯變換)}
The inverse Laplace transform of a complex function $F(s)$, denoted as $\mathcal{L}^{-1}\{F(s)\}(t)$ or $f(t)$, is defined as a real function such that
\[\mathcal{L}\{f(t)\}(s) = F(s),\]
where $\mathcal {L}$ denotes the Laplace transform.

Mellin's inverse formula, Bromwich integral, or Fourier–Mellin integral states that, the inverse Laplace transform of a complex function $F(s)$ is given by the line integral:
\[\mathcal{L}^{-1}\{F(s)\}(t)=f(t)=\frac{1}{2\pi i}\lim_{T\to\infty}\int_{\gamma-iT}^{\gamma+iT}e^{st}F(s)\dd{s},\]
where $\gamma$ is a real number such that it is greater than the real part of all singularities of $F$ and that $F$ is bounded on the line $s=\gamma$.
\sssc{(Bilateral or Two-sided) Laplace Transform ((雙邊)拉普拉斯變換)}
The (bilateral or two-sided) Laplace transform is an integral transform that converts a function of a real variable (usually $t$, in the time domain) to a function of a complex variable (usually $s$, in the complex-valued angular frequency domain, also known as $s$-domain or $s$-plane). The functions are often denoted in lowercase for the time-domain representation and uppercase for the frequency-domain.

The Laplace transform of a real function $f(t)$, denoted as $\mathcal{B}\{f(t)\}(s)$, is defined by the improper integral
\[\mathcal{B}\{f(t)\}(s) = \int_{-\infty}^{\infty} e^{-st} f(t)\,\mathrm{d}t.\]
\ssc{Fourier Transform (FT) (傅立葉變換)}
\sssc{Fourier Transform (FT)}
Fourier transform is an integral transform that converts a function of a real variable (usually $t$, in the time domain) to a function of another real variable (usually $\omega$, in the real-valued angular frequency domain). The functions are often denoted in lowercase for the time-domain representation and uppercase for the frequency-domain.

The Fourier transform, denoted as $\mathcal{F}\{f(t)\}(\omega)$ or $F(\omega)$, is defined by the improper integral
\[\mathcal{F}\{f(t)\}(\omega) = F(\omega) = \int_{-\infty}^{\infty} e^{-i\omega t} f(t)\,\mathrm{d}t.\]
\sssc{Inverse Fourier Transform (反傅立葉變換)}
The inverse Fourier transform of a complex function $F(s)$, denoted as $\mathcal{F}^{-1}\{F(s)\}(t)$ or $f(t)$, is defined as a real function such that
\[\mathcal{F}\{f(t)\}(s) = F(s),\]
where $\mathcal {F}$ denotes the Fourier transform.

The inverse Laplace transform of a complex function $F(s)$ is given by the line integral:
\[f(t) = \frac{1}{2\pi}\int_{-\infty}^{\infty} F(\omega) e^{i\omega t}\dd{\omega} .\]



\sct{Lists}
\ssc{List of Integrals}
\sssc{Rational function}
\[\int\frac{f'(x)}{f(x)}\,\mathrm{d}x=\ln|f(x)|+C.\]
\[\int\frac{1}{x^2+a^2}\,\mathrm{d}x=\frac{1}{a}\arctan\frac{x}{a}+C.\]
\[\int\frac{1}{x^2-a^2}\,\mathrm{d}x=\frac{1}{2a}\ln\frac{x-a}{x+a}+C.\]
\[\int\frac{1}{a^2-x^2}\,\mathrm{d}x=\frac{1}{2a}\ln\frac{a+x}{a+x}+C.\]
\[\int\frac{1}{ax+b}\,\mathrm{d}x=\frac{1}{a}\ln|ax+b|+C.\]
\[\int(ax+b)^n\,\mathrm{d}x=\frac{(ax+b)^{n-1}}{a(n+1)}+C,\quad n\neq -1.\]
\[\int\frac{x}{ax+b}\,\mathrm{d}x=\frac{x}{a}-\frac{b}{a^2}\ln|ax+b|+C.\]
\[\int\frac{x}{(ax+b)^2}\,\mathrm{d}x=\frac{b}{a^2(ax+b)}+\frac{1}{a^2}\ln|ax+b|+C.\]
\[\int x(ax+b)^n\,\mathrm{d}x=\frac{a(n+1)x-b}{a^2(n+1)(n+2)}(ax+b)^{n+1}+C,\quad n\notin\{-1,-2\}.\]
\sssc{Irrational function}
\[\int\sqrt{a^2+x^2}\,\mathrm{d}x=\frac{x}{2}\sqrt{a^2+x^2}+\frac{a^2}{2}\ln\qty(x+\sqrt{a^2+x^2})+C.\]
\[\int\sqrt{x^2-a^2}\,\mathrm{d}x=\frac{x}{2}\sqrt{x^2-a^2}-\frac{a^2}{2}\ln\qty(x+\sqrt{x^2-a^2})+C,\quad x^2>a^2.\]
\[\int\sqrt{a^2-x^2}\,\mathrm{d}x=\frac{x}{2}\sqrt{a^2-x^2}+\frac{a^2}{2}\arcsin\frac{x}{|a|}+C,\quad|a|\geq|x|.\]
{{{
\ssc{Lists of Laplace Transform}{{{
\ssc{Lists Fourier Transform}{{{



\section{Numerical differentiation (數值微分)}
\subsection{Newton's Method (牛頓法) or Newton-Raphson Method (牛頓-拉普森法)}
Newton's method, also known as Newton-Raphson method, is an iterative technique used to approximate the roots of a real-valued function. Given a function \( f(x) \) and an initial guess \( x_0 \) close to a root, Newton's method refines this guess by repeatedly applying the formula:
\[
x_{n+1} = x_n - \frac{f(x_n)}{f'(x_n)},n\in\mathbb{N}
\]
where:
\begin{itemize}
\item \( x_n \) is the current approximation,
\item \( f(x_n) \) is the value of the function at \( x_n \),
\item \( f'(x_n) \) is the derivative of \( f(x) \) evaluated at \( x_n \).
\end{itemize}



\section{Numerical integration (數值積分)}
\subsection{The trapezoidal rule (梯形法)}
Let $f$ be a continuous real-valued function on $[a,b]$, the trapezoidal rule gives the approximation
\[\int_a^bf(x)\,\mathrm{d}x\approx\frac{b-a}{2n}\left(2\left(\sum_{i=0}^nf(a+\frac{i}{n}(b-a))\right)-f(a)-f(b)\right).\]
The error is defined as
\[E_n=\int_a^bf(x)\,\mathrm{d}x-\frac{b-a}{2n}\left(2\left(\sum_{i=0}^nf(a+\frac{i}{n}(b-a))\right)-f(a)-f(b)\right).\]
When $\frac{\mathrm{d}^2f(x)}{\mathrm{d}x^2}$ is continuous on $[a,b]$, the error satisfies that
\[\abs{E_n}\leq\frac{(b-a)^3}{12n^2}\max_{a\leq x\leq b}\left(\abs{\frac{\mathrm{d}^2f(x)}{\mathrm{d}x^2}}\right).\]
\subsection{The Simpson's rule (辛普森法) or the Simpson's 1 or 3 rule}
Let $f$ be a continuous real-valued function on $[a,b]$, the Simpson's rule gives the approximation
\[\int_a^bf(x)\,\mathrm{d}x\approx\frac{b-a}{3n}\left(2\left(\sum_{i=0}^nf(a+\frac{i}{n}(b-a))\right)+2\left(\sum_{i=1}^{\frac{n}{2}}f(a+\frac{2i-1}{n}(b-a))\right)-f(a)-f(b)\right),\]
where $\frac{n}{2}\in\mathbb{N}$.

The error is defined as
\[E_n=\int_a^bf(x)\,\mathrm{d}x-\frac{b-a}{3n}\left(2\left(\sum_{i=0}^nf(a+\frac{i}{n}(b-a))\right)+2\left(\sum_{i=1}^{\frac{n}{2}}f(a+\frac{2i-1}{n}(b-a))\right)-f(a)-f(b)\right).\]
When $\frac{\mathrm{d}^4f(x)}{\mathrm{d}x^4}$ is continuous on $[a,b]$, the error satisfies that
\[\abs{E_n}\leq\frac{(b-a)^5}{180n^4}\max_{a\leq x\leq b}\left(\abs{\frac{\mathrm{d}^4f(x)}{\mathrm{d}x^4}}\right).\]
\end{document}
