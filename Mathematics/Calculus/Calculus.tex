\documentclass[a4paper,12pt]{report}
\setcounter{secnumdepth}{5}
\setcounter{tocdepth}{3}
\input{ or usr or share or latex-toolkit or template.tex}
\begin{document}
\title{Calculus}
\author{沈威宇}
\date{\temtoday}
\titletocdoc
\chapter{Calculus (微積分)}


\section{Function (函數)}
\subsection{Definition and notation}
A function is formed by three sets, the domain (定義域) $X$, the codomain (對應域) $Y$, and the graph $R$ that satisfy the three following conditions:
\[R\subseteq \{(x,y)\mid x\in X,y\in Y\}\]
\[\forall x\in X,\exists y\in Y,\left(x,y\right)\in R \]
\[ (x,y)\in R\land (x,z)\in R\implies y=z\]

A function $f$ that satisfies the above is denoted as:
\[f\colon X\to Y.\]
in which the domain $X$ is also denoted as $D_f$.

The range (值域), denoted as $R_f$ or $f(X)$, is defined as:
\[\{ y \mid \exists x \in X, (x, y) \in R \}.\]

If $x\in X$ and $(x, y) \in R$, we write $y = f(x)$, in which $f(x)$ is called the image of $x$ under $f$, $x$ is called the independent variable (自變數 or 獨立變數), and $y$ is called the dependent variable (應變數 or 依賴變數); and the function $f$ is also denoted as:
\[f \colon X \to Y;\, x \mapsto y.\]
\ssc{Type of Functions}
Consider a function $f$:
\[f\colon X\to Y;\,x\mapsto y.\]
\sssc{Injection (單射) or Injective function or One-to-one (一對一) function}
\[\forall a,b\in X\text{\ s.t.\ } f(a)=f(b)\colon a=b\]
\sssc{Many-to-one (多對一) function}
\[\exists a\neq b\in X\colon f(a)=f(b)\]
\sssc{Surjection (滿射 or 蓋射) or Surjective function or Onto function}
\[f(X)=Y\]
\sssc{Bijection (對射) or Bijective function or One-to-one (一對一) function or One-to-one correspondence (一一對應)}
Injective and surjective function.
\ssc{Increasing and Decreasing}
Consider a function $f$:
\[f\colon X\to Y;\,x\mapsto y,\]
such that $Y$ is a totally ordered set.
\sssc{Non-Decreasing (非遞減) or Monotone Increasing (單調遞增) function}
$f$ is non-decreasing on $I\subseteq X$ if and only if
\[\forall a,b\in I\colon a<b\implies f(a)\leq f(b).\]
$f$ is non-decreasing if and only if
\[\forall a,b\in X\colon a<b\implies f(a)\leq f(b).\]
\sssc{Strictly increasing (嚴格遞增) function}
$f$ is strictly increasing on $I\subseteq X$ if and only if
\[\forall a,b\in I\colon a<b\implies f(a)<f(b).\]
$f$ is strictly increasing if and only if
\[\forall a,b\in X\colon a<b\implies f(a)<f(b).\]
\sssc{Non-Increasing (非遞增) or Monotone Decreasing (單調遞減) function}
$f$ is non-increasing on $I\subseteq X$ if and only if
\[\forall a,b\in I\colon a<b\implies f(a)\geq f(b).\]
$f$ is non-increasing if and only if
\[\forall a,b\in X\colon a<b\implies f(a)\geq f(b).\]
\sssc{Strictly decreasing (嚴格遞減) function}
$f$ is strictly decreasing on $I\subseteq X$ if and only if
\[\forall a,b\in I\colon a<b\implies f(a)>f(b).\]
$f$ is strictly decreasing if and only if
\[\forall a,b\in X\colon a<b\implies f(a)>f(b).\]
\sssc{Monotone (單調) function}
$f$ is monotone on $I\subseteq X$ if and only if it is either non-increasing or non-decreasing on $I$.

$f$ is monotone if and only if it is either non-increasing or non-decreasing.
\ssc{Symmetry}
Given a function $f\colon X\to Y$ such that $\forall x\in X\colon -x\in X$.
\sssc{Even Function (偶函數)}
$f$ is an even function if for all $x\in X$, $f(x)=f(-x)$.
\sssc{Odd Function (奇函數)}
$f$ is an odd function if for all $x\in X$, $-f(x)=f(-x)$.
\ssc{Power Function}
\sssc{Power Function}
A power function if a function $f(x)=kx^a$ where $k\in\mathbb{C}$, called coefficient, and $a\in\mathbb{R}$, called exponent, are constants (some requires $k\in\mathbb{R}$ or $k=1$).
\sssc{Root Function}
A root function is a power function of which the exponent is $\frac{1}{n}$ where $n\in\mathbb{N}$.
\sssc{Reciprocal Function}
A reciprocal function is function $f(x)=\frac{1}{x}$.
\ssc{Polynomial (多項式)}
A polynomial is a function $P(x)=\sum_{k=0}^na_kx^k$ where $a_k\in\mathbb{R}$ or $\mathbb{C}$ are constants called coefficients and $n$ is a nonnegative integer called the degree of $P$.
\ssc{Rational Function (有理函數)}
A function $f$ is called an algebraic function if there exists two polynomials $P(x)$ and $Q(x)$ such that $f(x)=\frac{P(x)}{Q(x)}$ for all $x$ in the domain.
\ssc{Algebraic Function (代數函數)}
A function $f$ is called an algebraic function if there exists a polynomial $P(y,x)$ in two variables such that $P(f(x),x)=0$ for all $x$ in the domain.
\ssc{Transformation (變換)}
\sssc{Translation (平移)}
For any function $f\colon\mathbb{R}\to\mathbb{R}$, shifting $y=f(x)$ right by $h$ units and up by $k$ units on the $xy$ coordinate plane yields $y=f(x-h)+k$.
\sssc{Scaling (伸縮 or 縮放 or 拉伸)}
For any function $f\colon\mathbb{R}\to\mathbb{R}$, on the $xy$ coordinate plane, expand $y=f(x)$ vertically by $a$ times the original value with the $x$ axis as the reference line, and expand $y=af\qty(\frac{x}{b})$ horizontally by $b$ times the original value with the $y$ axis as the reference line, to obtain $y=af\qty(\frac{x}{b})$.
\ssc{Function composition (函數合成)}
For two functions $f\colon X\to Y$ and $g\colon V\to W$ such that $g(V)\subseteq X$, the composition of them, denoted as $(f\circ g)$, is defined as:
\[(f \circ g)\colon V\to Y;\,x\mapsto = f(g(x))\]
\ssc{Inverse function (反函數)}
For a bijective function $f\colon X\to Y$, the inverse of it, denoted as $f^{-1}$, is defined as:
\[f^{-1}\colon Y\to X;\,f(x)\mapsto x\]
\ssc{Power of a Function}
For a bijective function $f\colon X\to X$, $f^0$ is defined by:
\[f^0\colon X\to X;\,x\mapsto x,\]
$f^n(x)$ for any $n\in\mathbb{N}$ is defined by:
\[f^n\colon X\to X;\,x\mapsto f\left(f^{n-1}(x)\right),\]
and $f^{-n}(x)$ for any $n\in\mathbb{N}$ is defined by:
\[f^{-n}\colon X\to X;\,x\mapsto f^{-1}\left(f^{-n+1}(x)\right).\]
\ssc{Piecewise function (分段函數)}
A piecewise function is a function defined in the form:
\[f(x) =
\begin{cases}
f_1(x), & \quad x\in A_1, \\
f_2(x), & \quad x\in A_2, \\
\vdots \\
f_n(x), & \quad x \in A_n
\end{cases},\]
where
\[\bigcup_{i=1}^nA_i=D_f\land\forall i\neq j\land i,j\in\mathbb{N}\land i,j\leq n\colon A_i\cap A_j=\varnothing.\]
\subsection{Continuity (連續性)}
\begin{itemize}
\item If $\exists \lim_{x\to a} f(x)$ and $\lim_{x\to a} f(x)=f(a)$, we say $f(x)$ is continuous (連續的) at $x=a$; otherwise, we say $f(x)$ is discontinuous (不連續的) at $x=a$.
\item if $\exists\lim_{x\to a}f(x)\land\exists\lim_{x\to a}f(x)\neq f(a)$, we call $f(a)$ removable discontinuity.
\item if $\exists\lim_{x\to a^-}f(x)\land\exists\lim_{x\to a^+}f(x)\land\lim_{x\to a^-}f(x)\neq\lim_{x\to a^+}f(x)$, we call $f(a)$ jump discontinuity.
\item At least one of $\lim_{x\to a^-}f(x)$ and $\exists\lim_{x\to a^+}f(x)$ doesn't exist in $\mathbb{R}$, we call $f(a)$ essential discontinuity.
\item If $f(x)$ is continuous at all points in a open interval $(a,b)$, we say $f(x)$ is continuous on $(a,b)$.
\item If $f(x)$ is continuous on a open interval $(a,b)$, and $\lim_{x\to a^+}=f(a)$, we say $f(x)$ is continuous on the right-open interval $[a,b)$.
\item If $f(x)$ is continuous on a open interval $(a,b)$, and $\lim_{x\to b^-}=f(b)$, we say $f(x)$ is continuous on the left-open interval $(a,b]$.
\item If $f(x)$ is continuous on a open interval $(a,b)$, and $\lim_{x\to a^+}=f(a)\land\lim_{x\to b^-}=f(b)$, we say $f(x)$ is continuous on the closed interval $[a,b]$.
\item If $f(x)$ is continuous at all points in its domain, we say $f(x)$ is a continuous function (連續函數).
\item If both $f(x)$ and $g(x)$ are continuous at $x=a$ and $g(a)\in D_f$, then $(f\circ g)(x)$ is continuous at $x=a$, namely, $\lim_{x\to a}\qty((f\circ g))(x))=f\qty(\lim_{x\to a} g(x))=(f\circ g)(a)$.
\end{itemize}
\subsection{Indicator function (指示函數或示性函數) or characteristic function (特徵函數)}
An indicator function or a characteristic function of a subset $A$ of a set $X$ is a function that maps elements of the subset to one, and all other elements to zero, often denoted as $1_A$.



\section{Sequences and Series (數列與級數)}
\ssc{Sequence}
\begin{itemize}
\item\textbf{Sequence}: A sequence is a function of which the domain is a set $\{x\in\mathbb{Z}\land l\leq x\leq m\}$, in which $l$ usually equals to $0$ or $1$, usually denoted as $\langle a_n\rangle$, $\{a_n\}$, or $(a_n)$, sometimes denoted with domain as $\langle a_n\rangle_{n=l}^m$, $\{a_n\}_{n=l}^m$, or $(a_n)_{n=l}^m$, where the subscript $n$ refers to the $n$th element of the sequence, that is, the function value when the  independent variable is $n$. Below assume $l=1$.
\item\textbf{Finite Sequence (有限數列)}: A finite sequence is a sequence with finite terms, i.e. $l,m\in\mathbb{Z}$.
\item\textbf{Infinite sequence (無窮數列)}: An infinite sequence is a sequence with infinite terms, sometimes denoted with domain as $\langle a_n\rangle_{n=l}^\infty$. Unless otherwise specified, sequences usually refer to infinite sequences.
\item\textbf{Upper Bound (上界)}: $M$ is an upper bound of sequence $\langle a_n\rangle$ if and only if $\forall n\text{\ s.t.\ }\exists a_n\colon a_n\leq M$. We say $\langle a_n\rangle$ is bounded-above if it has a upper bound.
\item\textbf{Lower Bound (下界)}: $m$ is an lower bound of sequence $\langle a_n\rangle$ if and only if $\forall n\text{\ s.t.\ }\exists a_n\colon a_n\geq m$. We say $\langle a_n\rangle$ is bounded-below if it has a lower bound.
\item\textbf{Supremum or Least Upper Bound (最小上界)}: $M_0$ is the supremum or least upper bound of sequence $\langle a_n\rangle$, denoted as $\sup_n a_n$, if and only if for every upper bound $M$ of \langle a_n\rangle$, $M_0\leq M$.
\item\textbf{Infimum or Greatest Lower Bound (最大下界)}: $m_0$ is the infimum or greatest lower bound of sequence $\langle a_n\rangle$, denoted as $\inf_n a_n$, if and only if for every lower bound $m$ of \langle a_n\rangle$, $m_0\geq m$.
\item\textbf{Bounded}: A sequence is bounded if and only if it is bounded-above and bounded-below.
\end{itemize}
\ssc{Series (級數)}
\begin{itemize}
\item\textbf{Series}: The sum of the terms of a sequence.
\item\textbf{Finite Series (有限級數)}: The sum of the terms of a finite sequence.
\item\textbf{Infinite Series (無窮級數)}: The sum of the terms of an infinite sequence.
\end{itemize}



\section{Limit (極限)}
\subsection{Limit of Sequence}
\subsubsection{Definition}
For a sequence \(\langle a_n\rangle\), the limit of \(\langle a_n\rangle\) as \( n \) approaches \( \infty \) is defined as follows:
\[\lim_{n \to \infty} a_n = L \equiv \forall \epsilon > 0:\, \exists N \in\mathbb{N}\text{\ s.t.\ } n \geq N\implies |a_n - L| < \epsilon.\]
In other words, as \(n\) becomes arbitrarily large, \(a_n\) gets arbitrarily close to \(L\).
\subsubsection{Infinite Limits}
\[\lim_{n\to \infty}a_n=\infty \equiv \forall M > 0, \exists N \in\mathbb{N} \text{\ s.t.\ } n > N \implies a_n > M.\]
\[\lim_{n\to \infty}a_n=-\infty \equiv \forall M < 0, \exists N \in\mathbb{N} \text{\ s.t.\ } n > N \implies a_n < M.\]
\subsubsection{Existence of a Limit}
\[\exists\lim_{n\to \infty}a_n\iff\exists \,\text{finite}\,L\text{\ s.t.\ } \forall \epsilon > 0:\, \exists N \in\mathbb{N}\text{\ s.t.\ } n \geq N\implies |a_n - L| < \epsilon.\]
\subsubsection{Convergence (收斂) and Divergence (發散)}
A sequence \(\langle a_n\rangle\) converges to \(L\) if \(\exists \lim_{n \to \infty} a_n \land \lim_{n \to \infty} a_n = L\). If no such \(L\) exists, we say the sequence does not converge, namely, the sequence diverges.
\subsubsection{Limit Laws}
Given sequences $\langle a_n\rangle$ and $\langle b_n\rangle$ such that:
\[\exists\lim_{n\to\infty}a_n\land\exists\lim_{n\to\infty}b_n\]
and constant $c$, then:
\[ \lim_{x\to a}\langle a_n\rangle+\langle b_n\rangle=\lim_{x\to a}\langle a_n\rangle+\lim_{x\to a}\langle b_n\rangle\]
\[ \lim_{x\to a}\langle a_n\rangle-\langle b_n\rangle=\lim_{x\to a}\langle a_n\rangle-\lim_{x\to a}\langle b_n\rangle\]
\[\lim_{x\to a}c\langle a_n\rangle=c\lim_{x\to a}\langle a_n\rangle\]
\[ \lim_{x\to a}\langle a_n\rangle\langle b_n\rangle=\lim_{x\to a}\langle a_n\rangle\lim_{x\to a}\langle b_n\rangle\]
\[\langle b_n\rangle\neq 0\land\lim_{x\to a}\langle b_n\rangle\neq 0:\,\lim_{x\to a}\frac{\langle a_n\rangle}{\langle b_n\rangle}=\frac{\lim_{x\to a}\langle a_n\rangle}{\lim_{x\to a}\langle b_n\rangle}\]
\subsubsection{Squeeze (夾擠) or Sandwich (三明治) theorem}
Given sequences $\langle a_n\rangle$, $\langle b_n\rangle$, and $\langle c_n\rangle$ which:
\[a_n\leq c_n\leq b_n\]
and
\[\lim_{n\to\infty}a_n=\lim_{n\to\infty}b_n=L,\]
then: 
\[\lim_{n\to\infty}c_n=L.\]
\subsubsection{Monotone Convergence Theorem (單調收斂定理) or Completeness of the Real Number (實數的完備性)}
\textit{Statement.}
\begin{enumerate}[label=(\Alph*)]
\item For a non-decreasing and bounded-above sequence of real numbers $\langle a_n\rangle_{n\in\mathbb {N}}$:
\[\lim_{n\to\infty}a_n=\sup_n a_n\]
\item For a non-increasing and bounded-below sequence of real numbers $\langle a_n\rangle_{n\in\mathbb {N}}$:
\[\lim_{n\to\infty}a_n=\inf_n a_n\]
\end{enumerate}
\begin{proof}\mbox{}\\
Let $\{a_{n}\}_{n\in\mathbb {N}}$ be the set of values of $\langle a_n\rangle$. By assumption, $\{a_n\}$ is non-empty and bounded-above by $\sup_n a_n$. Let $c=\sup_n a_n$.
\[\forall\epsilon>0:\,\exists N\text{\ s.t.\ }c\geq a_N>c-\epsilon,\]
since otherwise $c-\epsilon$ is a strictly smaller upper bound of $\langle a_n\rangle$, contradicting the definition of the supremum. 

Then since $\langle a_n\rangle$ is non decreasing, and $c$ is an upper bound:
\[\forall\epsilon>0:\,\exists N\text{\ s.t.\ }\forall n>N:\,|c-a_n|=c-a_n\leq c-a_N=|c-a_N|<\epsilon.\]
The proof of the (B) part is analogous or follows from (A) by considering $\langle -a_{n}\rangle_{n\in \mathbb{N}}$.
\end{proof}
\textit{Statement.}

If $\langle a_n\rangle_{n\in\mathbb {N}}$ is a monotone sequence of real numbers, i.e., if 
$a_n\leq a_{n+1}$ for every $n\geq 1$ or $a_n\geq a_{n+1}$ for every $n\geq 1$, then this sequence has a finite limit if and only if the sequence is bounded.
\begin{proof}
"If"-direction: The proof follows directly from the proposition.

"Only If"-direction: By $(\epsilon,\delta)$-definition of limit, every sequence $\langle a_n\rangle_{n\in\mathbb {N}}$ with a finite limit $L$ is necessarily bounded.
\end{proof}
\subsection{A Limit for a Series (級數)}
\subsubsection{Definition}
Let:
\[S_n = \sum_{i=1}^n a_i,\]
where \(a_i\) are terms of a sequence. The limit of \(S_n\), denoted as \(\lim_{n\to\infty}S_n\) or \(\sum_{i=1}^{\infty}a_i\), is defined as the following:
\[\sum_{i=1}^{\infty}a_i = L \equiv \forall \epsilon > 0:\, \exists N \in\mathbb{N}\text{\ s.t.\ } n \geq N\implies |S_n - L| < \epsilon.\]
\subsubsection{Convergence and Divergence}
A series \(S_n\) converges to \(L\) if \(\exists \lim_{n \to \infty} S_n \land \lim_{n \to \infty} S_n = L\). If no such \(L\) exists, we say the series does not converge, namely, the series diverges.

A series $S_n=\sum_{i=1}^{\infty}a_i$ converges absolutely to $L$ if $\exists\lim_{n\to\infty}\sum_{i=1}^{\infty}\abs{a_1}\land\lim_{n\to\infty}\sum_{i=1}^{\infty}\abs{a_1}=L$. If $S_n$ is convergent but not convergent absolutely, we say $S_n$ converges conditionally.
\subsubsection{Limit Laws}
Given series $\sum_{i=1}^na_i$ and $\sum_{i=1}^nb_i$ such that:
\[\exists\lim_{n\to\infty}\sum_{i=1}^na_i\land\exists\lim_{n\to\infty}\sum_{i=1}^nb_i\]
and constant $c$, then:
\[ \sum_{i=1}^{\infty}a_i+\langle b_n\rangle=\sum_{i=1}^{\infty}a_i+\sum_{i=1}^{\infty}b_i\]
\[ \sum_{i=1}^{\infty}a_i-\langle b_n\rangle=\sum_{i=1}^{\infty}a_i-\sum_{i=1}^{\infty}b_i\]
\[\sum_{i=1}^{\infty}ca_i=c\sum_{i=1}^{\infty}a_i\]
\subsection{A Limit for a Function}
\subsubsection{Limit at Finity}
Let \(I\) be an interval containing the point \(a\). Let \( f(x) \) be a function defined on \(I\), except possibly at \(a\) itself. The limit of \( f(x) \) as \( x \) approaches \( a \) is defined as follows:
\[\lim_{x \to a} f(x) = L \equiv \forall \epsilon > 0:\, \exists \delta > 0 \text{\ s.t.\ } 0 < |x - a| < \delta \implies |f(x) - L| < \epsilon.\]
In other words, as \(x\) becomes arbitrarily close to \(a\), \(f(x)\) gets arbitrarily close to \(L\).
\subsubsection{Limit at Infinity}
\begin{itemize}
\item Let \(I\) be a left-bounded, right-unbounded interval with the point \(a\) being its endpoint on the left. Let \( f(x) \) be a function defined on \(I\). The limit of \( f(x) \) as \( x \) approaches \( \infty \) is defined as follows:
\[\lim_{x \to \infty} f(x) = L \equiv \forall \epsilon > 0: \, \exists M > a \text{\ s.t.\ } x > M \implies |f(x) - L| < \epsilon.\]
In other words, as \(x\) becomes arbitrarily big, \(f(x)\) gets arbitrarily close to \(L\). We say $f(x)$ converge to $L$ as $x\to\infty$ if $\lim_{x \to \infty} f(x) = L$.
\item Let \(I\) be a right-bounded, left-unbounded interval with the point \(a\) being its endpoint on the right. Let \( f(x) \) be a function defined on \(I\). The limit of \( f(x) \) as \( x \) approaches \( -\infty \) is defined as follows:
\[\lim_{x \to -\infty} f(x) = L \equiv \forall \epsilon > 0: \, \exists M < a \text{\ s.t.\ } x < M \implies |f(x) - L| < \epsilon.\]
In other words, as \(x\) becomes arbitrarily small, \(f(x)\) gets arbitrarily close to \(L\). We say $f(x)$ converge to $L$ as $x\to-\infty$ if $\lim_{x \to -\infty} f(x) = L$.
\end{itemize}
\subsubsection{Horizontal asymptote (水平漸近線)}
\[ \qty(\lim_{x \to \infty} f(x)=L \lor\lim_{x \to -\infty} f(x)=L) \iff \qty(y=L\tx{ is a horizontal asymptote of $y=f(x)$}).\]
\subsubsection{Slant asymptote (斜漸近線)}
\[ \qty(\lim_{x \to \infty} f(x)-(mx+b)=0 \lor\lim_{x \to -\infty} f(x)-(mx+b)=0) \iff \qty(y=mx+b\tx{\ is a slant asymptote of $y=f(x)$}).\]
\subsubsection{One-side Limits}
\begin{itemize}
\item \tb{Right-hand Limit (右極限)}: Let \(I\) be a left-open interval with the point \(a\) being its endpoint on the left. Let \( f(x) \) be a function defined on \(I\). The right-hand limit of \( f(x) \) as \( x \) approaches \( a \) is defined as follows:
\[\lim_{x \to a^+} f(x) = L \equiv \forall \epsilon > 0 :\,\exists \delta > 0 \text{\ s.t.\ } 0 < x - a < \delta \implies |f(x) - L| < \epsilon.\]
In other words, as \(x\) becomes arbitrarily close to \(a\) and is greater than \(a\), \(f(x)\) gets arbitrarily close to \(L\).
\item \tb{Left-hand Limit (左極限)}: Let \(I\) be a right-open interval with the point \(a\) being its endpoint on the right. Let \( f(x) \) be a function defined on \(I\). The left-hand limit of \( f(x) \) as \( x \) approaches \( a \) is defined as follows:
\[\lim_{x \to a^-} f(x) = L \equiv \forall \epsilon > 0 :\,\exists \delta > 0 \text{\ s.t.\ } 0 < a - x < \delta \implies |f(x) - L| < \epsilon.\]
In other words, as \(x\) becomes arbitrarily close to \(a\) and is less than \(a\), \(f(x)\) gets arbitrarily close to \(L\).
\end{itemize}
\subsubsection{Infinite Limits}
\[\lim_{x\to a}f(x)=\infty \equiv \forall M > 0, \exists \delta > 0 \text{\ s.t.\ } 0 < |x - a| < \delta \implies f(x) > M.\]
\[\lim_{x\to a}f(x)=-\infty \equiv \forall M < 0, \exists \delta > 0 \text{\ s.t.\ } 0 < |x - a| < \delta \implies f(x) < M.\]
\[\lim_{x\to\infty}f(x)=\infty \equiv \forall M > 0, \exists \delta > 0 \text{\ s.t.\ } x > \delta \implies f(x) > M.\]
\[\lim_{x\to\infty}f(x)=-\infty \equiv \forall M > 0, \exists \delta > 0 \text{\ s.t.\ } x > \delta \implies f(x) < M.\]
\[\lim_{x\to-\infty}f(x)=\infty \equiv \forall M > 0, \exists \delta < 0 \text{\ s.t.\ } x < \delta \implies f(x) > M.\]
\[\lim_{x\to-\infty}f(x)=-\infty \equiv \forall M > 0, \exists \delta < 0 \text{\ s.t.\ } x < \delta \implies f(x) < M.\]
\sssc{Vertical asymptote (鉛直漸近線)}
\[\qty(\exists a\in\mathbb{R}\colon\abs{\lim_{x \to a^+} f(x)}=\infty\lor\abs{\lim_{x \to a^-} f(x)}=\infty)\iff \qty(x=a\tx{\ is a vertical asymptote of $y=f(x)$}).\]
\subsubsection{Existence of Limits}
\[\exists\lim_{x \to a^+} f(x) \iff \exists \,\text{finite}\,L\text{\ s.t.\ }\forall \epsilon > 0:\, \exists \delta > 0 \text{\ s.t.\ } 0 < x - a < \delta \implies |f(x) - L| < \epsilon.\]
\[\exists\lim_{x \to a^-} f(x) \iff \exists \,\text{finite}\,L\text{\ s.t.\ }\forall \epsilon > 0 :\,\exists \delta > 0 \text{\ s.t.\ } 0 < a - x < \delta \implies |f(x) - L| < \epsilon.\]
\[\exists\lim_{x \to a} f(x) \iff \exists \lim_{x \to a^+} f(x)\land\exists \lim_{x \to a^-} f(x)\land\lim_{x \to a^+} f(x) = \lim_{x \to a^-} f(x).\]
\subsubsection{Limit Laws}
Let \(I\) be an interval containing a point \(a\). Given functions $f(x)$ and $g(x)$ defined on \(I\), except possibly at \(a\) itself, such that:
\[\exists\lim_{x\to a}f(x)\land\exists\lim_{x\to a}g(x),\]
and constant $c$, then:
\[ \lim_{x\to a}f(x)+g(x)=\lim_{x\to a}f(x)+\lim_{x\to a}g(x)\]
\[ \lim_{x\to a}f(x)-g(x)=\lim_{x\to a}f(x)-\lim_{x\to a}g(x)\]
\[\lim_{x\to a}cf(x)=c\lim_{x\to a}f(x)\]
\[ \lim_{x\to a}f(x)g(x)=\lim_{x\to a}f(x)\lim_{x\to a}g(x)\]
\[\lim_{x\to a}\frac{f(x)}{g(x)}=\frac{\lim_{x\to a}f(x)}{\lim_{x\to a}g(x)},\quad g(x)\neq 0\land\lim_{x\to a}g(x)\neq 0\]
\subsubsection{Squeeze or Sandwich theorem}
Let \(I\) be an interval containing the point \(a\). Let $f(x)$, $g(x)$, and $h(x)$ be functions defined on \(I\), except possibly at \(a\) itself which:
\[f(x)\leq h(x)\leq g(x)\]
and
\[\lim_{x\to a}f(x)=\lim_{x\to a}g(x)=L,\]
then: 
\[\lim_{x\to a}h(x)=L.\]



\section{Differentiation (微分)}
\ssc{Leibniz's notation (萊布尼茲符號) for differentiation}
Suppose a dependent variable $y$ represents a function $f$ of an independent variable $x$, that is,
\[y=f(x).\]
Then the derivative of the function $f$ can be written as
\[\frac{\mathrm{d}y}{\mathrm{d}x}\text{\ or\ }\frac{\mathrm{d}}{\mathrm{d}x}y\text{\ or\ }\frac{\mathrm{d}\qty(f(x))}{\mathrm{d}x}\text{\ or\ }\frac{\mathrm{d}}{\mathrm{d}x}\qty(f(x)).\]
The $n$th derivative of the function $f$ can be written as
\[\frac{\mathrm{d}^ny}{\mathrm{d}x^n}\text{\ or\ }\frac{\mathrm{d}^n}{\mathrm{d}x^n}y\text{\ or\ }\frac{\mathrm{d}^n\qty(f(x))}{\mathrm{d}x^n}\text{\ or\ }\frac{\mathrm{d}^n}{\mathrm{d}x^n}\qty(f(x)).\]
\ssc{Lagrange's notation (拉格朗日符號) for differentiation}
Suppose a dependent variable $y$ represents a function $f$ of an independent variable $x$, that is,
\[y=f(x).\]
Then the derivative of the function $f$ can be written as
\[f'(x),\]
the second derivative of the function $f$ can be written as
\[f''(x),\]
and so on.

The $n$th derivative of the function $f$ can also be written as
\[f^{(n)}(x).\]
\ssc{Newton's notation (牛頓符號), dot notation, or fluxions for differentiation}
Suppose a dependent variable $y$ represents a function $f$ of an independent variable $t$, that is,
\[y=f(t),\]
where $t$ usually represents time.

Then the derivative of the function $f$ can be written as
\[\dot{y},\]
the second derivative of the function $f$ can be written as
\[\ddot{y},\]
and so on.
\subsection{Fréchet derivative (弗蘭歇導數)}
Let $V$ and $W$ be normed vector spaces, and $U\subseteq V$ be an open subset of $V$. A function $f:\,U\to W$ is called Fréchet differentiable (弗蘭歇可微的) or differentiable at $x\in U$ if there exists a bounded linear operator $A:\,V\to W$ such that
\[\lim_{\|h\|_V\to 0}\frac{\|f(x+h)-f(x)-Ah\|_W}{\|h\|_V}=0.\]
If there exists such an operator $A$, it is unique, so we write $Df(x)=A$ and call it the Fréchet derivative or derivative of $f$ at $x$.

And we call the function $Df$:
\[Df:\,U\to B(V,W);\,x\mapsto Df(x)\]
Fréchet derivative function (弗蘭歇導函數) or derivative function (導函數) of $f$.

The derivative function of the derivative function of $f$ is called the second derivative function $f$; the derivative function of the $k$ derivative function of $f$ is called the $k+1$th derivative function $f$.
\ssc{Smoothness (光滑性或平滑性)}
A function $f$ that has a $k$th derivative that is continuous in its domain is said to be of class $C^k$ or be a $C^k$-function.

Generally, the term smooth function refers to a $C^{\infty}$-function. However, it may also mean "sufficiently differentiable" for the problem under consideration.
\subsection{Gateaux derivative (加托導數)}
Let $V$ and $W$ be locally convex topological vector spaces (LCTVSs), $U\subseteq V$ be an open subset of $V$, and a function $F:\,U\to W$. The Gateaux derivative $dF(u;\,\psi)$ of $F$ at $x$ in $U$ in the direction $\psi \in V$ is defined to be
\[\begin{aligned}
dF(u;\,\psi) &= \lim_{\tau\to 0}\frac{F(u+\tau \psi)-F(u)}{\tau}\\
&= \frac{\mathrm{d}}{\mathrm{d}\tau}F(u+\tau \psi)\big\vert_{\tau =0}
\end{aligned}\]
If the limit exists for all $\psi \in V$, then it is said that $F$ is Gateaux differentiable (加托可微的) at $u$.
\subsection{Taylor series (泰勒級數) or Taylor expansion (泰勒展開)}
Assume that $F:\,\mathbb{R}\to\mathbb{R}$ is an infinitely differentiable function, and its derivatives of every order exist on $\mathbb{R}$, then the Taylor series of $F$ at $a$ is
\[F(x) = \sum_{n\in\mathbb{N}_0} \frac{F^{(n)}(a)}{n!}(x-a)^n,\]
that is,
\[F(x) = \sum^k_{n=0} \frac{F^{(n)}(a)}{n!}(x-a)^n+\int_0^1\frac{(1-t)^k}{k!}F^{(k+1)}(a+t(x-a))(x-a)^{k+1}\,\mathrm{d}t.\]
Also, the $k$th-order approximation of $f$ near $a$ is
\[F(x) \approx \sum^k_{n=0} \frac{F^{(n)}(a)}{n!}(x-a)^n,\]
and the first-order approximation near $a$ is
\[F(x) \approx F(0)+F'(a)(x-a).\]
The Taylor series of $F$ at $0$ is called Maclaurin series (馬克勞林級數) or Maclaurin expansion (馬克勞林展開).
\subsection{Critical point (臨界點)}
Let $f$ be a real function and \( c \) be a point in \( D_f \) , if \( f'(c) = 0 \) or \( f' \) does not exist at \( c \), then \( c \) is a critical point of \( f \).
\subsection{Relative extremum (相對極值), local extremum (局部極值), or extremum (極值)}
Let $f$ be a real function and \( c \) be a point in \( D_f \), a relative maximum or maximum \( f(c) \) of \(f\) is said to occur at $c$ if there exists an open interval $I$ where \( c\in I \subseteq D_f \) such that \( \forall x\in I:\, f(c) \geq f(x) \).

Let $f$ be a real function and \( c \) be a point in \( D_f \), a relative minimum or minimum \( f(c) \) of \(f\) is said to occur at $c$ if there exists an open interval $I$ where \( c\in I \subseteq D_f \) such that \( \forall x\in I:\, f(c) \leq f(x) \).

The relative maximum and relative minimum are collectively called the relative extreme.
\subsection{Absolute extremum (絕對極值 or 最值) or global extremum (全域極值)}
Let $f$ be a real function and \( c \) be a point in \( D_f \), an absolute maximum (絕對極大值 or 最大值) \( f(c) \) of \(f\) is said to occur at $c$ if \( \forall x\in D_f:\, f(c) \geq f(x) \).

Let $f$ be a real function and \( c \) be a point in \( D_f \), an absolute minimum (絕對極小值 or 最小值) \( f(c) \) of \(f\) is said to occur at $c$ if \( \forall x\in D_f:\, f(c) \leq f(x) \).

The absolute maximum and absolute minimum are collectively called the absolute extreme.
\subsection{Concavity (凹性)}
Let $f:\,J\subseteq\mathbb{R}\to\mathbb{R}$ be differentiable on the open interval $I\subseteq J$. If $f'$ is strictly increasing on $I$, the graph of $f$ is said to concave upward on $I$; if $f'$ is strictly decreasing on $I$, the graph of $f$ is said to concave downward on $I$; if $f'$ is a constant on $I$, the graph of $f$ is said to be neither upward nor downward (or both upward and downward or undefined in some contexts) on $I$.
\subsection{Point of inflection or inflection point (反曲點 or 拐點)}
Let $f:\,I\subseteq\mathbb{R}\to\mathbb{R}$ be a continuous function and be differentiable on an open interval $J\subseteq I$, and let $a<b<c\land a,b,c\in J$. If the graph of $f$ is concave upward on interval $(a,b)$ and concave downward on interval $(b,c)$, or concave downward on interval $(a,b)$ and concave upward on interval $(b,c)$, then $(b,f(b))$ is called an inflection point of the graph of $f$.




\section{Integration (積分)}
\subsection{Riemann integral (黎曼積分) and Darboux integral (達布積分)}
Two equivalent definitions of definite integral (定積分).
\subsubsection{Partition of an interval}
A partition $P(x, n)$ of an interval $[a, b]$ is a finite sequence of numbers of the form
\[P(x, n):=\{x_i:\,a=x_0,b=x_n,\forall 1\leq i<j\leq n:\,x_i<x_j\}_{i=0}^n.\]
Each $[x_i, x_{i+1}]$ is called a sub-interval of the partition. The mesh or norm of a partition is defined to be the length of the longest sub-interval, that is,
\[\max\left(x_{i+1}-x_{i}\right),\quad i\in [0,n-1].\]
A tagged partition $P(x,n,\xi)$ of an interval $[a,b]$ is a partition together with a choice of a sample point within each of all $n$ sub-intervals, that is, numbers $\{\xi_i\}_{i=0}^{n-1}$ with $\xi_i\in [x_i,x_{i+1}]$ for each $i\in [0,n-1]$. The mesh of a tagged partition is the same as that of an ordinary partition.

Suppose that two tagged partitions $P(x,n,\xi)$ and $Q(y,m,\zeta)$ are both partitions of the interval $[a,b]$. We say that $Q(y,m,\zeta)$ is a refinement of $P(x,n,\xi)$ if for each integer $i\in [0,n]$, there exists an integer $r(i)\in [0,m]$ such that $x_i = y_{r(i)}$ and that $\forall i\in [0,n-1]:\,\exists j\in [r(i),r(i + 1)] \text{\ s.t.\ }\xi_i = \zeta_j$. That is, a tagged partition breaks up some of the sub-intervals and adds sample points where necessary, "refining" the accuracy of the partition.

We can turn the set of all tagged partitions into a directed set by saying that one tagged partition is greater than or equal to another if the former is a refinement of the latter.
\subsubsection{Riemann sum}
Let $f$ be a real-valued function defined on the interval $[a,b]$. The Riemann sum of $f$ with respect to the tagged partition $P(x,n,\xi)$ is defined to be
\[R(f,P):=\sum_{i=0}^{n-1}f(\xi_i)\left(x_{i+1}-x_i\right).\]
Each term in the sum is the product of the value of the function at a given point and the length of an interval. Consequently, each term represents the (signed) area of a rectangle with height $f(\xi_i)$ and width $x_{i + 1} − x_i$. The Riemann sum is the (signed) area of all the rectangles.
\subsubsection{Darboux sum}
Lower and upper Darboux sums of $f$ with respect to the partition $P(x,n)$ are two specific Riemann sums of which the tags are chosen to be the infimum and supremum (respectively) of $f$ on each sub-interval:
\[\begin{aligned}
L(f,P)&:=\sum_{i=0}^{n-1}\inf_{\xi\in [x_i,x_{i+1}]}f(\xi)(x_{i+1}-x_i),\\
U(f,P)&:=\sum_{i=0}^{n-1}\sup_{\xi\in [x_i,x_{i+1}]}f(\xi)(x_{i+1}-x_i).
\end{aligned}\]
\subsubsection{Riemann integral}
The Riemann integral of $f$ exists and equals $s$ if for all $\varepsilon > 0$, there exists $\delta > 0$ such that for any tagged partition $P(x,n,\xi)$ whose mesh is less than $\delta$,
\[\abs{R(f,P)-s}<\varepsilon .\]
\subsubsection{Darboux integral (達布積分)}
The Darboux integral of $f$ exists and equals $s$ if for all $\varepsilon > 0$, there exists $\delta > 0$ such that for any partition $P$ whose mesh is less than $\delta$,
\[\abs{U(f,P)-s}<\varepsilon \land \abs{L(f,P)-s}<\varepsilon.\]
\subsubsection{Integrability}
A function is Riemann-integrable if and only if it is Darboux-integrable.
\subsubsection{Improper integral (瑕積分)}
In the simplest case of a real-valued function of a single variable integrated in the sense of Riemann or Darboux over a single interval, $\int _{a}^{b}f(x)\dd{x}$ is an improper integrals if one or more of below conditions occur: 
\ben
\item$a=-\infty$
\item$b=\infty$
\item$f(x)$ is undefined or discontinuous somewhere on $[a,b]$
\een
If only the first or/and second condition occurs, we call the improper integral to be of the first type; if only the third condition occurs, we call the improper integral to be of the second type; otherwise, we call the improper integral to be of the third type.

The improper integral is defined with:
\bit
\item $a=-\infty$:
\[\int _{-\infty }^bf(x)\dd{x}=\lim _{a\to -\infty }\int _{a}^{b}f(x)\dd{x}.\]
\item $b=\infty$:
\[\int _{a}^{\infty }f(x)\dd{x}=\lim _{b\to \infty }\int _{a}^{b}f(x)\dd{x}.\]
\item $f(x)$ is undefined or discontinuous at $c$, in which $a<c<b$:
\[\int _{a}^{b}f(x)\dd{x}=\lim_{t\to c^-}\int _{a}^{t}f(x)\dd{x}+\lim_{t\to c^+}\int _{t}^{b}f(x)\dd{x}.\]
\eit

$\lim _{a\to -\infty }\int _{a}^{b}f(x)\dd{x}$ converges absolutely to $L$ if $\exists\lim _{a\to -\infty }\int _{a}^{b}\abs{f(x)}\dd{x}\land\lim _{a\to -\infty }\int _{a}^{b}\abs{f(x)}\dd{x}=L$.

$\lim _{b\to \infty }\int _{a}^{b}f(x)\dd{x}$ converges absolutely to $L$ if $\exists\lim _{b\to \infty }\int _{a}^{b}f(x)\dd{x}\dd{x}\land\lim _{b\to \infty }\int _{a}^{b}f(x)\dd{x}=L$.

If an improper integral is convergent but not convergent absolutely, we say it converges conditionally.
\subsection{Lebesgue integral (勒貝格積分)}
A definition of definite integral.

Below, we will define the Lebesgue integral of measurable functions from a measure space $(E,\Sigma,\mu)$ into $\mathbb{R}\cup\{-\infty,\infty\}$.
\subsubsection{Of an Indicator functions}
The integral of an indicator function of a measurable set $S$ is defined to be
\[\int 1_{S}\,\mathrm{d}{\mu} =\mu (S).\]
\paragraph{Simple functions}
Simple functions are finite real linear combinations of indicator functions. A simple function $s$ of the form
\[s:=\sum_ka_k1_{S_k},\]
where the coefficients $a_k$ are real numbers and $S_k$ are disjoint measurable sets, is called a measurable simple function. When the coefficients $a_k$ positive real numbers, $s$ is called a non-negative measurable simple function. The integral of a non-negative measurable simple function $\sum_ka_k1_{S_k}$ is defined to be
\[\int\left(\sum_ka_k1_{S_k}\right)\,\mathrm{d}\mu=\sum_ka_k\,\int 1_{S_k}\,\mathrm{d}\mu=\sum_ka_k\,\mu(S_k).\]
whether this sum is finite or $+\infty$.

If $B$ is a measurable subset of $E$ and $s:=\sum_ka_k1_{S_k}$ is a non-negative measurable simple function, one defines
\[\int_Bs\,\mathrm{d}\mu=\int 1_{B}\,s\,\mathrm {d} \mu =\sum _{k}a_{k}\,\mu (S_{k}\cap B).\]
\subsubsection{Non-negative measurable functions}
Let $f$ be a non-negative measurable function on some measurable subset $B$ of $E$. We define
\[\int_Bf\,\mathrm{d}\mu=\sup\left\{\int_Bs\,\mathrm{d}\mu:\,\forall x\in B:\,0\leq s(x)\leq f(x)\land s\text{ is a measurable simple function}\right\}.\]
\subsubsection{Signed functions}
Let $f$ be a measurable function from a measure set $E$ into $\mathbb{R}\cup\{-\infty,\infty\}$. We define
\[\begin{aligned}
f^{+}(x)&=
\begin{cases}
f(x)\hphantom{-}&\text{if }f(x)>0,\\
0&\text{otherwise},
\end{cases}\\
f^{-}(x)&=
\begin{cases}
-f(x&{\text{if }}f(x)<0,\\
0&\text{otherwise}.
\end{cases}
\end{aligned}\]
Note that both $f^+$ and $f^-$ are non-negative and that
\[f=f^+-f^-,\quad |f|=f^++f^-.\]
We say that the Lebesgue integral of $f$ exists, if
\[ \min \left(\int f^{+}\,\mathrm{d}\mu ,\int f^{-}\,\mathrm{d}\mu \right)<\infty .\]
In this case we define
\[ \int f\,\mathrm{d}\mu =\int f^{+}\,\mathrm{d}\mu -\int f^{-}\,\mathrm{d}\mu.\]
If
\[\int |f|\,\mathrm {d} \mu <\infty ,\]
we say that $f$ is Lebesgue integrable.
\ssc{Antiderivative (反導函數), inverse derivative, primitive function, primitive integral or indefinite integral (不定積分)}
An antiderivative of a continuous function $f$ is a differentiable function $F$ whose derivative is equal to the original function $f$, that is,
\[F'=f.\]
Suppose $f$ is a function of an independent variable $x$, then its antiderivative $F$ is denoted as
\[F=\int f\,\mathrm{d}x.\]
The process of solving for antiderivatives is called antidifferentiation (反微分) or indefinite integration (不定積分).



\section{Fundamental Theorem of Calculus (FTC) (微積分基本定理)}
\subsection{The First Theorem}
Let $F(x)$ be a differentiable function defined on $[a,b]$. Then:
\[\int_a^bF'(x)\dd{x}=F(b)-F(a).\]
\begin{proof}\mbox{}\\
By the definition of the Riemann integral:
\[
\int_a^b F'(x)\, \dd{x} = \lim_{n \to \infty} \sum_{i=1}^n F'(x_i^*) \Delta x_i,
\]
where \( \{x_i^*\} \) are sample points in the subintervals of a partition \( P = \{x_0, x_1, \dots, x_n\} \) of \([a, b]\), and \( \Delta x_i = x_i - x_{i-1} \).
By the Mean Value Theorem for derivatives, since \( F(x) \) is differentiable, there exists an adequately refined partition \( P = \{x_0, x_1, \dots, x_n\} \) such that on each subinterval \([x_{i-1}, x_i]\) there exists a point \( x_i^* \in [x_{i-1}, x_i] \) such that:
\[
F'(x_i^*) \cdot \Delta x_i = F(x_i) - F(x_{i-1}).
\]
Thus, the Riemann sum becomes:
\[
\sum_{i=1}^n F'(x_i^*) \Delta x_i = \sum_{i=1}^n \left(F(x_i) - F(x_{i-1})\right) = F(b) - F(a).
\]
\end{proof}
\subsection{The Second Theorem}
Let $f(x)$ be a continuous function defined on $[a,b]$. Then:
\[\dv{x}\int_a^xf(t)\dd{t}=f(x).\]
\begin{proof}
\[\dv{x}\qty(\int_a^xf(t)\,\dd{t})=\lim_{h\to 0}\frac{\int_a^{x+h}f(t)\,\dd{t}-\int_a^xf(t)\,\dd{t}}{h}\]
Using the additivity property of integrals:
\[\int_a^{x+h}f(t)\,\dd{t}-\int_a^xf(t)\,\dd{t}=\int_x^{x+h}f(t)\,\dd{t}\]
By the Mean Value Theorem for integrals, since \(f(t)\) is continuous on \([x,x+h]\), there exists a point \(c\in [x,x+h]\) such that:
\[\int_x^{x+h} f(t)\, \dd{t} = f(c) \cdot h.\]
Substituting into the difference quotient:
\[\frac{\int_x^{x+h}f(t)\,\dd{t}}{h} = f(c).\]
\[\lim_{h\to 0}\frac{\int_x^{x+h}f(t)\,\dd{t}}{h} = f(x).\]
\end{proof}



\section{Convention of multivariable (多變數 or 多變量 or 多元) calculus or multivariate calculus}
\subsection{Space convention}
\begin{itemize}
\item The domain of the funcitons or maps below are subsets of a Euclidean vector space. If not otherwise specified, the coordinates are the Cartesian coordinates, the norms are the Euclidean norms, and the measures are the Lebesgue measures.
\item $\mathbf{0}$ or $0$ refers to the zero tensor (零張量) in the interested Euclidean tensor space $V$, that is, it satisfies 
\[\forall\mathbf{v}\in V:\,\mathbf{v}+\mathbf{0}=\mathbf{v}.\]
\end{itemize}
\subsection{Operators notation convention}
\begin{itemize}
\item Dot product (點積) operator: $\cdot$
\item Cross product (叉積) operator: $\times$
\item Gradient (梯度) operator: $\nabla$
\item Divergence (散度) operator: $\nabla \cdot$
\item Curl (旋度) operator: $\nabla \times$
\item Directional derivative (方向導數) operator: $\cdot\nabla$
\item Laplace (拉普拉斯) operator: $\nabla^2$或$\Delta$
\item Line or Path integral operator: $\int$
\item Surface integral operator: $\iint$
\item Volume integration operator: $\iiint$
\item Closed line integral operator: $\oint$
\item Closed surface Integral Operator: $\oiint$
\item $\int\mathbf{F}\cdot\mathrm{d}\mathbf{S}$ is used as a shorthand for $\int(\mathbf{F}\cdot\mathbf{\hat{n}})\,\mathrm{d}S$, where $\hat{n}$ is the outward pointing unit normal at almost each point on $S$.
\end{itemize}



\section{Multivariable differentiation}
\subsection{Notation convention}
\begin{itemize}
\item Unit vector (單位向量): $\mathbf{e}_i$ is the unit vector in the $i$th direction, i.e., a vector with zero norm.
\item Independent variable vector: $\mathbf{x}=(x_1,x_2,\dots,x_n)$
\item Vector fields: $\mathbf{F}(\mathbf{x}) = \sum_{i=1}^n F_i(\mathbf{x}) \mathbf{e}_i$、$\mathbf{G}$
\item Scalar fields: $A(\mathbf{x})$、$B(\mathbf{x})$
\item Tensor fields: $f(\mathbf{x})$、$g(\mathbf{x})$
\item Three-dimensional tensor space field: $\mathbf{T}(\mathbf{x})$
\item The $i$-th component of the map $f$: $f_i$
\end{itemize}
\subsection{Gradient}
\[
\nabla f = \begin{pmatrix}\qty(\pdv{f}{x_1})^T & \qty(\pdv{f}{x_2})^T & \cdots & \qty(\pdv{f}{x_n})^T\end{pmatrix}
\]
The gradient of a scalar field is a vector field, the gradient of a vector field is a second-order tensor (matrix) field, and the gradient of a $k$-order tensor field is a $k+1$-order tensor field. 

In particular, the gradient of a scalar field $A$ is
\[
\nabla A = \sum_{i=1}^n \pdv{A}{x_i}e_i.
\]
And the gradient of a vector field $\mathbf{F}$ is also called the Jacobian matrix (雅可比矩陣) of it and also denoted as $\mathbf{J}(\mathbf{F})$, $J(\mathbf{F})$, or $J_{\mathbf{F}}$, of which the $( i,j )$th entry is
\[\mathbf{J}_{ij}=\frac{\partial F_i}{\partial x_j}.\]
The determinant $\det\left(J_{\mathbf{F}}\right)$ of a Jacobian matrix is called a Jacobian determinant, or Jacobian for short.
\subsection{Divergence}
\[
\nabla \cdot f = \sum_{i=1}^n\frac{\partial f_i}{\partial x_i}
\]
The divergence of a vector field is a scalar field, the gradient of a second-order tensor (i.e. matrix) field is a vector field, and the divergence of a $k+1$-order tensor field is a $k$-order tensor field.
\subsection{Curl}
The curl is only defined on three-dimensional vector field.
\[
\nabla \times \mathbf{T} = 
\begin{pmatrix}
\mathbf{e}_1 & \mathbf{e}_2 & \mathbf{e}_3 \\
\frac{\partial}{\partial x_1} & \frac{\partial}{\partial x_2} & \frac{\partial}{\partial x_3} \\
T_1 & T_2 & T_3 \\
\end{pmatrix}
\]
The curl of a three-dimensional vector field is a three-dimensional vector field.
\subsection{Directional derivative}
\[(\mathbf{f}\cdot\nabla)\mathbf{g}=\sum_{i=1}^n f_i\pdv{g}{x_i}\]
\subsection{Laplace operator}
\[
\nabla^2 f = \nabla \cdot (\nabla f) = \sum_{i=1}^n\frac{\partial^2 f}{\partial x_i^{\pht{i}2}}
\]
The Laplace operator applied to a tensor field is a tensor field of the same order and same dimension (but not necessarily the same field).
\subsection{Poisson's equation (卜瓦松 or 帕松 or 泊松方程)}
\[
\nabla^2 A = B(\mathbf{x})
\]
\subsection{Laplace's equation (拉普拉斯方程)}
\[
\nabla^2 A = 0
\]
A real function $A$ with real independent variables that is second-order differentiable for all independent variables is called a harmonic function if $A$ satisfies Laplace's equation.
\subsection{Multi-index notation (多重指標記號)}
Multiindex \( \alpha \) is a convenient notation for partial derivatives and polynomial expansions in multiple variables. Suppose there are \( n \) variables \( x_1, x_2, \dots, x_n \), then a multiindex is a vector of \( n \) non-negative integers: 
\[
\alpha = (\alpha_1, \alpha_2, \dots, \alpha_n), \quad \text{where } \alpha_i \in \mathbb{N}_0.
\]
Define: 
\begin{itemize}
\item Norm \( \|\alpha\| \): 
\[
\|\alpha\| = \alpha_1 + \alpha_2 + \cdots + \alpha_n.
\]
\item Factorial \( \alpha! \): 
\[
\alpha! = \alpha_1! \cdot \alpha_2! \cdot \cdots \cdot \alpha_n!.
\]
\item Power \( \mathbf{x}^\alpha \): 
If \( \mathbf{x} = (x_1, x_2, \dots, x_n) \), then
\[
\mathbf{x}^\alpha = x_1^{\alpha_1} \cdot x_2^{\alpha_2} \cdots x_n^{\alpha_n}.
\]
\item High-order mixed partial derivatives $D^\alpha f$: 
\[
D^\alpha f = \frac{\partial^{\|\alpha\|} f}{\partial x_1^{\alpha_1} \partial x_2^{\alpha_2} \cdots \partial x_n^{\alpha_n}}.
\]
\end{itemize}
\subsection{High-order derivative}
The $k$th order derivative of $\mathbf{F}:\,\mathbb{R}^n\to\mathbb{R}^m$, denoted as $\mathbf{F}^{(k)}(\mathbf{x})$ or $D^{k}\mathbf{F}(\mathbf{x})$, is a $(\mathbb{R}^n)^k\to\mathbb{R}$ function, where $(\mathbb{R}^n)^k$ is a Cartesian product of $k$ copies of $\mathbb{R}^n$ vector, that is,
\[D^{k}\mathbf{F}(\mathbf{x})=\sum_{\|\alpha\|=k} \left(D^\alpha \mathbf{F}(\mathbf{a})\right).\]
In particular, the first-order derivative of $\mathbf{F}$ is the gradient of it.
\subsection{Taylor expansion}
Assume that $\mathbf{F}:\,\mathbb{R}^n\to\mathbb{R}^m$ is an infinitely differentiable function, and its partial derivatives of every order exist on $\mathbb{R}^n$, then the Taylor expansion of $\mathbf{F}$ at $\mathbf{a}$ is
\[\mathbf{F}(\mathbf{x}) = \sum_{\|\alpha\|\in\mathbb{N}_0} \frac{D^\alpha \mathbf{F}(\mathbf{a})}{\alpha!} (\mathbf{x} - \mathbf{a})^\alpha,\]
that is,
\[\mathbf{F}(\mathbf{x}) = \sum_{\|\alpha\|\leq k} \frac{D^\alpha \mathbf{F}(\mathbf{a})}{\alpha!} (\mathbf{x} - \mathbf{a})^\alpha+\int_0^1\frac{(1-t)^k}{k!} D^{k+1}\mathbf{F}(\mathbf{a} + t(\mathbf{x} - \mathbf{a})) (\mathbf{x} - \mathbf{a})^{k+1} \, \mathrm{d}t.\]
Also, the $k$th-order approximation of $\mathbf{F}$ near $\mathbf{a}$ is
\[\mathbf{F}(\mathbf{x}) \approx \sum_{\|\alpha\|\leq k} \frac{D^\alpha \mathbf{F}(\mathbf{a})}{\alpha!} (\mathbf{x} - \mathbf{a})^\alpha,\]
and the first-order approximation near $\mathbf{a}$ is
\[\mathbf{F}(\mathbf{x}) \approx \mathbf{F}(\mathbf{a}) + \nabla \mathbf{F}(\mathbf{a}) \cdot (\mathbf{x} - \mathbf{a}).\]



\section{Line integral (線積分) or Path integral (路徑積分)}
\subsection{Scalar field line or path integral}
For a scalar field $A : \,U\subseteq \mathbb {R} ^{n}\to \mathbb {R}$ and the path $C \in U$, the line integral of $A$ is: 
\[\int _{C}A\,\mathrm {d} s=\int _{a}^{b}A(\mathbf{r}(t))\|\dv{t}\mathbf {r} (t)\|\,\mathrm {d} t,\]
where $\mathbf{r}:\, [a, b] \to C$ is a one-to-one parametric function with $\mathbf{r}(a)$ and $\mathbf{r}(b)$ being the two endpoints of the path $C$. 

$A$ is called the integral function, $C$ is called the integral path, and the result of the line integration does not depend on the parametric function $r$.
\subsection{Vector field line or path integral}
For a scalar field $\mathbf{F}: \,U\subseteq \mathbb {R} ^{n}\to \mathbb {R}^n$ and the path $C \in U$, the line integral of $\mathbf{F}$ is: 
\[\int _{C}\mathbf {F} (\mathbf {r} )\cdot \,\mathrm {d} \mathbf{r}=\int _{a}^{b}\mathbf {F} (\mathbf {r} (t))\cdot \dv{t}\mathbf {r} (t)\,\mathrm {d} t\]
where $\mathbf{r}:\, [a, b] \to C$ is a one-to-one parametric function with $\mathbf{r}(a)$ and $\mathbf{r}(b)$ being the two endpoints of the path $C$. 

$\mathbf{F}$ is called the integral function, $C$ is called the integral path, and the result of the line integration does not depend on the parametric function $\mathbf{r}$.
\subsection{Conservative field (保守場)}
A field $f$ whose domain is a subset $U$ of a Euclidean tensor space is called a conservative field if for all paths $C$ between point $a$ and $b$, the integral 
\[\int_Cf(\mathbf{x})\cdot\mathrm{d}\mathbf{x}\]
are the same.

This implies 
\begin{itemize}
\item For any closed path $C$, 
\[\int_Cf(\mathbf{x})\cdot\mathrm{d}\mathbf{x}=0.\]
\item If $\operatorname{dim}(U)=3$, then for any subset of $U$ where $f$ is smooth,
\[\nabla\times f=0.\]
\end{itemize}



\section{Fundamental theorem of multivariable calculus (多變數微積分基本定理)}
\subsection{Gradient theorem (梯度定理)}
Suppose $r$ is a oriented differentiable curve that starts at a point $\mathbf{p}$ and ends at a point $\mathbf{q}$. If $\mathbf{F}$ is a differentiable tensor field defined on a neighborhood of $\mathbf{F}$, then,
\[\int_r(\nabla\mathbf{F})\cdot\mathrm{d}\mathbf{r}=\mathbf{F}\left(\mathbf{q}\right)-\mathbf{F}\left(\mathbf{p}\right).\]
Gradient theorem is a special case of generalized Stokes theorem.
\subsection{Divergence theorem, Gauss's theorem, or Ostrogradsky's theorem (高斯散度定理)}
Suppose $V\subseteq\mathbb{R}^n$ is compact and has a piecewise smooth boundary $S$ (also indicated with $\partial V=S$). The closed, measurable set $\partial V$ is oriented by outward-pointing normals. If $F$ is a continuously differentiable vector field defined on a neighborhood of $V$, then,
\[\iiint_V\left(\nabla\cdot\mathbf {F}\right)\,\mathrm{d}V=\oiint_S\mathbf{F}\cdot\mathrm{d}\mathbf{S}\]
Divergence theorem is a special case of generalized Stokes theorem.
\subsection{Stokes' theorem (斯托克斯定理) or Kelvin–Stokes theorem}
Let $S$ be a positively oriented, piecewise smooth surface in $\mathbb{R}^3$ with boundary $\partial S\equiv L$. If a vector field $\mathbf{F}:\,\mathbb{R}^3\rightarrow\mathbb{R}^3$ is defined and has continuous first order partial derivatives in a region containing $S$, then,
\[\iint_S(\nabla\times\mathbf{F})\cdot \mathrm{d}\mathbf{S}=\oint_{L}\mathbf{F}\cdot\mathrm{d}\mathbf{L}\]
Stokes' theorem is a special case of generalized Stokes theorem.
\subsection{Green's theorem (格林定理或綠定理)}
Let $S$ be a positively oriented, piecewise smooth surface in $\mathbb{R}^2$ with boundary $\partial S\equiv L$. If scalar function $P,(x,y)\,Q(x,y)$ are defined and has continuous first order partial derivatives in a region containing $S$, then,
\[\oint_L (P\mathrm{d}x+Q\mathrm{d}y)=\iint_S\left(\frac{\partial Q}{\partial x}-\frac{\partial P}{\partial y}\right)\mathrm{d}x\mathrm{d}y\]
where the path of integration along C is counterclockwise.

Green's theorem is a special case of Stokes' theorem.
\subsection{Generalized Stokes theorem, Stokes–Cartan theorem, or fundamental theorem of multivariable calculus}
The generalized Stokes theorem says that the integral of a differential form $\omega$ over the boundary $\partial\Omega$ of some orientable manifold $\Omega$ is equal to the integral of its exterior derivative $\mathrm{d}\boldsymbol{\omega}$ over the whole of $\Omega$, i.e.,
\[\int _{\partial\Omega}\omega=\int_{\Omega}\mathrm{d}\boldsymbol{\omega}\]



\section{Differential theorems}
\ssc{Distributive over addition and subtraction}
Differentiation is distributive over addition and subtraction.
\ssc{Product rule (乘法定則)}
\[\dv{x}\qty(f(x)g(x))=f'(x)g(x)+f(x)g'(x).\]
\ssc{Quotient rule (除法定則)}
\[\dv{x}\qty(\frac{f(x)}{g(x)})=\frac{f'(x)g(x)-f(x)g'(x)}{\qty(g(x))^2}.\]
\ssc{Chain rule (連鎖律)}
\[\dv{x}\qty((f\circ g)(x))=f'\qty(g(x))g'(x).\]
\subsection{Intermediate Value Theorem, IVT (中間值定理)}
Let $f:\, I\subseteq\mathbb{R}\to\mathbb{R}$ be a continuous function, then
\[\forall [a,b]\subseteq I \text{\ s.t.\ } f(a)\neq f(b):\,k\in (f(a), f(b))\implies (\exists c\in (a,b) \text{\ s.t.\ }f(c)=k)\]
\subsection{Mean Value Theorem, MVT (均值定理)}
Let $f:\, I\subseteq\mathbb{R}\to\mathbb{R}$ be a continuous function, and $f$ is differentiable on an interval $(a, b)\subseteq I$, then
\[\exists c\in (a, b)\text{\ s.t.\ }f'(c)=\frac{f(b)-f(a)}{b-a}\]
\subsection{Extreme Value Theorem, EVT (極值定理)}
Let function $f:\, I\subseteq\mathbb{R}\to\mathbb{R}$ be continuous on an interval $[a,b]$, then
\[\exists c,d \in [a, b] \text{\ s.t.\ }\qty(\forall x \in [a, b]\colon f(c)\geq f(x)\geq f(d))\]
\subsection{L'Hôpital's rule (羅必達法則) or Bernoulli's rule}
Let \(I\) be an interval containing the point \(a\). Let \( f(x) \) and \( g(x) \) be functions defined on \(I\), except possibly at \(a\) itself. Let \( f(x) \) and \( g(x) \) be differentiable at all points except $a$ in $I$. If $\lim_{x\to a}f(x)=\lim_{x\to a}g(x)\in\{0,\infty,-\infty\}$, and $\exists \lim_{x\to a}\frac{f'(x)}{g'(x)}$, then:
\[\lim_{x\to a}\frac{f(x)}{g(x)}=\lim_{x\to a}\frac{f'(x)}{g'(x)}\]
\subsection{Relative extreme theorem}
A point in $I$ where a relative extreme of $f:\,I\subseteq\mathbb{R}\to\mathbb{R}$ occurs must be a critical point.
\subsection{Concavity theorem}
Let $f:\,J\subseteq\mathbb{R}\to\mathbb{R}$ be differentiable on the open interval $I\subseteq J$. If $\forall x\in I:\,f''>0$, then the graph of $f$ is concave upward on $I$; if $\forall x\in I:\,f''<0$, then the graph of $f$ is concave downward on $I$.
\subsection{Inflection point theorem}
If $(c,f(c))$ is an inflection point of $f$, then $f''(c)=0$ or $f''$ does not exist at $c$.
\subsection{Lagrange multiplier (method) (拉格朗日乘數 or 乘子(法))}
The Lagrange multiplier method is a method for finding the points where extremes occur of a differentiable function under constraints.
\sssc{Univariate form}
Let $f:\,\mathbb{R} \rightarrow \mathbb{R}$ and $g:\,\mathbb{R} \rightarrow \mathbb{R}$ . We want to find the points where extremes of \( f(x) \) under the constraint \( g(x) = \mathbf{0} \) occur. 

First, construct the Lagrangian function \( \mathcal{L}(x,\lambda) \):
\[\mathcal{L}(x,\lambda) = f(x) - \lambda \cdot g(x),\]
where \( \lambda\in\mathbb{R} \) is the Lagrange multiplier (拉格朗日乘數 or 乘子).

Find the derivative of $\mathcal{L}$ and set it to zero:
\[
\dv{\mathcal{L}}{x} = 0
\]
Solve the equation to find \( x \) and \( \lambda \) . 

\text{Statement:} Let $A$ be the set of all solutions for \( x \) such that $\dv{\mathcal{L}}{x} = 0$, and $B$ be the set of all points where extremes of \( f(x) \) under the constraint \( g(x) = \mathbf{0} \) occur. We claim that $B\subseteq A$.
\sssc{Multivariate form}
Let \( \mathbf{x} = (x_1, x_2, \dots, x_n) \) be the independent variable vector and $\mathbf{0}$ be the zero vector. Now we have $f:\,\mathbb{R}^n \rightarrow \mathbb{R}$ and $g:\,\mathbb{R}^n \rightarrow \mathbb{R}^c$. We want to find the points where extremes of \( f(\mathbf{x}) \) under the constraint \( g(\mathbf{x}) = \mathbf{0}\) occur. 

First, construct the Lagrangian function \( \mathcal{L}(\mathbf{x},\lambda) \):
\[
\mathcal{L}(\mathbf{x},\lambda) = f(\mathbf{x}) - \lambda \cdot g(\mathbf{x}),
\]
where \( \lambda\in\mathbb{R}^c \) is the Lagrange multiplier vector.

Find the gradient of $\mathcal{L}$ and set it to zero:
\[
\nabla \mathcal{L} = \mathbf{0}
\]
Solve the equation to find \( \mathbf{x} \) and \( \lambda \) . 

\text{Statement:} Let $A$ be the set of all solutions for \( \mathbf{x} \) such that $\nabla \mathcal{L} = \mathbf{0}$, and $B$ be the set of all points where extremes of \( f(\mathbf{x}) \) under the constraint \( g(\mathbf{x}) = \mathbf{0}\) occur. We claim that $B\subseteq A$.
\begin{proof}\mbox{}\\
Consider $\mathbf{x}^*\in B$. It must satisfy the constraint:
\[g(\mathbf{x}^*) = \mathbf{0}.\]
Any feasible point $\mathbf{x}$ near $\mathbf{x}^*$ must satisfy the constraint. We can represent $\mathbf{x}$ as:
\[\mathbf{x} = \mathbf{x}^* + \delta\mathbf{x},\]
where $\delta\mathbf{x}$ is a differential change tangent to the manifold defined by $g(\mathbf{x})$, that is, it lies in the kernel of $\nabla g(\mathbf{x}^*)$, that is,
\[g(\mathbf{x}^* + \delta\mathbf{x}) = \mathbf{0}.\]
Find the first-order approximation of $f$ at $\mathbf{x}^*$:
\[f(\mathbf{x}^*+ \delta\mathbf{x}) \approx f(\mathbf{x}^*) + \nabla f(\mathbf{x}^*) \cdot \delta\mathbf{x} + O(\|\delta\mathbf{x}\|^2)\]
Find the first-order approximation of $g$ at $\mathbf{x}^*$:
\[g(\mathbf{x}^*+ \delta\mathbf{x}) \approx g(\mathbf{x}^*) + \nabla g(\mathbf{x}^*) \cdot \delta\mathbf{x} + O(\|\delta\mathbf{x}\|^2)\]
Since $\mathbf{x}^*\in B$, for any feasible $\delta\mathbf{x}$ we must have:
\[\nabla f(\mathbf{x}^*) \cdot \delta\mathbf{x} = 0.\]
Since $g(\mathbf{x}^*) = \mathbf{0}$, we have
\[\nabla g(\mathbf{x}^*) \cdot \delta\mathbf{x} = O(\|\delta\mathbf{x}\|^2).\]
Because \( \delta\mathbf{x} \in \ker(\nabla g(\mathbf{x}^)) \), \( \nabla f(\mathbf{x}^) \) can be expressed as a linear combination of \( \nabla g(\mathbf{x}^) \) . This means that there exists a vector to $\lambda$ such that:
\[\nabla\mathcal{L} = \nabla \qty(f(\mathbf{x}) - \lambda \cdot g(\mathbf{x})) = \mathbf{0} \]
\end{proof}



\section{Integral theorems}
\ssc{Distributive over addition and subtraction}
Integration is distributive over addition and subtraction.
\subsection{Integration by substitution (代換積分法), integration by change of variables (換元積分法), u-substitution (u-代換), reverse chain rule, substitution theroem (代換定理), change of variables theorem (換元定理), or transformation theorem (變換定理)}
\subsubsection{Univariate form}
Let $g:\,I\subseteq\mathbb{R}\to\mathbb{R}$ be injective and differentiable on $[a,b]\subseteq I$, with $g'$ being integrable on $[a,b]$, and $f:\,K\supseteq g([a,b])\to\mathbb{R}$ be integrable on $g([a,b])$. Then:
\[\int_a^bf\circ g(x)\cdot g'(x)\,\mathrm{d}x=\int_{g(a)}^{g(b)}f(u)\,\mathrm{d}u,\]
and for $K=g([a,b])$:
\[\int f\circ g(x)\cdot g'(x)\,\mathrm{d}x=\int f(u)\,\mathrm{d}u.\]
\begin{proof}\mbox{}\\
Consider the interval \([a, b]\) partitioned as
\[P = \{a = x_0 < x_1 < \cdots < x_n = b\}.\]
For each subinterval \([x_{i-1}, x_i]\), let \(\xi_i \in [x_{i-1}, x_i]\) be a sample point. The Riemann sum for the left-hand side of the integral is:
\[S_P = \sum_{i=1}^n f(g(\xi_i)) g'(\xi_i) (x_i - x_{i-1}).\]
Since \(g\) is injective and differentiable, it is either strictly increasing or strictly decreasing on \([a, b]\). Assume \(g\) is strictly increasing (the proof for \(g\) strictly decreasing follows similarly).

Under this assumption, \(g\) maps \([a, b]\) to \([g(a), g(b)]\) (with \(g(a) < g(b)\)). Let \([g(a), g(b)]\) be partitioned as
\[Q = \{g(a) = u_0 < u_1 < \cdots < u_m = g(b)\},\]
where \(u_i = g(x_i)\). For \(g\) increasing, \((u_i - u_{i-1}) = g(x_i) - g(x_{i-1})\).

The Riemann sum for the right-hand side of the integral is:
\[T_Q = \sum_{i=1}^n f(u_i) (u_i - u_{i-1}).\]
Since \(u_i = g(x_i)\) and \(g'(\xi_i) \approx \frac{g(x_i) - g(x_{i-1})}{x_i - x_{i-1}}\), we rewrite:
\[ g'(\xi_i) (x_i - x_{i-1}) \approx g(x_i) - g(x_{i-1}) = u_i - u_{i-1}. \]
Thus, the left-hand Riemann sum \(S_P\) becomes:
\[ S_P = \sum_{i=1}^n f(g(\xi_i)) (u_i - u_{i-1}),\]
which matches the structure of the right-hand Riemann sum \(T_Q\) if we let \(\xi_i = g^{-1}(u_i)\).

As the partition \(P\) of \([a, b]\) gets finer, the Riemann sum \(S_P\) converges to \(\int_a^b f(g(x)) g'(x) \, \mathrm{d}x\). Similarly, as the partition \(Q\) of \([g(a), g(b)]\) gets finer, the Riemann sum \(T_Q\) converges to \(\int_{g(a)}^{g(b)} f(u) \, \mathrm{d}u\).
\end{proof}
\subsubsection{Multivariate form}
Let $\mathbf{T}:\,I\subseteq\mathbb{R}^n\to\mathbb{R}^n$ be injective and differentiable on $D\subseteq I$, with all elements of its gradient (i.e. Jacobian matrix) $\nabla\mathbf{T}$ being continuous on $D$, and $f:\,K\supseteq\mathbf{T}(D)\to\mathbb{R}$ be integrable on $\mathbf{T}(D)$. Then:
\[\int_{\mathbf{T}(D)}f(x_1\,x_2\,\dots\,x_n)\cdot\,\mathrm{d}x_1\,\mathrm{d}x_2\,\dots\,\mathrm{d}x_n=\int_D f(u_1\,u_2\,\dots\,u_n)\abs{\det\left(\nabla\mathbf{T}\right)}\,\mathrm{d}u_1\,\mathrm{d}u_2\,\dots\,\mathrm{d}u_n,\]
and for $K=\mathbf{T}(D)$:
\[\int f(x_1\,x_2\,\dots\,x_n)\cdot\,\mathrm{d}x_1\,\mathrm{d}x_2\,\dots\,\mathrm{d}x_n=\int f(u_1\,u_2\,\dots\,u_n)\abs{\det\left(\nabla\mathbf{T}\right)}\,\mathrm{d}u_1\,\mathrm{d}u_2\,\dots\,\mathrm{d}u_n.\]
\subsubsection{Measure theory form}
Let $X$ be a locally compact Hausdorff space equipped with a finite Radon measure $μ$, and let $Y$ be a \text{\textsigma}-compact Hausdorff space with a \text{\textsigma}-finite Radon measure $\rho$. Let $\phi:\,X\to Y$ be an absolutely continuous function, (which implies that $\mu(E)=0\implies\rho(\phi(E))=0$). Then there exists a real-valued Borel measurable function $w$ on $X$ such that for every Lebesgue integrable function $f:\,Y\to\mathbb{R}$, the function $(f\circ\phi)\cdot w$ is Lebesgue integrable on $X$, and for every open subset $U$ of $X$
\[\int_{\phi(U)}f(y)\,\mathrm{d}\rho(y)=\int_U(f\circ\phi)(x)\cdot w(x)\,\mathrm{d}\mu(x).\]
Furthermore, there exists some Borel measurable function $g$ such that 
\[w(x)=(g\circ\phi)(x).\]
\subsection{Integration by parts (分部積分法) or partial integration (部分積分法)}
\subsubsection{Theorem}
\[\frac{\mathrm{d}}{\mathrm{d}x}\prod_{i=1}^nf_i(x)=\sum_{j=1}^n\left(\frac{\mathrm{d}f_j(x)}{\mathrm{d}x}\frac{\prod_{\substack{i=1\\i\neq j}}^n f_i(x)}{f_j(x)}\right)\]
\subsubsection{Application}
Integration by parts is a heuristic rather than a purely mechanical process for solving integrals; given a single function to integrate, the typical strategy is to carefully separate this single function into a product of two functions such that the residual integral from the integration by parts formula is easier to evaluate than the single function.

The DETAIL rule or the LIATE rule is a rule of thumb for integration by parts. It involves choosing as u the function that comes first in the following list:
\begin{itemize}
\item L: Logarithmic function
\item I: Inverse trigonometric function
\item A: Algebraic function (such as polynomial function)
\item T: Trigonometric function
\item E: Exponential function
\end{itemize}



\section{Numerical differentiation (數值微分)}
\subsection{Newton's Method (牛頓法) or Newton-Raphson Method (牛頓-拉普森法)}
Newton's method, also known as Newton-Raphson method, is an iterative technique used to approximate the roots of a real-valued function. Given a function \( f(x) \) and an initial guess \( x_0 \) close to a root, Newton's method refines this guess by repeatedly applying the formula:
\[
x_{n+1} = x_n - \frac{f(x_n)}{f'(x_n)},n\in\mathbb{N}
\]
where:
\begin{itemize}
\item \( x_n \) is the current approximation,
\item \( f(x_n) \) is the value of the function at \( x_n \),
\item \( f'(x_n) \) is the derivative of \( f(x) \) evaluated at \( x_n \).
\end{itemize}
\ssc{(Unilateral or One-sided) Laplace Transform ((單邊)拉普拉斯變換)}
The (unilateral or one-sided) Laplace transform is an integral transform that converts a function of a real variable (usually $t$, in the time domain) to a function of a complex variable (usually $s$, in the complex-valued frequency domain, also known as $s$-domain, or $s$-plane). The functions are often denoted in lowercase for the time-domain representation and uppercase for the frequency-domain.

The Laplace transform, denoted as $\mathcal{L}\{f(t)\}(s)$ or $F(s)$ is defined (for suitable functions $f$) by the improper integral
\[\mathcal{L}\{f(t)\}(s) = F(s) = \int_0^{\infty} e^{-st} f(t)\,\mathrm{d}t.\]

If $f$ is integrable function, the Laplace transform $F(s)$ of $f$ converges provided that the limit 
\[\lim _{R\to \infty }\int _{0}^{R}e^{-st}f(t)\dd{t}\]
exists.

The Laplace transform converges absolutely if the integral
\[\int_0^{\infty} \abs{e^{-st} f(t)}\,\mathrm{d}t\]
exists as a proper Lebesgue integral.

The set of values for which $F(s)$ converges absolutely, called region of convergence (ROC), is either of the form $\Re(s) > a$ or $\Re(s) \geq a$, where $a$ is an extended real constant, i.e. $-\infty\leq a\leq\infty$.
\ssc{(Bilateral or Two-sided) Laplace Transform ((雙邊)拉普拉斯變換)}
The (bilateral or two-sided) Laplace transform is an integral transform that converts a function of a real variable (usually $t$, in the time domain) to a function of a complex variable (usually $s$, in the complex-valued angular frequency domain, also known as $s$-domain or $s$-plane). The functions are often denoted in lowercase for the time-domain representation and uppercase for the frequency-domain.

The Laplace transform, denoted as $\mathcal{B}\{f(t)\}(s)$ is defined (for suitable functions $f$) by the improper integral
\[\mathcal{B}\{f(t)\}(s) = \int_{-\infty}^{\infty} e^{-st} f(t)\,\mathrm{d}t.\]
\ssc{Fourier Transform (FT) (傅立葉變換)}
Fourier transform is an integral transform that converts a function of a real variable (usually $t$, in the time domain) to a function of another real variable (usually $\omega$, in the real-valued angular frequency domain). The functions are often denoted in lowercase for the time-domain representation and uppercase for the frequency-domain.

The Fourier transform, denoted as $\mathcal{F}\{f(t)\}(\omega)$ or $F(\omega)$ is defined (for suitable functions $f$) by the improper integral
\[\mathcal{F}\{f(t)\}(\omega) = F(\omega) = \int_{-\infty}^{\infty} e^{-i\omega t} f(t)\,\mathrm{d}t.\]



\section{Numerical integration (數值積分)}
\subsection{The trapezoidal rule (梯形法)}
Let $f$ be a continuous real-valued function on $[a,b]$, the trapezoidal rule gives the approximation
\[\int_a^bf(x)\,\mathrm{d}x\approx\frac{b-a}{2n}\left(2\left(\sum_{i=0}^nf(a+\frac{i}{n}(b-a))\right)-f(a)-f(b)\right).\]
The error is defined as
\[E_n=\int_a^bf(x)\,\mathrm{d}x-\frac{b-a}{2n}\left(2\left(\sum_{i=0}^nf(a+\frac{i}{n}(b-a))\right)-f(a)-f(b)\right).\]
When $\frac{\mathrm{d}^2f(x)}{\mathrm{d}x^2}$ is continuous on $[a,b]$, the error satisfies that
\[\abs{E_n}\leq\frac{(b-a)^3}{12n^2}\max_{a\leq x\leq b}\left(\abs{\frac{\mathrm{d}^2f(x)}{\mathrm{d}x^2}}\right).\]
\subsection{The Simpson's rule (辛普森法) or the Simpson's 1 or 3 rule}
Let $f$ be a continuous real-valued function on $[a,b]$, the Simpson's rule gives the approximation
\[\int_a^bf(x)\,\mathrm{d}x\approx\frac{b-a}{3n}\left(2\left(\sum_{i=0}^nf(a+\frac{i}{n}(b-a))\right)+2\left(\sum_{i=1}^{\frac{n}{2}}f(a+\frac{2i-1}{n}(b-a))\right)-f(a)-f(b)\right),\]
where $\frac{n}{2}\in\mathbb{N}$.

The error is defined as
\[E_n=\int_a^bf(x)\,\mathrm{d}x-\frac{b-a}{3n}\left(2\left(\sum_{i=0}^nf(a+\frac{i}{n}(b-a))\right)+2\left(\sum_{i=1}^{\frac{n}{2}}f(a+\frac{2i-1}{n}(b-a))\right)-f(a)-f(b)\right).\]
When $\frac{\mathrm{d}^4f(x)}{\mathrm{d}x^4}$ is continuous on $[a,b]$, the error satisfies that
\[\abs{E_n}\leq\frac{(b-a)^5}{180n^4}\max_{a\leq x\leq b}\left(\abs{\frac{\mathrm{d}^4f(x)}{\mathrm{d}x^4}}\right).\]



\section{Common Examples}
\subsection{Arithmetic progression or sequence (等差數列) and arithmetic series (等差級數)}
An arithmetic sequence is a sequence $\langle a_n\rangle=\langle a_1+(n-1)d\rangle$. 

Given $a$ and $b$, $\frac{a+b}{2}$ is called the median of an arithmetic sequence (等差中項).

An arithmetic series is a series $S_n=\sum_{i=1}^na_i$, where $\langle a_n\rangle$ is an arithmetic sequence.

\[\nexists\lim_{n\to\infty}a_n,\quad d\neq 0\]
\[\lim_{n\to\infty}a_n=a_1,\quad d=0\]
\[S_n=\frac{n}{2}\qty(a_1+a_n)=\frac{n}{2}\qty(2a_1+(n-1)d)=na_1+\frac{n(n-1)d}{2}\]
\[\nexists\lim_{n\to\infty}S_n,\quad a_1\neq 0\lor d\neq 0\]
\[\lim_{n\to\infty}S_n=0,\quad a_1=0\land d=0\]
\subsection{Geometric progression or sequence (等比 or 幾何數列) and geometric series (等比 or 幾何級數)}
A geometric sequence is a sequence $\langle a_n\rangle=\langle a_1\cdot r^{n-1}\rangle$, where $a_1r\neq 0$.

Given $a$ and $b$, $\pm\sqrt{ab}$ is called the median of an geometric sequence (等比中項).

A geometric series is a series $S_n=\sum_{i=1}^na_i$, where $\langle a_n\rangle$ is a geometric sequence.

\[S_n=\frac{a_1\qty(1-r^n)}{1-r},\quad r\neq 1\]
\[S_n=na_1,\quad r=1\]
\[\lim_{n\to\infty}S_n=\frac{a_1}{1-r},\quad \abs{r}<1\]
\[\nexists\lim_{n\to\infty}S_n,\quad \abs{r}\geq 1\]
\subsection{Riemann zeta function (黎曼 zeta 函數)}
\[\begin{aligned}
\zeta(s) &= \sum_{n=1}^\infty\frac{1}{n^s}\\
&= \frac{1}{\Gamma (s)}\int _0^\infty \frac {x^{s-1}}{e^x-1}\,\mathrm {d} x
\eam
\sssc{Harmonic Series (調和級數)}
\[S_n=\sum_{n=1}^n\frac{1}{n}\]
\[\nexists\sum_{n=1}^\infty\frac{1}{n}\]
\sssc{Basel Problem (巴塞爾問題)}
\[\zeta(2)=\frac{\pi^2}{6}\]
\sssc{Other Even Positive Integers}
\[\zeta(4)=\frac{\pi^4}{90}\]
\[\zeta(6)=\frac{\pi^6}{945}\]
\[\zeta(8)=\frac{\pi^8}{9450}\]
\[\zeta(10)=\frac{\pi^{10}}{93555}\]
\[\zeta(12)=\frac{691\pi^{12}}{638512875}\]
\[\zeta(14)=\frac{2\pi^{14}}{18243225}\]
\sssc{Infinity}
\[\lim_{n\to\infty}\zeta(n)=1\]
\subsection{Euler–Mascheroni constant (歐拉–馬斯克若尼常數)}
\[\begin{aligned}
\gamma &= \lim _{n\to \infty }\left(\left(\sum _{k=1}^n\frac {1}{k}\right)-\ln(n)\right)\\
&= \int _1^\infty \left(\frac{1}{\lfloor x\rfloor}-\frac{1}{x}\right)\,\mathrm{d}x
\end{aligned}\]
\subsection{Power series (冪級數)}
\[\begin{aligned}
\sum_{i=1}^ni &= \frac{n\qty(n+1)}{2}\\
\sum_{i=1}^ni^2 &= \frac{n\qty(n+1)\qty(2n+1)}{6}\\
\sum_{i=1}^ni^3 &= \qty(\frac{n(n+1)}{2})^2\\
\sum_{i=1}^ni^r &= n + \sum_{k=1}^{n-1} (n-k)((k+1)^r - k^r)\\
&= n + \sum_{k=1}^{n-1} (n-k)\sum_{j=0}^{r-1}\binom{r}{j}k^{j}
\end{aligned}\]
\subsection{Limits Involving Quotient Functions}
Let \( a \) and \( b \) be real numbers, set
\[A=\left\{f\colon U\subseteq\mathbb{R}\to\mathbb{R} \middle | f(x) = x \lor \ln(f(x)) \in A \lor e^{f\left(x\right)}  \in A \right\},\]
and function $f\in A$. Then:
\[\lim_{x \to \infty} \frac{\left(f\left(x\right)\right)^a}{\left(f\left(x\right)\right)^b} = \infty, \quad a > b \]
\[ \lim_{x \to \infty} \frac{af\left(x\right)}{bf\left(x\right)} = \frac{a}{b}, \quad b \neq 0 \]
\[ \lim_{x \to \infty}\frac{n^{af\left(x\right)}}{bf\left(x\right)} = \infty, \quad a,b > 0 \land  n > 1 \]
\[ \lim_{x \to \infty}\frac{n^{af\left(x\right)}}{bf\left(x\right)} = 0, \quad a,b > 0 \land  0\leq n<1 \]
\[ \lim_{x \to \infty}\frac{af\left(x\right)}{b\log_n f\left(x\right)} = \infty, \quad a,b > 0 \land  n > 1 \]
\ssc{Heaviside step function (黑維塞階躍函數)/unit step function (單位階躍函數)}
\[H(x)=
\begin{cases}0,\quad &x<0\\
1,\quad &x\geq 0
\end{cases},\]
where $H(0)$ is sometimes defined to be $\frac{1}{2}$ or other values instead.
\[H'(x)=\delta(x).\]
\ssc{Dirac delta function (狄拉克 δ 函數)}
\[
\delta(x) =
\begin{cases}
\infty, & x = 0, \\
0, & x \neq 0.
\end{cases}
\]
\[
\forall \epsilon > 0:\,\int_{a-\epsilon}^{a+\epsilon} f(x) \delta(x-a) \, \mathrm{d}x = f(a).
\]
\[
\int _{-\infty }^{\infty }\delta(x) \, \mathrm{d}x = 1.
\]
\[\int_0^x\delta(t)\,\mathrm{d}t=H(x).\]
\ssc{Common integrals}
\sssc{Rational function}
\[\int\frac{f'(x)}{f(x)}\,\mathrm{d}x=\ln|f(x)|+C.\]
\[\int\frac{1}{x^2+a^2}\,\mathrm{d}x=\frac{1}{a}\arctan\frac{x}{a}+C.\]
\[\int\frac{1}{x^2-a^2}\,\mathrm{d}x=\frac{1}{2a}\ln\frac{x-a}{x+a}+C.\]
\[\int\frac{1}{a^2-x^2}\,\mathrm{d}x=\frac{1}{2a}\ln\frac{a+x}{a+x}+C.\]
\[\int\frac{1}{ax+b}\,\mathrm{d}x=\frac{1}{a}\ln|ax+b|+C.\]
\[\int(ax+b)^n\,\mathrm{d}x=\frac{(ax+b)^{n-1}}{a(n+1)}+C,\quad n\neq -1.\]
\[\int\frac{x}{ax+b}\,\mathrm{d}x=\frac{x}{a}-\frac{b}{a^2}\ln|ax+b|+C.\]
\[\int\frac{x}{(ax+b)^2}\,\mathrm{d}x=\frac{b}{a^2(ax+b)}+\frac{1}{a^2}\ln|ax+b|+C.\]
\[\int x(ax+b)^n\,\mathrm{d}x=\frac{a(n+1)x-b}{a^2(n+1)(n+2)}(ax+b)^{n+1}+C,\quad n\notin\{-1,-2\}.\]
\sssc{Irrational function}
\[\int\sqrt{a^2+x^2}\,\mathrm{d}x=\frac{x}{2}\sqrt{a^2+x^2}+\frac{a^2}{2}\ln\qty(x+\sqrt{a^2+x^2})+C.\]
\[\int\sqrt{x^2-a^2}\,\mathrm{d}x=\frac{x}{2}\sqrt{x^2-a^2}-\frac{a^2}{2}\ln\qty(x+\sqrt{x^2-a^2})+C,\quad x^2>a^2.\]
\[\int\sqrt{a^2-x^2}\,\mathrm{d}x=\frac{x}{2}\sqrt{a^2-x^2}+\frac{a^2}{2}\arcsin\frac{x}{|a|}+C,\quad|a|\geq|x|.\]
\end{document}