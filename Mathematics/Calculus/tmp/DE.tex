\section{Differential Equation (微分方程)}
\ssc{Ordinary Differential Equations (常微分方程)}
\sssc{Characteristic Equation Method for Solving Linear Ordinary Differential Equations with Constant Coefficients (解常係數線性常微分方程的特徵方程法)}
\paragraph*{Equation to be solved:}
\[\sum_{i=0}^na_i\dv[i]{y}{x}=f(x)\]
where $a_i$ are constants.

\paragraph{Take the homogeneous equation:}
\[\sum_{i=0}^na_i\dv[i]{y_h}{x}=0\]

\paragraph{Take the characteristic equation:}

Assume the homogeneous solution is in the form of $y_h=e^{\lambda x}$. By substituting it into the homogeneous equation, we get the characteristic equation:
\[\sum_{i=0}^na_i\lambda^i=0\]
\paragraph{Find characteristic roots:}

Solving the above characteristic equation yields $m$ distinct complex roots, where the $i$th is $\lambda_i$ (the order does not matter).

\paragraph{Construct homogeneous solution:}

Let the root $\lambda_i$ be a repeated root with multiplicity of $k_i$. The corresponding homogeneous solution $y_h$ is:
\[y_h=\sum_{i=1}^me^{\Re(\lambda_i)x}\sum_{j=1}^{k_i}x^{j-1}\left(C_{ij}\cos(\Im{\lambda_i}x)+D_{ij}\sin(\Im{\lambda_i}x)\right)\]
where $C_{ij}$ and $D_{ij}$ are constants. If $\lambda_i$ is a real number, then $\left(C_{ij}\cos(\Im{r_i}x)+D_{ij}\sin(\Im{r_i}x)\right)$ degenerates to $C_{ij}$. If $f(x)=0$, that is, the original equation is a linear homogeneous ordinary differential equation with constant coefficients, then the homogeneous solution is the general solution.

\paragraph{Heuristic method for constructing particular solutions:}

The heuristic method is to assume a particular solution $y_p$ that satisfies the original non-homogeneous equation. The form of the particular solution depends on the inhomogeneous term $f(x)$ on the right-hand side. Some $f(x)$ are difficult to handle using the heuristic method.

If $f(x)$ has $k$ different terms, we can find each particular solutions $y_{pi}$ separately, and the total particular solution $y_p=\sum_{i=1}^ky_{pi}$. The particular solution of a term can be assumed to be the result of the term after a finite number of translations (i.e., replacing $x$ with $x-h$, where $h$ is a constant, or adding a constant to the expression) and or or stretches (i.e., replacing $x$ with $ax$, where $a$ is a constant, or multiplying the expression by a constant). However, if a particular solution is assumed to be a polynomial of a term in the homogeneous solution, the particular solution assumption must be multiplied by $a(x-h)$, where $a$ and $h$ are constants. If the particular solutions of different terms of $f(x)$ have a term of the same form, we can retain any one of them and skip the others. Common terms of $f(x)$ that can be handled by the heuristic method include:
\begin{itemize}
\item linear combinations of polynomials, exponential functions, and sine or cosine functions,
\item logarithmic function, and
\item tangent function.
\end{itemize}

\paragraph{Construct general solution:}

The general solution $y$ is:
\[y=y_h+y_p\]

\paragraph*{Example:}
Consider the second-order non-homogeneous linear ordinary differential equation:
\[y''-2y'+3y=\ln(x)+\tan(x)\]
The homogeneous equation is:
\[y''-2y'+3y=0\]
The characteristic equation is:
\[\lambda^2-2\lambda+3=0\]
The characteristic roots are:
\[\lambda=\frac{2\pm\sqrt{4-12}}{2}=1\pm\sqrt{2}i\]
The homogeneous solution is:
\[y_h=e^x(C\cos(\sqrt{2}x)+D\sin(\sqrt{2}x))\]
Assume the particular solution for $\ln(x)$ is $y_{p1}=A\ln(x)+B$.
\[y_{p1}'=\frac{A}{x}\]
\[y_{p1}''=-\frac{A}{x^2}\]
Substitute into the original equation:
\[-\frac{A}{x^2}-2\left(\frac{A}{x}\right)+3(A\ln(x)+B)=\ln(x)\]
\[-\frac{A}{x^2}-\frac{2A}{x}+3A\ln(x)+3B=\ln(x)\]
\[A=\frac{1}{3}\]
\[-\frac{A}{x^2}-\frac{2A}{x}+3B=0\]
\[B=0\]
\[y_{p1}=\frac{1}{3}\ln(x)\]
Assume the particular solution for $\ln(x)$ is $y_{p2}=C\tan(x)$.
\[y_{p2}'=C\sec^2(x)\]
\[y_{p2}''=2C\sec^2(x)\tan(x)\]
Substitute into the original equation:
\[2C\sec^2(x)\tan(x)-2(C\sec^2(x))+3(C\tan(x))=\tan(x)\]
\[2C\sec^2(x)\tan(x)-2C\sec^2(x)+3C\tan(x)=\tan(x)\]
\[(2C-2C+3C)\tan(x)=\tan(x)\]
\[C=\frac{1}{3}\]
\[y_{p2}=\frac{1}{3}\tan(x)\]
Construct the general solution:
\[y=y_h+y_{p1}+y_{p2}=e^x(C\cos(\sqrt{2}x)+D\sin(\sqrt{2}x))+\frac{1}{3}\ln(x)+\frac{1}{3}\tan(x)\]




Sure! Let’s state the **general theorem for solving first-order linear ODEs** using the integrating factor method, clearly and rigorously.

---

### Theorem (Integrating Factor Method for Linear ODEs)

Consider a first-order linear ODE of the form:

$$
y'(x) + P(x) y(x) = Q(x), \quad x \in I
$$

where $P(x)$ and $Q(x)$ are continuous on an interval $I$.

**Then:**

1. Define the **integrating factor**

$$
\mu(x) = e^{\int P(x)\, dx}
$$

2. Multiply both sides of the ODE by $\mu(x)$:

$$
\mu(x) y' + \mu(x) P(x) y = \mu(x) Q(x)
$$

3. The left-hand side becomes an exact derivative:

$$
\frac{d}{dx}[\mu(x) y(x)] = \mu(x) Q(x)
$$

4. Integrate both sides:

$$
\mu(x) y(x) = \int \mu(x) Q(x) \, dx + C
$$

5. Solve for $y(x)$:

$$
\boxed{y(x) = \frac{1}{\mu(x)} \left[ \int \mu(x) Q(x) \, dx + C \right]}
$$

---

### Notes:

* This method **works for any first-order linear ODE**, as long as $P$ and $Q$ are continuous.
* $C$ is an arbitrary constant determined by initial/boundary conditions.
* The key is that the integrating factor $\mu(x)$ **turns the left-hand side into a derivative** of $\mu(x)y$, which can then be directly integrated.

---

If you want, I can **also show a “visual intuition” of why the integrating factor works**, which often makes the theorem much easier to remember.

Do you want me to do that?



General solution
Singular solution 



critical points asymptomatically stable, unstable, or semi-stable