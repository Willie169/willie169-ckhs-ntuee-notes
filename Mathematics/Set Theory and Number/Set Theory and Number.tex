\documentclass[a4paper,12pt]{report}
\setcounter{secnumdepth}{5}
\setcounter{tocdepth}{3}
\input{/usr/share/LaTeX-ToolKit/template.tex}
\begin{document}
\title{Set Theory and Number}
\author{沈威宇}
\date{\temtoday}
\titletocdoc
\ch{Set Theory and Number}
\sct{Set Theory (集合論)}
\ssc{Basic Notations}
\sssc{Set (集合/集)}
A set, sometimes called a collection or family, is a collection of different things. Theses things are called elements (元素) or members (成員) if the set.

If $a$ is an element of a set $S$, we say that $a$ belongs to $S$ or is in $S$, denoted as $a\in S$; if $b$ is not an element of a set $S$, we say that $b$ does not belong to $S$ or is not in $S$, denoted as $b\notin S$.
\sssc{Specifying a set}
A set may be specified either by listing its elements or by a property that characterizes its elements.

\tb{Roster notation or enumeration notation} specifies a set by listing its elements between braces $\{\,\}$, separated by commas $,$, e.g. $\{\,\}$, $\{x\}$, and $\{x,y\}$. When there is a clear pattern for generating all set elements, one can use ellipses $\ldots$ for abbreviating the notation.

\tb{Set-builder notation} specifies a set as being the set of all elements that satisfy some logical formula. More precisely, if $P(x)$ is a logical formula depending on a variable ⁠$x$, which evaluates to true or false depending on the value of $x$, then
\[\{x\mid P(x)\}\]
or
\[\{x\colon P(x)\}\]
denotes the set of all $x$ of which $P(x)$ is true, in which the vertical bar $\mid$ or the colon $\colon$ is read as "such that", and the the whole formula can be read as "the set of all $x$ such that $P(x)$ is true".
\sssc{Empty set (空集合) and singleton (單元素集合)}
A set with no element in it is called the empty set, denoted as $\varnothing$. A set with one element in it is called a singleton.
\subsubsection{Subset (子集)}
\[A\subseteq B\iff (x\in A\implies x\in B)\]
\subsubsection{Proper subset (真子集)}
\[A\subset B\iff (A\subseteq B\land A\neq B)\]
\sssc{Superset (父集/母集/超集)}
\[B\supseteq A\iff (x\in A\implies x\in B)\]
\subsubsection{Intersection (交集)}
\[A\cap B\coloneq\{x\mid x\in A \land x\in B\}\]
\[\bigcap_{i=1}^n A_i\coloneq\left\{x\middle | \bigwedge_{i=1}^n x\in A_i\right\}\]
\subsubsection{Union (聯集)}
\[A\cup B\coloneq\{x\mid x\in A \lor x\in B\}\]
\[\bigcup_{i=1}^n A_i\coloneq\left\{x\middle | \bigvee_{i=1}^n x\in A_i\right\}\]
\subsubsection{Complement (差集)}
\[A\setminus B\coloneq\{x\mid x\in A \land x\notin B\}\]
\subsubsection{Cartesian product (笛卡爾積)}
\[\prod_{i=1}^nA_i=\{(a_1,a_2,\ldots a_n)\mid a_i\in A_i\},\]
and its elements are called ordered $n$-tuple, in which a $2$-tuple is also called a pair and $3$-tuple is also called a triple.
\subsubsection{Power set (冪集合)}
\[2^A\coloneq\{B\mid B\subseteq A\}\]
\sssc{Set scalar arithmetic operation}
\begin{itemize}
\item If $\forall a\in A$, $sa$ is defined, $sA\coloneq\{sa:\,a\in A\}$.
\item If $\forall a\in A$, $a+v$ is defined, $A+v\coloneq\{a+v:\,a\in A\}$.
\end{itemize}
\subsubsection{Kernel (核) of a set of sets}
The kernel of a set $\mathcal {B}\neq \varnothing$ of sets is defined to be:
\[\ker(\mathcal {B})\coloneq\bigcap_{B\in\mathcal {B}}B.\]
\subsection{Zermelo–Fraenkel Set Theory (策梅洛-弗蘭克爾集合論)}
Set theory, or more specifically, Zermelo–Fraenkel set theory, has been the standard way to provide rigorous foundations for all branches of mathematics since the first half of the 20th century. Zermelo–Fraenkel set theory with the axiom of choice included is abbreviated ZFC and denoted $\vdash_{ZFC}$; Zermelo–Fraenkel set theory with the axiom of choice excluded is abbreviated ZF and denoted $\vdash_{ZF}$; sometimes, either of them is denoted $\vdash$.
\subsubsection{Axiom of extensionality (外延公理)}
\[\forall x\forall y[\forall z(z\in x\Leftrightarrow z\in y)\Rightarrow x=y]\]
\subsubsection{Axiom of regularity (正則公理)}
\[\forall x\,(x\neq \varnothing \Rightarrow \exists y(y\in x\land y\cap x=\varnothing ))\]
\subsubsection{Axiom of separation (分類公理)/Axiom schema of specification (規範公理模式)}
Let $\varphi$ be any formula in the language of ZFC with all free variables among $x,z,w_{1},\ldots ,w_{n}$ so that $y$ is not free in 
$\varphi$. Then:
\[\forall z\forall w_{1}\forall w_{2}\ldots \forall w_{n}\exists y\forall x[x\in y\Leftrightarrow ((x\in z)\land \varphi (x,w_{1},w_{2},...,w_{n},z))]\]
This axiom can be used to prove the existence of the empty set, denoted as $\emptyset$ or $\varnothing$.

\tb{Axiom of empty set (空集公理)}:
\[\exists x , \forall y , (y \notin x) \]
The $\varnothing$ is defined as the $x$ above.
\subsubsection{Pairing Axiom (配對公理)}
\[ \forall x\forall y\exists z((x\in z)\land (y\in z))\]
\subsubsection{Union Axiom (聯集公理)}
\[ \forall {\mathcal {F}}\,\exists A\,\forall Y\,\forall x[(x\in Y\land Y\in {\mathcal {F}})\Rightarrow x\in A]\]
Although this formula doesn't directly assert the existence of $\cup \mathcal {F}$, the set $\cup \mathcal {F}$ can be constructed from $A$ in the above using the axiom schema of specification:
\[\cup \mathcal {F}=\{x\in A\mid\exists Y(x\in Y\land Y\in {\mathcal {F}})\}\]
\subsubsection{Axiom schema of replacement (替代公理模式)}
Let $\varphi$ be any formula in the language of ZFC with all free variables among $x,z,w_{1},\ldots ,w_{n}$ so that $B$ is not free in 
$\varphi$. Then:
\[ \forall A\forall w_{1}\forall w_{2}\ldots \forall w_{n}{\bigl [}\forall x(x\in A\Rightarrow \exists !y\,\varphi )\Rightarrow \exists B\ \forall x{\bigl (}x\in A\Rightarrow \exists y(y\in B\land \varphi ){\bigr )}{\bigr ]}\]
\subsubsection{Axiom of Infinity (無窮公理)}
\[ \exists X\left[\exists e(\forall z\,\neg (z\in e)\land e\in X)\land \forall y(y\in X\Rightarrow S(y)\in X)\right]\]
\subsubsection{Power Set Axiom (冪集公理)}
\[\forall A \exists P(A) \forall x (x \in P(A) \leftrightarrow x \subseteq A)\]
The $P(A)$ above is called power set (冪集) and denoted as $2^A$.
\subsubsection{Axiom of Choice (選擇公理)/Axiom of Well-ordering (良序公理)}
\[ \forall X\left[\varnothing \notin X\implies \exists f\colon X\rightarrow \bigcup _{A\in X}A\quad \forall A\in X\,(f(A)\in A)\right]\]
\ssc{Associativity, commutativity, and distributivity of intersection and union}
\begin{itemize}
\item Intersection has commutativity.
\item Union has commutativity.
\item Intersection has associativity and distributivity over intersection.
\item Union has associativity and distributivity over union.
\item Intersection has distributivity over union.
\item Union has distributivity over intersection.
\item Note: $(A\cap B)\cup C$ is not necessarily equal to $A\cap (B\cup C)$
\end{itemize}
\subsection{Russell's paradox or Russell's antinomy (羅素悖論)}
According to the unrestricted comprehension principle, for any sufficiently well-defined property, there is the set of all and only the objects that have that property. Russell's paradox states that, let
\[R\coloneq\{x\mid x\not \in x\}\]
then
\[R\in R\iff R\not \in R.\]

\textit{Statement.} $\vdash_{ZFC}\,\nexists U = \{z \mid z\text{\ is a set\ }\}$.
\begin{proof}\mbox{}\\
Assume that $\exists U = \{z \mid z\text{ is a set}\}$. Let $A=\{x\in U \mid x\notin x\}$. $A \in A\iff (A\in U \land A\notin A)$, but $A\in A \iff \neg A \notin A$, so $A\notin A$, so $\neg(A \in U \land A \notin A)$, so $A \notin U$. $A$ is a set, so $A\in U$. $\Rightarrow\Leftarrow$.
\end{proof}
\ssc{De Morgan's Law (笛摩根定律)}
\sssc{Universe (宇集)}
The universe is now often defined as "when the sets under discussion are all subsets of a given set, the given set is called the universe", denoted as $U$.
\sssc{Complement (補集/餘集)}
If a set $A$ is a subset of a given universe $U$, then the complement of $A$ given $U$ is defined as $U\setminus A$, denoted as $A'$, $\overline{A}$, or $A^C$.
\sssc{Complement laws}
\[(A')'=A\]
\[A\setminus B = A\cap B'\]
\[A\subseteq B \iff B' \subseteq A'\]
\sssc{De Morgan's Law}
\[(A\cup B)'=A'\cap B'\]
\[(A\cap B)'=A'\cup B'\]
\ssc{Almost all (幾乎所有)}
In mathematics, the term "almost all" means "all but a negligible quantity". More precisely, if $X$ is a set, "almost all elements of $X$" means "all elements of $X$ but those in a negligible subset of $X$". The meaning of "negligible" depends on the mathematical context; for instance, it can mean finite, countable, or null.
\ssc{Projection maps}
The $k$-th projection map for a Cartesian product of $n$ sets:
\[\prod_{i=1}^nX_i\]
is the function $\pi_k$
\[\pi_k\colon\prod_{i=1}^nX_i\to X_k;\;\pi_k(x_1,x_2,\ldots x_n)=x_k.\]
\ssc{Cover or covering (覆蓋)}
A cover or covering $C$ of a set $X$ is a family of subsets of $X$ such that:
\[\bigcup_{U\in C}U\supseteq X.\]
We say $C$ covers $X$.

A subset of a cover $C$ of $X$ that is also a cover of $X$ is a subcover of $C$.
\sct{Ordinal and Cardinal}
\ssc{Ordinal number or ordinal (序數)}
\sssc{Ordinal number or ordinal}
The ordinal number or ordinal is a measurement of the order of an element of a strict well ordered set.

The ordinal numbers is defined to ordered by inclusion to form a strictly well ordered set; the ordinal $0$ is defined as $\varnothing$; and each ordinal $>0$ is defined as the set of all smaller ordinals, that is, for an ordinal $a>0$:
\[a=\{b\mid b\tx{\ is an ordinal and\ }b<a\}.\]
This definition is known as the von Neumann definition of ordinals.

The ordinal number $\omega$, called the first infinite ordinal, is defined to be $\mathbb{N}_0$ in von Neumann definition of ordinals.
\sssc{Successor ordinal}
If an ordinal has a maximum $a$, then it is the next ordinal after $a$, then it is called a successor ordinal (of $a$), written $a+1$.
\sssc{Limit ordinal}
A nonzero ordinal that is not a successor is called a limit ordinal.
\subsection{Cardinality (勢 or 計數) and Cardinal number or cardinal (基數 or 量)}
\sssc{Cardinality and Cardinal number or cardinal}
The cardinal number or cardinal is a measurement of the cardinality (size) of a set $A$, denoted as $|A|$ or $n(A)$.

Two sets are said to be equinumerous or have the same cardinality if there exists a bijection between them.

The cardinality of a finite set is the number of its elements. A cardinal that is not finite is called an infinite cardinal.

Assuming the axiom of choice, the cardinality of a set $X$ is defined to be the least ordinal number $\alpha$ such that there is a bijection between $X$ and $\alpha$. This definition is known as the von Neumann cardinal assignment.
\sssc{Aleph numbers}
Aleph numbers, a system to describe infinite cardinals, denoted as $\aleph_0, \aleph_1, \aleph_2, \ldots \aleph_\omega, \aleph_{\omega+1}, \ldots$, in which the subscripts are ordinal numbers, are defined with:
\bit
\item $\aleph_0$ is defined as $\omega$;
\item for any successor ordinal $n$, $\aleph_n$ is defined inductively as:
\[\aleph_n=\min\{b\mid b\tx{\ is an cardinal and\ }b>\aleph_{n-1}\};\]
\item for any limit ordinal $\lambda$, $\aleph_{\lambda}$ is defined as:
\[\aleph_{\lambda}=\sup\{\aleph_{\alpha}\mid\alpha<\lambda}.\]
\eit
\sssc{Beth numbers}
Beth numbers, a system to describe infinite cardinals, denoted as $\beth_0, \beth_1, \beth_2, \ldots \beth_\omega, \beth_{\omega+1}, \ldots$, in which the subscripts are ordinal numbers, are defined with:
\bit
\item $\beth_0$ is defined as $\aleph_0$;
\item for any successor ordinal $n$, $\beth_n$ is defined inductively as:
\[\beth_n=2^{\aleph_{n-1}};\]
\item for any limit ordinal $\lambda$, $\aleph_{\lambda}$ is defined as:
\[\aleph_{\lambda}=\sup\{\beth_{\alpha}\mid\alpha<\lambda}.\]
\eit
\sssc{Countable infinity (可數無限)}
The cardinality of $\mathbb{N}_0$, called the cardinality of countable infinity (可數無限勢), is $\aleph_0$. A set with the same cardinality of $\mathbb{N}$ is called to be countable infinite (可數無限的).
\sssc{Uncountable infinity (不可數無限)}
The cardinality of $\mathbb{R}$, called the cardinality of continuity (連續勢), is $2^{\aleph_0}=\beth_1$, also denoted as $\mathfrak{c}$. A set with the same cardinality of $\mathbb{R}$ is called to be uncountable infinite (不可數無限的).
\sssc{Continuum hypothesis (CH)}
The continuum hypothesis (CH) states that, there is no set whose cardinality is strictly between that of the integers and the real numbers, that is, $2^{\aleph_0}=\aleph_1$.

CH is independent from ZFC, that is, proving or disproving the CH within ZFC is impossible.
\sssc{Generalized continuum hypothesis (GCH) or Cantor's aleph hypothesis}
The generalized continuum hypothesis (GCH) or Cantor's aleph hypothesis states that, if a set's cardinality lies between that of an infinite set $S$ and that of the power set $P(S)$ of $S$, then it has the same cardinality as either $S$ or $P(s)$, that is, for any infinite cardinal $\lambda$, there is no cardinal $\kappa$ such that $\lambda <\kappa <2^{\lambda }$, that is, $\aleph _{\alpha +1}=2^{\aleph _{\alpha }}$ for every ordinal $\alpha$.

GCH is independent from ZFC, that is, proving or disproving the GCH within ZFC is impossible.
\sct{Number Systems or Number Sets}
\ssc{Introduction}
A number is a mathematical object used to count, measure, and label.

A numeral system is a set of symbols, called numerals, and the rules for using them to represent numbers. Numerals can be used for counting (as with cardinal numbers), for ordering (as with ordinal numbers), labels (as with telephone numbers), and for codes (as with ISBNs). In common usage, a numeral is not clearly distinguished from the number that it represents.

Numbers can be classified into sets, called number sets or number systems, such as the natural numbers and the real numbers.
\ssc{Natural numbers (自然數)}
The natural numbers $\mathbb{N}$, aka counting numbers or whole numbers, are defined as positive integers $\mathbb{N}^*$, $\mathbb{N}_1$, or $\mathbb{N}^+$, that is, 1, 2, 3, and so on. Some sources define it as non-negative integers $\mathbb{N}^0$ or $\mathbb{N}_0$, that is, 0, 1, 2, 3, and so on.
\sssc{Second-order Peano Axioms (皮亞諾公理), Dedekind–Peano axioms, or Peano postulates (皮亞諾公設)}
A formal construction of the natural numbers. Below, we define $\bbN_0$; the definition of $\bbN_1$ can be constructed by simply replace $0$ with $1$ in the following axioms.
\bit
\item $0$ is a natural number.
\item Every natural number $n$ has a successor $S(n)$ which is also a natural number, that is, the natural numbers are closed under $S$.
\item $0$ is not a successor of any natural numbers, that is,
\[\forall n\in\bbN_0\colon S(n)\neq 0.\]
\item $S$ is an injection, that is,
\[\forall m,n\in\bbN_0\colon S(m)=S(n)\implies m=n\]
\eit
\item The (second-order) induction axiom: If $K$ is a set such that, $0$ is in $K$, and, for every natural number $n$, $n\in K\implies S(n)\in K$, then $\bbN_0\subseteq K$.
\eit
\sssc{Peano Arithmetic (皮亞諾算術)}
The axiomatization of arithmetic provided by Peano axioms.
\bit
\item \tb{Addition}: Addition is a function $\bbN_0^{\pht{0}2}\to\bbN_0$ defined recursively as
\[a+0=a,\]
\[a+S(b)=S(a+b).\]
\item \tb{Multiplication}: Multiplication is a function $\bbN_0^{\pht{0}2}\to\bbN_0$ defined given addition recursively as
\[a\cdot 0=0,\]
\[a\cdot S(b)=a+(a\cdot b).\]
\item \tb{Preservation of order under addition and multiplication}: $\mathbb{N}$ is a totally ordered set, that is, it is equipped with a transitive relation $\leq$ that is strongly connected and antisymmetric.
\bit
\item Preservation of order under addition:
\[\forall x,y,z\in\bbN_0\colon x\leq y\implies x+z\leq y+z.\]
\item Preservation of order under multiplication:
\[\forall x,y\in\bbN_0\colon 0\leq x\land 0\leq y\implies 0\leq xy.\]
\eit
\item \tb{Strong order}: The strong order $<$ over $\bbN_0$ is defined as:
\[\forall x,y\in\bbN_0\colon x<y\iff\qty(x\leq y\land x\neq y).\]
\eit
\sssc{First-order Peano Axioms, Dedekind–Peano axioms, or Peano postulates}
A weaker first-order theory can be obtained by explicitly adding the addition and multiplication operation and replacing the second-order induction axiom with a first-order axiom schema.
\bit
\item $0$ is a natural number.
\item Every natural number $n$ has a successor $S(n)$ which is also a natural number, that is, the natural numbers are closed under $S$.
\item $0$ is not a successor of any natural numbers, that is,
\[\forall n\in\bbN_0\colon S(n)\neq 0.\]
\item $S$ is an injection, that is,
\[\forall m,n\in\bbN_0\colon S(m)=S(n)\implies m=n\]
\item \tb{Addition}: Addition is a function $\bbN_0^{\pht{0}2}\to\bbN_0$ defined recursively as
\[a+0=a,\]
\[a+S(b)=S(a+b).\]
\item \tb{Multiplication}: Multiplication is a function $\bbN_0^{\pht{0}2}\to\bbN_0$ defined given addition recursively as
\[a\cdot 0=0,\]
\[a\cdot S(b)=a+(a\cdot b).\]
\item \tb{Preservation of order under addition and multiplication}: $\mathbb{N}$ is a totally ordered set, that is, it is equipped with a transitive relation $\leq$ that is strongly connected and antisymmetric.
\bit
\item Preservation of order under addition:
\[\forall x,y,z\in\bbN_0\colon x\leq y\implies x+z\leq y+z.\]
\item Preservation of order under multiplication:
\[\forall x,y\in\bbN_0\colon 0\leq x\land 0\leq y\implies 0\leq xy.\]
\eit
\item \tb{The first-order induction axiom (schema)}: For any formula $\varphi\qty(x,\bar{y})$ where $\bar{y}$ stands for $y_1,y_2,\ldots,y_k$ in the language of Peano arithmetic (i.e. addition and multiplication),
\[\forall\bar{y}\colon\qty(\varphi\qty(0,\bar{y})\land\forall x\colon\varphi\qty(x,\bar{y})\implies\varphi\qty(S(x),\bar{y}))\implies\forall x\colon\varphi\qty(x,\bar{y})).\]
\eit
\ssc{Integers (整數)}
Below, we will construct the set of integers (aka the integers) $\mathbb{Z}$.
\sssc{Equivalence classes of ordered pairs of natural numbers}
\bit
\item \tb{Equivalence relation}: Define an equivalence relation $\sim$ on $\bbN_0^{\pht{0}2}$ as
\[\forall(a,b),(c,d)\in\bbN_0^{\pht{0}2}\colon(a,b)\sim(c,d)\iff a+d=b+c.\]
An equivalence class of $(a,b)$ is
\[[(a,b)]=\qty\{(c,d)\in\bbN_0^{\pht{0}2}\middle|(a,b)\sim(c,d)\}.\]
The integers $\bbZ$ is defined as the quotient set of $\bbN_0^{\pht{0}2}$ by the equivalence relation $\sim$.
\item \tb{Addition}: Addition is a function $\bbZ^2\to\bbZ$ defined as
\[[(a,b)]+[(c,d)]\coloneq[(a+c,b+d)].\]
\item \tb{Multiplication}: Multiplication is a function $\bbZ^2\to\bbZ$ defined as
\[[(a,b)]\cdot [(c,d)]\coloneq[(ac+bd,ad+bc)].\]
\item Negation (or additive inverse) is a function $\bbZ\to\bbZ$ defined as
\[-[(a,b)]\coloneq[(b,a)].\]
\eit
\sssc{Subtraction}
Subtraction is a function $\bbZ^2\to\bbZ$ defined as:
\[[(a,b)]-[(c,d)]\coloneq[(a+d,b+c)].\]
\sssc{Order}
The standard weak order $\leq$ over $\bbZ$ is defined given order $\leq$ over $\bbN_0$ as:
\[[(a,b)]\leq[(c,d)]\iff a+d\leq b+c.\]
The standard strong order $<$ over $\bbZ$ is defined given order $<$ over $\bbN_0$ as:
\[[(a,b)]<[(c,d)]\iff a+d<b+c.\]
\sssc{Notation}
$[a,b]\in\bbZ$ is denoted by
\[\begin{cases}
a-b,\quad&b\leq a\\
-(b-a),\quad&a<b
\end{cases}.\]
\ssc{Rational numbers (有理數)}
Below, we will construct the set of rational numbers (aka the rationals or the rational numbers) $\mathbb{Q}$.
\sssc{Equivalence classes of ordered pairs of integers}
\bit
\item \tb{Equivalence relation}: Define an equivalence relation $\sim$ on $\bbZ\times\qty(\bbZ\setminus\{0\})$ as
\[\forall(a,b),(c,d)\in\bbZ\times\qty(\bbZ\setminus\{0\})\colon(a,b)\sim(c,d)\iff ad=bc.\]
An equivalence class of $(a,b)$ is
\[[(a,b)]=\qty\{(c,d)\in\bbZ\times\qty(\bbZ\setminus\{0\})\middle|(a,b)\sim(c,d)\}.\]
The rationals $\bbQ$ is defined as the quotient set of $\bbZ\times\qty(\bbZ\setminus\{0\})$ by the equivalence relation $\sim$.
\item \tb{Addition}: Addition is a function $\bbQ^2\to\bbQ$ defined as
\[[(a,b)]+[(c,d)]\coloneq[(ad+bc,bd)].\]
\item \tb{Multiplication}: Multiplication is a function $\bbQ^2\to\bbQ$ defined as
\[[(a,b)]\cdot [(c,d)]\coloneq[(ac,bd)].\]
\eit
\sssc{Subtraction}
Subtraction is a function $\bbQ^2\to\bbQ$ defined as:
\[[(a,b)]-[(c,d)]\coloneq[(ad-bc,bd)].\]
\sssc{Order}
The standard weak order $\leq$ over $\bbQ$ is defined given order $\leq$ over $\bbZ$ as:
\[[(a,b)]\leq[(c,d)]\iff ad\leq bc.\]
The standard strong order $<$ over $\bbQ$ is defined given order $<$ over $\bbZ$ as:
\[[(a,b)]<[(c,d)]\iff ad<bc.\]
\sssc{Negation or additive inverse}
Negation (or additive inverse) is a function $\bbQ\to\bbQ$ defined as
\[-[(a,b)]\coloneq[(-a,b)].\]
\sssc{Notation}
$[a,b]\in\bbQ$ is denoted by $\frac{a}{b}$.




\ssc{Real numbers (實數)}
The set of real numbers (aka, the reals) $\mathbb{R}$ can be constructed axiomatically with
\bit
\item \tb{Addition and multiplication}: The set $\mathbb{R}$ is a field under addition and multiplication. In other words,
\bit
\item Associativity of addition and multiplication:
\[\forall x,y,z\in\mathbb{R}\colon x+(y+z)=(x+y)+z\land x\cdot(y\cdot z)=(x\cdot y)\cdot z.\]
\item Commutativity of addition and multiplication:
\[\forall x,y\in\mathbb{R}\colon x+y=y+x\land x\cdot y=y\cdot x.\]
\item Distributivity of multiplication over addition:
\[\forall x,y,z\in\mathbb{R}\colon x\cdot(y+z)=(x\cdot y)+(x\cdot z).\]
\item Existence of additive identity:
\[\forall x\in\mathbb{R}\colon x+0=x.\]
\item Existence of multiplicative identity:
\[0\neq 1,\]
\[\forall x\in\mathbb{R}\colon x\times 1=x.\]
\item Existence of additive inverses:
\[\forall x\in\mathbb{R}\colon\exists(-x)\in\mathbb{R}\text{\ s.t.\ }x+(-x)=0.\]
\item Existence of multiplicative inverses:
\[\forall x\in\mathbb{R}\land x\neq 0\colon\exists x^{-1}\in\mathbb{R}\text{\ s.t.\ }x\cdot x^{-1}=1.\]
\eit
\item \tb{Total order}: The field $\mathbb{R}$ is ordered, meaning that there is a total order $\leq$ such that for all real numbers $x$, $y$ and $z$:
\bit
\item Preservation of order under addition:
\[\forall x,y,z\in\mathbb{R}\colon x\leq y\implies x+z\leq y+z.\]
\item Preservation of order under multiplication:
\[\forall x,y\in\mathbb{R}\colon 0\leq x\land 0\leq y\implies 0\leq xy.\]
\item The order is Dedekind-complete, meaning that every nonempty subset $S$ of $\mathbb{R}$ with an upper bound in $\mathbb{R}$ has a least upper bound (aka, supremum) in $\mathbb{R}$. This property applies to the real numbers but not to the rational numbers. For example, $\{x\in\mathbb{Q}\colon x^2<2\}$ has a rational upper bound (e.g., 1.42), but no least rational upper bound, because $\sqrt{2}$ is not rational.
\eit\eit
\ssc{Irrational numbers (無理數)}
\[\mathbb{R}\setminus\mathbb{Q}.\]
\ssc{Complex numbers (複數)}
\[\mathbb{C}=\{a+bi\mid a,b\in\mathbb{R}\},\]
in which $i=\sqrt{-1}$.
\sct{Positional Systems, Positional Numeral System, Place-value Notations, or Place-value}
\ssc{Definition}
Positional systems, positional numeral system, place-value notations, or place-value, usually denoted as extension to any base of the Hindu–Arabic numeral system, are classified by their base or radix, which is the number of symbols called digits used by the system.

Let the base (aka radix) $b\in\mathbb{N}_{>1}$. A base-$b$ (aka radix-$b$) number
\[N=\qty((-1)^m\ldots d_2d_1d_0.d_{-1}d_{-2}\ldots)_b,\quad m\in\{0,1\}\land\forall i\colon d_i\in\mathbb{N}_0\land d_i<b\]
represents the value
\[N =(-1)^m\sum_id_ib^i.\]

We can only represent a base in another base or in decimal (base-10), and we have to specify the base you use if not in decimal; otherwise, every base would be base-10.

To indicate the base of a base-$b$ number, we append subscript $_b$ after it. We also use prefix \verb|0b| (or \verb|0B|) to represent base-2 numbers, aka binary numbers, prefix \verb|0| (or \verb|0o|, \verb|0O|) to represent base-8 numbers, aka octal numbers, prefix \verb|0x| (or \verb|0X|) to represent base-16 numbers, aka hexadecimal number, and no prefix (or \verb|0d|, \verb|0D|) to represent base-10 numbers, aka decimal numbers.

Usually, when the radix $b>10_{10}$, $A$ (or $a$) is used to represent the digit $9+1$; when the radix $b>11_{10}$, $B$ (or $b$) is used to represent the digit $9+2$; and so on.
\ssc{Conversion between bases}
\sssc{General conversion algorithm}
We want to convert a number $N$ in base $x$ with $x\in\mathbb{N}_{>1}$,
\[N=((-1)^n\ldots d_2d_1d_0.d_{-1}d_{-2}\ldots)_x,\quad m\in\{0,1\},\]
into base $y\neq x$ (written in base $x$) with $y\in\mathbb{N}_{>1}$.
\ben
\item First, we take $I_1=(\ldots d_2d_1d_0)_x$ and repeatedly:
\ben
\item take $r_i=I_i\mod y$, convert it to base $y$, of which the result must be a single digit, and
\item take $I_{i+1}=\frac{I-r_i}{y}$,
\een
in the $i$th (start from $1$st) time, until the first time $I_{i+1}=0$, and let that time be the $k$th time.
\item Second, we take $J_1=(0.d^{-1}d^{-2}\ldots)_x$ and repeatedly
\ben
\item take $s_j=\lfloor J_j\cdot y\rfloor$, which may be more than $1$ digits if $x>y$ and is later treated directly as in base $y$, and
\item take $J_{j+1}=J_j\cdot y-s_j$,
\een
in the $j$th (start from $1$st) time, until the first time $J_{j+1}=0$, and let that time be the $l$th time.
\item Then, $N$ is
\[(-1)^n\qty(\sum_{i=1}^kr_i\times y^{i-1}+\sum_{j=1}^ls_j\times y^{-j}),\]
and $N$ in base $y$ can be obtained by first taking the integer part of the magnitude of it:
\[(r_kr_{k-1}\ldots r_2r_1)_y\]
and then obtaining the fractional part of the magnitude of it using vertical addition where the $i$th row is $s_i$ with the units digit of $s_i$ in the $i$th digit to the right of the radix point, for each $i$, where carry propagation is applied to ensure all digits are valid in base $y$, i.e. $<y$.
\een
This conversion algorithm is $O(m\log m)$, where $m$ is the total number of digits.

To convert an expression that is a radix-independent fuction $f$ of a vector of numbers $(p_1,p_2,\ldots)$, we convert each $p_i$ to the wanted base, let it be $q_i$, then the wanted result is $f(q_1,q_2,\ldots)$.
\sssc{Power-base conversion algorithm}
To convert a base-$b$ number into base $b^n$ with $b,n\in\mathbb{N}_{>1}$, we group the digits to the left and right of the radix point into blocks of $n$ digits starting from the digits closest to the radix point and convert each block to one digit in base $b^n$. This algorithm method is $O(m)$, where $m$ is the total number of digits.
\sssc{Root-base conversion algorithm}
To convert a base-$b^n$ number in into base $b$ with $b,n\in\mathbb{N}_{>1}$, we convert each digit to $n$ digits in base $b$ and concatenate them. This algorithm method is $O(m)$, where $m$ is the total number of digits.
\sssc{Product and quotient-base conversion algorithm}
We want to convert a number in base $x$ with $x\in\mathbb{N}_{>1}$,
\[N=((-1)^m\ldots d_2d_1d_0.d_{-1}d_{-2}\ldots)_x,\quad m\in\{0,1\},\]
into base $y=ax$ (written in base $x$) with $y\in\mathbb{N}_{>1}$.

We take $D_i=d_i\cdot a^{-i}$, which is later treated directly as in base $y$, for each integer $i$.

Then, $N$ is
\[(-1)^m\sum_iD_i\times y^i,\]
and $N$ in base $y$ can be obtained by vertical addition where the $i$th row is $D_i$ with the units digit of $D_i$ in the $(i+1)$th digit to the left of the radix point, for each $i$, where carry propagation is applied to ensure all digits are valid in base $y$, i.e. $<y$.

This conversion algorithm is $O(m)$, where $m$ is the total number of digits.
\sssc{Criterion of finite expansion after base conversion}
A finite rational number with the fractional part of the magnitude of it expressed in fraction in base $x$ with $x\in\mathbb{N}_{>1}$ being $D_x$ can be expressed in finite digits in radix point in base $y\neq x$ with $y\in\mathbb{N}_{>1}$ iff the reduced denominator (denominator cancelled the greatest common factor with the numerator) of $D_x$ has no prime factor that is not a prime factor of $y$.
\subsection{Binary}
A binary digit is called a bit, which is either $0$ or $1$. Binary arithmetic is the same as decimals, except that "invert" or "flip" means converting $0$ to $1$ and $1$ to $0$, and "complement" means inverting all bits.

The most significant bit (MSB) or most significant digit is the bit with the highest value place in a binary number and is the leftmost bit in standard binary notation; the least significant bit (LSB) or least significant digit is the bit with the lowest value place in a binary number and is the rightmost bit in standard binary notation.

$4$ bits is called a nibble; $8$ bits is called a byte.

A binary prefix is a unit prefix that indicates a multiple of a unit of measurement by an integer power of two. They are most often used in information technology as multipliers of bit and byte, in which the short prefixes are prefixed before b (representing bits) or B (representing bytes), and the long prefixes are prefixed before bits or bytes.
\begin{longtable}[c]{|c|c|c|c|c|}
\hline
Value & IEC (short) & IEC (long) & JEDEC (short) & JEDEC (long)\\\hline
$1024$ & Ki & kibi & K & kilo\\\hline
$1024^2$ & Mi & mebi & M & mega\\\hline
$1024^3$ & Gi & gibi & G & giga\\\hline
$1024^4$ & Ti & tebi & T & tera\\\hline
$1024^5$ & Pi & pebi & —\\\hline
$1024^6$ & Ei & exbi & —\\\hline
$1024^7$ & Zi & zebi & —\\\hline
$1024^8$ & Yi & yobi & —\\\hline
$1024^9$ & Ri & robi & —\\\hline
$1024^10$ & Qi & quebi & —\\\hline
\end{longtable}\FB
\end{document}
