\documentclass[a4paper,12pt]{article}
\setcounter{secnumdepth}{5}
\setcounter{tocdepth}{3}
\input{/usr/share/latex-toolkit/template.tex}
\begin{document}
\title{Topology}
\author{沈威宇}
\date{\temtoday}
\titletocdoc
\section{Topology (拓樸學)}
\subsection{Topological space (拓樸空間)}
A topological space consists of a set $X$ and a topology $\mathcal{T}$ on it, denoted as \( (X, \mathcal{T}) \). Where the set \( X \) is the set of points in the space, and the topology \( \mathcal{T} \) is the set of subsets of \( X \) that satisfies:
\begin{enumerate}
\item \(\varnothing,X\in\mathcal{T} \)
\item Closed under arbitrary unions: \(\forall\mathcal{S}\subseteq\mathcal{T}:\,\bigcup_{A\in\mathcal{S}}A\in\mathcal{T}\)
\item Closed under finite intersection: \(\forall\mathcal{S}\subseteq\mathcal{T}\text{ s.t. }\abs{\mathcal{S}}<\infty:\,\bigcap_{A\in\mathcal{S}}A\in\mathcal{T}\)
\end{enumerate}
\subsection{Homeomorphism (同胚) and isomorphism (同構)}
Topological spaces $(X,\mathcal{T}_X)$ and $(Y,\mathcal{T}_Y)$ are called homeomorphic if there exists a mapping $f:\,X\rightarrow Y$ between them such that $f$ is bijective and continuous and $f^{-1}$ is continuous, written as \( X \cong Y \), and $f$ is called a homeomorphism between them.

In linear algebra, when $X$ and $Y$ are homeomorphic vector spaces and $f$ is a linear map, $X$ and $Y$ are also called isomorphic, and a homeomorphism between them is also called an isomorphism.
\subsection{Open set (開集)}
In a topological space \( (X, \mathcal{T}) \), an open subset \( O \subseteq X \) is defined to be
\[O\in \mathcal{T}.\]
\subsection{Open neighborhood (開鄰域)}
In a topological space $X$, a open neighborhood of a point $P\in X$ is any open subset $O\subseteq X$ such that $P\in O$.
\subsection{Neighborhood (鄰域)}
In a topological space, a subset $U$ is a neighborhood of a point $P$ if and only if there exists an open set $O$ such that $P\in O\subseteq U$.
\ssc{Subspace topology}
The subspace topology of a topological space $(X,\mathcal{T})$ in a subset $I$ of $X$ is defined as the set 
\[\mathcal{I}=\{U\cap I | U\in\mathcal{T}\}.\]
\subsection{Limit point (極限點), cluster point, or accumulation point (集積點)}
Let $S$ be a subset of a topological space $X$. A point $x$ in $X$ is a limit point, cluster point, or accumulation point of the set $S$ if every neighborhood of $x$ contains at least one point of $S$ different from $x$ itself.
\subsection{Closure (閉包)}
The closure of a subset $S$ of points in a topological space consists of all points in $S$ together with all limit points of $S$.
\subsection{Closed set (閉集)}
A subset $A$ of a topological space $(X,\mathcal{T})$ is called closed if its complement $X\setminus A\in\mathcal{T}$.
\subsection{Borel set (博雷爾集)}
A Borel set $B$ is any set in a topological space $(X,\mathcal{T})$ that can be formed from open sets through the operations of countable union, countable intersection, and relative complement, that is:
\[\begin{aligned}
B\in &\{\bigcup_{O\in\mathcal{S}}O:\,\mathcal{S}\subseteq\mathcal{T},\abs{\mathcal{S}}<\infty\}\\
& \cup \{\bigcap_{O\in\mathcal{S}}O:\,\mathcal{S}\subseteq\mathcal{T},\abs{\mathcal{S}}<\infty\}\\
& \cup \{O\setminus P:\,O,P\in\mathcal{T}\}
\end{aligned}\]
\subsection{Filter}
A filter on a set $X$ is a family $\mathcal{B}$ of subsets of $X$ such that: 
\begin{enumerate}
\item $X\in\mathcal{B}$.
\item $\varnothing\in\mathcal{B}$.
\item $A\in\mathcal{B}\land B\in\mathcal{B}\implies A\cap B\in\mathcal{B}$.
\item $A\subseteq B\subseteq X\land A\in\mathcal{A}\in\mathcal{B}\implies B\in\mathcal{B}$.
\end{enumerate}
\subsection{Base or basis (基)}
Given a topological space $(X,\mathcal{T})$, a base (or basis) for the topology $\mathcal{T}$ (also called a base for $X$ if the topology is understood) is a family $\mathcal{B}\subseteq\mathcal{T}$ of open sets such that every open set of the topology can be represented as the union of some subfamily of $\mathcal{B}$.

The topology generated by a base $\mathcal{B}$, generally denoted by $\tau(\mathcal{B})$ can be defined to be follows: A subset $O\subseteq X$ is to be declared as open, if for all $x\in O$, there exists some $B\in\mathcal{B}$ such that $x\in B\subseteq O$.
\subsection{Prefilter or filter base}
$\mathcal{B}$ is called a prefilter if its upward closure $\uparrow\mathcal{B}$ is a filter.
\subsection{Connected space (連通空間)}
A topological space is disconnected if it can be expressed as the union of two disjoint non-empty open sets. Otherwise, it is connected.
\subsection{Path-connected space (路徑連通空間)}
A space $X$ is path-connected if for any two points $x,y\in X$, there existes a continuous function $f\colon[0,1]\to X$ such that $f(0)=x$ and $f(y)=1$. Otherwise, it is not path-connected.

Path-connectedness implies connectedness. However, connectedness does not imply path-connectedness, a counterexample is the topologist' sine curve:
\[S = \left\{ (x, \sin(1/x)) \mid x \in (0,1] \right\} \cup \{(0,y) \mid -1 \leq y \leq 1\} \subset \mathbb{R}^2.\]
\subsection{Simply connected space or 1-connected space (單連通空間)}
A space is simply connected if it is path-connected and that for any two points $x,y\in X$ and any two continuous functions $f\colon[0,1]\to X$ and $g\colon[0,1]\to X$ such that $f(0)=g(0)=x$ and $f(1)=g(1)=y$, there exists a continuous function $F\colon[0,1]\times[0,1]\to X$ such that $F(x,0)=f(x)$ and $F(x,1)=g(x)$. Otherwise, it is not simply connected.
\subsection{Compact space (緊緻空間)}
A topological space $X$ is said to be compact if for every collection $C$ of open subsets of $X$ such that
\[X=\bigcup_{S\in C}S,\]
there is a finite subcollection $F\subseteq C$ such that
\[X=\bigcup _{S\in F}S.\]
\subsection{Compact (緊緻) subset}
A subset $K$ of a topological space $X$ is said to be compact if for every collection $C$ of open subsets of $X$ such that
\[K\subseteq\bigcup_{S\in C}S,\]
there is a finite subcollection $F\subseteq C$ such that
\[K\subseteq\bigcup _{S\in F}S.\]
\subsection{Locally finite cover}
Let $X$ be a topological space, and $\{U_i\mid i\in I\}$ be a cover of $X$. We say that $U$ is locally finite if for all $x\in X$, there exists $V\ni x$ that is open in $X$ such that the set $\{i\in I\mid V\cap U_i\neq\varnothing\}$ is finite.
\ssc{Interior (內部) of subset}
The interior of a subset $S$ of a topological space $X$, denoted by $\operatorname{int}_XS$, $\operatorname{int}S$, $\operatorname{int}_X(S)$, $\operatorname{int}(S)$, or $S^{\circ}$, is defined as
\[\operatorname{int}(S) = \{ x \in S \mid \exists \text{\ open\ } U\subeteq X \text{\ s.t.\ } x \in U \subseteq S \}.\]
\ssc{Dense (稠密) subset}
A subset $A$ of a topological space $X$ is said to be a dense subset of $X$ if for every $x\in X$ and every neighborhood $U$ of $x$, $U\cap A\neq\varnothing$.
\subsection{Hausdorff space, separated space or T2 space (郝斯多夫空間、分離空間或T2空間)}
Points $x$ and $y$ in a topological space $X$ can be separated by neighborhoods if there exists a neighborhood $U$ of $x$ and a neighborhood $V$ of $y$ such that $U$ and $V$ are disjointed, i.e., $U\cap V=\varnothing$.

$X$ is a Hausdorff space if any two distinct points in $X$ are separated by neighborhoods. This condition is the third separation axiom (after T0 and T1), which is why Hausdorff spaces are also called T2 spaces. The name separated space is also used.
\subsection{Metric space (度量空間或賦距空間)}
Metric space is an ordered pair $(M, d)$ where $M$ is a set and $d$ is a metric on M, i.e., a function $d:\,M\times M\to\mathbb{R}$ satisfying the following axioms for all points $x,y,z\in M$:
\begin{enumerate}
\item The distance from a point to itself is zero: $d(x,x)=0$.
\item (Positivity) The distance between two distinct points is always positive:$x\neq y\implies d(x,y)>0$.
\item (Symmetry) The distance from $x$ to $y$ is always the same as the distance from $y$ to $x$: $d(x,y)=d(y,x)$.
\item The triangle inequality: $d(x,z)\leq d(x,y)+d(y,z)$.
\end{enumerate}

For a normed space $(X,\|\cdot\|)$, unless otherwise specified, it is equipped with a metric $d$ defined as
\[d(x,y)=\|x-y\|,\quad \forall x,y\in X.\]
\subsection{Cauchy sequence (柯西序列)}
Given a metric space $(x,d)$, a sequence of elements of $X$:
\[x_1,x_2,x_3,\ldots\]
is Cauchy if for every positive real number $\varepsilon$ there is a positive integer $N$ such that for all positive integers $m,n>N$:
\[d\left(x_m,x_n\right)<\varepsilon.\]
\subsection{Complete metric space (完備度量空間) or Cauchy space (柯西空間)}
A metric space $(x,d)$ is complete if every Cauchy sequence of points in $X$ has a limit that is also in $X$.
\subsection{Open ball (開球)}
In a metric space $(X,d)$, given a point $a$ and radius $r$, the open ball $B(a)_{<r}$ is defined to be:
\[B(a)_{<r}:=\left\{p\in X:\, d(a,p)<r\right\}.\]
\subsection{Closed ball (閉球)}
In a metric space $(X,d)$, given a point $a$ and radius $r$, the closed ball $B(a)_{\leq r}$ is defined to be:
\[B(a)_{\leq r}:=\left\{p\in X:\, d(a,p)\leq r\right\}.\]
\subsection{Topological field (拓樸域)}
A topological field is a topological space, such that addition, multiplication, the maps $a\mapsto -a$, and $a\mapsto a^{-1}$ are continuous maps with respect to the topology of the space.
\ssc{Net (網)}
\sssc{Net (網)}
A net in $X$, denoted as $x_{\bullet }=\left(x_{a}\right)_{a\in A}$ or $\left\langle x_{a}\right\rangle _{a\in A}$, is a function of the form $x_{\bullet }\colon A\to X$ whose domain $A$ is some directed set, and whose values are $x_{\bullet }(a)=x_{a}$ in a topological space $X$. Elements of a net's domain are called its indices.
\sssc{Eventually or residually}
A net $x_{\bullet }=\left(x_{a}\right)_{a\in A}$ in $X$ is said to be eventually or residually in a set $S$ if there exists some $a\in A$ such that for every $b\in A$ with $b\geq a$, the point $x_b\in S$.
\sssc{Limit of nets}
A point $x$ in a topological space $X$ is called a limit point or limit of a net $x_{\bullet }$ in $X$ if, for every open neighborhood $U$ of $x$, the net $x_{\bullet }$ is eventually in $U$, also called that the net converges to/towards $x$, and denoted as $x_{\bullet}\to x$, $x_a\to x$, $\lim x_{\bullet}\to x$, $\lim_{a\in A}x_a\to x$, or $\lim_ax_a\to x$.

If $\lim x_{\bullet}\to x$ and this limit is unique (i.e. $\lim x_{\bullet}\to y$ only if $x=y$), then one writes $\lim x_{\bullet}=x$, $\lim_{a\in A}x_a=x$, or $\lim_ax_a=x$.
\sssc{Cofinally (共尾地) or frequently}
A net $x_{\bullet }=\left(x_{a}\right)_{a\in A}$ in $X$ is said to be cofinally or frequently in a set $S$ if for every $a\in A$ there exists $b\in A$ such that $b\geq a\land x_b\in S$.
\sssc{Cluster point, or accumulation point (集積點)}
A point $x$ in a topological space $X$ is called a cluster point or accumulation point of a net $x_{\bullet }$ in $X$ if, for every open neighborhood $U$ of $x$, the net $x_{\bullet }$ is cofinally in $U$, also called that the net clusters to/towards $x$.

A limit point of a net is necessary a cluster point of it.
\sssc{Subnet or Willard-subnet}
A net $s_{\bullet }=\left(s_{i}\right)_{i\in I}$ is called a subnet or 
{\displaystyle s_{\bullet }} is called a subnet or Willard-subnet of another net $x_{\bullet }=\left(x_{a}\right)_{a\in A}$ if there exists an order-preserving map $h\colon I\to A$ such that $h(I)$ is a cofinal subset of $A$ and 
\[s_{i}=x_{h(i)}\quad \forall i\in I.\]
\sssc{Ultranet or universal net}
A net $x_{\bullet }$ in $X$ is called an ultranet or a universal net if for every subset $S\subseteq X$, $x_{\bullet }$ is eventually in $S$ or $X\setminus S$.

A point is a limit point of a ultranet if and only if it a cluster point of it.
\sssc{In Hausdorff spaces}
A net in a Hausdorff space has at most one limit; if it has one, the limit point of it is the only cluster point of it.
\end{document}
