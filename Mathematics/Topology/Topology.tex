\documentclass[a4paper,12pt]{article}
\setcounter{secnumdepth}{5}
\setcounter{tocdepth}{3}
\input{/usr/share/latex-toolkit/template.tex}
\begin{document}
\title{Topology}
\author{沈威宇}
\date{\temtoday}
\titletocdoc
\section{Topology (拓樸學)}
\subsection{Topological space (拓樸空間)}
A topological space consists of a set $X$ and a topology $\mathcal{T}$ on it, denoted as \( (X, \mathcal{T}) \). Where the set \( X \) is the set of points in the space, and the topology \( \mathcal{T} \) is the set of subsets of \( X \) that satisfies:
\begin{enumerate}
\item \(\varnothing,X\in\mathcal{T} \)
\item Closed under arbitrary unions: \(\forall\mathcal{S}\subseteq\mathcal{T}:\,\bigcup_{A\in\mathcal{S}}A\in\mathcal{T}\)
\item Closed under finite intersection: \(\forall\mathcal{S}\subseteq\mathcal{T}\text{ s.t. }\abs{\mathcal{S}}<\infty:\,\bigcap_{A\in\mathcal{S}}A\in\mathcal{T}\)
\end{enumerate}
\subsection{Homomorphism (同胚)}
Topological spaces $(X,\mathcal{T}_X)$ and $(Y,\mathcal{T}_Y)$ are called homeomorphic if there exists a mapping $f:\,X\to Y$ between them such that $f$ is bijective and continuous and $f^{-1}$ is continuous, written as \( X \cong Y \), and $f$ is called a homomorphism between them.
\subsection{Open set (開集)}
In a topological space \( (X, \mathcal{T}) \), an open subset \( O \subseteq X \) is defined to be
\[O\in \mathcal{T}.\]
\subsection{Open neighborhood (開鄰域)}
An open neighborhood of a point $P$ in a topological space $X$ is any open subset $O\subseteq X$ such that $P\in O$.

An open neighborhood of a subset $V$ in a topological space $X$ is any open subset $O\subseteq X$ such that $V\subseteq O$.
\ssc{Interior (內部) of subset}
The interior of a subset $S$ of a topological space $X$, denoted by $\operatorname{int}_XS$, $\operatorname{int}S$, $\operatorname{int}_X(S)$, $\operatorname{int}(S)$, or $S^{\circ}$, is defined as
\[\operatorname{int}(S) = \{ x \in S \mid \exists \text{\ open\ } U\subeteq X \text{\ s.t.\ } x \in U \subseteq S \}.\]
\subsection{Neighborhood (鄰域)}
A subset $U$ is a neighborhood of a point $P$ if and only if there exists an open set $O$ such that $P\in O\subseteq U$.

A subset $U$ is a neighborhood of a subset $V$ if and only if $V\subseteq\operatorname{int}(U)$.
\ssc{Subspace topology}
The subspace topology of a topological space $(X,\mathcal{T})$ in a subset $I$ of $X$ is defined as the set 
\[\mathcal{I}=\{U\cap I | U\in\mathcal{T}\}.\]
\subsection{Limit point (極限點), cluster point, or accumulation point (集積點)}
Let $S$ be a subset of a topological space $X$. A point $x$ in $X$ is a limit point, cluster point, or accumulation point of the set $S$ if every neighborhood of $x$ contains at least one point of $S$ different from $x$ itself.
\subsection{Closure (閉包)}
The closure of a subset $S$ of points in a topological space consists of all points in $S$ together with all limit points of $S$.
\subsection{Closed set (閉集)}
A subset $A$ of a topological space $(X,\mathcal{T})$ is called closed if its complement $X\setminus A\in\mathcal{T}$.
\subsection{Borel set (博雷爾集)}
A Borel set $B$ is any set in a topological space $(X,\mathcal{T})$ that can be formed from open sets through the operations of countable union, countable intersection, and relative complement, that is:
\[\begin{aligned}
B\in &\{\bigcup_{O\in\mathcal{S}}O:\,\mathcal{S}\subseteq\mathcal{T},\abs{\mathcal{S}}<\infty\}\\
& \cup \{\bigcap_{O\in\mathcal{S}}O:\,\mathcal{S}\subseteq\mathcal{T},\abs{\mathcal{S}}<\infty\}\\
& \cup \{O\setminus P:\,O,P\in\mathcal{T}\}
\end{aligned}\]
\subsection{Filter}
A filter on a set $X$ is a family $\mathcal{B}$ of subsets of $X$ such that: 
\begin{enumerate}
\item $X\in\mathcal{B}$.
\item $\varnothing\in\mathcal{B}$.
\item $A\in\mathcal{B}\land B\in\mathcal{B}\implies A\cap B\in\mathcal{B}$.
\item $A\subseteq B\subseteq X\land A\in\mathcal{A}\in\mathcal{B}\implies B\in\mathcal{B}$.
\end{enumerate}
\subsection{Base or basis (基)}
Given a topological space $(X,\mathcal{T})$, a base (or basis) for the topology $\mathcal{T}$ (also called a base for $X$ if the topology is understood) is a family $\mathcal{B}\subseteq\mathcal{T}$ of open sets such that every open set of the topology can be represented as the union of some subfamily of $\mathcal{B}$.

The topology generated by a base $\mathcal{B}$, generally denoted by $\tau(\mathcal{B})$ can be defined to be follows: A subset $O\subseteq X$ is to be declared as open, if for all $x\in O$, there exists some $B\in\mathcal{B}$ such that $x\in B\subseteq O$.
\subsection{Prefilter or filter base}
$\mathcal{B}$ is called a prefilter if its upward closure $\uparrow\mathcal{B}$ is a filter.
\ssc{Quotient space (商空間)}
Let $X$ be a topological space and $\cong$ be an equivalence relation on it. The quotient space $Y$ of $X$ under $\cong$ is the quotient set of $X$ under $\cong$ equipped with the quotient topology, defined as $U\subseteq Y$ is open in the quotient topology on $Y$ if and only if $\{x\in X\mid[x]\in U\}$ is open in $X$ and $V\subseteq Y$ is closed in the quotient topology on $Y$ if and only if $\{x\in X\mid[x]\in V\}$ is closed in $X$.
\ssc{Type of topological spaces}
\sssc{Distinct}
Point $x$ and point $y$ in a topological space are distinct if and only if they are not the same point.
\sssc{Kolmogorov space (柯爾莫果洛夫空間) or T$_0$ space}
Two points in a topological space are topological distinguishable if and only if there exists an open set that contains one of them but not the other.

A topological space is called a Kolmogorov space or T$_0$ space if and only if any two distinct points in it are topological distinguishable. This condition is called the T$_0$ condition or the first separation theorem.
\sssc{Kolmogorov quotient (space)}
The Kolmogorov quotient (space) of a topological space is the quotient space of it under the equivalence relation topological indistinguishablity.
\sssc{Fréchet space (弗蘭歇空間), accessible space, or T$_1$ space}
Two points $x$ and $y$ in a topological space are separated if and only if there exists open sets $U,V$ such that $x\in U\land y\notin U\land y\in V\land x\notin V$.

A topological space is called a Fréchet space, accessible space, or T$_1$ space if and only if any two distinct points in it are separated. This condition is called the T$_1$ condition or the second separation theorem. A topological space is a $T_1$ space implies it is a $T_0$ space.
\sssc{Symmetric space or R$_0$ space}
A topological space is called a symmetric space or R$_0$ space if and only if any two topological distinguishable points in it are separated.
\sssc{Hausdorff space (郝斯多夫空間), separated space (分離空間), or T$_2$ space}
Points $x$ and $y$ in a topological space are separated by neighborhoods if and only if there exists a neighborhood $U$ of $x$ and a neighborhood $V$ of $y$ such that $U$ and $V$ are disjointed.

A topological space is called a Hausdorff space, separated space, or T$_2$ space if and only if any two distinct points in it are separated by neighborhoods. This condition is called the T$_2$ condition or the third separation theorem. A topological space is a $T_2$ space implies it is a $T_1$ space.
\sssc{Urysohn space or T\(_{2\text{\textonehalf}}\) space}
Points $x$ and $y$ in a topological space are separated by closed neighborhoods if and only if there exists a closed neighborhood $U$ of $x$ and a closed neighborhood $V$ of $y$ such that $U$ and $V$ are disjointed.

A topological space is called a Urysohn space or T\(_{2\text{\textonehalf}}\) space if and only if any two distinct points in it are separated by closed neighborhoods. A topological space is a T\(_{2\text{\textonehalf}}\) space implies it is a $T_2$ space.
\sssc{Completely Hausdorff space, functionally Hausdorff space, or completely T$_2$ space}
Points $x$ and $y$ in a topological space $X$ are separated by a (continuous) function if and only if there exists a continuous function $f\colon X\to [0,1]$ such that $f(x)=0$ and $f(y)=1$.

A topological space is called a completely Hausdorff space, functionally Hausdorff space, or completely T$_2$ space if and only if any two distinct points in it are separated by a continuous function. A topological space is a completely T$_2$ space implies it is a T\(_{2\text{\textonehalf}}\) space.
\sssc{Regular space (正則空間)}
A topological space $X$ is called a regular space if and only if for any closed set $F$ and any point $x\notin F$, there exists a neighbourhood $U$ of $x$ and a neighbourhood $V$ of $F$ such that $U$ and $V$ are disjointed.
\sssc{Regular Hausdorff space or T$_3$ space}
A topological space is called a regular Hausdorff space or T$_3$ space if and only if it is a regular space and a Hausdorff space.
\sssc{Completely regular space}
A closed set $F$ and a point $x\notin F$ in a topological space $X$ are separated by a (continuous) function if and only if there exists a continuous function $f\colon X\to [0,1]$ such that $f(x)=0$ and $F\subseteq f^{-1}(\{1\})$.

A topological space $X$ is called a completely regular space if and only if any closed set $F$ and any point $x\notin F$ in it are separated by a (continuous) function. A topological space is a completely regular space implies it is a regular space.
\sssc{Completely regular Hausdorff space, Tychonoff space, completely T$_3$ space, or T\(_{3\text{\textonehalf}}\) space}
A topological space $X$ is called a completely regular Hausdorff space, Tychonoff space, completely T$_3$ space, or T\(_{3\text{\textonehalf}}\) space if and only if it is a completely regular space and a Hausdorff space. A topological space is a completely T$_3$ space implies it is a completely T$_2$ space.
\sssc{Normal space (正規空間)}
Disjoint closed sets $E$ and $F$ in a topological space are separated by neighborhoods if and only if there exists a neighborhood $U$ of $E$ and a neighborhood $V$ of $F$ such that $U$ and $V$ are disjointed.

A topological space is called a normal space if and only if any two distinct closed sets in it are separated by neighborhoods. A topological space is a normal space implies it is a regular space.
\sssc{Urysohn's lemma}
Two closed sets $E$ and $F$ in a topological space $X$ are separated by a (continuous) function if and only if there exists a continuous function $f\colon X\to [0,1]$ such that $\subseteq f^{-1}(\{0\})$ and $F\subseteq f^{-1}(\{1\})$.

The Urysohn's lemma states that, a topological space is a normal space if and only if any two disjoint closed sets in it are separated by a (continuous) function.
\sssc{Normal Hausdorff space or T$_4$ space}
A topological space is called a normal Hausdorff space or T$_4$ space if and only if it is a normal space and a $T_1$ space. A topological space is a normal Hausdorff space implies it is a regular Hausdorff space.
\sssc{Completely normal space or hereditarily normal space}
A topological space is called a completely normal space or hereditarily normal space if and only if any subspace of it is a normal space.
\sssc{Completely normal Hausdorff space, T$_5$ space, or completely T$_4$ space}
A topological space is called a completely normal Hausdorff space, T$_5$ space, or completely T$_4$ space if and only if it is a completely normal space and a T$_1$ space. A topological space is a completely normal Hausdorff space implies it is a normal Hausdorff space and a completely regular Hausdorff space.
\sssc{Perfectly normal space}
Two closed sets $E$ and $F$ in a topological space $X$ are precisely separated by a (continuous) function if and only if there exists a continuous function $f\colon X\to [0,1]$ such that $E=f^{-1}(\{0\})$ and $F=f^{-1}(\{1\})$.

A topological space is called a perfectly normal space if and only if any two disjoint closed sets in it are precisely separated by a (continuous) function. A topological space is a perfect normal space implies it is a completely normal space.
\sssc{Perfectly normal Hausdorff space, T$_6$ space, or perfectly T$_4$ space}
A topological space is called a perfectly normal Hausdorff space, T$_6$ space, or perfectly T$_4$ space if and only if it is a perfectly normal space and a $T_1$ space. A topological space is a perfectly normal Hausdorff space implies it is a completely normal Hausdorff space.
\sssc{Connected space (連通空間)}
A topological space is called disconnected if and only if it can be expressed as the union of two disjoint non-empty open sets; otherwise, it is called connected.
\sssc{Path-connected space (路徑連通空間)}
A topological space $X$ is called path-connected if and only if for any two points $x,y\in X$, there existes a continuous function $f\colon[0,1]\to X$ such that $f(0)=x$ and $f(1)=y$. A topological space is path-connected implies it is connected.
\sssc{Simply connected space or 1-connected space (單連通空間)}
A topological space is called simply connected if and only if it is path-connected, and that for any two points $x,y\in X$ and any two continuous functions $f\colon[0,1]\to X$ and $g\colon[0,1]\to X$ such that $f(0)=g(0)=x$ and $f(1)=g(1)=y$, there exists a continuous function $F\colon[0,1]\times[0,1]\to X$ such that $F(x,0)=f(x)$ and $F(x,1)=g(x)$.
\sssc{Compact space (緊緻空間)}
A topological space $X$ is called compact if and only if for every collection $C$ of open subsets of $X$ such that
\[X=\bigcup_{S\in C}S,\]
called a open cover, there is a finite subcollection $F\subseteq C$ such that
\[X=\bigcup _{S\in F}S,\]
called a finite subcover.
\subsection{Compact (緊緻) subset}
A subset $K$ of a topological space $X$ is said to be compact if for every collection $C$ of open subsets of $X$ such that
\[K\subseteq\bigcup_{S\in C}S,\]
called a open cover, there is a finite subcollection $F\subseteq C$ such that
\[K\subseteq\bigcup _{S\in F}S,\]
called a finite subcover.
\subsection{Locally finite cover}
Let $X$ be a topological space, and $\{U_i\mid i\in I\}$ be a cover of $X$. We say that $U$ is locally finite if for all $x\in X$, there exists $V\ni x$ that is open in $X$ such that the set $\{i\in I\mid V\cap U_i\neq\varnothing\}$ is finite.
\ssc{Dense (稠密) subset}
A subset $A$ of a topological space $X$ is said to be a dense subset of $X$ if for every $x\in X$ and every neighborhood $U$ of $x$, $U\cap A\neq\varnothing$.
\subsection{Metric space (度量空間 or 賦距空間)}
\sssc{Metric space}
Metric space is an ordered pair $(M, d)$ where $M$ is a set and $d$ is a metric on M, i.e., a function $d:\,M\times M\to\mathbb{R}$ satisfying the following axioms for all points $x,y,z\in M$:
\begin{enumerate}
\item The distance from a point to itself is zero: $d(x,x)=0$.
\item (Positivity) The distance between two distinct points is always positive:$x\neq y\implies d(x,y)>0$.
\item (Symmetry) The distance from $x$ to $y$ is always the same as the distance from $y$ to $x$: $d(x,y)=d(y,x)$.
\item The triangle inequality: $d(x,z)\leq d(x,y)+d(y,z)$.
\end{enumerate}

For a normed space $(X,\|\cdot\|)$, unless otherwise specified, it is equipped with a metric $d$ defined as
\[d(x,y)=\|x-y\|,\quad \forall x,y\in X.\]
\sssc{Cauchy sequence (柯西序列)}
Given a metric space $(x,d)$, a sequence of elements of $X$:
\[x_1,x_2,x_3,\ldots\]
is Cauchy if for every positive real number $\varepsilon$ there is a positive integer $N$ such that for all positive integers $m,n>N$:
\[d\left(x_m,x_n\right)<\varepsilon.\]
\sssc{Complete metric space (完備度量空間) or Cauchy space (柯西空間)}
A metric space $(X,d)$ is complete if every Cauchy sequence of points in $X$ has a limit that is also in $X$.
\sssc{Complete subset or Cauchy subset}
A subset $S$ of a metric space $(X,d)$ is complete if every Cauchy sequence of points in $S$ has a limit that is also in $S$.
\sssc{(Open) ball (開球)}
In a metric space $(X,d)$, the open ball or ball of radius $r>0$ and centered at the point $a\in X$, denoted as $B_{<r}(a)$, $B_{r}(a)$, or $B(a,r)$, is defined as:
\[B_{r}(a)\coloneq\left\{p\in X:\, d(a,p)<r\right\}.\]
\sssc{Closed ball (閉球)}
In a metric space $(X,d)$, the closed ball of radius $r\geq 0$ and centered at the point $a\in X$, denoted as $B_{\leq r}(a)$, is defined as:
\[B_{\leq r}(a)\coloneq\left\{p\in X:\, d(a,p)\leq r\right\}.\]
\sssc{Bounded (有界) subset}
A subset $S$ of a metric space $(M, d)$ is called bounded if and only if there exists $r > 0$ such that for all $s,t\in S$, we have $d(s, t) < r$.
\sssc{Totally bounded or precompact subset}
A subset $S$ of a metric space $(M, d)$ is called totally bounded or precompact if and only if there eixsts $r>0$ such that there exists a finite collection of open balls of radius $r$ that is a cover of $S$.
\sssc{$\varepsilon$-net}
A subset $E$ of a metric space $X$ is an $\varepsilon$-net for another subset $A$ of $X$ if
\[A\subseteq\bigcup_{x\in E}B(x,\varepsilon).\]

If a finite $\varepsilon$-net exists for every $\varepsilon>0$, then $A$ is totally bounded.
\sssc{Bolzano–Weierstrass Theorem (波爾查諾-魏爾斯特拉斯定理)}
Every bounded sequence in an Euclidean space $\bbR^n$ has a convergent subsequence.

\begin{proof}\mbox{}\\
    For $\bbR$, take any bounded sequence $(x_n)$ in $\bbR$. Then there exist real numbers $m,M$ such that
    \[m\leq x_n\leq M,\quad\forall n.\]
    So the sequence lies in the closed interval $[m,M]$.

    For $\varepsilon=1$. Divide $[m,M]$ into disjoint subintervals with length $\frac{M-m}{2^\varepsilon}$:
    \[\qty[m,\frac{m+M}{2}],\qty[\frac{m+M}{2},M].\]
    Since $(x_n)$ is infinite, by the pigeonhole principle, one of the subintervals must contain infinitely many terms of $(x_n)$. Denote it by $I_1$ and pick the subsequence $\qty(x_n^{(1)})$ of $(x_n)$ that is entirely in $I_1$.

    Repeat it for $\varepsilon=2,3,\ldots$ by dividing $I_{\varepsilon-1}$ into two halves, we can construct a nested sequence of subsequences
    \[\qty(x_n^{(1)})\supseteq \qty(x_n^{(2)})\supseteq\ldots\]
    where $\qty(x_n^{(k)})$ lies in a closed interval of length $\frac{M-m}{2^k}$, $I_k$.

    Pick the diagonal sequence $y_k=x_k^(k)$. Then
    \[y_k\in\qty(x_n^{(k)})\subseteq I_k,\]
    and the sequence $(y_k)$ is a convergent subsequence of $(x_n)$ because
    \[\forall N\in\bbN\colon m,n\geq N\implies |y_m-y_n|\leq\frac{M-m}{2^N}.\]

    To generalized to $\bbR^n$, consider each coordinate sequence in $\bbR$, it has a convergent subsequence. Diagonalization gives a subsequence converging in all coordinates.
\end{proof}
\sssc{A compact subset of a Hausdorff space is closed}
\textit{Statement:} If a subset $K$ of a Hausdorff space $X$ is compact, then it must be closed.

\begin{proof}\mbox{}\\
We want to show that $K$ is closed, that is, $X\setminus K$ is open.

For any $a\in X\setminus K$, for each $x\in K$, by the definition of Hausdorff space, there exist a disjoint open set $U_x$ such that $x\in U_x$, $a\in V_x$, and $U_x\cap V_x=\varnothing$.

The collection of $C=\{U_x\mid x\in K\}$ is an open cover of $K$. Since $K$ is compact, a finite subset $D$ of $C$ is a cover of $K$.

Corresponding to each $U_i\in D$, we have an open neighborhood $V_i$ of $a$ such that $U_i\cap V_i=\varnothing$. Let $V=\bigcap_{U_i\in D}V_i$. Then $V$ is an open neighborhood of $a$ and is disjoint from $\bigcap_{U_i\in D}U_i\supseteq K$, so
\[V\cap K=\varnothing.\]

For every $a\in X\setminus K$, we found an open neighborhood $V$ of $a$ such that
\[V\subseteq X\setminus K\land V\cap K=\varnothing,\]
so $K$ is closed.
\end{proof}
\sssc{A closed subset of a compact set is compact}
\textit{Statement:} If $K$ is a compact subset of a topological space $X$ and $F\subseteq K$ is closed, then $F$ is compact.

\begin{proof}\mbox{}\\
Let $C$ be an open cover of $F$. Then
\[C\cup\{X\setminus F\}\]
is an open cover of $K$.

Since $K$ is compact, there is a finite subcollection $D=C'\cup\{X\setminus F\}$ of $C\cup\{X\setminus F\}$ where $C'\subseteq C$ that is a cover of $K$. Then $C'$ is a finite subcollection of $C$ that is a cover of $F$. Thus $F$ is compact.
\end{proof}
\sssc{A compact subset of a metric space is bounded}
\textit{Statement:} If $K$ is a compact subset of a metric space $(X,d)$, then $K$ is bounded.

\begin{proof}\mbox{}\\
Since $K$ is compact, for any point $p\in X$, there exist a finite collection of open balls
\[\{B(p,n)\mid n\in\bbN\land n\leq m\in\bbN\}\]
that is a cover of $K$, then $K\subseteq B(p,m)$, so $K$ is bounded.
\end{proof}
\sssc{A complete subset of a complete metric space is closed}
\Textit{Statement:} If a subset $A$ of a complete metric space is compact, them $A$ is closed.

\begin{proof}\mbox{}\\
Since a metric space is necessarily a Hausdorff space, any convergent sequence must have a unique limit. Since $A$ is complete, every Cauchy sequence in $A$ converges to a point in $A$. Thus, $A$ is closed.
\end{proof}
\sssc{A closed subset of a complete metric space is complete}
\Textit{Statement:} If a subset $A$ of a complete metric space $X$ is closed, them $A$ is compact.

\begin{proof}\mbox{}\\
Let $(x_n)$ be a Cauchy sequence in $A$. Since the space is complete, it converges to some $x\in X$. Since $A$ is closed in $X$, any limit of a convergent sequence in $A$ must be in $A$. Thus, $A$ is complete.
\end{proof}
\sssc{A closed and bounded subset of an Euclidean space is compact}
\textit{Statement: }If a subset $A$ of an Euclidean space $\bbR^n$ is closed and bounded, then $A$ is compact.

\begin{proof}\mbox{}\\
Let $(x_k)$ be any sequence in $A$. We want to show that it has a convergent subsequence with its limit in $A$.

Since $A$ is bounded, there exists $M>0$ such that
    \[\|x_k\|\leq M,\quad\forall k.\]
    So
    \[x_k\in B(0,M),\quad\forall k.\]
    By the Bolzano–Weierstrass theorem, there exists a subsequence $(x_{k_j})$ that coverges to some point $x\in\bbR^n$.
    
    Since $A$ is closed, it is complete. Thus $x\in A$.
\end{proof}
\sssc{Heine–Borel Theorem (海涅-博雷爾定理)}
For a subset $S$ of an Euclidean space $\bbR^n$, the following two statements are equivalent:
\bit
\item $S$ is compact;
\item $S$ is closed and bounded.
\eit
\sssc{A complete and totally bounded subset of a complete metric space is compact}
\textit{Statement: }If a subset $A$ of a complete metric space $(x,d)$ is complete and totally bounded, then $A$ it is compact.

\begin{proof}\mbox{}\\
Take any sequence $(x_n)$ in $A$. We want to find a convergent subsequence whose limit lies in $A$.

For $\varepsilon=1$, since $A$ is totally bounded, there exists a finite cover of $A$ by closed balls of radius $1$:
    \[A\subseteq\bigcup_{i=1}^{N_1}B_1\qty(y_i).\]
    Since $(x_n)$ is infinite but the cover is finite, by the pigeonhole principle, one of these balls must contain infinitely many terms of $(x_n)$. Denote this ball by $B_1\qty(y_{i_1})$ and pick the subsequence $\qty(x_n^{(1)})$ of $(x_n)$ that is entirely in $B_1\qty(y_{i_1})$.

    Repeat it for $\varepsilon=\frac{1}{2},\frac{1}{3},\ldots$, we can construct a nested sequence of subsequences
    \[\qty(x_n^{(1)})\supseteq \qty(x_n^{(2)})\supseteq\ldots\]
    where $\qty(x_n^{(k)})$ lies in a closed ball of radius $\frac{1}{k}$, $B_{\frac{1}{k}}\qty(y_{i_k})$.

    Pick the diagonal sequence $y_k=x_k^{(k)}$. Then
    \[y_k\in\qty(x_n^{(k)})\subseteq B_{\frac{1}{k}}\qty(y_{i_k}),\]
    and the sequence $(y_k)$ is a Cauchy sequence in $A$ because
    \[\forall N\in\bbN\colon m,n\geq N\implies d(y_m,y_n)\leq\frac{2}{N}.\]
    Since $A$ is complete, $(y_k)$ must converges to some point $y\in A$.
\end{proof}
\sssc{Generalized/general Heine–Borel Theorem}
For a subset $S$ of a complete metric space $(X,d)$, the following two statements are equivalent:
\bit
\item $S$ is compact;
\item $S$ is complete and totally bounded.
\eit
\sssc{A totally bounded subset is bounded}
\Textit{Statement:} If a subset $A$ of a metric space $(X,d)$ is totally bounded, them $A$ is bounded.

\begin{proof}\mbox{}\\
By total boundedness, there exist finitely many points $x_1,x_2,\ldots x_n\in X$ such that:
\[A\subseteq\bigcup_{i=1}^nB(x_i,1).\]
Pick one of these centers, say $x_1$.

For any $a\in A$, there exists some $i\in\{1,2,\ldots n\}$ such that
\[d(a,x_i)<1.\]
By triangle inequality:
\[d(a,x_1)\leq d(a,x_i)+d(x_i,x_1)\leq 1+d(x_i,x_1).\]
Let:
\[M=1+\max_{i\in\{1,2,\ldots n\}}d(x_i,x_1).\]
So, for all $a\in A$:
\[d(a,x_1)<M.\]
Thus:
\[A\subseteq B(x_1,M).\]
Therefore, $A$ is bounded.
\end{proof}
\sssc{All norms on a finite-dimensional vector space are equivalent}
\textit{Statement: }Let $V$ be a finite-dimensional vector space over $\bbR$ or $\bbC$ and $\dim V=n$. If $\|\cdot\|_a
$ and $\|\cdot\|_b$ are two norm on $V$, then there exist constants $c,C>0$ such that for all $v\in V$,
\[c\|v\|_a\leq\|v\|_b\leq C\|v\|_a,\]
called that the two norms are equivalent, and $c,C$ are called the equivalence constants.

\begin{proof}\mbox{}\\
    First consider $V=\bbR^n$. The set
    \[S_a=\{x\in\bbR^n\mid\|x\|_a=1\}\]
    is closed and bounded in $\bbR^n$, hence compact.

    Define
    \[f(x)=\|x\|_b,\quad x\in S_a.\]
    Since $\|\cdot\|_b$ is continuous, $f$ is continuous on the compact set $S_a$. Therefore $f$ attains a minimum and maximum on $S_a$. So there exist $m,M\geq 0$ such that
    \[m\leq f(x)\leq M.\]
    For any $v\neq 0$, write
    \[u=\frac{v}{\|v\|_a}\in S_a.\]
    Then
    \[\|v\|_b=\|v\|_a\cdot\|u\|_b.\]
    So
    \[m\|v\|_a\leq\|v\|_b\leq M\|v\|_a.\]
    Thus $c=m,C=M$ are the equivalence constants.
\end{proof}
\sssc{A bounded subset of a finite-dimensional Banach space is totally bounded}
\Textit{Statement:} If a subset $A$ of a $n$-dimensional Banach space $X$ with $n\in\bbN$ is bounded, then $A$ is totally bounded.

\begin{proof}\mbox{}\\
    Pick a linear isomorphism
    \[T\colon\bbR^n\to X.\]
    Define a norm on $\bbbR^n$ by
    \[\|v\|_*\coloneq\|Tv\|_X,\quad\forall v\in\bbR^n.\]
    By that all norms on a finite-dimensional vector space are equivalent, the Euclidean norm $\|\cdot\|_2$ and $\|\cdot\|_*$ are equivalent, that is, there exist constants $c,C>0$ such that for every $v\in \bbR^n$,
    \[c\|v\|_2\leq\|v\|_*\leq C\|v\|_2.\qquad (1)\]
    Let $A\subseteq X$ be bounded. Then $B\coloneq T^{-1}(A)$ is bounded with respect to the metric induced by $\|\cdot\|_*$ and, by $(1)$, is bounded with respect to the metric induced by $\|\cdot\|_2$. So there is $R>0$ with $B_{\|\cdot\|_2}(0,R)\supseteq B$.
    For any $\varepsilon>0$ and $\delta\coloneq\frac{\varepsilon}{C}$, there is a finite collection of $B_{\|\cdot\|_2}(v_i,R)$ where $v_i\in\bbR^n$ that is a cover of $B_{\|\cdot\|_2}(0,R)$.

    Take any $b\in B$. Then $b\in B_{\|\cdot\|_2}(v_i,R)$ for some $v_i$, so $\|b-v_i\|_2<\delta$.

    By $(1)$, we get
    \[\|T(b)-T(v_i)\|_X=\|b-v_i\|_*\leq C\|b-v_i\|_2<C\delta=\varepsilon.\]
    Therefore the finite set of all $T(v_i)$ is an $\varepsilon$-net for $A$. Since $\varepsilon>0$ was arbitrary, $A$ is totally bounded.
\end{proof}
\subsection{Topological field (拓樸域)}
A topological field is a topological space, such that addition, multiplication, the maps $a\mapsto -a$, and $a\mapsto a^{-1}$ are continuous maps with respect to the topology of the space.
\ssc{Net (網)}
\sssc{Net (網)}
A net in $X$, denoted as $x_{\bullet }=\left(x_{a}\right)_{a\in A}$ or $\left\langle x_{a}\right\rangle _{a\in A}$, is a function of the form $x_{\bullet }\colon A\to X$ whose domain $A$ is some directed set, and whose values are $x_{\bullet }(a)=x_{a}$ in a topological space $X$. Elements of a net's domain are called its indices.
\sssc{Eventually or residually}
A net $x_{\bullet }=\left(x_{a}\right)_{a\in A}$ in $X$ is said to be eventually or residually in a set $S$ if there exists some $a\in A$ such that for every $b\in A$ with $b\geq a$, the point $x_b\in S$.
\sssc{Limit of nets}
A point $x$ in a topological space $X$ is called a limit point or limit of a net $x_{\bullet }$ in $X$ if, for every open neighborhood $U$ of $x$, the net $x_{\bullet }$ is eventually in $U$, also called that the net converges to/towards $x$, and denoted as $x_{\bullet}\to x$, $x_a\to x$, $\lim x_{\bullet}\to x$, $\lim_{a\in A}x_a\to x$, or $\lim_ax_a\to x$.

If $\lim x_{\bullet}\to x$ and this limit is unique (i.e. $\lim x_{\bullet}\to y$ only if $x=y$), then one writes $\lim x_{\bullet}=x$, $\lim_{a\in A}x_a=x$, or $\lim_ax_a=x$.
\sssc{Cofinally (共尾地) or frequently}
A net $x_{\bullet }=\left(x_{a}\right)_{a\in A}$ in $X$ is said to be cofinally or frequently in a set $S$ if for every $a\in A$ there exists $b\in A$ such that $b\geq a\land x_b\in S$.
\sssc{Cluster point, or accumulation point (集積點)}
A point $x$ in a topological space $X$ is called a cluster point or accumulation point of a net $x_{\bullet }$ in $X$ if, for every open neighborhood $U$ of $x$, the net $x_{\bullet }$ is cofinally in $U$, also called that the net clusters to/towards $x$.

A limit point of a net is necessary a cluster point of it.
\sssc{Subnet or Willard-subnet}
A net $s_{\bullet }=\left(s_{i}\right)_{i\in I}$ is called a subnet or 
{\displaystyle s_{\bullet }} is called a subnet or Willard-subnet of another net $x_{\bullet }=\left(x_{a}\right)_{a\in A}$ if there exists an order-preserving map $h\colon I\to A$ such that $h(I)$ is a cofinal subset of $A$ and 
\[s_{i}=x_{h(i)}\quad \forall i\in I.\]
\sssc{Ultranet or universal net}
A net $x_{\bullet }$ in $X$ is called an ultranet or a universal net if for every subset $S\subseteq X$, $x_{\bullet }$ is eventually in $S$ or $X\setminus S$.

A point is a limit point of a ultranet if and only if it a cluster point of it.
\sssc{In Hausdorff spaces}
A net in a Hausdorff space has at most one limit; if it has one, the limit point of it is the only cluster point of it.
\end{document}
