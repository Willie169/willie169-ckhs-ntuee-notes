\documentclass[a4paper,12pt]{article}
\setcounter{secnumdepth}{5}
\setcounter{tocdepth}{3}
\input{/usr/share/LaTeX-ToolKit/template.tex}
\begin{document}
\title{Abstract Algebra}
\author{沈威宇}
\date{\temtoday}
\titletocdoc
\sct{Abstract Algebra (抽象代數)}
\ssc{Algebraic structures (代數結構) or algebraic system (代數系統)}
An algebraic structure or algebraic system consists of a nonempty set $A$ (called the underlying set, carrier set or domain), a collection of operations on $A$ (typically binary operations such as addition and multiplication), and a finite set of identities (known as axioms) that these operations must satisfy.
\ssc{Operation (運算)}
An operation is a function from a set to itself.
\ssc{Common axioms}
\sssc{Equational axioms}
\bit
\item Commutativity (交換律): An operation $*$ is commutative if $x*y=y*x$ for every $x$ and $y$ in the algebraic structure.
\item Associativity (結合律): An operation $*$ is associative if $(x*y)*z=x*(y*z)$ for every $x$, $y$ and $z$ in the algebraic structure.
\item Left distributivity (左分配律): An operation $*$ is left-distributive with respect to another operation $+$ if $x*(y+z)=(x*y)+(x*z)$ for every $x$, $y$, and $z$ in the algebraic structure.
\item Right distributivity (右分配律): An operation $*$ is right-distributive with respect to another operation $+$ if $(y+z)*x=(y*x)+(z*x)$ for every $x$, $y$, and $z$ in the algebraic structure.
\item Distributivity (分配律): An operation $*$ is distributive with respect to another operation $+$ if it is both left-distributive and right-distributive with respect to $+$. If the operation $*$ is commutative, left and right distributivity are both equivalent to distributivity.
\eit
\sssc{Existential axioms}
\bit
\item Identity element: A binary operation $*$ has an identity element if there is an element $e$ such that $x*e=x$ and $e*x=x$ for any $x$ in the algebraic structure.
\item Inverse element: Given a binary operation $*$ that an identity element $e$, an element $x$ in the algebraic structure is invertible if there exists an element $\operatorname{inv}(x)$, called the inverse element, in the algebraic structure such that $x*\operatorname{inv}(x)=e$ and $\operatorname{inv}(x)*x=e$.
\eit
\ssc{Group-like algebraic structure}
\sssc{Magma (原群) or binar}
A magma or binar is a set $M$ with an operation $*$ such that
\[\forall a,b\colon a,b\in M\implies a*b\in M.\]
\sssc{Semigroup (半群)}
A semigroup is a magma equipped with an operation that is associative.
\sssc{Unital magma}
A unital magma is a magma equipped with an operation that has an identity element.
\sssc{Monoid (么半群, 單群, or 亞群)}
A monoid is a magma equipped with an operation that is associative and has an identity element.
\sssc{Quasigroup (擬群)}
A quasigroup is a magma equipped with an operation that is distributive.
\sssc{Loop (么擬群 or 圈)}
A loop is a magma equipped with an operation that is distributive and has an identity element.
\sssc{Associative quasigroup}
An associative quasigroup is a magma equipped with an operation that is associative and distributive.
\sssc{Group (群)}
A group is a magma equipped with an operation that is associative, distributive, and has an identity element.
\sssc{Abelian group (阿貝爾群) or commutative group (交換群)}
An abelian group, aka a commutative group, is a group equipped with an operation that is commutative.
\sssc{Field (域)}
A field is a ordered triple $(F,+,\cdot)$ of set $F$ and binary operations $+$ and $\cdot$ on $F$ called addition and multiplication such that
\bit
\item $(F,+)$ is an abelian group under addition with $0$ as the identity element, called addictive identity,
\item $(F\setminus\{0\},\cdot)$ is an abelian group under multiplication with $1$ as the identity element, called multiplicative identity, and
\item Distributivity of multiplication over addition:
\[a\cdot (b+c)=(a\cdot b)+(b\cdot c).\]
\eit
\end{document}
