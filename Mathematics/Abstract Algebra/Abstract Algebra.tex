\documentclass[a4paper,12pt]{article}
\setcounter{secnumdepth}{5}
\setcounter{tocdepth}{3}
\input{/usr/share/LaTeX-ToolKit/template.tex}
\begin{document}
\title{Abstract Algebra}
\author{沈威宇}
\date{\temtoday}
\titletocdoc
\sct{Abstract Algebra (抽象代數)}
\ssc{Algebraic structures (代數結構) or algebraic system (代數系統)}
An algebraic structure or algebraic system consists of a nonempty set $A$ (called the underlying set, carrier set or domain), a collection of operations on $A$ (typically binary operations such as addition and multiplication), and a finite set of identities (known as axioms) that these operations must satisfy.
\ssc{Operation (運算)}
An operation is a function from a set to itself.
\ssc{Common axioms}
\sssc{Equational axioms}
\bit
\item Commutativity (交換律): An operation $*$ is commutative if $x*y=y*x$ for every $x$ and $y$ in the algebraic structure.
\item Associativity (結合律): An operation $*$ is associative if $(x*y)*z=x*(y*z)$ for every $x$, $y$ and $z$ in the algebraic structure.
\item Left distributivity (左分配律): An operation $*$ is left-distributive with respect to another operation $+$ if $x*(y+z)=(x*y)+(x*z)$ for every $x$, $y$, and $z$ in the algebraic structure.
\item Right distributivity (右分配律): An operation $*$ is right-distributive with respect to another operation $+$ if $(y+z)*x=(y*x)+(z*x)$ for every $x$, $y$, and $z$ in the algebraic structure.
\item Distributivity (分配律): An operation $*$ is distributive with respect to another operation $+$ if it is both left-distributive and right-distributive with respect to $+$. If the operation $*$ is commutative, left and right distributivity are both equivalent to distributivity.
\eit
\sssc{Existential axioms}
\bit
\item Having an identity element: A binary operation $*$ has an identity element if there is an element $e$ such that $x*e=x$ and $e*x=x$ for any $x$ in the algebraic structure. Unless otherwise specified, the identity element of an operator denoted as $+$ is denoted as $0$, and the identity element of an operator denoted as $*$ is denoted as $1$.
\item Latin square property: A binary operation $*$ has Latin square property iff for every $a,b$ in the algebraic structure, there exists two unique $x,y$ in the algebraic structure such that $a*x=b$ and $x*a=b$.
\item Invertibility: Given a binary operation $*$ that an identity element $e$, an element $x$ in the algebraic structure is invertible if there exists an unique element $\operatorname{inv}(x)$, called the inverse element, in the algebraic structure such that $x*\operatorname{inv}(x)=e$ and $\operatorname{inv}(x)*x=e$. Unless otherwise specified, the inverse element of $a$ of an operator denoted as $+$ is denoted as $-a$, and the inverse element of $a$ of an operator denoted as $*$ is denoted as $a^{-1}$.
\eit
\ssc{Group-like algebraic structure}
\sssc{Magma (原群) or binar}
A magma or binar is a set $M$ with an operation $*$ such that
\[\forall a,b\colon a,b\in M\implies a*b\in M.\]
\sssc{Semigroup (半群)}
A semigroup is a magma equipped with an operation that is associative.
\sssc{Unital magma}
A unital magma is a magma equipped with an operation that has an identity element.
\sssc{Monoid (么半群, 單群, or 亞群)}
A monoid is a magma equipped with an operation that is associative and has an identity element.
\sssc{Commutative monoid or abelian monoid}
A commutative monoid or abelian monoid is a monoid equipped with an operation that is commutative.
\sssc{Quasigroup (擬群)}
A quasigroup is a magma equipped with an operation that has Latin square property.
\sssc{Loop (么擬群 or 圈)}
A loop is a magma equipped with an operation that has Latin square property and has an identity element.
\sssc{Associative quasigroup}
An associative quasigroup is a magma equipped with an operation that is associative and has Latin square property.
\sssc{Group (群)}
A group is a magma equipped with an operation that has an identity element, is associative, and every element has an inverse element, that is, a nonempty associative quasigroup.
\sssc{Abelian group (阿貝爾群) or commutative group (交換群)}
An abelian group, also called a commutative group, is a group equipped with an operation that is commutative.
\sssc{Group extension}
We say that a group $K$ is a subgroup of another group $L$ and that $L$ is a group extension of $K$ iff $K\subseteq L$ and that the operation of $K$ is that of $L$ restricted to $K$.

A trivial subgroup of a group is the singleton of the identity element of its operation.

A proper subgroup of a group is a subgroup of it that is a proper subset of it.
\sssc{One-parameter subgroup}
A map $\phi\colon\bbR\to G$ to some group equipped with operation $*$ is called a one-parameter subgroup of $G$ iff
\[\phi(t)*\phi(s)=\phi(t+s).\]
\ssc{Ring-like algebraic structure}
PLACEHOLDER
rngs ⊃ rings ⊃ commutative rings ⊃ integral domains ⊃ integrally closed domains ⊃ GCD domains ⊃ unique factorization domains ⊃ principal ideal domains ⊃ Euclidean domains ⊃ fields ⊃ algebraically closed fields
\sssc{Rng or non-unital ring}
A rng or non-unital ring is a ordered triple $(R,+,\cdot)$ of set $R$, binary operations $+$, and $\cdot$ on $R$ called addition and multiplication such that
\bit
\item $(R,+)$ is an abelian group under addition with $0$ as the identity element, called additive identity,
\item $(R,\cdot)$ is a semigroup under multiplication, and
\item Distributivity of multiplication over addition:
\[a\cdot (b+c)=(a\cdot b)+(b\cdot c).\]
\eit
\sssc{Ring}
A ring is a ordered triple $(R,+,\cdot)$ of set $R$, binary operations $+$, and $\cdot$ on $R$ called addition and multiplication such that
\bit
\item $(R,+)$ is an abelian group under addition with $0$ as the identity element, called additive identity,
\item $(R,\cdot)$ is a monoid under multiplication with $1$ as the identity element, called multiplicative identity, and
\item Distributivity of multiplication over addition:
\[a\cdot (b+c)=(a\cdot b)+(b\cdot c).\]
\eit
\sssc{Exponentiation to positive integer power}
Exponentiation of an element $x$ of a rng $R$ to a positive integer power $n$ is defined as
\[x^n=\prod_{i=1}^nx.\]
\sssc{Exponentiation to zero}
Exponentiation of an element $x$ of a ring $R$ to 0 is defined as 1.
\sssc{Ring homomorphism}
A ring homomorphism is a function from a ring $R$ to a ring $S$ that preserves addition, multiplication and multiplicative identity, that is,
\[f(a+b)=f(a)+f(b),\quad f(ab)=f(a)f(b),\quad f(1)=1,\]
for all $a,b\in R$.
\sssc{Ring extension}
We say that a ring $K$ is a subring of another ring $L$ and that $L$ is a ring extension of $K$ iff $K\subseteq L$ and that the operations of $K$ are those of $L$ restricted to $K$.
\sssc{Commutative ring}
A commutative ring is a ring whose multiplication is commutative.
\sssc{Center of ring}
The center of a ring $R$, denoted as $Z(R)$, is a commutative ring defined as a subring of $R$ consistig of all elements $x$ such that $xy=yx$ for all elements $y$ in $R$.
\sssc{Module}
A left module $M$ of a ring $R$, denoted as left $R$-module, is an abelian group $(M,+)$ and an operation $\cdot\colon R\times M\to M$ called scalar multiplication such that for all $r,s\in R$ and $x,y\in M$,
\[r\cdot(x+y)=r\cdot x+r\cdot y,\]
\[(r+s)\cdot x=r\cdot x+s\cdot x,\]
\[(r\cdot s)\cdot x=r\cdot(s\cdot x),\]
\[1\cdot x=x.\]
A right module $M$ of a ring $R$, denoted as right $R$-module, is an abelian group $(M,+)$ and an operation $\cdot\colon R\times M\to M$ called scalar multiplication such that for all $r,s\in R$ and $x,y\in M$,
\[r\cdot(x+y)=r\cdot x+r\cdot y,\]
\[(r+s)\cdot x=r\cdot x+s\cdot x,\]
\[(r\cdot s)\cdot x=s\cdot(r\cdot x),\]
\[1\cdot x=x.\]
If $R$ is a commutative ring, then left $R$-module is equivalent to right $R$-module, and they are collectively called $R$-module.
\sssc{Associative algebra}
An associative algebra $A$ over a commutative ring $R$, denoted as associative $R$-algebra or $R$-algebra, is a $R$-module $A$ of which $(A,+)$ is a ring and the scalar multiplication is such that for all $r\in R$ and $x,y\in A$,
\[r\cdot(x\cdot y)=(r\cdot x)\cdot y=x\cdot (r\cdot y).\]
\sssc{Commutative algebra}
A commutative algebra is an associative algebra that is also a commutative ring.
\sssc{Algebraic element or algebraic}
If $A$ is an associative algebra over a ring $K$, then an element $a$ of $A$ is an algebraic element or algebraic over $K$, if there exists some non-zero polynomial $g(x)\in K[x]$ such that $g(a)=0$. Elements of $A$ that are not algebraic over $K$ are transcendental over $K$.
\sssc{Field (域)}
A field $(F,+,\cdot)$ is a commutative ring with $1\neq 0$ and $(F\setminus\{0\},\cdot)$ being an abelian group.
\sssc{Field extension}
We say that a field $K$ is a subfield of another field $L$ and that $L$ is a field extension of $K$ iff $K\subseteq L$ and that the operations of $K$ are those of $L$ restricted to $K$.
\sssc{Algebraic extension}
An algebraic extension $L$ of a field $K$ is a field extension of $K$ that is an associative algebra over $K$ such that for all $x\in L$, $x$ is algebraic over $K$.
PLACEHOLDER
Polynomial
polynomial with coefficient in ring $R$.
Root
Polynomial ring
PLACEHOLDER
\sssc{Algebraically closed field}
A field $F$ is closed if every non-constant polynomial $p(x)\in F[x]$ has a root in $F$.
PLACEHOLDER
\sssc{Algebraic closure}
An algebraic closure of a field $K$ is an algebraic extension of $K$ that is algebraically closed.
PLACEHOLDER
Separable extension
Separable polynomial
PLACEHOLDER
\sssc{Finite field or Galois field}
PLACEHOLDER
\end{document}
