\documentclass[a4paper,12pt]{article}
\setcounter{secnumdepth}{5}
\setcounter{tocdepth}{3}
\newcounter{ZhRenew}
\setcounter{ZhRenew}{1}
\newcounter{SectionLanguage}
\setcounter{SectionLanguage}{1}
\input{/usr/share/latex-toolkit/template.tex}
\begin{document}
\title{解析幾何}
\author{沈威宇}
\date{\temtoday}
\titletocdoc
\sct{解析幾何(Analytic Geometry)}
下:空間為歐幾里德空間;位置$\mathbf{x}=(x_1,x_2,\ldots)$;第$i$軸正方向單位向量$\mathbf{e}_i$;一向量空間中兩點$P$、$Q$,$\ora{PQ}\coloneq Q-P$;$\hat{\mathbf{v}}$為$\mathbf{v}$正方向單位向量。
\ssc{正整數維空間}
$n\in\mathbb{N}$
\sssc{仿射子空間一般式}
$\mathbb{R}^n$中,一$n-1$維仿射子空間可以表示成:
\[\mathbf{n}\cdot\mathbf{x}+c=0\]
稱此表達方法為一般式。
\sssc{仿射子空間截距式}
$\mathbb{R}^n$中,$n-1$維仿射子空間:
\[E\colon\sum_{i=1}^n\frac{x_1}{a_1}=1\]
必通過$a_i\mathbf{e}_i$,$\forall i\in\mathbb{N}\land i\leq n$,稱此表達方法為截距式,稱$a_i$為第$i$軸截距。
\sssc{仿射子空間隱式方程/多面式}
$\mathbb{R}^n$中,一個$m\in\mathbb{N}\land m<n$維仿射子空間$E$可表示成:
\[\mathbf{A}\mathbf{x}=\mathbf{c},\quad\mathbf{A}\in\mathbb{R}^{(n-m)\times n}\land\operatorname{rank}(\mathbf{A})=n-m\land\mathbb{c}\in\mathbb{R}^{n-m}\]
稱此表達方法為隱式方程或多面式,對於三維空間中的直線特稱兩面式。令$\mb{A}$的第$i$列為$\mb{A}_i$、$\mb{c}$的第$i$列為$\mb{c}_i$,則$\mb{A}_i\mathbf{x}=\mathbf{c}_i$代表一個$(n-1)$維仿射子空間,且這$(n-m)$個$(n-1)$維仿射子空間的交集為$E$。對於三維空間中的直線$E$,特稱所有使得$E\subseteq F$的$(n-1)$維仿射子空間$F$的集合為平面系。
\sssc{直線參數式}
$\mathbb{R}^n$中,一直線可表示成:
\[\mathbf{x}=\mathbf{a}+\mathbf{v}t,\quad t\in\mathbb{R}\]
稱$\mathbf{v}$為此直線之方向向量,稱此表達方法為參數式。
\sssc{直線(對稱)比例式}
$\mathbb{R}^n$中,直線:
\[\forall i,j\in\mathbb{N}\land i<j\leq n\colon\frac{x_i-a_i}{v_i}=\frac{x_j-a_j}{v_j}\]
與直線
\[\mathbf{x}=(a_1,a_2,\ldots,a_n)+(v_1,v_2,\ldots,v_n)t,\quad t\in\mathbb{R}\]
相同,稱前一種表達方法為(對稱)比例式。
\sssc{點與仿射子空間關係}
$\mathbb{R}^n$中,一$(n-1)$維仿射子空間$E\colon\mathbf{n}\cdot\mathbf{x}+c=0$與一點$\mathbf{P}$可能的關係與其充要條件為:
\begin{itemize}
\item $P$在$E$上$\iff\mathbf{n}\cdot\mathbf{P}+c=0$。
\item $P$在$E$的$\mathbf{n}$方向半空間$\iff\mathbf{n}\cdot\mathbf{P}+c>0$。
\item $P$在$E$的$-\mathbf{n}$方向半空間$\iff\mathbf{n}\cdot\mathbf{P}+c<0$。
\end{itemize}
\sssc{兩仿射子空間關係}
$\mathbb{R}^n$中,$k$維仿射子空間$E$與$m$維仿射子空間,其中$k,m\in\mathbb{N}\land k\leq m<n$,可能的關係與其必要條件為:
\begin{itemize}
\item 重合:$k=m$
\item 平行但不相交:$k=m$
\item 相交但不平行。
\item 歪斜(不平行也不相交)。
\end{itemize}
相交時,令它們的一個交點$P$,兩者在該點上分別有法空間(normal space)$N_PE$、$N_PF$,定義$E$與$F$的主夾角(principal angles)為$N_PE$與$N_PF$的主夾角。
\sssc{點在仿射子空間的正射影點與對稱點}
$\mathbb{R}^n$中,一$(n-1)$維仿射子空間$E\colon\mathbf{n}\cdot\mathbf{x}+c=0$與一點$P$,$E$上距離$\mathbf{P}$最短的點為$\mathbf{Q}$,稱$\mb{P}$在$E$的正射影點/投影點,則:
\[\mathbf{Q}=\mathbf{P}-\frac{\mathbf{n}\cdot\mathbf{P}+c}{\abs{\mathbf{n}}^2}\mathbf{n}\]
$E$與$\mathbf{P}$距離為:
\[\abs{\mathbf{P}-\mathbf{Q}}=\frac{\abs{\mathbf{n}\cdot\mathbf{P}+c}}{\abs{\mathbf{n}}}\]
稱$2\mb{Q}-\mb{P}$為$\mb{P}$以$E$為對稱面的對稱點。
\sssc{平行仿射子空間間最短向量}
$\mathbb{R}^n$中,兩平行$(n-1)$維仿射子空間$E\colon\mathbf{n}\cdot\mathbf{x}+c=0$與$F\colon\mathbf{n}\cdot\mathbf{x}+d=0$,$F$上一點$\mathbf{P}$,$E$上距離$\mathbf{P}$最短的點為$\mathbf{Q}$,則:
\[\mathbf{Q}=\mathbf{P}-\frac{c-d}{\abs{\mathbf{n}}^2}\mathbf{n}\]
$E$與$F$距離為:
\[\abs{\mathbf{P}-\mathbf{Q}}=\frac{\abs{c-d}}{\abs{\mathbf{n}}}\]
\sssc{不平行仿射子空間夾角與分角面}
$\mathbb{R}^n$中,兩不平行$(n-1)$維仿射子空間$E\colon\mathbf{m}\cdot\mathbf{x}+c=0$與$F\colon\mathbf{n}\cdot\mathbf{x}+d=0$,則:
\begin{enumerate}
\item $E$、$F$夾角,稱兩面角,餘弦值為:
\[\pm\frac{\mathbf{m}\cdot\mathbf{n}}{\abs{\mathbf{m}}\cdot\abs{\mathbf{n}}}\]
若$\mathbf{m}\cdot\mathbf{n}\neq 0$,則$\pm$取與$\mathbf{m}\cdot\mathbf{n}$同號者為銳角餘弦值,取與$\mathbf{m}\cdot\mathbf{n}$異號者為鈍角餘弦值。
\item $E$、$F$的角平分/分角仿射子空間為:
\[\frac{\mathbf{m}\cdot\mathbf{x}+c}{\abs{\mathbf{m}}}=\pm\frac{\mathbf{n}\cdot\mathbf{x}+d}{\abs{\mathbf{n}}}\]
若$\mathbf{m}\cdot\mathbf{n}\neq 0$,則$\pm$取與$\mathbf{m}\cdot\mathbf{n}$同號者為鈍角分角面,取與$\mathbf{m}\cdot\mathbf{n}$異號者為銳角分角面。
\item $E$、$F$的交集為一$(n-2)$維仿射子空間,稱稜。
\end{enumerate}
\sssc{點積/內積的幾何意義}
兩向量$\mb{a}$、$\mb{b}$夾角$\theta$,則
\[\mb{a}\cdot\mb{b}=|\mb{a}\mb{b}\cos\theta|.\]
\sssc{多點共仿射子空間}
$\mathbb{R}^n$中,相異$k\leq n$點必共一$(k-1)$維仿射子空間。
\sssc{多點決定超體積域與仿射子空間}
$\mathbb{R}^n,\quad n\geq k$中,不共$(k-2)$維仿射子空間的$k$點的集合,可以決定一個$(k-1)$維超體積域,即其凸包,與一個$k$維仿射子空間,即其仿射包。
\sssc{線性組合}
$\mathbb{R}^n$中,不共$(n-2)$維仿射子空間的$n$點$P_1,P_2,\ldots,P_n$共一$(n-1)$維仿射子空間$E$,若:
\[P_n=\sum_{i=1}^{n-1}c_iP_i\]
且原點不在$E$上,則:
\[\sum_{i=1}^{n-1}c_i=1\]
(若原點在$E$上則不一定成立)
\sssc{分點公式(Section formula)/加權重心公式}
令$U$為所有由$\mathbb{R}^{n-1}$中不共$(n-2)$維仿射子空間的$n$點形成的序列形成的集合,$V$為所有由和為$1$的$n$個非零實數形成的序列形成的集合,$W$為所有由和為$1$的$n$個正實數形成的序列形成的集合,定義函數$f\colon U\to\mathbb{R}$使得$f(x)$為$x$的凸包的超體積,定義函數$g\colon S\to U,\quad S\subseteq U\times\mathbf{R}\times\mathbb{R}^{n-1}$使得$g(x,k,v)$為將$x$中的第$k$個元素換成$v$形成的序列,令序列$C$的第$k$個元素為$C_k$,則$\forall C\in V\land A\in U$:
\[K=\sum_{k=1}^nC_kA_k\iff\left(\forall j,k\in\mathbb{N}\land j,k\leq n\colon\frac{f\left(g(A,j,K)\right)}{c_j}=\frac{f\left(g(A,k,K)\right)}{c_k}\right)\]
$K$在$A$的仿射包中,特別地,若$C\in W$則$K$在$A$的凸包中。
\sssc{正射影圖形體積}
$\mathbb{R}^n$中,二$m$維仿射子空間$E$、$F$相交,其中$m\in\mathbb{N}\land m<n$,且$E$、$F$有主夾角$\theta_1,\theta_2,\ldots,\theta_m$,則$E$上一個$m$維圖形在$F$上的正射影的體積除以其原本體積為:
\[\prod_{i=1}^{m}\abs{\cos\theta_i}.\]
\sssc{超三角錐與超平行柱體積}
$\mathbb{R}^n$中,$n$個向量形成的$n\times n$矩陣為$M$,則它們所張的超三角錐體積為$\frac{\abs{\det(M)}}{n!}$,它們所張的超平行柱體積為$\det(M)$。
\sssc{過一點平面與座標軸所圍超三角錐體積最小值}
$\mathbb{R}^n$中,過一點$\mathbf{P}=(p_1,p_2,\ldots,p_n)$的$n-1$維仿射子空間$E$與所有座標軸所圍成的超三角錐體積$V$,其中$\prod_{i=1}^np_i\neq 0$,則當$E$為:
\[\sum_{i=1}^n\frac{x_i}{np_i}=1\]
時,即第$i$軸截距$np_i$時,$V$有最小值:
\[\frac{n^n\prod_{i=1}^n\abs{p_i}}{n!}.\]
\begin{proof}\mbox{}\\
設$E$為:
\[\sum_{i=1}^nn_ix_i=1\]
則$E$與各軸的交點為$\frac{\mb{e}_1}{n_1},\frac{\mb{e}_2}{n_2},\ldots,\frac{\mb{e}_n}{n_n}$,
\[V=\frac{1}{n!\prod_{i=1}^n\abs{n_i}}\]
限制條件:
\[\sum_{i=1}^nn_ip_i=1\]
利用拉格朗日乘數法:
\[\mathcal{L}\coloneq\frac{1}{n!\prod_{i=1}^nn_i}+\lambda\sum_{i=1}^nn_ip_i-\lambda\]
\[\frac{\partial\mathcal{L}}{\partial n_j}=\frac{-1}{n!\prod_{i=1}^nn_in_j}+\lambda p_j=0\]
\[C\coloneq n_jp_j=\frac{1}{\lambda  n!\prod_{i=1}^nn_i}\]
\[\sum_{i=1}^nn_ip_i=Cn=1\]
\[C=\frac{1}{n}\]
\[n_j=\frac{1}{np_j}\]
\end{proof}
\sssc{體積經線性變換}
$\mathbb{R}^n$中,經$n\times n$階矩陣$\mathbf{A}$的線性變換後,任一$n$維圖形的體積會變為原來的$|\det(\mathbf{A})|$倍。
\ssc{二維空間}
\sssc{直線}
\tb{一般式}:
\[ax+by+c=0\]
\tb{截距式}:
$ab\neq 0$
\[\frac{x}{p}+\frac{y}{q}=1\]
其中$p=-\frac{c}{a}$、$q=-\frac{c}{b}$分別為$x$、$y$截距。

\tb{斜截式}:
$b\neq 0$
\[y=mx+q\]
其中$m=-\frac{a}{b}$為斜率。

\tb{點斜式}:
$b\neq 0$
\[y-y_0=m(x-x_0)\]
其中$(x_0,y_0)$為該直線上任一點。
\sssc{直線、射線與線段參數式}
設相異兩點$A(x_1,y_1)$、$B(x_2,y_2)$:
\bit
\item 直線$\olra{AB}$的參數式:
\[\bcs x=x_1+(x_2-x_1)t\\y=y_1+(y_2-y_1)t\ecs,\quad t\in\mathbb{R}\]
\item 射線$\ora{AB}$的參數式:
\[\bcs x=x_1+(x_2-x_1)t\\y=y_1+(y_2-y_1)t\ecs,\quad t\geq 0\]
\item 線段$\ol{AB}$的參數式:
\[\bcs x=x_1+(x_2-x_1)t\\y=y_1+(y_2-y_1)t\ecs,\quad 0\leq t\leq 1\]
\eit
\sssc{分點公式擴展圖形}
\[\begin{aligned}
&\ora{AP}\coloneq x\ora{AB}+y\ora{AC},\quad x,y\in\mathbb{R}\\
\Rightarrow &\bcs
x+y=1&\iff P\tx{\ is on\ }\olra{BC},\\
x+y=1\land xy\geq 0&\iff P\tx{\ is on\ }\ol{BC},\\
x=0&\iff P\tx{\ is on\ }\olra{AC},\\
0\leq y\leq 1\land x=0&\iff P\tx{\ is on\ }\ol{AC},\\
y=0&\iff P\tx{\ is on\ }\olra{AB},\\
0\leq x\leq 1\land y=0&\iff P\tx{\ is on\ }\ol{AB},\\
x+y<1\land x>0\land y>0&\iff P\tx{\ is in\ }\Delta ABC\tx{(不含邊界)},\\
x+y>1\lor x<0\lor y<0&\iff P\tx{\ is outside of\ }\Delta ABC
\ecs
\end{aligned}\]
\sssc{兩直線關係}
直線$\mathbf{L}\colon ax+by+c=0$、$\mathbf{M}\colon dx+ey+f=0$:
\bit
\item 平行(含重合):$ae=bd$
\item 垂直:$ad+be=0$
\eit
直線$\mathbf{F}\colon y=mx+p$、$\mathbf{G}\colon y=nx+q$:
\bit
\item 平行(含重合):$m=n$
\item 垂直:$mn=-1$
\eit
\section{三維空間}
\sssc{叉積/外積的幾何意義}
兩向量$\mb{a}$、$\mb{b}$夾角$\theta$,則
\[|\mb{a}\times\mb{b}|=|\mb{a}\mb{b}\sin\theta|.\]
\sssc{四面體與平行六面體體積}
向量$\mathbf{A},\mathbf{B},\mathbf{C}$所張四面體體積為:
\[\frac{1}{6}\abs{\mathbf{A}\cdot(\mathbf{B}\times\mathbf{C})}\]
所張平行六面體體積為:
\[\abs{\mathbf{A}\cdot(\mathbf{B}\times\mathbf{C})}\]
\sssc{三垂線定理}
設相異點$B$、$C$和直線$L$均在平面$E$上,$A$不在$E$上。如果下列三個命題中有兩個成立,則剩下的一個也必定成立。
\[\ol{AB} \perp E\]
\[\ol{BC} \perp L \text{\ at\ } C\]
\[\ol{AC} \perp L \text{\ at\ } C\]
\sssc{兩歪斜線}
兩互相歪斜的直線$L\colon(x_0,y_0,z_0)+(a,b,c)t,\quad t\in\mathbb{R}$、$M\colon(x_1,y_1,z_1)+(d,e,f)k,\quad k\in\mathbb{R}$間必存在唯一公垂線,令為$S$,令$\mathbf{u}=(a,b,c)$,$\mathbf{v}=(d,e,f)$,$S$分別交$L$、$M$於$P$、$Q$,則:
\[\ora{PQ}\cdot\mathbf{u}=\ora{PQ}\cdot\mathbf{v}=0\]

令平行兩平面$E$、$F$分別包含$L$、$M$,則該二平面之法向量平行於$\qty(\mathbf{u}\times\mathbf{v})$,$M$上任一點與$E$距離=$L$上任一點與$F$距離=$L$與$M$的距離=$\ol{PQ}$。
\sssc{各體公式}
\[\tx{柱體體積}=\tx{底面積}\times \tx{高}\]
\[\tx{錐體體積}=\frac{1}{3}\tx{底面積}\times \tx{高}\]
\[\tx{球體體積}=\f{4}{3}\pi\times\tx{半徑}^3\]
\[\tx{球體表面積}=4\pi\times\tx{半徑}^2\]
\end{document}