\documentclass[a4paper,12pt]{article}
\setcounter{secnumdepth}{5}
\setcounter{tocdepth}{3}
\input{/usr/share/latex-toolkit/template.tex}
\begin{document}
\title{Complex Number}
\author{沈威宇}
\date{\temtoday}
\titletocdoc
\sct{Complex Number}
\ssc{Complex number (複數)}
\sssc{Complex Number}
The set of complex numbers $\bbC$ is defined as
\[\bbC\coloneq\{a+bi\mid a,b\in\bbR\},\]
where $i$ is defined as the only complex number such that $i^2=-1$ and is called the imaginary unit.

For a complex number $z=a+bi$ with $a,b\in\bbR$, $a$ is called its real part (實部) and denoted as $\Re(z)$, $\mathrm{Re}(z)$, or $\mathcal{Re}(z)$, $b$ is called its imaginary part (虛部) and denoted as $\Im(z)$, $\mathrm{Im}(z)$, or $\mathcal{Im}(z)$, $z$ is called an imaginary number (虛數) if $b\neq 0$, and $z$ is called a pure imaginary number (純虛數) if $a=0$ and $b\neq 0$.
\sssc{Addition, subtraction, and multiplication}
For two complex numbers $y=a+bi$ and $z=c+di$ with $a,b,c,d\in\bbR$:
\[y+z=(a+c)+(b+d)i.\]
\[y-z=(a-c)+(b-d)i.\]
\[y\cdot z=(ac-bd)+(ad+bc)i.\]
\sssc{Absolute value (絕對值), magnitude, radial distance (向徑), or modulus (模/模長)}
The absolute value, magnitude, radial distance, or modulus of a complex number $z=a+bi$ with $a,b\in\bbR$ is $\abs{z}=\sqrt{a^2+b^2}$.
\sssc{Conjugate (共軛)}
The conjugate of a complex number $z=a+bi$ with $a,b\in\bbR$, denoted as $\ol{z}$ and called z bar, is defined as $a-bi$.
\sssc{Division}
\[\frac{1}{z}=\frac{\bar{z}}{|z|^2},z\neq 0\]
\sssc{Properties}
For complex numbers $z,z_1,z_2$:
\[|z|=\sqrt{z\cdot\ol{z}}\]
\[\ol{z_1}+\ol{z_2}=\ol{z_1+z_2}.\]
\[\ol{z_1}-\ol{z_2}=\ol{z_1-z_2}.\]
\[\ol{z_1\cdot z_2}=\ol{z_1}\cdot\ol{z_2}.\]
\[\Re(\ol{z_1}\cdot z_2)=\Re(z_1\cdot\ol{z_2}).\]
\[\frac{\ol{z_1}}{\ol{z_2}}=\ol{\qty(\frac{z_1}{z_2})},\quad z_2\neq 0.\]
\[\ol{z^n}=\qty(\ol{z})^n,\quad zn\neq 0\land n\in\bbZ.\]
\ssc{Polynomials}
\sssc{Fundamental theorem of algebra (代數基本定理), d'Alembert's theorem, or the d'Alembert–Gauss theorem}
Every non-constant single-variable polynomial with complex coefficients has at least one complex root.

Every non-zero, single-variable, degree n polynomial with complex coefficients has, counted with multiplicity, exactly n complex roots.
\sssc{Equality of polynomials with real coefficients of complex number and its conjugate}
For all polynomials with real coefficients $f$, for all complex numbers $z$:
\[\ol{f(z)}=f\qty(\ol{z}).\]
Pair of imaginary roots (虛根成對定理): If a polynomial with real coefficients has an imaginary root, then its conjugate is also a root. Thus, an odd-degree polynomial with real coefficients must have at least one real root.
\subsection{Argument}
\subsubsection{Argument (輻角)}
For a complex number $z\neq 0$, the argument of $z$, denoted as $\arg(z)$, is definited as any real number $\varphi $ such that
\[z=|z|e^{i\varphi}.\]
\subsubsection{Principal argument (輻角主值/主輻角)}
For a complex number $z=\neq 0$, the principal argument of $z$, denoted as $\Arg(z)$, is definited as the argument of $z$ with in $(-\pi,\pi]$. (Some sources define it as the argument of $z$ with in $[0,2\pi)$.)
\sssc{Polar form (極式)}
The polar form of a complex number $z$ is
\[z=|z|(\cos\theta+i\sin\theta),\quad\theta\in\bbR.\]
\sssc{Properties}
For any real number $\theta,\varphi,a,b$:
\[\ol{e^{i\theta}}=e^{-i\theta}.\]
\[(ae^{i\theta})(be^{i\varphi})=(ab)e^{i(\theta+\varphi)}.\]
\[\frac{ae^{i\theta}}{be^{i\varphi}}=\frac{a}{b}e^{i(\theta-\varphi)}.\]
\ssc{Complex plane (複數平面), Argand plane (阿爾岡平面), or Gaussian plane (高斯平面)}
\sssc{Complex plane, Argand plane, or Gaussian plane}
The complex plane is the plane formed by the complex numbers, with a Cartesian coordinate system such that the horizontal x-axis, called the real axis, is formed by the real numbers, and the vertical y-axis, called the imaginary axis, is formed by the imaginary numbers.
\sssc{De Moivre's formula (隸美弗公式)}
\[(r(\cos\theta+i\sin\theta))^n=r^n(\cos(n\theta)+i\sin(n\theta)),\quad r\neq 0\land n\in\mathbb{Z}.\]
\sssc{Hyperbolic De Moivre's formula}
\[(r(\cosh\theta+i\sinh\theta))^n=r^n(\cosh(n\theta)+i\sinh(n\theta)),\quad r\neq 0\land n\in\mathbb{Z}.\]
\sssc{Dot product}
The dot product of two vectors in $\bbR^2$ corresponding to the two complex numbers $z_1$ and $z_2$ on the complex plane is $\Re(z_1\cdot\ol{z_2})$.
\ssc{Geometric meaning of exponentiations of complex numbers on complex plane}
\sssc{Raciporial of positive number exponentiation of 1}
For $n\in\bbN$, $1^{\frac{1}{n}}$, that are, the $n$ roots of $x^n=1$, that are, $\omega=\cos\frac{2\pi}{n}+i\sin\frac{2\pi}{n}$ to the power of $0$ to $(n-1)$, that are,
\[1^{\frac{1}{n}}=e^{i\frac{2\pi k}{n}},\quad k\in\mathbb{N}_0\land k<n,\]
are the $n$ vertices of a regular $n$-gon inscribed in a unit circle centered at the origin including $1$ on the complex plane.
\sssc{Raciporial of positive number exponentiation of nonzero complex number}
For $n\in\bbN$ and $z\in\bbC_{\neq 0}$, $z^{\frac{1}{n}}$, that are, the $n$ roots of $x^n=z$, that are, $\sqrt[n]{|z|}$ multiplied by $\omega=\cos\frac{2\pi}{n}+i\sin\frac{2\pi}{n}$ to the power of $0$ to $(n-1)$, that are,
\[z^{\frac{1}{n}}=\sqrt[n]{|z|}e^{i\frac{\Arg(z)+2\pi k}{n}},\quad k\in\mathbb{N}_0\land k<n,\]
are the $n$ vertices of a regular $n$-gon inscribed in a circle of radius $\sqrt[n]{|z|}$ centered at the origin including $\sqrt[n]{|z|}e^{i\frac{\Arg(z)}{n}}$ on the complex plane.
\sssc{Nonzero complex number exponentiation of nonzero complex number}
For $w,z\in\bbC_{\neq 0}$, $z^w$, that are,
\[\begin{aligned}
z^w&=e^{w\ln(z)}\\
&=\left(|z|e^{i\qty(\Arg(z)+2\pi k)}\right)^w\\
&=|z|^{\Re(w)}|z|^{\Im(w)i}e^{i\Re(w)\qty(\Arg(z)+2\pi k)}e^{-\Im(w)\qty(\Arg(z)+2\pi k)}\\
&=|z|^{\Re(w)}e^{-\Im(w)\qty(\Arg(z)+2\pi k)+i(\Re(w)\qty(\Arg(z)+2\pi k)+\Im(w)\ln|z|)},\\
&\quad k\in\mathbb{Z},
\end{aligned}\]
are points with equal angle difference $\Re(w)2\pi$ on the spiral
\[|z|^{\Re(w)}e^{-\Im(w)\theta}\left(\cos(\Re(w)\theta+\Im(w)\ln|z|)+i\sin(\Re(w)\theta+\Im(w)\ln|z|)\right)\]
of parameter $\theta\in\bbR$ including that of $\theta=\Arg(z)$ on the complex plane.

\tb{Modulus:}
\[\abs{z^w}=|z|^{\Re(w)}e^{-\Im(w)\qty(\Arg(z)+2\pi k)}.\]
\begin{itemize}
\item When $\Im(w) > 0$: The modulus decays exponentially with respect to $k$, approaching zero as $k$ increases and diverging to infinity as $k$ decreases.
\item When $\Im(w) < 0$: The modulus increases exponentially with respect to $k$, diverging to infinity as $k$ increases and approaching zero as $k$ decreases.
\item When $\Im(w) = 0$: The modulus is always $|z|^{\Re(w)}$, lying on a circle with radius $|z|^{\Re(w)}$ centered at the origin.
\end{itemize}
\tb{Angle (parameter $\theta$):}
\[\theta(k)=\Re(w)\qty(\Arg(z)+2\pi k)+\Im(w)\ln|z|.\]
\begin{itemize}
\item When $\Re(w) > 0$: The angle increases linearly with respect to $k$, rotating counterclockwise as $k$ increases and clockwise as $k$ decreases.
\item.When $\Re(w) < 0$: The angle decreases linearly with respect to $k$, rotating clockwise as $k$ increases and counterclockwise as $k$ decreases.
\item When $\Re(w) \in \mathbb{Z}$: The angle is always $\left(\Re(w)\Arg(z)+\Im(w)\ln|z|\right)\mod (2\pi)$, lying one the ray with angle $\Re(w)\Arg(z)+\Im(w)\ln|z|$.
\end{itemize}
\tb{Spiral direction:}
\begin{itemize}
\item When $\Re(w)\Im(w)>0$: The root lies in a clockwise spiral centered at the origin on the complex plane.
\item When $\Re(w)\Im(w)<0$: The root lies in a counterclockwise spiral centered at the origin on the complex plane.
\item When $\Im(w)=0$: The root lies in a circle centered at the origin on the complex plane.
\end{itemize}
\tb{Number of distinct roots:}
\begin{itemize}
\item When $\Im(w)=0\land\Re(w)\in\mathbb{Q}$: Suppose $\abs{\Re(w)}=\frac{m}{n}$ with $n\in\mathbb{N}$ and $\gcd(m,n)=1$, there are $n$ distinct roots, which are the $n$ vertices of a regular $n$-gon inscribed in a circle of radius $|z|^{\Re(w)}$ centered at the origin including $|z|^{\Re(w)}e^{i\Re(w)\Arg(z)}$ on the complex plane.
\item Otherwise: There are countably infinitely many distinct roots.
\end{itemize}
\end{document}