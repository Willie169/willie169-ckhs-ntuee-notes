\documentclass[a4paper,12pt]{report}
\setcounter{secnumdepth}{5}
\setcounter{tocdepth}{3}
\input{/usr/share/LaTeX-ToolKit/template.tex}
\begin{document}
\title{Trigonometric Functions}
\author{沈威宇}
\date{\temtoday}
\titletocdoc
\renewcommand{\arraystretch}{1.5}
\sct{Trigonometric Functions (三角函數)}
\subsection{Trigonometric Ratios (三角比) or Trigonometric Functions}
\sssc{Right triangle definitions for acute angles}
Given a right triangle with an acute angle being theta:
\begin{itemize}
  \item Sine ($\sin$) (正弦):
      \[\sin \theta = \frac{\text{opposite}}{\text{hypotenuse}}\]
  \item Cosine ($\cos$) (餘弦):
      \[\cos \theta = \frac{\text{adjacent}}{\text{hypotenuse}}\]
  \item Tangent ($\tan$) (正切):
      \[\tan \theta = \frac{\text{hypotenuse}}{\text{adjacent}}\]
  \item Cotangent ($\cot$) (餘切):
      \[\cot \theta = \frac{1}{\tan \theta}\]
  \item Secant ($\sec$) (正割):
      \[\sec \theta = \frac{1}{\cos \theta}\]
  \item Cosecant ($\csc$) (餘割):
      \[\csc \theta = \frac{1}{\sin \theta}\]
\end{itemize}
\sssc{Unit-circle definitions and properties}
Let $\Gamma$ be the ray obtained by rotating by an angle $\theta$ the positive half of the $x$-axis (counterclockwise rotation for $\theta>0$ and clockwise rotation for $\theta<0$). This ray intersects the unit circle with center $O=(0,0)$ at the point $A=(x_A,y_A)$. Let the line extended to by $\Gamma$ intersectes $x=1$ at $B=(1,y_B)$ and $y=1$ at $C=(x_C,1)$. The tangent line to the unit circle at $A$ intersects the $x$- and $y$-axes at points $D=(x_D,0)$ and $E=(0,y_E)$.
{\fontsize{6pt}{8pt}\selectfont
\begin{longtable}[c]{|p{0.05\textwidth}|p{0.04\textwidth}|p{0.04\textwidth}|p{0.12\textwidth}|p{0.12\textwidth}|p{0.05\textwidth}|p{0.05\textwidth}|p{0.15\textwidth}|p{0.18\textwidth}|}
    \hline
    Function & Symbol & Definition & Domain & Range & Period & Odd or even & Reflectional symmetric about & Rotationally symmetric of order 2 about \\\hline\endhead
    Sine & $\sin$ & $y_A$ & $\mathbb{R}$ & $[-1,1]$ & $2\pi$ & odd & $x\in\left\{x\middle|\qty(\frac{x}{\pi}-\frac{1}{2})\in\bbZ\right\}$ & \\\hline
    Cosine & $\cos$ & $x_A$ & $\mathbb{R}$ & $[-1,1]$ & $2\pi$ & even & $x\in\left\{x\middle|\frac{x}{\pi}\in\bbZ\right\}$ & \\\hline
    Tangent & $\tan$ & $y_B$ & $\bbR\setminus\left\{x\middle|\qty(\frac{x}{\pi}-\frac{1}{2})\in\bbZ\right\}$ & $\bbR$ & $\pi$ & odd & & $(x,y)\in\left\{x\middle|\qty(\frac{x}{\pi}-\frac{1}{2})\in\bbZ\right\}\times\{0\}$ \\\hline
    Cotangent & $\cot$ & $x_C$ & $\bbR\setminus\left\{x\middle|\frac{x}{\pi}\in\bbZ\right\}$ & $\bbR$ & $\pi$ & odd & & $(x,y)\in\left\{x\middle|\frac{x}{\pi}\in\bbZ\right\}\times\{0\}$ \\\hline
    Secant & $\sec$ & $x_D$ & $\bbR\setminus\left\{x\middle|\qty(\frac{x}{\pi}-\frac{1}{2})\in\bbZ\right\}$ & $(-\infty,-1]\cup[1,\infty)$ & $2\pi$ & even & $x\in\left\{x\middle|\frac{x}{\pi}\in\bbZ\right\}$ & $(x,y)\in\left\{x\middle|\qty(\frac{x}{\pi}-\frac{1}{2})\in\bbZ\right\}\times\{0\}$ \\\hline
    Cosecant & $\csc$ & $y_E$ & $\bbR\setminus\left\{x\middle|\frac{x}{\pi}\in\bbZ\right\}$ & $(-\infty,-1]\cup[1,\infty)$ & $2\pi$ & odd & $x\in\left\{x\middle|\qty(\frac{x}{\pi}-\frac{1}{2})\in\bbZ\right\}$ & $(x,y)\in\left\{x\middle|\frac{x}{\pi}\in\bbZ\right\}\times\{0\}$ \\\hline
\end{longtable}}
\sssc{Exponential definitions}
\[\ba
\sin x &= \frac{e^{ix}-e^{-ix}}{2i}\\
\cos x &= \frac{e^{ix}+e^{-ix}}{2}\\
\tan x &= -i\frac{e^{2ix}-1}{e^{2ix}+1},\quad x\neq\frac{\pi}{2}+k\pi,k\in\mathbb{Z}\\
\cot x &= i\frac{e^{2ix}+1}{e^{2ix}-1},\quad x\neq k\pi,k\in\mathbb{Z}\\
\sec x &= \frac{2e^{ix}}{e^{2ix}+1},\quad x\neq\pi+2k\pi,k\in\mathbb{Z}\\
\csc x &= i\frac{2e^{ix}}{e^{2ix}-1},\quad x\neq 2k\pi,k\in\mathbb{Z}
\ea\]
\sssc{Power notation}
\[\begin{aligned}
    &\sin^n x\coloneq\begin{cases}\qty(\sin x)^n,\quad & n\geq 0\\\arcsin x,\quad & n=-1\end{cases}\\
    &\cos^n x\coloneq\begin{cases}\qty(\cos x)^n,\quad & n\geq 0\\\arccos x,\quad & n=-1\end{cases}\\
    &\tan^n x\coloneq\begin{cases}\qty(\tan x)^n,\quad & n\geq 0\\\arctan x,\quad & n=-1\end{cases}\\
    &\cot^n x\coloneq\begin{cases}\qty(\cot x)^n,\quad & n\geq 0\\\arccot x,\quad & n=-1\end{cases}\\
    &\sec^n x\coloneq\begin{cases}\qty(\sec x)^n,\quad & n\geq 0\\\arcsec x,\quad & n=-1\end{cases}\\
    &\csc^n x\coloneq\begin{cases}\qty(\csc x)^n,\quad & n\geq 0\\\arccsc x,\quad & n=-1\end{cases}
\ea\]
\sssc{Trigonometric functions of important angles}
{\fontsize{8pt}{10pt}\selectfont
\begin{longtable}[c]{|p{0.1\textwidth}|p{0.1\textwidth}|p{0.2\textwidth}|p{0.2\textwidth}|p{0.2\textwidth}|}
\hline
Radian & Angle & $\sin$ & $\cos$ & $\tan$ \\\hline\endhead
0 & 0° & 0 & 1 & 0 \\\hline
$\frac{\pi}{2}$ & 90° & 1 & 0 & \\\hline
$\pi$ & 180° & 0 & -1 & 0 \\\hline
$\frac{3\pi}{2}$ & 270° & -1 & 0 & \\\hline
$\frac{\pi}{4}$ & 45° & $\frac{\sqrt{2}}{2}$ & $\frac{\sqrt{2}}{2}$ & $1$ \\\hline
$\frac{3\pi}{4}$ & 135° & $\frac{\sqrt{2}}{2}$ & $-\frac{\sqrt{2}}{2}$ & -1 \\\hline
$\frac{\pi}{6}$ & 30° & $\frac{1}{2}$ & $\frac{\sqrt{3}}{2}$ & $\frac{\sqrt{3}}{3}$ \\\hline
$\frac{\pi}{3}$ & 60 ° & $\frac{\sqrt{3}}{2}$ & $\frac{1}{2}$ & $\sqrt{3}$ \\\hline
$\frac{2\pi}{3}$ & 120° & $\frac{\sqrt{3}}{2}$ & $-\frac{1}{2}$ & $-\sqrt{3}$ \\\hline
$\frac{5\pi}{6}$ & 150° & $\frac{1}{2}$ & $-\frac{\sqrt{3}}{2}$ & $-\frac{\sqrt{3}}{3}$ \\\hline
$\frac{\pi}{12}$ & 15° & $\frac{\sqrt{6}-\sqrt{2}}{4}$ & $\frac{\sqrt{6}+\sqrt{2}}{4}$ & $2-\sqrt{3}$ \\\hline
$\frac{5\pi}{12}$ & 75° & $\frac{\sqrt{6}+\sqrt{2}}{4}$ & $\frac{\sqrt{6}-\sqrt{2}}{4}$ & $2+\sqrt{3}$ \\\hline
$\frac{\pi}{10}$ & 18° & $\frac{\sqrt{5}-1}{4}$ & $\frac{\sqrt{10+2\sqrt{5}}}{4}$ & $\frac{\sqrt{5}-1}{\sqrt{10+2\sqrt{5}}}$ \\\hline
$\frac{2\pi}{10}$ & 36° & $\frac{\sqrt{10-2\sqrt{5}}}{4}$ & $\frac{\sqrt{5}+1}{4}$ & $\frac{\sqrt{10-2\sqrt{5}}}{\sqrt{5}+1}$ \\\hline
$\frac{3\pi}{10}$ & 54° & $\frac{\sqrt{5}+1}{4}$ & $\frac{\sqrt{10-2\sqrt{5}}}{4}$ & $\frac{\sqrt{5}+1}{\sqrt{10-2\sqrt{5}}}$ \\\hline
$\frac{4\pi}{10}$ & 72° & $\frac{\sqrt{10+2\sqrt{5}}}{4}$ & $\frac{\sqrt{5}-1}{4}$ & $\frac{\sqrt{10+2\sqrt{5}}}{\sqrt{5}-1}$ \\\hline
& 37° & $\approx 0.6018$ & $\approx 0.7986$ & $\approx 0.7536$ \\\hline
& 53° & $\approx 0.7986$ & $\approx 0.6018$ & $\approx 1.3270$ \\\hline
\end{longtable}
\FB}
\ssc{Inverse trigonometric functions (反三角函數)}
\sssc{Definition}
{\fontsize{8pt}{10pt}\selectfont
\begin{longtable}[c]{|p{0.16\textwidth}|p{0.16\textwidth}|p{0.16\textwidth}|p{0.16\textwidth}|p{0.16\textwidth}|}
\hline
Function & Symbols & Definition & Domain & Range \\ 
\hline\endhead
    Inverse sine (反正弦) & \(y=\arcsin x=\sin^{-1}(x)=\asin(x)\) & \(x=\sin y\) & \([-1,1]\) & \([-\frac{\pi}{2},\frac{\pi}{2}]\) \\ \hline
    Inverse cosine (反餘弦) & \(y=\arccos x=\cos^{-1}(x)=\acos(x)\) & \(x=\cos y\) & \([-1,1]\) & \([0,\pi]\) \\ \hline
    Inverse tangent (反正切) & \(y=\arctan x=\tan^{-1}(x)=\atan(x)\) & \(x=\tan y\) & \(\mathbb{R}\) & \((-\frac{\pi}{2},\frac{\pi}{2})\) \\ \hline
    Inverse cotagent (反餘切) & \(y=\arccot x=\cot^{-1}(x)=\acot(x)\) & \(x=\cot y\) & \(\mathbb{R}\) & \((0,\pi)\) \\ \hline
    Inverse secant (反正割) & \(y=\arcsec x=\sec^{-1}(x)=\asec(x)\) & \(x=\sec y\) & \((-\infty,-1]\cup[1,+\infty)\) & \([0,\frac{\pi}{2})\cup(\frac{\pi}{2},\pi]\) \\ \hline
    Inverse cosecant (反餘割) & \(y=\arccsc x=\csc^{-1}(x)=\acsc(x)\) & \(x=\csc y\) & \((-\infty,-1]\cup[1,+\infty)\) & \([-\frac{\pi}{2},0)\cup(0,\frac{\pi}{2}]\) \\ \hline
\end{longtable}
    \FB}
\sssc{atan2 函數}
$\operatorname{atan2}$ or $\operatorname{arctan2}(y,x)\colon\bbR^2\setminus\{(0,0)\}$在$x>0$時返還$\tan(\theta)=\frac{y}{x}$在$(-\frac{\pi}{2},\frac{\pi}{2})$中的解,在$x<0$、$y\geq 0$時返還$\tan(\theta)=\frac{y}{x}$在$(\frac{\pi}{2},\pi)$中的解,在$x<0$、$y<0$時返還$\tan(\theta)=\frac{y}{x}$在$(-\pi,-\frac{\pi}{2})$中的解,在$x=0$、$y\neq 0$時返還$\frac{y}{\abs{y}}\frac{\pi}{2}$,在$x=y=0$時未定義。
\sssc{Trigonometric functions of inverse trigonometric functions}
{\fontsize{6pt}{8pt}\selectfont
\begin{longtable}[c]{|M{0.1\textwidth}|M{0.12\textwidth}|M{0.12\textwidth}|M{0.12\textwidth}|M{0.12\textwidth}|M{0.12\textwidth}|M{0.12\textwidth}|}
\hline
    \theta & \sin\theta & \cos\theta & \tan\theta & \cot\theta & \sec\theta & \csc\theta \\\hline\endhead
    \arcsin(x) & x & \sqrt{1-x^2} & \frac{x}{\sqrt{1-x^2}},\quad|x|<1 & \frac{\sqrt{1-x^2}}{x},\quad x\neq 0 & \frac{1}{\sqrt{1-x^2}},\quad|x|<1 & \frac{1}{x},\quad x\neq 0 \\\hline
    \arccos(x) & \sqrt{1-x^2} & x & \frac{\sqrt{1-x^2}}{x},\quad x\neq 0 & \frac{x}{\sqrt{1-x^2}},\quad|x|<1 & \frac{1}{x},\quad x\neq 0 & \frac{1}{\sqrt{1-x^2}},\quad|x|<1 \\\hline
    \arctan(x) & \frac{x}{\sqrt{1+x^2}} & \frac{1}{\sqrt{1+x^2}} & x & \frac{1}{x},\quad x\neq 0 & \sqrt{1+x^2} & \frac{\sqrt{1+x^2}}{x},\quad x\neq 0 \\\hline
    \arccot(x) & \frac{1}{\sqrt{1+x^2}} & \frac{x}{\sqrt{1+x^2}} & \frac{1}{x},\quad x\neq 0 & x & \frac{\sqrt{1+x^2}}{x},\quad x\neq 0 & \sqrt{1+x^2} \\\hline
    \arcsec(x) & \frac{\sqrt{x^2-1}}{|x|} & \frac{1}{x} & \sqrt{x^2-1}\operatorname{sgn}\qty(x) & \frac{\operatorname{sgn}\qty(x)}{\sqrt{x^2-1}},\quad|x|>1 & x & \frac{|x|}{\sqrt{x^2-1}} \\\hline
    \arccsc(x) & \frac{1}{x} & \frac{\sqrt{x^2-1}}{|x|} & \frac{\operatorname{sgn}\qty(x)}{\sqrt{x^2-1}},\quad|x|>1 & \sqrt{x^2-1}\operatorname{sgn}\qty(x) & \frac{|x|}{\sqrt{x^2-1}} & x \\\hline
\end{longtable}\FB}
\ssc{Identities}
\sssc{三角函數基本關係}
\begin{center}
\begin{tikzpicture}
  \foreach \a in {0,60,...,300}
    \node (P\a) at (\a:3) {};
  \draw (P0) -- (P60) -- (P120) -- (P180) -- (P240) -- (P300) -- (P0) -- cycle;
  \draw (P0) -- (P180);
  \draw (P60) -- (P240);
  \draw (P120) -- (P300);
  \draw (P0) -- (P300);
  \node[draw, fill=white, minimum size=1cm, anchor=center] at (P0) {$\cot$};
  \node[draw, fill=white, minimum size=1cm, anchor=center] at (P60) {$\cos$};
  \node[draw, fill=white, minimum size=1cm, anchor=center] at (P120) {$\sin$};
  \node[draw, fill=white, minimum size=1cm, anchor=center] at (P180) {$\tan$};
  \node[draw, fill=white, minimum size=1cm, anchor=center] at (P240) {$\sec$};
  \node[draw, fill=white, minimum size=1cm, anchor=center] at (P300) {$\csc$}
  \node[draw, fill=white, minimum size=1cm, anchor=center] at (0,0) {$1$};
\end{tikzpicture}
\end{center}
\bit
\item 名稱:左側三者為正;右側三者為餘;上面二者為弦;中間二者為切;下面二者為割。
\item 餘角關係:以鉛直軸為對稱軸,位於線對稱位置者,左$(\theta)=$右$\qty(\frac{\pi}{2}-\theta)$。
\item 倒數關係:三條通過中心點的線,其兩端者互為倒數,相乘為1。
\item 商數關係:六邊形周上,連續三個頂點形成的連線,其兩端者相乘等於中間者。
\item 平方(Pythagorean)關係:圖中有三個倒正三角形,其上方兩頂點者之平方和等於在下方頂點者之平方。
\eit
\sssc{平移關係}
\[\cos(x)=\sin(x+\frac{\pi}{2})=-\sin(x-\frac{\pi}{2}).\]
\[\cot(x)=-\tan(x+\frac{\pi}{2}).\]
\[\csc(x)=\sec(x-\frac{\pi}{2})=-\sec(x+\frac{\pi}{2}).\]
\[\arccos(x)=-\arcsin(x)+\frac{\pi}{2}.\]
\[\arccot(x)=-\arctan(x)+\frac{\pi}{2}.\]
\[\arccsc(x)=-\arcsec(x)+\frac{\pi}{2}.\]
\sssc{奇變偶不變,正負看象限}
今有函數$f$,已知其為$\sin$、$\cos$、$\tan$、$\sec$、$\csc$、$\cot$之一,且已知$f(\theta)$。欲求$f(\phi)$,其中$\phi=\pm\theta\pm n\frac{\pi}{2},\quad n\in\mathbb{Z}$。
\bit
\item 判斷方法:奇變偶不變,正負看象限。
\item 上句:奇偶指$n$之奇偶,變指倒數,即:若$n$為奇數則令$g(\theta)=\frac{1}{f\qty(\theta)}$,否則令$g(\theta)=f(\theta)$,則$|f(\phi)|=|g(\theta)|$。
\item 下句:象限指假設$[r,\theta]$在第一象限時,$[r,\phi]$之象限。令該象限中任意角度為$\omega$。令$k=\frac{f\qty(\phi)}{g\qty(\theta)}$。則$k=\frac{f\qty(\omega)}{\qty|f(\qty(\omega)|}$,即:
\eit
\begin{longtable}[c]{|p{0.16\textwidth}|p{0.16\textwidth}|p{0.16\textwidth}|p{0.16\textwidth}|p{0.16\textwidth}|}
\hline
\backslashbox{$f$}{象限} & 一 & 二 & 三 & 四 \\\hline
$\sin$ & + & + & - & - \\\hline
$\cos$ & + & - & - & + \\\hline
$\tan$ & + & - & + & - \\\hline
$\csc$ & + & + & - & - \\\hline
$\sec$ & + & - & - & + \\\hline
$\cot$ & + & - & + & - \\\hline
\end{longtable}\FB
\sssc{正切萬能公式}
\[\ba
&\sin\theta=\frac{2\tan\frac{\theta}{2}}{1+\tan^2\frac{\theta}{2}}\\
&\cos\theta=\frac{1-\tan^2\frac{\theta}{2}}{1+\tan^2\frac{\theta}{2}}\\
&\tan\theta=\frac{2\tan\frac{\theta}{2}}{1-\tan^2\frac{\theta}{2}}
\ea\]
\sssc{二倍角公式}
\[\ba
\sin 2\theta&=2\sin\theta\cos\theta\\
\cos 2\theta &=1-2\sin^2\theta\\
&=2\cos^2\theta-1\\
&=\cos^2\theta-\sin^2\theta
\ea\]
\sssc{半角公式或平方化倍角公式}
\[\ba
\sin^2\frac{\theta}{2}  &=\frac{1-\cos \theta}{2}\\
\cos^2\frac{\theta}{2}  &=\frac{1+\cos \theta}{2}\\
\tan^2\frac{\theta}{2}  &=\frac{1-\cos \theta}{1+\cos \theta}\\
\tan\frac{\theta}{2} &=\frac{\sin\theta}{1+\cos\theta}\\
&=\frac{1-\cos\theta}{\sin\theta}\\
&=\frac{1+\sin\theta-\cos\theta}{1+\sin\theta+\cos\theta}\\
&=\csc\theta-\cot\theta
\ea\]
\sssc{三倍角公式}
\[\sin 3\theta=3\sin\theta-4\sin^3\theta\]
\[\cos 3\theta=4\cos^3\theta-3\cos\theta\]
\[\tan 3\theta=\frac{3\tan\theta-\tan^3\theta}{1-3\tan^2\theta}\]
\sssc{和差角公式}
\[\sin\qty(\alpha +\beta)=\sin\alpha\cos\beta +\cos\alpha\sin\beta\]
\[\sin\qty(\alpha -\beta)=\sin\alpha\cos\beta -\cos\alpha\sin\beta\]
\[\cos\qty(\alpha +\beta)=\cos\alpha\cos\beta -\sin\alpha\sin\beta\]
\[\cos\qty(\alpha -\beta)=\cos\alpha\cos\beta +\sin\alpha\sin\beta\]
\[\tan\qty(\alpha +\beta)=\frac{\tan\alpha+\tan\beta}{1-\tan\alpha\tan\beta}\]
\[\tan\qty(\alpha -\beta)=\frac{\tan\alpha-\tan\beta}{1+\tan\alpha\tan\beta}\]
\[\cot\qty(\alpha +\beta)=\frac{\cot\alpha\cot\beta -1}{\cot\alpha +\cot\beta}\]
\[\cot\qty(\alpha -\beta)=\frac{\cot\alpha\cot\beta +1}{\cot\beta -\cot\alpha}\]
\[\sec\qty(\alpha +\beta)=\frac{\sec\alpha\sec\beta}{1-\tan\alpha\tan\beta}=\frac{\csc\alpha\csc\beta}{\cot\alpha\cot\beta-1}\]
\[\sec\qty(\alpha -\beta)=\frac{\sec\alpha\sec\beta}{1+\tan\alpha\tan\beta}=\frac{\csc\alpha\csc\beta}{\cot\alpha\cot\beta+1}\]
\[\csc\qty(\alpha +\beta)=\frac{\csc\alpha\csc\beta}{\cot\alpha+\cot\beta}=\frac{\sec\alpha\sec\beta}{\tan\alpha+\tan\beta}\]
\[\csc\qty(\alpha -\beta)=\frac{\csc\alpha\csc\beta}{\cot\beta-\cot\alpha}=\frac{\sec\alpha\sec\beta}{\tan\alpha-\tan\beta}\]
\sssc{平方化正切平方公式}
\[\sin^2\theta=\frac{\tan^2\theta}{1+\tan^2\theta}\]
\[\cos^2\theta=\frac{1}{1+\tan^2\theta}\]
\[\cot^2\theta=\frac{1}{\tan^2\theta}\]
\[\sec^2\theta=1+\tan^2\theta\]
\[\csc^2\theta=\frac{1+\tan^2\theta}{\tan^2\theta}\]
\sssc{三角形內角正切公式}
\[\tan\alpha+\tan\beta+\tan\qty(\pi-\alpha-\beta)=\tan\alpha\cdot\tan\beta\cdot\tan\qty(\pi-\alpha-\beta)\]
\sssc{正餘弦函數疊合}
\[\qty(a\sin\theta +b\cos\theta)^2\leq a^2+b^2,\quad a,b\in\mathbb{R}\]
\bma
a\sin x+b\cos x
&= \sqrt{a^2+b^2}\sin\qty(x+\tan^{-1}\left(\frac{b}{a}\right))\\
&= \sqrt{a^2+b^2}\cos\qty(x-\tan^{-1}\left(\frac{a}{b}\right))
\eam
\sssc{和差化積公式}
\bma
\sin\alpha +\sin\beta &= 2\sin\frac{\alpha +\beta}{2}\cos\frac{\alpha -\beta}{2}\\
\sin\alpha -\sin\beta &= 2\cos\frac{\alpha +\beta}{2}\sin\frac{\alpha -\beta}{2}\\
\cos\alpha +\cos\beta &= 2\cos\frac{\alpha +\beta}{2}\cos\frac{\alpha -\beta}{2}\\
\cos\alpha -\cos\beta &= -2\sin\frac{\alpha +\beta}{2}\sin\frac{\alpha -\beta}{2}
\eam
\sssc{積化和差公式}
\bma
2\sin\alpha\cos\beta &= \sin (\alpha +\beta)+\sin (\alpha -\beta)\\
2\cos\alpha\sin\beta &= \sin (\alpha +\beta)-\sin (\alpha -\beta)\\
2\cos\alpha\cos\beta &= \cos (\alpha +\beta)+\cos (\alpha -\beta)\\
2\sin\alpha\sin\beta &= -\cos (\alpha +\beta)+\cos (\alpha -\beta)
\eam
\sssc{連加公式}
\[\begin{aligned}
& \sum_{k=1}^{n} \sin(k\theta) = \frac{\sin\left(\frac{n\theta}{2}\right) \sin\left(\frac{(n+1)\theta}{2}\right)}{\sin\left(\frac{\theta}{2}\right)}.\\
& \sum_{k=1}^{n} \cos(k\theta) = \frac{\sin\left(\frac{n\theta}{2}\right) \cdot \cos\left(\frac{(n+1)\theta}{2}\right)}{\sin\left(\frac{\theta}{2}\right)}.
\end{aligned}\]
\sssc{正餘切和等於正餘割積公式}
\[\tan\theta+\cot\theta=\sec\theta\csc\theta\]
\sssc{正餘弦四次方和公式}
\[\sin^4\theta +\cos^4\theta = 1-2\sin^2\cos^2\theta=1-\frac{1}{2}\sin^2(2\theta)\]
\sssc{正餘弦四次方差公式}
\[\sin^4\theta -\cos^4\theta = \sin^2\theta -\cos^2\theta=-\cos(2\theta)\]
\sssc{正餘弦六次方和公式}
\[\sin^6\theta +\cos^6\theta = 1-3\sin^2\cos^2\theta=1-\frac{3}{4}\sin^2(2\theta)\]
\ssc{正弦連乘、餘切連加、餘割平方級數與餘切平方級數公式}
\[\prod_{k=0}^{n-1}\sin\qty(x+\frac{\pi k}{n})=2^{1-n}\sin\qty(nx)\]
\[\sum_{k=0}^{n-1}\cot\qty(x+\frac{\pi k}{n})=n\cot\qty(nx)\]
\[\sum_{k=0}^{n-1}\csc^2\qty(x+\frac{\pi k}{n})=n^2\csc^2\qty(nx)\]
\[\sum_{k=1}^{n-1}\csc^2\frac{\pi k}{n}=\frac{(n-1)(n+1)}{3}\]
\[\sum_{k=1}^{n-1}\cot^2\frac{\pi k}{n}=\frac{(n-1)(n-2)}{3}\]
\begin{proof}
\[\begin{aligned}
\prod_{k=0}^{n-1}\sin\qty(x+\frac{\pi k}{n})\\
&=\prod_{k=0}^{n-1}\frac{i}{2}\qty(e^{-i\qty(x+\frac{\pi k}{n})}-e^{i\qty(x+\frac{\pi k}{n})})\\
&=i^n2^{-n}\prod_{k=0}^{n-1}e^{-i\qty(x+\frac{\pi k}{n})}\prod_{k=0}^{n-1}(1-e^{2i\qty(x+\frac{\pi k}{n})})\\
&=i^n2^{-n}e^{-inx}e^{-i\pi\qty(\frac{n-1}{2})}
\prod_{k=0}^{n-1}(1-e^{2i\qty(x+\frac{\pi k}{n})})\\
&=i^n2^{-n}e^{-inx}i^{1-n}\prod_{k=0}^{n-1}(1-e^{2i\qty(x+\frac{\pi k}{n})})
\end{aligned}\]
考慮:
\[f(t)=t^n-e^{2inx}\]
$f(t)=0$的根為:
\[t=e^{2i\qty(x+\frac{\pi k}{n})},\quad k\in\mathbb{N}_0\land k<n\]
故:
\[f(t)=\prod_{k=0}^{n-1}(t-e^{2i\qty(x+\frac{\pi k}{n})})\]
\[\prod_{k=0}^{n-1}(1-e^{2i\qty(x+\frac{\pi k}{n})})=1-e^{2inx}\]
代回:
\[\begin{aligned}
\prod_{k=0}^{n-1}\sin\qty(x+\frac{\pi k}{n})\\
&=i^n2^{-n}e^{-inx}i^{1-n}(1-e^{2inx})\\
&=2^{-n}i(e^{-inx}-e^{inx})\\
&=2^{1-n}\sin(nx)
\end{aligned}\]
\[\sum_{k=0}^{n-1}\ln\abs{\sin\qty(x+\frac{\pi k}{n})}=(1-n)\ln(2)+\ln\abs{\sin\qty(nx)}\]
微分兩次:
\[\sum_{k=0}^{n-1}\cot\qty(x+\frac{\pi k}{n})=n\cot\qty(nx)\]
\[\sum_{k=0}^{n-1}\csc^2\qty(x+\frac{\pi k}{n})=n^2\csc^2\qty(nx)\]
\[\sum_{k=1}^{n-1}\csc^2\qty(x+\frac{\pi k}{n})=n^2\csc^2\qty(nx)-\csc^2(x)\]
\[\sum_{k=1}^{n-1}\csc^2\qty(\frac{\pi k}{n})=\lim_{x\to 0}n^2\csc^2\qty(nx)-\csc^2(x)=\frac{(n-1)(n+1)}{3}\]
\[\cot^2(x)=\csc^2(x)-1\]
\[\sum_{k=1}^{n-1}\cot^2\frac{\pi k}{n}=\sum_{k=1}^{n-1}\csc^2\frac{\pi k}{n}-n+1=\frac{n^2-3n+2}{3}\]
\end{proof}
\end{document}
