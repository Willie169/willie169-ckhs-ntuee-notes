\documentclass[a4paper,12pt]{article}
\setcounter{secnumdepth}{5}
\setcounter{tocdepth}{3}
\input{/usr/share/LaTeX-ToolKit/template.tex}
\begin{document}
\title{Binary Relation}
\author{沈威宇}
\date{\temtoday}
\titletocdoc
\sct{Binary Relation (二元關係)}
\ssc{Binary relation}
\sssc{Binary relation or correspondence}
A binary relation or correspondence $R$ over sets $X$ and $Y$ is a subset of $X\times Y$. The set $X$ is called the domain or set of departure of $R$, and the set $Y$ the codomain or set of destination of $R$. In order to specify the choices of the sets $X$ and $Y$, some authors define a binary relation or correspondence as an ordered triple $(X,Y,G)$, where $G$ is a subset of $X\times Y$ called the graph of the binary relation. The statement $(x,y)\in R$ reads "$x$ is $R$-related to $y$" and is denoted by $xRy$. The domain of definition or active domain of $R$ is the set of all $x\in X$ such that there exists $y\in Y$ such that $xRy$. The codomain of definition, active codomain, image, or range of $R$ is the set of all $y\in Y$ such that there exists $x\in X$ such that $xRy$. The field of $R$ is the union of its domain of definition and its codomain of definition.
\sssc{Transitive (binary) relation
{{{
\ssc{Equivalence relation (等價關係)}
\sssc{Equicalence relation}
An equivalence relation on a set $X$ is a binary relation $\cong$ on $X$ satisfying the three properties:
\bit
\item Reflexivity:
\[\forall a\in X\colon a\cong a,\]
\item Symmetry:
\[\forall a,b\in X\colon a\cong b\implies b\cong a,\]
\item Transitivity:
\[\forall a,b,c\in X\colon a\cong b\land b\cong a\implies a\cong c.\]
\eit
\sssc{Equivalence classs (等價類)}
An equivalence class of an element $a$ in a set $X$ under an equivalance relation $\cong$ on $X$ is defined as
\[[a]=\{x\in X\mid a\cong x\}.}\]
\sssc{Quotient set (商集)}
The quotient set $Y=X/\cong$ is the set of equivalence classes of elements of $X$ under equivalence relation $\cong$. 
\ssc{Ordered sets}
\sssc{Ordered set (序集)}
An ordered set is an ordered pair $P=(X,\leq )$ (or $P=(X,<)$) consisting of a set $X$, called the ground set of $P$, and an order (序) $\leq$ (or $<$), in which an order is a binary relation between two elements in $X$.

A non-strict order $\leq$ is an order such that $a=b\implies a\leq b$; a strict order $<$ is an order such that $a<b\implies a!=b$. An ordered set defined with a non-strict order is called a non-strictly ordered set; an ordered set defined with a strict order is called a strictly ordered set.

Unless otherwise specified,
\bit
\item an order denoted as $\leq$ is a non-strict order;
\item an order denoted as $<$ is a strict order;
\item an order without notation or explanation that makes it a strict order is considered a non-strict order;
\item for an non-strictly ordered set with order $\leq$, $\geq$ is defined as a binary relation between two elements $a,b$ in it such that $a\leq b\iff b\geq a$;
\item for an strictly ordered set with order $<$, $>$ is defined as a binary relation between two elements $a,b$ in it such that $a<b\iff b>a$;
\item for an strictly ordered set with order $<$, a non-strict order $\leq$ is defined as a binary relation between two elements $a,b$ in it such that $a\leq b\iff a<b\lor a=b$; and
\item for a specefic type $P$ of ordered sets (e.g. preordered set), strictly $P$ (e.g. strictly preordered set) is a type of strictly ordered set defined with the same definition of $P$ but replace non-strict order $\leq$ with strict order $<$ and remove the requirement of antisymmetry (if any).
\eit

When the meaning is clear from context and there is no ambiguity about the order, the groud set itself can sometimes represent the whole ordered set.
\sssc{Preordered set (預序集), preset, or quasiordered set}
A preordered set is an ordered pair $P=(X,\leq )$ consisting of a set $X$ and a preorder (預序), aka, quasiorder, $\leq$ on $X$, that is, for all $a,b,c\in X$ it must satisfy:
\bit
\item Reflexivity: $a\leq a$, and
\item Transitivity: $a\leq b\implies a+c\leq b+c$.
\eit
\sssc{Partially ordered set (偏序集) or poset}
A partially ordered set is an ordered pair $P=(X,\leq )$ consisting of a set $X$ and a partial order (偏序) $\leq$ on $X$, that is, for all $a,b,c\in X$ it must satisfy:
\begin{enumerate}
\item Reflexivity: $a\leq a$, i.e. every element is related to itself.
\item Antisymmetry: $a\leq b\land b\leq a\implies a=b$, i.e. no two distinct elements precede each other.
\item Transitivity: $a\leq b\land b\leq c\implies a\leq c$.
\end{enumerate}
\sssc{Upward closure}
Let $A$ be a subset of a poset $X$, the upward closure of $A$, denoted as $\uparrow A$, is defined as:
\[\uparrow A \coloneq \{ x \in X \mid\exists a \in A \text{\ s.t.\ } a \leq x \}.\]
\sssc{Totally ordered set (全序集) or linearly ordered set (線性順序集)}
A totally ordered set is an ordered pair $P=(X,\leq )$ consisting of a set $X$ and a total order (全序), aka, linear order (線性順序), $\leq$ on $X$, that is, for all $a,b,c\in X$ it must satisfy:
\begin{enumerate}
\item Reflexivity: $a\leq a$, i.e. every element is related to itself.
\item Antisymmetry: $a\leq b\land b\leq a\implies a=b$, i.e. no two distinct elements precede each other.
\item Transitivity: $a\leq b\land b\leq c\implies a\leq c$.
\item Strongly connectivity or totality: $a\leq b\lor b\leq a$.
\end{enumerate}
\sssc{Ordered field (有序域)}
A field $(F,+,\cdot)$ together with a total order $\leq$ on $F$ is an ordered field if the order satisfies the following properties for all $a,b,c\in F$:
\begin{enumerate}
\item $a\leq b\implies a+c\leq b+c$.
\item $0\leq a\land 0\leq b\implies 0\leq a\cdot b$.
\end{enumerate}
As usual, we write $a<b$ for $a\leq b$ and $a\neq b$. The notations $b\geq a$ and $b>a$ stand for $a\leq b$ and $a<b$, respectively. Elements $a\in F$ with $a>0$ are called positive.
\sssc{Well ordered set, well-ordered set, or woset (良序集)}
A well ordered set is a totally ordered set such that every non-empty subset of it has a least element in its order. The order of a woset is called a well order or well-order.
\sssc{Order homomorphism or Order-preserving function}
Given two posets $(S,\leq _{S})$ and $(T,\leq _{T})$, an order homomorphism or order-preserving function from $(S,\leq _{S})$ to $(T,\leq _{T})$ is a function $f$ from $S$ to $T$ with the property that, for every $x,y\in S$, $x \leq_S y$ implies $f(x)\leq _{T}f(y)$.
\sssc{Order isomorphism}
Given two posets $(S,\leq _{S})$ and $(T,\leq _{T})$, an order isomorphism from $(S,\leq _{S})$ to $(T,\leq _{T})$ is a bijective function $f$ from $S$ to $T$ with the property that, for every $x,y\in S$, $x \leq_S y$ if and only if $f(x)\leq _{T}f(y)$.

Two sets are called to be order-isomorphic if there exists an order isomorphism between them.
\sssc{Ordered by inclusion}
A family of sets is called to be ordered by inclusion if it is a poset such that for any two sets $A,B$ in it, $A\leq B\iff A\subseteq B$.
\sssc{Cofinal (共尾的) or frequent}
A subset $B\subseteq A$ of a preordered set $(A,\leq )$ is said to be cofinal or frequent in $A$ if for every $a\in A$, there exists a $b\in B$ such that $a\leq b$.
\sssc{Upper Bound (上界)}
$M$ is an upper bound of a preset $A$ if $a\leq M$ for every $a\in A$. We say a preset is bounded-above by $M$ if it has an upper bound $M$.

$M$ is an upper bound of a function $f$ of which the range is a preset if $M$ is an upper bound of its range. We say a function is bounded-above by $M$ if it has an upper bound $M$.

$M$ is an upper bound of a function $f$ of which the range is a preset on a subset $I$ of its domain if $M$ is an upper bound of $f(I)$. We say a function is bounded-above by $M$ on a subset $I$ of its domain if it has an upper bound $M$ on $I$.
\sssc{Lower Bound (下界)}
$m$ is an lower bound of a preset $A$ if $a\geq m$ for every $a\in A$. We say a preset is bounded-below by $m$ if it has an lower bound $m$.

$m$ is an lower bound of a function $f$ of which the range is a preset if $m$ is an lower bound of its range. We say a function is bounded-below by $m$ if it has an lower bound $m$.

$m$ is an lower bound of a function $f$ of which the range is a preset on a subset $I$ of its domain if $m$ is an lower bound of $f(I)$. We say a function is bounded-below by $m$ on a subset $I$ of its domain if it has an lower bound $m$ on $I$.
\sssc{Supremum or Least Upper Bound (最小上界)}
The supremum or least upper bound of a preset $A$, denoted as $\sup A$, is defined as $M_0$ in the set $\mathcal{M}$ of all upper bounds of $A$ such that $\forall M\in\mathcal{M}\colon M_0\leq M$.

The supremum or least upper bound of a function $f$ of which the range is a preset, denoted as $\sup_xf(x)$, is defined as $M_0$ in the set $\mathcal{M}$ of all upper bounds of $f$ such that $\forall M\in\mathcal{M}\colon M_0\leq M$.
\sssc{Infimum or Greatest Lower Bound (最大下界)}
The infimum or greatest lower bound of a preset $A$, denoted as $\inf A$, is defined as $m_0$ in the set $\mathcal{M}$ of all lower bounds of $A$ such that $\forall m\in\mathcal{M}\colon m_0\geq m$.

The infimum or greatest lower bound of a function $f$ of which the range is a preset, denoted as $\inf_xf(x)$, is defined as $m_0$ in the set $\mathcal{M}$ of all lower bounds of $f$ such that $\forall m\in\mathcal{M}\colon m_0\geq m$.
\sssc{Bounded (有界的)}
We say a preset or a function of which the range is a preset is bounded if it is bounded-above and bounded-below.

We say a function is bounded on a subset $I$ of its domain if it is bounded-above and bounded-below on $I$.
\sssc{Directed set (有向集), directed preset (有向預序集), or filtered set}
A directed set is a preset in which every finite subset has an upper bound.
\sssc{Intervals (區間)}
Unless otherwise specified, an interval defined for $\mathbb{R}$.

For a partially ordered set $(P,\leq)$ and $a,b\in P$, 
\bit
\item a closed interval (閉區間) $[a,b]$ is defined as
\[[a,b]\coloneq\{x\in P\mid a\leq x\leq b\};\]
\item an left-closed upward interval $[a,\infty)$ is defined as
\[[a,\infty)\coloneq\{x\in P\mid a\leq x\};\]
\item a right-closed downward interval $(-\infty,b]$ is defined as
\[(-\infty,b]\coloneq\{x\in P\mid x\leq b\};\]
\item the interval $(-\infty,\infty)$ is defined as
\[(-\infty,\infty)\coloneq P.\]
\eit

For a strictly partially ordered set $(P,<)$ and $a,b\in P$,
\begin{itemize}
\item an open interval (開區間) $(a,b)$ is defined as
\[(a,b)\coloneq\{x\in P\mid a<x<b\};\]
\item a right open interval (右開區間) $[a,b)$ is defined as
\[[a,b)\coloneq\{x\in P\mid a\leq x<b\};\]
\item a left open interval (左開區間) $(a,b]$ is defined as
\[(a,b]\coloneq\{x\in P\mid a<x\leq b\};\]
\item an left-open upward interval $(a,\infty)$ is defined as
\[(a,\infty)\coloneq\{x\in P\mid a<x\};\]
\item a right-open downward interval $(-\infty,b)$ is defined as
\[(-\infty,b)\coloneq\{x\in P\mid x<b\}.\]
\end{itemize}

Right open intervals and left open intervals are collectively called half open intervals (半開區間). All intervals with at least one of the endpoint being $\infty$ or $-\infty$ are collectively called infinite intervals (無限區間).
\ssc{Algebraic structures (代數結構) or algebraic system (代數系統)}
\sssc{Algebraic structure or algebraic system}
An algebraic structure or algebraic system consists of a nonempty set $A$ (called the underlying set, carrier set or domain), a collection of operations on $A$ (typically binary operations such as addition and multiplication), and a finite set of identities (known as axioms) that these operations must satisfy.
\sssc{Monoid (么半群, 單群, or 亞群)}
A monoid is a set equipped with an operation that combines any two elements of the set to produce a third element within the same set such that the following conditions hold: the operation is associative, and it has an identity element.
\sssc{Group (群)}
A group is a set equipped with an operation that combines any two elements of the set to produce a third element within the same set such that the following conditions hold: the operation is associative, it has an identity element, and every element of the set has an inverse element.
\sssc{Projection maps}
The $k$-th projection map for a Cartesian product of $n$ sets:
\[\prod_{i=1}^nX_i\]
is the function $\pi_k$
\[\pi_k\colon\prod_{i=1}^nX_i\to X_k;\;\pi_k(x_1,x_2,\ldots x_n)=x_k.\]
\end{document}
