\documentclass[a4paper,12pt]{article}
\setcounter{secnumdepth}{5}
\setcounter{tocdepth}{3}
\input{/usr/share/LaTeX-ToolKit/template.tex}
\begin{document}
\title{Binary Relation}
\author{沈威宇}
\date{\temtoday}
\titletocdoc
\sct{Binary Relation (二元關係)}
\ssc{Binary relation}
\sssc{Binary relation or correspondence}
A binary relation or correspondence $R$ over sets $X$ and $Y$ is a subset of $X\times Y$. The set $X$ is called the domain or set of departure of $R$, and the set $Y$ the codomain or set of destination of $R$. In order to specify the choices of the sets $X$ and $Y$, some authors define a binary relation or correspondence as an ordered triple $(X,Y,G)$, where $G$ is a subset of $X\times Y$ called the graph of the binary relation. The statement $(x,y)\in R$ reads "$x$ is $R$-related to $y$" and is denoted by $xRy$. The domain of definition or active domain of $R$ is the set of all $x\in X$ such that there exists $y\in Y$ such that $xRy$. The codomain of definition, active codomain, image, or range of $R$ is the set of all $y\in Y$ such that there exists $x\in X$ such that $xRy$. The field of $R$ is the union of its domain of definition and its codomain of definition.
\sssc{Homogeneous relation or endorelation}
A homogeneous relation or endorelation on a set $X$ is a binary relation between $X$ and itself, i.e. it is a subset of $X^2$, commonly phrased as "a (binary) relation on/over X".
\sssc{Transitive relation}
A homogeneous relation $R$ on the set $X$ is a transitive relation if
\[\forall a, b, c \in X\colon (aRb\land bRc)\implies aRc.\]
\sssc{Incomparability}
For a homogeneous relation $R$ on the set $X$, the homogeneous relation "incomparable with respect to $R$" is defined as "$a$ and $b$ in $X$ are incomparable with respect to $R$ if and only if $(\neg aRb)\land (\neg bRa)$".
\sssc{Important properties}
Some important properties that a homogeneous relation $R$ over a set $X$ may have include: for all $a,b,c\in X$,
\bit
\item Reflexive: $aRa$,
\item Irreflexive or anti-reflexive: $\neq aRa$,
\item Symmetric: $aRb\implies bRa$,
\item Asymmetric: $aRb\implies \neg bRa$ (which implies irreflexivity),
\item Antisymmetric: $aRb\land bRa\implies a=b$,
\item Transitive: $(aRb\land bRc)\implies aRc$,
\item Transitivity of incomparability or comparability modulo equivalence: the homogeneous relation "imcomparable with respect to $R$" is transitive, that is, an equivalence relation.
\item Connected: $a\neq b\implies (aRb\lor bRa)$ (which implies transitivity of incomparability),
\item Strongly connected or total: $aRb\lor bRa$ (which implies reflexivity and connectivity),
\item Dense: $aRb\implies (\exists z\in X\text{\ s.t.\ }aRz\land zRb$,
\item Well-founded: for every non-empty subset $S\subseteq X$, there exists some $m\in S$ such that $mRs$ for all $s\in S\setminus\{m\}$.
\eit
\ssc{Equivalence relation (等價關係)}
\sssc{Equivalence relation}
An equivalence relation on a set $X$ is a binary relation on $X$ that is reflexive, symmetric, and transitive, often denoted as $\cong$, $=$, $\equiv$, or $\sim$.
\sssc{Equivalence class (等價類)}
An equivalence class of an element $a$ in a set $X$ under an equivalance relation $\cong$ on $X$ is defined as
\[[a]=\{x\in X\mid a\cong x\}.\]
\sssc{Quotient set (商集)}
The quotient set $Y=X/\cong$ is the set of equivalence classes of elements of $X$ by equivalence relation $\cong$. 
\ssc{Order (序)}
\sssc{Ordered set (序集)}
An ordered set is an ordered pair $P=(X,\leq )$ (or $P=(X,<)$) consisting of a set $X$, called the ground set of $P$, and an order (序), aka ordering, $\leq$ (or $<$), in which an order is a transitive relation on $X$.

When the meaning is clear from context and there is no ambiguity about the order, the ground set itself can sometimes represents the whole ordered set.
\sssc{Preorder (預序), quasiorder, or quasi-order}
A preorder, quasiorder, or quasi-ordering is a transitive relation $\leq$ that is reflexive. 

A set equipped with a preorder is called a preordered set, preset, quasiordered set, or quasi-ordered set.
\sssc{Total preorder (全預序), total quasiorder, or total quasi-order}
A total preorder is a transitive relation $\leq$ that is strongly connected.

A set equipped with a total preorder is called a totally preordered set, totally preset, totally quasiordered set, or totally quasi-ordered set.
\sssc{Partial order (偏序)}
A partial order is a transitive relation $\leq$ that is reflexive and antisymmetric.

A set equipped with a partial order is called a partially ordered set or poset.
\sssc{Upward closure}
Let $A$ be a subset of a poset $X$, the upward closure of $A$, denoted as $\uparrow A$, is defined as:
\[\uparrow A \coloneq \{ x \in X \mid\exists a \in A \text{\ s.t.\ } a \leq x \}.\]
\sssc{Downward closure}
Let $A$ be a subset of a poset $X$, the downward closure of $A$, denoted as $\downarrow A$, is defined as:
\[\downarrow A \coloneq \{ x \in X \mid\exists a \in A \text{\ s.t.\ } a \geq x \}.\]
\sssc{Closed upward/upwards}
A subset $A$ of a of a poset $X$ is called closed upward/upwards iff
\[(x\in X\land y\in A\land x\geq y)\implies (x\in A).\]
\sssc{Closed downward/downwards}
A subset $A$ of a of a poset $X$ is called closed downward/downwards iff
\[(x\in X\land y\in A\land x\leq y)\implies (x\in A).\]
\sssc{Strict partial order, strict preorder, strict quasiorder, or strict quasi-order}
A strict partial order, strict preorder, strict quasiorder, or strict quasi-order is a transitive relation $<$ that is asymmetric.

A set equipped with a strict partial order is called a strictly partially ordered set, strictly preordered set, strictly preset, strictly quasiordered set, or strictly quasi-ordered set.

A strict partial order $<$ can be induced from a preorder $\leq$ with $a<b\iff(a\leq b\land \neg(b\leq a)$. A partial order $\leq$ can be induced from a strict partial order $<$ and an equivalence relation $=$ with $a\leq b\iff(a<b\lor a=b)$.
\sssc{Strict weak order}
A strict weak order is a transitive relation $<$ that is asymmetric and with transitivity of incomparability.

A strict weak order $<$ can be induced from a preorder $\leq$ with $a<b\iff(a\leq b\land \neg(b\leq a)$. A partial order $\leq$ can be induced from a strict weak order $<$ with $a\leq b\iff(a<b\lor a=b)$ where $=$ denotes the equivalence relation "incomparable with respect to $<$".
\sssc{Total order (全序) or linear order (線性順序)}
A total order or linear order is a transitive relation $\leq$ that is strongly connected and antisymmetric.

A set equipped with a total order is called a totally ordered set or linearly ordered set.
\sssc{Strict total order, strict linear order, or strict total preorder}
A strict total order, strict linear order, or strict total preorder is a transitive relation $<$ that is asymmetric and connected.

A set equipped with a strict total order is called a strictly totally ordered set, strictly linearly ordered set, or strictly totally preordered set.

A strict total order $<$ can be induced from a total preorder $\leq$ with $a<b\iff(a\leq b\land \neg(b\leq a)$. A total order $\leq$ can be induced from a strict total order $<$ with $a\leq b\iff(a<b\lor a=b)$ where $=$ denotes the equivalence relation "incomparable with respect to $<$".
\sssc{(Totally/linearly) ordered group (有序群)}
A group $(G,*)$ together with a total order $\leq$ on $F$ is a
\bit
\item left-(totally/linearly) ordered group if $\leq$ is left-invariant, that is
\[\forall a,b,c\in G\colon a\leq b\implies c*a\leq c*b.\]
\item right-(totally/linearly) ordered group if $\leq$ is right-invariant, that is
\[\forall a,b,c\in G\colon a\leq b\implies a*c\leq b*c.\]
\item bi-(totally/linearly) ordered group if $\leq$ is bi-invariant, that is it is both left- and right-invariant.
\eit
If $*$ is commutative, then left-ordred, right-ordered, and bi-ordered groups are the same, called (totally/linearly) ordered group.
\sssc{(Totally/linearly) ordered field (有序域)}
A field $(F,+,\cdot)$ together with a total order $\leq$ on $F$ is a (totally/linearly) ordered field if the order satisfies the following properties for all $a,b,c\in F$:
\[a\leq b\implies a+c\leq b+c,\]
\[0\leq a\land 0\leq b\implies 0\leq a\cdot b.\]
Elements $a\in F$ with $a>0$ are called positive; $a\in F$ with $a<0$ are called negative.
\sssc{Prewellorder (預良序)}
A prewellorder is a transitive relation $\leq$ that is strongly connected and well-founded.

A set equipped with a prewellorder is called a prewellordered set.
\sssc{Well order or well-order (良序)}
A well order or well-order is a transitive relation $\leq$ that is strongly connected, antisymmetric, and well-founded.

A set equipped with a well-order is called a well ordered set, well-ordered set, or woset.
\sssc{Strict well order or strict well-order}
A strict well order or strict well-order is a transitive relation $<$ that is asymmetric, connected, and well-founded.

A set equipped with a strict well order is called a strictly well ordered set or strictly well-ordered set.

A strict well order $<$ can be induced from a prewellorder $\leq$ with $a<b\iff(a\leq b\land \neg(b\leq a)$. A well order $\leq$ can be induced from a strict well order $<$ with $a\leq b\iff(a<b\lor a=b)$ where $=$ denotes the equivalence relation "incomparable with respect to $<$".
\sssc{Dedekind-complete}
A set $X$ equipped with a total order $\leq$ is Dedekind-complete if every nonempty subset $S$ of $X$ with an upper bound in $X$ has a least upper bound (aka supremum) in $X$.
\sssc{Order homomorphism or Order-preserving function}
Given two posets $(S,\leq _{S})$ and $(T,\leq _{T})$, an order homomorphism or order-preserving function from $(S,\leq _{S})$ to $(T,\leq _{T})$ is a function $f$ from $S$ to $T$ with the property that, for every $x,y\in S$, $x \leq_S y$ implies $f(x)\leq _{T}f(y)$.
\sssc{Order isomorphism}
Given two posets $(S,\leq _{S})$ and $(T,\leq _{T})$, an order isomorphism from $(S,\leq _{S})$ to $(T,\leq _{T})$ is a bijective function $f$ from $S$ to $T$ with the property that, for every $x,y\in S$, $x \leq_S y$ if and only if $f(x)\leq _{T}f(y)$.

Two sets are called to be order-isomorphic if there exists an order isomorphism between them.
\sssc{Ordered by inclusion}
A family of sets is called to be ordered by inclusion if it is a poset such that for any two sets $A,B$ in it, $A\leq B\iff A\subseteq B$.
\sssc{Cofinal (共尾的) or frequent}
A subset $B\subseteq A$ of a preordered set $(A,\leq )$ is said to be cofinal or frequent in $A$ if for every $a\in A$, there exists a $b\in B$ such that $a\leq b$.
\sssc{Upper Bound (上界)}
$M$ is an upper bound of a preset $A$ if $a\leq M$ for every $a\in A$. We say a preset is bounded(-)above by $M$ if it has an upper bound $M$.

$M$ is an upper bound of a function $f$ of which the range is a preset if $M$ is an upper bound of its range. We say a function is bounded-above by $M$ if it has an upper bound $M$.

$M$ is an upper bound of a function $f$ of which the range is a preset on a subset $I$ of its domain if $M$ is an upper bound of $f(I)$. We say a function is bounded-above by $M$ on a subset $I$ of its domain if it has an upper bound $M$ on $I$.
\sssc{Lower Bound (下界)}
$m$ is an lower bound of a preset $A$ if $a\geq m$ for every $a\in A$. We say a preset is bounded(-)below by $m$ if it has an lower bound $m$.

$m$ is an lower bound of a function $f$ of which the range is a preset if $m$ is an lower bound of its range. We say a function is bounded-below by $m$ if it has an lower bound $m$.

$m$ is an lower bound of a function $f$ of which the range is a preset on a subset $I$ of its domain if $m$ is an lower bound of $f(I)$. We say a function is bounded-below by $m$ on a subset $I$ of its domain if it has an lower bound $m$ on $I$.
\sssc{Supremum, join, or Least Upper Bound (最小上界)}
The supremum, join, or least upper bound of a preset $A$, denoted as $\sup A$, is defined as $M_0$ in the set $\mathcal{M}$ of all upper bounds of $A$ such that $\forall M\in\mathcal{M}\colon M_0\leq M$.

The supremum or least upper bound of a function $f$ of which the range is a preset, denoted as $\sup_xf(x)$, is defined as $M_0$ in the set $\mathcal{M}$ of all upper bounds of $f$ such that $\forall M\in\mathcal{M}\colon M_0\leq M$.
\sssc{Infimum, meet, or Greatest Lower Bound (最大下界)}
The infimum, meet, or greatest lower bound of a preset $A$, denoted as $\inf A$, is defined as $m_0$ in the set $\mathcal{M}$ of all lower bounds of $A$ such that $\forall m\in\mathcal{M}\colon m_0\geq m$.

The infimum or greatest lower bound of a function $f$ of which the range is a preset, denoted as $\inf_xf(x)$, is defined as $m_0$ in the set $\mathcal{M}$ of all lower bounds of $f$ such that $\forall m\in\mathcal{M}\colon m_0\geq m$.
\sssc{Bounded (有界的)}
We say a preset or a function of which the range is a preset of which the range is a preset is bounded if it is bounded-above and bounded-below.

We say a function of which the range is a preset is bounded on a subset $I$ of its domain if it is bounded-above and bounded-below on $I$.

We say a function $f$ of which the range is a preset is unbounded at a point $a$ in its domain if $f$ is unbounded on all neighborhood of $a$.
\sssc{Join-semilattice}
A join-semilattice is a partially ordered set such that every two-element $\{a,b\}$ subset has a join.
\sssc{Meet-semilattice}
A meet-semilattice is a partially ordered set such that every two-element $\{a,b\}$ subset has a meet.
\sssc{Lattice}
A lattice is a partially ordered set such that every two-element $\{a,b\}$ subset has a join and a meet.
\sssc{Directed set (有向集), directed preset (有向預序集), or filtered set}
A directed set is a preset in which every finite subset has an upper bound.
\sssc{Intervals (區間)}
Unless otherwise specified, an interval defined for $\mathbb{R}$.

For a partially ordered set $(P,\leq)$ and $a,b\in P$, 
\bit
\item a closed interval (閉區間) $[a,b]$ is defined as
\[[a,b]\coloneq\{x\in P\mid a\leq x\leq b\};\]
\item an left-closed upward interval $[a,\infty)$ is defined as
\[[a,\infty)\coloneq\{x\in P\mid a\leq x\};\]
\item a right-closed downward interval $(-\infty,b]$ is defined as
\[(-\infty,b]\coloneq\{x\in P\mid x\leq b\};\]
\item the open interval $(-\infty,\infty)$ is defined as
\[(-\infty,\infty)\coloneq P.\]
\eit

For a strictly partially ordered set $(P,<)$ and $a,b\in P$,
\begin{itemize}
\item an open interval (開區間) $(a,b)$ is defined as
\[(a,b)\coloneq\{x\in P\mid a<x<b\};\]
\item a right open interval (右開區間) $[a,b)$ is defined as
\[[a,b)\coloneq\{x\in P\mid a\leq x<b\};\]
\item a left open interval (左開區間) $(a,b]$ is defined as
\[(a,b]\coloneq\{x\in P\mid a<x\leq b\};\]
\item an left-open upward interval $(a,\infty)$ is defined as
\[(a,\infty)\coloneq\{x\in P\mid a<x\};\]
\item a right-open downward interval $(-\infty,b)$ is defined as
\[(-\infty,b)\coloneq\{x\in P\mid x<b\}.\]
\end{itemize}

Right open intervals and left open intervals are collectively called half open intervals (半開區間). All intervals with at least one of the endpoint being $\infty$ or $-\infty$ are collectively called infinite intervals (無限區間).
\end{document}
