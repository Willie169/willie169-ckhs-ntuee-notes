\documentclass[a4paper,12pt]{report}
\setcounter{secnumdepth}{5}
\setcounter{tocdepth}{3}
\newcounter{ZhRenew}
\setcounter{ZhRenew}{1}
\newcounter{SectionLanguage}
\setcounter{SectionLanguage}{1}
\input{/usr/share/latex-toolkit/template.tex}
\begin{document}
\title{組合數學}
\author{沈威宇}
\date{\temtoday}
\titletocdoc
\section{組合數學(Combinatorics)}
\subsection{計數原理(Counting principle)/基本計數原理(Fundamental principle of counting)}
\begin{itemize}
\item \tb{窮舉法(Proof by exhaustion)}:將所有可能一一列舉出而計算數目的方法。
\item \tb{樹狀圖(Tree structure)}:畫出樹狀圖列舉而計算數目的方法。
\item \tb{乘法原理(Rule of product or multiplication principle)}:若有$a$種方法做事$A$,$b$種方法做事$B$,則合共有$a\cdot b$種方法做事$A\land B$。
\item \tb{加法原理(Rule of sum or addition principle)}:若有$a$種方法做事$A$,$b$種方法做事$B$,則合共有$a+b$種方法做事$A\lor B$。
\end{itemize}
\ssc{冪集計數原理(Counting principle for power sets)}
\[n(2^A)=2^{n\qty(A)}\]
\ssc{排容原理/取捨原理(Principle of inclusion-exclusion or inclusion–exclusion principle, PIE)}
\textit{Statement:}
\[ \left|\bigcup _{i=1}^n A_i\right| = \sum _{\emptyset \neq J\subseteq \{a|a \in \mathbb{N} \wedge 1\leq a\leq n\}}\left((-1)^{|J|-1}\cdot{\Biggl |}\bigcap _{j\in J}A_j{\Biggr |}\right) \]

\tb{以元素證明:}
\begin{proof}
\[
\begin{aligned}
& \left|\bigcup _{i=1}^n A_i\right| = \sum _{\emptyset \neq J\subseteq \{a|a \in \mathbb{N} \wedge 1\leq a\leq n\}}\left((-1)^{|J|-1}\cdot{\Biggl |}\bigcap _{j\in J}A_j{\Biggr |}\right) \\
& \equiv 1_{\bigcup _{i=1}^n A_i} = \sum _{k=1}^{n}\left((-1)^{k-1}\cdot\sum _{I\subseteq \{a|a \in \mathbb{N} \wedge 1\leq a\leq n\} \wedge |I|=k}1_{A_{I}}\right) \\
& \equiv \forall \left( x \in \bigcup_{i=1}^n A_i, m = \sum_{i: x \in A_i}1\right): 1 = \sum _{k=1}^{m}\left((-1)^{k-1}\sum _{I\subset \{a|a \in \mathbb{N} \wedge 1\leq a\leq m\} \wedge |I|=k}1\right) \\
& \equiv \forall \left( x \in \bigcup_{i=1}^n A_i, m = \sum_{i: x \in A_i}1\right): {\binom{m}{0}} = \sum _{k=1}^{m}\left((-1)^{k-1}{\binom {m}{k}}\right) \\
& \equiv \forall \left( x \in \bigcup_{i=1}^n A_i, m = \sum_{i: x \in A_i}1\right): (1-1)^m = 0
\end{aligned}
\]
\end{proof}

\tb{以數學歸納法證明:}
\begin{proof}
\[
\begin{aligned}
& \text{當\ } n=2, |A_1\bigcup A_2| = |A_1|+|A_2| - |A_1\cap A_2)\text{,命題成立;} \\
& \text{假設當\ } n=k\geq 2, k \in \mathbb{N} \text{,命題成立,} \\
& \text{則當\ } n=k+1,\\
& \left| \bigcup_{i=1}^{k+1} A_i \right| \\
& = \left| \left(\bigcup_{i=1}^k A_i\right)\cup A_{k+1} \right| \\
& = \sum _{\emptyset \neq J\subseteq \{a|a \in \mathbb{N} \wedge 1\leq a\leq k+1\}}\left((-1)^{|J|-1}\cdot{\Biggl |}\bigcap _{j\in J}A_j{\Biggr |}\right) \\
& \text{,命題亦成立。} \\
& \text{由數學歸納法,得證。}
\end{aligned}
\]
\end{proof}
\subsection{階乘(Factorial)與伽瑪函數(Gamma function)}
\sssc{階乘(Factorial)}
\[n! = \begin{cases}\prod_{i=1}^n i, & \text{if } n \in \mathbb{N}, \\1, & \text{if } n = 0.\end{cases}\]
\sssc{伽瑪函數(Gamma function)}
階乘函數在正實部複數域上的擴展。
\[\Gamma(z) = \int_0^{\infty} t^{z-1} e^{-t} \, \mathrm{d}t, \quad \Re(z) > 0\]
其中$\Re(z)$指$z$之實部。\\
特別地,對於正實數$z$:
\[\Gamma(z) = (z-1)!\]
\subsection{排列(Permutation)}
\sssc{盡相異物排列}
從$n$相異物中取$m$個排成一直列($0\leq m\leq n$),正逆序視為二,其排列總數$P_m^n=\frac{n!}{\qty(n-m)!}$。
\sssc{不盡相異物排列}
設有$n$物,共有$k$種,第$i$種有$m_i$個。全取排成一直列,正逆序視為二,其排列總數為$\frac{n!}{\prod_{i=1}^n m_i!}$。
\sssc{盡相異物重複排列}
從$n$類相異物中,任取$m$個排成一直列,正逆序視為二,其中每類物品的個數均不小於$m$且可重複選取,其排列總數為$n^m$。
\subsection{錯排(Derangement)}
$n$相異物全取作直線排列,其中$m$物($m\leq n$)依次被限制不能排列於相異單一指定位置之排列,其方法數稱$D_m^n$;當$m=n$,錯排方法數稱$D_n$。
\subsubsection{全錯排}
\tb{遞迴式}:
\[
\begin{cases}
D_0 &= 1 \\
D_1 &= 0 \\
D_n &= (n-1) \cdot (D_{n-1}+D_{n-2})\quad\text{(當\protect\ }n > 2\text{)}
\end{cases}
\]
\tb{一般式}:
\[
\begin{aligned}
D_n &= n! \cdot \left(1+\sum_{i=1}^n \frac{(-1)^i}{i!} \right) \\
&= \sum_{i=0}^n (C_i^n \cdot (n-i)! \cdot (-1)^i
\end{aligned}
\]
\tb{錯排機率}:
\[ \lim_{n \to \infty} \frac{D_n}{n!} = \frac{1}{e} \]
\subsubsection{非全錯排}
\[ D_m^n = \sum_{i=0}^m \left((-1)^i \cdot C_i^m \cdot (n-i)!\right) \]
\subsection{環狀排列(Cyclic permutation)}
環狀排列:$n$物全取排列成環狀,不同方法之判定僅考慮相對位置,不考慮絕對位置,惟翻轉(順時針$\rightleftharpoons$逆時針)視為二種。
\subsubsection{盡相異物環狀排列}
\[ \frac{n!}{m\cdot (n-m)!} \]
\subsubsection{不盡相異物環狀排列}
$n$ 不盡相異物,全取作環狀排列,求其方法數。
\begin{itemize}
\item \textbf{法一:}\\
子循環:一個長度$n$的給定直線排列,若其前$\frac{n}{m}$物重複$m$次等同於原排列,且不存在$>1$的$k$使原排列的前$\frac{n}{m\cdot k}$物重複$k$次等同於原排列的前$\frac{n}{m}$物,則稱原排列的前$\frac{n}{m}$物為一個長度$\frac{n}{m}$的子循環,稱原排列之子循環長度為$\frac{n}{m}$、子循環數目為$m$。 \\
令該$n$不盡相異物的所有可能直線排列中,有$d_i$個之子循環長度為$\ell_i$,且$\sum_{i=1}^m d_i=$該$n$不盡相異物直線排列方法數,則:
\[ \text{所求}=\sum_{i=1}^m \frac{d_i}{\ell_i} \]
\item \textbf{法二:}\\
令該$n$不盡相異物可分為$k$相異類,第$i$類($1\leq i\leq k$)有$m_i$件相同物。最大公因數$\gcd(m_1, m_2, m_3, \ldots, m_k)=g$。令所求為$R$。
\begin{itemize}
\item 最大公因數$g$為$1$:
\[ R=\frac{\left(n-1\right)!}{\prod_{i=1}^k\left(m_i!\right)} \]
\item 最大公因數$g$為一質數$p$:
\[ R=\frac{1}{n}\cdot \left(\frac{n!}{\prod_{i=1}^k\left(m_i!\right)}-\frac{n!}{\prod_{i=1}^k\left(\frac{m_i}{p}!\right)}\right)+\frac{\left(\frac{n}{p}-1\right)!}{\prod_{i=1}^k\left(\frac{m_i}{p}!\right)} \]
\item 最大公因數$g$為任一正整數:
\[
\begin{aligned}
\text{令\protect\ }g & \text{之所有相異質因數由小到大依序為}p_1, p_2, p_3, \ldots, p_q。 \\
R &= \sum_{j=0}^q \left((-1)^j \cdot \sum_{\left|Y\right|=j} \frac{\left(\sum_{i=1}^k \frac{m_i}{\prod_{y \in Y} y}\right)!}{\prod_{i=1}^k \left(\frac{m_i}{\prod_{y \in Y} y}!\right)}\right) \\
& \text{,其中 } Y \subseteq \{p_1, p_2, p_3, \ldots, p_q\} \\
& \text{,定義 } \prod_{y \in \emptyset} y = 1
\end{aligned}
\]
\end{itemize}
\end{itemize}
\subsection{組合數(Combinatorics)}
組合數記作$\binom{n}{m}$或$C_m^n$。
\subsubsection{組合數非負整數域定義}
從$n$相異物中每次取$m$個為一組($0 \leq m \leq n, m,n \in \mathbb{N}_0$)之組合數,即:
\[\binom{n}{m} = \frac{n!}{m!\qty(n-m)!}, \quad 0 \leq m \leq n\land m,n \in \mathbb{N}_0.\]
\subsubsection{組合數非負實數域定義}
\[\binom{n}{m} = \frac{\prod_{i=0}^{m-1} \left(n-i\right)}{m!}, \quad 0 \leq m \leq n\land m \in \mathbb{N}_0\land n \in \mathbb{R}_{\geq 0}.\]
\subsubsection{組合數實數域定義}
\[\binom{n}{m} = 
\begin{cases}
& \frac{\prod_{i=0}^{m-1} \left(n-i\right)}{m!}, \quad 0 \leq m \leq n\land m \in \mathbb{N}_0\land n \in \mathbb{R}_{\geq 0},\\
& (-1)^m \cdot \binom{-n+m-1}{m}, \quad 0 \leq m \leq -n+m-1\land m \in \mathbb{N}_0\land n \in \mathbb{R}_{<0}.
\end{cases}\]
\subsubsection{組合數大於負一複數域定義}
\[\binom{n}{m} = \frac{\Gamma(n+1)}{\Gamma(m+1)\Gamma(n-m+1)}, \quad 0 \leq \Re(m) \leq \Re(n)\land m,n\in\mathbb{C}\land \Re(n),\Re(m),\Re(n-m) \in \mathbb{R}_{>-1}.\]
\subsubsection{組合數複數域定義}
\[\begin{aligned}
&\binom{n}{m} \\
=& \begin{cases}
\frac{\Gamma(n+1)}{\Gamma(m+1)\Gamma(n-m+1)}, \quad & 0 \leq \Re(m) \leq \Re(n)\land m,n\in\mathbb{C}\land \Re(n),\Re(m),\Re(n-m) \in \mathbb{R}_{>-1},\\
(-1)^m \cdot\binom{-n+m-1}{m}, \quad & 0 \leq \Re(m) \leq \Re(-n+m-1)\land m,n\in\mathbb{C}\land\Re(-n+m),\Re(m+1),\Re(-n) \in \mathbb{R}_{>0}.
\end{cases}
\end{aligned}\]
\sssc{二項式定理(Binomial theorem)}
\[\begin{aligned}
& (a+b)^n,\quad a,b,n\in\mathbb{C}\land a+b\neq 0\\
=& \begin{cases}
& \sum_{m=0}^n \binom{n}{m} \cdot a^m \cdot b^{n-m},\quad\Re(n)\in\mathbb{N}_0,\\
& \sum_{m=0}^{\infty} \binom{n}{m} \cdot a^m \cdot b^{n-m},\quad\Re(n)\notin\mathbb{N}_0\land\text{\ 此表達式收斂}.
\end{cases}\end{aligned}\]
\sssc{多項式定理(Multinomial theorem)}
\[\begin{aligned}
&\left(\sum_{i=1}^mx_i\right)^n,\quad x_i,n\in\mathbb{C}\land\sum_{i=1}^mx_i\neq 0\\
=& \begin{cases}
& \sum_{\substack{\sum_{i=1}^mk_i=n\\k_i\in\mathbb{N}_0}}\frac{n!}{\prod_{i=1}^n(k_i!)}\prod_{i=1}^nx_i^{\phantom{i}k_i},\quad \Re(n)\in\mathbb{N}_0,\\
& \sum_{k_i\in\mathbb{N}_0}\frac{\Gamma(n+1)}{\prod_{i=1}^n\Gamma(k_i+1)}\prod_{i=1}^nx_i^{\phantom{i}k_i},\quad\Re(n)\notin\mathbb{N}_0\land\text{\ 此表達式收斂}.
\end{cases}\end{aligned}\]
\sssc{范德蒙恆等式(Vandermonde identity)}
\[\sum_{k=0}^n\binom{n}{k}^2=\binom{2n}{n}.\]
\begin{proof}\mbox{}\\
Consider a lattice path from \((0,0)\) to \((n,n)\) in a grid where each step moves either right \((1,0)\) or up \((0,1)\).\\
Since we must choose \( n \) steps to move right out of the total \( 2n \) steps, the total number of such paths is given by:
\[\binom{2n}{n}.\]
Now, let’s count the same paths differently. Choose an intermediate point \((k, n-k)\) at step \( n \). The number of ways to reach this point from \((0,0)\) using \( k \) right steps and \( n-k \) up steps is \( \binom{n}{k} \). From \((k, n-k)\) to \((n,n)\), we need \( n-k \) right steps and \( k \) up steps, which can be done in \( \binom{n}{k} \) ways. Summing over all possible \( k \), we obtain:
\[\sum_{k=0}^{n}\binom{n}{k}^2.\]
\end{proof}
\sssc{重複組合}
由$n$類相異物中,任取$m$個為一組之方法數,其中每類物品的個數均不小於$m$且可重複選取。記作$H^n_m$。
\[
\begin{aligned}
H_m^n &= C^{n+m-1}_m \\
&= C^{n+m-1}_{n-1}
\end{aligned}
\]
\sssc{巴斯卡公式(Pascal's formula)}
\[C^n_m=C^{n-1}_{m-1}+C^{n-1}_m\]
\sssc{組合恆等式}
\[
\begin{aligned}
\sum_{k=0}^n C_k^n &= 2^n \\
\sum_{k=0}^n (-1)^k \cdot C_k^n &= 0 \\
\sum_{k=0}^n 2^k \cdot C_k^n &= 3^n \\
\sum_{k=0}^n k \cdot C_k^n &= n \cdot 2^{n-1} \\
\sum_{k=0}^n k^2 \cdot C_k^n &= n \cdot (n-1) \cdot 2^{n-2} + n \cdot 2^{n-1} \\
\sum_{k=3}^n C_2^k &= n \cdot 2^{n-1} \\
\left(\prod_{i=0}^x (k-i)\right) \cdot C^n_k &= \left(\prod_{i=0}^x (n-i)\right) \cdot C^{n-k-1}_{k-x-1} \quad x < k\land x \in \mathbb{N} \\
\sum_{k=0}^{n} \left(\left(\prod_{i=0}^x (k-i)\right) \cdot C^n_k\right) &= \sum_{k=0}^{n} \left(\left(\prod_{i=0}^x (n-i)\right) \cdot C^{n-k-1}_{k-x-1}\right) \\
&= \frac{n!}{\left(n-x-1\right)!} \cdot 2^{n-x-1} \\
& \quad x < n\land x \in \mathbb{N}, \text{定義}\,\forall k<0: C_k^n=0
\end{aligned}
\]
\sssc{巴斯卡三角形(Pascal's triangle)/楊輝三角形}
令最上面一列為第$0$列,向下每列遞增$1$;每一列最左之數為第$0$個,向右每個遞增$1$。巴斯卡三角形之第$n$列第$m$個數定義為$C_m^n$。
\[
\begin{aligned}
\begin{array}{ccccccc}
& & & 1 & & & \\
& & 1 & & 1 & & \\
& 1 & & 2 & & 1 & \\
1 & & 3 & & 3 & & 1 \\
& \vdots & & & & \vdots &
\end{array}
\end{aligned}
\]
\subsection{卡特蘭數(Catalan number)}
\sssc{卡特蘭數(Catalan number)}
所有在$n \times n$格點中不越過對角線的單調路徑的個數,一個單調路徑從格點左下角出發,在格點右上角結束,每一步均為向上或向右,記作$C_n$。

\tb{遞迴式}:
\[
\begin{cases}
C_0 &= 1 \\
C_n &= \sum_{k=1}^n C_{k-1}\cdot C_{n-k}\quad\text{(當\protect\ }n>0\text{)}
\end{cases}
\]
\begin{proof}
令對角線為$y=x$,第一次到達對角線上時的位置為$(k, k)$,則第一次到達對角線前的單調路徑數為$C_k$,第一次到達對角線後的單調路徑數為$C_{n-k}$,故對$n>0$成立。$C_0=C_1=0$,亦成立。
\end{proof}

\tb{一般式}:
\[
\begin{aligned}
C_n &= C_n^{2n} - C_{n-1}^{2n} \\
&= \frac{1}{n+1} \cdot C_n^{2n}
\end{aligned}
\]
\begin{proof}
自$(0, 0)$至$(n, n)$的單調路徑數為$C^{2n}_n$。考慮這些路徑中不符合卡特蘭數定義者,其第一次跨越對角線$y=x$的點必在$y=x+1$上,將接下來的路徑對$y=x+1$鏡射,則終點$(n,n)$變為$(n-1, n+1)$。在此$(n-1) \times (n+1)$格點中自$(0, 0)$至$(n-1, n+1)$的單調路徑數為$C^{2n}_{n-1}$。故$C_n = C_n^{2n} - C_{n-1}^{2n} = \frac{1}{n+1} \cdot C_n^{2n}$。
\end{proof}
\subsubsection{非正方形格點的卡特蘭數}
所有在$n \times k$格點中不越過對角線的單調路徑的個數,一個單調路徑從格點左下角出發,在格點右上角結束,每一步均為向上或向右,記作$C_{n,k}$。

\tb{一般式}:
\[C_{n,k}=
\begin{cases}
C_k^{n+k}-C_{k-1}^{n+k} & \quad\text{(當\protect\ }n \geq k\text{)} \\
C_n^{n+k}-C_{n-1}^{n+k} & \quad\text{(當\protect\ }n \leq k\text{)}
\end{cases}
\]
\end{document}