\documentclass[a4paper,12pt]{article}
\setcounter{secnumdepth}{5}
\setcounter{tocdepth}{3}
\newcounter{ZhRenew}
\setcounter{ZhRenew}{1}
\newcounter{SectionLanguage}
\setcounter{SectionLanguage}{1}
\input{/usr/share/latex-toolkit/template.tex}
\begin{document}
\title{數理邏輯}
\author{沈威宇}
\date{\temtoday}
\titletocdoc
\section{數理邏輯(Mathematical Logic)}
\ssc{陳述(Statement)/命題(Proposition)}
一個陳述性句子,要麼為真,要麼為假,但不能同時為真或假,如「2+2=4」。
\ssc{表達式(Expression)}
表達式是表示值或數學物件的符號組合,例如數字、變數、運算子和函數。 它本身不斷言任何內容,並且在對其進行評估之前不能將其分類為真或假,如「2 + 2」。
\ssc{條件(Condition)}
指定必須滿足的要求或限制的陳述。
\ssc{非命題(Negation)}
「$\neg P$」稱「$P$」的非命題。
\ssc{否命題(Inverse)}
「$\neg P \implies \neg Q$」為「$P \implies Q$」的否命題。
\ssc{逆命題(Converse)}
「$Q \implies P$」為「$P \implies Q$」的逆命題。
\ssc{否逆/逆否命題(Contraposition)/對偶命題}
「\( \neg B \implies \neg A\)」為「\( A \implies B \)」的逆否命題。逆否命題$\iff$原命題。
\ssc{邏輯的笛摩根定律}
\[\neg (P\land Q)\iff (\neg P)\lor (\neg Q)\]
\[\neg (P\lor Q)\iff (\neg P)\land (\neg Q)\]
\ssc{充分條件(Sufficient condition)}
 「如果\( A \) 為真則保證\( B \) 為真」$\equiv$「條件\( A \) 是條件\( B \) 的充分條件」$\equiv$「\( A \implies B \)」。
\ssc{必要條件(Necessary condition)}
「如果\( A \) 為真則保證\( B \) 為真」$\equiv$「條件\( B \) 是條件\( A \) 的必要條件」$\equiv$「\( A \implies B \)」$\equiv$「\( \neg B \implies \neg A\)」。
\ssc{充要條件(Sufficient and necessary condition)/充分必要條件}
「\( (A \implies B) \land (B \implies A) \)」$\equiv$「\( A \iff B\)」$\equiv$「條件\( A \) 是條件\( B \) 的充分必要條件」$\equiv$「條件\( B \) 是條件\( A \) 的充分必要條件」。
\end{document}