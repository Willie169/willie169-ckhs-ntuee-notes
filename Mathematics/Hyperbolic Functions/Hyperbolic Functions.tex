\documentclass[a4paper,12pt]{report}
\setcounter{secnumdepth}{5}
\setcounter{tocdepth}{3}
\input{/usr/share/LaTeX-ToolKit/template.tex}
\begin{document}
\title{Hyperbolic Functions}
\author{沈威宇}
\date{\temtoday}
\titletocdoc
\sct{Hyperbolic Functions (雙曲函數)}
\ssc{Hyperbolic Functions}
\sssc{Definition}
\begin{longtable}[c]{|p{0.16\textwidth}|p{0.16\textwidth}|p{0.16\textwidth}|p{0.16\textwidth}|p{0.16\textwidth}|}
\hline
Function & Symbols & Definition & Domain & Range \\
\hline\endhead
    Hyperbolic sine (雙曲正弦) & $\sinh x$ & $\frac{e^{x}-e^{-x}}{2}$ & $\bbR$ & $\bbR$ \\\hline
    Hyperbolic cosine (雙曲餘弦) & $\cosh x$ & $\frac{e^{x}+e^{-x}}{2}$ & $\bbR$ & $(1,\infty)$ \\\hline
    Hyperbolic tangent (雙曲正切) & $\tanh x$ & $\frac{\sinh(x)}{\cosh(x)}$ & $\bbR$ & $(-1,1)$ \\\hline
    Hyperbolic cotangent (雙曲餘切) & $\coth x$ & $\frac{\cosh(x)}{\sinh(x)}$ & $\bbR\setminus\{0\}$ & $(-\infty,-1)\cup(1,\infty)$ \\\hline
    Hyperbolic secant (雙曲正割) & $\sech x$ & $\frac{1}{\cosh(x)}$ & $\bbR$ & $(0,1]$ \\\hline
    Hyperbolic cosine (雙曲餘弦) & $\cosh x$ & $\frac{1}{\sinh(x)}$ & $\bbR\setminus\{0\}$ & $\bbR\setminus\{0\}$ \\\hline
\end{longtable}
\FB
\sssc{Power notation}
\[\ba
&\sinh^n x\coloneq\begin{cases}\qty(\sinh x)^n,\quad n\geq 0\\\arcsinh x,\quad n=-1\end{cases}\\
&\cosh^n x\coloneq\begin{cases}\qty(\cosh x)^n,\quad n\geq 0\\\arccosh x,\quad n=-1\end{cases}\\
&\tanh^n x\coloneq\begin{cases}\qty(\tanh x)^n,\quad n\geq 0\\\arctanh x,\quad n=-1\end{cases}\\
&\coth^n x\coloneq\begin{cases}\qty(\coth x)^n,\quad n\geq 0\\\arccoth x,\quad n=-1\end{cases}\\
&\sech^n x\coloneq\begin{cases}\qty(\sech x)^n,\quad n\geq 0\\\arcsech x,\quad n=-1\end{cases}\\
&\csch^n x\coloneq\begin{cases}\qty(\csch x)^n,\quad n\geq 0\\\arccsch x,\quad n=-1\end{cases}
\ea\]
\ssc{Inverse hyperbolic function (反雙曲函數)}
\sssc{Definition}
\begin{longtable}[c]{|p{0.16\textwidth}|p{0.16\textwidth}|p{0.16\textwidth}|p{0.16\textwidth}|p{0.16\textwidth}|}
\hline
Function & Symbols & Definition & Domain & Range \\
\hline\endhead
    Inverse hyperbolic sine (反雙曲正弦) & \(y=\arcsinh x=\sinh^{-1}(x)=\asinh(x)\) & \(x=\sinh y\) & \(\bbR\) & \(\bbR\) \\ \hline
    Inverse hyperbolic cosine (反雙曲餘弦) & \(y=\arccosh x=\cosh^{-1}(x)=\acosh(x)\) & \(x=\cosh y\) & \([1,\infty)\) & \([0,\infty)\) \\ \hline
    Inverse hyperbolic tangent (反雙曲正切) & \(y=\arctanh x=\tanh^{-1}(x)=\atanh(x)\) & \(x=\tanh y\) & \((-1,1)\) & $\bbR$ \\ \hline
    Inverse hyperbolic cotagent (反雙曲餘切) & \(y=\arccoth x=\coth^{-1}(x)=\acoth(x)\) & \(x=\coth y\) & \((-\infty,-1)\cup(1,\infty)\) & \(\bbR\setminus\{0\}\) \\ \hline
    Inverse hyperbolic secant (反雙曲正割) & \(y=\arcsech x=\sech^{-1}(x)=\asech(x)\) & \(x=\sech y\) & \((0,1]\) & $[0,\infty)$ \\ \hline
    Inverse hyperbolic cosecant (反雙曲餘割) & \(y=\arccsch x=\csch^{-1}(x)=\acsch(x)\) & \(x=\csch y\) & \(\bbR\setminus\{0\}\) & \(\bbR\setminus\{0\}\) \\ \hline
\end{longtable}
\FB
\sssc{Logarithmatic forms}
\bma
\operatorname{arcsinh} &= \ln\left(x+\sqrt{x^2+1}\right)\\
\operatorname{arccosh} &= \ln\left(x+\sqrt{x^{2}-1}\right),\quad x\geq 1\\
\operatorname{arctanh} &= \frac{1}{2}\ln\left(\frac{1+x}{1-x}\right),\quad\abs{x}<1\\
\operatorname{arccoth} &= \frac{1}{2}\ln\left(\frac{x+1}{x-1}\right),\quad\abs{x}>1\\
\operatorname{arcsech} &= \ln\left(\frac{1}{x}+\frac{\sqrt{1-x^2}}{x}\right),\quad 0<x\leq 1\\
\operatorname{arccsch} &= \ln\left(\frac{1}{x}+\frac{\sqrt{1+x^2}}{\abs{x}}\right),\quad x\neq 0
\eam
\sssc{Hyperbolic functions of inverse hyperbolic functions}
\begin{longtable}[c]{|m{0.1\textwidth}|m{0.12\textwidth}|m{0.12\textwidth}|m{0.12\textwidth}|m{0.12\textwidth}|m{0.12\textwidth}|m{0.12\textwidth}|}
\hline
    \theta & \sinh\theta & \cosh\theta & \tanh\theta & \coth\theta & \sech\theta & \csch\theta \\\hline\endhead
    \arcsinh(x) & x & \sqrt{1+x^2} & \frac{x}{\sqrt{1+x^2}} & \frac{\sqrt{1+x^2}}{x},\quad x\neq 0 & \frac{1}{\sqrt{1+x^2}} & \frac{1}{x},\quad x\neq 0 \\\hline
    \arccosh(x) & \sqrt{x^2-1} & x & \frac{\sqrt{x^2-1}}{x} & \frac{x}{\sqrt{x^2-1}} & \frac{1}{x} & \frac{1}{\sqrt{x^2-1}} \\\hline
    \arctanh(x) & \frac{x}{\sqrt{1-x^2}} & \frac{1}{\sqrt{1-x^2}} & x & \frac{1}{x},\quad x\neq 0 & \sqrt{1-x^2} & \frac{\sqrt{1-x^2}}{x},\quad x\neq 0 \\\hline
    \arccoth(x) & \frac{1}{\sqrt{x^2-1}} & \frac{|x|}{\sqrt{x^2-1}} & \frac{1}{x} & x & \frac{\sqrt{x^2-1}}{|x|} & \sqrt{x^2-1} \\\hline
    \arcsech(x) & \frac{\sqrt{1-x^2}}{x} &{{{ \frac{1}{x} & \sqrt{x^2-1}\operatorname{sgn}\qty(x) & \frac{\operatorname{sgn}\qty(x)}{\sqrt{x^2-1}},\quad|x|>1 & x & \frac{|x|}{\sqrt{x^2-1}} \\\hline
    \arccsc(x) & \frac{1}{x} & \frac{\sqrt{x^2-1}}{|x|} & \frac{\operatorname{sgn}\qty(x)}{\sqrt{x^2-1}},\quad|x|>1 & \sqrt{x^2-1}\operatorname{sgn}\qty(x) & \frac{|x|}{\sqrt{x^2-1}} & x \\\hline
\end{longtable}\FB
\ssc{Identities}
\sssc{正切萬能公式}
\[\sinh\theta=\frac{2\tanh\frac{\theta}{2}}{1-\tanh^2\frac{\theta}{2}}\]
\[\cosh\theta=\frac{1+\tanh^2\frac{\theta}{2}}{1-\tanh^2\frac{\theta}{2}}\]
\[\tanh\theta=\frac{2\tanh\frac{\theta}{2}}{1+\tanh^2\frac{\theta}{2}}\]



















\end{document}

