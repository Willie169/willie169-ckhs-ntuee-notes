\documentclass[a4paper,12pt]{article}
\setcounter{secnumdepth}{5}
\setcounter{tocdepth}{3}
\input{/usr/share/LaTeX-ToolKit/template.tex}
\renewcommand{\arraystretch}{1.5}
\begin{document}
\title{Microelectronics}
\author{沈威宇}
\date{\temtoday}
\titletocdoc
\sct{Microelectronics}
\ssc{Signals}
\sssc{Signal}
A signal is a function that conveys information about a phenomenon. Signal processing is usually most conveniently performed by electronic systems, and voltage signals are studied here.
\sssc{Transducer}
A transducer is a device that convert a signal to a voltage signal.
\sssc{Signal source}
A signal source is typically represented in the Thévenin form, a voltage source in a series connection with a resistance, or the Norton form, a current source in a parallel connection with a resistance.

