\documentclass[a4paper,12pt]{article}
\setcounter{secnumdepth}{5}
\setcounter{tocdepth}{3}
\input{/usr/share/LaTeX-ToolKit/template.tex}
\begin{document}
\title{Logic Design}
\author{沈威宇}
\date{\temtoday}
\titletocdoc
\sct{Logic Design}
\ssc{Digital Systems and Switching Circuits}
\begin{itemize}
\item\textbf{Digital system or digital circuit}: Deals with signals that have discrete values. Theoretically, for a given input, the output is exactly correct.
\item\textbf{Analog system}: Deals with signals that vary continuously over time. The output might have an error depending on the accuracy of the components used.
\item\textbf{System design}: The highest level of the design of digital systems, where you break the system into subsystems, specify what each subsystem do, and determine the interconnection and control of the subsystems.
\item\textbf{(Digital) logic design}: The middle level of the design of digital systems, where you specify the logic operations inside each subsystem.
\item\textbf{Circuit (ckt) design}: The lowest level of the design of digital systems, where you specify the electronic components and their interconnection to form the system.
\item\textbf{Switching (logic) circuit}: A switching circuit has one or more inputs and one or more outputs which take on discrete values. Many of a digital system's subsystems take the form of a switching circuit.
\item\textbf{Combinational (logic) circuit}: A circuit whose output value depends only on the present input value.
\item\textbf{Sequential (logic) circuit}: A circuit whose output value depends on the present input value and past input values.
\item\textbf{Logic gate}: An electronic device that performs a basic Boolean operation on one or more binary inputs (0 or 1) to produce a single binary output (0 or 1).
\item\textbf{Flip-flop}: A bistable electronic device, meaning it has two stable states representing logic 0 and 1. It's a basic memory element used in sequential circuits.
\end{itemize}
\subsection{Boolean Algebra}
The basic mathematics needed for the study of logic design of digital systems is Boolean algebra. Boolean algebra was introduced by George Boole in his first book The Mathematical Analysis of Logic (1847), and set forth more fully in his An Investigation of the Laws of Thought (1854).

Switching devices are essentially two-state devices (e.g. switches which are open or closed and transistors with high or low output voltages). Consequently, we will emphasize the special case of Boolean algebra in which all of the variables assume only one of two values, 0 (false) or 1 (true), called Boolean variables; this two-valued Boolean algebra is also called switching algebra.

In a switch circuit, 0 (usually) represents an open switch, and 1 represents a closed circuit. In general, 0 and 1 can be used to represent the two states in any binary-valued system.

A Boolean function is a logical operation performed on one or more binary inputs that produces a single binary output.

A logic gate is a device that performs a Boolean function. Below, electronic symbols follow ANSI standards.
\subsection{Logical Operators}
\sssc{NOT / Negation / Inversion / Complement}
\begin{itemize}
\item Symbol: $A'$, $\neg A$, $\mathord{\sim}A$, or $\ol{A}$
\item Definition: Invert the value.
\item Truth Table:
\begin{longtable}[c]{|m|m|}
\hline
A & A'\\\hline
0 & 1\\\hline
1 & 0\\\hline
\end{longtable}
\item Logic gate: \esym{not port}
\eit 
\sssc{AND / Conjunction / Logical Multiplication}
\begin{itemize}
\item Symbol: $A \cdot B$, $A*B$, or $AB$
\item Definition: True if both inputs are true and false otherwise.
\item Truth Table:
\begin{longtable}[c]{|m|m|m|}
\hline
A & B & A\cdot B\\\hline
0 & 0 & 0\\\hline
1 & 0 & 0\\\hline
0 & 1 & 0\\\hline
1 & 1 & 1\\\hline
\end{longtable}
\item Logic gate: \esym{and port}
\eit
\sssc{OR / Disjunction / Logical Addition / Inclusive OR}
\begin{itemize}
\item Symbol: $A + B$
\item Definition: True if at least one input is true and false otherwise.
\item Truth Table:
\begin{longtable}[c]{|m|m|m|}
\hline
A & B & A+B\\\hline
0 & 0 & 0\\\hline
1 & 0 & 1\\\hline
0 & 1 & 1\\\hline
1 & 1 & 1\\\hline
\end{longtable}
\item Logic gate: \esym{or port}
\eit
\sssc{XOR / Exclusive OR}
\begin{itemize}
\item Symbol: $A \oplus B$
\item Definition: $A'B+AB'$
\item Truth Table:
\begin{longtable}[c]{|m|m|m|}
\hline
A & B & A\oplus B\\\hline
0 & 0 & 0\\\hline
1 & 0 & 1\\\hline
0 & 1 & 1\\\hline
1 & 1 & 0\\\hline
\end{longtable}
\item Logic gate: \esym{xor port}
\eit
\sssc{NAND / Not AND}
\begin{itemize}
\item Definition: $(AB)'$
\item Truth Table:
\begin{longtable}[c]{|m|m|m|}
\hline
A & B & (A\cdot B)'\\\hline
0 & 0 & 1\\\hline
1 & 0 & 1\\\hline
0 & 1 & 1\\\hline
1 & 1 & 0\\\hline
\end{longtable}
\item Logic gate: \esym{nand port}
\eit
\sssc{NOR / Not OR}
\begin{itemize}
\item Definition: $(A+B)'$
\item Truth Table:
\begin{longtable}[c]{|m|m|m|}
\hline
A & B & (A+B)'\\\hline
0 & 0 & 1\\\hline
1 & 0 & 0\\\hline
0 & 1 & 0\\\hline
1 & 1 & 0\\\hline
\end{longtable}
\item Logic gate: \esym{nor port}
\eit
\sssc{XNOR / Logical equivalence / Exclusive NOR}
\begin{itemize}
\item Symbol: $A\equiv B$ or $A\odot B$
\item Definition: $(A\oplus B)'$
\item Truth Table:
\begin{longtable}[c]{|m|m|m|}
\hline
A & B & (A\oplus B)'\\\hline
0 & 0 & 1\\\hline
1 & 0 & 0\\\hline
0 & 1 & 0\\\hline
1 & 1 & 1\\\hline
\end{longtable}
\item Logic gate: \esym{xnor port}
\eit
\ssc{Boolean Arithmetic}
\sssc{Boolean expression}
A Boolean expression is formed by application of the logic operations to one or more Boolean variables or constants. 

\bit
\item\tb{Literals}: The simplest expressions, consisting of a single constant or variable or its complement, such as $0$, $X$, or $Y′$ .
\item\tb{Product terms}: A product term is a product of literals or a literal.
\item\tb{Sum terms}: A sum term is a sum of literals or a literal.
\eit

In a Boolean expression, parentheses are added as needed to specify the order in which the operations are performed; without parentheses, complementation is performed first, and then AND, and then OR; and the precedence of XOR and XNOR are either right higher to OR or right lower to OR depending on the context.
\sssc{Boolean function}
A Boolean function or a switching function is a function of which the domain is $\{1,0\}^n$ in which $n\in\mathbb{N}$ and the codomain is $\{1,0\}$.

The ON-set of a Boolean function $f$ is the set of all input combinations such that $f$ of them is $1$. Each element in the ON-set of $f$ corresponds to a row in the truth table of $f$ in which the output is $1$. The OFF-set of a Boolean function $f$ is the set of all input combinations such that $f$ of them is $0$. Each element in the OFF-set of $f$ corresponds to a row in the truth table of $f$ in which the output is $0$.
\sssc{Truth table}
A truth table, also called a table of combinations, specifies the corresponding output values for all possible combinations of input values for a Boolean function in the order such that if two input combinations have the same inputs in the first $i$ variables and $A_{i+1}$ of the first one is $0$, $A_{i+1}$ of the second one is $1$, then the first is put prior to the second. A truth table for an $n$-variable Boolean function $f\qty(A_1,A_2,\ldots A_n)$ have $2^n$ rows and is:
\begin{longtable}[c]{|m|m|m|m|m|}
\hline
A_1 & A_2 & \ldots & A_n & f\\\hline
0 & 0 & \ldots & 0 & f\qty(0,0,\ldots 0)\\\hline
0 & 0 & \ldots & 1 & f\qty(0,0,\ldots 1)\\\hline
\vdots & \vdots & \vdots & \ddots & \vdots\\\hline
1 & 1 & \ldots & 1 & f\qty(1,1,\ldots 1)\\\hline
\end{longtable}
\sssc{Sum-of-products (SOP) form}
An expression is said to be in SOP form if it consists of a sum (OR) of product (AND) or single variable terms, in which it is called to be degenerate when some of the terms are single variables.

Every Boolean expression can be expressed in SOP form. And a Boolean expression may have more than one SOP forms.
\sssc{Product-of-sums (POS) form}
An expression is said to be in POS form if it consists of a product (AND) of sum (OR) or single variable terms, in which it is called to be degenerate when some of the terms are single variables.

Every Boolean expression can be expressed in POS form. And a Boolean expression may have more than one POS forms.
\sssc{Dual}
The dual of an Boolean expression or equation is another Boolean expression or equation obtained by simplifing the original expression or equation such that it only consists of literals, AND, and OR, and then replacing all AND in it with OR, all OR in it with AND, and all constants in it with their complements.
\sssc{Boolean Vector Functions or Boolean Multi-output Functions}
A Boolean vector function or a Boolean multi-output function is a function $f\colon\{1,0\}^n\to\{1,0\}^m$. A Boolean vector function can be specified with truth table or as a tuple of $m$ Boolean functions.
\ssc{Theorems}
\sssc{Idempotent Laws}
\[XX = X,\quad X + X = X.\]
\sssc{Involution Law}
\[(X′)′ = X.\]
\sssc{Laws of Complementarity}
\[X X′ = 0,\quad X + X′ = 1.\]
\sssc{Commutativity of AND, OR, XOR, and XNOR}
\[X Y=Y X,\quad X+Y=Y+X,\]
\[X\oplus Y=Y\oplus X,\quad X\odot Y=Y\odot X.\]
\sssc{Associativity of AND, OR, XOR, and XNOR}
\[X Y Z=X (Y Z),\quad X+Y+Z=X+(Y+Z),\]
\[X\oplus Y\oplus Z=X\oplus (Y\oplus Z),\quad X\odot Y\odot Z=X\odot (Y\odot Z).\]
\sssc{Distributivity of AND over OR and OR over AND}
\[X (Y+Z)=X Y+X Z.\]
\[X+YZ=(X+Y)(X+Z).\]
\sssc{XOR of ANDs Theorem or Distributivity of AND over XOR}
\[X Y\oplus X Z=X (Y\oplus Z).\]
\begin{proof}
\[X Y\oplus X Z=XY(XZ)'+(XY)'XZ=XYX'+XYZ'+X'XZ+Y'XZ=XYZ'+XY'Z=X(Y\oplus Z).\]
\end{proof}
\sssc{XOR of ORs Theorem}
\[(X+Y)\oplus (X+Z)=X' (Y\oplus Z).\]
\begin{proof}
\[(X+Y)\oplus (X+Z)=(X+Y)(X+Z)'+(X+Y)'(X+Z)=XX'Z'+YX'Z'+X'Y'X+X'Y'Z=X'YZ'+X'Y'Z=X'(Y\oplus Z).\]
\end{proof}
\sssc{XNOR of ORs Theorem or Distributivity of OR over XNOR}
\[(X+Y)\odot (X+Z)=X+(Y\odot Z).\]
\begin{proof}
\[(X+Y)\odot (X+Z)=(X+Y)(X+Z)+(X+Y)'(X+Z)'=X+YZ+X'Y'Z'=X+YZ+Y'Z'=X+(Y\odot Z).\]
\end{proof}
\sssc{XNOR of ANDs Theorem}
\[XY\odot XZ=X'+(Y\odot Z).\]
\begin{proof}
\[XY\odot XZ=XYZ+(XY)'(XZ)'=XYZ+(X'+Y')(X'+Z')=XYZ+X'+Y'Z'=X'+(Y\odot Z).\]
\end{proof}
\sssc{DeMorgan's Laws}
\[(\sum_{i=1}^nX_i)′=\prod_{i=1}^nX_i',\quad (\prod_{i=1}^nX_i)'=\sum_{i=1}^nX_i',\]
\[(\bigoplus_{i=1}^nX_i)′=\bigodot_{i=1}^nX_i',\quad (\bigodot_{i=1}^nX_i)'=\bigoplus_{i=1}^nX_i'.\]
\sssc{Axiom of Equality}
For an one-to-one Boolean function $f$, a Boolean expression $A$ equals another Boolean expression $B$ if and only if $f(A)$ equals $f(B)$.
\sssc{Duality Principle}
A Boolean equation is an identity if and only if the dual of it is an identity.
\sssc{Uniting Theorems}
\[XY+XY'=X,\quad (X+Y)(X+Y')=X.\]
\sssc{Absorption Theorems}
\[X+XY=X,\quad X(X+Y)=X.\]
\sssc{Elimination Theorems}
\[X+X'Y=X+Y,\quad X(X'+Y)=XY.\]
\sssc{Consensus Theorems}
The consensus theorems involve eliminate one term from an expression in SOP or POS form, in which the eliminated term is called the consensus term.
\[XY+X′Z+YZ=XY+X′Z.\]
\begin{proof}
\[\begin{aligned}
XY+X′Z+YZ&=XY+X'Z+(X+X')YZ\\
&=XY+X'Z+XYZ+X'YZ\\
&=XY+X′Z.
\end{aligned}\]
\end{proof}
\[(X+Y)(X′+Z)(Y+Z)=(X+Y)(X′+Z).\]
\begin{proof}
\[\begin{aligned}
(X+Y)(X′+Z)(Y+Z)&=(X+Y)(X'+Z)(X+X')(Y+Z)\\
&=(X+Y)(X'+Z)(X+Y+Z)(X'+Y+Z)\\
&=(X+Y)(X′+Z).
\end{aligned}\]
\end{proof}
\sssc{Boole's expansion theorem, Shannon's expansion theorem, Shannon decomposition, or fundamental theorem of Boolean algebra}
The theorem states that
\[F=x\cdot F_x+x'\cdot F_{x'},\]
where $F$ is any Boolean function, $x$ is a variable, and $F_x$ and $F_{x'}$, sometimes called the positive and negative Shannon cofactors, respectively, of $F$ with respect to $x$, are $F$ with the independent variable $x$ set to $1$ and to $0$ respectively.
\sssc{Combination of Distributivity and Consensus Theorem}
\[(X+Y)(X'+Z)=XZ+X'Y\]
\begin{proof}
\[(X+Y)(X'+Z)=0+XZ+X'Y+YZ=XZ+X'Y\]
\end{proof}
\sssc{XOR and XNOR Series Theorems}
\[\bigoplus_{i=1}^nX_i=\qty(\sum_{i=1}^nX_i)\mod 2.\]
\[\bigodot_{i=1}^nX_i=1-\qty(\sum_{i=1}^nX_i)\mod 2=\qty(1+\sum_{i=1}^nX_i)\mod 2.\]
\[\qty(\bigoplus_{i=1}^nX_i)'=\qty(\bigoplus_{i=1}^{j-1}X_i)\oplus\qty(X_j)'\oplus\qty(\bigoplus_{i=j+1}^nX_i),\quad\forall j\leq n\land j\in\mathbb{N},\,\forall n\text{\ s.t.\ }\frac{n}{2}\in\mathbb{N}.\]
\begin{proof}\mbox{}\\
Case $n=2$:
\[\ba
(X_1\oplus X_2)'&=(X_1X_2'+X_1'X_2)'=(X_1X_2')'(X_1'X_2)'\\
&=(X_1'+X_2)(X_1+X_2')=(X_1'X_2'+X_1X_2)\\
&=X_1'\oplus X_2=X_1\oplus X_2'
\ea\]
Prove by mathematical induction. Assume it holds for $n=k$ and $n=2$. We want to prove that it holds for $n=k+2$.
\[\ba
\qty(\bigoplus_{i=1}^{k+2})'&=\qty(\bigoplus_{i=1}^kX_i)'\oplus X_{k+1}\oplus X_{k+2}\\
&=\qty(\bigoplus_{i=1}^{j-1}X_i)\oplus\qty(X_j)'\oplus\qty(\bigoplus_{i=j+1}^kX_i)\oplus X_{k+1}\oplus X_{k+2},\quad \forall j\leq k\land j\in\mathbb{N}
\ea\]
\[\ba
\qty(\bigoplus_{i=1}^{k+2})'&=X_1\oplus X_2\oplus\qty(\bigoplus_{i=3}^{k+2}X_i)'\\
&=X_1\oplus X_2\oplus\qty(\bigoplus_{i=3}^{j-1}X_i)\oplus\qty(X_j)'\oplus\qty(\bigoplus_{i=j+1}^{k+2}X_i),\quad \forall 3\leq j\leq k+2\land j\in\mathbb{N}
\ea\]
\end{proof}
\[\bigoplus_{i=1}^nX_i=\qty(\bigodot_{i=1}^nX_i)',\quad\forall n\text{\ s.t.\ }\frac{n}{2}\in\mathbb{N}.\]
\begin{proof}\mbox{}\\
By the definitions, it holds for case $n=2$.

Prove by mathematical induction. Assume it holds for $n=k$ and $n=2$. We want to prove that it holds for $n=k+2$.
\[\ba
\bigoplus_{i=1}^{k+2}X_i&=\qty(\bigoplus_{i=1}^kX_i)\oplus X_{k+1}\oplus X_{k+2}\\
&=\qty(\bigodot_{i=1}^kX_i)'\oplus X_{k+1}\oplus X_{k+2}\\
&=\qty(\bigodot_{i=1}^{k+1}X_i)\oplus X_{k+2}\\
&=\qty(\bigodot_{i=1}^{k+2}X_i)'
\ea\]
\end{proof}
\[\bigoplus_{i=1}^nX_i=\bigodot_{i=1}^nX_i,\quad\forall n\text{\ s.t.\ }\frac{n-1}{2}\in\mathbb{N}.\]
\begin{proof}\mbox{}\\
By
\[\bigoplus_{i=1}^nX_i=\qty(\bigodot_{i=1}^nX_i)',\quad\forall n\text{\ s.t.\ }\frac{n}{2}\in\mathbb{N}.\]
For $n$ such that $\frac{n-1}{2}\in\mathbb{N}$,
\[\ba
\bigoplus_{i=1}^nX_i&=\bigoplus_{i=1}^{n-1}X_i\oplus X_n\\
&=\bigoplus_{i=1}^{n-1}X_i\oplus X_n\\
&=\qty(\bigodot_{i=1}^{n-1}X_i)'\oplus X_n\\
&=\bigodot_{i=1}^nX_i
\ea\]
\end{proof}
\sssc{Consensus of XOR Theorem}
\[X\oplus Y+X\oplus Z+Y\oplus Z=X\oplus Y+X\oplus Z=X\oplus Y+Y\oplus Z=X\oplus Z+Y\oplus Z=XY'+X'Z+YZ'=X'Y+XZ'+Y'Z\]
\begin{proof}
\[X\oplus Y+X\oplus Z=XY'+X'Y+XZ'+X'Z=XY'+X'Y+XZ'+X'Z+YZ'+Y'Z=X\oplus Y+X\oplus Z+Y\oplus Z=XY'+X'Z+YZ'=X'Y+XZ'+Y'Z\]
\end{proof}
\ssc{Minterm and Maxterm Expansions, Canonical Expansions, or Standard Expansion}
\sssc{Minterm expansion, canonical SOP, or standard SOP}
A minterm of a completely specified Boolean function $f\colon\{1,0\}^n\to\{1,0\}$, denoted as $m_i$ for the input combination in the $(i+1)$th row of the truth table of $f$ that is in the ON-set of $f$, is defined for any input combinations in the ON-set of $f$ as
\[m_i=\prod_{k=1}^ny_k,\]
in which $y_k$ is defined as the $k$th input variable, i.e. $x_k$, if the $k$th input in that input combination is $1$ and as the complement of the $k$th input variable, i.e. $(x_k)'$, if the $k$th input in that input combination is $0$.

Minterm expansion, canonical SOP, or standard SOP is an expression of a completely specified Boolean function as a sum of minterms of it. Let $S$ be the set of all integer $i$ such that the input combination in the $(i+1)$th row of the truth table of $f$ is in the ON-set of $f$. Then the minterm expansion, canonical SOP, or standard SOP of $f$ is
\[f=\sum_{i\in S}m_i,\]
also denoted as
\[f=\sum m(i\in S).\]

For a given completely specified Boolean function, there exists a unique minterm expansion of it.

For example, given a function $f(a,b,c)$ with truth table
\begin{longtable}[c]{|m|m|m|m|}
\hline
a & b & c & f\\\hline
0 & 0 & 0 & 0\\\hline
0 & 0 & 1 & 0\\\hline
0 & 1 & 0 & 1\\\hline
0 & 1 & 1 & 0\\\hline
1 & 0 & 0 & 0\\\hline
1 & 0 & 1 & 1\\\hline
1 & 1 & 0 & 1\\\hline
1 & 1 & 1 & 0\\\hline
\end{longtable}
, the minterm expansion of it is
\[f=m_2+m_5+m_6=\sum m(2,5,6).\]
\sssc{Maxterm expansion, canonical POS, or standard POS}
A maxterm of a completely specified Boolean function $f\colon\{1,0\}^n\to\{1,0\}$, denoted as $M_i$ for the input combination in the $(i+1)$th row of the truth table of $f$ that is in the OFF-set of $f$, is defined for any input combinations in the OFF-set of $f$ as
\[M_i=\sum_{k=1}^ny_k,\]
in which $y_k$ is defined as the $k$th input variable, i.e. $x_k$, if the $k$th input in that input combination is $0$ and as the complement of the $k$th input variable, i.e. $(x_k)'$, if the $k$th input in that input combination is $1$.

Maxterm expansion, canonical POS, or standard POS is an expression of a completely specified Boolean function as a product of maxterms of it. Let $T$ be the set of all integer $i$ such that the input combination in the $(i+1)$th row of the truth table of $f$ is in the OFF-set of $f$. Then the maxterm expansion, canonical POS, or standard POS of $f$ is
\[f=\prod_{i\in T}M_i,\]
also denoted as 
\[f=\prod M(i\in T).\]

For a given completely specified Boolean function, there exists a unique maxterm expansion of it.

For example, given a function $f(a,b,c)$ with truth table
\begin{longtable}[c]{|m|m|m|m|}
\hline
a & b & c & f\\\hline
0 & 0 & 0 & 0\\\hline
0 & 0 & 1 & 0\\\hline
0 & 1 & 0 & 1\\\hline
0 & 1 & 1 & 0\\\hline
1 & 0 & 0 & 0\\\hline
1 & 0 & 1 & 1\\\hline
1 & 1 & 0 & 1\\\hline
1 & 1 & 1 & 0\\\hline
\end{longtable}
, the maxterm expansion of it is
\[f=M_0M_1M_3M_4M_7=\prod M(0,1,3,4,7).\]
\sssc{Conversion between function and its complement}
Let $f$ be a completely specified Boolean function with minterm expansion
\[f=\sum m(i\in S),\]
and maxterm expansion
\[f=\prod M(i\in T),\]
where $S$ be the set of all integer $i$ such that the input combination in the $(i+1)$th row of the truth table of $f$ is in the ON-set of $f$, $T$ be the set of all integer $i$ such that the input combination in the $(i+1)$th row of the truth table of $f$ is in the OFF-set of $f$.

Then, the complement $f'$ of has minterm expansion 
\[f'=\sum m(i\in T),\]
and maxterm expansion
\[f'=\prod M(i\in S).\]
\ssc{Minimal form}
\sssc{Implicant}
Given a Boolean function $f$ of $n$ variables, a product term $P$ is an implicant of $f$ iff for every combination of values of the $n$ variables for which $P = 1$, $f$ is also equal to $1$.
\sssc{Prime implicant}
A prime implicant of a function $f$ is an implicant of $f$ which is no longer an implicant of $f$ if any literal is deleted from it.
\sssc{Essential prime implicant}
An essential prime implicant of a function $f$ is a prime implicant $P$ of $f$ such that there exists a minterm $m$ of $f$ such that $m$ implies $P$ and for any other prime implicant $Q$ of $f$, $m$ implies $Q'$.
\sssc{Minimal SOP form or minimum sum}
A SOP form of a Boolean function is called minimal if it has the fewest number of terms out of all SOP forms of the function, and every product term in it can not have any variable in it be eliminated. For a given Boolean function, there may exist more than one minimal SOP forms of it.

A minimal SOP form of a Boolean function must consist of some of its prime implicants (but not necessarily all), that is, if a SOP form contains implicants that are not prime implicants, it is not minimal.

A minimal SOP form of a Boolean function must contain all of its essential prime implicants.
\sssc{Minimal POS form or minimum product}
A POS form of a Boolean function is called minimal if it has the fewest number of terms out of all POS forms of the function, and every sum term in it can not have any variable in it be eliminated. For a given Boolean function, there may exist more than one minimal POS forms of it.

One can simplify a Boolean function to minimal POS form by:
\ben
\item taking dual of the expression,
\item simplifying it to minimal SOP form, and
\item taking dual of it.
\een
\ssc{Incompletely Specified Boolean Functions (ISF) or Don't-care Functions}
An incompletely specified function (ISF) or a don't-care function is a Boolean function but in which for some input combinations, the output is not defined or irrelevant, often denoted as $X$. Those input combinations are called don't-care conditions. The definition of ON-set and OFF-set of an ISF is the same as completely specified function; the Don't-Care set or DC-set of an ISF is the set of all don't-care conditions.

The output for the don't-care conditions can be assigned either $0$ or $1$ such that the minimal SOP (or POS) forms have the fewest terms.

An ISF $f$ with ON-set $S$, OFF-set $T$, and DC-set $D$ is sometimes written similar to minterm expansion as
\[f=\sum m(i\in S)+\sum d(i\in D),\]
and similar to maxterm expansion as
\[f=\prod M(i\in T)+\prod d(i\in D),\]
where each $d_i$ is called a don't-care terms.
\ssc{Karnaugh Maps}
A Karnaugh Map, aka a K-map, is a grid that visualize the ways to simplify a completely or incompletely specified Boolean function to a minimal SOP form.
\sssc{Draw Karnaugh Maps of less than five variables}
If the ouput of the input combination of a cell is $1$, we write $1$ in that cell; if the ouput of the input combination of a cell is $0$, we can leave that cell empty or write $0$ in that cell; if the input combination of a cell is a don't care condition, we write $X$ in that cell.

The order 00, 01, 11, 10 is the gray code order, which is such that adjacent cells in the K-map differ by only one variable.

The notation of a cell is the input combination of that cell (thus a binary number), called binary notation, or the decimal integer in BCD representated by the input combination of that cell, called decimal notation.

A K-map for a $2$-variable Boolean function $f(A,B)$ is
\begin{longtable}[c]{c|c|c|}
\multicolumn{1}{c}{\thead{\backslashbox{$B$}{$A$}}} & \multicolumn{1}{c}{\thead{0}} & \multicolumn{1}{c}{\thead{1}} \\\cline{2-3}
\multicolumn{1}{c|}{\thead{0}} & $f(0,0)$ & $f(1,0)$ \\\cline{2-3}
\multicolumn{1}{c|}{\thead{1}} & $f(0,1)$ & $f(1,1)$ \\\cline{2-3}
\end{longtable}

A K-map for a $3$-variable Boolean function $f(A,B,C)$ is
\begin{longtable}[c]{c|c|c|}
\multicolumn{1}{c}{\thead{\backslashbox{$BC$}{$A$}}} & \multicolumn{1}{c}{\thead{0}} & \multicolumn{1}{c}{\thead{1}} \\\cline{2-3}
\multicolumn{1}{c|}{\thead{00}} & $f(0,0,0)$ & $f(1,0,0)$ \\\cline{2-3}
\multicolumn{1}{c|}{\thead{01}} & $f(0,0,1)$ & $f(1,0,1)$ \\\cline{2-3}
\multicolumn{1}{c|}{\thead{11}} & $f(0,1,1)$ & $f(1,1,1)$ \\\cline{2-3}
\multicolumn{1}{c|}{\thead{10}} & $f(0,1,0)$ & $f(1,1,0)$ \\\cline{2-3}
\end{longtable}

A K map for a $4$-variable Boolean function $f(A,B,C,D)$ is
\begin{longtable}[c]{c|c|c|c|c|}
\multicolumn{1}{c}{\thead{\backslashbox{$CD$}{$AB$}}} & \multicolumn{1}{c}{\thead{00}} & \multicolumn{1}{c}{\thead{01}} & \multicolumn{1}{c}{\thead{11}} & \multicolumn{1}{c}{\thead{10}} \\\cline{2-5}
\multicolumn{1}{c|}{\thead{00}} & $f(0,0,0,0)$ & $f(0,1,0,0)$ & $f(1,1,0,0)$ & $f(1,0,0,0)$ \\\cline{2-5}
\multicolumn{1}{c|}{\thead{01}} & $f(0,0,0,1)$ & $f(0,1,0,1)$ & $f(1,1,0,1)$ & $f(1,0,0,1)$ \\\cline{2-5}
\multicolumn{1}{c|}{\thead{11}} & $f(0,0,1,1)$ & $f(0,1,1,1)$ & $f(1,1,1,1)$ & $f(1,0,1,1)$ \\\cline{2-5}
\multicolumn{1}{c|}{\thead{10}} & $f(0,0,1,0)$ & $f(0,1,1,0)$ & $f(1,1,1,0)$ & $f(1,0,1,0)$ \\\cline{2-5}
\end{longtable}
\sssc{Group 1's}
The first step of simplifying a Boolean function using K-maps is grouping 1's (minterms).

A group is a rectangle of $2^m\times 2^n$ cells that are either $1$ (minterm) or $X$ (don't-care condition), in which $m,n\in\mathbb{N}_0$, the leftmost and rightmost columns are horizontally adjacent, and the top and bottom rows are vertically adjacent. Cells can be in multiple groups.

Each group represents an implicant of $f$ by the following rules:
\ben
\item Let the set of all input variables that are $1$ in all cells in the group be $O$.
\item Let the set of all input variables that are $0$ in all cells in the group be $Z$.
\item The implicant represented by that group is $\prod_{o\in O}o\prod_{z\in Z}z'$.
\een
\sssc{Select collections of groups}
An allowed collection of groups must follow the following rules:
\bit
\item Any $1$ must be in at least one group.
\item If any $1$ is only covered by one possible group, then that group must be chosen, that is, any group representing an essential prime implicant must be chosen.
\item If any group $g$ is completely covered by another group, $g$ must not be chosen, that is, a group chosen must represent a prime implicant.
\eit
Find the allowed collections of groups that contain the fewest groups. For each one of them, the sum of the implicants corresponding to the groups in it is a minimal SOP form of the given function.
\sssc{Five-variable Karnaugh Maps}
One way of drawing a five-variable K-map is by placing one four-variable K-map on top of another, that is, for a $5$-variable Boolean function $f(A,B,C,D,E)$, the K-map is
{\fontsize{8pt}{12pt}\selectfont
\begin{longtable}[c]{cc|c|c|c|c|}
& \multicolumn{1}{c}{\thead{\backslashbox{$BC$}{$DE$}}} & \multicolumn{1}{c}{\thead{00}} & \multicolumn{1}{c}{\thead{01}} & \multicolumn{1}{c}{\thead{11}} & \multicolumn{1}{c}{\thead{10}} \\\cline{3-6}
& \multicolumn{1}{c|}{\thead{00}} & \backslashbox{$f(1,0,0,0,0)$}{$f(0,0,0,0,0)$} & \backslashbox{$f(1,0,1,0,0)$}{$f(0,0,1,0,0)$} & \backslashbox{$f(1,1,1,0,0)$}{$f(0,1,1,0,0)$} & \backslashbox{$f(1,1,0,0,0)$}{$f(0,1,0,0,0)$} \\\cline{3-6}
$A$ & \multicolumn{1}{c|}{\thead{01}} & \backslashbox{$f(1,0,0,0,1)$}{$f(0,0,0,0,1)$} & \backslashbox{$f(1,0,1,0,1)$}{$f(0,0,1,0,1)$} & \backslashbox{$f(1,1,1,0,1)$}{$f(0,1,1,0,1)$} & \backslashbox{$f(1,1,0,0,1)$}{$f(0,1,0,0,1)$} \\\cline{3-6}
\backslashbox{$1$}{$0$} & \multicolumn{1}{c|}{\thead{11}} & \backslashbox{$f(1,0,0,1,1)$}{$f(0,0,0,1,1)$} & \backslashbox{$f(1,0,1,1,1)$}{$f(0,0,1,1,1)$} & \backslashbox{$f(1,1,1,1,1)$}{$f(0,1,1,1,1)$} & \backslashbox{$f(1,1,0,1,1)$}{$f(0,1,0,1,1)$} \\\cline{3-6}
& \multicolumn{1}{c|}{\thead{10}} & \backslashbox{$f(1,0,0,1,0)$}{$f(0,0,0,1,0)$} & \backslashbox{$f(1,0,1,1,0)$}{$f(0,0,1,1,0)$} & \backslashbox{$f(1,1,1,1,0)$}{$f(0,1,1,1,0)$} & \backslashbox{$f(1,1,0,1,0)$}{$f(0,1,0,1,0)$} \\\cline{3-6}
\end{longtable}}
in which the adjacency between cells in the same layer of four-variable K-map is the same as that in four-variable K-map; the upper and lower triangles in the same square are adjacent; and a group is a cuboid of $2^l\times 2^m\times 2^n$ cells that are either $1$ (minterm) or $X$ (don't-care condition), in which $l,m,n\in\mathbb{N}_0$.
\ssc{Map-entered variable (MEV) or Variable entrant map (VEM)}
Map-entered variable (MEV), aka variable entrant map (VEM), simplification is an extension of Karnaugh maps used to simplify Boolean functions with typically more than four variables by allowing some variables to be entered as symbols or expressions inside the K-map cells rather than expanding the map's dimensions.
\sssc{One MEV}
Suppose we have an $n$-variable (usually five-variable) completely or incompletely specified Boolean function $f$ with $E$ being one of the input variable of $f$ and being the MEV.
\ben
\item Draw a $(n-1)$-variable K-map for input variables except $E$. In each cell, let the $n-1$-variable input combination corresponding to the cell be $P$, where $f(P,E)$ denotes a function of $E$ where the variables except $E$ be their values in $P$:
\bit
\item If $f(P,1)=f(P,0)=X$, enter $X$.
\item If $f(P,1)=1$ and $f(P,0)=0$, enter $E$.
\item If $f(P,1)=0$ and $f(P,0)=1$, enter $E'$.
\item If $f(P,1)=f(P,0)=1$, enter $1$.
\item If $f(P,1)=1$ and $f(P,0)=X$, enter $1$.
\item If $f(P,1)=X$ and $f(P,0)=1$, enter $1$.
\item If $f(P,1)=0=f(P,0)=0$, leave the cell empty or enter $0$.
\item If $f(P,1)=0$ and $f(P,0)=X$, leave the cell empty or enter $0$.
\item If $f(P,1)=X$ and $f(P,0)=0$, leave the cell empty or enter $0$.
\eit
That is, let the set of products of $n-1$ literals that corresponding to a $1$ cell be $A_1$, the set of products of $n-1$ literals that corresponding to a $E$ cell be $A_E$, the set of products of $n-1$ literals that corresponding to a $E'$ cell be $A_e$, and the set of products of $n-1$ literals that corresponding to a $X$ cell be $A_X$, we write $f$ as:
\[f=\sum m(i\mid m_i\in A_1)+E\sum m(i\mid m_i\in A_E)+E'\sum m(i\mid m_i\in A_e)+\sum d(i\mid m_i\in A_X),\]
where $m_i$ is minterms in $(n-1)$-variable truth tables.
\item Group 1's in same way as K-map, that is, $1$ and $X$ are allowed in a group. Each such group corresponding to a $(n-1)$-variable implicant.
\item Group $E$s in same way as K-map except replacing $1$ with $E$ and $X$ with $1$ and $X$, that is, $E$, $1$, and $X$ are allowed in a group. Each such group corresponding to a $n$-variable implicant with $E$ be one of the literals
\item Group $E'$s in same way as K-map except replacing $1$ with $E'$ and $X$ with $1$ and $X$, that is, $E'$, $1$, and $X$ are allowed in a group. Each such group corresponding to a $n$-variable implicant with $E'$ be one of the literals.
\item An allowed collection of groups must follow the following rules:
\bit
\item Any $1$, $E$, or $E'$ must be in at least one group.
\item If any $1$, $E$, or $E'$ is only covered by one possible group, then that group must be chosen, that is, any group representing an essential prime implicant must be chosen.
\item If any group $g$ is completely covered by another group, $g$ must not be chosen, that is, a group chosen must represent a prime implicant.
\eit
\item Find the allowed collections of groups that contain the fewest groups. For each one of them, the sum of the implicants corresponding to the groups in it is a minimal SOP form of the given function.
\eit
\sssc{Multiple MEVs}
Suppose we have an $n$-variable completely or incompletely specified Boolean function $f$ in which there exist $m$ (usually $n-4$) variables $E_1,E_2,\ldots E_m$ chosen as MEVs such that given a $(n-m)$-variable input combination $P$ of the other variables, the set of all possible values of $f(P,Q,E_i)$ is a subset of $\{0,X\}$ or a subset of $\{1,X\}$, with $X$ denoting don't-care, for any fixed $(m-1)$-variable input combination $Q$ of the MEVs except $E_i$, where $f(P,Q,E_i)$ denotes a function of $E_i$ where the output when $E_i$ be $x\in\{1,0\}$, $f(P,Q,x)$ equals the output of $f$ when the variables except MEVs be their values in $P$, the MEVs except $E_i$ be their values in $Q$, and $E_i$ be $x$.
\ben
\item Draw a $(n-m)$-variable K-map for input variables except $E$ and $F$. In each cell, let the $(n-m)$-variable input combination corresponding to the cell be $P$, for any $i\in\mathbb{N}_{\leq m}$:
\bit
\item If $f(P,R)=X$ for any fixed $m$-variable input combination $Q$ of the MEVs, where $f(P,Q)$ denotes the output of $f$ when the variables except MEVs be their values in $P$ and the MEVs be their values in $Q$, enter $X$.
\item If the set of all possible values of $f(P,Q,1)$ is $\{1\}$ and the set of all possible values of $f(P,Q,0)$ is $\{0\}$, for any fixed $(m-1)$-variable input combination $Q$ of the MEVs except $E_i$, enter $E_i$.
\item If the set of all possible values of $f(P,Q,1)$ is $\{0\}$ and the set of all possible values of $f(P,Q,0)$ is $\{1\}$, for any fixed $(m-1)$-variable input combination $Q$ of the MEVs except $E_i$, enter $E_i'$.
\item If the set of all possible values of $f(P,Q,1)$ and the set of all possible values of $f(P,Q,0)$ are both $\{1\}$ for any fixed $(m-1)$-variable input combination $Q$ of the MEVs except $E_i$, enter $1$.
\item If the set of all possible values of $f(P,Q,1)$ is $\{1\}$ and the set of all possible values of $f(P,Q,0)$ is $\{X\}$, for any fixed $(m-1)$-variable input combination $Q$ of the MEVs except $E_i$, enter $1$.
\item If the set of all possible values of $f(P,Q,1)$ is $\{X\}$ and the set of all possible values of $f(P,Q,0)$ is $\{1\}$, for any fixed $(m-1)$-variable input combination $Q$ of the MEVs except $E_i$, enter $1$.
\item If the set of all possible values of $f(P,Q,1)$ and the set of all possible values of $f(P,Q,0)$ are both $\{0\}$ for any fixed $(m-1)$-variable input combination $Q$ of the MEVs except $E_i$, leave the cell empty or enter $0$.
\item If the set of all possible values of $f(P,Q,1)$ is $\{0\}$ and the set of all possible values of $f(P,Q,0)$ is $\{X\}$, for any fixed $(m-1)$-variable input combination $Q$ of the MEVs except $E_i$, leave the cell empty or enter $0$.
\item If the set of all possible values of $f(P,Q,1)$ is $\{X\}$ and the set of all possible values of $f(P,Q,0)$ is $\{0\}$, for any fixed $(m-1)$-variable input combination $Q$ of the MEVs except $E_i$, leave the cell empty or enter $0$.
\eit
That is, let the set of products of $n-m$ literals that corresponding to a $1$ cell be $A_1$, the set of products of $n-m$ literals that corresponding to a $E_i$ cell be $A_{E_i}$, the set of products of $n-m$ literals that corresponding to a $E_i'$ cell be $A_{e_i}$, and the set of products of $n-m$ literals that corresponding to a $X$ cell be $A_X$, we write $f$ as:
\[f=\sum m(i\mid m_i\in A_1)+\sum_{i=1}^mE_i\sum m(i\mid m_i\in A_{E_i})+\sum_{i=1}^mE_i'\sum m(i\mid m_i\in A_{e_i})+\sum d(i\mid m_i\in A_X),\]
where $m_i$ is minterms in $(n-m)$-variable truth tables.
\item Group 1's in same way as K-map, that is, $1$ and $X$ are allowed in a group. Each such group corresponding to a $(n-m)$-variable implicant.
\item For each $i\in\bbN_{\leq m}$, group $E_i$s in same way as K-map except replacing $1$ with $E_i$ and $X$ with $1$ and $X$, that is, $E_i$, $1$, and $X$ are allowed in a group. Each such group corresponding to a $(n-m+1)$-variable implicant with $E_i$ be one of the literals
\item For each $i\in\bbN_{\leq m}$, group $E_i'$s in same way as K-map except replacing $1$ with $E_i'$ and $X$ with $1$ and $X$, that is, $E_i'$, $1$, and $X$ are allowed in a group. Each such group corresponding to a $(n-m+1)$-variable implicant with $E_i'$ be one of the literals.
\item An allowed collection of groups must follow the following rules:
\bit
\item Any $1$, $E_i$, or $E_i'$ must be in at least one group.
\item If any $1$, $E_i$, or $E_i'$ is only covered by one possible group, then that group must be chosen, that is, any group representing an essential prime implicant must be chosen.
\item If any group $g$ is completely covered by another group, $g$ must not be chosen, that is, a group chosen must represent a prime implicant.
\eit
\item Find the allowed collections of groups that contain the fewest groups. For each one of them, the sum of the implicants corresponding to the groups in it is a minimal SOP form of the given function.
\eit
\ssc{Quine-McCluskey Method}
The Quine-McCluskey method (along with the Petrick's method) is a systematic procedure to find the minimal SOP forms of a completely or incompletely specified Boolean function. Below, we first discuss about completely specified Boolean function.
\sssc{Minterm expansion}
First, we convert the function minterm expansion.

Below, we will take
\[f(a,b,c,d)=\sum m(0,1,2,5,6,7,8,9,10,14)\]
for example.
\sssc{Group minterms}
Group minterms with the same number of $1$'s in the input combination together and list them. Let the group of minterms with $k$ $1$'s be Group $k$. Group $k$ and Group $(k+1)$ are called adjacent.

For $f$:
\begin{longtable}[c]{|c|m|mmmm|}
\hline
Group 0 & 0 & 0 & 0 & 0 & 0 \\\hline
Group 1 & 1 & 0 & 0 & 0 & 1 \\\cline{2-6}
& 2 & 0 & 0 & 1 & 0 \\\cline{2-6}
& 8 & 1 & 0 & 0 & 0 \\\hline
Group 2 & 5 & 0 & 1 & 0 & 1 \\\cline{2-6}
& 6 & 0 & 1 & 1 & 0 \\\cline{2-6}
& 9 & 1 & 0 & 0 & 1 \\\cline{2-6}
& 10 & 1 & 0 & 1 & 0 \\\hline
Group 3 & 7 & 0 & 1 & 1 & 1 \\\cline{2-6}
& 14 & 1 & 1 & 1 & 0 \\\hline
\end{longtable}
\sssc{Combine minterms}
Compare minterms from adjacent groups and combine minterms that differ in only one input bit by
\[XY+XY'=X,\]
where $X$ is a product of literals and $Y$ is a literal.

(Note: It is obvious that two terms not in adjacent groups can't be combined by $XY+XY'=X$.)

Create next list by writing combined terms with $X$ preserved and $Y$, the differing bit, by $-$ (don't care), and writing not-touched terms as is.

Repeat combinations and create next list until no further combining is possible. The terms in this last list are the prime implicants of $f$.

For $f$:

List 1:
\begin{longtable}[c]{|c|m|mmmm|}
\hline
Group 0 & 0 & 0 & 0 & 0 & 0 \\\hline
Group 1 & 1 & 0 & 0 & 0 & 1 \\\cline{2-6}
& 2 & 0 & 0 & 1 & 0 \\\cline{2-6}
& 8 & 1 & 0 & 0 & 0 \\\hline
Group 2 & 5 & 0 & 1 & 0 & 1 \\\cline{2-6}
& 6 & 0 & 1 & 1 & 0 \\\cline{2-6}
& 9 & 1 & 0 & 0 & 1 \\\cline{2-6}
& 10 & 1 & 0 & 1 & 0 \\\hline
Group 3 & 7 & 0 & 1 & 1 & 1 \\\cline{2-6}
& 14 & 1 & 1 & 1 & 0 \\\hline
\end{longtable}

List 2:
\begin{longtable}[c]{|c|m|mmmm|}
\hline
Group 0 & 0,1 & 0 & 0 & 0 & - \\\cline{2-6}
& 0,2 & 0 & 0 & - & 0 \\\cline{2-6}
& 0,8 & - & 0 & 0 & 0 \\\hline
Group 1 & 1,5 & 0 & - & 0 & 1 \\\cline{2-6}
& 1,9 & - & 0 & 0 & 1 \\\cline{2-6}
& 2,6 & 0 & - & 1 & 0 \\\cline{2-6}
& 2,10 & - & 0 & 1 & 0 \\\cline{2-6}
& 8,9 & 1 & 0 & 0 & - \\\cline{2-6}
& 8,10 & 1 & 0 & - & 0 \\\hline
Group 2 & 5,7 & 0 & 1 & - & 1 \\\cline{2-6}
& 6,7 & 0 & 1 & 1 & - \\\cline{2-6}
& 6,14 & - & 1 & 1 & 0 \\\cline{2-6}
& 10,14 & 1 & - & 1 & 0 \\\hline
\end{longtable}

List 3:
\begin{longtable}[c]{|c|m|mmmm|}
\hline
Group 0 & 0,1,8,9 & - & 0 & 0 & - \\\cline{2-6}
& 0,2,8,10 & - & 0 & - & 0 \\\hline
Group 1 & 1,5 & 0 & - & 0 & 1 \\\cline{2-6}
& 2,6,10,14 & - & - & 1 & 0 \\\hline
Group 2 & 5,7 & 0 & 1 & - & 1 \\\cline{2-6}
& 6,7 & 0 & 1 & 1 & - \\\hline
\end{longtable}
\sssc{Create prime implicant chart}
Create the prime implicant chart, a chart where the column are minterms and rows are prime implicants and each cell is marked $\times$ if the corresponding minterm is implied by the corresponding prime implicant.

For $f$:
\begin{longtable}[c]{m|mmmmmmmmmm}
\hline
& 0 & 1 & 2 & 5 & 6 & 7 & 8 & 9 & 10 & 14 \\\hline
(0,1,8,9) b'c' & \times & \times & & & & & \times & \times & & \\
(0,2,8,10) b'd' & \times & & \times & & & & \times & & \times & \\
(1,5) a'c'd & & \times & & \times & & & & & & \\
(2,6,10,14) cd' & & & \times & & \times & & & & \times & \times \\
(5,7) a'bd & & & & \times & & \times & & & & \\
(6,7) a'bc & & & & & \times & \times & & & & \\
\end{longtable}
\sssc{Select ssential prime implicants}
When a prime implicant is selected for inclusion in the minimal SOP form, the chart is reduced by deleting the corresponding row and the columns which correspond to the minterms covered by that prime implicant.

If a column has only one $\times$, the prime implicant of that row is an essential prime implicant.

Select all essential prime implicants.

For $f$: Select $(0,1,8,9) b'c'$ to cover $m_9$, and select $(2,6,10,14) cd'$ to cover $m_{14}$.
\sssc{Petrick's method}
Given the reduced chart, number all prime implicants as $P_1,P_2,\ldots$.

For $f$:
\[P_1=a'c'd,\quad P_2=a'bd,\quad P_3=a'bc.\]

For each remaining minterm $m_i$, construct a sum term of all prime implicants that covers it, and let $P$ be the product of them.

For $f$:
\[P=(P_1+P_2)(P_2+P_3).\]

Expand $P$ into canonical SOP form (with each $P_i$ considered a literal).

For $f$:
\[P=P_1P_2+P_2+P_2P_3.\]

Reduce $P$ by applying
\[X+XY=X\]
until no more such reduction is possible.

For $f$:
\[P=P_2.\]

Each term in reduced $P$ represents a solution, that is, a set of prime implicants which covers all minterms in the reduced chart. The sets with the least number of literals in it are the minimal SOP forms of $f$.

For $f$: $P_2$ is the only term in reduced $P$, thus the only minimal SOP form is 
\[f=P_2=b'c'+cd'+a'bd.\]
\sssc{Incompletely specified Boolean functions}
For incompletely specified Boolean functions, we do the following change to the process for completely specified Boolean functions:
\ben
\item Treat don't-care terms as minterms when grouping minterm.
\item Omit don't-care terms columns when creating the prime implicant chart.
\een
\ssc{Two-level circuit}
The maximum number of gates cascaded in series between a circuit input and the output is referred to as the number of levels of gates. Thus, a function written in sum-of-products form or in product-of-sums form corresponds directly to a two-level gate circuit. As is usually the case in digital circuits where the gates are driven from flip-flop outputs, we will assume that all variables and their complements are available as circuit inputs.
For this reason, we will not normally count inverters which are connected directly to input variables when determining the number of levels in a circuit.





\end{document}
