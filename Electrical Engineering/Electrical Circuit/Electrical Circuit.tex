\documentclass[a4paper,12pt]{article}
\setcounter{secnumdepth}{5}
\setcounter{tocdepth}{3}
\input{/usr/share/LaTeX-ToolKit/template.tex}
\begin{document}
\title{Electrical Circuit}
\author{沈威宇}
\date{\temtoday}
\titletocdoc
\sct{Electrical Circuit}
\ssc{Introduction}
\sssc{Electrical network}
An electrical network is an interconnection of electrical components or a model of such an interconnection, consisting of electrical elements.
\sssc{Electrical element}
Electrical elements are conceptual abstractions representing idealized electrical components, such as resistors, capacitors, and inductors, used in the analysis of electrical networks.
\sssc{Lumped-element model, lumped-parameter model, or lumped-component model}
A simplified and idealized representation of an electrical network that assumes all components are concentrated at a single point and their behavior can be described by idealized mathematical models, that is, the lumped-matter discipline:
\bit
\item The change of the magnetic flux in time through the surface surrounded by a loop is zero.
\item The change of the charge in time inside conductor is zero.
\item Signal timescales of interest are much larger than propagation delay of electromagnetic waves across the lumped element.
\eit
such that the system is a dynamic system whose state space is a finite dimension, called lumped-parameter system.

In contrast, distributed parameter systems have infinite-dimensional state space.

We will concentrate on lumped-parameter networks in this text.
\ssc{Circuit element}
A circuit element is a two terminal electrical network that can be completely characterized by its voltage-current relationship, that is, with time $t$, voltage $v(t)$, and current $i(t)$, for some given three variable function $f$,
\[f\qty(v(t),i(t),t)=0.\]

We label + and - on the two terminal repsectively and voltage $v(t)$ between them to define the voltage $v(t)$ of the element as the electric potential difference of + terminal relative to - terminal.

We label arrow and current $i(t)$ on it to define the reference direction of the current $i(t)$ of the element.

If the current arrow enters the + terminal, the element is said to follow the passive sign convention. Under this convention, the power of the element absorbs is
\[p(t)=v(t)i(t),\]
where negative power means delivering.
\sssc{Ohm’s law}
The resistance $R$ of a circuit element is the magnitude of the fraction of its voltage over its current, while the conductance $G$ is $\frac{1}{R}$. Resistors discussed in this text are assumed to be linear conductor, that is, $R$ is independent from voltage and current.
\sssc{Wire or lead}
Circuit elements in a network are connected with wires (aka leads), which are assumed to have zero resistance.
\begin{center}
\begin{circuitikz}
\draw (0,0) to[short, l=Wire] (3,0);
\draw (5,0) to[crossing, l=Wire Crossing] (8,0);
\end{circuitikz}
\end{center}
\sssc{Sources}
Ideal sources are assumed to have zero resistance.

Independent sources:
\bit
\item An independent voltage source enforces a prescribed voltage $v_s(t)$ across its terminals.
\item An independent current source enforces a prescribed current $i_s(t)$ through its terminals.
\eit
Dependent or controlled sources:
\bit
\item A voltage-controlled voltage source enforces a voltage $v=\mu v_x$ across its terminals, where $v_x$ is control voltage and constant $\mu$ is dimensionless voltage gain.
\item A current-controlled voltage source enforces a voltage $v=ri_x$ across its terminals, where $i_x$ is control current and constant $r$ is transresistance.
\item A voltage-controlled current source enforces a current $i=gv_x$ through its terminals, where $v_x$ is control voltage and constant $g$ is transconductance.
\item A current-controlled current source enforces a cuurent $i=\beta i_x$ through its terminals, where $i_x$ is control current and constant $\beta$ is dimensionless current gain.
\eit
\begin{center}
\begin{circuitikz}
\draw (0,0) node[vsourcesinshape]{} (0,1) node[above]{Source};
\draw (5,0) node[vsourceAMshape]{} (5,1) node[above]{Independent Voltage Source};
\draw (10,0) node[isourceAMshape]{} (10,1) node[above]{Independent Current Source};
\end{circuitikz}
\begin{circuitikz}
\draw (0,0) node[cvsourceAMshape]{} (0,1) node[above]{Controlled Voltage Source};
\draw (5,0) node[cisourceAMshape]{} (5,1) node[above]{Controlled Current Source};
\end{circuitikz}
\end{center}
\sssc{Node}
A node is a point in a network where two or more elements are connected.
\sssc{Path}
A path is any route through which current can flow from one node to another.
\sssc{Branch}
A branch is a single element or a group of elements in series that connects two nodes.
\sssc{Loop}
A loop is any closed path in a network where you can start at a node, traverse branches, and return to the starting node without retracing any branch.
\sssc{Electrical circuit or closed circuit}
An electrical circuit is a network containing loops.
\sssc{Open circuit}
An open circuit is an electrical network that is not a circuit.
\sssc{Ground}
A ground is a node designated to have 0 electric potential. All other electric potential in the network are measured relative to the grounds.
\bit
\item \tb{Earth ground}: Earth ground is a point in the network that is physically connected to the Earth.
\item \tb{Signal ground}: Signal ground is a ground in the network provided by an external signal, which can be float relative to Earth and such circuit is called float. An example of signal ground is a GND pin on a microcontroller board.
\item \tb{Chassis ground}: Chassis ground is a point in the circuit that is physically connected to a devices's metal frame. The metal frame acts as a shield against electromagnetic interference and can be connected to Earth for safety in AC-powered devices.
\eit
\begin{center}
\begin{circuitikz}
\draw (0,0) node[ground]{} (0,0) node[above]{Earth Ground};
\draw (5,0) node[sground]{} (5,0) node[above]{Signal Ground};
\draw (10,0) node[cground]{} (10,0) node[above]{Chassis Ground};
\end{circuitikz}
\end{center}
\sssc{Kirchhoff's current law (KCL), Kirchhoff's first law, or Kirchhoff's junction rule}
The algebraic sum of the currents entering any node is zero.
\sssc{Kirchhoff's voltage law (KVL), Kirchhoff's second law, or Kirchhoff's loop rule}
The algebraic sum of the voltages around any closed path is zero.
\sssc{Series connection}
Two or more elements are connected in series if they are connected along a single path, and each component has the same current through it, equal to the current through the network. The voltage across the network is equal to the sum of the voltages across each element. The resistance of the network is equal to the sum of the resistances of each element.
\sssc{Parallel connection}
Two or more elements are connected in parallel if they are connected along multiple paths from one node to another, and each component has the same voltage across it, equal to the voltage through the network. The current through the network is equal to the sum of the current through each element. The conductance of the network is equal to the sum of the conductances of each element.
\sssc{Y-Δ transformation and Δ-Y transformation}
\bit
\item Y (aka wye or star) network: a common central node $N$, three terminals $N_1,N_2,N_3$, and three resistors with resistances $R_1,R_2,R_3$ connecting $N_1$ and $N$, $N_2$ and $N$, and $N_3$ and $N$ respectively.
\item Δ (delta): three terminals $N_1,N_2,N_3$, and three resistors with resistances $R_{12},R_{23},R_{31}$ connecting $N_1$ and $N_2$, $N_2$ and $N_3$, and $N_3$ and $N_1$ respectively.
\item Y-Δ (aka wye-delta or star-delta) transformation: The two network are equivalent if
\[R_{12}=R_1+R_2+\frac{R_1R_2}{R_3},\]
\[R_{23}=R_2+R_3+\frac{R_2R_3}{R_1},\]
\[R_{31}=R_3+R_1+\frac{R_3R_1}{R_2}.\]
\item Δ-Y (aka delta-wye or star-delta) transformation: The two network are equivalent if
\[R_1=\frac{R_{12}R_{31}}{R_{12}+R_{23}+R_{31}},\]
\[R_2=\frac{R_{23}R_{12}}{R_{12}+R_{23}+R_{31}},\]
\[R_3=\frac{R_{31}R_{23}}{R_{12}+R_{23}+R_{31}}.\]
\eit
\sssc{Generalized Y-Δ transformation and Δ-Y transformation}
For all integer $n\geq 3$:
\bit
\item $n$-star network: a common central node $N$, $n$ terminals $N_1,N_2,\ldots,N_n$, and $n$ resistors with resistances $R_i$ connecting $N_i$ and $N$ for all $i\in\bbN\land i\leq n$.
\item $n$-gon network: $n$ terminals $N_1,N_2,\ldots,N_n$, and $\frac{n(n-1)}{2}$ resistors with resistances $R_{ij}$ connecting $N_i$ and $N_j$ for all $i,j\in\bbN\land i<j\leq n$.
\item Generalized Y-Δ transformation: The two network are equivalent if
\[R_{ij}=R_i+R_j+\frac{R_iR_j}{-R_i-R_j+\sum_{k=1}^nR_k}.\]
\item Generalized Δ-Y (aka delta-wye or star-delta) transformation: The two network are equivalent if
\[R_i=\frac{\prod_{k=1}^{i-1}R_{ki}\prod_{k=i+1}^nR_{ik}}{\sum_{k=1}^{n-1}\sum_{l=k+1}^nR_{kl}}.\]
\eit










\subsection{Circuit symbols}
\subsubsection{Switches}
\begin{center}
\begin{circuitikz}
\draw (0,0) to[switch, l=Switch] (3,0);
\draw (5,0) node[spdt]{} (5,0.5) node[above]{SPDT Switch};
\end{circuitikz}
\end{center}
\subsubsection{Resistors}
\begin{center}
\begin{circuitikz}
\draw (0,0) to[R, l=Resistor] (3,0);
\draw (5,0) to[vR, l=Variable Resistor] (8,0);
\draw (10,0) to[thermistor, l=Thermistor] (13,0);
\end{circuitikz}
\end{center}
\begin{center}
\begin{circuitikz}
\draw (0,0) to[ldR, l=Light Dependent Resistor] (3,0);
\draw (5,0) to[varistor, l=Varistor] (8,0);
\end{circuitikz}
\end{center}
\subsubsection{Capacitors and Inductor}
\begin{center}
\begin{circuitikz}
\draw (5,0) to[C, l=Capacitor] (8,0);
\draw (10,0) to[L, l=Inductor] (13,0);
\end{circuitikz}
\end{center}
\subsubsection{Diodes}
\begin{center}
\begin{circuitikz}
\draw (0,0) to[D, l=Diode] (3,0);
\draw (5,0) to[zDo, l=Zener Diode] (8,0);
\draw (10,0) to[pDo, l=Photodiode] (13,0);
\draw (15,0) to[led, l=LED] (18,0);
\end{circuitikz}
\end{center}
\subsubsection{Transistors}
\begin{center}
\begin{circuitikz}
\draw (0,0) node[npn]{} (0,1) node[above]{NPN Transistor};
\draw (5,0) node[pnp]{} (5,1) node[above]{PNP Transistor};
\draw (10,0) node[nmos]{} (10,1) node[above]{N-MOSFET};
\draw (15,0) node[pmos]{} (15,1) node[above]{P-MOSFET};
\end{circuitikz}
\end{center}
\subsubsection{Operational Amplifiers}
\begin{center}
\begin{circuitikz}
\draw (0,0) node[op amp]{} (0,1) node[above]{Op-Amp};
\end{circuitikz}
\end{center}
\subsubsection{Measuring Instruments}
\begin{center}
\begin{circuitikz}
\draw (0,0) to[voltmeter, l=Voltmeter] (3,0);
\draw (5,0) to[ammeter, l=Ammeter] (8,0);
\draw (10,0) to[ohmmeter, l=Ohmmeter] (13,0);
\end{circuitikz}
\end{center}
\subsubsection{Transformers}
\begin{center}
\begin{circuitikz}
\draw (0,0) node[transformer]{Transformer} (3,0);
\draw (5,0) node[transformer core]{Transformer Core} (8,0);
\end{circuitikz}
\end{center}
\ssc{Sources}



\ssc{Linear electrical network}
\sssc{Linear electrical network}
A linear electrical network is an electrical network which obeys the superposition principle that the output voltage is a linear combination of the input voltages.
\sssc{Thévenin's theorem}
Any linear time-invariant electricali network containing only voltage sources, current sources, and resistances with two terminals is equivalent to a combination of a voltage source in a series connection with a resistance, such equivalent circuit is called Thévenin equivalent or Thévenin form.
\sssc{Norton's theorem or Mayer–Norton theorem}
Any linear time-invariant electrical network containing only voltage sources, current sources, and resistances with two terminals is equivalent to a combination of a current source in a parallel connection with a resistance, such equivalent circuit is called Norton equivalent or Norton form.

