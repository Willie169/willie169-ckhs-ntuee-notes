\documentclass[a4paper,12pt]{article}
\setcounter{secnumdepth}{5}
\setcounter{tocdepth}{3}
\input{/usr/share/LaTeX-ToolKit/template.tex}
\begin{document}
\title{Electrical Circuit}
\author{沈威宇}
\date{\temtoday}
\titletocdoc
\sct{Electrical Circuit}
\ssc{Introduction}
\sssc{Electrical network}
An electrical network is an interconnection of electrical components or a model of such an interconnection, consisting of electrical elements.
\sssc{Electrical element}
Electrical elements are conceptual abstractions representing idealized electrical components, such as resistors, capacitors, and inductors, used in the analysis of electrical networks and denoted with circuit symbols, also called electronic symbol, in circuit diagrams.
\sssc{Passive element and active element}
A passive element is an element that cannot generate power. An active element is an element that can generate power.
\sssc{Lumped-element model, lumped-parameter model, or lumped-component model}
A simplified and idealized representation of an electrical network that assumes all components are concentrated at a single point and their behavior can be described by idealized mathematical models, that is, the lumped-matter discipline:
\bit
\item The change of the magnetic flux in time through the surface surrounded by a loop is zero.
\item The change of the charge in time inside conductor is zero.
\item Signal timescales of interest are much larger than propagation delay of electromagnetic waves across the lumped element.
\eit
such that the system is a dynamic system whose state space is a finite dimension, called lumped-parameter system.

In contrast, distributed parameter systems have infinite-dimensional state space.

We will concentrate on lumped-parameter networks in this text.
\ssc{Circuit element}
A circuit element is a two terminal electrical network that can be completely characterized by its voltage-current relationship, that is, with time $t$, voltage $v(t)$, and current $i(t)$, for some given three variable function $f$,
\[f\qty(v(t),i(t),t)=0.\]

We label + and - on the two terminal repsectively and voltage $v(t)$ between them to define the voltage $v(t)$ of the element as the electric potential difference of + terminal relative to - terminal.

We label arrow and current $i(t)$ on it to define the reference direction of the current $i(t)$ of the element.

If the current arrow enters the + terminal, the element is said to follow the passive sign convention. Under this convention, the power absorbed by the element is
\[p(t)=v(t)i(t),\]
with negative power means delivering.
\sssc{Node}
A node is a point in a network where two or more elements are connected.
\sssc{Pole or terminal}
A pole or a terminal is a node that is available for connection to an external network.
\sssc{Path}
A path is any route through which current can flow from one node to another.
\sssc{Branch}
A branch is a single element or a group of elements in series that connects two nodes.
\sssc{Loop}
A loop is any closed path in a network where you can start at a node, traverse branches, and return to the starting node without retracing any branch.
\sssc{Mesh}
A mesh is a loop that contains no other loop inside it.
\sssc{Electrical circuit or closed circuit}
An electrical circuit is a network containing loops.
\sssc{Open circuit}
An open circuit is an electrical network that is not a circuit.
\sssc{Port}
A port is a pair of terminals through which current can enter or leave a network and satisfies the port condition: the current flowing into one pole from outside is equal to the current flowing out of the other pole to outside.

An $n$-port network is a network with $2n$ poles forming $n$ ports.
\sssc{Resistor}
An ideal resistor is a circuit element with a resistance and no capacitance or inductance.
\begin{center}
\begin{circuitikz}
\draw (0,0) to[R, l=Resistor] (3,0);
\draw (5,0) to[vR, l=Variable Resistor] (8,0);
\draw (10,0) to[thermistor, l=Thermistor] (13,0);
\end{circuitikz}
\begin{circuitikz}
\draw (0,0) to[ldR, l=Light Dependent Resistor] (3,0);
\draw (5,0) to[varistor, l=Varistor] (8,0);
\end{circuitikz}
\end{center}
\subsubsection{Capacitor}
An ideal capacitor is a circuit element with a capacitance and no resistance or inductance.
\begin{center}
\begin{circuitikz}
\draw (0,0) to[C, l=Capacitor] (3,0);
\end{circuitikz}
\end{center}
\subsubsection{Inductor}
An ideal inductor is a circuit element with an inductance and no resistance or capacitance.
\begin{center}
\begin{circuitikz}
\draw (0,0) to[L, l=Inductor] (3,0);
\end{circuitikz}
\end{center}
\sssc{Sinusoidal alternating current}
The voltage and current of a sinusoidal AC
\[v(t)=v_m\cos(\omega t+\phi),\]
\[i(t)=i_m\cos(\omega t+\theta),\]
are represented in phasors $v_m\angle\phi$ and $i_m\angle\theta$, which can be treated as complex numbers $v_me^{i\phi}$ and $i_me^{i\theta}$.

When the context is clear, sinusoidal AC may be called simply AC.

A frequency range of sinusoidal AC is called bandwidth.

Here, all currents discussed are either sinusoidal AC or DC.
\sssc{Reactance}
Let the capacitance of a circuit element under alternating current at an angular frequency of $\omega$ be $C$. Then its capacitive reactance $X_C$ is defined as:
\[X_C=\frac{1}{\omega C}.\]
There is also a definition that is the opposite of the above definition, i.e., the capacitive reactance is negative, which is not used here.

Let the inductance of a circuit element under alternating current with an angular frequency of $\omega$ be $L$. Then its inductive reactance $X_L$ is defined as:
\[X_L=\omega L.\]
The reactance $X$ of a circuit element is defined as:
\[X=X_L-X_C.\]
A circuit element is said to have net capacitive reactance if $X<0$ and have net inductive reactance if $X>0$.
\sssc{Impedance}
The impedance $Z$ of a one-port network is defined as a complex number with its resistance as real part and its reactance as imaginary part:
\[Z=R+jX,\]
where $j=\sqrt{-1}$ is the imaginary unit (since $i$ is used by current).

For AC in phasor notation, we can write
\[v=Zi.\]
\sssc{Admittance}
The admittance $Y$ of a one-port network is defined as the reciprocal of its impedance:
\[Y=\frac{1}{Z}.\]
\sssc{Conductance}
The conductance $G$ of a circuit element is defined as the real part of its admittance, for circuit element under direct current, that is, the reciprocal of its resistance.
\sssc{Susceptance}
The susceptance $B$ of a circuit element is defined as the imaginary part of its admittance:
\[Y=G+jB.\]
\sssc{Phase shift by impedance}
A circuit element with resistance $R$ and reactance $X$ introduces a change $\arctan(\frac{X}{R})$ in phase angle difference of current relative to voltage to the output current relative to input current. For in-phase input AC, circuit element with $X>0$ makes output current lags voltage, and circuit element with $X<0$ makes output current leads voltage.
\sssc{Wire or lead}
Circuit elements in a network are connected with wires (aka leads), which are assumed to have zero impedance.
\begin{center}
\begin{circuitikz}
\draw (0,0) to[short, l=Wire] (3,0);
\draw (5,0) to[crossing, l=Wire Crossing] (8,0);
\end{circuitikz}
\end{center}
Adding a wire directly between two nodes is called adding a short circuit.
\sssc{Linear element and Ohm's law}
A linear element is a passive element whose resistance $R$, capacitance $C$, and inductance $L$ are assumed to be independent of voltage and current, that is, equivalent to a one-port network consists of ideal resistors, capacitors, and inductors, that is, the voltage-current relationship under sinusoidal AC is a complex (real for DC) equation $v(t)=Z\cdot i(t)$ ($v(t)=R\cdot i(t)$) with voltage $v(t)$, current $i(t)$, and impedance $Z$ (resistance $R$ for DC). The assumption for resistance is the Ohm's law.
\sssc{Souces}
Ideal sources are active elements assumed to have zero impedance.

\tb{Independent sources:}
\bit
\item An independent voltage source enforces a prescribed DC or sinusoidal AC voltage $v_s(t)$ across its terminals.
\item An independent current source enforces a prescribed DC or sinusoidal AC current $i_s(t)$ through its terminals.
\eit
\begin{center}
\begin{circuitikz}
\draw (0,0) node[vsourcesinshape]{} (0,1) node[above]{Source};
\draw (5,0) node[vsourceAMshape]{} (5,1) node[above]{Independent Voltage Source};
\draw (10,0) node[isourceAMshape]{} (10,1) node[above]{Independent Current Source};
\end{circuitikz}
\end{center}
\tb{Linear dependent or controlled sources:}
\bit
\item A voltage-controlled voltage source enforces a DC or sinusoidal AC voltage $v_s(t)=\mu v_x(t)$ across its terminals, where $v_x(t)$ is control voltage and real for DC or complex for AC constant $\mu$ is dimensionless voltage gain.
\item A current-controlled voltage source enforces a DC or sinusoidal AC voltage $v_s(t)=ri_x(t)$ across its terminals, where $i_x(t)$ is control current and real for DC or complex for AC constant $r$ is transresistance.
\item A voltage-controlled current source enforces a DC or sinusoidal AC current $i_s(t)=gv_x(t)$ through its terminals, where $v_x(t)$ is control voltage and real for DC or complex for AC constant $g$ is transconductance.
\item A current-controlled current source enforces a DC or sinusoidal AC current $i_s(t)=\beta i_x(t)$ through its terminals, where $i_x(t)$ is control current and real for DC or complex for AC constant $\beta$ is dimensionless current gain.
\eit
\begin{center}
\begin{circuitikz}
\draw (0,0) node[cvsourceAMshape]{} (0,1) node[above]{Controlled Voltage Source};
\draw (5,0) node[cisourceAMshape]{} (5,1) node[above]{Controlled Current Source};
\end{circuitikz}
\end{center}
\sssc{Practical souces}
A practical independent voltage source can be approximated with an ideal independent voltage source in series connection with a resistor.

A practical independent current source can be approximated with an ideal independent current source in parallel connection with a resistor.
\sssc{Ground}
A ground is a node designated to have 0 electric potential. All other electric potential in the network are measured relative to the grounds.
\bit
\item \tb{Earth ground}: Earth ground is a point in the network that is physically connected to the Earth.
\item \tb{Signal ground}: Signal ground is a ground in the network provided by an external signal, which can be float relative to Earth and such circuit is called float. An example of signal ground is a GND pin on a microcontroller board.
\item \tb{Chassis ground}: Chassis ground is a point in the circuit that is physically connected to a devices's metal frame. The metal frame acts as a shield against electromagnetic interference and can be connected to Earth for safety in AC-powered devices.
\eit
\begin{center}
\begin{circuitikz}
\draw (0,0) node[ground]{} (0,0) node[above]{Earth Ground};
\draw (5,0) node[sground]{} (5,0) node[above]{Signal Ground};
\draw (10,0) node[cground]{} (10,0) node[above]{Chassis Ground};
\end{circuitikz}
\end{center}
\sssc{Kirchhoff's current law (KCL), Kirchhoff's first law, or Kirchhoff's junction rule}
The algebraic sum of the currents entering any node is zero.
\sssc{Kirchhoff's voltage law (KVL), Kirchhoff's second law, or Kirchhoff's loop rule}
The algebraic sum of the voltages around any closed path is zero.
\sssc{Network as graph}
A network is a graph with nodes being vertices and branches without nodes inside except ends being edges.
\sssc{Planar network}
A network that is planar as a graph.
\sssc{Linear network and linear circuit}
A linear network is a network which consists of only linear elements, independent sources, and linear dependent sources.

A linear circuit is a circuit which obeys the superposition principle of voltages and currents (but not power) that the voltages across or currents through all one-port subnetwork, called responses or response functions, are linear combinations of the input voltages from independent voltage sources and input currents from independent current sources in it, called forcing functions, that is, consists of only linear elements, independent sources, and linear dependent sources.
\sssc{Series connection}
Two or more elements are connected in series if they are connected along a single path, and each component has the same current through it, equal to the current through the network. The voltage across the network is equal to the sum of the voltages across each element. For linear networks, the impedance of the network is equal to the sum of the impedances of each element.
\sssc{Parallel connection}
Two or more elements are connected in parallel if they are connected along multiple paths from one node to another, and each component has the same voltage across it, equal to the voltage through the network. The current through the network is equal to the sum of the current through each element. For linear networks, the admittance of the network is equal to the sum of the admittances of each element.
\sssc{Bilateral network}
A bilateral electrical network is an electrical network whose behaviors remain the same regardless of the direction of current flow or applied voltage.
\ssc{Nodal Analysis or Node-voltage Analysis of Linear Circuit under DC or Sinusoidal AC}
\sssc{Supernode}
If a voltage source is connected between two nodes, enclose both nodes and the source as a supernode.
\sssc{Linear circuit as graph}
A linear circuit is a graph with nodes with all voltage sources enclosed as supernodes being vertices, branches without nodes inside except ends being edges, and admittances (or conductance for DC) being weights.
\sssc{Admittance matrix}
The admittance (or conductance for DC) matrix of a linear circuit is the Laplacian matrix of it.
\sssc{Nodal analysis or node-voltage analysis}
\ben
\item Form supernodes to enclose all voltage sources.
\item The equations given by KCL for all nodes form a system of linear equations:
\[\mb{Y}\mb{v}=\mb{i},\]
where $\mb{v}$ is the column vector of nodal voltages (directly as variables) relative to a common reference node, $\mb{i}$ is the vector of current injected into the nodes, where each entry $i_k$ is zero if node $k$ is not connected to a current source and $i_s$ if node $k$ is connected to a current source of current $i_s$ injected to the node (negative for absorption, direct substituted for VCCS and CCCS), and $\mb{Y}$ is the admittance (or conductance for DC) matrix of the linear circuit.
\item Additional linear equations of VCVS and CCVS are added to the linear system.
\een
\ssc{Mesh Analysis of Planar Linear Circuit under DC or Sinusoidal AC}
\sssc{Supermesh}
If a current source lies on two or more meshes, exclude the edge with the current source and merge the meshes into one mesh.
\sssc{Mesh analysis}
\ben
\item Indentify all meshes and form supermeshes to exclude all current sourcs.
\item For each mesh, assign a mesh current variable for it with an arbitrary but consistent direction.
\item Write the linear equation of mesh current variables given by KVL for every mesh.
\item Additional linear equations of current sources are addded to the linear system.
\item Additional linear equations of dependent sources are addded to the linear system.
\een
\ssc{Transformations}
\sssc{Source transformation}
An independent voltage source  $v$ in series connection with an impedance $Z=R+jX$ is equivalent to an independent current source $i=\frac{v}{Z}$ in parallel connection with the same impedance $Z$.
\sssc{Y-Δ transformation and Δ-Y transformation}
\bit
\item Y (wye or star) network: a common central node $N$, three terminals $N_1,N_2,N_3$, and three circuit elements with impedances $Z_1,Z_2,Z_3$ connecting $N_1$ and $N$, $N_2$ and $N$, and $N_3$ and $N$ respectively.
\item Δ (delta or triangle) network: three terminals $N_1,N_2,N_3$, and three circuit elements with impedances $Z_{12},Z_{23},Z_{31}$ connecting $N_1$ and $N_2$, $N_2$ and $N_3$, and $N_3$ and $N_1$ respectively.
\item Y-Δ transformation: The two networks are equivalent if
\[Z_{12}=Z_1+Z_2+\frac{Z_1Z_2}{Z_3},\]
\[Z_{23}=Z_2+Z_3+\frac{Z_2Z_3}{Z_1},\]
\[Z_{31}=Z_3+Z_1+\frac{Z_3Z_1}{Z_2}.\]
\item Δ-Y transformation: The two networks are equivalent if
\[Z_1=\frac{Z_{12}Z_{31}}{Z_{12}+Z_{23}+Z_{31}},\]
\[Z_2=\frac{Z_{23}Z_{12}}{Z_{12}+Z_{23}+Z_{31}},\]
\[Z_3=\frac{Z_{31}Z_{23}}{Z_{12}+Z_{23}+Z_{31}}.\]
\eit
\sssc{Star-mesh transformation}
For all integer $n\geq 3$:
\bit
\item $n$-star network: a common central node $N$, $n$ terminals $N_1,N_2,\ldots,N_n$, and $n$ circuit elements with impedances $Z_i$ connecting $N_i$ and $N$ for all $i\in\bbN\land i\leq n$.
\item $n$-mesh (complete graph) network: $n$ terminals $N_1,N_2,\ldots,N_n$, and $\frac{n(n-1)}{2}$ circuit elements  with impedances $Z_{ij}$ connecting $N_i$ and $N_j$ for all $i,j\in\bbN\land i<j\leq n$.
\item Star-mesh transformation: The two networks are equivalent if
\[Z_{ij}=Z_i+Z_j+\frac{Z_iZ_j}{-Z_i-Z_j+\sum_{k=1}^nZ_k}.\]
\eit
\ssc{Thévenin's theorem and Norton's theorem}
\sssc{Thévenin's theorem}
Any linear network with two terminals under DC or sinusoidal AC is equivalent to a voltage source with a voltage called Thévenin voltage in a series connection with a linear element (or resistor for DC) with an impedance (or resistance for DC) called Thévenin impedance (or resistance for DC), such equivalent circuit is called Thévenin equivalent.
\sssc{Norton's theorem}
Any linear network with two terminals under DC or sinusoidal AC is equivalent to a current source with a current called Norton current in a parallel connection with a linear element (or resistor for DC) with an impedance (or resistance for DC) called Norton impedance (or resistance for DC), such equivalent circuit is called Norton equivalent.
\sssc{Finding Thévenin voltage}
Remove any load, and the open-circuit voltage (phasor for AC) is the Thévenin voltage.
\sssc{Finding Norton current}
Add a short circuit between the two terminals, and the current on the it is the Norton current.
\sssc{Finding Thévenin impedance or Norton impedance}
Thévenin impedance and Norton impedance are the same and can be found with:
\bit
\item \tb{Source deactivation method}: Replace all independent voltage sources with short circuits and independent current sources with open circuits, and the impedance between the two terminals is the Thévenin impedance or Norton impedance.
\item \tb{Division method}: The Thévenin impedance or Norton impedance is the Thévenin voltage divided by the Norton current.
\eit
\sssc{Maximum power theorem}
A linear network delivers maximum power to a load impedance when it is equal to the Thévenin impedance or its conjugate of the network.
\ssc{Linear circuit superposition procedure}
\ben
\item Select one of the independent sources. Set all other independent sources to zero. This means voltage sources are replaced with short circuits and current sources are replaced with open circuits. Leave dependent sources in the circuit.
\item Analyze the simplified circuit to find the desired currents and voltages.
\item Repeat previous steps until each independent source has been considered.
\item Sum the currents and voltages obtained from the separate analyses above.
\een
\ssc{SPICE}
\sssc{SPICE}
SPICE (Simulation Program with Integrated Circuit Emphasis) is a general-purpose, open-source analog electronic circuit simulator. It is a program used in integrated circuit and board-level design to check the integrity of circuit designs and to predict circuit behavior.
\sssc{Ngspice}
An open-source successor of SPICE.
\ssc{Operational Amplifier (op amp or op-amp)}
\sssc{Definition}
An op amp has five terminals:
\bit
\item $v_+$: non-inverting input voltage of the terminal at the $+$ sign;
\item $v_-$: inverting input voltage of the terminal at the $-$ sign;
\item $v_o$: output voltage of the terminal at the opposite side of inputs;
\item $v_{CC+}$: positive supply voltage, denoted on the $+$ side of an op-amp or omitted;
\item $v_{CC-}$: negative supply voltage, denoted on the $-$ side of an op-amp or omitted.
\eit
\begin{center}
\begin{circuitikz}
\draw (0,0) node[op amp] (op amp) {};
\end{circuitikz}
\end{center}
\[v_o=A(v_+-v_-),\]
where $A$ is open-loop gain.
\sssc{Ideal op amp}
\bit
\item Infinite open-loop gain: $A=\infty$.
\item Zero input offset voltage: $v_o=0$ when $v_+=v_-$.
\item Infinite input impedance: No current flow into either of the input terminals.
\item Zero output impedance: Can drive any load without output voltage change.
\item Infinite bandwidth: Gain is independent of frequency.
\eit
\sssc{Inverting amplifier}
An inverting amplifier consists of an op amp, an input resistor $R_i$, and a feedback resistor $R_f$, where input voltage $v_{in}$ is connected to $v_-$ through $R_i$, $v_+$ is connected to ground, output voltage $v_{out}$ is connected to $v_o$, and $v_o$ is fed back to $v_-$ through $R_f$.
\[i_{in}=\frac{v_{in}}{R_{in}}=i_f=\frac{-v_{out}}{R_f},\]
\[v_{out}=-\frac{R_f}{R_{in}}v_{in}.\]
\sssc{Noninverting amplifier}
A noninverting amplifier consist of an op amp, a ground resistor $R_g$, and a feedback resistor $R_f$, where input voltage $v_{in}$ is connected to $v_+$, $v_-$ is connected ground through $R_g$, output voltage $v_{out}$ is connected to $v_o$, and $v_o$ is fed back to $v_-$ through $R_f$.
\[v_{in}=v_+=v_-=v_{out}\frac{R_g}{R_f+R_g},\]
\[v_{out}=\qty(1+\frac{R_f}{R_g})v_{in}.\]
\sssc{Summing amplifier}
A summing amplifier consists of an op amp, input voltages $v_1,v_2,\ldots,v_n$, $n$ input resistors $R_1,R_2,\ldots,R_n$, and a feedback resistor $R_f$, where for any $k\in\bbN\land k\leq n$, $v_k$ is connected to $v_-$ through $R_k$, $v_+$ is connected to ground, output voltage $v_{out}$ is connected to $v_o$, and $v_o$ is fed back to $v_-$ through $R_f$.
\[i_k=\frac{v_k}{R_k},\]
\[\sum_{k=1}^ni_k=i_f=\frac{-v_{out}}{R_f},\]
\[v_{out}=-R_f\sum_{k=1}^n\frac{v_k}{R_k}.\]
When $R_k=R$ for all $k\in\bbN\land k\leq n$:
\[v_{out}=-\frac{R_f}{R}\sum_{k=1}^nv_k.\]
\sssc{Differential amplifier}
A differential amplifier consists of a minuend $v_1$, a subtrahend $v_2$, and four resistors $R_1,R_2,R_3,R_4$ satisfying the condition
\[\frac{R_1}{R_2}=\frac{R_3}{R_4},\]
where $v_1,v_2$ are connected to $v_-,v_+$ through $R_1,R_2$ respectively, output voltage $v_{out}$ is connected to $v_o$, $v_o$ is fed back to $v_-$ through $R_3$, and $v_+$ is connected to ground through $R_4$.
\[v_+=v_2\frac{R_4}{R_2+R_4},\]
\[\frac{v_1-v_-}{R_1}=\frac{v_--v_{out}}{R_3},\]
\[\frac{v_1}{R_2}-v_2\frac{R_4}{R_2+R_4}\frac{R_2+R_4}{R_2R_4}+\frac{v_{out}}{R_4}=0,\]
\[v_{out}=\frac{R_4}{R_2}\qty(v_2-v_1).\]
When $R_1=R_2=R_3=R_4$:
\[v_{out}=(v_2-v_1).\]







\subsubsection{Switches}
\begin{center}
\begin{circuitikz}
\draw (0,0) to[switch, l=Switch] (3,0);
\draw (5,0) node[spdt]{} (5,0.5) node[above]{SPDT Switch};
\end{circuitikz}
\end{center}
\subsubsection{Diodes}
\begin{center}
\begin{circuitikz}
\draw (0,0) to[D, l=Diode] (3,0);
\draw (5,0) to[zDo, l=Zener Diode] (8,0);
\draw (10,0) to[pDo, l=Photodiode] (13,0);
\draw (15,0) to[led, l=LED] (18,0);
\end{circuitikz}
\end{center}
\subsubsection{Transistors}
\begin{center}
\begin{circuitikz}
\draw (0,0) node[npn]{} (0,1) node[above]{NPN Transistor};
\draw (5,0) node[pnp]{} (5,1) node[above]{PNP Transistor};
\draw (10,0) node[nmos]{} (10,1) node[above]{N-MOSFET};
\draw (15,0) node[pmos]{} (15,1) node[above]{P-MOSFET};
\end{circuitikz}
\end{center}
\subsubsection{Measuring Instruments}
\begin{center}
\begin{circuitikz}
\draw (0,0) to[voltmeter, l=Voltmeter] (3,0);
\draw (5,0) to[ammeter, l=Ammeter] (8,0);
\draw (10,0) to[ohmmeter, l=Ohmmeter] (13,0);
\end{circuitikz}
\end{center}
\subsubsection{Transformers}
\begin{center}
\begin{circuitikz}
\draw (0,0) node[transformer]{Transformer} (3,0);
\draw (5,0) node[transformer core]{Transformer Core} (8,0);
\end{circuitikz}
\end{center}








\ssc{RC 與 RL 濾波電路(filter circuits)}
PLACEHOLDER
\subsection{RLC 與 LC 諧振電路(Resonant circuits)}
\subsubsection{常電壓串聯 RLC 與 LC 電路}
常電壓串聯 RLC 電路:常電壓$V$直流電源、電阻$R$、電感$L$、電容$C$串聯。

常電壓串聯 LC 電路:常電壓$V$直流電源、電感$L$、電容$C$串聯,即令$R=0$。
\[RI(t)+L\dv{I(t)}{t}+\frac{1}{C}\int _{-\infty}^tI(\tau)\,\mathrm{d}\tau=V\]
\[\dv[2]{I(t)}{t}+\frac{R}{L}\dv{I(t)}{t}+\frac{1}{LC}I(t)=0\]
定義奈培頻率(neper frequency)/衰減量(attenuation)$\alpha=\frac{R}{2L}$與角共振頻率(angular resonance frequency)$\omega_0=\frac{1}{\sqrt{LC}}$。
\[\dv[2]{I(t)}{t}+2\alpha\dv{I(t)}{t}+\omega_0^{\phantom{0}2}I(t)=0\]
定義阻尼係數(damping factor)$\zeta$:
\[\zeta=\frac{\alpha}{\omega_0}=\frac{R}{2}\sqrt{\frac{C}{L}}\]
特徵根:
\[s_1=-\omega_0\zeta+\omega_0\sqrt{\zeta^2-1}\]
\[s_2=-\omega_0\zeta-\omega_0\sqrt{\zeta^2-1}\]
過阻尼響應(Overdamped response)($\zeta>1$)的$I(t)$通解為:
\[I(t)=A_1e^{-\omega_0\left(\zeta+\sqrt{\zeta^2-1}\right)t}+A_2e^{-\omega_0\left(\zeta-\sqrt{\zeta^2-1}\right)t}\]
臨界阻尼響應(Critically damped response)($\zeta=1$)的$I(t)$通解為:
\[I(t)=A_1e^{-\omega_0\zeta t}+A_2e^{-\omega_0\zeta t}\]
欠阻尼響應(Underdamped response)($0<\zeta<1$)的$I(t)$通解為:
\[I(t)=A_1e^{-\omega_0\zeta t}\cos\left(\omega_0\sqrt{1-\zeta^2}t\right)+A_2e^{-\omega_0\zeta t}\sin\left(\omega_0\sqrt{1-\zeta^2}t\right)\]
無阻尼響應(Undamped response)($\zeta=0$即$R=0$)的$I(t)$通解為:
\[I(t)=A_1\cos\left(\omega_0t\right)+A_2\sin\left(\omega_0t\right)\]
\subsubsection{常電流並聯 RLC 與 LC 電路}
常電流並聯 RLC 電路:常電流$I$直流電源、電阻$R$、電感$L$、電容$C$並聯。

常電流並聯 LC 電路:常電流$I$直流電源、電感$L$、電容$C$並聯,即令 $\frac{1}{R}=0$。
\[\frac{1}{R}V(t)+C\dv{V(t)}{t}+\frac{1}{L}\int _{-\infty}^tV(\tau)\,\mathrm{d}\tau=I\]
\[\dv[2]{V(t)}{t}+\frac{1}{RC}\dv{V(t)}{t}+\frac{1}{LC}V(t)=0\]
定義奈培頻率(neper frequency)/衰減量(attenuation)$\alpha=\frac{1}{2RC}$與角共振頻率(angular resonance frequency)$\omega_0=\frac{1}{\sqrt{LC}}$。
\[\dv[2]{V(t)}{t}+2\alpha\dv{V(t)}{t}+\omega_0^{\phantom{0}2}V(t)=0\]
定義阻尼係數(damping factor)$\zeta$:
\[\zeta=\frac{\alpha}{\omega_0}=\frac{1}{2R}\sqrt{\frac{L}{C}}\]
特徵根:
\[s_1=-\omega_0\zeta+\omega_0\sqrt{\zeta^2-1}\]
\[s_2=-\omega_0\zeta-\omega_0\sqrt{\zeta^2-1}\]
過阻尼響應(Overdamped response)($\zeta>1$)的$V(t)$通解為:
\[V(t)=A_1e^{-\omega_0\left(\zeta+\sqrt{\zeta^2-1}\right)t}+A_2e^{-\omega_0\left(\zeta-\sqrt{\zeta^2-1}\right)t}\]
臨界阻尼響應(Critically damped response)($\zeta=1$)的$V(t)$通解為:
\[V(t)=A_1e^{-\omega_0\zeta t}+A_2e^{-\omega_0\zeta t}\]
欠阻尼響應(Underdamped response)($0<\zeta<1$)的$V(t)$通解為:
\[V(t)=A_1e^{-\omega_0\zeta t}\cos\left(\omega_0\sqrt{1-\zeta^2}t\right)+A_2e^{-\omega_0\zeta t}\sin\left(\omega_0\sqrt{1-\zeta^2}t\right)\]
無阻尼響應(Undamped response)($\zeta=0$即$\frac{1}{R}=0$)的$V(t)$通解為:
\[V(t)=A_1\cos\left(\omega_0t\right)+A_2\sin\left(\omega_0t\right)\]



