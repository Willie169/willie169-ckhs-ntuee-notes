\documentclass[a4paper,12pt]{article}
\setcounter{secnumdepth}{5}
\setcounter{tocdepth}{3}
\input{/usr/share/LaTeX-ToolKit/template.tex}
\begin{document}
\title{Electrical Circuit}
\author{沈威宇}
\date{\temtoday}
\titletocdoc
\sct{Electrical Circuit}
\ssc{Introduction}
\sssc{Convention}
\bit
\item Time $t$ (s)
\item Complex frequency (Laplace domain) $s$ (Hz = 1 / s)
\item Angular frequency $\omega$ (Hz = 1 / s)
\item Voltage $v$ (time domain) $V$ (phasor or Laplace domain) (volt = V)
\item Current $i$ (time domain) $I$ (phasor or Laplace domain) (ampere = A)
\item Power $p$ (watt = W = A V)
\item Imaginary unit (since $i$ is used by current) $j=\sqrt{-1}$
\item Resistance $R$ (ohm = \textOmega = V / A)
\item Capacitance $C$ (farad = F = A s / V)
\item Inductance $L$ (henry = H = V s / A)
\item Reactance $X$ (\textOmega)
\item Impedance $Z$ (\textOmega)
\item Admittance $Y$ (siemens = S = 1 / \textOmega)
\item Conductance $G$ (S)
\item Susceptance $B$ (S)
\item Time constant $\tau$ (s)
\item Dirac delta function $\delta(t)$
\item Unit step function $u(t)$
\item Impulse response function $h(t)$
\item Transfer function $H(s)$
\eit
\sssc{Electrical network}
An electrical network or simply network is an interconnection of electrical components or a model of such an interconnection, consisting of electrical elements.
\sssc{Electrical element}
Electrical elements or simply elements are conceptual abstractions representing idealized electrical components, such as resistors, capacitors, and inductors, used in the analysis of electrical networks and denoted with circuit symbols, also called electronic symbol, in circuit diagrams.
\sssc{Direct current (DC)}
A direct current is a current that is one-directional and with constant voltage and current over time.
\sssc{Alternating current (AC)}
An alternating current is a current whose voltage and current change over time and are of zero Cesàro mean over $[0,\infty)$.
\sssc{Steady-state sinusoidal alternating current and phasor domain}
A steady-state sinusoidal alternating current is a current whose voltage $v(t)$ and current $i(t)$ are of the form:
\[v(t)=v_m\cos(\omeg t+\varphi),\]
\[i(t)=i_m\cos(\omega t+\theta),\]
where $\omega$ is the angular frequency of them.

They can be represented in phasor as
\[V=v_me^{j\varphi},\]
\[I=i_me^{j\theta}.\]
\sssc{General current and Laplace domain}
Given a general current with voltage $v(t)$ and current $i(t)$ over time since $t=0$, we can perform Laplace transforms,
\[V(s)=\mathcal{L}\{v(t)\}(s)=\int_0^{\infty}e^{-st}v(t)\dd{t},\]
\[I(s)=\mathcal{L}\{i(t)\}(s)=\int_0^{\infty}e^{-st}i(t)\dd{t},\]
to obtain its voltage $V(s)$ and current $I(s)$ as a function of Laplace domain $s$.
\sssc{Resistance and resistor}
The resistance $R$ of an element is the voltage across it over the current through it under DC:
\[v(t)=Ri(t).\]
\[V(s)=RI(s).\]
An ideal resistor is a circuit element with a resistance and no capacitance or inductance.
\begin{center}
\begin{circuitikz}
\draw (0,0) to[R, l=Resistor] (3,0);
\draw (5,0) to[vR, l=Variable Resistor] (8,0);
\draw (10,0) to[thermistor, l=Thermistor] (13,0);
\end{circuitikz}
\begin{circuitikz}
\draw (0,0) to[ldR, l=Light Dependent Resistor] (3,0);
\draw (5,0) to[varistor, l=Varistor] (8,0);
\end{circuitikz}
\end{center}
Power absorbed:
\[p(t)=\frac{\qty(v(t))^2}{R}=\qty(i(t))^2R.\]
\sssc{Capacitance and capacitor}
An ideal capacitor is a circuit element with a capacitance and no resistance or inductance.
\begin{center}
\begin{circuitikz}
\draw (0,0) to[C, l=Capacitor] (3,0);
\end{circuitikz}
\end{center}
The capacitance $C$ of an element is the current through it divided by the derivative of the voltage across it with respect to time, in which voltage must not be allowed to jump instantaneously from one value to another.

Time domain:
\[i(t)=C\dv{v(t)}{t}.\]
Phasor domain (steady-state sinusoidal AC with angular frequency $\omega$):
\[I=Cj\omega V.\]
Laplace domain:
\[I(s)=CsV(s)-Cv(0).\]
\[V(s)=\frac{I(s)}{Cs}+\frac{v(0)}{s}.\]
A capacitor with capacitance $C$ is equivalent to a capacitance $C$ without initial energy in parallel connection with an independent current source enforcing current $Cv(0)\delta(t)$ (time domain), i.e., $Cv(0)$ (Laplace domain), or a capacitance $C$ without initial energy in series connection with an independent voltage source enforcing voltage $-v(0)u(t)$ (time domain), i.e., $-\frac{v(0)}{s}$ (Laplace domain).

Integral voltage-current relationship:
\[v(t)=\frac{1}{C}\int_{t_0}^ti(\tau)\dd{\tau}+v(t_0^{\pht{0}-})=\frac{1}{C}\int_{-\infty}^ti(\tau)\dd{\tau}.\]
Power absorbed:
\[p(t)=i(t)v(t)=Cv(t)\dv{v(t)}{t}.\]
Energy stored:
\[\int_{-\infty}^tp(t)\dd{t}=\frac{1}{2}C\qty(v(t))^2.\]
\sssc{Inductance and inductor}
An ideal inductor is a circuit element with an inductance and no resistance or capacitance.
\begin{center}
\begin{circuitikz}
\draw (0,0) to[L, l=Inductor] (3,0);
\end{circuitikz}
\end{center}
The inductance $L$ of an element is the voltage across it divided by the derivative of the current through it with respect to time, in which the current must not be allowed to jump instantaneously from one value to another.

Time domain:
\[v(t)=L\dv{i(t)}{t}.\]
Phasor domain (steady-state sinusoidal AC with angular frequency $\omega$):
\[V=Lj\omega I.\]
Laplace domain:
\[V(s)=LsI(s)-Li(0).\]
\[I(s)=\frac{V(s)}{Ls}+\frac{i(0)}{s}.\]
An inductor with inductance $L$ is equivalent to a inductance $L$ without initial energy in series connection with an independent voltage source enforcing voltage $Li(0)\delta(t)$ (time domain), i.e., $Li(0)$ (Laplace domain), or a inductance $L$ without initial energy in parallel connection with an independent current source enforcing current $-i(0)u(t)$ (time domain), i.e., $-\frac{i(0)}{s}$ (Laplace domain).

Integral voltage-current relationship:
\[i(t)=\frac{1}{L}\int_{t_0}^tv(\tau)\dd{\tau}+i(t_0^{\pht{0}-})=\frac{1}{L}\int_{-\infty}^tv(\tau)\dd{\tau}.\]
Power absorbed:
\[p(t)=i(t)v(t)=Li(t)\dv{i(t)}{t}.\]
Energy stored:
\[\int_{-\infty}^tp(t)\dd{t}=\frac{1}{2}L\qty(i(t))^2.\]
\sssc{Reactance (phasor domain)}
The (capacitive) reactance $X_C$ of a one-port network with capacitance $C$ under steady-state sinusoidal AC of angular frequency $\omega$ is defined as:
\[X_C=-\frac{1}{C\omega}.\]
The (inductive) reactance $X_L$ of a one-port network with inductance $L$ under steady-state sinusoidal AC of angular frequency $\omega$ is defined as:
\[X_L=L\omega.\]
The reactance $X$ of a one-port network under steady-state sinusoidal AC of angular frequency $\omega$ is defined as:
\[X=X_L+X_C.\]
Another choice is to define $X_C$ as $\frac{1}{C\omega}$ and $X=X_L-X_C$.
\sssc{Impedance (phasor domain)}
The impedance $Z$ of a one-port network under steady-state sinusoidal AC of angular frequency $\omega$ is defined as:
\[V=ZI,\]
and is such that
\[Z=R+jX.\]
\sssc{Impedance (Laplace domain)}
The impedance $Z(s)$ of a one-port network is defined as:
\[V(s)=Z(s)I(s),\]
and is such that for a one-port network with capacitance $C$:
\[Z(s)=\frac{1}{Cs};\]
and for a one-port network with inductance $L$:
\[Z(s)=Ls.\]
The impedance in phasor domain can be obtained by assigning $s=j\omega$.
\sssc{Admittance (phasor domain)}
The admittance $Y$ of a one-port network is defined as the reciprocal of its impedance $Z$.
\sssc{Admittance (Laplace domain)}
The admittance $Y(s)$ of a one-port network is defined as the reciprocal of its impedance $Z(s)$.
\sssc{Conductance (phasor domain)}
The conductance $G$ of a circuit element is defined as the real part of its admittance.
\sssc{Susceptance (phasor domain)}
The susceptance $B$ of a circuit element is defined as the imaginary part of its admittance.
\sssc{Active element}
An active element is an element capable of furnishing a power with Cesàro mean over $[0,\infty)$ greater than zero to some external network.
\sssc{Passive element}
A passive element is an element that is not an active element.
\sssc{Lumped-element model, lumped-parameter model, or lumped-component model}
A simplified and idealized representation of an electrical network that assumes all components are concentrated at a single point and their behavior can be described by idealized mathematical models, that is, the lumped-matter discipline:
\bit
\item The change of the magnetic flux in time through the surface surrounded by a loop is zero.
\item The change of the charge in time inside conductor is zero.
\item Signal timescales of interest are much larger than propagation delay of electromagnetic waves across the lumped element.
\eit
such that the system is a dynamic system whose state space is a finite dimension, called lumped-parameter system.

In contrast, distributed parameter systems have infinite-dimensional state space.

We will concentrate on lumped-parameter networks in this text.
\sssc{Time-invariant network}
Elements' properties and connections don't change over time. We will concentrate on time-invariant networks in this text.
\sssc{Circuit element}
A circuit element is a two terminal electrical network that can be completely characterized by its voltage-current relationship.

We label + and - on the two terminal repsectively and voltage $v$ between them to define the voltage $v$ of the element as the electric potential difference of + terminal relative to - terminal.

We label arrow and current $i$ on it to define the reference direction of the current $i$ of the element.

If the current arrow enters the + terminal, the element is said to follow the passive sign convention. Under this convention, the power absorbed by the element is
\[p(t)=v(t)i(t),\]
with negative power means delivering.
\sssc{Node}
A node is a point in a network where two or more elements are connected.
\sssc{Pole or terminal}
A pole or a terminal is a node that is available for connection to an external network.
\sssc{Path}
A path is any route through which current can flow from one node to another.
\sssc{Branch}
A branch is a single element or a group of elements in series that connects two nodes.
\sssc{Loop}
A loop is any closed path in a network where you can start at a node, traverse branches, and return to the starting node without retracing any branch.
\sssc{(Essential) mesh}
An (essential) mesh is a loop $\ell$ such that there is no other loop whose surface is a subset of the surface surrounded by $\ell$.
\sssc{Electrical circuit or closed circuit}
An electrical circuit is a network containing loops.
\sssc{Open circuit}
An open circuit is a path with infinite (or very high) impedance (thus not actually a path).
\sssc{Short circuit}
A short circuit is a path with zero (or very low) impedance.
\sssc{Port}
A port is a pair of terminals through which current can enter or leave a network and satisfies the port condition: the current flowing into one pole from outside is equal to the current flowing out of the other pole to outside.

An $n$-port network is a network with $2n$ poles forming $n$ ports.
\sssc{Wire or lead}
Circuit elements in a network are connected with wires (aka leads), which are assumed to have zero impedance.
\begin{center}
\begin{circuitikz}
\draw (0,0) to[short, l=Wire] (3,0);
\draw (5,0) to[crossing, l=Wire Crossing] (8,0);
\end{circuitikz}
\end{center}
Adding a wire directly between two nodes is called adding a short circuit.
\sssc{Kirchhoff's current law (KCL), Kirchhoff's first law, or Kirchhoff's junction rule}
The algebraic sum of the currents entering any node is zero.
\sssc{Kirchhoff's voltage law (KVL), Kirchhoff's second law, or Kirchhoff's loop rule}
The algebraic sum of the voltages around any closed path is zero.
\sssc{Network as graph}
A network is a graph with nodes being vertices and branches without nodes inside except ends being edges.
\sssc{Planar network}
A network that is planar as a graph.
\sssc{Series connection}
Two or more elements are connected in series if they are connected along a single path, and each component has the same current through it, equal to the current through the network. The voltage across the network is equal to the sum of the voltages across each element.
\sssc{Parallel connection}
Two or more elements are connected in parallel if they are connected along multiple paths from one node to another, and each component has the same voltage across it, equal to the voltage through the network. The current through the network is equal to the sum of the current through each element.
\sssc{Bilateral network}
A bilateral electrical network is an electrical network whose behaviors remain the same regardless of the direction of current flow or applied voltage.
\sssc{Ground}
A ground is a node designated to have 0 electric potential. All other electric potential in the network are measured relative to the grounds.
\bit
\item \tb{Earth ground}: Earth ground is a point in the network that is physically connected to the Earth.
\item \tb{Signal ground}: Signal ground is a ground in the network provided by an external signal, which can be float relative to Earth and such circuit is called float. An example of signal ground is a GND pin on a microcontroller board.
\item \tb{Chassis ground}: Chassis ground is a point in the circuit that is physically connected to a devices's metal frame. The metal frame acts as a shield against electromagnetic interference and can be connected to Earth for safety in AC-powered devices.
\eit
\begin{center}
\begin{circuitikz}
\draw (0,0) node[ground]{} (0,0) node[above]{Earth Ground};
\draw (5,0) node[sground]{} (5,0) node[above]{Signal Ground};
\draw (10,0) node[cground]{} (10,0) node[above]{Chassis Ground};
\end{circuitikz}
\end{center}
\sssc{SPICE}
SPICE (Simulation Program with Integrated Circuit Emphasis) is a general-purpose, open-source analog electronic circuit simulator. It is a program used in integrated circuit and board-level design to check the integrity of circuit designs and to predict circuit behavior.
\ssc{Souce}
\sssc{Ideal sources}
Ideal voltage sources are active elements with zero output resistance. To turn off an ideal voltage source means replace it with short circuit.

Ideal current sources are active elements with infinite output resistance. To turn off an ideal current source means replace it with open circuit.
\sssc{Independent source}
\bit
\item An independent voltage source enforces a prescribed DC or sinusoidal AC voltage $v_s$ across its terminals.
\item An independent current source enforces a prescribed DC or sinusoidal AC current $i_s$ through its terminals.
\eit
\begin{center}
\begin{circuitikz}
\draw (0,0) node[vsourcesinshape]{} (0,1) node[above]{Source};
\draw (5,0) node[vsourceAMshape]{} (5,1) node[above]{Independent Voltage Source};
\draw (10,0) node[isourceAMshape]{} (10,1) node[above]{Independent Current Source};
\end{circuitikz}
\end{center}
\sssc{Linear dependent or controlled source}
\bit
\item A voltage-controlled voltage source enforces a DC or sinusoidal AC voltage $v_s=\mu v_x$ across its terminals, where $v_x$ is control voltage and real for DC or complex for AC constant $\mu$ is dimensionless voltage gain.
\item A current-controlled voltage source enforces a DC or sinusoidal AC voltage $v_s=ri_x$ across its terminals, where $i_x$ is control current and real for DC or complex for AC constant $r$ is transresistance.
\item A voltage-controlled current source enforces a DC or sinusoidal AC current $i_s=gv_x$ through its terminals, where $v_x$ is control voltage and real for DC or complex for AC constant $g$ is transconductance.
\item A current-controlled current source enforces a DC or sinusoidal AC current $i_s=\beta i_x$ through its terminals, where $i_x$ is control current and real for DC or complex for AC constant $\beta$ is dimensionless current gain.
\eit
\begin{center}
\begin{circuitikz}
\draw (0,0) node[cvsourceAMshape]{} (0,1) node[above]{Controlled Voltage Source};
\draw (5,0) node[cisourceAMshape]{} (5,1) node[above]{Controlled Current Source};
\end{circuitikz}
\end{center}
\sssc{Practical souce}
A practical independent voltage source can be approximated with an ideal independent voltage source in series connection with a resistor.

A practical independent current source can be approximated with an ideal independent current source in parallel connection with a resistor.
\ssc{Linear circuit}
\sssc{Linear element}
A linear element is a passive element whose resistance $R$, capacitance $C$, and inductance $L$ are independent of voltage and current, that is, equivalent to a one-port network consists of ideal resistors, capacitors, and inductors, that is, the voltage-current relationship is a complex equation $v=Z\cdot i$ ($v=R\cdot i$ for DC) with voltage $v$, current $i$, and impedance $Z$ (resistance $R$ for DC). The assumption for resistance is the Ohm's law.
\sssc{Linear network}
A linear network is a network which consists of only linear elements, independent sources, and linear dependent sources.
\sssc{Linear circuit}
A linear circuit is a circuit which obeys the superposition principle of voltages and currents (but not power) that the voltages across or currents through all one-port subnetwork, called responses or response functions, are linear combinations of the input voltages from independent voltage sources and input currents from independent current sources in it, called forcing functions, that is, consists of only linear elements, independent sources, and linear dependent sources.

All circuits discussed in this text are causal continuous-time linear time-invariant (LTI) systems.
\sssc{Types of responses}
\bit
\item\tb{Natural, zero-input, homogeneous, or unforced response}: Response when all independent sources are turned off.
\item\tb{Forced or zero-state response}: Response when there are no initial energy.
\item\tb{(Total) response}: The response considering both independent sources and initial energies, which is the sum of natural response and forced response.
\item\tb{Transient response}: A response that converges to a constant.
\eit
\sssc{Series connection}
For linear networks, the impedance of the network is equal to the sum of the impedance of each element.
\sssc{Parallel connection}
For linear networks, the admittance of the network is equal to the sum of the admittance of each element.
\sssc{Linear circuit superposition analysis}
\ben
\item Select one of the independent sources. Set all other independent sources to zero. This means voltage sources are replaced with short circuits and current sources are replaced with open circuits. Leave dependent sources in the circuit.
\item Analyze the simplified circuit to find the desired currents and voltages.
\item Repeat previous steps until each independent source has been considered.
\item Sum the currents and voltages obtained from the separate analyses above.
\een
\sssc{Linear circuit as graph}
A linear circuit is a graph with nodes with all voltage sources enclosed as supernodes being vertices, branches without nodes inside except ends being edges, and admittances (or conductance for DC) being weights.
\sssc{Dual}
Two linear circuit are dual if their governing equations in terms of variables become identical after systematically exchanging voltage and current, capacitance and inductance, impedance and admittance, open circuit and short circuit, KCL and KVL, Thévenin's theorem and Norton's theorem, series and parallel, node and loop, etc.

Two linear circuit are exact dual if their governing equations in terms of numerical values become identical after systematically exchanging voltage and current, capacitance and inductance, impedance and admittance, open circuit and short circuit, KCL and KVL, Thévenin's theorem and Norton's theorem, series and parallel, node and loop, etc.
\ssc{Thévenin's theorem and Norton's theorem}
\sssc{Thévenin's theorem}
Any linear network with two terminals is equivalent to a voltage source with a voltage called Thévenin voltage in a series connection with a linear element (or resistor for DC) with an impedance (or resistance for DC) called Thévenin impedance (or resistance for DC), such equivalent circuit is called Thévenin equivalent.
\sssc{Norton's theorem}
Any linear network with two terminals is equivalent to a current source with a current called Norton current in a parallel connection with a linear element (or resistor for DC) with an impedance (or resistance for DC) called Norton impedance (or resistance for DC), such equivalent circuit is called Norton equivalent.
\sssc{Finding Thévenin voltage}
Remove any load, and the open-circuit voltage is the Thévenin voltage.
\sssc{Finding Norton current}
Add a short circuit between the two terminals, and the current on the it is the Norton current.
\sssc{Finding Thévenin impedance or Norton impedance}
Thévenin impedance and Norton impedance are the same and can be found with:
\bit
\item \tb{Source deactivation method}: Replace all independent voltage sources with short circuits and independent current sources with open circuits, and the impedance between the two terminals is the Thévenin impedance or Norton impedance.
\item \tb{Division method}: The Thévenin impedance or Norton impedance is the Thévenin voltage divided by the Norton current.
\eit
\sssc{Maximum power theorem}
A linear network delivers maximum power to a load impedance when it is equal to the Thévenin impedance or its conjugate of the network.
\ssc{Nodal analysis or node-voltage analysis of linear circuits}
\sssc{(Nodal) admittance matrix}
The (nodal) admittance (conductance for DC) matrix of a linear circuit that is a simple graph with all sources omitted is the Laplacian matrix of it with all sources omitted.
\sssc{Supernode method}
\ben
\item Choose a reference node and assign a voltage for it (typically 0).
\item For each pair of nodes, if there exists an edge with zero impedance and no source, merge the two nodes, and discard all edges between them.
\item For each pair of nodes, use series and parallel connections to combine all edges without sources between them.
\item For each collection of voltage sources in series connection between two nodes, combine them to one.
\item For each collection of current sources in parallel connection between two nodes, combine them to one.
\item For each pair of node with a voltage source in series connection with an nonzero impedance (resistance for DC) connecting them, convert it to its Norton equivalent.
\item For each voltage source whose either endpoint is the reference node, the nodal voltage of the other endpoint is immediately known.
\item For each node with unknown voltage, assign a nodal voltage variable.
\item For each voltage source whose neither endpoint is the reference node, enclose both nodes and the source as a supernode.
\item Excluding all supernodes and all edges connected to supernode, form a system of linear equations given by KCL:
\[\mb{Y}\mb{v}=\mb{i},\]
where $\mb{v}$ is the column vector of the nodal voltage variables and known nodal voltages, $\mb{i}$ is the column vector of current injected into the nodes, where each entry $i_k$ is zero if node $k$ is not connected to a current source and $i_s$ if node $k$ is connected to a current source of current $i_s$ injected to the node (negative for absorption, VCCSs and CCCSs directly replaced with the control functions), and $\mb{Y}$ is the admittance (conductance for DC) matrix of the linear circuit.
\item For each supernode, write the equation given by KCL in terms of nodal voltage variables, that is, the sum of the currents leave the boundary of the supernode is zero.
\item For each voltage source, write the equation where the enforced voltage in one side and the difference of two nodal voltage variables in the other side.
\item Solve the system of linear equations.
\een
\sssc{MNA}
PLACEHOLDER
\ssc{Mesh analysis of planar linear circuit}
\ben
\item For each pair of nodes, if there exists an edge with zero impedance and no source, merge the two nodes, and discard all edges between them.
\item For each pair of nodes, use series and parallel connections to combine all edges without sources between them.
\item For each collection of voltage sources in series connection between two nodes, combine them to one.
\item For each collection of current sources in parallel connection between two nodes, combine them to one.
\item For each pair of node with a current source in parallel connection with an nonzero impedance (resistance for DC) connecting them, convert it to its Thévenin equivalent.
\item Indentify all meshes.
\item Assign an arbitrary but consistent direction (clockwise or counterclockwise) of current for all meshes, and assign a mesh current variable for each mesh.
\item For each current source, if it lies on two meshes, exclude the edge with the current source, and merge the two meshes into one, called a supermesh.
\item For each mesh (including supermesh), write the linear equation given by KVL with VCVSs and CCVSs directly replaced with the control functions.
\item For each current source, write the equation where the enforced current in one side and the result of additions and subtractions of mesh current variables in the other side.
\item Solve the system of linear equations.
\een
\ssc{Transformations}
\sssc{Source transformation}
An independent voltage source $v$ in series connection with an impedance $Z=R+jX$ is equivalent to an independent current source $i=\frac{v}{Z}$ in parallel connection with the same impedance $Z$.
\sssc{Y-Δ transformation and Δ-Y transformation}
\bit
\item Y (wye or star) network: a common central node $N$, three terminals $N_1,N_2,N_3$, and three circuit elements with impedances $Z_1,Z_2,Z_3$ connecting $N_1$ and $N$, $N_2$ and $N$, and $N_3$ and $N$ respectively.
\item Δ (delta or triangle) network: three terminals $N_1,N_2,N_3$, and three circuit elements with impedances $Z_{12},Z_{23},Z_{31}$ connecting $N_1$ and $N_2$, $N_2$ and $N_3$, and $N_3$ and $N_1$ respectively.
\item Y-Δ transformation: The two networks are equivalent if
\[Z_{12}=Z_1+Z_2+\frac{Z_1Z_2}{Z_3},\]
\[Z_{23}=Z_2+Z_3+\frac{Z_2Z_3}{Z_1},\]
\[Z_{31}=Z_3+Z_1+\frac{Z_3Z_1}{Z_2}.\]
\item Δ-Y transformation: The two networks are equivalent if
\[Z_1=\frac{Z_{12}Z_{31}}{Z_{12}+Z_{23}+Z_{31}},\]
\[Z_2=\frac{Z_{23}Z_{12}}{Z_{12}+Z_{23}+Z_{31}},\]
\[Z_3=\frac{Z_{31}Z_{23}}{Z_{12}+Z_{23}+Z_{31}}.\]
\eit
\sssc{Star-mesh transformation}
For all integer $n\geq 3$:
\bit
\item $n$-star network: a common central node $N$, $n$ terminals $N_1,N_2,\ldots,N_n$, and $n$ circuit elements with impedances $Z_i$ connecting $N_i$ and $N$ for all $i\in\bbN\land i\leq n$.
\item $n$-mesh (complete graph) network: $n$ terminals $N_1,N_2,\ldots,N_n$, and $\frac{n(n-1)}{2}$ circuit elements  with impedances $Z_{ij}$ connecting $N_i$ and $N_j$ for all $i,j\in\bbN\land i<j\leq n$.
\item Star-mesh transformation: The two networks are equivalent if
\[Z_{ij}=Z_i+Z_j+\frac{Z_iZ_j}{-Z_i-Z_j+\sum_{k=1}^nZ_k}.\]
\eit
\ssc{RL, RC, LC, and RLC Circuits}
\sssc{RL circuit}
A (series) RL circuit or filter consists of a resistor with resistance $R$, an inductor with inductance $L$, and an independent voltage source $v_{in}$ in series connection.

Natural response:
\[Ri+L\dv{i}{t}=0\]
\[i=i(0)e^{-\frac{R}{L}t}\]
\[v_R=-v_L=Ri(0)e^{-\frac{R}{L}t}\]
\[\tau_i=\tau_{v_R}=\tau_{v_L}=\frac{L}{R}\]
\[p_R=-p_L=R(i(0))^2e^{-\frac{2R}{L}t}\]
\[\int_0^{\infty}p_R\dd{t}=-\int_0^{\infty}p_L\dd{t}=\frac{1}{2}L(i(0))^2\]
\[I=\frac{Li(0)}{R+Ls}\]
\[V_R=-V_L=\frac{RLi(0)}{R+Ls}\]
\begin{proof}
\[Ri+L\dv{i}{t}=0\]
\[Lr+R=0\]
\[r=-\frac{R}{L}\]
\[i=Ae^{-\frac{R}{L}t}\]
\[i=i(0)e^{-\frac{R}{L}t}\]
\[v_R=-v_L=Ri(0)e^{-\frac{R}{L}t}\]
\[p_R=-p_L=R(i(0))^2e^{-\frac{2R}{L}t}\]
\[\ba
\int_0^{\infty}p_R\dd{t}&=R(i(0))^2\evlv(-\frac{L}{2R}e^{-\frac{2R}{L}t})_0^{\infty}\\
&=\frac{1}{2}L(i(0))^2
\ea\]
\[I=\frac{Li(0)}{R+Ls}\]
\[V_R=-V_L=\frac{RLi(0)}{R+Ls}\]
Check consistency:
\[\mathcal{L}\{i\}=\frac{i(0)}{s+\frac{R}{L}}=\frac{Li(0)}{R+Ls}\]
\[\mathcal{L}\{v_R\}=Ri(0)\frac{1}{s+\frac{R}{L}}=\frac{RLi(0)}{R+Ls}\]
\end{proof}
Forced response:
\[Ri+L\dv{i}{t}=v_{in}\]
\[i=\frac{1}{L}\int_0^te^{\frac{R}{L}\tau}v_{in}(\tau)\dd{\tau}e^{-\frac{R}{L}t}\]
\[v_R=v_{in}-v_L=\frac{R}{L}\int_0^te^{\frac{R}{L}\tau}v_{in}(\tau)\dd{\tau}e^{-\frac{R}{L}t}\]
\[I=\frac{V_{in}}{R+Ls}\]
\[V_R=\frac{RV_{in}}{R+Ls}\]
\[V_L=\frac{LsV_{in}}{R+Ls}\]
\begin{proof}
\[Ri+L\dv{i}{t}=v_{in}\]
\[\dv{i}{t}+\frac{R}{L}i=\frac{v_{in}}{L}\]
\[\mu(t)=e^{\int\frac{R}{L}\dd{t}}=e^{\frac{R}{L}t}\]
\[\dv{}{t}\qty(e^{\frac{R}{L}t}i)=e^{\frac{R}{L}t}\frac{v_{in}}{L}\]
\[i=\frac{1}{L}\int_0^te^{\frac{R}{L}\tau}v_{in}(\tau)\dd{\tau}e^{-\frac{R}{L}t}\]
\[I=\frac{V_{in}}{R+Ls}\]
Check consistency:
\[\ba
\mathcal{L}\{i\}&=\frac{1}{L}\mathcal{L}\{\int_0^te^{\frac{R}{L}\tau}v_{in}(\tau)u(t-\tau)\dd{\tau}\}\qty(s+\frac{R}{L})\\
&=\frac{1}{L}\mathcal{L}\{e^{\frac{R}{L}t}v_{in}(t)\}\qty(s+\frac{R}{L})\mathcal{L}\{u(t)\}\qty(s+\frac{R}{L})\\
&=\frac{1}{L}V_{in}\qty(s-\frac{R}{L}+\frac{R}{L})\frac{1}{s+\frac{R}{L}}\\
&=\frac{V_{in}}{R+Ls}
\ea\]
\end{proof}
Total response:
\[i=\qty(i(0)+\frac{1}{L}\int_0^te^{\frac{R}{L}\tau}v_{in}(\tau)\dd{\tau})e^{-\frac{R}{L}t}\]
\[v_R=v_{in}-v_L=R\qty(i(0)+\frac{1}{L}\int_0^te^{\frac{R}{L}\tau}v_{in}(\tau)\dd{\tau})e^{-\frac{R}{L}t}\]
\[I=\frac{Li(0)+V_{in}}{R+Ls}\]
\[V_R=\frac{RLi(0)+RV_{in}}{R+Ls}\]
\[V_L=\frac{LsV_{in}-RLi(0)}{R+Ls}\]
Impulse response and transfer function:
\[h_i=\frac{1}{L}u(t)e^{-\frac{R}{L}t}\]
\[h_{v_R}=\frac{R}{L}u(t)e^{-\frac{R}{L}t}\]
\[h_{v_L}=\delta(t)-\frac{R}{L}u(t)e^{-\frac{R}{L}t}\]
\[H_I=\frac{1}{R+Ls}\]
\[H_{V_R}=\frac{R}{R+Ls}\]
\[H_{V_L}=\frac{Ls}{R+Ls}\]
\begin{proof}
\[h_i=\frac{1}{L}\int_0^te^{\frac{R}{L}\tau}\delta(\tau)\dd{\tau}e^{-\frac{R}{L}t}=\frac{1}{L}u(t)e^{-\frac{R}{L}t}\]
Check consistency:
\[\mathca{L}\{h_i\}=\frac{1}{L\qty(s+\frac{R}{L})}=\frac{1}{R+Ls}\]
\[\mathca{L}\{h_{V_L}\}=1-\frac{R}{R+Ls}=\frac{Ls}{R+Ls}\]
\end{proof}
$V_R$ being low-pass filter:
\[\lim_{\omega\to 0}\abs{H_{V_R}(j\omega)}=1\]
\[\lim_{\omega\to\infty}\abs{H_{V_R}(j\omega)}=0\]
\begin{proof}
\[\lim_{\omega\to 0}\abs{H_{V_R}(j\omega)}=\lim_{\omega\to 0}\abs{\frac{R}{R+Lj\omega}}=1\]
\[\lim_{\omega\to\infty}\abs{H_{V_R}(j\omega)}=\lim_{\omega\to\infty}\abs{\frac{R}{R+Lj\omega}}=0\]
\end{proof}
$V_L$ being high-pass filter:
\[\lim_{\omega\to 0}\abs{H_{V_L}(j\omega)}=0\]
\[\lim_{\omega\to\infty}\abs{H_{V_L}(j\omega)}=1\]
\begin{proof}
\[\lim_{\omega\to 0}\abs{H_{V_L}(j\omega)}=\lim_{\omega\to 0}\abs{\frac{Lj\omega}{R+Lj\omega}}=0\]
\[\lim_{\omega\to\infty}\abs{H_{V_L}(j\omega)}=\lim_{\omega\to\infty}\abs{\frac{Lj\omega}{R+Lj\omega}}=1\]
\end{proof}
\sssc{RC circuit}
A (series) RC circuit or filter consists of a resistor with resistance $R$, a capacitor with capacitance $C$, and an independent voltage source $v_{in}$ in series connection.

Natural response:
\[\frac{v_C}{R}+C\dv{v_C}{t}=0\]
\[v_C=-v_R=v_C(0)e^{-\frac{1}{RC}t}\]
\[i=-\frac{1}{R}v_C(0)e^{-\frac{1}{RC}t}\]
\[\tau_{v_C}=\tau_{v_R}=\tau_i=\frac{1}{RC}\]
\[p_R=-p_C=\frac{1}{R}(v_C(0))^2e^{-\frac{2}{RC}t}\]
\[\int_0^{\infty}p_R\dd{t}=-\int_0^{\infty}p_C\dd{t}=\frac{1}{2}C(v_C(0))^2\]
\[V_C=-V_R=\frac{RCv_C(0)}{1+RCs}\]
\[I=-\frac{Cv_C(0)}{1+RCs}\]
\begin{proof}
\[v_R+v_C=0\]
\[\frac{v_R}{R}-C\dv{v_C}{t}=0\]
\[\frac{v_C}{R}+C\dv{v_C}{t}=0\]
\[Cr+\frac{1}{R}=0\]
\[r=-\frac{1}{RC}\]
\[v_C=Ae^{-\frac{1}{RC}t}\]
\[v_C=v_C(0)e^{-\frac{1}{RC}t}\]
\[i=-\frac{1}{R}v_C(0)e^{-\frac{1}{RC}t}\]
\[p_R=-p_C=\frac{1}{R}(v_C(0))^2e^{-\frac{2}{RC}t}\]
\[\ba
\int_0^{\infty}p_R\dd{t}&=\frac{1}{R}(v_C(0))^2\evlv{-\frac{RC}{2}e^{-\frac{2}{RC}t}}_0^{\infty}\\
&=\frac{1}{2}C(v_C(0))^2
\ea\]
\[I=-\frac{v_C(0)}{s\qty(\frac{1}{Cs}+R)}=-\frac{Cv_C(0)}{1+RCs}\]
\[V_C=\frac{I+Cv_C(0)}{Cs}=\frac{RCv_C(0)}{1+RCs}\]
Check consistency:
\[\mathcal{L}\{v_C\}=\frac{v_C(0)}{s+\frac{1}{RC}}=\frac{RCv_C(0)}{1+RCs}\]
\[\mathcal{L}\{i\}=-\frac{1}{R}v_C(0)\frac{1}{s+\frac{1}{RC}}=-\frac{Cv_C(0)}{1+RCs}\]
\end{proof}
Forced response:
\[\frac{v_C-v_{in}}{R}+C\dv{v_C}{t}=0\]
\[v_C=v_{in}-v_R=\frac{1}{RC}\int_0^te^{\frac{1}{RC}\tau}v_{in}(\tau)\dd{\tau}e^{-\frac{1}{RC}t}\]
\[i=\frac{1}{R}\qty(v_{in}-\frac{1}{RC}\int_0^te^{\frac{1}{RC}\tau}v_{in}(\tau)\dd{\tau}e^{-\frac{1}{RC}t})\]
\[V_C=\frac{V_{in}}{1+RCs}\]
\[V_R=\frac{RCsV_{in}}{1+RCs}\]
\[I=-\frac{CsV_{in}}{1+RCs}\]
\begin{proof}
\[v_R+v_C=v_{in}\]
\[\frac{v_R}{R}-C\dv{v_C}{t}=0\]
\[\frac{v_C-v_{in}}{R}+C\dv{v_C}{t}=0\]
\[\dv{v_C}{t}+\frac{v_C}{RC}=\frac{v_{in}}{RC}\]
\[\mu(t)=e^{\int\frac{1}{RC}\dd{t}}=e^{\frac{1}{RC}t}\]
\[\dv{}{t}\qty(e^{\frac{1}{RC}t}v_C)=e^{\frac{1}{RC}t}\frac{v_{in}}{RC}\]
\[v_C=\frac{1}{RC}\int_0^te^{\frac{1}{RC}\tau}v_{in}(\tau)\dd{\tau}e^{-\frac{1}{RC}t}\]
\[V_C=\frac{1}{\frac{1}{R}+Cs}CV_{in}=\frac{RCV_{in}}{1+RCs}\]
Check consistency:
\[\ba
\mathcal{L}\{v_C\}&=\frac{1}{RC}\mathcal{L}\{\int_0^te^{\frac{1}{RC}\tau}v_{in}(\tau)u(t-\tau)\dd{\tau}\}\qty(s+\frac{1}{RC})\\
&=\frac{1}{RC}\mathcal{L}\{e^{\frac{1}{RC}t}v_{in}(t)\}\qty(s+\frac{1}{RC})\mathcal{L}\{u(t)\}\qty(s+\frac{1}{RC})\\
&=\frac{1}{RC}V_{in}\qty(s-\frac{1}{RC}+\frac{1}{RC})\frac{1}{s+\frac{1}{RC}}\\
&=\frac{V_{in}}{1+RCs}
\ea\]
\end{proof}
Total response:
\[v_C=v_{in}-v_R=\qty(v_C(0)+\frac{1}{RC}\int_0^te^{\frac{1}{RC}\tau}v_{in}(\tau)\dd{\tau})e^{-\frac{1}{RC}t}\]
\[i=\frac{1}{R}\qty(v_{in}-\qty(v_C(0)+\frac{1}{RC}\int_0^te^{\frac{1}{RC}\tau}v_{in}(\tau)\dd{\tau})e^{-\frac{1}{RC}t})\]
\[V_C=\frac{RCv_C(0)+V_{in}}{1+RCs}\]
\[V_R=\frac{-RCv_C(0)+RCsV_{in}}{1+RCs}\]
\[I=-\frac{Cv_C(0)+CsV_{in}}{1+RCs}\]
Impulse response and transfer function:
\[h_{v_C}=\frac{1}{RC}u(t)e^{-\frac{1}{RC}t}\]
\[h_{v_R}=\delta(t)-\frac{1}{RC}u(t)e^{-\frac{1}{RC}t}\]
\[h_i=\frac{1}{R}\qty(\delta(t)-\frac{1}{RC}u(t)e^{-\frac{1}{RC}t})\]
\[H_{V_C}=\frac{1}{1+RCs}\]
\[H_{V_R}=\frac{RCs}{1+RCs}\]
\[H_I=-\frac{Cs}{1+RCs}\]
\begin{proof}
\[h_{v_C}=\frac{1}{RC}\int_0^te^{\frac{1}{RC}\tau}\delta(\tau)\dd{\tau}e^{-\frac{1}{RC}t}=\frac{1}{RC}u(t)e^{-\frac{1}{RC}t}\]
Check consistency:
\[\mathca{L}\{h_{v_C}\}=\frac{1}{RC(s+\frac{1}{RC})}=\frac{1}{1+RCs}\]
\[\mathca{L}\{h_i\}=\frac{1}{R}\qty(1-\frac{1}{1+RCs})=-\frac{Cs}{1+RCs}\]
\end{proof}
$V_C$ being low-pass filter:
\[\lim_{\omega\to 0}\abs{H_{V_C}(j\omega)}=1\]
\[\lim_{\omega\to\infty}\abs{H_{V_C}(j\omega)}=0\]
\begin{proof}
\[\lim_{\omega\to 0}\abs{H_{V_C}(j\omega)}=\lim_{\omega\to 0}\abs{\frac{1}{1+RCj\omega}}=1\]
\[\lim_{\omega\to\infty}\abs{H_{V_C}(j\omega)}=\lim_{\omega\to\infty}\abs{\frac{1}{1+RCj\omega}}=0\]
\end{proof}
$V_R$ being high-pass filter:
\[\lim_{\omega\to 0}\abs{H_{V_R}(j\omega)}=0\]
\[\lim_{\omega\to\infty}\abs{H_{V_R}(j\omega)}=1\]
\begin{proof}
\[\lim_{\omega\to 0}\abs{H_{V_R}(j\omega)}=\lim_{\omega\to 0}\abs{\frac{RCj\omega}{1+RCj\omega}}=0\]
\[\lim_{\omega\to\infty}\abs{H_{V_R}(j\omega)}=\lim_{\omega\to\infty}\abs{\frac{RCj\omega}{1+RCj\omega}}=1\]
\end{proof}
\sssc{Series LC circuit}
A series LC (resonant, oscillator, tank, or tuned) circuit consists of an inductor with inductance $L$, a capacitor with capacitance $C$, and an independent voltage source $v_{in}$ in series connection. Angular resonance frequency (角共振頻率) $\omega_0$.

Natural response: Total energy stored in inductor and capacitor $E$.
\[LC\dv[2]{i}{t}+i=0\]
\[\omega_0=\frac{1}{\sqrt{LC}}\]
\[\dv[2]{i}{t}+\omega_0^{\pht{0}2}i=0\]
\[\ba
i&=i(0)\cos\qty(\frac{1}{\sqrt{LC}}t)-v_C(0)\sqrt{\frac{C}{L}}\sin\qty(\frac{1}{\sqrt{LC}}t)\\
&=i(0)\cos\qty(\omega_0t)-v_C(0)C\omega_0\sin\qty(\omega_0t)
\ea\]
\[\ba
v_C&=-v_L\\
&=i(0)\sqrt{\frac{L}{C}}\sin\qty(\frac{1}{\sqrt{LC}}t)+v_C(0)\cos\qty(\frac{1}{\sqrt{LC}}t)\\
&=i(0)L\omega_0\sin\qty(\omega_0t)+v_C(0)\cos\qty(\omega_0t)
\ea\]
\[\ba
p_C&=-p_L\\
&=\frac{1}{2}\qty((i(0))^2\sqrt{\frac{L}{C}}-(v_C(0))^2\sqrt{\frac{C}{L}})\sin\qty(\frac{2}{\sqrt{LC}}t)+i(0)v_C(0)\cos\qty(\frac{2}{\sqrt{LC}}t)\\
&=\frac{1}{2}\qty((i(0))^2L\omega_0-(v_C(0))^2C\omega_0)\sin\qty(2\omega_0t)+i(0)v_C(0)\cos\qty(2\omega_0t)
\ea\]
\[\lim_{n\to\infty}\frac{1}{n}\int_0^np_C\dd{t}=-\lim_{n\to\infty}\frac{1}{n}\int_0^np_L\dd{t}=0\]
\[E=\frac{1}{2}\qty(C(v_C(0))^2+L(I(0))^2)\]
\[\ba
I&=\frac{LCsi(0)-Cv_C(0)}{LCs^2+1}\\
&=\frac{si(0)-C\omega_0^{\pht{0}2}v_C(0)}{s^2+\omega_0^{\pht{0}2}}
\ea\]
\[\ba
V_C&=-V_L\\
&=\frac{LCsv_C(0)+Li(0)}{LCs^2+1}\\
&=\frac{sv_C(0)+L\omega_0^{\pht{0}2}i(0)}{s^2+\omega_0^{\pht{0}2}}
\ea\]
\begin{proof}
\[v_C+v_L=0\]
\[i=C\dv{v_C}{t}\]
\[v_C=-v_L=-L\dv{i}{t}\]
\[LC\dv[2]{i}{t}+i=0\]
\[LCr^2+1=0\]
\[r=\pm\frac{1}{\sqrt{LC}}j\]
\[\omega_0=\frac{1}{\sqrt{LC}}\]
\[\dv[2]{i}{t}+\omega_0^{\pht{0}2}i=0\]
\[i=A\cos\qty(\frac{1}{\sqrt{LC}}t)+B\sin\qty(\frac{1}{\sqrt{LC}}t)\]
\[A=i(0)\]
\[L\dv{i}{t}(0)+v_C(0)=0\]
\[LB\frac{1}{\sqrt{LC}}+v_C(0)=0\]
\[B=-v_C(0)\sqrt{\frac{C}{L}}=-v_C(0)\frac{1}{L\omega_0}\]
\[\ba
i&=i(0)\cos\qty(\frac{1}{\sqrt{LC}}t)-v_C(0)\sqrt{\frac{C}{L}}\sin\qty(\frac{1}{\sqrt{LC}}t)\\
&=i(0)\cos\qty(\omega_0t)-v_C(0)C\omega_0\sin\qty(\omega_0t)
\ea\]
\[\ba
v_C&=-v_L=-L\dv{i}{t}\\
&=i(0)\sqrt{\frac{L}{C}}\sin\qty(\frac{1}{\sqrt{LC}}t)+v_C(0)\cos\qty(\frac{1}{\sqrt{LC}}t)\\
&=i(0)L\omega_0\sin\qty(\omega_0t)+v_C(0)\cos\qty(\omega_0t)
\ea\]
\[\ba
p_C&=-p_L\\
&=\qty(i(0)\cos\qty(\frac{1}{\sqrt{LC}}t)-v_C(0)\sqrt{\frac{C}{L}}\sin\qty(\frac{1}{\sqrt{LC}}t))\qty(i(0)\sqrt{\frac{L}{C}}\sin\qty(\frac{1}{\sqrt{LC}}t)+v_C(0)\cos\qty(\frac{1}{\sqrt{LC}}t))\\
&=(i(0))^2\sqrt{\frac{L}{C}}\cos\qty(\frac{1}{\sqrt{LC}}t)\sin\qty(\frac{1}{\sqrt{LC}}t)+i(0)v_C(0)\cos^2\qty(\frac{1}{\sqrt{LC}}t)-i(0)v_C(0)\sin^2\qty(\frac{1}{\sqrt{LC}}t)-(v_C(0))^2\sqrt{\frac{C}{L}}\sin\qty(\frac{1}{\sqrt{LC}}t)\cos\qty(\frac{1}{\sqrt{LC}}t)\\
&=\frac{1}{2}\qty((i(0))^2\sqrt{\frac{L}{C}}-(v_C(0))^2\sqrt{\frac{C}{L}})\sin\qty(\frac{2}{\sqrt{LC}}t)+i(0)v_C(0)\cos\qty(\frac{2}{\sqrt{LC}}t)\\
&=\frac{1}{2}\qty((i(0))^2L\omega_0-(v_C(0))^2C\omega_0)\sin\qty(2\omega_0t)+i(0)v_C(0)\cos\qty(2\omega_0t)
\ea\]
\[\ba
I&=\qty(Li(0)-\frac{v_C(0)}{s})\frac{1}{Ls+\frac{1}{Cs}}\\
&=\frac{LCsi(0)-Cv_C(0)}{LCs^2+1}\\
&=\frac{\omega_0^{\pht{0}-2}si(0)-Cv_C(0)}{\omega_0^{\pht{0}-2}s^2+1}\\
&=\frac{si(0)-C\omega_0^{\pht{0}2}v_C(0)}{s^2+\omega_0^{\pht{0}2}}
\ea\]
\[\ba
V_C&=-V_L=-LsI+Li(0)\\
&=\frac{-L^2Cs^2i(0)+LCsv_C(0)+L^2Cs^2i(0)+Li(0)}{LCs^2+1}\\
&=\frac{LCsv_C(0)+Li(0)}{LCs^2+1}\\
&=\frac{\omega_0^{\pht{0}-2}sv_C(0)+Li(0)}{\omega_0^{\pht{0}-2}s^2+1}\\
&=\frac{sv_C(0)+L\omega_0^{\pht{0}2}i(0)}{s^2+\omega_0^{\pht{0}2}}
\ea\]
Check consistency:
\[\ba
\mathcal{L}\{i\}&=i(0)\frac{s}{s^2+\frac{1}{LC}}-v_C(0)\sqrt{\frac{C}{L}}\frac{\frac{1}{\sqrt{LC}}}{s^2+\frac{1}{LC}}\\
&=\frac{LCsi(0)-Cv_C(0)}{LCs^2+1}
\ea\]
\[\ba
\mathcal{L}\{v_C\}&=i(0)\sqrt{\frac{L}{C}}\frac{\frac{1}{\sqrt{LC}}}{s^2+\frac{1}{LC}}+v_C(0)\frac{s}{s^2+\frac{1}{LC}}\\
&=\frac{LCsv_C(0)+Li(0)}{LCs^2+1}
\ea\]
\end{proof}
Forced response:
\[LC\dv[2]{i}{t}+i=C\dv{v_{in}}{t}\]
\[\dv[2]{i}{t}+\omega_0^{\pht{0}2}i=\frac{1}{L}\dv{v_{in}}{t}\]
\[\ba
i&=\frac{1}{L}\int_0^t\cos\qty(\frac{1}{\sqrt{LC}}(t-\tau))v_{in}(\tau)\dd{\tau}\\
&=\frac{1}{L}\int_0^t\cos\qty(\omega_0(t-\tau))v_{in}(\tau)\dd{\tau}
\ea\]
\[\ba
v_C&=v_{in}-v_L\\
&=\frac{1}{\sqrt{LC}}\int_0^t\sin\qty(\frac{1}{\sqrt{LC}}(t-\tau))v_{in}(\tau)\dd{\tau}\\
&=\omega_0\int_0^t\sin\qty(\omega_0(t-\tau))v_{in}(\tau)\dd{\tau}
\ea\]
\[\ba
I&=\frac{CsV_{in}}{LCs^2+1}\\
&=\frac{\omega_0^{\pht{0}2}CsV_{in}}{s^2+\omega_0^{\pht{0}2}}
\ea\]
\[\ba
V_C&=\frac{V_{in}}{LCs^2+1}\\
&=\frac{\omega_0^{\pht{0}2}}V_{in}}{s^2+\omega_0^{\pht{0}2}}}
\ea\]
\[\ba
V_L&=\frac{LCs^2V_{in}}{LCs^2+1}\\
&=\frac{s^2V_{in}}{s^2+\omega_0^{\pht{0}2}}
\ea\]
\begin{proof}
\[v_L+v_C=v_{in}\]
\[i=C\dv{v_C}{t}\]
\[v_C=v_{in}-v_L=v_{in}-L\dv{i}{t}\]
\[i=C\dv{v_{in}}{t}-LC\dv[2]{i}{t}\]
\[LC\dv[2]{i}{t}+i=C\dv{v_{in}}{t}\]
\[\dv[2]{i}{t}+\frac{1}{LC}i=\frac{1}{L}\dv{v_{in}}{t}\]
\[\dv[2]{i}{t}+\omega_0^{\pht{0}2}i=\frac{1}{L}\dv{v_{in}}{t}\]
\[i_1=\cos\qty(\frac{1}{\sqrt{LC}}t)\]
\[i_2=\sin\qty(\frac{1}{\sqrt{LC}}t)\]
\[\frac{1}{i_1i_2'-i_2i_1'}=\sqrt{LC}\]
\[\ba
i&=\sqrt{\frac{C}{L}}\qty(-\cos\qty(\frac{1}{\sqrt{LC}}t)\int_0^t\sin\qty(\frac{1}{\sqrt{LC}}\tau)\dv{v_{in}}{\tau}\dd{\tau}+\sin\qty(\frac{1}{\sqrt{LC}}t)\int\cos\qty(\frac{1}{\sqrt{LC}}\tau)\dv{v_{in}}{\tau}\dd{\tau})\\
&=\sqrt{\frac{C}{L}}\qty(-\cos\qty(\frac{1}{\sqrt{LC}}t)\qty(\sin\qty(\frac{1}{\sqrt{LC}}t)v_{in}-\frac{1}{\sqrt{LC}}\int_0^t\cos\qty(\frac{1}{\sqrt{LC}}\tau)v_{in}(\tau)\dd{t})+\sin\qty(\frac{1}{\sqrt{LC}}t)\qty(\cos\qty(\frac{1}{\sqrt{LC}}t)v_{in}+\frac{1}{\sqrt{LC}}\int_0^t\sin\qty(\frac{1}{\sqrt{LC}}\tau)v_{in}(\tau)\dd{\tau}))\\
&=-\sqrt{\frac{C}{L}}\cos\qty(\frac{1}{\sqrt{LC}}t)\sin\qty(\frac{1}{\sqrt{LC}}t)v_{in}+\frac{1}{L}\cos\qty(\frac{1}{\sqrt{LC}}t)\int_0^t\cos\qty(\frac{1}{\sqrt{LC}}\tau)v_{in}(\tau)\dd{\tau}+\sqrt{\frac{C}{L}}\sin\qty(\frac{1}{\sqrt{LC}}t)\cos\qty(\frac{1}{\sqrt{LC}}t)v_{in}+\frac{1}{L}\sin\qty(\frac{1}{\sqrt{LC}}t)\int_0^t\sin\qty(\frac{1}{\sqrt{LC}}\tau)v_{in}(\tau)\dd{\tau}\\
&=\frac{1}{L}\qty(\cos\qty(\frac{1}{\sqrt{LC}}t)\int_0^t\cos\qty(\frac{1}{\sqrt{LC}}\tau)v_{in}(\tau)\dd{\tau}+\sin\qty(\frac{1}{\sqrt{LC}}t)\int_0^t\sin\qty(\frac{1}{\sqrt{LC}}\tau)v_{in}(\tau)\dd{\tau})\\
&=\frac{1}{L}\int_0^t\qty(\cos\qty(\frac{1}{\sqrt{LC}}t)\cos\qty(\frac{1}{\sqrt{LC}}\tau)+\sin\qty(\frac{1}{\sqrt{LC}}t)\sin\qty(\frac{1}{\sqrt{LC}}\tau))v_{in}(\tau)\dd{\tau}\\
&=\frac{1}{L}\int_0^t\cos\qty(\frac{1}{\sqrt{LC}}(t-\tau))v_{in}(\tau)\dd{\tau}\\
&=\frac{1}{L}\int_0^t\cos\qty(\omega_0(t-\tau))v_{in}(\tau)\dd{\tau}
\ea\]
\[\ba
v_C&=v_{in}-v_L=v_{in}-L\dv{i}{t}\\
&=v_{in}-\cos(0)v_{in}-\int_0^t\pdv{}{t}\qty(\cos\qty(\frac{1}{\sqrt{LC}}(t-\tau))v_{in}(\tau))\dd{\tau}\\
&=\frac{1}{\sqrt{LC}}\int_0^t\sin\qty(\frac{1}{\sqrt{LC}}(t-\tau))v_{in}(\tau)\dd{\tau}\\
&=\omega_0\int_0^t\sin\qty(\omega_0(t-\tau))v_{in}(\tau)\dd{\tau}
\ea\]
\[\ba
I&=\frac{V_{in}}{Ls+\frac{1}{Cs}}\\
&=\frac{CsV_{in}}{LCs^2+1}\\
&=\frac{CsV_{in}}{\omega_0^{\pht{0}-2}s^2+1}\\
&=\frac{\omega_0^{\pht{0}2}CsV_{in}}{s^2+\omega_0^{\pht{0}2}}
\ea\]
\[\ba
V_L&=\frac{LCs^2V_{in}}{LCs^2+1}\\
&=\frac{s^2V_{in}}{s^2+\omega_0^{\pht{0}2}}
\ea\]
\[\ba
V_C&=\frac{V_{in}}{LCs^2+1}\\
&=\frac{\omega_0^{\pht{0}2}V_{in}}{s^2+\omega_0^{\pht{0}2}}
\ea\]
Check consistency:
\[\mathcal{L}\{i\}=\frac{1}{L}\frac{s}{s^2+\frac{1}{LC}}V_{in}=\frac{CsV_{in}}{LCs^2+1}\]
\[\mathcal{L}\{v_C\}=\frac{1}{\sqrt{LC}}\frac{\frac{1}{\sqrt{LC}}}{s^2+\frac{1}{LC}}V_{in}=\frac{V_{in}}{LCs^2+1}\]
\end{proof}
Total response:
\[\ba
i&=i(0)\cos\qty(\frac{1}{\sqrt{LC}}t)-v_C(0)\sqrt{\frac{C}{L}}\sin\qty(\frac{1}{\sqrt{LC}}t)+\frac{1}{L}\int_0^t\cos\qty(\frac{1}{\sqrt{LC}}(t-\tau))v_{in}(\tau)\dd{\tau}\\
&=i(0)\cos\qty(\omega_0t)-v_C(0)C\omega_0\sin\qty(\omega_0t)+\frac{1}{L}\int_0^t\cos\qty(\omega_0(t-\tau))v_{in}(\tau)\dd{\tau}
\ea\]
\[\ba
v_C&=v_{in}-v_L\\
&=i(0)\sqrt{\frac{L}{C}}\sin\qty(\frac{1}{\sqrt{LC}}t)+v_C(0)\cos\qty(\frac{1}{\sqrt{LC}}t)+\frac{1}{\sqrt{LC}}\int_0^t\sin\qty(\frac{1}{\sqrt{LC}}(t-\tau))v_{in}(\tau)\dd{\tau}\\
&=i(0)L\omega_0\sin\qty(\omega_0t)+v_C(0)\cos\qty(\omega_0t)+\omega_0\int_0^t\sin\qty(\omega_0(t-\tau))v_{in}(\tau)\dd{\tau}
\ea\]
\[\ba
I&=\frac{LCsi(0)-Cv_C(0)+CsV_{in}}{LCs^2+1}\\
&=\frac{si(0)-C\omega_0^{\pht{0}2}v_C(0)+\omega_0^{\pht{0}2}CsV_{in}}{s^2+\omega_0^{\pht{0}2}}
\ea\]
\[\ba
V_C&=\frac{LCsv_C(0)+Li(0)+V_{in}}{LCs^2+1}\\
&=\frac{sv_C(0)+L\omega_0^{\pht{0}2}i(0)+\omega_0^{\pht{0}2}}V_{in}}{s^2+\omega_0^{\pht{0}2}}}
\ea\]
\[\ba
V_L&=\frac{LCs^2V_{in}-LCsv_C(0)-Li(0)}{LCs^2+1}\\
&=\frac{s^2V_{in}-sv_C(0)-L\omega_0^{\pht{0}2}i(0)}{s^2+\omega_0^{\pht{0}2}}
\ea\]
Impulse response and transfer function:
\[\ba
h_i&=\frac{1}{L}\cos\qty(\frac{1}{\sqrt{LC}}t)u(t)\\
&=\frac{1}{L}\cos\qty(\omega_0t)u(t)
\ea\]
\[\ba
h_{v_C}&=\delta(t)-h_{v_L}\\
&=\frac{1}{\sqrt{LC}}\sin\qty(\frac{1}{\sqrt{LC}}t)\\
&=\omega_0\sin\qty(\omega_0t)
\ea\]
\[\ba
H_I&=\frac{Cs}{LCs^2+1}\\
&=\frac{\omega_0^{\pht{0}2}Cs}{s^2+\omega_0^{\pht{0}2}}
\ea\]
\[\ba
H_{V_C}&=\frac{1}{LCs^2+1}\\
&=\frac{\omega_0^{\pht{0}2}}}{s^2+\omega_0^{\pht{0}2}}}
\ea\]
\[\ba
H_{V_L}&=\frac{LCs^2}{LCs^2+1}\\
&=\frac{s^2}{s^2+\omega_0^{\pht{0}2}}
\ea\]
$V_C$ being low-pass filter:
\[\lim_{\omega\to 0}\abs{H_{V_C}(j\omega)}=1\]
\[\lim_{\omega\to\infty}\abs{H_{V_C}(j\omega)}=0\]
\begin{proof}
\[\lim_{\omega\to 0}\abs{H_{V_C}(j\omega)}=\lim_{\omega\to 0}\abs{\frac{1}{1-LC\omega}}=1\]
\[\lim_{\omega\to\infty}\abs{H_{V_C}(j\omega)}=\lim_{\omega\to\infty}\abs{\frac{1}{1-LC\omega}}=0\]
\end{proof}
$V_L$ being high-pass filter:
\[\lim_{\omega\to 0}\abs{H_{V_L}(j\omega)}=0\]
\[\lim_{\omega\to\infty}\abs{H_{V_L}(j\omega)}=1\]
\begin{proof}
\[\lim_{\omega\to 0}\abs{H_{V_L}(j\omega)}=\lim_{\omega\to 0}\abs{\frac{-LC\omega^2}{1-LC\omega^2}}=0\]
\[\lim_{\omega\to\infty}\abs{H_{V_L}(j\omega)}=\lim_{\omega\to\infty}\abs{\frac{-LC\omega^2}{1-LC\omega^2}}=1\]
\end{proof}
\sssc{Parallel LC circuit}
A parallel LC (resonant, oscillator, tank, or tuned) circuit consists of an inductor with inductance $L$, a capacitor with capacitance $C$, and an independent current source $i_{in}$ in parallel connection. Angular resonance frequency $\omega_0$.

Natural response: Total energy stored in inductor and capacitor $E$.
\[LC\dv[2]{v}{t}+v=0\]
\[\omega_0=\frac{1}{\sqrt{LC}}\]
\[\dv[2]{v}{t}+\omega_0^{\pht{0}2}v=0\]
\[\ba
v&=v(0)\cos\qty(\frac{1}{\sqrt{LC}}t)-i_L(0)\sqrt{\frac{L}{C}}\sin\qty(\frac{1}{\sqrt{LC}}t)\\
&=v(0)\cos\qty(\omega_0t)-i_L(0)L\omega_0\sin\qty(\omega_0t)
\ea\]
\[\ba
i_L&=-i_C\\
&=v(0)\sqrt{\frac{C}{L}}\sin\qty(\frac{1}{\sqrt{LC}}t)+i_L(0)\cos\qty(\frac{1}{\sqrt{LC}}t)\\
&=v(0)\sqrt{\frac{C}{L}}\sin\qty(\frac{1}{\sqrt{LC}}t)+i_L(0)\cos\qty(\frac{1}{\sqrt{LC}}t)
\ea\]
\[\ba
p_L&=-p_C\\
&=\frac{1}{2}\qty((v(0))^2\sqrt{\frac{C}{L}}-(i_L(0))^2)\sin\qty(\frac{2}{\sqrt{LC}}t)+v(0)i_L(0)\cos\qty(\frac{2}{\sqrt{LC}}t)\\
&=\frac{1}{2}\qty((v(0))^2C\omega_0-(i_L(0))^2)\sin\qty(2\omega_0t)+v(0)i_L(0)\cos\qty(2\omega_0t)
\ea\]
\[\lim_{n\to\infty}\frac{1}{n}\int_0^np_C\dd{t}=-\lim_{n\to\infty}\frac{1}{n}\int_0^np_L\dd{t}=0\]
\[E=\frac{1}{2}\qty(C(v_C(0))^2+L(I(0))^2)\]
\[\ba
V&=\frac{LCsv(0)-Li_L(0)}{LCs^2+1}\\
&=\frac{sv(0)-L\omega_0^{\pht{0}2}i_L(0)}{s^2+\omega_0^{\pht{0}2}}
\ea\]
\[\ba
I_L&=-I_C\\
&=\frac{Cv(0)+LCsi_L(0)}{LCs^2+1}\\
&=\frac{C\omega_0^{\pht{0}2}v(0)+si_L(0)}{s^2+\omega_0^{\pht{0}2}}
\ea\]
\begin{proof}
\[i_L+i_C=0\]
\[v=L\dv{i_L}{t}\]
\[i_L=-i_C=-C\dv{v}{t}\]
\[LC\dv[2]{v}{t}+v=0\]
\[LCr^2+1=0\]
\[r=\pm\frac{1}{\sqrt{LC}}j\]
\[\omega_0=\frac{1}{\sqrt{LC}}\]
\[\dv[2]{v}{t}+\omega_0^{\pht{0}2}v=0\]
\[v=A\cos\qty(\frac{1}{\sqrt{LC}}t)+B\sin\qty(\frac{1}{\sqrt{LC}}t)\]
\[A=v(0)\]
\[i_L=v(0)\sqrt{\frac{C}{L}}\sin\qty(\frac{1}{\sqrt{LC}}t)-B\sqrt{\frac{C}{L}}\cos\qty(\frac{1}{\sqrt{LC}}t)\]
\[B=-i_L(0)\sqrt{\frac{L}{C}}\]
\[\ba
v&=v(0)\cos\qty(\frac{1}{\sqrt{LC}}t)-i_L(0)\sqrt{\frac{L}{C}}\sin\qty(\frac{1}{\sqrt{LC}}t)\\
&=v(0)\cos\qty(\omega_0t)-i_L(0)L\omega_0\sin\qty(\omega_0t)
\ea\]
\[\ba
i_L&=-i_C=-C\dv{v}{t}\\
&=v(0)\sqrt{\frac{C}{L}}\sin\qty(\frac{1}{\sqrt{LC}}t)+i_L(0)\cos\qty(\frac{1}{\sqrt{LC}}t)\\
&=v(0)\sqrt{\frac{C}{L}}\sin\qty(\frac{1}{\sqrt{LC}}t)+i_L(0)\cos\qty(\frac{1}{\sqrt{LC}}t)
\ea\]
\[\ba
p_L&=-p_C\\
&=\qty(v(0)\cos\qty(\frac{1}{\sqrt{LC}}t)-i_L(0)\sqrt{\frac{L}{C}}\sin\qty(\frac{1}{\sqrt{LC}}t))\qty(v(0)\sqrt{\frac{C}{L}}\sin\qty(\frac{1}{\sqrt{LC}}t)+i_L(0)\cos\qty(\frac{1}{\sqrt{LC}}t))\\
&=(v(0))^2\sqrt{\frac{C}{L}}\sin\qty(\frac{1}{\sqrt{LC}}t)\cos\qty(\frac{1}{\sqrt{LC}}t)+v(0)i_L(0)\cos^2\qty(\frac{1}{\sqrt{LC}}t)-v(0)i_L(0)\sin^2\qty(\frac{1}{\sqrt{LC}}t)-(i_L(0))^2\sin\qty(\frac{1}{\sqrt{LC}}t)\cos\qty(\frac{1}{\sqrt{LC}}t)\\
&=\frac{1}{2}\qty((v(0))^2\sqrt{\frac{C}{L}}-(i_L(0))^2)\sin\qty(\frac{2}{\sqrt{LC}}t)+v(0)i_L(0)\cos\qty(\frac{2}{\sqrt{LC}}t)\\
&=\frac{1}{2}\qty((v(0))^2C\omega_0-(i_L(0))^2)\sin\qty(2\omega_0t)+v(0)i_L(0)\cos\qty(2\omega_0t)
\ea\]
\[I_L+I_C=0\]
\[\ba
V&=\frac{Cv(0)-\frac{i_L(0)}{s}}{\frac{1}{Ls}+Cs}\\
&=\frac{LCsv(0)-Li_L(0)}{LCs^2+1}\\
&=\frac{\omega_0^{\pht{0}-2}sv(0)-Li_L(0)}{\omega_0^{\pht{0}-2}s^2+1}\\
&=\frac{sv(0)-L\omega_0^{\pht{0}2}i_L(0)}{s^2+\omega_0^{\pht{0}2}}
\ea\]
\[\ba
I_L&=-I_C\\
&=\frac{V}{Ls}+\frac{i_L(0)}{s}\\
&=\frac{Csv(0)-i_L(0)+LCs^2i_L(0)+i_L(0)}{LCs^3+s}\\
&=\frac{Cv(0)+LCsi_L(0)}{LCs^2+1}\\
&=\frac{Cv(0)+\omega_0^{\pht{0}-2}si_L(0)}{\omega_0^{\pht{0}-2}s^2+1}\\
&=\frac{C\omega_0^{\pht{0}2}v(0)+si_L(0)}{s^2+\omega_0^{\pht{0}2}}
\ea\]
Check consistency:
\[\ba
\mathcal{L}\{V\}&=v(0)\frac{s}{s^2+\frac{1}{LC}}-i_L(0)\sqrt{\frac{L}{C}}\frac{\frac{1}{\sqrt{LC}}{s^2+\frac{1}{LC}}\\
&=\frac{LCsv(0)-Li_L(0)}{LCs^2+1}
\ea\]
\[\ba
\mathcal{L}\{i_L\}&=v(0)\sqrt{\frac{C}{L}}\frac{\frac{1}{\sqrt{LC}}{s^2+\frac{1}{LC}}+i_L(0)\frac{s}{s^2+\frac{1}{LC}}\\
\ea\]
\end{proof}
Forced response:
\[LC\dv[2]{v}{t}+v=\dv{i_{in}}{t}\]
\[\dv[2]{v}{t}+\omega_0^{\pht{0}2}v=\omega_0^{\pht{0}2}\dv{i_{in}}{t}\]
\[\ba
v&=\frac{1}{C}\int_0^t\cos\qty(\frac{1}{\sqrt{LC}}(t-\tau))i_{in}(\tau)\dd{\tau}\\
&=\frac{1}{C}\int_0^t\cos\qty(\omega_0(t-\tau))i_{in}(\tau)\dd{\tau}
\ea\]
\[\ba
i_L&=i_{in}-i_C\\
&=\frac{1}{\sqrt{LC}}\int_0^t\sin\qty(\frac{1}{\sqrt{LC}}(t-\tau))i_{in}(\tau)\dd{\tau}\\
&=\omega_0\int_0^t\sin\qty(\omega_0(t-\tau))i_{in}(\tau)\dd{\tau}
\ea\]
\[\ba
V&=\frac{LsI_{in}}{LCs^2+1}\\
&=\frac{\omega_0^{\pht{0}2}LsI_{in}}{s^2+\omega_0^{\pht{0}2}}
\ea\]
\[\ba
I_L&=\frac{I_{in}}{LCs^2+1}\\
&=\frac{\omega_0^{\pht{0}2}I_{in}}{s^2+\omega_0^{\pht{0}2}}
\ea\]
\[\ba
I_C&=\frac{LCs^2I_{in}}{LCs^2+1}\\
&=\frac{s^2I_{in}}{s^2+\omega_0^{\pht{0}2}}
\ea\]
\begin{proof}
\[i_L+i_C=i_{in}\]
\[v=L\dv{i_L}{t}\]
\[i_L=i_{in}-i_C=i_{in}-C\dv{v}{t}\]
\[v=L\dv{i_{in}}{t}-LC\dv[2]{v}{t}\]
\[LC\dv[2]{v}{t}+v=\dv{i_{in}}{t}\]
\[\dv[2]{v}{t}+\frac{v}{LC}=\frac{1}{C}\dv{i_{in}}{t}\]
\[\dv[2]{v}{t}+\omega_0^{\pht{0}2}v=\omega_0^{\pht{0}2}\dv{i_{in}}{t}\]
\[v_1=\cos\qty(\frac{1}{\sqrt{LC}}t)\]
\[v_2=\sin\qty(\frac{1}{\sqrt{LC}}t)\]
\[\frac{1}{v_1v_2'-v_2v_1'}=\sqrt{LC}\]
\[\ba
v&=\sqrt{\frac{L}{C}}\qty(-\cos\qty(\frac{1}{\sqrt{LC}}t)\int_0^t\sin\qty(\frac{1}{\sqrt{LC}}\tau)\dv{i_{in}}{\tau}\dd{\tau}+\sin\qty(\frac{1}{\sqrt{LC}}t)\int\cos\qty(\frac{1}{\sqrt{LC}}\tau)\dv{i_{in}}{\tau}\dd{\tau})\\
&=\sqrt{\frac{L}{C}}\qty(-\cos\qty(\frac{1}{\sqrt{LC}}t)\qty(\sin\qty(\frac{1}{\sqrt{LC}}t)i_{in}-\frac{1}{\sqrt{LC}}\int_0^t\cos\qty(\frac{1}{\sqrt{LC}}\tau)i_{in}(\tau)\dd{t})+\sin\qty(\frac{1}{\sqrt{LC}}t)\qty(\cos\qty(\frac{1}{\sqrt{LC}}t)i_{in}+\frac{1}{\sqrt{LC}}\int_0^t\sin\qty(\frac{1}{\sqrt{LC}}\tau)i_{in}(\tau)\dd{\tau}))\\
&=-\sqrt{\frac{L}{C}}\cos\qty(\frac{1}{\sqrt{LC}}t)\sin\qty(\frac{1}{\sqrt{LC}}t)i_{in}+\frac{1}{C}\cos\qty(\frac{1}{\sqrt{LC}}t)\int_0^t\cos\qty(\frac{1}{\sqrt{LC}}\tau)i_{in}(\tau)\dd{\tau}+\sqrt{\frac{L}{C}}\sin\qty(\frac{1}{\sqrt{LC}}t)\cos\qty(\frac{1}{\sqrt{LC}}t)i_{in}+\frac{1}{C}\sin\qty(\frac{1}{\sqrt{LC}}t)\int_0^t\sin\qty(\frac{1}{\sqrt{LC}}\tau)i_{in}(\tau)\dd{\tau}\\
&=\frac{1}{C}\qty(\cos\qty(\frac{1}{\sqrt{LC}}t)\int_0^t\cos\qty(\frac{1}{\sqrt{LC}}\tau)i_{in}(\tau)\dd{\tau}+\sin\qty(\frac{1}{\sqrt{LC}}t)\int_0^t\sin\qty(\frac{1}{\sqrt{LC}}\tau)i_{in}(\tau)\dd{\tau})\\
&=\frac{1}{C}\int_0^t\qty(\cos\qty(\frac{1}{\sqrt{LC}}t)\cos\qty(\frac{1}{\sqrt{LC}}\tau)+\sin\qty(\frac{1}{\sqrt{LC}}t)\sin\qty(\frac{1}{\sqrt{LC}}\tau))i_{in}(\tau)\dd{\tau}\\
&=\frac{1}{C}\int_0^t\cos\qty(\frac{1}{\sqrt{LC}}(t-\tau))i_{in}(\tau)\dd{\tau}\\
&=\frac{1}{C}\int_0^t\cos\qty(\omega_0(t-\tau))i_{in}(\tau)\dd{\tau}
\ea\]
\[\ba
i_L&=i_{in}-i_C=i_{in}-C\dv{v}{t}\\
&=i_{in}-\cos(0)i_{in}-\int_0^t\pdv{}{t}\qty(\cos\qty(\frac{1}{\sqrt{LC}}(t-\tau))i_{in}(\tau))\dd{\tau}\\
&=\frac{1}{\sqrt{LC}}\int_0^t\sin\qty(\frac{1}{\sqrt{LC}}(t-\tau))i_{in}(\tau)\dd{\tau}\\
&=\omega_0\int_0^t\sin\qty(\omega_0(t-\tau))i_{in}(\tau)\dd{\tau}
\ea\]
\[\ba
V&=\frac{I_{in}}{\frac{1}{Ls}+Cs}\\
&=\frac{LsI_{in}}{LCs^2+1}\\
&=\frac{LsI_{in}}{\omega_0^{\pht{0}-2}s^2+1}\\
&=\frac{\omega_0^{\pht{0}2}LsI_{in}}{s^2+\omega_0^{\pht{0}2}}
\ea\]
\[\ba
I_L&=\frac{I_{in}}{LCs^2+1}\\
&=\frac{\omega_0^{\pht{0}2}I_{in}}{s^2+\omega_0^{\pht{0}2}}
\ea\]
\[\ba
I_C&=\frac{LCs^2I_{in}}{LCs^2+1}\\
&=\frac{s^2I_{in}}{s^2+\omega_0^{\pht{0}2}}
\ea\]
Check consistency:
\[\mathcal{L}\{v\}=\frac{1}{C}\frac{s}{s^2+\frac{1}{LC}}I_{in}=\frac{LsI_{in}}{LCs^2+1}\]
\[\mathcal{L}\{i_L\}=\frac{1}{\sqrt{LC}}\frac{\frac{1}{\sqrt{LC}}}{s^2+\frac{1}{LC}}I_{in}=\frac{I_{in}}{LCs^2+1}\]
\end{proof}
Total response:
\[\ba
v&=\frac{1}{C}\int_0^t\cos\qty(\frac{1}{\sqrt{LC}}(t-\tau))i_{in}(\tau)\dd{\tau}+v(0)\cos\qty(\frac{1}{\sqrt{LC}}t)-i_L(0)\sqrt{\frac{L}{C}}\sin\qty(\frac{1}{\sqrt{LC}}t)\\
&=\frac{1}{C}\int_0^t\cos\qty(\omega_0(t-\tau))i_{in}(\tau)\dd{\tau}+v(0)\cos\qty(\omega_0t)-i_L(0)L\omega_0\sin\qty(\omega_0t)
\ea\]
\[\ba
i_L&=i_{in}-i_C\\
&=\frac{1}{\sqrt{LC}}\int_0^t\sin\qty(\frac{1}{\sqrt{LC}}(t-\tau))i_{in}(\tau)\dd{\tau}+v(0)\sqrt{\frac{C}{L}}\sin\qty(\frac{1}{\sqrt{LC}}t)+i_L(0)\cos\qty(\frac{1}{\sqrt{LC}}t)\\
&=\omega_0\int_0^t\sin\qty(\omega_0(t-\tau))i_{in}(\tau)\dd{\tau}+v(0)\sqrt{\frac{C}{L}}\sin\qty(\frac{1}{\sqrt{LC}}t)+i_L(0)\cos\qty(\frac{1}{\sqrt{LC}}t)
\ea\]
\[\ba
V&=\frac{LsI_{in}+LCsv(0)-Li_L(0)}{LCs^2+1}\\
&=\frac{\omega_0^{\pht{0}2}LsI_{in}+sv(0)-L\omega_0^{\pht{0}2}i_L(0)}{s^2+\omega_0^{\pht{0}2}}
\ea\]
\[\ba
I_L&=\frac{I_{in}+Cv(0)+LCsi_L(0)}{LCs^2+1}\\
&=\frac{\omega_0^{\pht{0}2}I_{in}+C\omega_0^{\pht{0}2}v(0)+si_L(0)}{s^2+\omega_0^{\pht{0}2}}
\ea\]
\[\ba
I_C&=\frac{LCs^2I_{in}-Cv(0)-LCsi_L(0)}{LCs^2+1}\\
&=\frac{s^2I_{in}-C\omega_0^{\pht{0}2}v(0)-si_L(0)}{s^2+\omega_0^{\pht{0}2}}
\ea\]
Impulse response and transfer function:
\[\ba
h_i&=\frac{1}{C}\cos\qty(\frac{1}{\sqrt{LC}}t)u(t)\\
&=\frac{1}{C}\cos\qty(\omega_0t)u(t)
\ea\]
\[\ba
h_{i_L}&=\delta(t)-h_{i_C}\\
&=\frac{1}{\sqrt{LC}}\sin\qty(\frac{1}{\sqrt{LC}}t)\\
&=\omega_0\sin\qty(\omega_0t)
\ea\]
\[\ba
H_V&=\frac{Ls}{LCs^2+1}\\
&=\frac{\omega_0^{\pht{0}2}Ls}{s^2+\omega_0^{\pht{0}2}}
\ea\]
\[\ba
H_{I_L}&=\frac{1}{LCs^2+1}\\
&=\frac{\omega_0^{\pht{0}2}}}{s^2+\omega_0^{\pht{0}2}}}
\ea\]
\[\ba
H_{I_C}&=\frac{LCs^2}{LCs^2+1}\\
&=\frac{s^2}{s^2+\omega_0^{\pht{0}2}}
\ea\]
$I_L$ being low-pass filter:
\[\lim_{\omega\to 0}\abs{H_{I_L}(j\omega)}=1\]
\[\lim_{\omega\to\infty}\abs{H_{I_L}(j\omega)}=0\]
\begin{proof}
\[\lim_{\omega\to 0}\abs{H_{I_L}(j\omega)}=\lim_{\omega\to 0}\abs{\frac{1}{1-LC\omega}}=1\]
\[\lim_{\omega\to\infty}\abs{H_{I_L}(j\omega)}=\lim_{\omega\to\infty}\abs{\frac{1}{1-LC\omega}}=0\]
\end{proof}
$I_C$ being high-pass filter:
\[\lim_{\omega\to 0}\abs{H_{I_C}(j\omega)}=0\]
\[\lim_{\omega\to\infty}\abs{H_{I_C}(j\omega)}=1\]
\begin{proof}
\[\lim_{\omega\to 0}\abs{H_{I_C}(j\omega)}=\lim_{\omega\to 0}\abs{\frac{-LC\omega^2}{1-LC\omega^2}}=0\]
\[\lim_{\omega\to\infty}\abs{H_{I_C}(j\omega)}=\lim_{\omega\to\infty}\abs{\frac{-LC\omega^2}{1-LC\omega^2}}=1\]
\end{proof}
















\sssc{Series RLC circuit}
A series RLC (oscillator) circuit consists of a resistor with resistance $R$, an inductor with inductance $L$, a capacitor with capacitance $C$, and an independent voltage source $v_{in}$ in series connection. Neper frequency (奈培頻率) or attenuation (衰減量) $\alpha$, angular resonance frequency $\omega_0$, damping factor (阻尼係數) $\zeta$, characteristic roots $r$.
\[\dv[2]{i}{t}+\frac{R}{L}\dv{i}{t}+\frac{i}{LC}=0\]
\[\alpha=\frac{R}{2L}\]
\[\omega_0=\frac{1}{\sqrt{LC}}\]
\[\zeta=\frac{R}{2}\sqrt{\frac{C}{L}}\]
\[\dv[2]{i}{t}+2\omega_0\zeta\dv{i}{t}+\omega_0^{\pht{0}2}i=0\]
\[r=-\omega_0\qty(\zeta\pm\sqrt{\zeta^2-1})\]
\begin{proof}
\[v_R+v_C+v_L=0\]
\[v_R=Ri\]
\[i=C\dv{v_C}{t}\]
\[v_L=L\dv{i}{t}\]
\[R\dv{i}{t}+\frac{i}{C}+L\dv[2]{i}{t}=0\]
\[\dv[2]{i}{t}+\frac{R}{L}\dv{i}{t}+\frac{i}{LC}=0\]
\[\alpha=\frac{R}{2L}\]
\[\omega_0=\frac{1}{\sqrt{LC}}\]
\[\zeta=\frac{\alpha}{\omega_0}=\frac{R}{2}\sqrt{\frac{C}{L}}\]
\[\dv[2]{i}{t}+2\omega_0\zeta\dv{i}{t}+\omega_0^{\pht{0}2}i=0\]
\[r^2+2\omega_0\zeta+\omega_0^{\pht{0}2}=0\]
\[\ba
r&=\frac{-2\omega_0\zeta\pm\sqrt{4\omega_0^{\pht{0}2}\zeta^2-4\omega_0^{\pht{0}2}}}{2}\\
&=-\omega_0\qty(\zeta\pm\sqrt{\zeta^2-1})
\ea\]
\end{proof}
Natural response: Total energy stored in inductor and capacitor $E$.
\bit
\item $\zeta>1$ Overdamped response (過阻尼響應):
\[i=-\frac{i(0)R\qty(\zeta-\sqrt{\zeta^2-1})+2v_C(0)\zeta}{2R\sqrt{\zeta^2-1}}e^{-\omega_0\qty(\zeta-\sqrt{\zeta^2-1})t}+\frac{i(0)R\qty(\zeta+\sqrt{\zeta^2-1})+2v_C(0)\zeta}{2R\sqrt{\zeta^2-1}}e^{-\omega_0\qty(\zeta+\sqrt{\zeta^2-1})t}\]
\[v_R=-\frac{i(0)R\qty(\zeta-\sqrt{\zeta^2-1})+2v_C(0)\zeta}{2\sqrt{\zeta^2-1}}e^{-\omega_0\qty(\zeta-\sqrt{\zeta^2-1})t}+\frac{i(0)R\qty(\zeta+\sqrt{\zeta^2-1})+2v_C(0)\zeta}{2\sqrt{\zeta^2-1}}e^{-\omega_0\qty(\zeta+\sqrt{\zeta^2-1})t}\]
\[v_L=\frac{i(0)R\qty(\zeta-\sqrt{\zeta^2-1})+2v_C(0)\zeta}{2\sqrt{\zeta^2-1}}\frac{\zeta-\sqrt{\zeta^2-1}}{2\zeta}e^{-\omega_0\qty(\zeta-\sqrt{\zeta^2-1})t}-\frac{i(0)R\qty(\zeta+\sqrt{\zeta^2-1})+2v_C(0)\zeta}{2\sqrt{\zeta^2-1}}\frac{\zeta+\sqrt{\zeta^2-1}}{2\zeta}e^{-\omega_0\qty(\zeta+\sqrt{\zeta^2-1})t}\]
\[v_C=\frac{i(0)R+2v_C(0)\zeta\qty(\zeta+\sqrt{\zeta^2-1})}{4\zeta\sqrt{\zeta^2-1}}e^{-\omega_0\qty(\zeta-\sqrt{\zeta^2-1})t}-\frac{i(0)R+2v_C(0)\zeta\qty(\zeta-\sqrt{\zeta^2-1})}{4\zeta\sqrt{\zeta^2-1}}e^{-\omega_0\qty(\zeta+\sqrt{\zeta^2-1})t}\]
\begin{proof}
\[i=Ae^{-\omega_0\qty(\zeta-\sqrt{\zeta^2-1})t}+Be^{-\omega_0\qty(\zeta+\sqrt{\zeta^2-1})t}\]
\[i(0)=A+B\]
\[v_R=ARe^{-\omega_0\qty(\zeta-\sqrt{\zeta^2-1})t}+BRe^{-\omega_0\qty(\zeta+\sqrt{\zeta^2-1})t}\]
\[\ba
v_L&=L\dv{i}{t}\\
&=-AR\frac{\zeta-\sqrt{\zeta^2-1}}{2\zeta}e^{-\omega_0\qty(\zeta-\sqrt{\zeta^2-1})t}-BR\frac{\zeta+\sqrt{\zeta^2-1}}{2\zeta}e^{-\omega_0\qty(\zeta+\sqrt{\zeta^2-1})t}
\ea\]
\[
\ba
v_C&=-v_R-v_L\\
&=-ARe^{-\omega_0\qty(\zeta-\sqrt{\zeta^2-1})t}-BRe^{-\omega_0\qty(\zeta+\sqrt{\zeta^2-1})t}+AR\frac{\zeta-\sqrt{\zeta^2-1}}{2\zeta}e^{-\omega_0\qty(\zeta-\sqrt{\zeta^2-1})t}+BR\frac{\zeta+\sqrt{\zeta^2-1}}{2\zeta}e^{-\omega_0\qty(\zeta+\sqrt{\zeta^2-1})t}\\
&=-AR\frac{\zeta+\sqrt{\zeta^2-1}}{2\zeta}e^{-\omega_0\qty(\zeta-\sqrt{\zeta^2-1})t}-BR\frac{\zeta-\sqrt{\zeta^2-1}}{2\zeta}e^{-\omega_0\qty(\zeta+\sqrt{\zeta^2-1})t}
\ea\]
\[\ba
v_C(0)&=-AR\frac{\zeta+\sqrt{\zeta^2-1}}{2\zeta}-BR\frac{\zeta-\sqrt{\zeta^2-1}}{2\zeta}\\
&=-AR\frac{\zeta+\sqrt{\zeta^2-1}}{2\zeta}-i(0)R\frac{\zeta-\sqrt{\zeta^2-1}}{2\zeta}+AR\frac{\zeta-\sqrt{\zeta^2-1}}{2\zeta}\\
&=\frac{-i(0)R\qty(\zeta-\sqrt{\zeta^2-1})-2AR\sqrt{\zeta^2-1}}{2\zeta}
\ea\]
\[A=-\frac{i(0)R\qty(\zeta-\sqrt{\zeta^2-1})+2v_C(0)\zeta}{2R\sqrt{\zeta^2-1}}\]
\[B=i(0)-A=\frac{i(0)R\qty(\zeta+\sqrt{\zeta^2-1})+2v_C(0)\zeta}{2R\sqrt{\zeta^2-1}}\]
\[i=-\frac{i(0)R\qty(\zeta-\sqrt{\zeta^2-1})+2v_C(0)\zeta}{2R\sqrt{\zeta^2-1}}e^{-\omega_0\qty(\zeta-\sqrt{\zeta^2-1})t}+\frac{i(0)R\qty(\zeta+\sqrt{\zeta^2-1})+2v_C(0)\zeta}{2R\sqrt{\zeta^2-1}}e^{-\omega_0\qty(\zeta+\sqrt{\zeta^2-1})t}\]
\[\ba
v_C&=\frac{i(0)R\qty(\zeta-\sqrt{\zeta^2-1})+2v_C(0)\zeta}{2R\sqrt{\zeta^2-1}}R\frac{\zeta+\sqrt{\zeta^2-1}}{2\zeta}e^{-\omega_0\qty(\zeta-\sqrt{\zeta^2-1})t}-\frac{i(0)R\qty(\zeta+\sqrt{\zeta^2-1})+2v_C(0)\zeta}{2R\sqrt{\zeta^2-1}}R\frac{\zeta-\sqrt{\zeta^2-1}}{2\zeta}e^{-\omega_0\qty(\zeta+\sqrt{\zeta^2-1})t}\\
&=\frac{i(0)R+2v_C(0)\zeta\qty(\zeta+\sqrt{\zeta^2-1})}{4\zeta\sqrt{\zeta^2-1}}e^{-\omega_0\qty(\zeta-\sqrt{\zeta^2-1})t}-\frac{i(0)R+2v_C(0)\zeta\qty(\zeta-\sqrt{\zeta^2-1})}{4\zeta\sqrt{\zeta^2-1}}e^{-\omega_0\qty(\zeta+\sqrt{\zeta^2-1})t}
\ea\]
Check $i=C\dv{v_C}{t}$:
\[\ba
i&=C\dv{v_C}{t}\\
&=-\frac{i(0)R+2v_C(0)\zeta\qty(\zeta+\sqrt{\zeta^2-1})}{4\zeta\sqrt{\zeta^2-1}}\frac{2\zeta}{R}\qty(\zeta-\sqrt{\zeta^2-1})e^{-\omega_0\qty(\zeta-\sqrt{\zeta^2-1})t}+\frac{i(0)R+2v_C(0)\zeta\qty(\zeta-\sqrt{\zeta^2-1})}{4\zeta\sqrt{\zeta^2-1}}\frac{2\zeta}{R}\qty(\zeta+\sqrt{\zeta^2-1})e^{-\omega_0\qty(\zeta+\sqrt{\zeta^2-1})t}\\
&=-\frac{i(0)R\qty(\zeta-\sqrt{\zeta^2-1})+2v_C(0)\zeta}{2R\sqrt{\zeta^2-1}}e^{-\omega_0\qty(\zeta-\sqrt{\zeta^2-1})t}+\frac{i(0)R\qty(\zeta+\sqrt{\zeta^2-1})+2v_C(0)\zeta}{2R\sqrt{\zeta^2-1}}e^{-\omega_0\qty(\zeta+\sqrt{\zeta^2-1})t}
\ea\]
\[v_L=\frac{i(0)R\qty(\zeta-\sqrt{\zeta^2-1})+2v_C(0)\zeta}{2\sqrt{\zeta^2-1}}\frac{\zeta-\sqrt{\zeta^2-1}}{2\zeta}e^{-\omega_0\qty(\zeta-\sqrt{\zeta^2-1})t}-\frac{i(0)R\qty(\zeta+\sqrt{\zeta^2-1})+2v_C(0)\zeta}{2\sqrt{\zeta^2-1}}\frac{\zeta+\sqrt{\zeta^2-1}}{2\zeta}e^{-\omega_0\qty(\zeta+\sqrt{\zeta^2-1})t}\]
\end{proof}
\item $\zeta=1$ Critically damped response (臨界阻尼響應):
\[i=\qty(i(0)-\qty(i(0)+\frac{2v_C(0)}{R})\omega_0t)e^{-\omega_0t}\]
\[v_R=\qty(i(0)R-\qty(i(0)R+2v_C(0))\omega_0t)e^{-\omega_0t}\]
\[v_L=-\qty(i(0)R+v_C(0)+\qty(\frac{1}{2}i(0)R+v_C(0))\omega_0t)e^{-\omega_0t}\]
\[v_C=\qty(v_C(0)+\qty(\frac{1}{2}i(0)R+v_C(0))\omega_0t)e^{-\omega_0t}\]
\begin{proof}
\[i=(A+Bt)e^{-\omega_0t}\]
\[i(0)=A\]
\[i=(i(0)+Bt)e^{-\omega_0t}\]
\[v_R=(i(0)+Bt)Re^{-\omega_0t}\]
\[\ba
v_L&=L\dv{i}{t}\\
&=L\qty(-(i(0)+Bt)\omega_0+B)e^{-\omega_0t}\\
&=\qty(-i(0)\frac{R}{2}+\frac{BR}{2\omega_0}-\frac{BR}{2}t)e^{-\omega_0t}
\ea\]
\[\ba
v_C&=-v_R-v_L\\
&=\qty(-i(0)R-BRt+i(0)\frac{R}{2}-\frac{BR}{2\omega_0}+\frac{BR}{2}t)e^{-\omega_0t}\\
&=\qty(-i(0)\frac{R}{2}-\frac{BR}{2\omega_0}-\frac{BR}{2}t)e^{-\omega_0t}
\ea\]
\[v_C(0)=-i(0)\frac{R}{2}-\frac{BR}{2\omega_0}\]
\[\ba
B&=-\qty(i(0)\frac{R}{2}+v_C(0))\frac{2\omega_0}{R}\\
&=-\frac{\qty(i(0)R+2v_C(0))\omega_0}{R}\\
&=-\qty(i(0)+\frac{2v_C(0)}{R})\omega_0
\ea\]
\[i=\qty(i(0)-\qty(i(0)+\frac{2v_C(0)}{R})\omega_0t)e^{-\omega_0t}\]
\[\ba
v_C&=\qty(-i(0)\frac{R}{2}+\frac{i(0)R\omega_0+2v_C(0)\omega_0}{2\omega_0}+\frac{i(0)R\omega_0+2v_C(0)\omega_0}{2}t)e^{-\omega_0t}\\
&=\qty(v_C(0)+\qty(\frac{1}{2}i(0)R+v_C(0))\omega_0t)e^{-\omega_0t}
\ea\]
Check $i=C\dv{v_C}{t}$:
\[\ba
i&=C\dv{v_C}{t}\\
&=\qty(-v_C(0)\frac{2}{R}+\qty(\frac{1}{2}i(0)R+v_C(0))\frac{2}{R}-\qty(\frac{1}{2}i(0)R+v_C(0))\frac{2}{R}\omega_0t)e^{-\omega_0t}\\
&=\qty(i(0)-\qty(\frac{1}{2}i(0)R+v_C(0))\frac{2}{R}\omega_0t)e^{-\omega_0t}\\
&=(i(0)-\qty(i(0)+\frac{2v_C(0)}{R})\omega_0t)e^{-\omega_0t}
\ea\]
\[v_R=\qty(i(0)R-\qty(i(0)R+2v_C(0))\omega_0t)e^{-\omega_0t}\]
\[v_L=-\qty(i(0)R+v_C(0)+\qty(\frac{1}{2}i(0)R+v_C(0))\omega_0t)e^{-\omega_0t}\]
\end{proof}
\item $0<\zeta<1$ Underdamped response (欠阻尼響應):




\begin{proof}
\[i=e^{-\omega_0\zeta t}\qty(A\cos\qty(\omega_0\sqrt{1-\zeta^2}t)+B\sin\qty(\omega_0\sqrt{1-\zeta^2}t))\]
\[A=i(0)\]
\[i=e^{-\omega_0\zeta t}\qty(i(0)\cos\qty(\omega_0\sqrt{1-\zeta^2}t)+B\sin\qty(\omega_0\sqrt{1-\zeta^2}t))\]
\[v_R=Re^{-\omega_0\zeta t}\qty(i(0)\cos\qty(\omega_0\sqrt{1-\zeta^2}t)+B\sin\qty(\omega_0\sqrt{1-\zeta^2}t))\]
\[\ba
v_L&=L\dv{i}{t}\\
&=-\frac{R}{2\zeta}\zeta e^{-\omega_0\zeta t}\qty(i(0)\cos\qty(\omega_0\sqrt{1-\zeta^2}t)+B\sin\qty(\omega_0\sqrt{1-\zeta^2}t))+\frac{R}{2\zeta}\sqrt{1-\zeta^2}e^{-\omega_0\zeta t}\qty(-i(0)\sin\qty(\omega_0\sqrt{1-\zeta^2}t)+B\cos\qty(\omega_0\sqrt{1-\zeta^2}t))\\
&=\frac{R}{2\zeta}e^{-\omega_0\zeta t}\qty(-i(0)\zeta\cos\qty(\omega_0\sqrt{1-\zeta^2}t)-B\zeta\sin\qty(\omega_0\sqrt{1-\zeta^2}t)-i(0)\sqrt{1-\zeta^2}\sin\qty(\omega_0\sqrt{1-\zeta^2}t)+B\sqrt{1-\zeta^2}\cos\qty(\omega_0\sqrt{1-\zeta^2}t))\\
&=\frac{R}{2\zeta}e^{-\omega_0\zeta t}\qty(\qty(B\sqrt{1-\zeta^2}-i(0)\zeta)\cos\qty(\omega_0\sqrt{1-\zeta^2}t)-\qty(B\zeta+i(0)\sqrt{1-\zeta^2})\sin\qty(\omega_0\sqrt{1-\zeta^2}t))
\ea\]
\[\ba
v_C&=-v_R-v_L\\
&=\frac{R}{2\zeta}e^{-\omega_0\zeta t}\qty(-2i(0)\zeta\cos\qty(\omega_0\sqrt{1-\zeta^2}t)-2B\zeta\sin\qty(\omega_0\sqrt{1-\zeta^2}t)-\qty(B\sqrt{1-\zeta^2}-i(0)\zeta)\cos\qty(\omega_0\sqrt{1-\zeta^2}t)+\qty(B\zeta+i(0)\sqrt{1-\zeta^2})\sin\qty(\omega_0\sqrt{1-\zeta^2}t))\\
&=\frac{R}{2\zeta}e^{-\omega_0\zeta t}\qty(-\qty(B\sqrt{1-\zeta^2}+i(0)\zeta)\cos\qty(\omega_0\sqrt{1-\zeta^2}t)+\qty(-B\zeta+i(0)\sqrt{1-\zeta^2})\sin\qty(\omega_0\sqrt{1-\zeta^2}t))
\ea\]
\[v_C(0)=-\frac{R}{2\zeta}\qty(B\sqrt{1-\zeta^2}+i(0)\zeta)\]
\[B=-\qty(\frac{2v_C(0)}{R}+i(0))\frac{\zeta}{\sqrt{1-\zeta^2}}\]
\[i=e^{-\omega_0\zeta t}\qty(i(0)\cos\qty(\omega_0\sqrt{1-\zeta^2}t)-\qty(\frac{2v_C(0)}{R}+i(0))\frac{\zeta}{\sqrt{1-\zeta^2}}\sin\qty(\omega_0\sqrt{1-\zeta^2}t))\]
\[\ba
v_C&=\frac{R}{2\zeta}e^{-\omega_0\zeta t}\qty(\qty(\qty(\frac{2v_C(0)}{R}+i(0))\frac{\zeta}{\sqrt{1-\zeta^2}}\sqrt{1-\zeta^2}-i(0)\zeta)\cos\qty(\omega_0\sqrt{1-\zeta^2}t)+\qty(\qty(\frac{2v_C(0)}{R}+i(0))\frac{\zeta}{\sqrt{1-\zeta^2}}\zeta+i(0)\sqrt{1-\zeta^2})\sin\qty(\omega_0\sqrt{1-\zeta^2}t))\\
&=\frac{R}{2\zeta}e^{-\omega_0\zeta t}\qty(\frac{2v_C(0)}{R}\zeta\cos\qty(\omega_0\sqrt{1-\zeta^2}t)+\qty(\qty(\frac{2v_C(0)}{R}+i(0))\zeta^2+i(0)\qty(1-\zeta^2))\frac{1}{\sqrt{1-\zeta^2}}\sin\qty(\omega_0\sqrt{1-\zeta^2}t))\\





Check $i=C\dv{v_C}{t}$:
\[\ba
i&=C\dv{v_C}{t}\\








\[\omega_0=\frac{1}{\sqrt{LC}}\]
\[\zeta=\frac{R}{2}\sqrt{\frac{C}{L}}\]
\[L\omega_0=\frac{R}{2\zeta}\]
\[C\omega_0=\frac{2\zeta}{R}\]

\end{proof}
\eit













Parallel
$\alpha=\frac{1}{2RC}$
$\omega_0=\frac{1}{\sqrt{LC}}$
\[\zeta=\frac{\alpha}{\omega_0}=\frac{1}{2R}\sqrt{\frac{L}{C}}\]
\[s_1=-\omega_0\zeta+\omega_0\sqrt{\zeta^2-1}\]
\[s_2=-\omega_0\zeta-\omega_0\sqrt{\zeta^2-1}\]




\ssc{Operational Amplifier (op amp or op-amp)}
\sssc{Definition}
An op amp has five terminals:
\bit
\item $v_+$: non-inverting input voltage of the terminal at the $+$ sign;
\item $v_-$: inverting input voltage of the terminal at the $-$ sign;
\item $v_{out}$: output voltage of the terminal at the opposite side of inputs;
\item $v_{CC+}$: positive supply voltage, denoted on the $+$ side of an op-amp or omitted;
\item $v_{CC-}$: negative supply voltage, denoted on the $-$ side of an op-amp or omitted.
\eit
\begin{center}
\begin{circuitikz}
\draw (0,0) node[op amp] (op amp) {};
\end{circuitikz}
\end{center}
\[v_{out}=A(v_+-v_--v_{os}),\]
where $A$ is open-loop gain, and $v_{os}$ is input offset voltage.
\sssc{Ideal op amp}
\bit
\item Infinite open-loop gain, infinite bandwidth, and zero output impedance: $A=\infty$ and is independent of frequency and output voltage.
\item Zero input offset voltage: $v_{os}=0$.
\item No current flow into either of the input terminals.
\eit
\sssc{Inverting amplifier}
An inverting amplifier consists of an op amp, an input resistor $R_i$, and a feedback resistor $R_f$, where input voltage $v_{in}$ is connected to $v_-$ through $R_i$, $v_+$ is connected to ground, and $v_{out}$ is fed back to $v_-$ through $R_f$.
\[i_{in}=\frac{v_{in}}{R_{in}}=i_f=\frac{-v_{out}}{R_f},\]
\[v_{out}=-\frac{R_f}{R_{in}}v_{in}.\]
\sssc{Noninverting amplifier}
A noninverting amplifier consist of an op amp, a ground resistor $R_g$, and a feedback resistor $R_f$, where input voltage $v_{in}$ is connected to $v_+$, $v_-$ is connected ground through $R_g$, and $v_{out}$ is fed back to $v_-$ through $R_f$.
\[v_{in}=v_+=v_-=v_{out}\frac{R_g}{R_f+R_g},\]
\[v_{out}=\qty(1+\frac{R_f}{R_g})v_{in}.\]
\sssc{Summing amplifier}
A summing amplifier consists of an op amp, input voltages $v_1,v_2,\ldots,v_n$, $n$ input resistors $R_1,R_2,\ldots,R_n$, and a feedback resistor $R_f$, where for any $k\in\bbN\land k\leq n$, $v_k$ is connected to $v_-$ through $R_k$, $v_+$ is connected to ground, and $v_{out}$ is fed back to $v_-$ through $R_f$.
\[i_k=\frac{v_k}{R_k},\]
\[\sum_{k=1}^ni_k=i_f=\frac{-v_{out}}{R_f},\]
\[v_{out}=-R_f\sum_{k=1}^n\frac{v_k}{R_k}.\]
When $R_k=R$ for all $k\in\bbN\land k\leq n$:
\[v_{out}=-\frac{R_f}{R}\sum_{k=1}^nv_k.\]
\sssc{Differential amplifier}
A differential amplifier consists of a minuend $v_1$, a subtrahend $v_2$, and four resistors $R_1,R_2,R_3,R_4$ satisfying the condition
\[\frac{R_1}{R_2}=\frac{R_3}{R_4},\]
where $v_1,v_2$ are connected to $v_-,v_+$ through $R_1,R_2$ respectively, $v_{out}$ is fed back to $v_-$ through $R_3$, and $v_+$ is connected to ground through $R_4$.
\[v_+=v_2\frac{R_4}{R_2+R_4},\]
\[\frac{v_1-v_-}{R_1}=\frac{v_--v_{out}}{R_3},\]
\[\frac{v_1}{R_2}-v_2\frac{R_4}{R_2+R_4}\frac{R_2+R_4}{R_2R_4}+\frac{v_{out}}{R_4}=0,\]
\[v_{out}=\frac{R_4}{R_2}\qty(v_2-v_1).\]
When $R_1=R_2=R_3=R_4$:
\[v_{out}=(v_2-v_1).\]







\subsubsection{Switches}
\begin{center}
\begin{circuitikz}
\draw (0,0) to[switch, l=Switch] (3,0);
\draw (5,0) node[spdt]{} (5,0.5) node[above]{SPDT Switch};
\end{circuitikz}
\end{center}
\subsubsection{Diodes}
\begin{center}
\begin{circuitikz}
\draw (0,0) to[D, l=Diode] (3,0);
\draw (5,0) to[zDo, l=Zener Diode] (8,0);
\draw (10,0) to[pDo, l=Photodiode] (13,0);
\draw (15,0) to[led, l=LED] (18,0);
\end{circuitikz}
\end{center}
\subsubsection{Transistors}
\begin{center}
\begin{circuitikz}
\draw (0,0) node[npn]{} (0,1) node[above]{NPN Transistor};
\draw (5,0) node[pnp]{} (5,1) node[above]{PNP Transistor};
\draw (10,0) node[nmos]{} (10,1) node[above]{N-MOSFET};
\draw (15,0) node[pmos]{} (15,1) node[above]{P-MOSFET};
\end{circuitikz}
\end{center}
\subsubsection{Measuring Instruments}
\begin{center}
\begin{circuitikz}
\draw (0,0) to[voltmeter, l=Voltmeter] (3,0);
\draw (5,0) to[ammeter, l=Ammeter] (8,0);
\draw (10,0) to[ohmmeter, l=Ohmmeter] (13,0);
\end{circuitikz}
\end{center}
\subsubsection{Transformers}
\begin{center}
\begin{circuitikz}
\draw (0,0) node[transformer]{Transformer} (3,0);
\draw (5,0) node[transformer core]{Transformer Core} (8,0);
\end{circuitikz}
\end{center}
\end{document}
