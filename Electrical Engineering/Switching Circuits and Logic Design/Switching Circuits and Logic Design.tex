\documentclass[a4paper,12pt]{article}
\setcounter{secnumdepth}{5}
\setcounter{tocdepth}{3}
\input{/usr/share/latex-toolkit/template.tex}
\begin{document}
\title{Switching Circuits and Logic Design}
\author{沈威宇}
\date{\temtoday}
\titletocdoc
\section{Digital Systems and Switching Circuits}
\begin{itemize}
\item\textbf{Digital system}: Deals with signals that have discrete values. Theoretically, for a given input, the output is exactly correct.
\item\textbf{Analog system}: Deals with signals that vary continuously over time. The output might have an error depending on the accuracy of the components used.
\item\textbf{System design}: The highest level of the design of digital systems, where you break the system into subsystems, specify what each subsystem do, and determine the interconnection and control of the subsystems.
\item\textbf{Logic design}: The middle level of the design of digital systems, where you specify the logic operations inside each subsystem.
\item\textbf{Circuit design}: The lowest level of the design of digital systems, where you specify the electronic components and their interconnection to form the system.
\item\textbf{Switching circuit}: A switching circuit has one or more inputs and one or more outputs which take on discrete values. Many of a digital system’s subsystems take the form of a switching circuit.
\item\textbf{Combinational circuit}: A circuit whose output value depends only on the present input value.
\item\textbf{Sequential circuit}: A circuit whose output value depends on the present input value and past input values.
\item\textbf{Logic gate}: An electronic device that performs a basic Boolean operation on one or more binary inputs (0 or 1) to produce a single binary output (0 or 1).
\end{itemize}
