\documentclass[a4paper,12pt]{article}
\setcounter{secnumdepth}{5}
\setcounter{tocdepth}{3}
\input{/usr/share/LaTeX-ToolKit/template.tex}
\renewcommand{\arraystretch}{1.5}
\begin{document}
\title{Switching Circuit and Logic Design}
\author{沈威宇}
\date{\temtoday}
\titletocdoc
\sct{Switching Circuit and Logic Design}
\ssc{Introduction to Switching Circuit}
\begin{itemize}
\item\textbf{Digital system}: Deals with signals that have discrete values. Theoretically, for a given input, the output is exactly correct.
\item\textbf{Digital circuit (ckt) or logic circuit}: A circuit that deals with binary input and output signals, that is, either 0 or 1.
\item\textbf{Analog system}: Deals with signals that vary continuously over time. The output might have an error depending on the accuracy of the components used.
\item\textbf{System design}: The highest level of the design of systems, where you break the system into subsystems, specify what each subsystem do, and determine the interconnection and control of the subsystems.
\item\textbf{(Digital) logic design}: The middle level of the design of digital systems, where you specify the logic operations inside each subsystem.
\item\textbf{Circuit design}: The lowest level of the design of digital systems, where you specify the electronic components and their interconnection to form the system.
\item\textbf{Switching (logic) circuit}: A switching circuit has one or more inputs and one or more outputs which take on discrete values. Many of a digital system's subsystems take the form of a switching circuit.
\item\textbf{Combinational (logic) circuit}: A circuit whose output value depends only on the present input value.
\item\textbf{Sequential (logic) circuit}: A circuit whose output value depends on the present input value and past input values.
\item\textbf{Switch, switching device, or switch element}: A component that opens or closes a circuit.
\item\textbf{Flip-flop}: A bistable electronic device, meaning it has two stable states representing logic 0 and 1. It's a basic memory element used in sequential circuits.
\end{itemize}
\subsection{Introduction to Boolean Algebra}
The basic mathematics needed for the study of logic design of digital systems is Boolean algebra. Boolean algebra was introduced by George Boole in his first book The Mathematical Analysis of Logic (1847), and set forth more fully in his An Investigation of the Laws of Thought (1854).

Switching devices are essentially two-state devices (e.g. switches which are open or closed and transistors with high or low output voltages). Consequently, we will emphasize the special case of Boolean algebra in which all of the variables assume only one of two values, 0 (false) or 1 (true), called Boolean variables; this two-valued Boolean algebra is also called switching algebra.

In a switch circuit, 0 (usually) represents an open switch, and 1 represents a closed circuit. In general, 0 and 1 can be used to represent the two states in any binary-valued system.

A Boolean function is a logical operation performed on one or more binary inputs that produces a single binary output.
\subsection{Logical Operators}
\sssc{NOT / Negation / Inversion / Complement}
\begin{itemize}
\item Symbol: $A'$, $\neg A$, $\mathord{\sim}A$, or $\ol{A}$
\item Definition: One input, one output. True if the input is false and false otherwise.
\eit 
\sssc{AND / Conjunction / Logical Multiplication}
\begin{itemize}
\item Symbol: $A \cdot B$, $A*B$, or $AB$
\item Definition: Multiple inputs, one output. True if all inputs are true and false otherwise.
\eit
\sssc{OR / Disjunction / Logical Addition / Inclusive OR}
\begin{itemize}
\item Symbol: $A + B$
\item Definition: Multiple inputs, one output. True if at least one input is true and false otherwise.
\eit
\sssc{NAND / Not AND}
Definition: Multiple inputs, one output. True if all inputs are false and false otherwise.
\sssc{NOR / Not OR}
Definition: Multiple inputs, one output. True if at least one input is false and false otherwise.
\sssc{XOR / Exclusive OR}
\begin{itemize}
\item Symbol: $A \oplus B$
\item Definition: Multiple inputs, one output. True if odd number of inputs are true and false otherwise.
\eit
\sssc{XNOR / Logical equivalence / Exclusive NOR}
\begin{itemize}
\item Symbol: $A\equiv B$ or $A\odot B$
\item Definition: Multiple inputs, one output. True if even number (including 0) of inputs are true and false otherwise.
\eit
\sssc{IMPLY / Logical conditional}
\begin{itemize}
\item Symbol: $A \rightarrow B$
\item Definition: Two inputs, one output. $A\rightarrow B$ is defined as $A'+B$.
\eit
\sssc{NIMPLY / Material nonimplication}
\begin{itemize}
\item Symbol: $A\nrightarrow B$
\item Definition: Two inputs, one output. $A\nrightarrow B$ is defined as $AB'$.
\eit
\ssc{Boolean Arithmetic}
\sssc{Boolean expression}
A Boolean expression is formed by application of the logic operations to one or more Boolean variables or constants. 

\bit
\item\tb{Literals}: The simplest expressions, consisting of a single constant or variable or its complement, such as $0$, $X$, or $Y′$ .
\item\tb{Product terms}: A product term is a product of literals or a literal.
\item\tb{Sum terms}: A sum term is a sum of literals or a literal.
\eit

In a Boolean expression, parentheses are added as needed to specify the order in which the operations are performed; without parentheses, complementation is performed first, and then AND, and then OR; and the precedence of XOR and XNOR are either right higher to OR or right lower to OR depending on the context.
\sssc{Boolean function}
A Boolean function or a switching function is a function of which the domain is $\{1,0\}^n$ in which $n\in\mathbb{N}$ and the codomain is $\{1,0\}$.

The ON-set of a Boolean function $f$ is the set of all input combinations such that $f$ of them is $1$. Each element in the ON-set of $f$ corresponds to a row in the truth table of $f$ in which the output is $1$. The OFF-set of a Boolean function $f$ is the set of all input combinations such that $f$ of them is $0$. Each element in the OFF-set of $f$ corresponds to a row in the truth table of $f$ in which the output is $0$.
\sssc{Truth table}
A truth table, also called a table of combinations, specifies the corresponding output values for all possible combinations of input values for a Boolean function in the order such that if two input combinations have the same inputs in the first $i$ variables and $A_{i+1}$ of the first one is $0$, $A_{i+1}$ of the second one is $1$, then the first is put prior to the second. A truth table for an $n$-variable Boolean function $f\qty(A_1,A_2,\ldots A_n)$ have $2^n$ rows and is:
\begin{longtable}[c]{|m|m|m|m|m|}
\hline
A_1 & A_2 & \ldots & A_n & f\\\hline
0 & 0 & \ldots & 0 & f\qty(0,0,\ldots 0)\\\hline
0 & 0 & \ldots & 1 & f\qty(0,0,\ldots 1)\\\hline
\vdots & \vdots & \vdots & \ddots & \vdots\\\hline
1 & 1 & \ldots & 1 & f\qty(1,1,\ldots 1)\\\hline
\end{longtable}
\sssc{Sum-of-products (SOP) form}
An expression is said to be in SOP form if it consists of a sum (OR) of product (AND) or single variable terms, in which it is called to be degenerate when some of the terms are single variables.

Every Boolean expression can be expressed in SOP form. And a Boolean expression may have more than one SOP forms.
\sssc{Product-of-sums (POS) form}
An expression is said to be in POS form if it consists of a product (AND) of sum (OR) or single variable terms, in which it is called to be degenerate when some of the terms are single variables.

Every Boolean expression can be expressed in POS form. And a Boolean expression may have more than one POS forms.
\sssc{Dual}
The dual of an Boolean expression or equation is another Boolean expression or equation obtained by simplifing the original expression or equation such that it only consists of literals, AND, and OR, and then replacing all AND in it with OR, all OR in it with AND, and all constants in it with their complements.
\sssc{Functionally complete}
A functionally complete set is a set of logic operators and constant sources (0 or 1) that can express all possible Boolean functions of any number of input variables. A minimal functionally complete set is a functionally complete set such that removing any one element from it makes it no longer functionally complete. Some minimal functionally complete sets are $\{\tx{AND},\tx{NOT}\}$, $\{\tx{OR},\tx{NOT}\}$, $\{\tx{XOR},\tx{NOT}\}$, $\{\tx{XNOR},\tx{NOT}\}$, $\{\tx{NAND}\}$, $\{\tx{NOR}\}$, $\{\tx{IMPLY},\tx{0}\}$, and $\{\tx{NIMPLY},\tx{1}\}$, in which NAND and NOR are called universal.
\sssc{Boolean Vector Functions or Boolean Multi-output Functions}
A Boolean vector function or a Boolean multi-output function is a function $f\colon\{1,0\}^n\to\{1,0\}^m$. A Boolean vector function can be specified with truth table or as a tuple of $m$ Boolean functions.
\ssc{Logic gate}
A logic gate is an electronic device that performs a Boolean function.

Logic gates symbols (inputs at left, first input named A, second input named B):
\begin{longtable}[c]{|p{0.2\tw}|p{0.2\tw}|p{0.2\tw}|p{0.2\tw}|}
\hline
Type & Distinctive shape (ANSI/IEEE Std 91/91a-1991 & Rectangular shape (IEEE Std 91/91a-1991/IEC 60617-12:1997) & Boolean function \\\hline\endhead
(Logic) buffer (gate) & \cktus{buffer gate}{n} & \cktiec{buffer gate}{n} & $A$ \\\hline
NOT gate / inverter & \cktus{not gate}{n} & \cktiec{not gate}{n} & $A'$ \\\hline
AND gate & \cktus{and gate}{nn} & \cktiec{and gate}{nn} & $AB$ \\\hline 
OR gate & \cktus{or gate}{nn} & \cktiec{or gate}{nn} & $A+B$ \\\hline
NAND gate & \cktus{nand gate}{nn} & \cktiec{nand gate}{nn} & $(AB)'$ \\\hline
NOR gate & \cktus{nor gate}{nn} & \cktiec{nor gate}{nn} & $(A+B)'$ \\\hline
XOR gate & \cktus{xor gate}{nn} & \cktiec{xor gate}{nn} & $A\oplus B$ \\\hline
XNOR gate & \cktus{xnor gate}{nn} & \cktiec{xnor gate}{nn} & $A\odot B$ \\\hline
IMPLY gate & \cktus{or gate}{in} & \cktiec{or gate}{in} & $A\rightarrow B$ \\\hline
NIMPLY gate & \cktus{nor gate}{in} & \cktiec{nor gate}{in} & $A\nrightarrow B$ \\\hline
\end{longtable}

An empty circle, called inversion bubble or bubble, means inverting the input before inputting to the gate when at a input to a gate, and means inverting the output before outputting from the gate when at the output from a gate.

By using inversion bubbles at the inputs instead of the output in logic gates symbol except IMPLY and NIMPLY gates, using inversion bubbles at the output and second input instead of the first input in IMPLY gate symbol, and using inversion bubbles at the second input instead of the first input and the output in NIMPLY gate symbol, we obtain the alternative gate symbols:
\begin{longtable}[c]{|p{0.2\tw}|p{0.2\tw}|p{0.2\tw}|p{0.2\tw}|}
\hline
Type & Distinctive shape (ANSI/IEEE Std 91/91a-1991 & Rectangular shape (IEEE Std 91/91a-1991/IEC 60617-12:1997) & Boolean function \\\hline\endhead
(Logic) buffer (gate) & \cktus{not gate}{i} & \cktiec{not gate}{i} & $A$ \\\hline
NOT gate / inverter & \cktus{buffer gate}{i} & \cktiec{buffer gate}{i} & $A'$ \\\hline
AND gate & \cktus{nor gate}{ii} & \cktiec{nor gate}{ii} & $AB$ \\\hline 
OR gate & \cktus{nand gate}{ii} & \cktiec{nand gate}{ii} & $A+B$ \\\hline
NAND gate & \cktus{or gate}{ii} & \cktiec{or gate}{ii} & $(AB)'$ \\\hline
NOR gate & \cktus{and gate}{ii} & \cktiec{and gate}{ii} & $(A+B)'$ \\\hline
XOR gate & \cktus{xnor gate}{ii} & \cktiec{xnor gate}{ii} & $A\oplus B$ \\\hline
XNOR gate & \cktus{xor gate}{ii} & \cktiec{xor gate}{ii} & $A\odot B$ \\\hline
IMPLY gate & \cktus{nand gate}{ni} & \cktiec{nand gate}{ni} & $A\rightarrow B$ \\\hline
NIMPLY gate & \cktus{and gate}{ni} & \cktiec{and gate}{ni} & $A\nrightarrow B$ \\\hline
\end{longtable}
\ssc{Theorems}
\sssc{Idempotent Laws}
\[XX = X,\quad X + X = X.\]
\sssc{Involution Law}
\[(X′)′ = X.\]
\sssc{Laws of Complementarity}
\[X X′ = 0,\quad X + X′ = 1.\]
\sssc{Commutativity of AND, OR, XOR, and XNOR}
\[X Y=Y X,\quad X+Y=Y+X,\]
\[X\oplus Y=Y\oplus X,\quad X\odot Y=Y\odot X.\]
\sssc{Associativity of AND, OR, XOR, and XNOR}
\[X Y Z=X (Y Z),\quad X+Y+Z=X+(Y+Z),\]
\[X\oplus Y\oplus Z=X\oplus (Y\oplus Z),\quad X\odot Y\odot Z=X\odot (Y\odot Z).\]
\sssc{Distributivity of AND over OR and OR over AND}
\[X (Y+Z)=X Y+X Z.\]
\[X+YZ=(X+Y)(X+Z).\]
\sssc{XOR of ANDs Theorem or Distributivity of AND over XOR}
\[X Y\oplus X Z=X (Y\oplus Z).\]
\begin{proof}
\[X Y\oplus X Z=XY(XZ)'+(XY)'XZ=XYX'+XYZ'+X'XZ+Y'XZ=XYZ'+XY'Z=X(Y\oplus Z).\]
\end{proof}
\sssc{XOR of ORs Theorem}
\[(X+Y)\oplus (X+Z)=X' (Y\oplus Z).\]
\begin{proof}
\[(X+Y)\oplus (X+Z)=(X+Y)(X+Z)'+(X+Y)'(X+Z)=XX'Z'+YX'Z'+X'Y'X+X'Y'Z=X'YZ'+X'Y'Z=X'(Y\oplus Z).\]
\end{proof}
\sssc{XNOR of ORs Theorem or Distributivity of OR over XNOR}
\[(X+Y)\odot (X+Z)=X+(Y\odot Z).\]
\begin{proof}
\[(X+Y)\odot (X+Z)=(X+Y)(X+Z)+(X+Y)'(X+Z)'=X+YZ+X'Y'Z'=X+YZ+Y'Z'=X+(Y\odot Z).\]
\end{proof}
\sssc{XNOR of ANDs Theorem}
\[XY\odot XZ=X'+(Y\odot Z).\]
\begin{proof}
\[XY\odot XZ=XYZ+(XY)'(XZ)'=XYZ+(X'+Y')(X'+Z')=XYZ+X'+Y'Z'=X'+(Y\odot Z).\]
\end{proof}
\sssc{DeMorgan's Laws}
\[(\sum_{i=1}^nX_i)′=\prod_{i=1}^nX_i',\quad (\prod_{i=1}^nX_i)'=\sum_{i=1}^nX_i',\]
\[(\bigoplus_{i=1}^nX_i)′=\bigodot_{i=1}^nX_i',\quad (\bigodot_{i=1}^nX_i)'=\bigoplus_{i=1}^nX_i'.\]
\sssc{Axiom of Equality}
For an one-to-one Boolean function $f$, a Boolean expression $A$ equals another Boolean expression $B$ if and only if $f(A)$ equals $f(B)$.
\sssc{Duality Principle}
A Boolean equation is an identity if and only if the dual of it is an identity.
\sssc{Uniting Theorems}
\[XY+XY'=X,\quad (X+Y)(X+Y')=X.\]
\sssc{Absorption Theorems}
\[X+XY=X,\quad X(X+Y)=X.\]
\sssc{Elimination Theorems}
\[X+X'Y=X+Y,\quad X(X'+Y)=XY.\]
\sssc{Consensus Theorems}
The consensus theorems involve eliminate one term from an expression in SOP or POS form, in which the eliminated term is called the consensus term.
\[XY+X′Z+YZ=XY+X′Z.\]
\begin{proof}
\[\begin{aligned}
XY+X′Z+YZ&=XY+X'Z+(X+X')YZ\\
&=XY+X'Z+XYZ+X'YZ\\
&=XY+X′Z.
\end{aligned}\]
\end{proof}
\[(X+Y)(X′+Z)(Y+Z)=(X+Y)(X′+Z).\]
\begin{proof}
\[\begin{aligned}
(X+Y)(X′+Z)(Y+Z)&=(X+Y)(X'+Z)(X+X')(Y+Z)\\
&=(X+Y)(X'+Z)(X+Y+Z)(X'+Y+Z)\\
&=(X+Y)(X′+Z).
\end{aligned}\]
\end{proof}
\sssc{Shannon's expansion theorem, Shannon decomposition, Boole's expansion theorem, or fundamental theorem of Boolean algebra}
The theorem states that
\[F=x\cdot F_x+x'\cdot F_{x'},\]
where $F$ is any $n$-variable Boolean function, $x$ is any independent variable of $F$, $F_x$ and $F_{x'}$, sometimes called the positive and negative Shannon cofactors, respectively, of $F$ with respect to $x$, are $(n-1)$-variable Boolean functions defined as $F$ with the $x$ set to $1$ and to $0$ respectively, and the RHS is called a Shannon's expansion or Boole's expansion of $F$.
\sssc{Combination of Distributivity and Consensus Theorem}
\[(X+Y)(X'+Z)=XZ+X'Y\]
\begin{proof}
\[(X+Y)(X'+Z)=0+XZ+X'Y+YZ=XZ+X'Y\]
\end{proof}
\sssc{XOR and XNOR Series Theorems}
\[\bigoplus_{i=1}^nX_i=\qty(\sum_{i=1}^nX_i)\mod 2.\]
\[\bigodot_{i=1}^nX_i=1-\qty(\sum_{i=1}^nX_i)\mod 2=\qty(1+\sum_{i=1}^nX_i)\mod 2.\]
\[\qty(\bigoplus_{i=1}^nX_i)'=\qty(\bigoplus_{i=1}^{j-1}X_i)\oplus\qty(X_j)'\oplus\qty(\bigoplus_{i=j+1}^nX_i),\quad\forall j\leq n\land j\in\mathbb{N},\,\forall n\text{\ s.t.\ }\frac{n}{2}\in\mathbb{N}.\]
\begin{proof}\mbox{}\\
Case $n=2$:
\[\ba
(X_1\oplus X_2)'&=(X_1X_2'+X_1'X_2)'=(X_1X_2')'(X_1'X_2)'\\
&=(X_1'+X_2)(X_1+X_2')=(X_1'X_2'+X_1X_2)\\
&=X_1'\oplus X_2=X_1\oplus X_2'
\ea\]
Prove by mathematical induction. Assume it holds for $n=k$ and $n=2$. We want to prove that it holds for $n=k+2$.
\[\ba
\qty(\bigoplus_{i=1}^{k+2})'&=\qty(\bigoplus_{i=1}^kX_i)'\oplus X_{k+1}\oplus X_{k+2}\\
&=\qty(\bigoplus_{i=1}^{j-1}X_i)\oplus\qty(X_j)'\oplus\qty(\bigoplus_{i=j+1}^kX_i)\oplus X_{k+1}\oplus X_{k+2},\quad \forall j\leq k\land j\in\mathbb{N}
\ea\]
\[\ba
\qty(\bigoplus_{i=1}^{k+2})'&=X_1\oplus X_2\oplus\qty(\bigoplus_{i=3}^{k+2}X_i)'\\
&=X_1\oplus X_2\oplus\qty(\bigoplus_{i=3}^{j-1}X_i)\oplus\qty(X_j)'\oplus\qty(\bigoplus_{i=j+1}^{k+2}X_i),\quad \forall 3\leq j\leq k+2\land j\in\mathbb{N}
\ea\]
\end{proof}
\[\bigoplus_{i=1}^nX_i=\qty(\bigodot_{i=1}^nX_i)',\quad\forall n\text{\ s.t.\ }\frac{n}{2}\in\mathbb{N}.\]
\begin{proof}\mbox{}\\
By the definitions, it holds for case $n=2$.

Prove by mathematical induction. Assume it holds for $n=k$ and $n=2$. We want to prove that it holds for $n=k+2$.
\[\ba
\bigoplus_{i=1}^{k+2}X_i&=\qty(\bigoplus_{i=1}^kX_i)\oplus X_{k+1}\oplus X_{k+2}\\
&=\qty(\bigodot_{i=1}^kX_i)'\oplus X_{k+1}\oplus X_{k+2}\\
&=\qty(\bigodot_{i=1}^{k+1}X_i)\oplus X_{k+2}\\
&=\qty(\bigodot_{i=1}^{k+2}X_i)'
\ea\]
\end{proof}
\[\bigoplus_{i=1}^nX_i=\bigodot_{i=1}^nX_i,\quad\forall n\text{\ s.t.\ }\frac{n-1}{2}\in\mathbb{N}.\]
\begin{proof}\mbox{}\\
By
\[\bigoplus_{i=1}^nX_i=\qty(\bigodot_{i=1}^nX_i)',\quad\forall n\text{\ s.t.\ }\frac{n}{2}\in\mathbb{N}.\]
For $n$ such that $\frac{n-1}{2}\in\mathbb{N}$,
\[\ba
\bigoplus_{i=1}^nX_i&=\bigoplus_{i=1}^{n-1}X_i\oplus X_n\\
&=\bigoplus_{i=1}^{n-1}X_i\oplus X_n\\
&=\qty(\bigodot_{i=1}^{n-1}X_i)'\oplus X_n\\
&=\bigodot_{i=1}^nX_i
\ea\]
\end{proof}
\sssc{Consensus of XOR Theorem}
\[X\oplus Y+X\oplus Z+Y\oplus Z=X\oplus Y+X\oplus Z=X\oplus Y+Y\oplus Z=X\oplus Z+Y\oplus Z=XY'+X'Z+YZ'=X'Y+XZ'+Y'Z\]
\begin{proof}
\[X\oplus Y+X\oplus Z=XY'+X'Y+XZ'+X'Z=XY'+X'Y+XZ'+X'Z+YZ'+Y'Z=X\oplus Y+X\oplus Z+Y\oplus Z=XY'+X'Z+YZ'=X'Y+XZ'+Y'Z\]
\end{proof}
\ssc{Minterm and Maxterm Expansions, Canonical Expansions, or Standard Expansion}
\sssc{Minterm expansion, canonical SOP, or standard SOP}
A minterm of a completely specified Boolean function $f\colon\{1,0\}^n\to\{1,0\}$, denoted as $m_i$ for the input combination in the $(i+1)$th row of the truth table of $f$ that is in the ON-set of $f$, is defined for any input combinations in the ON-set of $f$ as
\[m_i=\prod_{k=1}^ny_k,\]
in which $y_k$ is defined as the $k$th input variable, i.e. $x_k$, if the $k$th input in that input combination is $1$ and as the complement of the $k$th input variable, i.e. $(x_k)'$, if the $k$th input in that input combination is $0$.

Minterm expansion, canonical SOP, or standard SOP is an expression of a completely specified Boolean function as a sum of minterms of it. Let $S$ be the set of all integer $i$ such that the input combination in the $(i+1)$th row of the truth table of $f$ is in the ON-set of $f$. Then the minterm expansion, canonical SOP, or standard SOP of $f$ is
\[f=\sum_{i\in S}m_i,\]
also denoted as
\[f=\sum m(i\in S).\]

For a given completely specified Boolean function, there exists a unique minterm expansion of it.

For example, given a function $f(a,b,c)$ with truth table
\begin{longtable}[c]{|m|m|m|m|}
\hline
a & b & c & f\\\hline
0 & 0 & 0 & 0\\\hline
0 & 0 & 1 & 0\\\hline
0 & 1 & 0 & 1\\\hline
0 & 1 & 1 & 0\\\hline
1 & 0 & 0 & 0\\\hline
1 & 0 & 1 & 1\\\hline
1 & 1 & 0 & 1\\\hline
1 & 1 & 1 & 0\\\hline
\end{longtable}
, the minterm expansion of it is
\[f=m_2+m_5+m_6=\sum m(2,5,6).\]
\sssc{Maxterm expansion, canonical POS, or standard POS}
A maxterm of a completely specified Boolean function $f\colon\{1,0\}^n\to\{1,0\}$, denoted as $M_i$ for the input combination in the $(i+1)$th row of the truth table of $f$ that is in the OFF-set of $f$, is defined for any input combinations in the OFF-set of $f$ as
\[M_i=\sum_{k=1}^ny_k,\]
in which $y_k$ is defined as the $k$th input variable, i.e. $x_k$, if the $k$th input in that input combination is $0$ and as the complement of the $k$th input variable, i.e. $(x_k)'$, if the $k$th input in that input combination is $1$.

Maxterm expansion, canonical POS, or standard POS is an expression of a completely specified Boolean function as a product of maxterms of it. Let $T$ be the set of all integer $i$ such that the input combination in the $(i+1)$th row of the truth table of $f$ is in the OFF-set of $f$. Then the maxterm expansion, canonical POS, or standard POS of $f$ is
\[f=\prod_{i\in T}M_i,\]
also denoted as 
\[f=\prod M(i\in T).\]

For a given completely specified Boolean function, there exists a unique maxterm expansion of it.

For example, given a function $f(a,b,c)$ with truth table
\begin{longtable}[c]{|m|m|m|m|}
\hline
a & b & c & f\\\hline
0 & 0 & 0 & 0\\\hline
0 & 0 & 1 & 0\\\hline
0 & 1 & 0 & 1\\\hline
0 & 1 & 1 & 0\\\hline
1 & 0 & 0 & 0\\\hline
1 & 0 & 1 & 1\\\hline
1 & 1 & 0 & 1\\\hline
1 & 1 & 1 & 0\\\hline
\end{longtable}
, the maxterm expansion of it is
\[f=M_0M_1M_3M_4M_7=\prod M(0,1,3,4,7).\]
\sssc{Conversion between function and its complement}
Let $f$ be a completely specified Boolean function with minterm expansion
\[f=\sum m(i\in S),\]
and maxterm expansion
\[f=\prod M(i\in T),\]
where $S$ be the set of all integer $i$ such that the input combination in the $(i+1)$th row of the truth table of $f$ is in the ON-set of $f$, $T$ be the set of all integer $i$ such that the input combination in the $(i+1)$th row of the truth table of $f$ is in the OFF-set of $f$.

Then, the complement $f'$ of has minterm expansion 
\[f'=\sum m(i\in T),\]
and maxterm expansion
\[f'=\prod M(i\in S).\]
\ssc{Minimum form}
\sssc{Implicant or 1-term}
Given a Boolean function $f$ of $n$ variables, a product term $P$ is an implicant (aka 1-term) of $f$ iff for every combination of values of the $n$ variables for which $P = 1$, $f$ is also equal to $1$.
\sssc{Implicate or 0-term}
Given a Boolean function $f$ of $n$ variables, a sum term $P$ is an implicate (aka 0-term) of $f$ iff for every combination of values of the $n$ variables for which $P = 0$, $f$ is also equal to $0$.
\sssc{Prime implicant}
A prime implicant of a function $f$ is an implicant of $f$ which is no longer an implicant of $f$ if any literal is deleted from it.
\sssc{Prime implicate}
A prime implicate of a function $f$ is an implicate of $f$ which is no longer an implicate of $f$ if any literal is deleted from it.
\sssc{Essential prime implicant}
An essential prime implicant of a function $f$ is a prime implicant $P$ of $f$ such that there exists a minterm $m$ of $f$ such that $m$ implies $P$ and for any other prime implicant $Q$ of $f$, $m$ doesn't implies $Q$.
\sssc{Essential prime implicate}
An essential prime implicate of a function $f$ is a prime implicate $P$ of $f$ such that there exists a maxterm $M$ of $f$ such that $P$ implies $M$ and for any other prime implicate $Q$ of $f$, $Q$ doesn't implies $M$.
\sssc{Minimum SOP form or minimum sum (of prime implicants)}
A SOP form of a Boolean function is called minimum if it has the fewest number of terms out of all SOP forms of the function, and every product term in it can not have any variable in it be eliminated. For a given Boolean function, there may exist more than one minimum SOP forms of it.

A minimum SOP form of a Boolean function must consist of some of its prime implicants (but not necessarily all). If a SOP form contains implicants that are not prime implicants, it is not a minimum SOP form.

A minimum SOP form of a Boolean function must contain all of its essential prime implicants.
\sssc{Minimum POS form or minimum product (of prime implicates)}
A POS form of a Boolean function is called minimum if it has the fewest number of terms out of all POS forms of the function, and every sum term in it can not have any variable in it be eliminated. For a given Boolean function, there may exist more than one minimum POS forms of it.

A minimum POS form of a Boolean function must consist of some of its prime implicates (but not necessarily all). If a POS form contains implicates that are not prime implicates, it is not a minimum POS form.

A minimum POS form of a Boolean function must contain all of its essential prime implicates.

One can simplify a Boolean function to minimum POS form by:
\ben
\item taking dual of the expression,
\item simplifying it to minimum SOP form, and
\item taking dual of it.
\een
\ssc{Incompletely Specified Boolean Functions (ISF) or Don't-care Functions}
An incompletely specified function (ISF) or a don't-care function is a Boolean function but in which for some input combinations, the output is not defined or irrelevant, often denoted as $X$. Those input combinations are called don't-care conditions. The definition of ON-set and OFF-set of an ISF is the same as completely specified function; the Don't-Care set or DC-set of an ISF is the set of all don't-care conditions.

The output for the don't-care conditions can be assigned either $0$ or $1$ such that the minimum SOP (or POS) forms have the fewest terms.

An ISF $f$ with ON-set $S$, OFF-set $T$, and DC-set $D$ is sometimes written similar to minterm expansion as
\[f=\sum m(i\in S)+\sum d(i\in D),\]
and similar to maxterm expansion as
\[f=\prod M(i\in T)+\prod d(i\in D),\]
where each $d_i$ is called a don't-care terms.
\ssc{Karnaugh Maps}
A Karnaugh Map, aka a K-map, is a grid that visualize the ways to simplify a completely or incompletely specified Boolean function to a minimum SOP form.
\sssc{Draw Karnaugh Maps of less than five variables}
If the ouput of the input combination of a cell is $1$, we write $1$ in that cell; if the ouput of the input combination of a cell is $0$, we can leave that cell empty or write $0$ in that cell; if the input combination of a cell is a don't care condition, we write $X$ in that cell.

The order 00, 01, 11, 10 is the gray code order, which is such that adjacent cells in the K-map differ by only one variable.

The notation of a cell is the input combination of that cell (thus a binary number), called binary notation, or the decimal integer in BCD representated by the input combination of that cell, called decimal notation.

A K-map for a $2$-variable Boolean function $f(A,B)$ is
\begin{longtable}[c]{c|c|c|}
\multicolumn{1}{c}{\thead{\backslashbox{$B$}{$A$}}} & \multicolumn{1}{c}{\thead{0}} & \multicolumn{1}{c}{\thead{1}} \\\cline{2-3}
\multicolumn{1}{c|}{\thead{0}} & $f(0,0)$ & $f(1,0)$ \\\cline{2-3}
\multicolumn{1}{c|}{\thead{1}} & $f(0,1)$ & $f(1,1)$ \\\cline{2-3}
\end{longtable}

A K-map for a $3$-variable Boolean function $f(A,B,C)$ is
\begin{longtable}[c]{c|c|c|}
\multicolumn{1}{c}{\thead{\backslashbox{$BC$}{$A$}}} & \multicolumn{1}{c}{\thead{0}} & \multicolumn{1}{c}{\thead{1}} \\\cline{2-3}
\multicolumn{1}{c|}{\thead{00}} & $f(0,0,0)$ & $f(1,0,0)$ \\\cline{2-3}
\multicolumn{1}{c|}{\thead{01}} & $f(0,0,1)$ & $f(1,0,1)$ \\\cline{2-3}
\multicolumn{1}{c|}{\thead{11}} & $f(0,1,1)$ & $f(1,1,1)$ \\\cline{2-3}
\multicolumn{1}{c|}{\thead{10}} & $f(0,1,0)$ & $f(1,1,0)$ \\\cline{2-3}
\end{longtable}

A K map for a $4$-variable Boolean function $f(A,B,C,D)$ is
\begin{longtable}[c]{c|c|c|c|c|}
\multicolumn{1}{c}{\thead{\backslashbox{$CD$}{$AB$}}} & \multicolumn{1}{c}{\thead{00}} & \multicolumn{1}{c}{\thead{01}} & \multicolumn{1}{c}{\thead{11}} & \multicolumn{1}{c}{\thead{10}} \\\cline{2-5}
\multicolumn{1}{c|}{\thead{00}} & $f(0,0,0,0)$ & $f(0,1,0,0)$ & $f(1,1,0,0)$ & $f(1,0,0,0)$ \\\cline{2-5}
\multicolumn{1}{c|}{\thead{01}} & $f(0,0,0,1)$ & $f(0,1,0,1)$ & $f(1,1,0,1)$ & $f(1,0,0,1)$ \\\cline{2-5}
\multicolumn{1}{c|}{\thead{11}} & $f(0,0,1,1)$ & $f(0,1,1,1)$ & $f(1,1,1,1)$ & $f(1,0,1,1)$ \\\cline{2-5}
\multicolumn{1}{c|}{\thead{10}} & $f(0,0,1,0)$ & $f(0,1,1,0)$ & $f(1,1,1,0)$ & $f(1,0,1,0)$ \\\cline{2-5}
\end{longtable}
\sssc{Group 1's}
The first step of simplifying a Boolean function using K-maps is grouping 1's (minterms).

A group (or loop) is a rectangle of $2^m\times 2^n$ cells that are either $1$ (minterm) or $X$ (don't-care condition), in which $m,n\in\mathbb{N}_0$, the leftmost and rightmost columns are horizontally adjacent, and the top and bottom rows are vertically adjacent. Cells can be in multiple groups.

Each group represents an implicant of $f$ by the following rules:
\ben
\item Let the set of all input variables that are $1$ in all cells in the group be $O$.
\item Let the set of all input variables that are $0$ in all cells in the group be $Z$.
\item The implicant represented by that group is $\prod_{o\in O}o\prod_{z\in Z}z'$.
\een
\sssc{Select collections of groups}
An allowed collection of groups must follow the following rules:
\bit
\item Any $1$ must be in at least one group.
\item If any $1$ is only covered by one possible group, then that group must be chosen, that is, any group representing an essential prime implicant must be chosen.
\item If any group $g$ is completely covered by another group, $g$ must not be chosen, that is, a group chosen must represent a prime implicant.
\eit
Find the allowed collections of groups that contain the fewest groups. For each one of them, the sum of the implicants corresponding to the groups in it is a minimum SOP form of the given function.
\sssc{Five-variable Karnaugh Maps}
One way of drawing a five-variable K-map is by placing one four-variable K-map on top of another, that is, for a $5$-variable Boolean function $f(A,B,C,D,E)$, the K-map is
{\fontsize{8pt}{12pt}\selectfont
\begin{longtable}[c]{cc|c|c|c|c|}
& \multicolumn{1}{c}{\thead{\backslashbox{$BC$}{$DE$}}} & \multicolumn{1}{c}{\thead{00}} & \multicolumn{1}{c}{\thead{01}} & \multicolumn{1}{c}{\thead{11}} & \multicolumn{1}{c}{\thead{10}} \\\cline{3-6}
& \multicolumn{1}{c|}{\thead{00}} & \backslashbox{$f(1,0,0,0,0)$}{$f(0,0,0,0,0)$} & \backslashbox{$f(1,0,1,0,0)$}{$f(0,0,1,0,0)$} & \backslashbox{$f(1,1,1,0,0)$}{$f(0,1,1,0,0)$} & \backslashbox{$f(1,1,0,0,0)$}{$f(0,1,0,0,0)$} \\\cline{3-6}
$A$ & \multicolumn{1}{c|}{\thead{01}} & \backslashbox{$f(1,0,0,0,1)$}{$f(0,0,0,0,1)$} & \backslashbox{$f(1,0,1,0,1)$}{$f(0,0,1,0,1)$} & \backslashbox{$f(1,1,1,0,1)$}{$f(0,1,1,0,1)$} & \backslashbox{$f(1,1,0,0,1)$}{$f(0,1,0,0,1)$} \\\cline{3-6}
\backslashbox{$1$}{$0$} & \multicolumn{1}{c|}{\thead{11}} & \backslashbox{$f(1,0,0,1,1)$}{$f(0,0,0,1,1)$} & \backslashbox{$f(1,0,1,1,1)$}{$f(0,0,1,1,1)$} & \backslashbox{$f(1,1,1,1,1)$}{$f(0,1,1,1,1)$} & \backslashbox{$f(1,1,0,1,1)$}{$f(0,1,0,1,1)$} \\\cline{3-6}
& \multicolumn{1}{c|}{\thead{10}} & \backslashbox{$f(1,0,0,1,0)$}{$f(0,0,0,1,0)$} & \backslashbox{$f(1,0,1,1,0)$}{$f(0,0,1,1,0)$} & \backslashbox{$f(1,1,1,1,0)$}{$f(0,1,1,1,0)$} & \backslashbox{$f(1,1,0,1,0)$}{$f(0,1,0,1,0)$} \\\cline{3-6}
\end{longtable}}
in which the adjacency between cells in the same layer of four-variable K-map is the same as that in four-variable K-map; the upper and lower triangles in the same square are adjacent; and a group is a cuboid of $2^l\times 2^m\times 2^n$ cells that are either $1$ (minterm) or $X$ (don't-care condition), in which $l,m,n\in\mathbb{N}_0$.
\sssc{Finding minimum POS form}
We can use K-maps to find minimum POS form of a completely or incompletely specified Boolean function by grouping $0$s instead of $1$s ($0$s and $X$s are allowed in a group), in which each group corresponds to a sum term, and the product of the sum terms corresponding to the groups in an allowed (covering all 0's) collections that contain the fewest groups is a minimum POS form of the Boolean function.
\ssc{Map-entered variable (MEV) or Variable entrant map (VEM)}
Map-entered variable (MEV), aka variable entrant map (VEM), simplification is an extension of Karnaugh maps used to simplify Boolean functions with typically more than four variables by allowing some variables to be entered as symbols or expressions inside the K-map cells rather than expanding the map's dimensions.
\sssc{One MEV}
Suppose we have an $n$-variable (usually five-variable) completely or incompletely specified Boolean function $f$ with $E$ being one of the input variable of $f$ and being the MEV.
\ben
\item Draw a $(n-1)$-variable K-map for input variables except $E$. In each cell, let the $n-1$-variable input combination corresponding to the cell be $P$, where $f(P,E)$ denotes a function of $E$ where the variables except $E$ be their values in $P$:
\bit
\item If $f(P,1)=f(P,0)=X$, enter $X$.
\item If $f(P,1)=1$ and $f(P,0)=0$, enter $E$.
\item If $f(P,1)=0$ and $f(P,0)=1$, enter $E'$.
\item If $f(P,1)=f(P,0)=1$, enter $1$.
\item If $f(P,1)=1$ and $f(P,0)=X$, enter $1$.
\item If $f(P,1)=X$ and $f(P,0)=1$, enter $1$.
\item If $f(P,1)=0=f(P,0)=0$, leave the cell empty or enter $0$.
\item If $f(P,1)=0$ and $f(P,0)=X$, leave the cell empty or enter $0$.
\item If $f(P,1)=X$ and $f(P,0)=0$, leave the cell empty or enter $0$.
\eit
That is, let the set of products of $n-1$ literals that corresponding to a $1$ cell be $A_1$, the set of products of $n-1$ literals that corresponding to a $E$ cell be $A_E$, the set of products of $n-1$ literals that corresponding to a $E'$ cell be $A_e$, and the set of products of $n-1$ literals that corresponding to a $X$ cell be $A_X$, we write $f$ as:
\[f=\sum m(i\mid m_i\in A_1)+E\sum m(i\mid m_i\in A_E)+E'\sum m(i\mid m_i\in A_e)+\sum d(i\mid m_i\in A_X),\]
where $m_i$ is minterms in $(n-1)$-variable truth tables.
\item Group $1$s in same way as K-map ($1$s and $X$s are allowed in a group). Each group corresponding to a $(n-1)$-variable implicant.
\item Group $E$s in same way as K-map but treating $1$s as $X$s ($E$s, $1$s, and $X$s are allowed in a group). Each group corresponding to a $n$-variable implicant with $E$ be one of the literals.
\item Group $E'$s in same way as K-map but treating $1$s as $X$s ($E'$s, $1$s, and $X$s are allowed in a group). Each group corresponding to a $n$-variable implicant with $E'$ be one of the literals.
\item An allowed collection of groups must follow the following rules:
\bit
\item Any $1$, $E$, or $E'$ must be in at least one group.
\item If any $1$, $E$, or $E'$ is only covered by one possible group, then that group must be chosen, that is, any group representing an essential prime implicant must be chosen.
\item If any group $g$ is completely covered by another group, $g$ must not be chosen, that is, a group chosen must represent a prime implicant.
\eit
\item Find the allowed collections of groups that contain the fewest groups. For each one of them, the sum of the implicants corresponding to the groups in it is a minimum SOP form of the given function.
\een
\sssc{Multiple MEVs}
Suppose we have an $n$-variable completely or incompletely specified Boolean function $f$ in which there exist $m$ (usually $n-4$) variables $E_1,E_2,\ldots E_m$ chosen as MEVs such that given a $(n-m)$-variable input combination $P$ of the other variables, the set of all possible values of $f(P,Q,E_i)$ is a subset of $\{0,X\}$ or a subset of $\{1,X\}$, with $X$ denoting don't-care, for any fixed $(m-1)$-variable input combination $Q$ of the MEVs except $E_i$, where $f(P,Q,E_i)$ denotes a function of $E_i$ where the output when $E_i$ be $x\in\{1,0\}$, $f(P,Q,x)$ equals the output of $f$ when the variables except MEVs be their values in $P$, the MEVs except $E_i$ be their values in $Q$, and $E_i$ be $x$.
\ben
\item Draw a $(n-m)$-variable K-map for input variables except $E$ and $F$. In each cell, let the $(n-m)$-variable input combination corresponding to the cell be $P$, for any $i\in\mathbb{N}_{\leq m}$:
\bit
\item If $f(P,R)=X$ for any fixed $m$-variable input combination $Q$ of the MEVs, where $f(P,Q)$ denotes the output of $f$ when the variables except MEVs be their values in $P$ and the MEVs be their values in $Q$, enter $X$.
\item If the set of all possible values of $f(P,Q,1)$ is $\{1\}$ and the set of all possible values of $f(P,Q,0)$ is $\{0\}$, for any fixed $(m-1)$-variable input combination $Q$ of the MEVs except $E_i$, enter $E_i$.
\item If the set of all possible values of $f(P,Q,1)$ is $\{0\}$ and the set of all possible values of $f(P,Q,0)$ is $\{1\}$, for any fixed $(m-1)$-variable input combination $Q$ of the MEVs except $E_i$, enter $E_i'$.
\item If the set of all possible values of $f(P,Q,1)$ and the set of all possible values of $f(P,Q,0)$ are both $\{1\}$ for any fixed $(m-1)$-variable input combination $Q$ of the MEVs except $E_i$, enter $1$.
\item If the set of all possible values of $f(P,Q,1)$ is $\{1\}$ and the set of all possible values of $f(P,Q,0)$ is $\{X\}$, for any fixed $(m-1)$-variable input combination $Q$ of the MEVs except $E_i$, enter $1$.
\item If the set of all possible values of $f(P,Q,1)$ is $\{X\}$ and the set of all possible values of $f(P,Q,0)$ is $\{1\}$, for any fixed $(m-1)$-variable input combination $Q$ of the MEVs except $E_i$, enter $1$.
\item If the set of all possible values of $f(P,Q,1)$ and the set of all possible values of $f(P,Q,0)$ are both $\{0\}$ for any fixed $(m-1)$-variable input combination $Q$ of the MEVs except $E_i$, leave the cell empty or enter $0$.
\item If the set of all possible values of $f(P,Q,1)$ is $\{0\}$ and the set of all possible values of $f(P,Q,0)$ is $\{X\}$, for any fixed $(m-1)$-variable input combination $Q$ of the MEVs except $E_i$, leave the cell empty or enter $0$.
\item If the set of all possible values of $f(P,Q,1)$ is $\{X\}$ and the set of all possible values of $f(P,Q,0)$ is $\{0\}$, for any fixed $(m-1)$-variable input combination $Q$ of the MEVs except $E_i$, leave the cell empty or enter $0$.
\eit
That is, let the set of products of $n-m$ literals that corresponding to a $1$ cell be $A_1$, the set of products of $n-m$ literals that corresponding to a $E_i$ cell be $A_{E_i}$, the set of products of $n-m$ literals that corresponding to a $E_i'$ cell be $A_{e_i}$, and the set of products of $n-m$ literals that corresponding to a $X$ cell be $A_X$, we write $f$ as:
\[f=\sum m(i\mid m_i\in A_1)+\sum_{i=1}^mE_i\sum m(i\mid m_i\in A_{E_i})+\sum_{i=1}^mE_i'\sum m(i\mid m_i\in A_{e_i})+\sum d(i\mid m_i\in A_X),\]
where $m_i$ is minterms in $(n-m)$-variable truth tables.
\item Group $1$s in same way as K-map ($1$s and $X$s are allowed in a group). Each group corresponding to a $(n-1)$-variable implicant.
\item For each $i\in\bbN_{\leq m}$, group $E_i$s in same way as K-map but treating $1$s as $X$s ($E_i$s, $1$s, and $X$s are allowed in a group). Each group corresponding to a $(n-m+1)$-variable implicant with $E_i$ be one of the literals.
\item For each $i\in\bbN_{\leq m}$, group $E_i'$s in same way as K-map but treating $1$s as $X$s ($E_i'$s, $1$s, and $X$s are allowed in a group). Each group corresponding to a $(n-m+1)$-variable implicant with $E_i'$ be one of the literals.
\item An allowed collection of groups must follow the following rules:
\bit
\item Any $1$, $E_i$, or $E_i'$ must be in at least one group.
\item If any $1$, $E_i$, or $E_i'$ is only covered by one possible group, then that group must be chosen, that is, any group representing an essential prime implicant must be chosen.
\item If any group $g$ is completely covered by another group, $g$ must not be chosen, that is, a group chosen must represent a prime implicant.
\eit
\item Find the allowed collections of groups that contain the fewest groups. For each one of them, the sum of the implicants corresponding to the groups in it is a minimum SOP form of the given function.
\een
\ssc{Quine-McCluskey Method}
The Quine-McCluskey method (along with the Petrick's method) is a systematic procedure to find the minimum SOP forms of a completely or incompletely specified Boolean function. Below, we first discuss about completely specified Boolean function.
\sssc{Minterm expansion}
First, we convert the function minterm expansion.

Below, we will take
\[f(a,b,c,d)=\sum m(0,1,2,5,6,7,8,9,10,14)\]
for example.
\sssc{Group minterms}
Group minterms with the same number of $1$'s in the input combination together and list them. Let the group of minterms with $k$ $1$'s be Group $k$. Group $k$ and Group $(k+1)$ are called adjacent.

For $f$:
\begin{longtable}[c]{|c|m|mmmm|}
\hline
Group 0 & 0 & 0 & 0 & 0 & 0 \\\hline
Group 1 & 1 & 0 & 0 & 0 & 1 \\\cline{2-6}
& 2 & 0 & 0 & 1 & 0 \\\cline{2-6}
& 8 & 1 & 0 & 0 & 0 \\\hline
Group 2 & 5 & 0 & 1 & 0 & 1 \\\cline{2-6}
& 6 & 0 & 1 & 1 & 0 \\\cline{2-6}
& 9 & 1 & 0 & 0 & 1 \\\cline{2-6}
& 10 & 1 & 0 & 1 & 0 \\\hline
Group 3 & 7 & 0 & 1 & 1 & 1 \\\cline{2-6}
& 14 & 1 & 1 & 1 & 0 \\\hline
\end{longtable}
\sssc{Combine minterms}
Compare minterms from adjacent groups and combine minterms that differ in only one input bit by
\[XY+XY'=X,\]
where $X$ is a product of literals and $Y$ is a literal.

(Note: It is obvious that two terms not in adjacent groups can't be combined by $XY+XY'=X$.)

Create next list by writing combined terms with $X$ preserved and $Y$, the differing bit, by $-$ (don't care), and writing not-touched terms as is.

Repeat combinations and create next list until no further combining is possible. The terms in this last list are the prime implicants of $f$.

For $f$:

List 1:
\begin{longtable}[c]{|c|m|mmmm|}
\hline
Group 0 & 0 & 0 & 0 & 0 & 0 \\\hline
Group 1 & 1 & 0 & 0 & 0 & 1 \\\cline{2-6}
& 2 & 0 & 0 & 1 & 0 \\\cline{2-6}
& 8 & 1 & 0 & 0 & 0 \\\hline
Group 2 & 5 & 0 & 1 & 0 & 1 \\\cline{2-6}
& 6 & 0 & 1 & 1 & 0 \\\cline{2-6}
& 9 & 1 & 0 & 0 & 1 \\\cline{2-6}
& 10 & 1 & 0 & 1 & 0 \\\hline
Group 3 & 7 & 0 & 1 & 1 & 1 \\\cline{2-6}
& 14 & 1 & 1 & 1 & 0 \\\hline
\end{longtable}

List 2:
\begin{longtable}[c]{|c|m|mmmm|}
\hline
Group 0 & 0,1 & 0 & 0 & 0 & - \\\cline{2-6}
& 0,2 & 0 & 0 & - & 0 \\\cline{2-6}
& 0,8 & - & 0 & 0 & 0 \\\hline
Group 1 & 1,5 & 0 & - & 0 & 1 \\\cline{2-6}
& 1,9 & - & 0 & 0 & 1 \\\cline{2-6}
& 2,6 & 0 & - & 1 & 0 \\\cline{2-6}
& 2,10 & - & 0 & 1 & 0 \\\cline{2-6}
& 8,9 & 1 & 0 & 0 & - \\\cline{2-6}
& 8,10 & 1 & 0 & - & 0 \\\hline
Group 2 & 5,7 & 0 & 1 & - & 1 \\\cline{2-6}
& 6,7 & 0 & 1 & 1 & - \\\cline{2-6}
& 6,14 & - & 1 & 1 & 0 \\\cline{2-6}
& 10,14 & 1 & - & 1 & 0 \\\hline
\end{longtable}

List 3:
\begin{longtable}[c]{|c|m|mmmm|}
\hline
Group 0 & 0,1,8,9 & - & 0 & 0 & - \\\cline{2-6}
& 0,2,8,10 & - & 0 & - & 0 \\\hline
Group 1 & 1,5 & 0 & - & 0 & 1 \\\cline{2-6}
& 2,6,10,14 & - & - & 1 & 0 \\\hline
Group 2 & 5,7 & 0 & 1 & - & 1 \\\cline{2-6}
& 6,7 & 0 & 1 & 1 & - \\\hline
\end{longtable}
\sssc{Create prime implicant chart}
Create the prime implicant chart, a chart where the column are minterms and rows are prime implicants and each cell is marked $\times$ if the corresponding minterm is implied by the corresponding prime implicant.

For $f$:
\begin{longtable}[c]{m|mmmmmmmmmm}
\hline
& 0 & 1 & 2 & 5 & 6 & 7 & 8 & 9 & 10 & 14 \\\hline
(0,1,8,9) b'c' & \times & \times & & & & & \times & \times & & \\
(0,2,8,10) b'd' & \times & & \times & & & & \times & & \times & \\
(1,5) a'c'd & & \times & & \times & & & & & & \\
(2,6,10,14) cd' & & & \times & & \times & & & & \times & \times \\
(5,7) a'bd & & & & \times & & \times & & & & \\
(6,7) a'bc & & & & & \times & \times & & & & \\
\end{longtable}
\sssc{Select ssential prime implicants}
When a prime implicant is selected for inclusion in the minimum SOP form, the chart is reduced by deleting the corresponding row and the columns which correspond to the minterms covered by that prime implicant.

If a column has only one $\times$, the prime implicant of that row is an essential prime implicant.

Select all essential prime implicants.

For $f$: Select $(0,1,8,9) b'c'$ to cover $m_9$, and select $(2,6,10,14) cd'$ to cover $m_{14}$.
\sssc{Petrick's method}
Given the reduced chart, number all prime implicants as $P_1,P_2,\ldots$.

For $f$:
\[P_1=a'c'd,\quad P_2=a'bd,\quad P_3=a'bc.\]

For each remaining minterm $m_i$, construct a sum term of all prime implicants that covers it, and let $P$ be the product of them.

For $f$:
\[P=(P_1+P_2)(P_2+P_3).\]

Expand $P$ into canonical SOP form (with each $P_i$ considered a literal).

For $f$:
\[P=P_1P_2+P_2+P_2P_3.\]

Reduce $P$ by applying
\[X+XY=X\]
until no more such reduction is possible.

For $f$:
\[P=P_2.\]

Each term in reduced $P$ represents a solution, that is, a set of prime implicants which covers all minterms in the reduced chart. The sets with the least number of literals in it are the minimum SOP forms of $f$.

For $f$: $P_2$ is the only term in reduced $P$, thus the only minimum SOP form is 
\[f=P_2=b'c'+cd'+a'bd.\]
\sssc{Incompletely specified Boolean functions}
For incompletely specified Boolean functions, we do the following change to the process for completely specified Boolean functions:
\ben
\item Treat don't-care terms as minterms when grouping minterm.
\item Omit don't-care terms columns when creating the prime implicant chart.
\een
\ssc{Multi-level combinational circuits}
\sssc{Introduction}
The maximum number of gates cascaded in series between a circuit input and the output is referred to as the number of levels of gates. Thus, a function written in sum-of-products form or in product-of-sums form corresponds directly to a two-level gate circuit. As is usually the case in digital circuits where the gates are driven from flip-flop outputs, we will assume that all variables and their complements are available as circuit inputs. For this reason, we will not normally count inverters which are connected directly to input variables.

The level starting with the output gate is numbered 1.

Given multi-input logic gates $A_1,A_2,\ldots A_m$, $A_1-A_2-\ldots A_mi$ circuit means a $m$-level circuit composed of a level of $A_1$ gates followed by a level of $A_2$ gate followed by $\ldots$ an $A_m$ gate at the output.

Given multi-input logic gates $A_1,A_2,\ldots A_m$, circuit of $A_1,A_2,\ldots A_m$ gates or $A_1,A_2,\ldots A_m$-gates circuits implies no particular ordering of the gates.
\sssc{Design of circuits of AND and OR gates}
The number of levels in an AND-OR circuit can be increased by factoring the sum-of-products expression from which it was derived. Similarly, the number of levels in an OR-AND circuit can be increased by multiplying out some of the terms in the product-of-sums expression from which it was derived. Sometimes doing so will reduce the required number of gates and gate inputs, and thus reduce the cost of building the circuit. In many application, the number of gates which can be cascaded is limited by gate delays. When the input of a gate is switched, there is a finite time before the output changes. When several gates are cascaded, the time between an input change and the corresponding change in the circuit output may become excessive and slow down the operation of the digital system.

In general, to be sure of obtaining a minimum solution, one must find both the circuit with the AND-gate output and the one with the OR-gate output. If an expression for $f'$ has $n$ levels, the complement of that expression is an $n$-level expression for $f$. Therefore, to realize $f$ as an $n$-level circuit with an AND-gate output, one procedure is first to find an $n$-level expression for $f'$ with an OR operation at the output level and then complement the expression for $f'$.
\sssc{Degeneracy}
A $m$-level circuit is degenerate iff it can degenerate into less-than-$m$-level. A circuit is non-degenerate iff it's not degenerate.

There are $8$ degenerate forms, which are two-level circuits:
\bit
\item AND-AND, which can degenerate into one AND,
\item OR-OR, which can degenerate into one OR,
\item AND-NAND, which can degenerate into one NAND,
\item OR-NOR, which can degenerate into one NOR,
\item NAND-NOR, which can degenerate into one AND,
\item NOR-NAND, which can degenerate into one OR,
\item NAND-OR, which can degenerate into one NAND,
\item NOR-AND, which can degenerate into one NOR.
\eit

A circuit is degenerate iff any two-level subcircuit of it is of the $8$ degenerate form.

A non-deegenerate circuit is an implementation of a sum-of-product-of-sum-$\ldots$product or product-of-sum-of-product-$\ldots$sum form.
\sssc{Design of two-level circuits}
There are $16$ forms of two-level circuit consist of AND, OR, NAND, and NOR gates, $8$ of which are the degenerate forms, $4$ of which are implementation of SOP form, and $4$ of which are implementation of POS form.

The non-degenerate forms that implement SOP form are:
\bit
\item AND-OR, which can be directly derived from the SOP form,
\item NAND-NAND, which can be converted from AND-OR form by replacing all gates with NAND gates and inverting all literals that are inputs to level 1, which can be derived from AND-OR form by taking double negation and applying De Morgan's law on level 1,
\item OR-NAND, which can be converted from AND-OR form by replacing AND gates with OR gates and OR gate with NAND gate and inverting all literals, which can be derived from NAND-NAND form by applying De Morgan's law on level 2, or from NOR-OR form by applying De Morgan's law on level 1,
\item NOR-OR, which can be converted from AND-OR form by replacing AND gates with NOR gates and inverting all inputs to level 2, which can be derived from AND-OR form by applying De Morgan's law on level 2.
\eit

The non-degenerate forms that implement POS form are
\bit
\item OR-AND, which can be directly derived from the POS form,
\item NOR-NOR, which can be converted from OR-AND form by replacing all gates with NOR gates and inverting all literals that are inputs to level 1, which can be derived from OR-AND form by taking double negation and applying De Morgan's law on level 1,
\item AND-NOR, which can be converted from OR-AND form by replacing OR gates with AND gates and AND gate with NOR gate and inverting all literals, which can be derived from NOR-NOR form by applying De Morgan's law on level 2, or from NAND-AND form by applying De Morgan's law on level 1,
\item NAND-AND, which can be converted from OR-AND form by replacing OR gates with NAND gates and inverting all inputs to level 2, which can be derived from OR-AND form by applying De Morgan's law on level 2.
\eit

The procedure of designing a two-level circuit is
\ben
\item Simplify the Boolean function to be realized to SOP or POS form.
\item Design an AND-OR or OR-AND circuit.
\item Convert it to the wanted form.
\een
\sssc{Design of multi-level NAND- and NOR-gate circuits}
The procedure of designing a multi-level NAND-gate circuit is
\ben
\item Simplify the Boolean function to be realized to the wanted sum-of-product-of-sum-$\ldots$product form.
\item Design an AND-OR-AND-$\ldots$OR or OR-AND-OR$\ldots$AND-OR circuit.
\item Replace all gates with NAND gates and invert all literals that are inputs to levels of odd numbers.
\een
The procedure of designing a multi-level NOR-gate circuit is
\ben
\item Simplify the Boolean function to be realized to the wanted product-of-sum-of-product-$\ldots$sum form.
\item Design an OR-AND-OR$\ldots$AND or AND-OR-AND-$\ldots$OR-AND circuit.
\item Replace all gates with NOR gates and invert all literals that are inputs to levels of odd numbers.
\een
\sssc{Circuit conversion using inversion bubbles}
Adding inversion bubbles to both ends of any interconnection does not change the Boolean function realized by the circuit.

Therefore, the gate type at levels $i$ and $i+1$ can be changed by inserting inversion bubbles on both sides of all connections between levels $i+1$ and $i$. After this operation, the gates at level $i+1$ may appear with double inversion bubbles at output and can be cancelled together, the gates at level $i$ will appear in their alternate symbols (with input bubbles) and can be changed to the standard form. The changing of level $i+1$ after this operation is
\begin{longtable}[c]{|c|c|}
\hline
Original gate & New gate \\\hline\endhead
Buffer gate & Not gate \\\hline
Not gate & Buffer gate \\\hline
AND gate & NAND gate \\\hline
OR gate & NOR gate \\\hline
NAND gate & AND gate \\\hline
NOR gate & OR gate \\\hline
XOR gate & XNOR gate \\\hline
XNOR gate & XOR gate \\\hline
\end{longtable}
The changing of level $i$ after this operation is
\begin{longtable}[c]{|c|c|}
\hline
Original gate & New gate \\\hline\endhead
Buffer gate & Not gate \\\hline
Not gate & Buffer gate \\\hline
AND gate & NOR gate \\\hline
OR gate & NAND gate \\\hline
NAND gate & OR gate \\\hline
NOR gate & AND gate \\\hline
XOR gate & XNOR gate \\\hline
XNOR gate & XOR gate \\\hline
\end{longtable}
\sssc{Limited gate fan-ins}
In practical logic design problems, the maximum number of inputs on each gate (or the fan-in) is limited. Depending on the type of gates used, this limit may be two, three, four, eight, or some other number. If a realization of a circuit requires more gate inputs than allowed, factoring the logic expression to obtain a more-level realization is necessary.
\sssc{Buffer}
A gate output can only be connected to a limited number of other device inputs without degrading the performance of a digital system. A simple buffer may be used to increase the driving capability of a gate output.
\ssc{Multiple-output combinational circuits}
\sssc{Minimal cost realization}
Solution of digital design problems often requires the realization of several completely or incompletely specified Boolean functions of the same variables, that is, a completely or incompletely specified Boolean vector function. Although each function could be realized separately, the use of some gates in common between two or more functions sometimes leads to a more economical realization. Thus in realizing multiple-output circuits, the use of a minimum SOP for each function does not necessarily lead to a minimum cost realization for the circuit as a whole.

When designing multiple-output circuits, you should try to minimize the total number of gates required. If several solutions require the same number of gates, the one with the minimum number of gate inputs should be chosen.
\sssc{Karnaugh Maps}
\ben
\item Draw K-map for each output.
\item Group 1's in the same definition in K-maps for single Boolean function; however the collection of groups is the collection of all groups on all maps, in which groups including same cells on different maps are the same and corresponding to only one element in the collection.
\item An allowed collection of groups must follow the following rules:
\bit
\item Any $1$ must be in at least one group.
\item If any $1$ is only covered by one possible group, then that group must be choose. The product represented by the group is called an essential prime implicant of the Boolean vector function.
\item If any group $g$ is completely covered by another group, $g$ must not be chosen. The product represented by a group that is allowed to be chosen according to this rule is called an prime implicant of the Boolean vector function.
\eit
\item Find the allowed collections of groups that contain the fewest groups. For each one of them, the sum of the products corresponding to the groups in it derives a minimal-gate SOP realization of the Boolean vector function.
\item Among the realization(s), find the one with fewest literal inputs, that is the minimum cost realization of the Boolean vector function.
\een
\sssc{Design of multiple-output NAND- and NOR-gate circuits}
The procedure of designing a multiple-output NAND-gate circuit is
\ben
\item Simplify the Boolean vector function to be realized to the wanted sum-of-product-of-sum-$\ldots$product form.
\item Design an AND-OR-AND-$\ldots$OR or OR-AND-OR$\ldots$AND-OR circuit.
\item Replace all gates with NAND gates and invert all literals that are inputs to levels of odd numbers.
\een
The procedure of designing a multiple-output NOR-gate circuit is
\ben
\item Simplify the Boolean vector function to be realized to the wanted product-of-sum-of-product-$\ldots$sum form.
\item Design an OR-AND-OR$\ldots$AND or AND-OR-AND-$\ldots$OR-AND circuit.
\item Replace all gates with NOR gates and invert all literals that are inputs to levels of odd numbers.
\een
\ssc{Delays and Hazards}
\sssc{Propagation delays and timing diagram}
When the input to a logic gate is changed, the output will not change instantaneously but change after a finite delay. If the change in output is delayed by time $\varepsilon$ with respect to the input, we say that this gate has a propagation delay of $\varepsilon$. In practice, the propagation delay for a 0 to 1 output change may be different than the delay for a 1 to 0 change. Propagation delays for integrated circuit gates may be as short as a few nanoseconds, and in many cases these delays can be neglected. However, in the analysis of some types of sequential circuits, even short delays may be important.

Timing diagram shows various signals in the circuit as a function of time, often in waveform with time being the horizontal axis and each vertical block plots one signal with the same time scale. The vertical edges on the diagram where the signals change from 0 to 1 are called rising edges; the vertical edges on the diagram where the signals change from 1 to 0 are called falling edges.
\sssc{Inertial delays}
If a logic gate will not respond to an input change that is shorter than a certain minimum pulse width $\varepsilon$, we say that this gate has a inertial delay of $\varepsilon$. Quite often the inertial delay value is assumed to be the same as the propagation delay of the gate. In contrast, if a gate always responds to input changes (with a propagation delay), no matter how closely spaced the input changes may be, the gate is said to have an ideal or transport delay.
\sssc{Hazards}
When the input to a combinational circuit changes, unwanted switching transients may appear in the output. These transients occur when different paths from input to output have different propagation delays. If, in response to any single input change and for some combination of propagation delays, a circuit output may momentarily go to 0 when it should remain a constant 1, we say that the circuit has a (static) 1-hazard (or static-1 hazard). Similarly, if the output may momentarily go to 1 when it should remain a 0, we say that the circuit has a (static) 0-hazard (or static-0 hazard). If, when the output is supposed to change from 0 to 1 (or 1 to 0), the output may change three or more times, we say that the circuit has a dynamic hazard. In each case the steady-state output of the circuit is correct, but a switching transient appears at the circuit output when the input is changed.

Hazards in a two-level circuit:
\bit
\item Hazards in a two-level circuit realizing a SOP form can be detected using a Karnaugh map, in which if any two adjacent 1's are not covered by a same group corresponding to a product term, a 1-hazard exists for the transition between the two 1's, which corresponds to an eliminated consensus term. For an $n$-variable map, this transition occurs when one variable changes and the other $n-1$ variables are held constant. We can eliminate a hazard by adding a group (product term) that covers the two 1's.
\item Hazards in a two-level circuit realizing a POS form can be detected using a Karnaugh map, in which if any two adjacent 0's are not covered by a same group corresponding to a sum term, a 0-hazard exists for the transition between the two 0's, which corresponds to an eliminated consensus term. For an $n$-variable map, this transition occurs when one variable changes and the other $n-1$ variables are held constant. We can eliminate a hazard by adding a group (sum term) that covers the two 0's.
\eit

Hazards in a multi-level circuit can be detected by deriving either a SOP or POS expression for the circuit that represents a two-level circuit containing the same hazards as the original circuit. The SOP or POS expression is derived in the normal manner except that the complementation laws, i.e. $XX'=0$ and $X+X'=1$, are not used. Consequently:
\bit
\item The resulting SOP expression may contain products of the form $xx'\alpha$, where $\alpha$ is a product of literals or nothing. In the SOP expression, a product of the form $xx'\alpha$ represents a pseudo AND gate that may temporarily have the output value 1 as $x$ changes if $\alpha=1$, which causes static 0- or dynamic hazards.
\item The resulting POS expression may contain sums of the form $x+x'+\beta$, where $\beta$ is a sum of literals or nothing. In the POS expression, a sum of the form $x+x'+\beta$ represents a pseudo OR gate that may temporarily have the output value 0 as $x$ changes if $\beta=0$, which causes static 1- or dynamic hazards.
\eit

To design a circuit  which is free of static and dynamic hazards, the following procedures may be used:
\ben
\item Find a SOP expression for the function in which every pair of adjacent 1's is covered by a 1-term. (The sum of all prime implicants always satisfies this condition.) A two-level AND-OR circuit based on this expression will be free of 1-, 0-, and dynamic hazards.
\item If a different form of the circuit is desired, manipulate the expression to the desired form but treat each independent variable and its inverse as two independent variables to prevent introduction of hazards.
\een
or
\ben
\item Find a POS expression for the function in which every pair of adjacent 0's is covered by a 0-term. (The product of all prime implicate always satisfies this condition.) A two-level OR-AND circuit based on this expression will be free of 1-, 0-, and dynamic hazards.
\item If a different form of the circuit is desired, manipulate the expression to the desired form but treat each independent variable and its inverse as two independent variables to prevent introduction of hazards.
\een

It should be emphasized that the discussion of hazards and the possibility of resulting glitches in this section has assumed that only a single input can change at a time and that no other input will change until the circuit has stabilized. If more than one input can change at one time, then nearly all circuits will contain hazards, and they cannot be eliminated by modifying the circuit implementation.
\ssc{Simulation and testing of logic circuits}
\sssc{Verilog, SystemVerilog, and VHDL}
A hardware description language (HDL) is a specialized computer language used to describe the structure and behavior of electronic circuits.

Verilog, standardized as IEEE 1364, is a hardware description language used to model and simulate electronic circuits.

SystemVerilog, an extension of Verilog, standardized as IEEE 1800, is a hardware description and hardware verification language used to model, simulate, and test electronic circuits.

VHDL (VHSIC Hardware Description Language), standardized as IEEE Std 1076, is a hardware description language used to model and simulate electronic circuits.
\sssc{Four-valued logic}
The two logic values, 0 and 1, are not sufficient for simulating logic circuits. At times, the value of a gate input or output may be unknown, which is represented by X (\verb|x| in Verilog/SystemVerilog). At other times we may have no logic signal at an input, as in the case of an open circuit when an input is not connected to any output, called high impedance or hi-Z/Hi-Z connection, which is represented by Z (\verb|z| in Verilog/SystemVerilog).

The logical connectives are defined as follows (without loss of commutativity of operators):
\[X\cdot0=Z\cdot0=0,\]
\[X\cdot1=X\cdot X=X\cdot Z=Z\cdot1=Z\cdot Z=X,\]
\[X+1=Z+1=1,\]
\[X+0=X+X=X+Z=Z+0=Z+Z=X.\]
\sssc{Simulation and testing}
A simple simulator for combinational logic works as follows:
\ben
\item The circuit inputs are applied to the first set of gates in the circuit, and the out-puts of those gates are calculated.
\item The outputs of the gates which changed in the previous step are fed into the next level of gate inputs. If the input to any gate has changed, then the output of that gate is calculated.
\item Step 2 is repeated until no more changes in gate inputs occur. The circuit is then in a steady-state condition, and the outputs may be read.
\item Steps 1 through 3 are repeated every time a circuit input changes.
\een
If a circuit output is wrong for some set of input values, this may be due to several possible causes:
\bit
\item incorrect design
\iten Gates connected wrong
\item Wrong input signals to the circuit
\eit
If the circuit is physically built, other possible causes include
\bit
\item Defective gates
\item Defective connecting wires
\eit
It is very easy to locate the problem systematically by starting at the output and working back through the circuit until the trouble is located.
\ssc{Integrated Circuit (IC)}
An integrated circuit (IC), also known as a microchip or simply chip, is a compact assembly of electronic circuits, in which the components are fabricated onto a thin piece (called chip) of semiconductor material, most commonly silicon.

Integrated circuits can be broadly classified into analog, digital and mixed-signal, consisting of analog and digital signaling on the same IC.
\sssc{Integrated circuit package}
An integrated circuit package is the physical case that holds the silicon chip (die) and provides the electrical and mechanical connection between the chip’s microscopic circuits and the outside world. An IC package typically has:
\bit
\item Silicon die (chip): The actual IC.
\item Bond wires: Typically tiny gold or aluminum wires.
\item Encapsulation: Protective material that shields the die from damage.
\item Pins or leads: Metal terminal that connect the IC to a printed circuit board (PCB).
\eit
\sssc{Scales}
\begin{longtable}[c]{|c|c|c|c|c|}
    \hline
    Acronym & Name & Year & Transistor count & Logic gates number \\\hline
    SSI & small-scale integration & 1964 & 1 to 10 & 1 to 12 \\\hline
    MSI & medium-scale integration & 1968 & 10 to 500 & 13 to 99 \\\hline
    LSI & large-scale integration & 1971 & 500 to 20000 & 100 to 9999 \\\hline
    VLSI & very-large-scale integration & 1980 & 20000 and more (or 20000 to 1000000) & 10000 and more (or 10000 to 99999) \\\hline
    (ULSI & ultra-large-scale integration & 1964 & 1000000 and more & 100000 and more) \\\hline
\end{longtable}

It is generally uneconomical to design digital systems using only SSI and MSI integrated circuits. By using LSI and VLSI ICs, the required number of integrated circuit packages is greatly reduced, and the cost of mounting, wiring, designing, and maintaining may be significantly lower.
\ssc{Combinational circuits}
\sssc{Active high and low}
For some type of circuits, active high (aka active-high or noninverted) outputs means the outputs are noninverted; active low (aka active-low or inverted) outputs means the outputs are logically equivalent to being inverted. Active high (aka active-high or noninverted) inputs means the inputs are noninverted before inputted; active low (aka active-low or inverted) outputs means the inputs are logically equivalent to being inverted before inputted.

Unless otherwise specified, a type of circuit with active high and low version of inputs and/or outputs are assumed to be active high inputs and/or outputs.
\sssc{Bus}
Several signals that perform a common function may be grouped together to form a bus, represented with a single heavy line optionally with a diagonal slash through it specifying the number of bits in the bus.
\sssc{Multiplexer (MUX)}
A multiplexer (MUX) or data selector has a group of data inputs and a group of control inputs (aka select inputs or select lines) and one output. The control inputs are used to select one of the data inputs and connect it to the output terminal. The symbol for a MUX is an isosceles trapezoid, where the longer parallel side contains the data input pins, one of the legs contains the control inputs, and the shorter parallel side contains the output pin. Multiplexers are frequently used in digital system design to select the data which is to be processed or stored.

A $2^n$-to-1 MUX has $2^n$ data inputs $D_0,D_1,\ldots D_{2^n-1}$, $n$ control inputs $S_0,S_1,\ldots S_{n-1}$, and one output $Y$, with $Y$ given by
\[Y=\sum_{i=0}^{2^{n-1}}D_i\cdot m_i(S_0,S_1,\ldots S_{n-1}),\]
where $m_i$ is the minterm corresponding to $i$ for $n$-variable functions of $S_0,S_1,\ldots S_{n-1}$. When $m_i=1$, that is, the binary number $S_0S_1\ldots S_{n-1}$ is $i$, $I_i$ is the selected input.

Alternatively, $Y$ can be written as
\[Y=\prod_{i=0}^{2^{n-1}}\qty(D_i+M_i(S_0,S_1,\ldots S_{n-1})),\]
where $M_i$ is the maxterm corresponding to $i$ for $n$-variable functions of $S_0,S_1,\ldots S_{n-1}$.

We can implement a MUX with a two- or multi-level circuit that realizes the SOP or POS form given above.

A MUX with active high outputs outputs the selected input directly, while a MUX with active low outputs outputs the inverse of the selected input. A MUX can also have an additional input $E$ called enable (signal) and works as discussed above when $E=1$ and outputs $0$ when $E=0$.

A $2^n$-to-1 multiplexer can be realized with $\qty(2^{n-m+1}-1)$ $2^m$-to-1 multiplexers ($m\leq n$).

Given a $n$-variable switching function $F$, $2^m$ $m$-variable subfunctions of $F$ can be obtained using Shannon's expansion of the function and compose back to $F$ with one $2^{n-m}$-to-1 multiplexer. Take $m=1$, we can get a realization of $F$ with one $2^{n-1}$-to-1 multiplexer.
\sssc{Three-state buffer or tri-state buffer}
A three-state buffer (or tri-state buffer) is a logic gate that has three stable states: logic HIGH (1), logic LOW (0), and high impedance (Z). In the high impedance state, the output of the buffer is effectively disconnected from the subsequenct circuit. A three-state buffer has one data input and one output like a simple buffer and one additional input $B$ called enable (signal) and works as a simple buffer when $B=1$ and outputs high impedance when $B=0$.

When a bus is driven by three-state buffers, we call it a three-state bus.

We can also add an inversion bubble at enable signal such that it works as a simple buffer when $B=0$ and outputs high impedance when $B=1$. We can also add an inversion bubble at either data input or output such that when it doesn't outputs high impedance, it works as an inverter.

We can implement a 2-to-1 MUX by using data inputs of the MUX as data inputs of two tri-state buffers, using the control input and its inverse as enable signal of the two tri-state buffers respectively, and simply connecting the two outputs of the two tri-state buffers to the output of the MUX.
\sssc{Bidirectional I/O pin}
Integrated circuits are often designed using bidirectional (or bi-directional) I/O pins for input and output to save pins and wiring. A bidirectional pin is a single physical pin that can act as either an input or an output, but not both at the same time, depending on the control signal. To accomplish this, the circuit output is connected to the pin through a three-state buffer. When the buffer is enabled, the pin is driven with the output signal; when the buffer is disabled, an external source can drive the pin.
\sssc{Decoders}
A decoder is a combinational circuit that converts binary input codes into a single active output line. An $n$-to-$2^n$ line decoder has $n$ input signals and $2^n$ output signals realizing a one-to-one Boolean vector function $\{0,1\}^n\to\{0,1\}^{2^n}$ that map each input combination to an output vector of which the 1-norm is $1$ (noninverted/active-high outputs) or $2^n-1$ (inverted/active-low outputs).

Each possible output of a decoder corresponds to a minterm of the inputs. Thus, we can OR them together to get the active-high outputs or NAND them together to get the active-low outputs.
\sssc{Normal encoders}
A normal encoder is a combinational circuit that realize the inverse of a Boolean vector function realized by a decoder. A $2^n$-to-$n$ normal encoder realizes the inverse of a Boolean vector function $\{0,1\}^n\to\{0,1\}^{2^n}$ realized by a decoder, of which the domain of the inverse is a subset of $\{0,1\}^{2^n}$ that has $n$ elements, each of which with the 1-norm being $1$ (noninverted/active-high inputs) or $(2^n-1)$ (inverted/active-low inputs). If the input combination is not in the domain, the outputs are undefined.
\sssc{Priority encoders}
A priority encoder resolves the undefined behaviour of normal encoders by assigning priority to the inputs.

A $2^n$-to-$n$ active-high inputs and outputs priority encoder has $2^n$ inputs and $(n+1)$ outputs. If all inputs are 0, the $(n+1)$-th output, called valid (V) or enable output (EO), is 0 and the other $n$ outputs are undefined; otherwise the $(n+1)$-th output is 1 and the other $n$ outputs are determined by a loop: \texttt{for (int i=1; i<=2^n; i=i+1)} (in which \verb|2^n| means $2$ to the power of $n$):
\bit
\item If the input of $i$th priority is $1$, the output combination is the output combination of a active-high inputs normal encoder when that input is $1$ and all other inputs are $0$.
\eit

The priority of inputs are usually as: for any $1\leq i<2^n$, the $(i+1)$-th input is prior to the $i$-th input.

An active-low inputs priority encoder invert the inputs; an active-low outputs priority encoder invert the outputs.
\ssc{Read-only memory (ROM)}
\sssc{Introduction and Realization}
A read-only memory (ROM) is a non-volatile (data is kept even when power is off) memory consists of an array of semiconductor devices that are interconnected to store an array of binary data. Once binary data is stored in the ROM, it can be read out whenever desired, but the data that is stored cannot be changed under normal operating conditions. The block diagram of a ROM is typically a rectangle with input lines at the left side and output lines at the buttom.

A ROM which has $n$ input lines and $m$ output lines contains an array of $2^n$ words, each word of width $m$ bits, called a $2^n$(-word)$\times m$-bit ROM. An input combination serves as an address to select one of the $2^n$ words and output it from the ROM, that is, a ROM basically consists of a decoder and a memory array. Typical sizes for commercially available ROMs range from 32 words $\times$ 4 bits to 512K words $\times$ 8 bits, or larger.

A $2^n$-word$\times m$-bit ROM stores $m$ $n$-variable truth tables by a $n$-to-$2^n$ decoder and a grid consists of $2^n$ rows (decoder outputs) intersecting each of the $m$ columns (wires that connect ground to ROM outputs, which outputs 0 if no 1 signal from the input lines is inputted), sometimes called OR plane, where in each of the $2^n\times m$ row-column intersection points, each corresponding to one bit in the memory array, a switching element, often a diode, is either present and connected or not. If absent or not connected (logic 0), the signal value from the row isn't passed to the column; if present and connected (logic 1), represented by a $\times$ on the diagram, the signal value from the row is passed to the column.
\sssc{Types}
Common types of ROMs:
\bit
\item Mask-programmable ROMs: Programmed permanently during manufacturing using photolithography by selectively including or omitting the switching elements at the cells. The mask used in photolithography is expensive, so the use of mask-programmable ROMs is economically feasible only if a large quantity (typically several thousand or more) is required with the same data array.
\item Programmable ROMs (PROM): Manufactured with switching elements at all cells and programmable once (one-time programmable (OTP)) by the user using a PROM programmer (aka PROM burner), which sends high-voltage pulses to burn tiny fuses in the cell to create permanent 1's and 0's.
\item Erasable Programmable ROMs (EPROM): Manufactured with each cell contains a control gate that receives normal programming or read voltages and a floating gate that is completely insulated by oxide and can store electric charge and with quartz window on top, programmable by the user using a PROM programmer (aka PROM burner), which sends high-voltage pulses to inject electrons into floating gates, and erasable by the user by expose it to intense UV light that provides photons providing energy to remove trapped electrons from the floating gate simultaneously at all celles.
\item Electrically Erasable Programmable ROMs (EEPROM): Similar to EPROM but erasing process is done electrically to remove trapped electrons from the floating gate one bit or one byte at a time (instead of all cells simultaneously) only a limited number of times, typically 100 to 1000 times.
\item Flash memory: Derived from EEPROM but optimized to be faster, denser, cheaper, and erasable in blocks or pages.

NOR and NAND flash:
\bit
\item NOR Flash: Output lines are connected in parallel and cells are erased in blocks and with random access for read. Writing and erasing is slower and cell size is larger.
\item NAND Flash: Output lines are connected in series and cells are erased in pages and with sequential access for read only. Writing and erasing is faster and cell size is smaller.
\eit
Type of cells:
\begin{longtable}[c]{|c|c|c|}
\hline
Type & Bits per cell & Characteristics \\\hline
Single-level cell (SLC) & 1 & Fast, reliable, expensive \\\hline
Multi-level cell (MLC) & 2 & Slower, cheaper \\\hline
TLC (Triple-Level Cell) & 3 & Used in consumer solid-state drives (SSDs) \\\hline
QLC (Quad-Level Cell) & 4 & Denser, slower, less endurable \\\hline
\end{longtable}
Flash memories usually have built-in programming and erase capability so that data can be written to the flash memory while it is in place in a circuit without the need for a separate programmer.
\ssc{Programmable Logic Devices (PLDs)}
A programmable logic device (PLD) is a general name for a digital integrated circuit capable of being programmed to provide a variety of different logic function. When a digital system is designed using a PLD, changes in the design can easily be made by changing the programming of the PLD without having to change the wiring in the system, which leads to lower cost designs.
\sssc{Programmable Logic Arrays (PLAs)}
A programmable logic array (PLA) performs the same basic function as a ROM. A PLA with $n$ inputs and $m$ outputs can realize $m$ functions of $n$ variables. The internal organization of the PLA is different from that of the ROM. The decoder is replaced with an AND array (aka AND plane) which realizes selected product terms of the input variables. The OR array (aka OR plane) ORs together the product terms needed to form the output functions, so a PLA implements a sum-of-products expression, while a ROM directly implements a truth table. Product terms are formed in the AND array by connecting switching elements at appropriate points in the array. Outputs are formed in the OR array by connecting switching elements at appropriate points in the array. 

The contents of a PLA can be specified by a PLA table, in which the first column are product terms, the second columns are input combinations in the product terms with $-$ indicating don't care, and the third columns are output combinations corresponding to the product terms, in which the first column is sometimes omitted.

When the number of input variables is small, a PROM may be more economical to use than a PLA. However, when the number of input variables is large, PLAs often provide a more economical solution than PROMs.

Type:
\bit
\item Mask-programmable logic arrays: Programmed permanently during manufacturing using photolithography. The mask used in photolithography is expensive, so the use of mask-programmable logic arrays is economically feasible only if a large quantity (typically several thousand or more) is required with the same AND and OR arrays.
\item Field-programmable logic arrays (FPLAs): Manufactured blank and programmable by the user. Technology of EPROM, EEPROM, or Flash memory may be used. Typically slower and larger than mask-programmable logic arrays.
\eit
\sssc{Programmable Array Logic (PAL)}
A programmable array logic (PAL) is similar to a programmable logic array but in which the AND array is programmable and the OR array is fixed. PALs are less expensive than the more general PLAs and easier to program. For this reason, logic designers frequently use PALs to replace individual logic gates when several logic functions must be realized.

A buffer is used because each PAL input must drive many AND gate inputs. When the PAL is programmed, some of the interconnection points are programmed to make the desired connections to the AND gate inputs, represented by $X$'s on the diagram.

When designing with PALs, we must simplify our logic equations and try to fit them into one (or more) of the available PALs. Unlike the more general PLA, the AND terms cannot be shared among two or more OR gates; therefore, each function to be realized can be simplified by itself without regard to common terms. For a given type of PAL, the number of AND terms that feed each output OR gate is fixed and limited. If the number of AND terms in a simplified function is too large, we may be forced to choose a PAL with more gate inputs and fewer outputs.
\sssc{Complex Programmable Logic Devices (CPLDs)}
As integrated circuit technology continues to improve, more and more gates can be placed on a single chip. This has allowed the development of complex programmable logic devices (CPLDs). Instead of a single PAL or PLA on a chip, many PALs or PLAs can be placed on a single CPLD chip and interconnected. When storage elements such as flip-flops are also included on the same IC, a small digital system can be implemented with a single CPLD.

Take Xilinx XCR3064XL CPLD for example. This CPLD has four function blocks, each of which has 16 associated macrocells and is a programmable AND-OR array that is configured as a PLA. Each macrocell contains a flip-flop and multiplexers that route signals from the function block to the input-output (I/O) block or to the interconnect array (IA). The IA selects signals from the macrocell outputs or I/O blocks and connects them to function block inputs. The I/O blocks connect the signals from the interior of the CPLD to the bi-directional I/O pins on the IC.
\sssc{Field-Programmable Gate Arrays (FGPAs)}
An FPGA is an IC that consists of an array of identical logic cells, also called configurable logic blocks (CLBs) or function generators, with programmable interconnections and surrounded by I/O blocks that connect the CLB signals to IC pins. The user can program the functions realized by each logic cell and the connections between the cells. Each CLB contains a lookup table (LUT), flip-flops, and multiplexers. An $n$-input LUT is essentially a reprogrammable $2^n$-word-$\times$-1-bit ROM that stores the truth table for the $n$-variable function being generated.

Given a $n$-variable switching function $F$ and a FGPA with $m$-input LUTs ($m<n$), $2^{n-m}$ $m$-variable subfunctions of $F$ can be obtained using Shannon's expansion of the function and compose back to $F$ with one $2^{n-m}$-to-1 multiplexer, which can be composed of $\qty(2^{n-m-k+1}-1)$ $2^k$-to-1 multiplexers ($k\leq n-m$).
\end{document}

