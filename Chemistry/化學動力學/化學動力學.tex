\documentclass[a4paper,12pt]{article}
\setcounter{secnumdepth}{5}
\setcounter{tocdepth}{3}
\newcounter{ZhRenew}
\setcounter{ZhRenew}{1}
\newcounter{SectionLanguage}
\setcounter{SectionLanguage}{1}
\input{/usr/share/LaTeX-ToolKit/template.tex}
\begin{document}
\title{化學動力學}
\author{沈威宇}
\date{\temtoday}
\titletocdoc
\section{化學動力學(Chemical Kinetics)/反應動力學(Reaction Kinetics)}
\subsection{總論}
\subsubsection{反應進度(Extent of reaction)}
指反應物消耗量或生成物生成量,用於量化反應進行的程度,其中量指體積莫耳濃度或正比於之的其他單位(下稱濃度)。
\subsubsection{反應速率(Reaction rate)}
指反應進度時變率。
\sssc{影響反應速率的因素比較}
\begin{longtable}[c]{|p{0.22\textwidth}|p{0.22\textwidth}|p{0.22\textwidth}|p{0.22\textwidth}|}
\hline
影響因素 & 濃度增加 & 升溫 & 加入催化劑 \\ \hline
改變反應自發與否 & 可能改變 & 可能改變 & 不變 \\ \hline
活化能(低限能) & 不變 & 高中:不變;實際:可能改變但多不顯著 & 正逆反應等量變小 \\ \hline
反應熱 & 不變 & 可能改變但多不顯著 & 不變 \\ \hline
總碰撞頻率 & 變大 & 變大 & 不變 \\ \hline
有效碰撞頻率 & 變大 & 變大 & 正逆反應等倍放大 \\ \hline
有效碰撞分率 & 不變 & 變大 & 正逆反應等倍放大 \\ \hline
速率常數 & 不變 & 變大 & 正逆反應等倍放大 \\ \hline
產率 & 可能改變 & 可能改變 & 不變 \\ \hline
\end{longtable}\FB
\subsection{(反應)速率定律(式)/方程式((Reaction) rate law/equation)}
反應速率與反應物濃度的定量關係式。
\subsubsection{(反應)速率定律(式)/方程式((Reaction) rate law/equation)}
令一只有一個速率決定步驟(rate-determining step)的反應:
\[\sum_{i=1}^ma_i\mathbf{A}_i\ce{->}\sum_{i=i}^nb_i\mathbf{B}_i\]
有催化劑與溶劑$\qty(\mathbf{C}_1,\mathbf{C}_2,\ldots,\mathbf{C}_o)$,其中:
\[\mathscr{S}=\{\mathbf{A}_1,\mathbf{A}_2,\ldots,\mathbf{A}_p,\mathbf{B}_1,\mathbf{B}_2,\ldots,\mathbf{B}_q,\mathbf{C}_1,\mathbf{C}_2,\ldots,\mathbf{C}_s\},\quad p\leq m\land q\leq n\land s\leq o\]
是所有前有提及之物質中可變濃度者之集合,並假設所有有效接觸面積不變。

則速率定律式為:
\[r=k\prod_{\mathbf{S}_i\in\mathscr{S}}[\mathbf{S}_i]^{c_i}\]

其中:
\begin{itemize}
\item $[\mathbf{A}]$指$\mathbf{A}$的體積莫耳濃度,並可以替換成任意與之正比的度量。
\item $r$:反應速率,可為任意$-\dv{[\mathbf{A}_i]}{t}$或$\dv{[\mathbf{B}_i]}{t}$。
\item $a_i$:反應式中反應物$\mathbf{A}_i$的平衡係數。
\item $b_i$:反應式中生成物$\mathbf{B}_i$的平衡係數。
\item $c_i$:$\mathbf{S}_i$的反應級數(order)。
\item $\sum_{\mathbf{S}_i\in\mathscr{S}}c_i$:反應(總)級數,稱該反應為$\sum_{\mathbf{S}_i\in\mathscr{S}}c_i$級反應($\sum_{\mathbf{S}_i\in\mathscr{S}}c_i$th-order reaction)。
\item $k$:(反應)速率常數/係數((reaction) rate constant/coefficient)。
\end{itemize}
\subsubsection{說明}
\begin{itemize}
\item 化學反應中一般將接觸面積變化不大的固體、稀溶液中的溶劑視為不可變濃度,後者如 NTP 稀水溶液中水的濃度可視為恆為 $\frac{500}{18}$ M。
\item 壓力影響氣相物質的濃度,但對凝相物質幾乎沒有影響(除非改變晶型等)。對於理想氣體,分壓正比於體積莫耳濃度,故對於偏差較小的情況可使用之作為濃度單位。理想氣體的分壓$p$與體積莫耳濃度$C_M$關係為:
\[p=C_MRT\]
\item 反應總級數$x$的反應稱$x$級反應。由速率常數的單位可推得反應總級數,$x$級反應速率常數之單位為$M^{1-x}s^{-1}$。速率常數的單位常省略不寫。
\item 反應速率之不同選擇間,速率常數互為乘上一個常數。令$1\leq f\neq g\leq p$、$1\leq h\neq i\leq q$,$k_f$、$k_g$、$k_h$與$k_i$分別為選用$-\dv{[\mathbf{A}_f]}{t}$、$-\dv{[\mathbf{A}_g]}{t}$、$\dv{[\mathbf{B}_h]}{t}$與$-\dv{[\mathbf{B}_i]}{t}$作為反應速率時的速率常數,則:
\[-\dv{[\mathbf{A}_f]}{t}\frac{1}{a_f}=-\dv{[\mathbf{A}_g]}{t}\frac{1}{a_g}=\dv{[\mathbf{B}_h]}{t}\frac{1}{b_h}=\dv{[\mathbf{B}_i]}{t}\frac{1}{b_i}\]
\[\frac{k_f}{a_f}=\frac{k_g}{a_g}=\frac{k_h}{b_h}=\frac{k_i}{b_i}\]
\item 速率常數依賴於反應物本性、溫度、反應活化能,不依賴於反應熱、反應物或生成物濃度。
\item 測量時,以實驗先測得各反應物級數,接著再以反應速率推得速率常數。
\item 半生期(Half-life)$t_{1/2}$:指一反應物濃度減半所需時間,正比於速率常數的倒數,可能依賴於初始濃度。
\item 放射性衰變可視為單一反應物一級反應,惟濃度改為莫耳數。
\item 秒錶反應(clock reaction)/化學鐘(chemical clock):由於可檢測量的時鐘物種的存在,在可預測的誘導時間之後出現可觀察的特性。
\end{itemize}
\subsubsection{單一反應物基元反應速率定律式形式與半生期(Half-life)}
令一基元反應只有單一反應物,其濃度$a$作為時間的函數,$t=0$時濃度$a_0$,反應速率$r=-\dv{a}{t}$,速率常數$k$,濃度單位$M$、時間單位$s$:
\begin{longtable}[c]{|p{0.1\textwidth}|p{0.15\textwidth}|p{0.2\textwidth}|p{0.2\textwidth}|p{0.15\textwidth}|}
\hline
反應級數 & 速率定律式微分形式 & 速率定律式積分形式 & 半生期 & 速率常數單位 \\\hline
0 & $r=k$ & $a=a_0-kt$ & $\frac{a_0}{2k}$ & M s$^{-1}$ \\\hline
1 & $r=ka$ & $a=a_0\cdot e^{-kt}$ & $\frac{\ln(2)}{k}\approx\frac{0.693}{k}$ & s$^{-1}$ \\\hline
2 & $r=ka^2$ & $a=\frac{a_0}{1+a_0\cdot kt}$ & $\frac{1}{ka_0}$ & M$^{-1}$ s$^{-1}$ \\\hline
\end{longtable}\FloatBarrier
\subsection{碰撞理論(Collision theory)}
\sssc{主張}
\ben
\item 反應物粒子必須相互碰撞,才有可能發生化學反應。
\item 不是所有的碰撞都會發生化學反應。能夠發生化學反應的碰撞稱為有效碰撞(effective collision)。
\item 化學反應速率的快慢取決於有效碰撞的頻率。
\een
\subsubsection{低限能(Threshold energy)}
阿瑞尼斯(Svante Arrhenius)提出。$E_a$,發生有效碰撞所需的最低動能。只有動能超過低限能的粒子才有可能發生有效碰撞。該能量在反應中轉換為電位能(斷鍵等)。必為正。

碰撞理論中有效碰撞分率理論值不考慮立體因子(Steric factor),故:
\[\text{有效碰撞分率理論值}=\frac{\text{有效碰撞頻率理論值}}{\text{總碰撞頻率}}=\frac{\text{超越低限能分子數}}{\text{總分子數}}\]
\sssc{立體因子(Steric factor)}
$\rho$,只有適當的碰撞位置與方向(化學鍵生成與斷裂處等)方能引發化學反應,例如 \ce{CO(g) + NO2(g) -> CO2(g) + NO(g)} 中 \ce{CO} 之 \ce{C} 與 \ce{NO2} 之一 \ce{O} 以足夠相對速率碰撞。立體因子定義為速率常數的實驗值與碰撞理論預測值之間的比率。

考慮立體因子(Steric factor)後的實際情況:
\[\text{有效碰撞分率}=\frac{\text{有效碰撞頻率}}{\text{總碰撞頻率}}=\rho\cdot\frac{\text{超越低限能分子數}}{\text{總分子數}}=\rho\cdot\text{有效碰撞分率理論值}\]
\subsubsection{活化複合體/活化錯合物(Activated complex)與活化能(Activation energy)}
波拉尼(Polányi Mihály)等提出。化學反應的過程中,碰撞開始後、產物生成前,會先經過一個(後經反應機構(Reaction mechanism)修正,應為每個基元反應(Elementary reaction)經過一個)動能減少、位能增加的一高位能、極不穩定的過渡(狀)態(Transition state),其原鍵結未完全破壞,新鍵結未完全形成,稱活化複合體/活化錯合物(Activated complex),其存在時間極短,不易偵測,其位能減去最初反應物之位能為該反應的(正反應)活化能((forward reaction) activation energy)$E_a$(後經反應機構修正,反應的(正反應)活化能為其反應過程中位能最高時的位能減去最初反應物之位能),其值等於該反應的低限能(Threshold energy)。活化複合體會接著減少位能、增加動能,轉變成生成物或變回原來的反應物。活化複合體之位能減去最終生成物之位能為逆反應活化能$E_r$(reverse reaction activation energy)(後經反應機構修正,逆反應活化能為其反應過程中位能最高時的位能減去最終生成物之位能)。

活化能依賴於反應物本性、反應機構(活化複合體種類)、催化劑、抑制劑,不依賴於反應熱、各物質濃度,高中:不依賴於溫度;實際:依賴於溫度但效應多不顯著。

活化能愈大,能量障壁愈高,反應愈難發生,反應速率愈慢,速率常數愈小;活化能愈小,能量障壁愈低,反應愈易發生,反應速率愈快,速率常數愈大。
\subsection{反應機構/反應機理(Reaction mechanism)}
有時也視為碰撞理論的一部分。
\sssc{碰撞理論與反應速率定律式的矛盾}
\bit
\item 碰撞理論無法解釋為何部分反應物的濃度對反應速率沒有影響,如\ce{NO2(g) + CO(g) -> NO(g) + CO2(g)}的速率定律式為$r=k[\ce{NO2}]^2$。
\item 碰撞理論預測大於二個反應物粒子的反應之反應物應難以瞬間同時碰撞,但實際上許多大於二個反應物粒子的反應反應速率甚快,如\ce{H2O2(aq) + 3I-(aq) + 2H+(aq) -> I3-(aq) + 2H2O(l)}。
\eit
\sssc{主張}
化學反應不一定是單一步驟完成,而是由一系列連續而無法再分解為更簡單步驟的基元/基本反應/步驟(Elementary reaction/step)所構成。

一個合理的反應機構須滿足:
\bit
\item 所有基本步驟相加會得到該反應的反應式。
\item 由反應機構推導出的速率定律式應與實驗結果一致。
\eit
\subsubsection{基元/基本反應/步驟(Elementary reaction/step)}
只有一個步驟的反應。假設所有接觸面積不變,其各可變濃度反應物之平衡係數即其在速率定律式中之級數,其餘物質均不在速率定律式中。令有基元反應:
\[\sum_{i=1}^ma_i\mathbf{A}_i\ce{->}\sum_{i=i}^nb_i\mathbf{B}_i\]
反應物中:
\[\{\mathbf{A}_1,\mathbf{A}_2,\ldots,\mathbf{A}_p\},\quad p\leq m\]
為可變濃度物質,其餘反應物為不可變濃度物質,則該反應的速率定律式為:
\[r=k\prod_{i=1}^p[\mathbf{A}_i]^{a_i}\]

多數基元反應為一或二級。如 \ce{CO(g) + NO2(g) -> CO2(g) + NO(g)}。
\subsubsection{中間產物/反應中間體(Reaction intermediate)}
非基元反應中,於反應過程中先被釋放出來,而後又參與反應,不出現於淨反應中之物質。
\subsubsection{速率決定/控制步驟(Rate-determining step, RDS, r/d step)/瓶頸反應}
指一個反應機構中速率最慢(即正反應活化能最大)的基元反應步驟,其決定了整個反應的速率:
\begin{itemize}
\item 如果速率決定步驟之反應物為全反應之反應物的子集,則全反應之速率定律式即速率決定步驟之速率定律式。
\item 如果速率決定步驟之前有其他步驟且速率決定步驟遠慢於其前所有步驟,則其前步驟幾乎處於平衡狀態,稱預平衡(Pre-equilibrium)/快速平衡(fast equilibrium)。此時應將速率決定步驟的反應物中屬全反應中之中間產物者,改以其前步驟之平衡常數乘以各全反應中之反應物與生成物之濃度或其倒數表示,代入速率決定步驟之速率定律式中,並將各須乘上的平衡常數、速率決定步驟之速率定律式的速率常數與全反應之反應速率與速率決定步驟之反應速率之比值之積作為新的速率常數,方得全反應之速率定律式。
\end{itemize}
\subsubsection{氫氣與氯化碘生成碘與氯化氫之反應}
全反應:
\[\ce{H2(g) + 2ICl(g) ->[$\Delta$] I2(g) + 2HCl(g)}\]

反應機構:
\begin{enumerate}
\item \ce{H2(g) + ICl(g) -> HI(g) + HCl(g)}(慢),反應速率:$r_1=-\dv{[\ce{H2}]}{t}=k_1[\ce{H2}][\ce{ICl}]$
\item \ce{HI(g) + ICl(g) -> I2(g) + HCl(g)}(快)
\end{enumerate}

說明:
\begin{itemize}
\item 速率決定步驟為第一步驟。
\item \ce{HI(g)}為中間產物。
\item 全反應之速率定律式為:
\[r=-\dv{[\ce{H2}]}{t}=k_1[\ce{H2}][\ce{ICl}]\]
\end{itemize}
\subsubsection{三氯甲烷與氯氣生成四氯化碳與氯化氫之反應}
全反應:
\[\ce{CHCl3(g) + Cl2(g) -> CCl4(g) + HCl(g)}\]

反應機構:
\begin{enumerate}
\item \ce{Cl2(g) -> 2Cl(g)}(快),平衡常數:$K_1=\frac{[\ce{Cl}]^2}{[\ce{Cl2}]}$
\item \ce{Cl(g) + CHCl3(g) -> CCl4(g) + H(g)}(慢),反應速率:$r_2=-\dv{[\ce{CHCl3}]}{t}=k_2[\ce{Cl}][\ce{CHCl3}]$
\item \ce{H(g) + Cl(g) -> HCl(g)}(快)
\end{enumerate}

說明:
\begin{itemize}
\item 速率決定步驟為第二步驟。
\item \ce{Cl(g)}、\ce{H(g)}為中間產物。
\item 全反應之速率定律式為:
\[\begin{aligned}
r&=-\dv{[\ce{Cl}]}{t}=r_2=k_2[\ce{Cl}][\ce{CHCl3}]\\
&=k_2\sqrt{K_1}[\ce{Cl2}]^{\frac{1}{2}}[\ce{CHCl3}]
\end{aligned}\]
\end{itemize}
\subsubsection{臭氧生成氧氣之反應}
全反應:
\[\ce{2O3(g) -> 3O2(g)}\]

反應機構:
\begin{enumerate}
\item \ce{O3(g) -> O2(g) + O(g)}(快),平衡常數:$K_1=\frac{[\ce{O2}][\ce{O}]}{[\ce{O3}]}$
\item \ce{O3(g) + O(g) -> 2O2(g)}(慢),反應速率:$r_2=\text{[\ce{O3}] 被此反應消耗速率}=k_2[\ce{O3}][\ce{O}]$
\end{enumerate}

說明:
\begin{itemize}
\item 速率決定步驟為第二步驟。
\item \ce{O(g)}為中間產物。
\item 全反應之速率定律式為:
\[\begin{aligned}
r&=-\dv{[\ce{O3}]}{t}=2r_2=2k_2[\ce{O3}][\ce{O}]\\
&=2k_2K_1[\ce{O3}]^2[\ce{O2}]^{-1}
\end{aligned}\]
\end{itemize}
\subsubsection{臭氧在一氧化氮催化下生成氧氣之反應}
全反應:
\[\ce{2O3(g) ->[\ce{NO}] 3O2(g)}\]

反應機構:
\begin{enumerate}
\item \ce{O3(g) + NO(g) -> O2(g) + NO2(g)}(快),平衡常數:$K_1=\frac{[\ce{NO2}][\ce{O2}]}{[\ce{O3}][\ce{NO}]}$
\item \ce{NO2(g) -> NO(g) + O(g)}(慢),反應速率:$r_2=\text{[\ce{NO2}] 被此反應消耗速率}=k_2[\ce{NO2}]$
\item \ce{O3(g) + O(g) -> 2O2(g)}(快)
\end{enumerate}

說明:
\begin{itemize}
\item 速率決定步驟為第二步驟。
\item \ce{NO(g)}為催化劑。
\item \ce{NO2(g)}、\ce{O(g)}為中間產物。
\item 全反應之速率定律式為:
\[\begin{aligned}
r&=-\dv{[\ce{O3}]}{t}=2r_2=2k_2[\ce{NO2}]\\
&=2k_2K_1[\ce{O3}][\ce{NO}][\ce{O2}]^{-1}
\end{aligned}\]
\end{itemize}
\subsection{溫度效應}
溫度增加,所有反應之速率均增加,吸熱反應平衡狀態右移,放熱反應平衡狀態左移,反應機構與各活化能均不變。
\subsubsection{阿瑞尼斯方程(Arrhenius equation)}
令一反應速率常數$k$、指數前因子(Pre-exponential factor)$A$、活化能$E_a$,阿瑞尼斯方程指出速率常數作為絕對溫度的函數的近似:
\[k=Ae^{-\frac{E_a}{RT}}\]
其中指數前因子一個依賴於反應與反應機構但不依賴於溫度、各物質濃度的常數。
\subsubsection{溫度與超過低限能的粒子比例正相關}
超過低限能的反應物粒子增加,有效碰撞分率增加,是溫度效應的主要原因。

令波茲曼常數$k_B$,$a = \sqrt{\frac{k_BT}{m}}$,粒子速率$v$,反應物粒子質量$m$,$u=\sqrt{\frac{2E_a}{m}}$,有效碰撞分率理論值=超過低限能的粒子比例$P$:
\[\begin{aligned}
P=&\int_u^\infty \sqrt{\frac{2}{\pi}}\frac{v^2}{a^3} e ^{\frac{-v^2}{2a^2}}\,\mathrm{d}v \\
=& \sqrt{\frac{2}{\pi}} \frac{u}{a}  e ^{-\frac{u^2}{2a^2}} + 1 - \text{erf}\left(\frac{u}{\sqrt{2}a}\right)\\
=& \sqrt{\frac{4E_a}{k_BT\pi}} e^{-\frac{2E_a}{k_BT}} + 1 - \text{erf}\left(\sqrt{\frac{E_a}{k_BT}}\right)
\end{aligned}\]
\subsubsection{溫度與總碰撞頻率正相關}
溫度增加,通量增加,總碰撞頻率增加,總碰撞頻率與平均速率成正比,是溫度效應的次要原因。以 \ce{H2(g) + I2(g) -> 2HI(g)} 為例,700 K 反應速率約是 556 K 的 1445 倍,但平均速率僅約 1.12 倍,可見此非主因。
\subsubsection{近似}
理想氣體間活化能 50 kJ mol$^{-1}$ 的反應,溫度增加$\Delta T$,反應速率自$r_1$變為$r_2$:
\[\frac{r_2}{r_1} \approx 2^{\frac{\Delta T}{10}}\]

通常分子較大者溫度效應愈大。如一些蛋白質變性,溫度增加$\Delta T$,反應速率自$r_1$變為$r_2$:
\[\frac{r_2}{r_1} \approx 50^{\frac{\Delta T}{10}}\]
\subsection{物質本性對反應速率的效應}
\subsubsection{常溫常壓下各類反應速率通則}
\begin{itemize}
\item 涉及化學鍵的破壞或形成能量愈大,反應速率通常愈慢,如反應速率(\ce{CH4 + Br2 -> CH3Br + HBr})>(\ce{C2H4 + Br2 -> CH2BrCH2Br})
\item 不涉及化學鍵的破壞或形成者,涉及電子之轉移愈多,反應速率通常愈慢。
\item 不涉及化學鍵的破壞或形成亦不涉及電子之轉移者,反應速率通常甚快。
\item 所有反應物與生成物均為水中離子的反應通常甚快。
\item 無機反應通常快於有機反應,離子間反應通常快於分子間反應。
\item 涉及共價網狀固體的破壞者通常極慢,幾乎無法觀察到反應發生。
\item 立體結構愈複雜的反應物,立體因子通常愈小,反應速率愈慢,惟立體障礙的影響通常遠小於電子轉移與化學鍵破壞或形成。
\item 流體勻相反應產生安定產物者通常較有固體參與的反應快。
\item 愈多步驟與須愈多粒子碰撞的反應通常愈慢。
\item 通常:阿瑞尼斯酸鹼中和(極快,$10^{-7}$ 秒內完成99\%)>錯離子生成(極快,因需成鍵故較阿瑞尼斯酸鹼中和慢,如\ce{Cu^{2+}(aq) + 4NH3(aq) -> Cu(NH3)4^{2+}(aq)})$\approx$沉澱(極快,因需排入晶格故較阿瑞尼斯酸鹼中和慢,如\ce{Ba^{2+}(aq) + SO4^{2-}(aq) -> BaSO4(s)})>離子電子轉移反應(快,如\ce{Fe^{2+} + Ce^{4+} -> Fe^{3+} + Ce^{3+}})>氧化還原反應有斷鍵重組者(中,斷鍵重組愈多愈慢,斷鍵較少者如 \ce{5Fe^{2+} + MnO4-(aq) + 8H+(aq) -> 5Fe^{3+}(aq) + Mn^{2+}(aq) + 4H2O(l)},斷鍵較多者如 \ce{5C2O4^{2-}(aq) + 2MnO4-(aq) + 16H+(aq) -> 10CO2(g) + 2Mn^{2+}(aq) + 8H2O(l)})>有機反應(慢甚至幾乎不反應,常需催化或高溫高壓等方進行,其中通常斷非共振 π 鍵>斷共振非芳香 π 鍵>斷芳香性共軛 π 鍵或 σ 鍵,路易斯酸鹼中和>自由基機制)>燃燒反應(如 \ce{2H2(g) + O2(g) -> 2H2O(l)} 10$^{17}$ 秒方約一半的氫氣反應)。
\end{itemize}
\subsubsection{反應物活性}
同類型反應,反應物活性愈大愈快,如:
\bit
\item (\ce{2K(s) + 2H2O(l) -> 2KOH(aq) + H2(g)})>(\ce{2Na(s) + 2H2O(l) -> 2NaOH(aq) + H2(g)})
\item (\ce{H2(g) + F2(g) -> 2HF(g)})>(\ce{H2(g) + Cl2(g) -> 2HCl(g)}>(\ce{H2(g) + Br2(l) -> 2HBr(g)}>(\ce{H2(g) + I2(s) -> 2HI(g)}
\eit
\subsubsection{溶劑對有機反應速率的效應}
\begin{itemize}
\item 溶劑解離出質子的能力愈大,SN1、E1、AE 反應通常愈快,SN2、AN 反應通常愈慢。
\item 溶劑形成分子間氫鍵的能力愈大,SN1、E1 反應通常愈快,SN2、E2、AN 反應通常愈慢。
\item 溶劑極性愈大,SN1、E1 反應通常愈快,SN2、E2 反應通常愈慢。
\end{itemize}
\sssc{一些較快的分子反應}
涉及分子的反應通常慢,但一些反應速率較快,如:
\bit
\item 碳酸鹽固體投入酸性溶液,碳酸根與氫離子反應成水與二氧化碳,如:\ce{CaCO3(s) + 2HCl(aq) -> CaCl2(aq) + H2O(l) + CO2(g)}
\item \ce{NH3(g) + HCl(g) -> NH4Cl(s)},白色煙霧狀,數秒。
\item \ce{2NO(g) + O2(g) -> 2NO2(g)}、\ce{3NO(g) + O3(g) -> 3NO2(g)},紅棕色,約7-8秒
\item \ce{H2(g) + F2(g) -> 2HF(g)},活性極大,在暗處仍極快。
\item \ce{H2(g) + Cl2(g) -> 2HCl(g)},活性大,照光處極快。
\item \ce{P4(s) + 5O2(g) -> P4O10(s)}、\ce{P4(s) + 3O2(g) -> P4O6(s)},白磷在空氣中易自燃,須存放在液體中。雖成鍵甚多,但反應不慢。
\eit
\subsection{接觸面積效應}
\sssc{比表面積(Specific surface area, SSA)}
物質顆粒的表面積除以體積。顆粒愈小或孔洞愈多,比表面積愈大。
\sssc{勻相反應(Homogeneous reaction)}
反應物能混合成單一相而沒有界面的反應。通常為流體溶液。如所有氣態反應、阿瑞尼斯酸鹼中和反應。勻相反應物之影響可用速率定律式表示。
\sssc{非勻相/異相反應(Heterogeneous reaction)}
反應物不能混合成單一相而具有界面的反應。通常為液相與氣相、不互溶的液相與液相、液相與固相或氣相與固相。如活性金屬固體與酸溶液的反應、固體的燃燒反應。

與主要流相反應物不同相者以接觸面積而非體積莫耳濃度影響反應速率,基元反應速率正比於其有效接觸面積。若增加接觸面積,碰撞次數會增加,故可加快反應速率,但不會增加產率。

相似反應,非勻相反應通常較勻相反應慢。

固體反應物之接觸面積一般可視為正比於其表面積。

接觸面積常視為常數。
\sssc{增加固體接觸面積的方法}
\bit
\item 攪拌,同時可增加溫度,常用於溶解反應。
\item 將固體研磨成更小顆粒,如鐵釘在純氧只能加熱至通紅的熱源可以使等質量相同成分的鋼絲絨劇烈燃燒、粉末狀固體溶解或與液態溶液反應較塊狀固體快。
\eit
\sssc{粉塵爆炸/塵爆(Dust explosion)}
粉塵(Dust)定義為顆粒直徑小於500微米的顆粒,因比表面積大,反應速率快。

粉塵爆炸指懸浮在空間中的可燃粉塵快速燃燒爆炸。

粉塵爆炸的條件為:
\bit
\item 適當濃度的可燃粉塵懸浮在空間中,
\item 有足夠的氧化劑(通常是空氣中的氧氣),與
\item 火源或強烈振動或摩擦等產生高溫。
\eit

較容易發生粉塵爆炸的粉塵有鋁粉、鋅粉、鎂粉、鐵粉、塑膠原料粉末、穀物粉末、糖分、奶粉、花粉等,適當濃度下,只要接觸到火源或高溫,一旦少數顆粒被點燃,很快就能再點燃其他粉塵導致爆炸。
\sssc{鈍化(Passivation)}
指材料外的保護層,使材料不易與外界反應,如氧化鋁緻密保護層可保護內部之鋁不氧化。
\ssc{催化劑/觸媒(Catalyst)}
\sssc{催化劑/觸媒(Catalyst)}
參與反應,而後又釋放出來,不出現於淨反應中、反應前後質量不變之物質,催化劑通過改變反應機構,等量降低正反應與逆反應的活化能(和與其相等的低限能),造成正反應與逆反應的速率常數、有效碰撞頻率、有效碰撞分率均等倍放大,這種現象稱催化(catalysis)。

催化劑會縮短反應達到平衡或耗盡一種反應物的時間,但不改變平衡狀態與平衡常數、產率、反應熱、動能分布,也不會讓原先不自發的反應變得自發。

催化劑不同產物可能不同,如:
\[\ce{CO(g) + 3H2(g) ->[\ce{Ni}\tx{, 100°C, 1 atm}] CH4(g) + H2O(l)}\]
\[\ce{CO(g) + 2H2(g) ->[\ce{ZnO$\cdot$Cr2O3}\tx{, 400°C, 5000 atm}] CH3OH(g)}\]

不存在可以通過改變反應機構從而增加反應活化能的物質。
\sssc{催化活度/活性/活力(Catalytic activity)}
每秒使反應進度提高一莫耳的催化劑定義為具有一開特(katal, kat)的催化活度。
\sssc{(反應)抑制劑((Reaction) inhibitor)}
抑制催化劑降低活化能之能力的物質。
\subsubsection{勻相催化}
催化劑與反應物能混合成單一相而沒有界面的催化。可通過增加催化劑之濃度來增加其與反應物間的碰撞頻率,從而加快反應速率。勻相催化劑之影響可用速率定律式表示。通常催化效率較佳,但通常較不易回收重複使用。如:
\bit
\item \ce{2O3(g) ->[\tx{\ce{Cl2(g)}或\ce{NO(g)}或含氯化合物(g)}] 3O2(g)}
\item \ce{2H2O2(aq) ->[\tx{\ce{Fe^{2+}(aq)}\text{或}\ce{Fe^{3+}(aq)}\text{或}\ce{I-(aq)}}] 2H2O(l) + O2(g)}
\item \ce{HCOOH(aq) ->[\ce{H+}] CO(g) + H2O(l)}
\item 質子催化順丁烯二酸變為反丁烯二酸。
\item 質子催化甲酸分解成一氧化碳與水。
\item 部分酶促反應。
\eit
\sssc{非勻相催化}
催化劑與反應物不能混合成單一相而具有界面的催化。反應模式為,反應物擴散到催化劑表面並吸附於其表面,進行反應後離開,故反應速率正比於非勻相催化劑的有效接觸面積。通常較易回收重複使用,但通常催化效率較差。固態催化劑多製成多孔結構或粉末以增加接觸面積,如空氣過濾器、口罩、活性碳。如:
\bit
\item \ce{2H2O2(aq) ->[\ce{MnO2(s)}] 2H2O(l) + O2(g)}
\item \ce{2KClO3(aq) ->[\ce{MnO2(s)}] 2KCl(aq) + 3O2(g)}
\item 哈柏法(Haber process)製備氨。
\item 接觸法(Contact process)製備硫酸
\item 汽機車觸媒轉化器。
\item 鐵、鎳、銠、鈀、鉑、五氧化二釩等固相催化劑催化氫化反應。
\item 齊格勒-納塔催化劑(Ziegler–Natta catalyst)、菲力普斯催化劑(Phillips catalyst)/Phillips supported chromium 催化劑或茂金屬(Metallocene)催化劑催化烯類的加成聚合反應。
\eit

膠體溶液的分散質作為溶液中物質反應的催化劑屬於非勻相催化,但因接觸面積甚大,故多亦甚快。如:
\bit
\item 奈米顆粒催化反應。
\item 部分酶促反應。
\eit
\sssc{輔/助催化劑(Catalyst support)/載體(Carrier)}
增加非勻相催化反應接觸面積的物質,通常是比表面積較大的固體物質。
\sssc{光觸媒(Photocatalyst)}
指加速光化學(photochemical)反應的催化劑,而這種現象稱光催化(photocatalysis),如汽機車觸媒轉化器。
\sssc{自催化(Autocatalysis)}
指一個化學反應所生成之部分產物為該反應之催化劑。如:
\[\ce{5C2O4^{2-}(aq) + 2MnO4-(aq) + 16H+(aq) ->[\ce{Mn^{2+}(aq)}] 2Mn^{2+}(aq) + 10CO2(g) + 8H2O(l)}\]
\end{document}