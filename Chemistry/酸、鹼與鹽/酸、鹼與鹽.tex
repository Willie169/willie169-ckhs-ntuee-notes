\documentclass[a4paper,12pt]{article}
\setcounter{secnumdepth}{5}
\setcounter{tocdepth}{3}
\newcounter{ZhRenew}
\setcounter{ZhRenew}{1}
\newcounter{SectionLanguage}
\setcounter{SectionLanguage}{1}
\input{/usr/share/latex-toolkit/template.tex}
\begin{document}
\title{酸、鹼與鹽}
\author{沈威宇}
\date{\temtoday}
\titletocdoc
\section{酸、鹼與鹽(Acids, Bases, and Salts)}
\ssc{電解質(Electrolyte)}
\sssc{電解質(Electrolyte)與其強度(Strength)}
濃度指體積莫耳濃度 M=mol 溶質/L 溶液。
\begin{itemize}
\item 電解質(Electrolyte):指經由離子移動來導電的物質。
\item 溶液導電度約正比於溶解之離子總莫耳濃度。
\item 強(strong)電解質:完全解離之物質。強電解質之解離後離子濃度正比於其濃度,即解離度恆為 1。
\item 弱(weak)電解質:非強電解質的電解質。環境中相同離子濃度愈大,解離度愈小(同離子效應)。
\item 解離常數單位相同之弱電解質,其解離常數愈大,同溫同濃度下之解離度愈大。
\item 定溫下,給定弱電解質之濃度愈大,解離度愈小。
\item 解離後為二離子之弱電解質,解離度$\alpha$對總濃度$C$和解離常數$K$的函數為:
\[\alpha = \frac{\sqrt{K^2+4KC}-K}{2C}\]
解離後一種離子之濃度$M$對總濃度$C$和解離常數$K$的函數為:
\[M = \frac{\sqrt{K^2+4KC}-K}{2}\]
未解離之原弱電解質濃度$C_x$對總濃度$C$和解離常數$K$的函數為:
\[C_x = C-\frac{\sqrt{K^2+4KC}-K}{2}\]
若$C\gg K$(通常取$C\geq 1000 K$或$\alpha\leq 3\%$),常近似為:
\[\alpha=\sqrt{\frac{K}{C}}\]
\[M=\sqrt{KC}\]
\[C_x=C\]
\end{itemize}
\subsection{歷史上關於酸的學說}
\subsubsection{拉瓦節(Antoine Lavoisier)}
氧是酸中不可或缺的元素。(錯誤)
\subsubsection{戴維(Humphry Davy)}
氫是酸中不可或缺的元素。(對於布—洛酸正確,對於路易斯酸錯誤)
\subsubsection{李比希(Justus von Liebig)}
只有物質中的氫容易被金屬置換的含氫化合物才是酸性化合物。(對於阿瑞尼斯酸大致正確,對於路易斯酸錯誤)
\ssc{阿瑞尼斯(Svante August Arrhenius)的酸鹼電離論/電離說/阿瑞尼斯酸鹼學說(Arrhenius acid-base theory)}
\sssc{阿瑞尼斯酸鹼定義}
\bit
\item 僅適用於水溶液。
\item 凡是在水中可解離出\ce{H+(aq)}的物質為阿瑞尼斯酸。
\item 凡是在水中可解離出\ce{OH-(aq)}的物質為阿瑞尼斯鹼。
\item 阿瑞尼斯酸鹼中和指淨離子反應式為\ce{H+(aq) + OH-(aq) -> H2O(l)}的反應。
\eit
\sssc{阿瑞尼斯酸鹼的分類}
\begin{itemize}
\item 在水中可解離出$n$個\ce{H+}者稱n元(或質子)酸,$n=1$者稱單元酸/單質子酸,$n>1$者稱多元酸/多質子酸。
\item 在水中可解離出$n$個\ce{OH-}或接收$n$個質子者稱n元鹼,$n=1$者稱單元鹼,$n>1$者稱多元鹼。
\end{itemize}
\subsubsection{酸解離常數(Acid dissociation constant)}
酸\ce{HA$^{n+}$}的酸解離常數$K_a$為:
\[K_a=\frac{[\ce{H+}][\ce{A$^{(n-1)+}$}]}{[\ce{HA$^{n+}$}]}\]

$m$元酸\ce{H$_m$A$^{n+}$}的第$i$酸解離常數$K_{ai}$為:
\[K_{ai}=\frac{[\ce{H+}][\ce{H$_{m-i}$A$^{(n-i)+}$}]}{[\ce{H$_{m-i+1}$A$^{(n-i+1)+}$}]}\]
\subsubsection{鹼解離常數(Base dissociation constant)}
鹼\ce{B$^{n-}$}的鹼解離常數$K_b$為:
\[K_b=\frac{[\ce{HB$^{(n-1)-}$}][\ce{OH-}]}{[\ce{B$^{n-}$}]}\]

鹼\ce{BOH$^{n-}$}的鹼解離常數$K_b$為:
\[K_b=\frac{[\ce{B$^{(n-1)-}$}][\ce{OH-}]}{[\ce{BOH$^{n-}$}]}\]

$m$元鹼\ce{B$^{n-}$}的第$i$鹼解離常數$K_{bi}$為:
\[K_{bi}=\frac{[\ce{H$_i$B$^{(n-i)-}$}][\ce{OH-}]}{[\ce{B$^{(n-i+1)-}$}]}\]

$m$元鹼\ce{B(OH)$_m^{\phantom{m}n-}$}的第$i$鹼解離常數$K_{bi}$為:
\[K_{bi}=\frac{[\ce{B(OH)$_{m-i}^{\phantom{m-i}(n-i)-}$}][\ce{OH-}]}{[\ce{B(OH)$_{m-i+1}^{\phantom{m-i+1}(n-i+1)-}$}]}\]
\sssc{水的離子積(Ion product of water)}
水的解離反應:
\[\ce{H2O(l) + 57.27\tx{\ kJ} <=> H+(aq) + OH-(aq)}\]
水的離子積定義為:
\[K_w = [\ce{H+}][\ce{OH-}]\]
水的離子積等於水的解離常數乘以其體積莫耳濃度(假設溶質可忽略、密度恆為 1000 kg/m$^3$,則體積莫耳濃度恆為 55.$\bar{5}$ M)。因反應熱 +57.27 kJ,故離子積略正比於絕對溫度的對數,0°C 時 \( K_w = 1.3 \times 10^{-15} \)、25°C 時 \( K_w = 1 \times 10^{-14} \)、60°C 時 \( K_w = 1 \times 10^{-13} \)。25°C 時,水的解離常數為 \scinote{1.8}{-16} M、解離度為 \scinote{1.8}{-9}。
\subsubsection{pH 值/酸鹼值、pOH 值與 p$K_w$值}
索任生(Søren Peter Lauritz Sørensen)發明,p 代表$-\log$。
\[\text{pH}=-\log[\ce{H+}]\]
\[\text{pOH}=-\log[\ce{OH-}]\]
\[\text{p}K_w=-\log(K_w)=\text{pH}+\text{pOH}\]
\subsubsection{酸度(Acidity Grade)}
魯貝克(Henk van Lubeck)提出。
\[\tx{AG}=\log\left(\frac{[H^+]}{[OH^-]}\right)=\tx{pOH}-\tx{pH}\]
\subsubsection{阿瑞尼斯酸鹼的強度(Strength)}
單一物種之酸鹼強度依此。
\begin{itemize}
\item 阿瑞尼斯酸鹼理論以解離度的大小來定義酸鹼的強弱。
\item 強酸:每莫耳解離質子莫耳數大於等於一之物質。穩定者包含 \ce{HClO4}, \ce{HI}, \ce{HBr}, \ce{HCl}, \ce{H2SO4}, \ce{HNO3}。常近似為$n$[\ce{H+}]=[\ce{A^{n-}}]。
\item 強鹼:每莫耳解離氫氧根離子莫耳數大於等於一之物質。穩定者包含除了鋰外的鹼金屬的氫氧化物、\ce{Ca(OH)2}、\ce{Sr(OH)2}、\ce{Ba(OH)2}。常近似為$n$[\ce{OH-}]=[\ce{B^{n+}}]。
\item 弱酸:非強酸的酸。環境愈鹼,解離度愈高。如 \ce{CH3COOH}、\ce{H2CO3}、\ce{H2C2O4}、\ce{HF}(因鍵能大故不易解離)、\ce{HClO}、\ce{H3PO4}、\ce{H2S}、\ce{H2SO3}、\ce{HNO2}、\ce{HCN}。
\item 弱鹼:非強鹼的鹼。環境愈酸,解離度愈高。如 \ce{NH3}、\ce{NaHCO3}、\ce{Na2CO3}、\ce{Mg(OH)2}、\ce{Al(OH)3}、\ce{Fe(OH)3}。
\end{itemize}
\sssc{阿瑞尼斯酸的命名}
\begin{itemize}
\item 有機酸:依有機化合物命名。
\item 無機非含氧酸:氣相稱某化氫者,水溶液稱氫某酸。
\item 無機含氧酸:
\begin{itemize}
\item 中心原子(團)符合其價電子數者或最為人熟知者稱某酸,硝酸例外。
\item 中心原子氧化數較某酸高者稱過某酸、氧化數較某酸低者稱亞某酸、氧化數較亞某酸低者稱次某酸,某為主族元素者通常氧化數差二,過錳酸差一。
\end{itemize}
\end{itemize}
\sssc{阿瑞尼斯鹼的命名}
\begin{itemize}
\item 有機鹼:依有機化合物命名。
\item 氨\ce{NH3}。
\item 含氫氧根離子鹼:
\begin{itemize}
\item 金屬符合其價電子數者或最為人熟知者稱氫氧化某。
\item 氫氧化某離子稱氫氧化某,如氫氧化亞鐵,或附其羅馬數字價數於金屬名稱後,如氫氧化鐵(II)。
\end{itemize}
\end{itemize}
\sssc{水溶液之酸鹼性定義}
\begin{itemize}
\item 酸性水溶液:pH<pOH
\item 中性水溶液:pH=pOH
\item 鹼性水溶液:pH>pOH
\end{itemize}
\sssc{酸鹼溶液中的反應}
酸性溶液中的反應式寫法反應物與生成物不可包含\ce{OH-},應以\ce{H+}與\ce{H2O}平衡之;鹼性溶液中的反應式寫法反應物與生成物不可包含\ce{H+},應以\ce{OH-}與\ce{H2O}平衡之。
\sssc{阿瑞尼斯酸水溶液通性}
\bit
\item 為電解質。
\item 可與碳酸((氫)鹽)水溶液反應生成二氧化碳。
\item 可與亞硫酸((氫)鹽)水溶液反應生成二氧化硫。
\item 可與硫化氫/金屬硫(氫)化物水溶液反應生成硫化氫。
\item 可與還原電位小於氫的金屬反應生成氫氣。
\item 可與阿瑞尼斯鹼中和。
\item 一般具酸味、具腐蝕性。
\item 含氧酸一般中心原子之氧化數愈大酸性愈強,但酸解離常數次磷酸>亞磷酸>磷酸(同濃度酸性強弱有稱次磷酸>亞磷酸>磷酸與亞磷酸>次磷酸>磷酸者)。
\eit
\sssc{阿瑞尼斯鹼水溶液通性}
\bit
\item 為電解質。
\item 可與氨(鹽)水溶液反應生成氨。
\item 可與兩性金屬反應生成氫氣。
\item 可與阿瑞尼斯酸中和。
\item 一般具苦澀味、具滑膩感、具腐蝕性。
\eit
\sssc{莫耳中和熱(molar heat of neutralization)}
指酸鹼中和後產生鹽類和一莫耳水所放出的熱。25°C、1 bar 下,強酸與強鹼中和之莫耳中和熱必為 57.27 kJ,若有一者為弱酸或弱鹼,則莫耳中和熱必小於 57.27 kJ。若形成沉澱,其另吸放熱不計入中和熱。
\ssc{布侖斯惕(Johannes Nicolaus Brønsted)與洛瑞(Thomas Martin Lowry)的酸鹼質子理論/布侖斯惕-洛瑞酸鹼理論/布—洛酸鹼學說(Brønsted–Lowry acid–base theory)/質子說/共軛(Conjugate)酸鹼對學說}
不限於水溶液。
\sssc{布侖斯惕-洛瑞酸鹼/布—洛酸鹼定義與共軛酸鹼對}
\bit
\item 不限於水溶液,基於反應定義。
\item 凡是在反應中釋放質子的物質(質子予體)為布—洛酸。
\item 一布—洛酸之共軛鹼(Conjugate base)為其釋放一個質子所形成之物質。
\item 凡是在反應中接受質子的物質(質子受體)為布—洛鹼。
\item 一布—洛鹼之共軛酸(Conjugate acid)為其接受一個質子所形成之物質。
\item 互為共軛酸與共軛鹼的物質為共軛酸鹼對。
\item 布—洛酸鹼中和指涉及質子之轉移的反應。
\eit
\sssc{布—洛酸鹼的性質}
\bit
\item 凡阿瑞尼斯酸必為布—洛酸。
\item 凡阿瑞尼斯鹼必為布—洛鹼。
\item 布—洛酸通常可作為氫鍵供體(如醇、阿瑞尼斯酸)。
\item 布—洛鹼通常具有孤電子對(如陰離子)、π 電子(如烯)或自由電子(如金屬)。
\item 一共軛酸鹼對之共軛酸之酸解離常數與共軛鹼之鹼解離常數之積等於水的離子積。
\eit
\sssc{兩性(Amphoteric)物質}
指可作為布—洛酸亦可作為布—洛鹼之物質。如:\ce{H2O}, \ce{Be}, \ce{Be(OH)2}, \ce{Al}, \ce{Al2O3}, \ce{Al(OH)3}, \ce{Ga}, \ce{Ga2O3}, \ce{Ga(OH)3}, \ce{Sn}, \ce{SnO}, \ce{Sn(OH)2}, \ce{Pb}, \ce{PbO}, \ce{Pb(OH)2}, \ce{Cr}, \ce{Cr(H2O)3(OH)3}, \ce{Cr2O3}, \ce{Zn}, \ce{ZnO}, \ce{Zn(OH)2}, \ce{HSO4-}, \ce{HS2O3-}, \ce{HSO3-}, \ce{HCO3-}, \ce{HS-}, \ce{H2PO3-}, \ce{H2PO4^{2-}}, \ce{H3PO4-}, \ce{HC2O4-}, \ce{H2AsO4-}, \ce{HAsO4^{2-}}, \ce{C6H4(COOH)(COO-)}, 胺基酸。
\subsubsection{布—洛酸鹼的強度(Strength)}
化學反應中之酸鹼強度依此。

布—洛酸鹼理論以提供與接受質子之能力的強弱來定義酸鹼的強弱。相等當量濃度之物質甲與物質乙,凡物質甲自物質乙獲得質子之反應速率大於其逆反應之速率,稱甲的鹼性強於乙之共軛鹼的鹼性,稱甲之共軛酸的酸性弱於乙的酸性。

由於在水溶液系統中阿瑞尼斯酸或鹼強度愈大者反應時提供或接受質子的能力就愈強,因此阿瑞尼斯酸鹼理論與布—洛酸鹼理論得到的酸鹼強度之排序是一致的。
\ssc{路易斯(Gilbert Newton Lewis)的路易斯酸鹼理論(Lewis acid–base theory)}
\sssc{路易斯酸鹼定義}
\bit
\item 不限於質子之轉移。
\item 凡是能夠接受電子對的物質為路易斯酸。
\item 凡是能夠供給電子對的物質為路易斯鹼。
\item 路易斯酸鹼中和指路易斯鹼提供電子對給路易斯酸形成配位共價鍵的反應,其生成物稱路易斯加合物(Lewis adduct)。
\eit
\sssc{路易斯酸鹼的性質}
\bit
\item 凡布—洛酸必為路易斯酸。
\item 凡布—洛鹼必為路易斯鹼。
\item 路易斯酸通常具有空軌域、缺電子(electron-deficient)(如陽離子、\ce{BF3})或可作為氫鍵供體。
\item 路易斯鹼通常具有孤電子對、π 電子或自由電子。
\eit
\subsection{鹽(Salt)}
酸鹼中和形成的離子化合物。
\sssc{單鹽(Simple salt)}
酸鹼中和時,酸的氫離子被其他陽離子取代或鹼的氫氧離子被其他陰離子取代,依其取代程度,可將單鹽分為正鹽、酸式鹽與鹼式鹽。
\subsubsection{正鹽(Neutral salt)}
酸中可游離的氫離子全部被金屬離子或銨離子取代。非含氧酸形成者稱某化某,含氧酸形成者稱某酸某。
\subsubsection{酸式鹽(Acid salt)}
其中還有可解離的氫離子,由多元酸與鹼反應產生。如酸性的亞硫酸氫鈉/酸式亞硫酸鈉\ce{NaHSO3}、酸性的硫酸氫鈉/酸式硫酸鈉\ce{NaHSO4}、酸性的磷酸二氫鈉\ce{NaH2PO4}、酸性的亞磷酸氫鈉/酸式亞磷酸鈉\ce{NaH2PO3}、鹼性的碳酸氫鈉/酸式碳酸鈉\ce{NaHCO3}、鹼性的硫化氫鈉/酸式硫化鈉\ce{NaHS}、鹼性的磷酸氫二鈉\ce{Na2HPO4}。
\subsubsection{鹼式鹽(Alkali/base salt)}
其中還有可解離的氫氧離子,由多元鹼與酸反應產生。如氯化氫氧鈣/鹼式氯化鈣\ce{Ca(OH)Cl}、醋酸氫氧銅/鹼式醋酸銅\ce{Cu(OH)(CH3COO)}、碳酸二氫氧銅/鹼式碳酸銅\ce{Cu2(OH)2CO3}、水合二碳酸六氫氧鋅/鹼式碳酸鋅\ce{Zn5(OH)6(CO3)2$\cdot$H2O}、二碳酸二氫氧鉛/鹼式碳酸鉛\ce{Pb3(OH)2(CO3)2}、硝酸氫氧鉍\ce{Bi(OH)(NO3)2}、硝酸二氫氧鉍\ce{Bi(OH)2NO3}。
\subsubsection{複鹽(Double salt)}
指含有兩種或以上非氫氧根離子之陰離子或含有兩種或以上非氫離子之陽離子,但有固定組成的晶型的簡單鹽類,復溶於水會離解出所有的簡單離子,屬於化合物。通常由不同種鹽類混合後再結晶形成。如明礬\ce{KAl(SO4)2 $\cdot $ 12H2O}、硫酸銨鎂\ce{MgNH4PO4}、硫酸銨亞鐵\ce{Fe(NH4)2(SO4)2}。
\subsubsection{錯鹽(Complex salt)}
含有錯離子的鹽類,屬於配位化合物,某些錯鹽與複鹽類似,其組成陽離子(通常為過渡金屬)或陰離子不只一種,但錯鹽溶於水會解離出錯離子。如黃血鹽/亞鐵氰化鉀/六氰亞鐵酸鉀\ce{K4[Fe(CN)6](s)}溶於水解離出\ce{4K+(aq)}與六氰亞鐵酸根錯離子\ce{[Fe(CN)6]^{4-}(aq)}、赤血鹽/鐵氰化鉀/六氰鐵酸鉀\ce{K3[Fe(CN)6](s)}溶於水解離出六氰鐵酸根錯離子\ce{[Fe(CN)6]^{3-}(aq)}與鉀離子、氯化二氨銀\ce{Ag(NH3)2Cl(s)}溶於水解離出銀氨錯離子\ce{Ag(NH3)2+(aq)}與氯離子、硫酸四氨銅\ce{Cu(NH3)4SO4(s)}溶於水解離出銅氨錯離子\ce{Cu(NH3)4^{2+}(aq)}與硫酸根離子、六氟鋁酸鈉\ce{Na3AlF6(s)}溶於水解離出六氟鋁酸根錯離子\ce{AlF6^{3-}(aq)}與鈉離子。
\subsubsection{正鹽水溶液的酸鹼性}
\begin{itemize}
\item 若陽離子會水解(Hydrolysis)或解離產生質子,其$K_a$為$K_w$除以原該鹼之$K_b$,如\ce{NH4+}之$K_a$為$K_w$除以\ce{NH3}之$K_b$。
\item 若陰離子會水解或解離產生氫氧離子,其$K_b$為$K_w$除以原該酸之$K_a$,如\ce{F-}之$K_b$為$K_w$除以\ce{HF}之$K_a$。
\item 若陽離子的$K_a$大於陰離子的$K_b$,即陰離子之原酸之$K_a$大於陽離子之原鹼之$K_b$,或陰離子不產生氫氧離子但陽離子產生質子,水溶液為酸性,每莫耳的[\ce{H+}]為陽離子(有多個應同計)的$K_a$減去陰離子(有多個應同計)的$K_b$。
\item 若陽離子的$K_a$小於陰離子的$K_b$,即陰離子之原酸之$K_a$小於陽離子之原鹼之$K_b$,或陽離子不產生質子但陰離子產生氫氧離子,水溶液為鹼性,每莫耳的[\ce{OH-}]為陰離子(有多個應同計)的$K_b$減去陽離子(有多個應同計)的$K_a$。
\item 若陽離子的$K_a$等於陰離子的$K_b$,即陰離子之原酸之$K_a$等於陽離子之原鹼之$K_b$,或陽離子不產生質子且陰離子不產生氫氧離子,水溶液為中性。
\end{itemize}
\subsubsection{酸式鹽水溶液的酸鹼性}
\begin{itemize}
\item 陽離子同正鹽。
\item 陰離子之等效$K_b$等於$\sum_{i=1}^m\prod_{j=1}^iK_{bi}-\sum{i=1}^n\prod_{j=1}^iK_{ai}$,其中$m$為原(無法為共軛鹼的)共軛酸要變成該陰離子所需解離的質子數,$n$為該陰離子之可解離氫離子數,$K_{bi}$等於$K_w$除以$K_{a(m-i+1)}$,$K_{ai}$等於原共軛酸之$K_{ai+m}$。
\item 酸鹼判斷同正鹽,惟陰離子之$K_b$為其等效$K_b$。
\item 如\ce{NaHSO3},\ce{Na+}不水解,\ce{HSO3-}之$K_b$即$K_w$除以\ce{H2SO3}之$K_{a1}$,$K_a$即\ce{H2SO3}之$K_{a2}$,因$K_a>K_b$故呈酸性。
\item 如\ce{NaHCO3},\ce{Na+}不水解,\ce{HCO3-}之$K_b$即$K_w$除以\ce{H2CO3}之$K_{a1}$,$K_a$即\ce{H2CO3}之$K_{a2}$,因$K_a<K_b$故呈鹼性。
\end{itemize}
\subsubsection{鹼式鹽水溶液的酸鹼性}
酸式鹽之相反。
\subsubsection{複鹽水溶液的酸鹼性}
各陽離子有(等效)$K_a$者相加,各陰離子有(等效)$K_b$者相加,餘同單鹽。
\subsubsection{錯鹽水溶液的酸鹼性}
同複鹽。
\subsubsection{常見離子水溶液的酸鹼性}
不水解者均為中性。
\begin{longtable}[c]{|p{0.1\tw}|p{0.4\tw}|p{0.3\tw}|}
\hline
行為 & 陽離子 & 陰離子 \\\hline\endhead
中性 & 氫氧化物易溶之金屬離子:\ce{Na+}, \ce{K+}, \ce{Rb+}, \ce{Cs+}, \ce{Ca^{2+}}, \ce{Sr^{2+}}, \ce{Ba^{2+}} & 強酸根離子:\ce{ClO4-}, \ce{I-}, \ce{Br-}, \ce{Cl-}, \ce{SO4^{2-}}, \ce{NO3-} \\\hline
阿瑞尼斯酸 & \ce{NH4+}, 氫氧化物難溶之金屬離子 & 中度酸的酸氫根離子:\ce{HSO4-}, \ce{HSO3-}, \ce{H2PO4-}, \ce{H2PO3-}, \ce{HC2O4-} \\\hline
阿瑞尼斯鹼 & 無 & 弱酸酸根離子、極弱酸的酸氫根離子:\ce{HPO4^{2-}}, \ce{HCO3-}, \ce{HS-} \\\hline
\end{longtable}\FB
\subsection{緩衝溶液(Buffer solution)}
\subsubsection{組成與性質}
緩衝溶液是一對共軛酸鹼對同時顯著存在的溶液,或兩性物質顯著存在的溶液,加水稀釋後 pH 值幾乎不變,加入供質子莫耳數小於共軛鹼莫耳數之強酸或受質子莫耳數小於共軛酸莫耳數之強鹼後 pH 值變化很小。
\subsubsection{亨德森-哈塞爾巴爾赫方程式(Henderson-Hasselbalch equation)}
忽略水的解離並假設活性係數為$1$下,一對共軛酸 A 與共軛鹼 B 同時顯著存在的緩衝溶液,A 之酸解離常數 $K_a$,其 pH 值為:
\[\text{pH}=\text{p}K_a+\log\left(\frac{\tx{[B]}}{\tx{[A]}}\right)\]
\subsubsection{緩衝能力(Buffer capacity)}
$\beta$,令加入之強酸或強鹼之最大供或受質子莫耳數$n$:
\[\beta=\abs{\dv{n}{(\text{pH})}}\]
\subsubsection{酸型緩衝溶液}
弱酸及其共軛鹼同時顯著存在的溶液。

配置方法:
\begin{itemize}
\item 弱酸加其可溶鹽:如\ce{CH3COOH + CH3COONa}、\ce{HCN + NaCN}。
\item 過量弱酸加少量鹼:如過量\ce{CH3COOH}加少量\ce{NaOH}、過量\ce{HCN}加少量\ce{NaOH}
\item 過量弱酸之可溶鹽加少量酸:如過量\ce{CH3COONa}加少量\ce{HCl}
\end{itemize}
\subsubsection{鹼型緩衝溶液}
弱鹼及其共軛酸同時顯著存在的溶液。

配置方法:
\begin{itemize}
\item 弱鹼加其可溶鹽:如\ce{NH3 + NH4Cl}
\item 過量弱鹼加少量酸:如過量\ce{NH3}加少量\ce{HCl}
\item 過量弱鹼之可溶鹽加少量鹼:如過量\ce{NH4Cl}加少量\ce{NaOH}
\end{itemize}
\subsubsection{兩性型緩衝溶液}
配置方法:
\begin{itemize}
\item 弱酸與弱鹼之可溶鹽:如\ce{NH4CN}、\ce{Cu(CH3COO)2}
\item 弱酸之可溶鹽加弱鹼之可溶鹽:如\ce{NaCN + NH4Cl}
\item 可溶酸式鹽:如\ce{NaHCO3}、\ce{Na2HPO4}、\ce{NaH2PO4}
\item 可溶鹼式鹽:如\ce{Bi(OH)2NO3}
\item 多質子酸加其可溶鹽:如\ce{H2CO3 + NaHCO3}、\ce{H2CO3 + NaCO3}、\ce{H2SO3 + Na2SO3}、\ce{H3PO4 + NaH2PO4}、\ce{H3PO4 + Na2HPO4}
\item 不同氫數之多元酸(氫)鹽:如\ce{NaHCO3 + NaCO3}、\ce{Na2HPO4 + NaH2PO4}、\ce{Na3PO4 + NaH2PO4}
\item 可溶兩性分子:如胺基酸
\end{itemize}
\sssc{生物緩衝溶液}
人類血液即為極佳的緩衝溶液,有碳酸/碳酸氫根、磷酸二氫根/磷酸氫根等多對共軛酸鹼對,使 pH 值維持在 7.40$\pm$0.05。
\subsection{酸鹼滴定(Acid-base titration)}
\subsubsection{滴定過程中 pH 值對時間曲線}
\begin{itemize}
\item 強酸滴定強鹼或反之:起初改變較慢,而後接近中性處改變極快,其中達當量點時為中性,曲線旋轉對稱於當量點。淨反應式(忽略沉澱):\ce{H+(aq) + OH-(aq) -> H2O(l)}。
\item 強鹼滴定弱酸:起初改變稍慢,隨弱酸之共軛鹼增加,形成緩衝溶液之現象,改變極慢,而後隨著共軛酸耗盡在接近當量點處改變極快,其中達當量點時為弱鹼。淨反應式:弱酸(aq)+\ce{OH-(aq) -> }弱酸之共軛鹼(aq)+\ce{H2O(l)}。
\item 強酸滴定弱鹼:起初改變稍慢,隨弱鹼之共軛酸增加,形成緩衝溶液之現象,改變極慢,而後隨著共軛鹼耗盡在接近當量點處改變極快,其中達當量點時為弱酸。淨反應式:\ce{H+(aq) + }弱鹼(aq)\ce{->}弱鹼之共軛酸(aq)+\ce{H2O(l)}。
\end{itemize}
\subsubsection{當量點與半當量點濃度}
令滴定過程溫度與壓力不變,水的離子積$K_w$,待測液當量濃度$C_l\gg\sqrt{K_w}$、體積$V_l$、解離常數$K_l$,滴定液當量濃度$C_t\gg\sqrt{K_w}$、達半當量點與當量點時已加入體積分別為$V_H$與$V_E$,達半當量點與當量點時已提供的滴定液溶質當量數(滴定液為鹼則當量為該溶質式量除以每莫耳該溶質可接收質子莫耳數;滴定液為酸則當量為該溶質式量除以每莫耳該溶質可解離質子莫耳數)分別為$C_H\cdot (V_l+V_H)$與$C_E\cdot (V_l+V_E)$,達半當量點與當量點時待測液提供之反應物(滴定液為鹼則為質子,滴定液為酸則為氫氧根離子)當量濃度分別為$R_H$與$R_E$:
\begin{itemize}
\item 任意酸鹼滴定:
\[2V_H=V_E=\frac{C_tV_l}{C_l}\]
\[C_H=\frac{C_lV_l}{2(V_l+V_H)}=\frac{C_l^{\phantom{l}2}}{2C_l+C_t}\]
\[C_E=\frac{C_lV_l}{V_l+V_E}=\frac{C_l^{\phantom{l}2}}{C_l+C_t}\]
\item 強鹼滴定強酸或強酸滴定強鹼:
\[R_H=C_H\]
\[R_E=\sqrt{K_w}\]
\item 強鹼滴定弱酸或強酸滴定弱鹼:
\[R_H=K_l\]
\[R_E=\sqrt{\frac{K_wK_l}{C_E}}\]
\begin{proof}\mbox{}\\
令待測液提供的非質子或氫氧跟離子之離子(包含溶液中與沉澱)當量數$C\cdot (V_l+V_H)$。
\[C_E+R_E=\frac{K_w}{R_E}+C\]
\[\frac{R_EC}{C_E-C}=K_l\]
\[R_E^{\phantom{E}3}+(C_E+K_l)R_E^{\phantom{E}2}-K_wR_E-K_wK_l=0\]
\[C_ER_E^{\phantom{E}2}-K_wR_E-K_wK_l=0\]
\[R_E=\sqrt{\frac{K_wK_l}{C_E}}\]
\end{proof}
\end{itemize}
\subsubsection{酸鹼滴定實驗}
\bit
\item 酸滴定鹼,先精稱碳酸鈉,使完全溶解後,以之標定鹽酸,再以標定後的鹽酸滴定未知鹼。
\item 鹼滴定酸,先精稱鄰苯二甲酸氫鉀(KHP),使完全溶解後,以之標定氫氧化鈉,再以標定後的氫氧化鈉滴定未知酸。
\item 強酸滴定弱鹼,選用弱酸性範圍指示劑;強鹼滴定弱酸;選用弱鹼性範圍指示劑;強酸滴定強鹼或強鹼滴定強酸,選用弱酸至弱鹼範圍指示劑皆可。
\eit
\subsection{酸鹼指示劑(Acid-base indicator)}
\sssc{酸鹼指示劑(Acid-base indicator)}
酸鹼指示劑為與其共軛鹼顏色不同的有機弱酸或有機與其共軛酸顏色不同的弱鹼,在不同酸鹼性環境下轉化成相應的酸式或鹼式而顯示不同顏色。若變色範圍(pH 值)為$x$到$y$,小於等於$x$為酸式顏色,大於等於$y$為鹼式顏色,$x$、$y$之間為兩者混合之顏色,且愈接近$x$者顏色愈接近酸式顏色、愈接近$y$者顏色愈接近鹼式顏色。
\sssc{常見酸鹼指示劑}
\begin{longtable}[c]{|p{0.5\textwidth}|p{0.1\textwidth}|p{0.1\textwidth}|p{0.1\textwidth}|}
\hline
\textbf{指示劑名稱} & \textbf{酸式顏色} & \textbf{鹼式顏色} & \textbf{變色範圍(pH 值)} \\\hline
\endhead
石蕊(Litmus) & 紅 & 藍 & 4.5-8.3 \\\hline
酚酞(Phenolphthalein) & 無 & 紅 & 8.3-10.0 \\\hline
酚紅(Phenol Red) & 黃 & 品紅 & 6.8-8.2 \\\hline
百里酚藍(Thymol blue)第一變色範圍 & 紅 & 黃 & 1.2-2.8 \\\hline
百里酚藍(Thymol blue)第二變色範圍 & 黃 & 藍 & 8.0-9.6 \\\hline
溴酚藍(Bromophenol Blue) & 黃 & 藍 & 3.0-4.6 \\\hline
溴百里酚藍/溴瑞香草酚藍(Bromothymol Blue, BTB) & 黃 & 藍 & 6.0-7.6 \\\hline
甲基紅(Methyl Red) & 紅 & 黃 & 4.4-6.2 \\\hline
甲基橙(Methyl Orange) & 紅 & 黃 & 3.1-4.4 \\\hline
甲基黃(Methyl Yellow) & 紅 & 黃 & 2.9-4.0 \\\hline
結晶紫(Crystal Violet)/甲基紫 10B(Methyl Violet 10B)/龍膽紫(Gentian Violet) & 黃 & 紫 & -1.0-2.0 \\\hline
甲酚紅(Cresol Red) & 黃 & 品紅 & 7.2-8.8 \\\hline
溴甲酚綠(Bromocresol Green, BCG) & 黃 & 藍 & 3.8-5.4 \\\hline
溴甲酚紫(Bromocresol Purple, BCP) & 黃 & 洋紅 & 5.2-6.8 \\\hline
剛果紅(Congo Red) & 藍 & 紅 & 3.0-5.2 \\\hline
中性紅(Neutral Red) & 紅 & 黃 & 6.8-8.0 \\\hline
孔雀綠(Malachite Green)第一變色範圍 & 黃 & 綠 & 0.2-1.8 \\\hline
孔雀綠(Malachite Green)第二變色範圍 & 綠 & 無 & 11.5-13.2 \\\hline
酸性橙 5(Acid orange 5)/金蓮花素 OO(Tropaeolin OO) & 紅 & 黃 & 1.4-3.2 \\\hline
茜素黃 R(Alizarine Yellow R) & 黃 & 紅 & 10.1-12.0 \\\hline
\end{longtable}\FloatBarrier
\sssc{廣用指試劑/通用指示劑(Universal Indicator)}
酸鹼指示成分包含酚酞、甲基紅、甲基橙、溴百里酚藍與百里酚藍,可測量 pH 值 1.2 至 10.0,自酸至鹼呈紅色至紫色,波長漸短,pH=7 時約為黃色。
\end{document}