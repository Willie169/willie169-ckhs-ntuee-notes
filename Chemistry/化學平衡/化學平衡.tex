\documentclass[a4paper,12pt]{article}
\setcounter{secnumdepth}{5}
\setcounter{tocdepth}{3}
\newcounter{ZhRenew}
\setcounter{ZhRenew}{1}
\newcounter{SectionLanguage}
\setcounter{SectionLanguage}{1}
\input{/usr/share/LaTeX-ToolKit/template.tex}
\begin{document}
\title{化學平衡}
\author{沈威宇}
\date{\temtoday}
\titletocdoc
\section{化學平衡(Chemical Equilibrium)}
\ssc{可逆反應(Reversible reaction)}
指反應物間能互相反應形成生成物,生成物間亦能互相反應形成反應物的反應。由熱力學可知,所有化學反應都是可逆的。惟有時將其中一側反應速率遠大於另一側之反應式微不可逆反應。
\subsection{化學平衡(Chemical equilibrium)}
指一孤立系統中所有化學反應的正逆反應速率(取同一物質)均相等的狀態,此時反應物和生成物濃度、分壓、顏色、體積、溫度等巨觀性質不再改變,但正逆反應速率均大於零,呈現動態平衡(dynamic equilibrium)。反應物耗盡不是該反應達成化學平衡。
\ssc{熱力(學)控制(Thermodynamic control)與動力(學)控制(Kinetic control)}
\bit
\item 若一化學反應,其正逆反應的活化能都較低(相對於當時的內能而言),容易達成化學平衡,則稱熱力學控制。
\item 若一化學反應,其逆反應的活化能較高(相對於當時的內能而言)而反應速率極慢,難以達成化學平衡,則稱動力學控制。
\eit
\ssc{平衡常數(Equilibrium constant)}
由伯特洛(Berthelot)首先發現此性質,由古伯格(Guldberg)和瓦格(Waage)於1893年提出並發展。
\sssc{(化學)平衡定律式/平衡常數表示式}
令孤立系統中一可逆反應:
\[\sum_{i=1}^ma_i\mathbf{A}_i\ce{<=>}\sum_{i=i}^nb_i\mathbf{B}_i\]
其中$(\mathbf{A}_i)_{i=1}^p$與$(\mathbf{B}_i)_{i=1}^q$是可變濃度物質,其餘為不可變濃度物質,並假設所有不可變濃度物質均不影響化學平衡。

則平衡定律式為:
\[K=\frac{\prod_{i=1}^q\qty(\lim_{t\to\infty}[\mathbf{B}_i])^{b_i}}{\prod_{i=1}^p\qty(\lim_{t\to\infty}[\mathbf{A}_i])^{a_i}}\]

其中:
\begin{itemize}
\item $\lim_{t\to\infty}[\mathbf{A}]$指平衡狀態時$\mathbf{A}$的體積莫耳濃度,並可以替換成任意與之正比的度量。
\item $a_i$:反應式中反應物$\mathbf{A}_i$的平衡係數。
\item $b_i$:反應式中生成物$\mathbf{B}_i$的平衡係數。
\eit
\sssc{質量作用定律(Law of mass action)}
由古伯格和瓦格提出。指出,反應之平衡常數等於正反應速率常數除以逆反應速率常數,等於平衡狀態生成物濃度之係數次方之積除以反應物濃度之係數次方之積。
\begin{proof}\mbox{}\\
令一基元反應,各反應物濃度的係數次方之積對時間的函數$\mathbf{A}(t)$,各生成物濃度的係數次方之積對時間的函數$\mathbf{B}(t)$,以$\mathbf{A}$消耗速率為反應速率的正反應速率常數$j$,以$\mathbf{A}$生成速率為反應速率的逆反應速率常數$k$。有:
\[\frac{\mathrm{d}\mathbf{A}}{\mathrm{d}t}=k\mathbf{B}-j\mathbf{A}\]
令:
\[\mathbf{A}(0)=a\]
\[\mathbf{B}(0)=b\]
依據質量守恆定律:
\[\mathbf{A}+\mathbf{B}=a+b\]
求$\mathbf{A}(t)$、$\mathbf{B}(t)$和平衡常數$K=\frac{\lim_{t\to\infty}\mathbf{B}}{\lim_{t\to\infty}\mathbf{A}}$:
\[\frac{\mathrm{d}\mathbf{A}}{\mathrm{d}t}=-(j+k)\mathbf{A}+k(a+b)\]
令:
\[\mathbf{A}=ce^{-(j+k)t}+d\]
\[-c(j+k)e^{-(j+k)t}=-c(j+k)e^{-(j+k)t}-d(j+k)+k(a+b)\]
\[d=\frac{k(a+b)}{j+k}\]
\[c+d=a\]
\[c=\frac{ja-kb}{j+k}\]
\[\mathbf{A}=\frac{ja-kb}{j+k}e^{-(j+k)t}+\frac{k(a+b)}{j+k}\]
\[\mathbf{B}=a+b-\mathbf{A}=-\frac{ja-kb}{j+k}e^{-(j+k)t}+\frac{j(a+b)}{j+k}\]
\[\lim_{t\to\infty}\mathbf{A}=\frac{k(a+b)}{j+k}\]
\[\lim_{t\to\infty}\mathbf{B}=\frac{j(a+b)}{j+k}\]
\[K=\frac{\lim_{t\to\infty}\mathbf{B}}{\lim_{t\to\infty}\mathbf{A}}=\frac{j}{k}\]
由於基元反應係數即級數,連乘後全反應的平衡定律式必可將級數次方約分為係數次方。
\end{proof}
\subsubsection{說明}
\bit
\item 正反應係數和$x$、逆反應係數和$y$之可逆反應,平衡常數之單位為$M^{x-y}$。平衡常數的單位常省略不寫。
\item 反應式逆寫,平衡常數變為原平衡常數之倒數;反應式乘以$n$倍,平衡常數變為原平衡常數之$n$次方;反應式相加,平衡常數相乘;反應式減去另一反應式,平衡常數除以該反應式的平衡常數。
\item 反應式兩側有相同物質但不同相時不可相約分,例如碘的萃取:
\[\ce{I2(aq) <=> I2(CClO4)}\]
\item 對於一個可逆反應,所有相同溫度的平衡狀態中,都有相同的平衡常數,即一可逆反應之平衡常數僅依賴於溫度與反應本身,不依賴於反應機構、各物質濃度。
\item 對於孤立系統中的一個可逆反應,若兩初始狀態依反應式當量比在兩側間移動使有相同的反應進度(計算時常移動到使其中一側至少一個反應物耗盡),並使溫度與壓力相同或溫度與體積相同,可以得到相同的狀態,則該二初始狀態之平衡狀態相同。
\eit
\sssc{溫度效應}
令正反應活化能$E_a$、指數前因子$A$,逆反應活化能$E_r$、指數前因子$B$,阿瑞尼斯方程與質量作用定律指出,平衡常數$K$近似為:
\[K=\frac{Ae^{-\frac{E_a}{RT}}}{Be^{-\frac{E_r}{RT}}}=\frac{A}{B}e^{-\frac{\Delta H}{RT}}\]
\sssc{數值近似}
一般計算時:
\begin{itemize}
\item 平衡常數$>10^5$可視為反應不可逆向右。
\item 平衡常數$<10^{-5}$可視為反應不可逆向左。
\item 量級差 5 以上之數相加減,小者可視為0。
\end{itemize}
\ssc{非平衡狀態}
\sssc{反應商(Reaction quotient)}
反應商$Q$定義為將當下狀態之濃度代入平衡常數表示式中,即:
\[K=\frac{\prod_{j=1}^q[\mathbf{B}_j]^{d_j}}{\prod_{i=1}^p[\mathbf{A}_i]^{c_i}}\]
\begin{itemize}
\item $Q<K\iff$正反應速率大於逆反應速率,反應向右進行。
\item $Q=K\iff$正反應速率等於逆反應速率,反應處於平衡狀態。
\item $Q>K\iff$正反應速率小於逆反應速率,反應向左進行。
\end{itemize}
\sssc{勒沙特列原理(Le Chatelier principle)}
化學平衡達成後,若改變系統中的某些因素,如濃度、壓力、溫度,往往會破壞平衡狀態,此時反應就會朝向試圖抵銷造成平衡狀態破壞的改變的方向移動。
\begin{itemize}
\item 對於濃度,可以平衡常數求出。
\item 對於溫度,放熱反應溫度升高時平衡常數變小,吸熱反應溫度升高時平衡常數變大。
\item 對於改變壓力、體積、填充或取出反應物、生成物或不活潑氣體等,可分別就濃度與溫度因此的改變討論。
\end{itemize}
\ssc{各種平衡常數與其衍生常數}
\sssc{莫耳分率平衡常數(Mole fraction equilibrium constant)}
$K_x$,指所有濃度皆以莫耳分率表示時的平衡常數。
\sssc{濃度平衡常數(Concentration equilibrium constant)}
$K_c$,指所有濃度皆以體積莫耳濃度表示時的平衡常數。
\sssc{壓力平衡常數(Pressure equilibrium constant)}
$K_p$,指所有濃度皆以(有效)分壓表示時的平衡常數。

令孤立系統中一只有一個速率決定步驟的可逆反應:
\[\sum_{i=1}^ma_i\mathbf{A}_i\ce{<=>}\sum_{i=i}^nb_i\mathbf{B}_i\]
其中$(\mathbf{A}_i)_{i=1}^p$與$(\mathbf{B}_i)_{i=1}^q$是可變濃度物質,其餘為不可變濃度物質,並假設所有有效接觸面積不變。

\[K_p=K_c(RT)^{\sum_{i=1}^qb_i-\sum_{i=1}^pa_i}\]
\sssc{解離常數(Dissociation constant)}
$K_h$,指解離/分解(Dissociation)反應的平衡常數。
\subsubsection{同離子效應(Common ion effect)}
指有相同種類離子在溶液中時,解離時會出現該種離子之物質解離度降低,如氯化銀在硝酸銀或氯化鈉溶液中溶解度降低。
\subsubsection{溶(解)度積(常數)(Solibility product (constant))}
$K_{sp}$,指溶解反應的濃度平衡常數。
\subsubsection{離子積(Ion product)}
$Q$,溶解度積常數的平衡定律式中各物質皆為離子時的反應商。
\begin{itemize}
\item $Q<K\iff$溶液未飽和。
\item $Q=K\iff$溶液恰飽和。
\item $Q>K\iff$溶液過飽和。
\end{itemize}
\subsubsection{解離度(Degree of dissociation)}
\[\alpha\coloneq\frac{\text{平衡時分解的莫耳數}}{\text{全部未分解時的莫耳數}}\]
常以解離百分率(Percent dissociation)表示。
\subsubsection{定溫定容氣相反應與無同離子效應稀溶液反應解離常數與解離度關係}
令定溫定容理想氣體氣相反應或無同離子效應溶質體積不計溶液反應:
\[A \ce{<=>} \sum_{i=1}^nb_iB_i\]
有平衡常數$K$,所有濃度使用相同單位,解離度$\alpha$,全部均未解離時$A$的濃度 $Z$,$b=\sum_{i=1}^nb_i$,則:
\[K=Z^{b-1}\frac{\alpha^b}{1-\alpha}\prod_{i=1}^nb_i^{b_i}\]
\subsubsection{無同離子效應稀溶液溶解度積常數與溶解度關係}
令無同離子效應溶質體積不計溶解反應:
\[A\ce{<=>}\sum_{i=1}^nb_iB_i\]
有溶解度積常數$K_{sp}$,無同離子效應下體積莫耳溶解度$S$,$b=\sum_{i=1}^nb_i$,則:
\[K_{sp}=S^b\prod_{i=1}^nb_i^{b_i}\]
\subsubsection{無旁觀物質的定溫令定壓氣相反應與液相反應解離常數與解離度關係}
無旁觀物質的定溫定壓理想氣體氣相反應或溶劑莫耳數不計液相反應:
\[A \ce{<=>} \sum_{i=1}^nb_iB_i\]
有平衡常數$K$,所有濃度使用相同單位,解離度$\alpha$,全部均未解離時$A$的濃度 $Z$,$b=\sum_{i=1}^nb_i$,則:
\[K=\qty(\frac{Z}{1+\alpha(b-1)})^{b-1}\frac{\alpha^b}{1-\alpha}\prod_{i=1}^nb_i^{b_i}\]
\subsection{常見反應的平衡常數}
\begin{longtable}{|p{0.6\textwidth}|p{0.1\textwidth}|p{0.1\textwidth}|}
\hline
反應 & 攝氏度 & 平衡常數\\\hline
\ce{H2(g) + I2(g) <=> 2HI(g)} & 448 & $K_p=50$\\\hline
\ce{N2O4(g) <=> 2NO2(g)} & 100 & $K_p=0.066$ atm\\\cline{1-3}
& 25 & $K_p=0.113$ atm\\\cline{1-3}
& 0 & $K_p=58$ atm\\\hline
\ce{N2(g) + 3H2(g) <=> 2NH3(g)} & 450 & $K_p=4.28\times 10^{-5}$ atm$^{-2}$\\\hline
\ce{H2O(l) <=> H+(aq) + OH-(aq)} & 25 & $K_c=1.0\times 10^{-14}$ M$^2$\\\hline
\ce{AgI(s) <=> Ag+(aq) + I-(aq)} & 25 & $K_c=8.3\times 10^{-17}$ M$^2$\\\hline
\end{longtable}\FB
\end{document}