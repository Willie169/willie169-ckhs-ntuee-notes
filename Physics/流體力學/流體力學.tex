\documentclass[a4paper,12pt]{article}
\setcounter{secnumdepth}{5}
\setcounter{tocdepth}{3}
\newcounter{ZhRenew}
\setcounter{ZhRenew}{1}
\newcounter{SectionLanguage}
\setcounter{SectionLanguage}{1}
\input{/usr/share/latex-toolkit/template.tex}
\begin{document}
\title{流體力學}
\author{沈威宇}
\date{\temtoday}
\titletocdoc
\sct{流體力學(Fluid Mechanics)}
\ssc{符號約定}
\begin{itemize}
\item $t$:時間
\item $\mathbf{v}$:流速場
\item $\rho$:密度
\item $V$:控制體積
\item $M$:流體分子分子量
\item $n$:流體分子分子數
\item $p$:壓力
\item $
\item $R$:理想氣體常數
\item $T$:絕對溫度
\item $u$:每個流體分子的平均內能
\item $\frac{\mathrm{D}}{\mathrm{D}t}$:物質導數(Material derivative)/隨質導數(Substantial derivative)
\item $\mu$:動力黏度(Dynamic viscosity)
\item $\nu$:運動黏度(Kinetic viscosity)
\item $C$:阻力係數(Drag coefficient)
\item $L$:特徵長度(Characteristic length)
\item $v$:特徵長度(Characteristic speed)
\end{itemize}
\ssc{流體運動(Fluid motion)}
\sssc{物質導數(Material derivative)/隨質導數(Substantial derivative)}
對於任意場$\mb{y}$,其物質導數$\frac{\mathrm{D}\mb{y}}{\mathrm{D}t}$定義為:
\[\frac{\mathrm{D}\mb{y}}{\mathrm{D}t}=\pdv{\mb{y}}{t}+\mathbf{v}\cdot(\nabla \mb{y}).\]
\sssc{連續性方程(Continuity equation)/質量守恆}
\[\frac{\partial\rho}{\partial t}+\nabla\cdot(\rho\mathbf{v})=0\]
\sssc{納維-斯托克斯方程式(Navier–Stokes equations)/動量守恆}
\[\rho\frac{\partial\mathbf{v}}{\partial t}+\rho(\mathbf{v}\cdot\nabla)\mathbf{v}=-\nabla p\]
\sssc{歐拉方程式(Euler equations)}
對於零黏度、零導熱率的流體:
\[\begin{aligned}
\frac{\mathrm{D}\rho}{\mathrm{D}t}&=-\rho\nabla\cdot\mathbf {v}\\
\frac{\mathrm{D}\mathbf{v}}{\mathrm{D}t}&=-\frac{\nabla p}{\rho}+\frac{\partial\mathbf{v}}{\partial t}\\
\frac{\mathrm{D}u}{\mathrm{D}t}&=-\frac{p}{\rho}\nabla\cdot\mathbf{v}
\end{aligned}\]
\sssc{不可壓縮約束}
對於不可壓縮的流體:
\[\frac{\mathrm{D}\rho}{\mathrm{D}t}=0\]
\[\frac{\mathrm{D}u}{\mathrm{D}t}=0\]
\[\nabla\cdot\mathbf{v}=0\]
\ssc{狀態方程式(Equation of state)}
描述流體的熱力學狀態,形式為:
\[f(p,V,T)=0\]
例如理想氣體方程式為:
\[pV-nRT=0\]
\ssc{Characteristic length (特徵長度) and characteristic speed (特徵速率)}
Typical length and speed of the system, chosen for relevance and simplicity, not uniqueness.
\ssc{Viscosity (黏度/黏滯性) and drag (阻力)}
\sssc{Drag, fluid resistance, or viscous force (阻力)}
A force acting opposite to the direction of motion of any object moving with respect to a surrounding fluid.
\sssc{Viscosity of Newtonian fluid (牛頓流體)}
Let the viscous stress tensor ($3\times 3$) be $p$ (Pa)$. Then any fluid such that there exists a constant $3\times 3$ tensor or scalar $\mu$ (Pa s), called the dynamic viscosity (動力黏度), such that
\[p=\mu\nabla v.\]
is called Newtonian.

And the kinetic viscosity (運動黏度) $\nu$ is defined as
\[\nu=\frac{\mu}{\rho}.\]
\sssc{Reynolds number, Re (雷諾數)}
\[\operatorname{Re}=\frac{\rho v L}{\mu}=\frac{v L}{\nu}\]
\sssc{Drag equation (阻力方程式)}
The drag force $\mb{F}$ of fluid with density $\rho$ and high Reynolds number ($\operatorname{Re}\gg 1$) on an object with effective cross-sectional area (截面積) $A$ and velocity $\mb{v}$ relative to the fluid is given by
\[\mb{F}=-\frac{1}{2}C\rho A\abs{\mb{v}}^2\hat{\mb{v}},\]
where $C$ is the drag coefficient (阻力係數), a dimensionless number, which is not actually a constant but can be treated as a constant in some scenarios.
\sssc{Laminar flow (層流)}
Laminar flow is the property of fluid particles to follow smooth paths in layers, with each layer moving smoothly past the adjacent layers without mixing.
\sssc{Stokes' law (斯托克定律)}
In a laminar flow of fluid of homogeneous (uniform in composition) material with zero Reynolds number and dynamic viscosity $\mu$, and that particles do not interfere with each other, a sphere particle with radius $a$ and smooth surfaces moving in velocity $\mb{v}$ in the fluid is subjected to a drag force $\mb{F}$, known as Stokes' drag, given by
\[\mb{F}=-6\pi\mu a\mb{v}.\]
\end{document}