\documentclass[a4paper,12pt]{report}
\setcounter{secnumdepth}{5}
\setcounter{tocdepth}{3}
\newcounter{ZhRenew}
\setcounter{ZhRenew}{1}
\newcounter{SectionLanguage}
\setcounter{SectionLanguage}{1}
\input{/usr/share/latex-toolkit/template.tex}
\begin{document}
\title{演化生物學與生物分類學}
\author{沈威宇}
\date{\temtoday}
\titletocdoc
\chapter{演化生物學(Evolutionary biology)與生物分類學(Taxonomy)}
\section{亞里斯多德(Aristotle)的大自然的階梯(Scala Naturae)}
亞里斯多德是古希臘哲學家和博物學家。相信神創論(Creationism),認為生物由神所創造,完美且永恆不變。相信地球上有了水和土後就有形成生命的能力。他提出大自然的階梯,即存在鎖鏈(Great chain of being)。這是一個從依次排列、固定不變的階梯,由下到上分為礦物(Minerals)植物(Plants)、動物(Animals)、人類(Humanity)、天使(Angelic beings)、上帝(God)。
\section{林奈的林奈分類系統(Linnaean taxonomy)與
二名法(Binominal nomenclature)}
\begin{itemize}
\item 卡爾·林奈(Carl Linnaeus),瑞典人,是現代生物分類學的奠基人,被稱為「分類學之父」。
\item 相信神創論,認為物種不會改變。
\item 二名法又稱雙名法,用拉丁文的屬名和種小名(specific epithet)來命名,其中屬名為名詞,首字大寫,種小名為形容詞均小寫,通常兩字均斜體或加底線,如智人為\ti{Homo sapiens}。
\item 林奈分類系統依照物的型態相似性分類,分為動物界、植物界和礦物界,並進一步細分為界、門、綱、目、科、屬、種七個等級,以彰顯神創萬物的完美秩序。動物界、植物界屬於生物,故為二界系統。
\end{itemize}
\section{布豐的演化論啟蒙}
喬治-路易·勒克萊爾,布豐伯爵(Georges-Louis Leclerc, Comte de Buffon):18世紀法國博物學家,提出物種並不是不變的,認為生物會因為環境的影響而發生變化。
\section{拉馬克的用盡廢退說/拉馬克主義}
讓-巴蒂斯特·拉馬克(Jean-Baptiste Lamarck):法國博物學家,1809年提出第一個系統的演化理論。
\subsection{理論要義}
\begin{itemize}
\item 用盡廢退:認為生物器官的使用會使其發達,不使用則會退化。例如,長頸鹿因為要吃高處的樹葉,長期伸長脖子,導致脖子變長。
\item 獲得性遺傳:為何適應環境所產生的後天變化會遺傳給後代。例如,長脖子的長頸鹿將這一特徵遺傳給下一代。
\end{itemize}
\subsection{參酌證據}
比對現生物種的化石標本,發現生物型態構造隨生存需求改變。
\subsection{主張}
\begin{itemize}
\item 物種起源:各自起源,而非存在共祖。
\item 環境影響:促進變異,而非僅提供選汰壓力。
\item 變異來源:來自生物內在的驅動力,即型態構造根據需要有方向性的發生改變,而非變異本來就存在。
\end{itemize}
\subsection{魏斯曼老鼠剪尾實驗}
奧古斯特·魏斯曼(August Weismann),18世紀德國生物學家,他剪掉老鼠的尾巴並觀察其後代,發現即使多代老鼠都被剪尾,其後代仍然長有正常的尾巴,這證明了獲得性狀並不會遺傳,反駁了拉馬克的用進廢退說。
\section{達爾文與華萊士的天擇說}
\subsection{自然選擇/天擇(Natural selection)}
\begin{itemize}
\item 變異(Variation):生物個體間的表徵具有差異。
\item 過度繁殖(Overproduction):生物體往往產生大量的後代,超過環境所能支持的數量。這導致競爭資源,如食物、水和棲息地。
\item 競爭(Competition):由於資源有限,個體之間必然發生競爭。具有某些特定表徵的個體可能在某些環境下較其他個體容易在生存競爭中獲勝,即是適應環境者。
\item 適者生存(Survival of the fittest):適應環境者更有可能生存下來,並繁殖後代。
\item 遺傳(Heredity):獲勝個體的有利性狀和表徵會通過遺傳傳給後代,從而在種群中逐漸積累,導致物種的進化。
\item 適應(Adaptation):隨著時間的推移,物種會根據環境變化進化出更適應環境的性狀和表徵。這種適應可導致物種的形成和變異。
\end{itemize}
\subsection{人擇(Artificial selection)}
指人類針對特定性狀進行育種,使這些性狀的表現逐漸強化,而人們不需要的性狀則可能逐漸消匿的過程。
\subsection{性擇(Sexual selection)}
\subsubsection{性別二型性(Sexual dimorphism)/兩性異形}
指生物的雌、雄個體有與生殖沒有直接關係的特徵(第二性徵)的差異,如體型、顏色、用作求偶或打鬥的身體器官。
\subsubsection{性擇}
用於解釋同一性別的個體(通常是雄性)對交配機會的競爭如何促進性狀的演化。同一物種的兩個性別之間,通常有至少一個性別必須競爭取得有限的交配機會。由於個體間存在可遺傳的差異,造成有的個體在競爭中較為成功,此較成功的個體將此差異給後代,便造成性擇演化。通常雌性在生殖過程中投資較多,因此對交配對象較挑剔,所以性擇是作用在雄性的性狀上,但在性別角色相反的海馬等海龍科魚類上,則是作用在雌性。
\subsection{共同祖先與演化}
\begin{itemize}
\item 共同祖先:所有生物具有共同祖先,後經天擇的過程演化成後代物種。
\item 親緣關係樹:將物種的親緣關係繪製為一棵樹,其基部就是共同祖先,末端即是衍生出來的後代物種。將親緣關係樹在某一分支處翻轉並不影響親緣關係。達爾文手稿中,T 型分支為現存物種,I 型分支為滅絕物種。
\item 輻射演化(Adaptive radiation):指一群生物從共同祖先開始,在相對較短的地質時間內迅速分化成多個新物種的過程。這些新物種適應了不同的生態位和環境條件。通常發生在以下情況:
\begin{itemize}
\item 新環境的開放:例如,當一個物種進入一個新的生態區域,這裡的生態位尚未被占據。
\item 進化創新:當一個物種演化出新的特徵或能力,使其能夠利用之前未能利用的資源或環境。
\item 競爭壓力的減少:例如,由於大規模滅絕事件的發生,使得競爭對手減少,留下許多空閒的生態位。
\end{itemize}
\end{itemize}
\subsection{由來}
\begin{itemize}
\item 查爾斯·達爾文(Charles Darwin)與阿爾弗雷德·華萊士(Alfred Russel Wallace):演化論的共同創立者。他們分別獨立提出了自然選擇的概念。
\item 達爾文觀察加拉巴哥群島達爾文雀(Darwin's finches):達爾文在1831-1836年乘坐英國小獵犬號進行環球航行,途經南美洲再到加拉巴哥群島,對生物進行了詳細觀察。他注意到群島上不同島嶼的雀鳥(後來稱為達爾文雀)在喙的形態上存在顯著差異,這些差異與它們的食物來源有關,例如食蟲者較食種子者喙長。達爾文原本以為該等雀鳥屬不同科,後經鳥類學家鑑定為同科,達爾文推論這些雀鳥是由可能來自南美洲大陸的共同祖先經演化而來。
\item 達爾文觀察家鴿提出人擇:達爾文飼養並觀察了家鴿,注意到家鴿個體間有不同表徵,人類可以通過選擇性繁殖培育出具有特定表徵的鴿子,提出人擇。
\item 達爾文觀察孔雀尾羽提出性擇:達爾文觀察了孔雀尾羽,注意到尾羽不利於適應環境,故提出性擇。
\item 達爾文受萊爾地質學原理的啟發:達爾文受查爾斯·萊爾(Charles Lyell)的《地質學原理》一書啟發,該書主張地球的地質變化是通過長時間內的漸進過程實現的。
\item 達爾文受拉馬克演化理論啟發:拉馬克提出有系統的演化理論,啟發達爾文的天擇說。
\item 達爾文受馬爾薩斯人口論的啟發:托馬斯·馬爾薩斯(Thomas Malthus)《人口論》提出人口增長速度快於食物供應,子代數量可能超過環境負載力,導致生存競爭。達爾文受之啟發。
\item 華萊士觀察南美洲與東南亞甲蟲:華萊士在南美洲與東南亞觀察甲蟲,注意到兩地區距離愈接近,物種組成愈相似,發展出對物種起源的看法。
\item 華萊士線(Wallace Line):阿爾弗雷德·華萊士在東南亞地區進行研究時,觀察到生物分布的顯著差異,並提出了華萊士線(Wallace Line),該線將東南亞分為兩個區域,分別代表亞洲和澳大利亞的生物區系。
\item 林奈學會發表:1858年,華萊士去信達爾文,發現兩人想法一致。同年,兩人在林奈學會共同發表了自然選擇理論的論文。
\item 物種起源:1859年,達爾文發表了《物種起源》(On the Origin of Species)一書,其中詳細闡述了自然選擇理論(天擇說),認為生物通過自然選擇適應環境,優勢特徵逐漸累積,導致物種演化。
\end{itemize}
\section{分子演化的中性理論(Neutral theory of molecular evolution)/中性演化理論}
日本遺傳學家木村資生在1968年提出。\\
認為在分子遺傳學層次上,基因變化大多數是中性突變,對生物個體的生殖與生存既沒有好處也沒有壞處,不受自然選擇影響,而存在遺傳漂變。中性突變指突然變異產生的遺傳物質和原本的遺傳物質之間沒有適應性差異。
\section{現代綜合進化論(Modern Synthmesis)}
20世紀中期發展,將達爾文的自然選擇演化理論與孟德爾遺傳學相結合,形成了現代演化生物學的基礎。\\
現今的演化生物學家認為,自然選擇理論與中性理論是能夠並立且互補的。\\
現代綜合進化論的主要內容:
\begin{itemize}
\item 基因突變:基因突變是遺傳變異的根本來源。
\item 遺傳重組:通過有性生殖等,基因重組產生新的遺傳組合。
\item 自然選擇:自然選擇是適應性變化的主要驅動力。
\item 遺傳漂變(Genetic drift):小群體中,基因頻率可能隨機變動,導致遺傳漂變。
\item 物種形成:新物種形成涉及隔離機制,導致種群分化。
\end{itemize}
\section{演化的證據}
生物演化的證據可以用來驗證演化學者提出生物具有共同祖先的理論,並釐清生物間的親緣關係。
\subsection{化石證據}
化石指古代生物遺體或活動留下來的痕跡,其中前者指生物化石,後者指生痕化石。提供了關於古代生命形式的直接證據,顯示了生物隨時間變化的過程。例如,始祖鳥(Archaeopteryx)顯示了從恐龍到現代鳥類的過渡特徵;魚石螈(Tiktaalik)顯示了從魚到四足動物的過渡。
\subsection{比較解剖學證據}
比較不同物種的解剖結構可以揭示它們之間的演化關係。
\subsubsection{同源構造(Homologous Structures)}
指不同物種的結構,儘管在形態和功能上可能不同,但在演化上具有共同的祖先。這些結構在胚胎發育中起源於相同的組織。例如:哺乳動物的前肢(如人的手臂、蝙蝠的翅膀、鯨魚的鰭)都是同源構造,因為它們在骨骼結構上有相似之處,顯示它們具有共同的祖先。
\subsubsection{同功構造(Analogous Structures)}
指不同物種的結構,儘管在功能上相似,但在演化上沒有共同的祖先。這些結構通常是由於相似的環境壓力和自然選擇,導致不同物種獨立地演化出相似的功能。例如:蝙蝠的翅膀和昆蟲的翅膀是同功構造,因為它們都用於飛行,但在結構和演化來源上完全不同。
\subsubsection{痕跡構造(Vestigial Structures)}
指在現代生物體中已經退化或失去原來功能的結構,這些結構在其祖先中是具有功能的。例如:人類的闌尾,因為它在現代人類中並沒有明顯的功能,但在草食性祖先中可能有助於消化纖維;鯨魚的骨盆和腿骨遺跡顯示出鯨魚的陸地祖先。
\subsection{胚胎學證據}
胚胎學研究生物從受精卵到成體(有性能力)的發育過程,可以發現它們之間的演化聯繫。例如:不同脊椎動物的胚胎在早期階段顯示出相似的特徵,例如魚類、鳥類和哺乳動物的胚胎都有咽囊和尾,其中魚類的咽囊發育為鰓,人類則發育為咽喉。
\subsection{分子生物學證據}
通過比較不同物種的DNA和蛋白質胺基酸序列,可以揭示它們之間的親緣關係。例如:人類與黑猩猩的DNA序列相似度超過98\%,表明它們有共同的祖先;粒線體DNA的變異用於追溯人類的母系血統;人類與恆河猴的血紅素胺基酸序列差異僅8個,與八目鰻相差125個,故人與恆河猴親緣關係較近。
\subsection{生物地理學證據}
生物地理學主要研究物種在地理分布上的範圍、模式與原因,可以提供演化證據。\\
種緣中心:某些類群生物的祖先最早出現的地區。例如:達爾文雀的種源中心可能在南美洲,並顯示了在不同加拉巴哥群島上不同物種的適應輻射;澳大利亞的有袋類哺乳動物與世界其他地方的胎盤(由母體子宮內膜與胎兒部分組織構成,內有微血管,使胎兒可與母親交換氣體與獲得養分)類哺乳動物分化,科學家推測有袋類可能起源於中生代盤古大陸今亞洲與北美洲處,後其他大洲的有袋類受後來出現的胎盤類競爭而絕大部分滅絕,僅澳洲與其他大洲隔離未受到胎盤類競爭。
\section{林奈之後的分類系統}
\subsection{恩斯特·海克爾(Ernst Haeckel)的三界說}
19世紀後,科學家利用光學顯微鏡發現許多微小的單細胞生物。海克爾是德國生物學家和哲學家,他在1866年提出了三界說,將生物界分為多細胞異營生物的動物界(Animalia)、多細胞自營生物的植物界(Plantae)和單細胞真核生物的原生生物界(Protista)。他首次提出了原生生物界這一概念。
\subsection{埃德·查頓(Édouard Chatton)的四界說先驅}
20世紀,電影顯微鏡的發明與細胞生物學的進展使科學家發現沒有細胞核的原核生物。查頓是法國生物學家,他在1925年提出將生物分為原核生物和真核生物的概念。
\subsection{赫伯特·科普蘭(Herbert F. Copeland)的四界說}
科普蘭是美國生物學家,他在1956年提出了四界說,將生物分為原核生物界(Monera)、原生生物界(Protista)、植物界(Plantae)和動物界(Animalia)。
\subsection{羅伯特·懷塔克(Robert Whittaker)的五界說}
懷塔克是美國生物學家,發現真菌營養方式和細胞壁成分明顯與植物不同,而在1969年提出五界說,將生物分為真菌界(Fungi)、植物界(Plantae)、動物界(Animalia)、原核生物界(Monera)和原生生物界(Protista)。
\subsection{卡爾·渥易斯(Carl Woese)的六界說}
渥易斯是美國科學家,他在1977年提出六界說。基於分子生物學特別是16S rRNA序列的差異,將原核生物分成真細菌界(Bacteria)和古菌界(Archaea),加上真核生物原有的四界共六界。
\subsection{湯瑪斯·卡弗利爾-史密斯(Thomas Cavalier-Smith)的六界說}
史密斯是英國生物學家,他在1981年提出了六界說,將生物分為細菌界(Bacteria)、古菌界(Archaea)、原生生物界(Protozoa)、真菌界(Fungi)、植物界(Plantae)、動物界(Animalia)。其中原生生物界從懷塔克五界系統的僅有單細胞生物變成包含單細胞和簡單多細胞真核生物。
\subsection{卡爾·渥易斯(Carl Woese)的三域系統}
1990年,渥易斯比對動物、植物與真菌的RNA序列後,發現此三類生物均源自特定類群的原生生物,因此合併上述四類為真核生物。經由rRNA序列比對,他發現古菌與真核生物的親緣關係較與真細菌近,因此將六界生物分別歸類為細菌域(Bacteria)、古菌域(Archaea)和真核生物域(Eukarya)。真細菌和古菌隨同屬原核生物,但細胞成分不同,真細菌如藍綠菌、大腸桿菌等,細胞壁主要成分為肽聚醣,而古菌則否,甚至沒有細胞壁。許多古菌如甲烷菌、極端嗜鹽菌與極端嗜熱菌等生活在極端環境中。
\section{分支系統學(Cladistics)}
早期分類學者多以主觀認定的型態作為分類依據,例如林奈。但其所根據的特徵可能不是共祖或衍徵,而是同功構造,因此無法表現生物之間實際的演化關係。現代分支系統學受到演化論的影響,基於進化關係進行分類,強調生物的共同祖先和演化支系,由威利·亨尼希(Willi Hennig)在20世紀中期提出,並且在現代生物分類學中得到了廣泛應用。具有共同祖先的物種歸為同一類是親緣關係重建的原則,許多學者並主張生物分類均應為單系群。
\subsection{單系群、並系群與多系群}
\subsubsection{單系群(Monophyletic Group)}
單系群是指由一個共同祖先及其所有後代組成的一個演化支系。這一分類單元反映了真實的進化關係。例如,哺乳綱。
\subsubsection{並系群(Paraphyletic Group)}
並系群是指由一個共同祖先及其部分後代組成的一個演化支系。這一分類單元不包括該祖先的所有後代,因此不完全反映真實的進化關係。例如,爬行動物如果不包括鳥類的話,就構成一個並系群。
\subsubsection{多系群(Polyphyletic Group)}
多系群是指由不具有共同最近祖先的生物組成的一個分類單元。這一分類單元通常是基於相似的形態或功能特徵,而不是共同的祖先。例如,如果將蝙蝠和鳥類歸為一類,這就是一個多系群,因為它們的飛行能力是獨立進化的。
\subsection{祖徵與衍徵}
祖徵和衍徵用於描述生物的特徵或特徵集合在演化過程中的變化和分佈,有助於重建親緣關係。
\subsubsection{祖徵(Plesiomorphy)}
祖徵指的是一個特徵或特徵集合,這些特徵存在於物種的祖先中,並且在這些物種的演化分支中普遍存在。換句話說,祖徵是指由一個共同祖先繼承的原始特徵,它並沒有在後代中發生重要的改變或演化。例如,脊椎動物的脊柱是一個祖徵,因為它存在於所有脊椎動物的共同祖先中。
\subsubsection{衍徵(Apomorphy)}
衍徵指的是在演化過程中出現的新特徵或特徵集合,這些特徵是由共同祖先傳遞給其後代的衍生性狀。換句話說,衍徵是進化的新獲得特徵,它反映了物種或群體在演化過程中的特化或變化。例如,鳥羽類的羽毛是一個衍徵,因為它是鳥羽類特有的特徵,與其共同祖先和其他動物的祖徵不同。
\subsection{四足總綱(Tetrapoda)親緣關係的重建}
\subsubsection{傳統分類學名稱定義}
\begin{itemize}
\item 爬行動物/爬蟲類/爬蟲綱(Reptilia):指不包括鳥類和哺乳類在內的所有羊膜動物,因此屬於一個並系群。
\item 羊膜動物 Amniota:指具有羊膜的四足總綱動物,因此屬於一個單系群。羊膜,指因為在陸地生活,胚胎非如魚類、兩生類直接接觸外界有水的環境,故演化出以羊膜包裹胚胎使之在羊水中發育。
\item 哺乳綱 Mammalia:所有具有毛髮和乳腺作為共衍徵的物種的最近共同祖先,與其所有後代。
\item 兩生類/兩棲類/兩生動物/兩棲動物/兩生綱/兩棲綱 Amphibia:指不包括羊膜動物在內的所有四足總綱生物,因此屬於一個並系群。
\item 龜鱉目 Testudines:所有現存龜鱉的最近共同祖先,與其所有後代。
\item 蜥蜴亞目 Lacertilia:有鱗目 Squamata 除了蛇亞目 Serpentes 的物種。
\item 蛇亞目 Serpentes:所有現存蛇類的最近共同祖先,與其所有後代嗎。
\item 非鳥恐龍:現今定義之恐龍總目除了新鳥亞綱。
\end{itemize}
\subsubsection{現代分類學名稱定義}
\begin{itemize}
\item 羊膜動物 Amniota:與傳統分類學相同。
\item 兩生綱/兩棲綱 Amphibia:指四足總綱蛙形類的一支,因此屬於一個單系群。
\item 滑體亞綱 Lissamphibia:所有現存兩生類的最近共同祖先,與其所有後代。
\item 哺乳綱 Mammalia:與傳統分類學相同。
\item 鳥羽類 Avifilopluma:有羽毛的物種。是鳥綱 Aves 的一種定義。
\item 鳥跖(蹠)類 Avemetatarsalia:主龍類中,所有親緣關係與鳥類較近、與鱷魚較遠的物種。是鳥綱的一種定義。
\item 鳥翼類 Avialae:恐龍總目中,擁有布滿羽毛的翅膀作為共衍徵,並可將翅膀用於拍打飛行的所有物種的最近共同祖先,與其所有後代。是鳥綱的一種定義。
\item 新鳥亞綱 Neornithes:所有現存鳥類的最近共同祖先,與其所有後代。是鳥綱的一種定義。
\item 鱷形超目 Crocodylomorpha:所有現存鱷魚的最近共同祖先,與其所有後代。
\item 龜鱉目 Testudines:與傳統分類學相同。
\item 泛龜類 Pantestudines:主龍形亞綱中,與海龜的關係比與鱷魚和鳥類的關係更近的物種。
\item 蛇亞目 Serpentes:與傳統分類學相同。
\item 恐龍總目 Dinosauria:鳥跖類的一支。
\end{itemize}
\subsubsection{傳統分類到現代分類的變化}
\begin{itemize}
\item 將羊膜動物與兩生綱分開古今並無不同,儘管兩者的兩生綱定義略有不同。
\item 羊膜動物中,哺乳類與鳥類都是內溫動物(恆溫動物、溫血動物),會以身體代謝產生熱能來調節體溫,爬蟲類則為外溫動物(變溫動物、冷血動物)。因此傳統分類學者以為鳥類與哺乳類的親緣關係較爬蟲類近。
\item 傳統分類學者以為爬蟲類不具羽毛,實則部分恐龍也具有羽毛。
\item 19世紀德國發現始祖鳥化石,同時具有鳥類與爬蟲類特徵,各地亦接續發現有羽恐龍化石,例如中華龍鳥,顯示鳥類與恐龍的親緣關係較哺乳類近。
\item 分子生物學顯示鳥類與爬蟲類的親緣關係較與哺乳類近。
\end{itemize}
\subsubsection{四足總綱現代分支系統學分類}
依照親緣關係重建的四足總綱現代分支系統學分類應如以下樹狀圖:
\nthm
\scalebox{0.4}{
\begin{minipage}{\textwidth}
\begin{center}
\vspace*{\fill}
\begin{figure}[H]
\centering
\begin{tikzpicture}[
  level 1/.style = {sibling distance=10em},
  level 2/.style = {sibling distance=10em},
  level 3/.style = {sibling distance=40em},
  level 4/.style = {sibling distance=20em},
  level 5/.style = {sibling distance=10em},
  level 6/.style = {sibling distance=10em},
  level 7/.style = {sibling distance=10em},
  level 8/.style = {sibling distance=10em},
  level distance=10em,
  every node/.style = {
    shape=rectangle, draw, align=center,
    fill=white,
    text width=8em,
    minimum height=1em,
    font=\scriptsize
  }]
  \node {動物界 Animalia \\ | \\ 脊索動物門 Chordata \\ | \\ 脊椎動物亞門 Vertebrata \\ | \\ 有頜(hé)下門 Gnathostomata \\ | \\ 硬骨魚高綱 Osteichthyes \\ | \\ 肉鰭魚總綱 Sarcopterygii \\ | \\ 四足總綱 Tetrapoda}
    child { node {蛙形類 Batrachomorpha \\ | \\ 兩生綱 Amphibia \\ | \\ 滑體亞綱 Lissamphibia}}
    child { node {爬行形類 Reptiliomorpha \\ | \\ 羊膜動物 Amniota}
      child { node {合弓綱 Synapsida \\ | \\ 哺乳綱 Mammalia} }
      child { node {蜥形綱 Sauropsida \\ | \\ 真爬行動物 Eureptilia \\ | \\ 盧默龍類 Romeriida \\ | \\ 雙孔亞綱 Diapsida \\ | \\ 新雙弓類 Neodiapsida \\ | \\ 蜥類 Sauria}
        child { node {鱗龍形下綱 Lepidosauromorpha \\ | \\ 鱗龍總目 Lepidosauria}
          child { node {有鱗目 Squamata}
            child { node {雙足蜥亞目 Dibamia} }
            child { node {壁虎亞目 Gekkota} }
            child { node {正蜥亞目 Laterata} }
            child { node {石龍子亞目 Scincomorpha} }
            child { node {有毒類 Toxicofera}
              child { node {蛇蜥亞目 Anguimorpha} }
              child { node {鬣蜥亞目 Iguania} }
              child { node {蠎形類 Pythonomorpha \\ | \\ 蛇形類 Ophidiomorpha \\ | \\ 泛蛇類 Ophidia \\ | \\ 蛇亞目 Serpentes} }
            }
          }
          child { node {喙頭蜥目 Sphenodontia} }
        }
        child { node {主龍形亞綱 Archelosauria}
          child { node {泛龜類 Pantestudines}
            child { node {龜鱉目 Testudines}
            }
          }
          child { node {主龍形下綱 Archosauromorpha \\ | \\ 主龍形類 Archosauriformes \\ | \\ 真鱷腳類 Eucrocopoda \\ | \\ 主龍類 Archosauria}
            child { node {偽鱷類 Pseudosuchia \\ | \\ 鱷形超目 Crocodylomorpha} }
            child { node {鳥跖(蹠)類 Avemetatarsalia}
              child { node {翼龍目 Pterosauria} }
              child { node {恐龍總目 Dinosauria}
                child { node {鳥臀目 Ornithischia} }
                child { node {蜥臀目 Saurischia}
                  child { node {蜥腳亞目 Sauropodomorpha} }
                  child { node {獸腳亞目 Theropoda \\ | \\ 新獸腳類 Neotheropoda \\ | \\ 鳥翼類 Avialae}
                    child { node {真鳥翼類 Euavialae \\ | \\ 短尾鳥類 Avebrevicauda \\ | \\ 尾綜骨鳥類 Pygostylia \\ | \\ 鳥胸骨類 Ornithothoraces \\ | \\ 真鳥類 Euornithes \\ | \\ 今鳥型類 Ornithuromorpha \\ | \\ 扇尾類 Ornithurae}
                      child { node {新鳥亞綱 Neornithes} }
                    }
                  }
                }
              }
            }
          }
        }
      }
    };
\end{tikzpicture}
\caption{四足總綱現代分支系統學分類樹狀圖}
\end{figure}\FloatBarrier
\vspace*{\fill}
\end{center}
\end{minipage}
}
\nthm
\section{生物與環境交互作用的歷程}
\begin{enumerate}
\item 最古老的化石是約35億年前的單細胞原核生物。
\item 太古代時,地球氧氣濃度非常低,厭氧菌興盛,生物多為異營。
\item 隨著行光合作用的藍綠菌出現,大氣氧濃度在約24億年前開始增加,使厭氧菌生存範圍減少。
\item 古生代石炭紀陸域植物繁盛,有許多體型巨大的昆蟲,如翼展75公分的巨脈蜻蜓。科學家提出假說認為當時陸域植物繁盛使大氣氧濃度達約35\%,較利於巨大昆蟲生存,因為昆蟲體型大小受到氣體交換效率(呼吸效率)的限制,體積愈大,可進行氣體交換的表面積相對於其體積就愈小。
\item 中生代出現恐龍、哺乳動物。
\item 中生代末恐龍大滅絕,僅留下鳥類。
\item 新生代是開花植物與哺乳類優勢的時代。
\item 約20萬年前智人出現,改變環境。
\end{enumerate}
\end{document}