\documentclass[a4paper,12pt]{article}
\setcounter{secnumdepth}{5}
\setcounter{tocdepth}{3}
\newcounter{ZhRenew}
\setcounter{ZhRenew}{1}
\newcounter{SectionLanguage}
\setcounter{SectionLanguage}{1}
\input{/usr/share/latex-toolkit/template.tex}
\begin{document}
\title{物理學簡史}
\author{沈威宇}
\date{\temtoday}
\titletocdoc
\section{物理學簡史}
\begin{itemize}
\item B.C.420:德謨克利特(Democritus)提出原子(atom)。
\item B.C.320:亞里斯多德(Aristotle)提出物體運動的解釋。
\item B.C.250:阿基米得(Archimede)發現液體浮力的阿基米德定律(Archimedes' principle)。
\item 150:托勒密(Ptolemy)提出地心模型(Geocentric model)。
\item 1265:培根(Francis Bacon)發現球面像差(Spherical aberration)。
\item 1543:哥白尼(Nicolaus Copernicus)著「天體運行論(On the Revolutions of the Heavenly Spheres)」,發表日心說(Heliocentrism)。
\item 1609:伽利略(Galileo Galilei)使用望遠鏡觀測行星。
\item 1609:克卜勒(Johannes Kepler Kepler)著「新天文學(Astronomia nova)」,發表克卜勒第一及第二行星運動定律(Kepler's First and Second Laws of Planetary Motion)。
\item 1615:司乃耳(Willebrord Snellius)發表折射的司乃耳定律(Snell's law)。
\item 1619:克卜勒著「世界的和諧(Harmonices Mundi)」,發表克卜勒第三行星運動定律(Kepler's Third Law of Planetary Motion)。
\item 1632:伽利略著「關於兩大世界體系的對話(Dialogue Concerning the Two Chief World Systems)」。
\item 1634:伽利略提出慣性定律(Law of Inertia)。
\item 1653:帕斯卡(Blaise Pascal)提出關於流體壓力的帕斯卡定律(Pascal's law)。
\item 1661:波以耳發表波以耳定律(Boyle's law)。
\item 1666:牛頓(Isaac Newton)進行三稜鏡色散實驗。
\item 1676:隆美耳(Ole Rømer)從木星蝕現象推算光速。
\item 1678:惠更斯(Christiaan Huygens)提出光的波動說和以太(ether)。
\item 1678:虎克(Robert Hooke)發表關於彈性的虎克定律(Hooke's law)。
\item 1687:牛頓出版「自然哲學的數學原理(Philosophiae Naturalis Principia Mathematica)」,發表牛頓運動定律(Newton's Laws of Motion)與牛頓萬有引力定律(Newton's Law of Universal Gravitation)。
\item 1704:牛頓出版「光學(Opticks)」。
\item 1747:富蘭克林(Benjamin Franklin)用電荷守恆說明摩擦起電現象。
\item 1752:富蘭克林以風箏實驗證明閃電和摩擦起電之電性質相同。
\item 1785:庫侖(Charles-Augustin de Coulomb)提出庫侖定律(Coulomb's law)。
\item 1789:拉瓦節(Antoine-Laurent de Lavoisier)發表最早的元素列表,並提出質量守恆定律(Conservation of mass)。
\item 1798:卡文迪西(Henry Cavendish)完成卡文迪西試驗(Cavendish experiment),是首個在實驗室內完成的測量兩地球以外物體間的萬有引力實驗,也是首個求出萬有引力常數和地球質量的實驗。
\item 1799:普魯斯特(Joseph Proust)提出定比定律(Law of definite proportions)。
\item 十八世紀末:賈法尼(Luigi Galvani)以黃銅解剖刀碰觸鐵盤內的青蛙產生抽搐,稱動物電(animal electricity),以賈法尼理論(Galvanism)解釋,伏特(Alessandro Volta)反對,並在舌頭上放錫箔、下放銀幣,接觸時感到微微酸疼,提出舌頭上的導電物質使二種金屬間產生電位差。
\item 1800:伏特發明鋅圓板、吸滿食鹽水的布與銀圓板交疊的伏打電堆(Voltaic pile)。
\item 1801:楊格(Thomas Young)做光的楊格雙狹縫干涉實驗(Young's double-slit interference experiment),並用波動說解釋光的干涉。
\item 1803:道耳頓(John Dalton)提出原子論(Atomic theory)。
\item 1804:道耳頓提出倍比定律(Law of multiple proportions)。
\item 1808:蓋呂薩克(Joseph Louis Gay-Lussac)提出氣體化合體積定律(Law of gas combining volume)。
\item 1808:道爾頓出版「化學新體系(A New System of Chemical Philosophy)」。
\item 1811:亞佛加厥(Amedeo Avogadro)提出分子說與亞佛加厥定律(Avogadro's law)。
\item 1814:弗朗和斐(Joseph von Fraunhofer)造大型稜鏡行色散實驗,發現太陽光譜中存在 500 多條暗線,稱弗朗和斐線(Fraunhofer lines)。
\item 1818:菲涅耳(Augustin-Jean Fresnel)提出惠更斯-菲涅耳原理(Huygens–Fresnel principle)。
\item 1820:厄斯特(Hans Christian Ørsted)發表關於電流對磁針作用的實驗。
\item 1820:安培(André-Marie Ampère)把正電荷的流動方向定為電流的方向,並提出右手定則(Right-hand rule)。
\item 1820:必歐(Jean-Baptiste Biot)和沙伐(Félix Savart)發表必歐–沙伐定律(Biot–Savart law)。
\item 1821:弗朗和斐發明分光儀(Spectrometer)。
\item 1824:卡諾(Nicolas Léonard Sadi Carnot)發現無法將熱量完全轉換為功。
\item 1825:安培提出二長直載流導線間相互作用力之定律。
\item 1827:布朗(Robert Brown)發現布朗運動(Brownian motion)。
\item 1829:德貝萊納(Johann Wolfgang Döbereiner)提出德貝萊納三元素組(Döbereiner's triads)。
\item 1830:亨利(Joseph Henry)發現自感現象。
\item 1831:法拉第(Michael Faraday)提出法拉第電磁感應定律(Faraday's law of induction)。
\item 1833:法拉第提出法拉第電解定律(Faraday's laws of electrolysis)。
\item 1834:格雷姆(Thomas Graham)提出關於氣體擴散的格雷姆定律(Graham's law)。
\item 1840:焦耳(James Prescott Joule)提出由電磁線圈產生的熱與由電源產生的熱受同一定律支配。
\item 1842:都卜勒(Christian Doppler)提出都卜勒效應(Doppler effect)。
\item 1844:冷次(Emil Lenz)提出金屬電阻與溫度線性正相關。
\item 1847:亥姆霍茲(Hermann von Helmholtz)著「論力的守恆(Über die Erhaltung der Kraft)」。
\item 1848:克耳文(William Thomson, 1st Baron Kelvin)提出克耳文溫標(Kelvin scale of temperature)。
\item 1848:焦耳提出氣體運動論(Kinetic theory of gases)。
\item 1849:克希荷夫(Gustav Kirchhoff)發表穩態電路的克希荷夫電路定律(Kirchhoff's circuit laws)。
\item 1849:傅科(Léon Foucault)發現同種金屬的明線與暗線光譜相同。
\item 1850:克勞修斯(Rudolf Clausius)提出熱力學第二定律(Second law of thermodynamics)。
\item 1852:埃格斯特朗(Anders Jonas Ångström)提出可以通過光譜分析辨識元素。
\item 1853:埃格斯特朗提出電火花光譜分別來自金屬電極和氣體。
\item 1855:克耳文把力的守恆改稱為能量守恆。
\item 1855:本生(Robert Bunsen)發明本生燈(Bunsen burner)。
\item 1859:克希荷夫與本生發現各元素氣體發射光譜各不相同。
\item 1864:紐蘭(John Newlands)提出八度律(Law of Octaves)。
\item 1864:馬克士威(James Clerk Maxwell)發表「電磁場的動力學理論(A Dynamical Theory of the Electromagnetic Field)」,並預測電磁波的存在。
\item 1868:埃格斯特朗著「太陽光譜研究」,其中發表標準太陽光譜圖表,並在氣體放電光譜中找到部分氫原子光譜譜線並精確測量其波長。
\item 1869:門得列夫(Dmitri Mendeleev)提出依照原子量排列的元素週期表(Periodic Table),並成功預言部分元素。
\item 1869:瑞利(John William Strutt, 3rd Baron Rayleigh)提出瑞利散射(Rayleigh  scattering),成功解釋天空顏色。
\item 1870:焦耳測定熱功當量。
\item 1873:凡得瓦(Johannes Diderik van der Waals)提出關於氣體狀態的凡得瓦方程式。
\item 1873:馬克士威完成「電磁學通論(A Treatise on Electricity and Magnetism)」。
\item 1875:簽訂米制公約(Convention du Mètre, Metre Convention),建立國際度量衡局(Bureau international des poids et mesures, International Bureau of Weights and Measures, BIPM)、國際度量衡大會(Conférence générale des poids et mesures, General Conference on Weights and Measures, CGPM)與國際度量衡委員會(Comité international des poids et mesures, International Committee for Weights and Measures, CIPM)。
\item 1876:高德斯坦(Eugen Goldstein)發現氣體放電管從負極發出的射線,並命名為「陰極射線(Cathode ray)」。
\item 1877:波茲曼(Ludwig Boltzmann)提出波茲曼熵公式(Boltzmann's entropy formula)。
\item 1877至1894:瑞利進行聲學與氣體密度研究。
\item 1878:克魯克斯(William Crookes)進行陰極射線實驗。
\item 1881:斯托尼(George Johnstone Stoney)提出「電子(electron)」。
\item 1885:巴耳末(Johann Jakob Balmer)發現氫原子光譜的巴耳末系列,並提出巴耳末公式。
\item 1887:赫茲(Heinrich Hertz)以火花隙實驗證實電磁波,並發現紫外光照在負極板上更易放電。
\item 1887-1889:赫茲證實電磁波的反射、折射等波動性質。
\item 1890:芮德伯(Johannes Rydberg)提出芮德伯公式(Rydberg formula)。
\item 1892:勞侖茲(Hendrik Antoon Lorentz)著《馬克士威電磁理論及其在運動物體中的應用》(La théorie électromagnétique de Maxwell et son application aux corps mouvants)。
\item 1893:維恩(Wilhelm Wien)提出維恩位移定律(Wien's displacement law)。
\item 1894:威爾遜(Charles Thomson Rees Wilson)發明雲霧室(Cloud chamber)。
\item 1894:瑞利與拉姆齊(William Ramsay)發現氬,分別獲1904年諾貝爾物理獎與化學獎。
\item 1895:侖琴(Wilhelm Röntgen)發現 X 射線,並提出制動輻射(Bremsstrahlung)原理,獲1901年諾貝爾物理獎。
\item 1896:貝克勒(Henri Becquerel)發現鈾鹽可發出使底片感光的射線,是首次發現天然放射性,獲1903年諾貝爾物理獎。
\item 1896:維恩提出維恩分布定律(Wien's distribution law),獲1911年諾貝爾物理獎。
\item 1896:馬可尼(Guglielmo Marconi)發明火花隙發射器(Spark gap transmitter)。
\item 1897:J. J. 湯姆森(Joseph John Thomson)發表陰極射線實驗結果,命名陰極射線之粒子為電子,並測得電子荷質比,獲1906年諾貝爾物理獎。
\item 1897:瑪麗亞•居里(Maria Curie)與皮耶•居里(Pierre Curie)夫婦發現釷具有放射性。
\item 1898:居里夫婦發現釙和鐳,發現鐳之天然放射性約為鈾之一百萬倍,且有兩種不同射線,即後由拉塞福命名為 α 射線與 β 射線,共同獲1903年獲諾貝爾物理獎。
\item 1898:勞侖茲著《論離子電荷與質量引起的光學現象》(Optische verschijnselen die met de lading en de massa der ionen in verband staan)提出關於電磁學與光學的理論,獲1902年諾貝爾物理獎。
\item 1899:拉塞福(Ernest Rutherford, 1st Baron Rutherford of Nelson)讓放射性元素衰變放出之射線通過磁場分離出帶正電和負電的射線,並命名之為 α 射線與 β 射線。
\item 1899:J. J. 湯姆森利用紫外光照射陰極射線管陰極發生光電效應(Photoelectric effect)。
\item 1900:瑞立(John William Strutt, 3rd Baron Rayleigh)與京士(James Jeans)提出關於黑體輻射的瑞立–京士定律(Rayleigh–Jeans law)。
\item 1900:普朗克(Max Planck)提出能量量子化的量子論與關於黑體輻射的普朗克定律(Planck's law)。
\item 1900:維拉得(Paul Ulrich Villard)發現一種穿透力強且不受電磁場偏折的射線,後由拉塞福命名為 γ 射線。
\item 1901:馬可尼以火花隙發射器首次達成跨大西洋無線電報傳輸。
\item 1902:雷納(Philipp von Lenard)進行光電效應實驗並提出底限頻率,獲1905年諾貝爾物理獎。
\item 1903:拉塞福命名 γ 射線。
\item 1903:拉塞福與索迪(Frederick Soddy)發現氡氣,並提出放射性衰變(Radioactive decay)定律,分別獲1908年和1921年諾貝爾化學獎。
\item 1904:J. J. 湯姆森提出原子的葡萄乾布丁模型(Plum pudding model)。
\item 1905:愛因斯坦(Albert Einstein)發表狹義相對論(Special relativity);發表布朗運動的理論;發表光量子論(Light quantum theory)與愛因斯坦光電方程式(Einstein's photoelectric equation),解釋光電效應;獲1921年諾貝爾物理獎。
\item 1906:萊曼(Theodore Lyman IV)發現氫原子光譜的萊曼系列。
\item 1907:拉塞福證明 α 粒子是遊離的氦原子。
\item 1908:佩蘭(Jean Baptiste Perrin)以實驗證實愛因斯坦關於布朗運動的理論,獲1926年諾貝爾物理獎。
\item 1908:瑞茲(Walther Ritz)提出關於原子光譜的芮德伯–瑞茲組合原理(Rydberg–Ritz combination principle)。
\item 1908:帕申(Friedrich Paschen)發現氫原子光譜的帕申系列。
\item 1909:密立坎(Robert Andrews Millikan)做油滴實驗(Oil drop experiment),測得電子電量。
\item 1909:拉塞福、蓋革(Hans Geiger)和馬士登(Ernest Marsden)研究 α 粒子散射實驗。
\item 1911:拉塞福以 α 粒子金箔散射實驗發現原子核,並提出原子的行星模型。
\item 1911:居里夫婦分離出純鐳,獲1911年諾貝爾化學獎。
\item 1912:勞厄(Max von Laue)發現 X 射線晶體繞射現象,並確認X射線為電磁波,獲1914年諾貝爾物理獎。
\item 1913:J. J. 湯姆森利用質譜儀(Mass spectrometry)發現同位素。
\item 1913:布拉格父子(W. H. Bragg and W. L. Bragg)研究 X 射線晶體繞射,提出布拉格公式(Bragg's law),並測量 X 射線波長與晶體布拉格面間距(Interplanar spacing),獲1915年諾貝爾物理獎。
\item 1913:波耳(Niels Bohr)提出波耳模型(Bohr model)解釋氫原子光譜,獲1922年諾貝爾物理獎。
\item 1914:莫斯利(Henry Moseley)發現關於原子光譜的莫斯利定律(Moseley's law),認為原子序才是決定元素化學性質的主要因素,並以之重新排序元素週期表。
\item 1915:愛因斯坦發表廣義相對論(General relativity)。
\item 1916:密立坎發表光電效應實驗結果,以截止電壓證實愛因斯坦光電方程式,因對基本電荷與光電效應的研究獲1923年獲貝爾物理獎。
\item 1916:科塞爾(Walther Kossel)提出關於離子鍵的理論。
\item 1916:路易斯(Gilbert Newton Lewis)發明路易斯電子點式(Lewis electron dot structures),並提出八隅體規則(Octet rule)。
\item 1919:拉塞福以 α 粒子轟擊氮原子核發現一粒子,即後稱之質子,是史上首次人工核反應。
\item 1920:拉塞福發現質子即氫-1核並命名之為質子,並預測中子存在。
\item 1923:康普頓(Arthur Holly Compton)發現康普頓散射(Compton scattering),證實光子的存在。
\item 1924:德布羅意(Louis de Broglie)提出物質波(Matter wave)假說,獲1929年諾貝爾物理獎。
\item 1925:洪德(Friedrich Hund)提出洪德定則前兩條。
\item 1925:包立(Wolfgang Pauli)發表包立不相容原理(Pauli exclusion principle)。
\item 1925:玻色(Satyendra Nath Bose)提出玻色–愛因斯坦統計(Bose–Einstein statistics)。
\item 1925:海森堡(Werner Heisenberg)發表矩陣力學(Matrix mechanics)處理粒子運動,成功解釋原子光譜。
\item 1926:費米與狄拉克(Paul Adrien Maurine Dirac)提出費米–狄拉克統計(Fermi–Dirac statistics)。
\item 1926:波恩(Max Born)提出物質波函數的平方描述粒子在某一時刻出現在某一位置的機率,獲1954年諾貝爾物理獎。
\item 1926:薛丁格(Erwin Schrödinger)利用波函數(Wave function)描述物質波,並提出波函數必須遵循的微分方程,稱薛丁格方程(Schrödinger's equation),以波動力學(Wave mechanics)提出原子軌域(Atomic orbital)成功解釋原子光譜,並證明了波動力學與矩陣力學的等價性,統稱量子力學(Quantum mechanics),獲1933年諾貝爾物理獎。
\item 1927:達維森(Clinton Davisson)、格末(Lester Germer)以戴維森–革末實驗(Davisson–Germer experiment),即電子的鎳晶體繞射,證實電子的物質波,獲1937年諾貝爾物理獎。
\item 1927:G. P. 湯姆森(George Paget Thomson)以電子的金屬箔繞射實驗證實電子的物質波,獲1937年諾貝爾物理獎。
\item 1927:波耳發表互補性(Complementarity)。
\item 1927:海森堡(Werner Heisenberg)提出不確定性原理(Uncertainty principle)。
\item 1927:海特勒(Walter Heitler)與倫敦(Fritz London)提出價鍵理論(Valence bond theory)的前身。
\item 1927:洪德提出洪德定則第三條。
\item 1928:狄拉克提出狄拉克方程式(Dirac equation),並開展相對論性量子力學(relativistic quantum mechanics),獲1933年諾貝爾物理獎。
\item 1929:瓊斯(John Lennard-Jones)提出分子軌域理論(Molecular orbital theory)的許多內容。
\item 1930:狄拉克(Paul Dirac)預測正電子(Positron)。
\item 1930:包立提出 β 衰變中會放出一沒有電荷、沒有質量、自旋1/2的粒子,後被命名為微中子(neutrino),獲1995年諾貝爾物理獎。
\item 1931:包立提出價鍵理論與混成軌域(Orbital hybridization)。
\item 1931:波特(Walther Bothe)發現釙發出的 α 射線落在鈹、硼或鋰上會產生一種穿透力極強且不受電場影響的射線,後發現為中子束;居里夫婦重做該實驗,並以之射在石蠟板上發現其放出質子,以為該射線是 γ 射線。
\item 1932:安德森(Carl David Anderson)利用雲霧室(Cloud chamber)發現正電子,獲1926年諾貝爾物理獎。
\item 1932:費米(Enrico Fermi)以中子轟擊各原子序的原子核,發現其會進入不穩定激發態後經 β 衰變轉變成原子序多一的原子核。
\item 1932:查兌克(James Chadwick)以 α 粒子轟擊鈹原子核發現中子。
\item 1932:鮑林(Linus Carl Pauling)提出鮑林電負度(Pauling electronegativity)。
\item 1933:費米命名微中子(neutrino)並描述其性質。
\item 1934:馬利肯(Robert Sanderson Mulliken)提出馬利肯電負度(Mulliken electronegativity)。
\item 1934:契忍可夫(Pavel Cherenkov)發現契忍可夫輻射(Cherenkov radiation)。
\item 1934:湯川秀樹(Hideki Yukawa)提出介子(meson)。
\item 1934:居里夫婦以 α 粒子轟擊鋁-27得磷-31並快速放出中子變成磷-30,是首次人工產生放射性元素,稱人工誘發放射性(Artificial induced radioactivity),共同獲1935年諾貝爾化學獎。
\item 1934:費米提出弱核力(Weak nuclear force)解釋 β 衰變;發現與氫原子碰撞後的慢中子更易誘發核反應;以慢中子轟擊鈾製得原子序93的錼,是首次製得超鈾元素(Transuranium elements),獲1938年諾貝爾物理獎;進行鈾的核分裂(Nuclear fission)實驗;提出物質與反物質的產生與湮滅(Annihilation)。
\item 1935:湯川秀樹提出強核力(Strong nuclear force)。
\item 1938:哈恩(Otto Hahn)、邁特納(Lise Meitner)、施特拉斯曼(Fritz Strassmann)與弗里施(Otto Frisch)確認中子轟擊鈾發生核分裂。
\item 1938:費米提出核分裂連鎖反應(Chain reaction)的概念。
\item 1938:貝特(Hans Albrecht Bethe)提出關於恆星核融合的理論。
\item 1938:居里夫婦證實核分裂連鎖反應。
\item 1939:波耳和惠勒(John Archibald Wheeler)建立核分裂學說。
\item 1941:康普頓與費米算出 U-235 的臨界質量。
\item 1942:美國開始曼哈頓計畫(Manhattan Project),研發原子彈。
\item 1942:費米建造史上首座核反應爐——芝加哥1號堆(Chicago Pile-1, CP-1),進行鈾的受控核分裂反應。
\item 1944:西博格(Glenn T. Seaborg)提出關於錒系元素的理論。
\item 1945:美國奧本海默(J. Robert Oppenheimer)、班布里奇(Kenneth Bainbridge)、費米等進行三位一體(Trinity)核試驗,是史上首次核爆炸。
\item 1945:廣島與長崎原子彈爆炸(Atomic bombings of Hiroshima and Nagasaki)。
\item 1947:鮑威爾(Cecil Powell)、 Hugh Muirhead、César Lattes 和 Giuseppe Occhialini 發現介子。
\item 1947:諾基亞貝爾實驗室(Nokia Bell Labs)的巴丁(John Bardeen)、布拉頓(Walter Houser Brattain)和肖克利(William Shockley)以鍺材料發明點接觸電晶體(point-contact transistor),是史上首個半導體電子元件,獲1956年諾貝爾物理獎。
\item 1948:費曼(Richard Feynman)提出費曼圖(Feynman diagram),獲1965年諾貝爾物理獎。
\item 1948:第九屆國際計量大會(CGPM)委託進行一項研究,評估科學、技術和教育界的測量需求,並「為所有遵守《米制公約》的國家製定一個適用的單一計量單位體系提出建議」。
\item 1951:羅莎琳•富蘭克林(Rosalie Franklin)研究 X 射線繞射,發現 A-DNA 與 B-DNA 結構轉換與磷酸基團位於 DNA 螺旋之外。
\item 1952:美國進行常春藤麥克(Ivy Mike)氫彈試爆。
\item 1953:華生(James Watson)、克里克(Francis Crick)與威爾金斯(Maurice Wilkins)提出 DNA 的雙股螺旋結構,獲1962年諾貝爾生理暨醫學獎。
\item 1954:第十屆國際度量衡大會基於公尺/米-公斤-秒單位制(metre, kilogram, second system of units, MKS system of units)決定建立國際單位制/SI 制(Système International d'Unités, International System of Units, SI)。
\item 1956:科溫(Clyde Cowan)與萊因斯(Frederick Reines)發現微中子(Neutrino)。
\item 1956:雷諾斯(Frederick Reines)和柯旺發現反微中子(Antineutrino)。
\item 1958:蓋爾曼(Murray Gell-Mann)、費曼、George Sudarshan 與 Robert Marshak 發現了物理弱相互作用的手性結構,並發展向量減軸向量理論(vector minus axial vector theory)。
\item 1960:第十一屆國際度量衡大會通過採納國際單位制。
\item 1961:瓊森(Claus Jönsson)做電子的楊格雙狹縫實驗。
\item 1964:蓋爾曼提出夸克(quark)。
\item 1964:布勞特(Robert Brout)、恩格勒(François Englert)、希格斯(Peter Higgs)、古拉尼(Gerald Stanford Guralnik)、哈庚(Carl Richard Hagen)、基博爾(Thomas Walter Bannerman Kibble)提出希格斯玻色子(Higgs boson)
\item 1965:彭齊亞斯(Arno Penzias)與威爾遜(Robert Wilson)發現宇宙微波背景(Cosmic Microwave Background)。
\item 1968:美國史丹佛線性加速器中心(Stanford Linear Accelerator Center, SLAC)發現夸克。
\item 1969:蓋爾曼提出核子由夸克組成的理論。
\item 1974:格拉肖(Sheldon Glashow)、薩拉姆(Abdus Salam)與溫伯格(Steven Weinberg)提出電弱交互作用(Electroweak interaction),描述電磁力和弱核力在高能下的統一性質,並預言 W 與 Z 玻色子(Boson)的存在。
\item 1977:白川英樹、麥克德爾米德(Alan Graham MacDiarmid)與希格(Alan Jay Heeger)發明摻碘聚乙炔,共同獲2000年諾貝爾化學獎。
\item 1980:西博格(Glenn T. Seaborg)在加速器內轟擊鉍-209使核轉變為金-197。
\item 1983:歐洲核子研究中心(European Organization for Nuclear Research, CERN)的實驗中,魯比亞(Carlo Rubbia)與 Simon van der Meer 發現 W 與 Z 玻色子,證實電弱交互作用。
\item 1991:O. Carnal 和 J. Mlynek 做 α 粒子的楊格雙狹縫實驗。
\item 1992:赤崎勇、天野浩與中村修二以氮化鎵材料發明藍光 LED,共同獲2014年諾貝爾物理獎。
\item 1998:芮斯(Adam Riess)、施密特(Brian Schmidt)與裴穆特(Saul Perlmutter)發現宇宙加速膨脹。
\item 2004:蓋姆(Andre Konstantin Geim)領導的團隊利用普通膠帶沾黏石墨首次製造出單層石墨烯,蓋姆及其學生諾沃肖洛夫(Konstantin Novoselov)共同獲2010年諾貝爾物理獎。
\item 2012:歐洲核子研究中心的大型強子對撞機(Large Hadron Collider, LHC)實驗中發現希格斯玻色子,證實標準模型(Standard model)。
\item 2017:雷射干涉重力波天文台(Laser Interferometer Gravitational-Wave Observatory, LIGO)和室女座干涉儀(Virgo interferometer, VIRGO)觀測到兩個中子星併合的重力波事件 GW170817。
\item 2018:第二十六屆國際度量衡大會上通過採納重新定義公斤的提案,並於2019年5月開始生效。
\end{itemize}
\end{document}