\documentclass[a4paper,12pt]{report}
\setcounter{secnumdepth}{5}
\setcounter{tocdepth}{3}
\newcounter{ZhRenew}
\setcounter{ZhRenew}{1}
\newcounter{SectionLanguage}
\setcounter{SectionLanguage}{1}
\input{/usr/share/latex-toolkit/template.tex}
\begin{document}
\title{化學鍵與固態物理學}
\author{沈威宇}
\date{\temtoday}
\titletocdoc
\ch{化學鍵(Chemical Bonding)與固態物理學(Solid-State Physics)}
此文件因字型問題將「金翁」(U+9393)一字寫作「金翁」二字。
\sct{化學鍵(Chemical bond)總論}
\ssc{化學鍵(Chemical bond)}
化學鍵是原子靠近時彼此之電子經由某種交流方式使位能降低的交互作用,包含共價鍵(Covalent bonds)、離子鍵(Ionic bonds)與金屬鍵(Metallic bonds)。
\ssc{成鍵(Bonding)、鍵長/鍵距(Bond length)與鍵能(Bond energy)}
原子或離子間靜電吸引力和排斥力相同時達平衡且靜電位能最小時稱成鍵,此時兩者間距離為鍵長/鍵距(Bond length),位能較兩者距離為無限遠時減少的量為鍵能(Bond energy),即打破該鍵所須的能量。若將兩者距離拉遠則吸引力位能增加量大於排斥力位能減少量,若將兩者距離拉近則排斥力位能增加量大於吸引力位能減少量,總電位能均上升。
\ssc{原子或分子間交互作用的穩定能(Stabilization energy)}
穩定能指該交互作用發生相較於不發生時的位能改變量,為負。對於化學鍵與氫鍵,穩定能乘以負一稱鍵能(bond energy)。通常:
\begin{itemize}
\item \tb{共價鍵(Covalent bonds)、離子鍵(Ionic bonds)}:150-500 kJ/mol
\item \tb{金屬鍵(Metallic bonds)}:50-150 kJ/mol
\item \tb{氫鍵(Hydrogen bonds)}:5-40 kJ/mol
\item \tb{凡得瓦力(Van der Waals forces)}:<5 kJ/mol
\end{itemize}
\ssc{鍵三角 (Bond triangle)/Van Arkel-Ketelaar triangle}
利用電負度推測兩原子間鍵結為離子鍵、共價鍵或金屬鍵的成分的圖表,$x$軸為兩原子的電負度平均,$y$軸為兩原子的電負度差,略為左下小三角形為金屬鍵,右下大三角形除去與金屬鍵三角形重疊處為共價鍵,剩餘中上平行四邊形為離子鍵。
\ssc{不同化學鍵形成的物質比較表}
\begin{longtable}[c]{|p{0.06\tw}|p{0.16\tw}|p{0.06\tw}|p{0.12\tw}|p{0.12\tw}|p{0.12\tw}|p{0.1\tw}|p{0.06\tw}|}
\hline
物質 & 化學鍵 & 延展性 & 硬度與脆度 & 溶解性 & 可溶於水者水溶液導電性 & 固態導電性 & 液態導電性 \\\cline{2-8}
& 熔沸點 & 常溫常壓狀態 & 元素/化合物 & 僅有式量 & 成分元素 & 構成單位 & \\\hline
\endhead
離子晶體 & 離子鍵 & 否 & 硬且脆,易脆裂為一定晶面與晶形 & 多數可溶於極性溶液 & 是 & 否 & 是 \\\cline{2-8}
& 次高,略為晶格能愈小熔點愈低,如 NaCl 801°C、MgO 2800°C & 固 & 化合物 & 是 & 金屬陽離子與/或銨根離子與非金屬陰離子與/或酸根離子 & 陰陽離子(團) & \\\hline
共價網狀固體 & 共價鍵 & 否 & 三度空間網狀排列者硬度極大,二度空間網狀排列者層間易滑動故質軟 & 否 & 否 & 除少數具連續共軛體系者外否,是者如石墨為導體、矽為半導體 & 否,無液態 \\\cline{2-8}
& 最高,鑽石、石墨熔點均>3500°C & 固 & 都有 & 是 & 通常是非金屬與/或類金屬,如 Si、SiO$_2$、SiC、B、C & 原子 & \\\hline
分子 & 共價鍵 & 否 & 小 & 同類互溶,非極性者較多 & 溶於水為酸鹼者是,前者如 HX、SO$_2$、SO$_3$、NO$_2$、羧酸,後者如 NH$_3$、胺 & 除少數具連續共軛體系者外否,是者如摻碘聚乙炔 & 否 \\\cline{2-8}
& 最低,因僅需破壞分子間作用力 & 都有 & 都有 & 否,有分子量 & 非金屬為主,亦有金屬如 AlCl$_3$ & 分子 & \\\hline
金屬晶體 & 金屬鍵 & 是 & 多數大,鹼金屬軟,略與族數正相關 & 除鹼金屬、Ca、Ba、Sr 溶於水外否 & 是,鹼 & 是 & 是 \\\cline{2-8}
& 跨距大,略介於離子晶體與共價分子之間,熔點最高者 W 3422°C,最低者 Hg -39°C & 
Hg 液,其餘固 & 都有 & 是 & 金屬 & 原子 & \\\hline
\end{longtable}\FB


\sct{離子鍵(Ionic bonds)}
數種陰陽離子(團)因彼此所帶的電荷,藉由庫侖靜電力形成的化學鍵。無方向性。
\ssc{離子晶體(Ionic crystals)}
\sssc{定義}
陽離子與陰離子以總電荷為零的比例規則交替排列於離子晶體中,呈現出規則的幾何外形。
\sssc{典型特性}
\bit
\item 質硬而脆,無延展性,因為受到應力使陰陽離子錯位,造成離子間斥力大於引力,使晶體碎裂。
\item 解離與熔融態均可導電。
\item 熔化或汽化須破壞晶格與離子鍵,故熔沸點較共價分子高。
\eit
\ssc{同型與不同型離子}
同型離子指電荷數絕對值相同的陰陽離子。不同型離子指電荷數絕對值不同的陰陽離子。
\ssc{穩定能}
\sssc{定義}
指將陰陽離子團破壞為分開的氣態成分陰陽離子所需的能量乘以負一。
\sssc{無共價性同型離子對(Ion pair)穩定能}
一個陰離子和一個陽離子相鄰可達到之最小位能狀態,穩定能為庫侖常數乘以陰陽離子電荷乘積除以陰陽離子半徑和。
\sssc{無共價性同型離子正方穩定能}
二個陰離子和二個陽離子交錯排列於平面正方形中可達到之最小位能狀態,穩定能為庫侖常數乘以$(4-\sqrt{2})$陰陽離子電荷乘積除以陰陽離子半徑和。
\sssc{無共價性同型離子立方穩定能}
四個陰離子和四個陽離子交錯排列於立體正方體中可達到之最小位能狀態,穩定能為庫侖常數乘以$(12-6\sqrt{2}+\frac{4\sqrt{3}}{3})$陰陽離子電荷乘積除以陰陽離子半徑和。
\sssc{晶格能(Lattice energy)}
一莫耳氣態陰陽離子形成離子晶體的能量變化,必為負,如 NaCl 的晶格能為 -786 kJ mol$^{-1}$ 因為 \ce{Na+(g) + Cl-(g) -> NaCl(s) + 786} kJ mol$^{-1}$。晶格能量值愈大,通常熔點愈高,晶格愈穩定,硬度愈大。與離子所帶電荷、半徑、晶體結構等有關。
\ssc{極化現象與法陽斯規則(Fajans' rules)}
極化現象發生於離子鍵與極性共價鍵,極化現象愈大,離子性愈小,鍵能愈小,陰離子/負偶極作為親核試劑/路易斯鹼愈強。依法陽斯規則:
\bit
\item 陽離子/正偶極半徑愈小或電荷愈大,愈易極化陰離子/負偶極。
\item 陰離子/負偶極半徑愈大或電荷愈大,愈易被陽離子/正偶極極化。
\item 過渡元素通常比典型元素容易極化陰離子,典型元素中 \ce{Li+}, \ce{Be^{2+}}, \ce{Al^{3+}}, \ce{Pb^{4+}}, \ce{Sn^{4+}} 極化能力較強,可極化鹵素離子等,有些甚至形成分子物質,如\ce{AlCl3},可溶於某些有機溶劑,如乙醇。
\eit
\ssc{晶格能量值比較經驗法則}
\sssc{同型離子}
略正比於電荷數乘積除以離子半徑和。如熔點:\ce{MgO}>\ce{LiF}>\ce{NaCl}>\ce{KI}。
\sssc{不同型離子}
略為:
\bit
\item 先比電負度差:電負度差$\Delta EN$愈大,離子性愈大,離子鍵愈強。如熔點:\ce{CaCl2}>\ce{MgCl2}>\ce{BeCl2},但 \ce{Na2O}<\ce{Al2O3}<\ce{MgO}。
\item 次比陽離子價數:陽離子價數愈高,離子性愈弱,離子鍵愈弱。如熔點:\ce{PbCl2}>\ce{PbCl4}、\ce{MgO}>\ce{Al2O3},但\ce{Na2O}<\ce{Al2O3}<\ce{MgO}。
\item 末比極化現象:極化現象愈大,離子性愈小,離子鍵愈弱。
\eit
\ssc{同型離子晶體堆積的半徑比準則(Radius ratio rule)}
同型離子晶體堆積方式為陽離子半徑除以陰離子半徑的函數,稱半徑比準則。
\begin{longtable}[c]{|c|c|}
\hline
陽離子半徑除以陰離子半徑 & 堆積方式 \\\hline\endhead
[0.155, 0.225) & 平面三角 \\\hline
[0.225, 0.414) & B3/ZnS \\\hline
[0.414, 0.732) & B1/NaCl \\\hline
[0.732, 1) & B2/CsCl \\\hline
\end{longtable}\FB


\sct{金屬(Metal)}
\ssc{金屬晶體(Metallic crystals)與金屬鍵(Metallic bonds)}
金屬原子因游離能較低且多空價軌域故易於失去價電子,失去價電子後的金屬陽離子規則排列於金屬晶體中,失去的價電子則在整個金屬晶體中自由移動,該等電子稱自由電子(free electrons)或離域電子(delocalized electrons),如同金屬陽離子浸在電子海(sea of electrons)中,形成金屬晶體,此時自由電子與金屬陽離子的吸引力稱金屬鍵。無方向性。
\sssc{金屬鍵強弱}
指金屬硬度、熔點、汽化熱大小等。由電荷密度及晶體堆積方式共同決定。
\sssc{典型特性}
\bit
\item 熱的良導體:因為離域電子傳遞無序動能的能力優於晶格振動,故金屬晶體與具連續共軛系統的共價網狀固體的導熱性一般優於分子晶體、無連續共軛系統的共價網狀固體與離子晶體。
\item 電的良導體:當一定數量電子進入金屬一端,等量的電子自另一端外流。溫度愈高,金屬離子震動愈劇烈,導電性愈差。
\item 延展性佳:金屬離子可在電子海中滑動,故延展性佳,可被拉成細絲或敲成薄片。延展性最高的金屬純質為鉑,次之為金,它們 1 g 可拉成約 4000 m 細絲。
\item 具高光澤:當光照射到金屬表面時,金屬原子振動能階吸收各頻率電磁波使電子激發,當電子躍遷至低能階時放出金屬光澤。多數金屬為銀白至銀灰色,銅為紅色,金與銫為金色。
\item 硬度:跨度較大。小於三維共價網狀固體,並多大於大部分的共價分子晶體與離子晶體。部分金屬較軟,如鈉可用刀片切割。
\item 熔沸點:跨度較大。小於共價網狀固體、多大於大部分共價分子,並多小於離子性較強的離子晶體。部分金屬熔沸點小於共價分子。汞熔點 -39°C、沸點 357°C;鎢熔點 3422°C、沸點 5660°C。
\eit
\ssc{金屬鍵能比較經驗法則}
\sssc{相同堆積方式}
略正比於價電子數的平方除以金屬半徑,故:
\bit
\item 同族相同堆積方式金屬原子序愈大者金屬鍵愈弱。如:鹼金屬均為體心立方,熔點原子序愈大愈低。
\item 同週期相同堆積方式金屬原子序愈大者金屬鍵愈強。
\eit
\sssc{不同堆積方式}
受堆積方式影響。
\bit
\item 有時仍略為價電子數的平方除以金屬半徑愈大者金屬鍵愈強。如熔點:Na<Mg<Al。
\item 有時則否。如熔點:Be>Ca>Sr>Ba>Mg。
\eit
\ssc{合金(Alloy)}
一種由兩種或多種元素組成的材料,其中至少有一種是金屬。分為金屬間化合物(intermetallic compound)、固溶體(solid solution)和非勻相混合物,後兩者較常見。
\sssc{金屬間化合物(Intermetallic compound)}
由兩種或多種元素以固定的比例組成,組成元素之間形成較強且具有金屬鍵與共價鍵性質的鍵結,形成具有特定化學計量比的化合物,具有特定有序的晶體結構。相較於主要組分金屬純質,通常熔點更高、更耐腐蝕、硬度更大、脆度更大、延展性更差、導電度更小、導熱性更差。如鈦鋁\ce{TiAl}和鎳鋁\ce{Ni3Al}。

C、P、S 等非金屬與金屬間化合物之合金有時也視為金屬間化合物有時則否,它們通常在製造過程中加入\ce{C}、\ce{P4}、\ce{S8}製成,其非金屬與金屬間鍵結為共價鍵,通常熔點更低、硬度更大、脆度更大、延展性更差、導電度更小、導熱性更差。
\sssc{金屬固溶體(solid solution)}
由一種金屬元素作為溶劑,溶解另一種或多種作為溶質的物質形成,溶質可以是金屬或非金屬,並且它們的比例不須固定。固溶體保持溶劑金屬的晶體結構,溶質元素隨機地占據溶劑的晶格點或晶格間隙中,前者稱置換式,後者稱間隙式,形成均勻的單相固體。相較於溶劑金屬純質,通常熔點更低、更耐腐蝕、硬度更大、延展性更佳、導電度更小、導熱性更差。如黃銅為鋅溶於銅的置換式固溶體、生鐵為碳溶於鐵的間隙式固溶體。
\sssc{熔融法}
將組分加熱至熔點,使其混合形成均勻熔液,然後冷卻成合金固體。
\sssc{粉末冶金法}
將組分粉末混合,然後在高壓和高溫下燒結成合金固體。
\ssc{金屬的應用}
\bit
\item 鑄造:銅、鐵、鋁、錫等。
\item 飾品:金、鉑、銀等,其化性小,光澤美麗。
\item 還原劑:鉀、鈉、鋁等,其失去電子的傾向大。
\item 燈絲:鎢的熔點高,白熾燈泡的燈絲材料主要成分通常是鎢。
\item 燈管:日光燈管內含汞蒸氣。
\item 催化劑:很多過渡元素是一些反應的催化劑。
\eit


\sct{共價鍵(Covalent bond)}
兩原子間共用價電子,藉由原子核與共用電子對的靜電引力形成的化學鍵,有方向性。
\ssc{電子}
\sssc{(電子)對((electron) pair)}
同一軌域中的兩自旋相反的電子。
\sssc{孤(電子)對/孤對電子(lone pair, l.p.)}
只屬於一個原子的電子對。僅受本身原子核吸引故電子雲較寬廣。
\sssc{鍵結(電子)對/鍵結對電子(bonding pair, b.p.)/共用(電子)對/共用對電子(shared pair)}
由參與共價鍵的二個原子共用的電子對。同時受兩原子核吸引故電子雲較細長。
\sssc{不成對電子(Unpaired electron)}
其所屬軌域只有其一個電子的電子。電子雲機率密度較低。
\sssc{離域/非定域(Delocalized)電子}
不局限於特定幾個原子間的電子,如共軛系統中的離域 π 電子與金屬鍵中的自由電子(Free electrons)。
\sssc{鍵級/鍵序(Bond order)/鍵數}
一個共價鍵的鍵結電子數除以二。對於具有共振結構者,一鍵的(有效)鍵級為該鍵的有效鍵結電子數除以二。鍵級為$k>0$之鍵稱$k$(重)鍵,鍵級為 1 之鍵稱單(重)鍵,鍵級為 2 之鍵稱雙(重)鍵,鍵級為 3 之鍵稱參(重)鍵,鍵級 >1 之鍵稱多(重)鍵。
\ssc{分子(Molecules)}
\sssc{定義}
原子間以共價鍵形成的具有物質特性的最小粒子,或具有物質特性的單原子,可以分子式表示,後者稱單原子分子(monatomic molecule)。具有離子團的離子晶體因離子團不能單獨表現物質特性,故不是分子。
\sssc{分子晶體(Molecular crystals)/分子固體(Molecular solid)}
分子間以微弱的凡德瓦力或氫鍵維持分子固體結構中的相對位置。
\sssc{典型特性}
固體質軟、結構鬆散、無延展性、除少數具連續共軛體系者外不可導電。熔沸點低,因為熔化或汽化僅需破壞微弱的分子間作用力。分子物質中沸點較高者如 \ce{S8} 444.6°C,高於鈉,非單原子中沸點最低者為 \ce{H2} -252.9°C,單原子中沸點最低者為 \ce{He} -268.9°C。
\ssc{共價網狀固體(Covalent network solid)}
\sssc{定義}
原子間以共價鍵形成一維、二維或三維排列且結構可以無限延伸的物質,僅能以實驗式表示。
\sssc{典型特性}
質硬,熔沸點最高,除少數具連續共軛體系者外不可導電,不溶。
\ssc{雙中心雙電子共價鍵}
\sssc{典型共價鍵}
兩原子各提供一個半填滿的軌域,且其中電子自旋相反。
\sssc{配位共價鍵(Coordinate covalent bond)}
一原子提供全滿軌域(即原本的孤對電子),另一原子提供空軌域,稱由前者配位電子對給後者。
\ssc{共價分子的結構判斷}
\sssc{路易斯的八隅體規則(Octet rule)/八隅體學說}
除\ce{H}、\ce{He}有二個價電子時化性最安定外,其餘元素價電子八個化性最安定,因為填滿價軌域時能量最低。
\sssc{自由基(Radical or free radical)}
具有不成對電子的分子,如:\ce{NO}、\ce{NO2}。
\sssc{超價分子(hypervalent molecule)}
具有價電子數超過路易斯的八隅體規則的原子的分子,一般為中心原子為 d 軌域參與鍵結的原子,如:\ce{PCl5}、\ce{PF5}、\ce{SF6}、\ce{I3}、\ce{XeF2}。
\sssc{缺電子分子(electron-deficientmolecule)}
具有價電子數低於路易斯的八隅體規則的原子的分子,一般為中心原子不足四個價電子的原子,故無法形成四個兩原子各提供一個電子的典型共價鍵,如:\ce{BF3}、\ce{BeCl2}、\ce{BH2}。
\sssc{N-X-L notation(N-X-L 符號)}
N 為中心原子的價電子數,X 為中心原子的元素符號,L 為連接到中心原子的原子數,間以 - 相連,如\ce{XeF2}為 10-Xe-2。
\sssc{形式電荷(Formal charge, FC)}
是一種過分考慮共價性的計算方法。使用印度-阿拉伯數字表示,並不省略正號。必須滿足:
\bit
\item 原子上的孤電子對被分配到該原子。
\item 原子周圍每個共價鍵中的一半電子被分配給該原子。
\item 形式電荷總和等於電荷數。
\eit
通常合理結構應為滿足八隅體後依以下規則最有利者:
\ben
\item 形式電荷的絕對值之和愈小者。
\item 負形式電荷位於電負性愈強的原子上者。
\item 帶同種形式電荷的原子距離愈遠者。
\een
\sssc{單中心分子的結構判斷經驗法則}
\ben
\item 以氫以外電負度最小者作為中心原子。
\item 將各周圍原子與中心原子以單鍵相連。
\item 剩餘電子以孤電子對填滿周圍原子使它們為八隅體。
\item 剩餘電子以孤電子對填入中心原子。
\item 若中心原子仍不足八隅體,則把周圍原子的孤電子對變鍵結電子對,形成多鍵,直到中心原子滿足八隅體。
\item 若以上步驟有多種不同可能結構,則取各原子形式電荷絕對值之和最小者。
\item 若以上步驟有多種不同可能結構,則均為共振結構。
\een
\ssc{鍵角(Bond angle)}
一原子上兩共價鍵的夾角。
\ssc{鍵長與鍵能比較經驗法則}
\sssc{鍵長}
略為:
\ben
\item 先比週期,週期小者短。
\item 次比鍵級,鍵級大者短。
\item 末比兩側原子共價半徑和,兩側原子共價半徑和小者短。單鍵鍵長約等於兩側原子共價半徑和。
\een
\sssc{鍵能}
略為:
\ben
\item 先比鍵級,鍵級大者強。
\item 次比電負度差,電負度差愈大,極性與離子性愈大,鍵能愈強。
\item 再次比鍵長,鍵長愈短,因靜電位能與距離呈反比,鍵能愈強。惟有例外之氟斥效應:鍵能:\ce{Cl2}>\ce{Br2}>\ce{F2}>\ce{I2}。
\item 同鍵級、同參與鍵結的原子,兩側提供的軌域平均能階愈低,鍵能愈強。一側鍵結軌域相同下,鍵能依另一側鍵結軌域:s>sp>sp$^2$>sp$^3$>sp$^3$d。
\een
\subsection{鍵離解能(Bond dissociation energy, D$_0$, BDE)}
鍵解離能是 A−B 透過均裂解開得到片段 A 和 B 時的焓變,這些片段通常是自由基物質或原子。
\ssc{常見共價鍵的鍵解離能}
\begin{longtable}[c]{|c|c|}
\hline
Bond & (Approximate) BDE (kJ/mol)\\\hline
\endhead
H-H & 436\\\hline
甲烷 C-H & 439\\\hline
乙烷 C-H & 423\\\hline
異丙烷 C-H & 414\\\hline
叔丁烷 C-H & 404\\\hline
三甲胺 C-H & 381\\\hline
二甲醚 C-H & 385\\\hline
丙酮 C-H & 402\\\hline
乙烯 C-H & 464\\\hline
乙炔 C-H & 556\\\hline
苯 C-H & 473\\\hline
O-H & 460\\\hline
N-H & 390\\\hline
C-C & 350\\\hline
苯 C-C & 481\\\hline
C=C & 611\\\hline
C$\equiv$C & 835\\\hline
N-C & 300\\\hline
O-C & 350\\\hline
H-F & 570\\\hline
C-F & 450\\\hline
N-F & 270\\\hline
O-F & 180\\\hline
S-F & 310\\\hline
H-Cl & 432\\\hline
C-Cl & 330\\\hline
N-CI & 200\\\hline
O-CI & 200\\\hline
H-Br & 366\\\hline
C-Br & 270\\\hline
N-Br & 240\\\hline
O-Br & 210\\\hline
S-Br & 210\\\hline
H-I & 298\\\hline
C-I & 240\\\hline
O-I & 220\\\hline
S-Cl & 250\\\hline
H-N & 390\\\hline
C-N & 300\\\hline
N-N & 240\\\hline
N=N & 418\\\hline
N$\equiv$N & 945\\\hline
O-N & 200\\\hline
H-O & 460\\\hline
C-O & 350\\\hline
C=O & 732\\\hline
N-O & 200\\\hline
O-O & 180\\\hline
O=O & 498\\\hline
H-S & 340\\\hline
C-S & 260\\\hline
C=S & 352\\\hline
O=S & 469\\\hline
S-S & 225\\\hline
H-P & 326\\\hline
C-P & 263\\\hline
N-P& 209\\\hline
O-P & 502\\\hline
F-F & 159\\\hline
Cl-Cl & 243\\\hline
Br-Br & 193\\\hline
I-I & 151\\\hline
\end{longtable}
\FloatBarrier
\ssc{價鍵理論(Valence bond theory, VB theory)}
\sssc{主張}
當原子間以共用電子形成共價鍵結合時,以兩原子軌域的重疊(overlap)說明兩原子核同時吸引一鍵結電子對的理論。
\sssc{鍵結軌域}
\bit
\item \tb{軌域方向}:s 軌域沒有方向性,其餘軌域有方向,如 p$_x$軌域之$x$軸正向與$x$軸負向為頭(head),$yz$平面上任意方向皆為側/肩(lobe)。
\item \tb{核間軸(Internuclear axis)}:兩原子核中心之連線。
\item \tb{σ 鍵(σ bond)}:兩原子之軌域頭碰頭重疊(head-on overlap),可旋轉,斷之所須能量最高,s 軌域只能參與 σ 鍵。
\item  \tb{π 鍵(π bond)}:兩原子之軌域側對側/肩並肩/橫向重疊(laterally overlap),核間軸電子雲密度為零,不可旋轉,故有 π 鍵之化合物可能因此形成順反異構物,斷之所須能量低於 σ 鍵。
\item \tb{鍵級與鍵種}:單鍵必為 σ 鍵,雙、參鍵則再加上 π 鍵。
\eit
\sssc{軌域混成/雜化(Orbital hybridization)}
\tb{主張}:鮑林提出。中心原子在鍵結時,其原子軌域可能重新混合,稱軌域混成/雜化(Orbital hybridization),使原先能階不同的軌域簡併(degenerate)成同一能階,稱混成/雜化軌域(Hybrid orbital),其中必須填充有孤電子或 σ 鍵的共用電子對。用於解釋分子的形狀與單以原子軌域參與鍵結時的預測不同。

\tb{命名}:混成軌域命名為各參與混成的軌域依能階低者在前排序,寫下各副軌域符號,並將參與混成的該種軌域數量右上標,如:一個 s 軌域與二個同一殼層的 p 軌域混成的三個軌域稱 sp$^2$,一個 s 軌域、三個同一殼層的 p 軌域與二個同一殼層的 d 軌域混成的五個軌域稱 sp$^3$d$^2$,一個 s 軌域、三個同一殼層的 p 軌域與二個上一殼層的 d 軌域混成的五個軌域稱 d$^2$sp$^3$。

\tb{電子提升(Promotion of electron)}:部分認為,在混成前將參與混成的軌域中能量較低的一種內有成對電子者,其一會先躍遷至將參與混成的軌域中能量最低的空軌域,使鍵結前將參與混成的每個混成軌域最多只有一個電子,稱此過程電子提升。有電子提升者如\ce{BeH2} 的Be中一個 2s 軌域的電子躍遷至 2p 軌域再混成為兩個 sp 軌域;無電子提升者如\ce{H2O}之O、\ce{NH3}之N均直接混成為四個 sp$^3$ 軌域。

\tb{條件}:
\bit
\item 同一原子內能量相近的軌域方能參與混成。
\item 混成軌域數量與參與混成的軌域數量相同。
\item 混成軌域能量簡併、形狀相同,能量介於原先軌域能量之間。
\item 混成前原子軌域在同一平面者,混成軌域仍在該平面上;混成前原子軌域不在同一平面者,混成軌域亦不在同一平面上。
\item 混成軌域彼此間以符合是否在同一平面上之條件下距離最遠、斥力最小、能量最低的方位分布在原子周圍。
\item 鍵結後所有混成軌域都有一對 σ 電子或一個孤電子或一對孤電子,若有 π 鍵則由更高能量軌域形成而不以混成軌域形成,故原子的混成軌域個數等於其 σ 鍵數 $+\left\lceil\frac{\tx{孤電子數}}{2}\right\rceil$。
\eit
\ssc{分子軌域/軌道理論(Molecular Orbital Theory, MO Theory, MOT)}
\sssc{主張}
當原子間以共用電子形成共價鍵結合時,以原子軌域組合成分子軌域/軌道(Molecular orbitals),電子填充於其中說明的理論。

MO 理論中的鍵結更加離域,使其較價鍵理論更適用於具有等效非整數鍵級的分子,且可以解釋紫外可見光光譜(UV-VIS)。
\sssc{原子軌域/軌道線性組合法(Linear combination of atomic orbitals method, LCAO method)}
每個分子都有一組分子軌域,並近似為組成原子軌域的簡單加權和,使軌道波函數被修改,即電子雲形狀被改變。加權係數係根據位能最小化得出。此近似有三個主要要求:
\bit
\item 原子軌域對稱性匹配。頭對頭方能成 σ 鍵、肩並肩方能成 π 鍵,頭對肩則不可成鍵。

以兩核間軸為 z 軸,s, p, d 軌域的對稱性匹配為:
\begin{longtable}[c]{|c|c|c|}
\hline
原子軌域 & σ 鍵對稱性匹配 & π 鍵對稱性匹配 \\\hline\endhead
s & s, p$_z$, d$_{z^2}$ & 無 \\\hline
p$_x$ & 無 & p$_x$, d$_{xz}$ \\\hline
p$_y$ & 無 & p$_y$, d$_{yz}$ \\\hline
p$_z$ & s, p$_z$, d$_{z^2}$ & 無 \\\hline
d$_{xy}$ & 無 & 無 \\\hline
d$_{xz}$ & 無 & p$_x$, d$_{xz}$ \\\hline
d$_{yz}$ & 無 & p$_y$, d$_{yz}$ \\\hline
d$_{x^2-y^2}$ & 無 & 無 \\\hline
d$_{z^2}$ & s, p$_z$, d$_{z^2}$ & 無 \\\hline
\end{longtable}\FB
\item 原子軌域在空間內足夠近。
\item 原子軌域能階相近。若能量差很大,分子軌域形成時,能量的變化就會變小,使電子能量的降低不足以形成顯著的鍵結。
\eit
\sssc{分子軌域(Molecular orbitals)}
\bit
\item 成鍵軌域/鍵結軌域(Bonding orbital):電子雲密度在兩個原子之間較大,其中電子將傾向於將兩個原子保持在一起。能量較原先的原子軌域低。形狀與名稱同價鍵理論中的鍵結軌域。
\item 反鍵(結)軌域(Antibonding orbital):電子雲密度在兩個原子背後(遠離另一原子的方向)較大,其中電子將傾向於將兩個原子核拉開。能量較原先的原子軌域高。通過在成鍵軌域名稱後添加星號來表示,如反鍵結 π 記作 π *。通常在多鍵中才可能有電子填入其中。
\item 非鍵(結)軌域(Non-bonding orbital)。
\eit

\tb{電子填充}:優先填入能量較低的軌域,多數服從洪德定則。
\sssc{鍵級}
\[\tx{鍵級}=\frac{\tx{鍵結軌域電子數}-\tx{反鍵結軌域電子數}}{2}\]
\sssc{π* 軌域非成對電子}
在一些化合物中存在 π* 軌域非成對電子,貢獻 -0.5 鍵級,但總鍵級和總鍵能都還是正的,較價鍵理論更好地解釋了一些化合物的穩定性,如 NO 間,σ 軌域、σ* 軌域與兩個 π 軌域各有一對電子,π* 軌域有一個不成對電子,為 2.5 鍵。
\ssc{分子光譜(Molecular spectrum)/帶狀光譜(Band spectrum)}
指分子發射光譜/明帶光譜與吸收光譜/暗帶光譜,屬特徵光譜,除了原子既有的電子躍遷,各個原子核、電子間的相互作用與耦合造成更多其組分元素光譜之聯集所沒有的譜線,故略呈帶狀,前者之明帶與後者之暗帶重合,可用於得到分子與電子結構的資訊。可用分子軌域理論解釋,但無法用價鍵理論解釋。
\sssc{電子光譜(Electronic spectrum)}
涉及分子不同電子能階之間的躍遷,通常發生紫外線至可見光區。
\sssc{旋轉光譜(Rotational spectrum)}
涉及分子不同旋轉能階之間的躍遷,通常發生在微波至遠紅外線區。
\sssc{振動光譜(Vibrational)}
涉及分子不同振動能階之間的躍遷,通常發生在紅外線區。
\ssc{共振(Resonance)}
\sssc{共振(Resonance)}
存在不只一個電子配置不同但原子排列相同的合理結構(在八隅體規則與形式電荷等中占相近優勢)時,均為共振結構,實際結構結構是所有共振結構的混成體,且能量總是低於所有共振結構,稱共振。

實際結構的分子光譜不同於任一共振結構,並因為更穩定通常反應性低於任一共振結構的非共振相似物質,如苯比無共振烯類更穩定而較難發生加成反應。

具共振的物質的結構式通常將不同共振結構均繪出並在其間以 $\rightleftharpoons$ 表示共振,或以虛線表示只存在於某些共振結構的鍵結,其中環狀共軛系統的虛線可改繪成環中的圓。
\sssc{共振能(Resonance energy)/穩定能(Stabilization energy)}
指共振結構中能量最低者減去實際結構之能量。如苯的共振能是 36 kJ/mol。
\sssc{價鍵理論中的共振}
鍵級為共振結構之算數平均,如\ce{O3}鍵級為$\frac{3}{2}$、\ce{NO3-}鍵級為$\frac{5}{3}$、\ce{CO3^{2-}}鍵級為$\frac{4}{3}$、苯的碳碳鍵級為$\frac{3}{2}$、石墨鍵級為$\frac{4}{3}$。實際混成軌域數目與共振結構中混成軌域數目最多者相同。無法很好地解釋實際結構中的電子雲形狀等,並在一些情況無法很好地解釋其穩定性。
\sssc{分子軌域理論中的共振}
分子軌域理論用離域分子軌域解釋共振。離域成鍵軌域跨越每個參與共享電子對的原子,並將電子填入其中,故該等原子必然是共平面或至少接近平面的。如苯的六個碳的 2p$_z$ 軌域共同形成三個離域 π 軌域與三個離域 π* 軌域,六個電子恰填滿三個離域 π 軌域。
\sssc{共軛系統(Conjugated system)}
π 電子在多個原子之間的離域的現象,形成一個更穩定的電子雲。只要以 σ 鍵連成的鏈中,每個連續原子都有未參與混成的相互平行的 p 軌域,且其中每對相鄰的兩個原子至少有一個具有 π 電子或 p 軌域孤電子,該系統就可以被認為是共軛的。有$k$個原子與$n$個 π 電子參與的共軛系統記作 Π$_k^n$。

具有交替的單鍵與多鍵或相間的多鍵與孤電子者多具共軛,如:苯 Π$_6^6$、氮雜苯 Π$_6^6$、1,3-丁二烯 Π$_4^4$、氧雜環戊-2,4-二烯 Π$_5^6$(O的一對 l.p. 參與)、環庚-2,4,6-三烯-1-金翁離子 Π$_7^6$、乙酸根 Π$_3^4$(O$^-$的一對 l.p. 參與)、乙醯胺 Π$_3^4$(N的一對 l.p. 參與)、丙-2-烯醛 Π$_4^4$。

在具有連續 Π$_n^n$ 共軛系統的一維或二維結構中,無數個 p 軌域共同形成能量較低、基態時填滿電子的離域 π 能帶,即價帶,與能量較高、基態時沒有電子的離域 π* 能帶,即導帶,前者頂與後者底的能階差為能隙,若能隙足夠小則可導電。可以通過 p 型摻雜在 π 能帶即價帶引入電洞、通過 n 型摻雜在 π* 能帶即導帶引入電子。如:石墨的無數個未參與混成的 2p$_z$ 軌域平行重疊成離域 π 鍵共軛系統,離域 π 能帶即價帶,離域 π* 能帶即導帶,能隙為零,可導電;聚乙炔能隙較大,導電性差,摻雜後導電性上升。
\sssc{芳香性(Aromaticity)與休克爾規則(Hückel’s Rule)}
芳香性(Aromaticity)描述不飽和鍵、孤對電子或空軌域的共軛環表現出比單獨共軛所預期的更強的穩定性,如苯。

休克爾規則(Hückel’s Rule)是一個經驗法則,用來判斷環狀分子是否具有芳香性,指出,一個化合物要滿足芳香性,必須符合以下條件:
\bit
\item 環狀(Cyclic):分子必須形成封閉的環狀結構。
\item 共軛(Conjugated):環內的 π 電子必須能夠離域。
\item 平面(Planar):分子必須能夠維持平面結構,使 π 電子可以有效地離域。
\item 具有 $4n+2$ 個 π 電子($n$為非負整數)。
\eit
\ssc{超共軛(Hyperconjugation)/σ-共軛(σ-conjugation)/非鍵共振(no-bond resonance)}
指 σ 軌域向同一原子上的非鍵或反鍵軌域捐獻電子雲密度的現象,可增加穩定性,並可以有效的解釋許多化學性質。其主要效應有:
\bit
\item 縮短供電子 σ 鍵鍵長,如丙炔碳碳單鍵鍵長遠小於丙烷。
\item 增加偶極矩,如1,1,1-三氯乙烷的偶極矩遠大於三氯甲烷。
\item 負生成熱大於其鍵能總和、氫化加成反應放熱小於乙烯。
\item 增加碳正離子的穩定性,如穩定性\ce{(CH3)3C+}>\ce{(CH3)2CH+}>\ce{CH3CH2+}>\ce{CH3+}。
\eit
\sssc{負超共軛(Negative hyperconjugation)}
指空 σ* 軌域自同一原子上的成鍵或非鍵軌域吸取電子雲密度的現象,可增加穩定性,如:\ce{CFH2NH-}以共振結構表示為 FC(H)(H)N$^-$H \ce{<=>} F$^-$.C(H)(H)=NH。
\sssc{三中心四電子鍵(3-center 4-electron (3c–4e) bond)/皮門特爾–朗德爾三中心模型(Pimentel–Rundle three-center model)}
三個原子各提供一個原子軌域組合成三個分子軌域,一個成鍵軌域離域於所有三個中心,一個非鍵軌域定域於外圍兩個中心,被四個電子占據,用於解釋某些超價分子。如:
\bit
\item 三碘離子\ce{I3-}為直線形,鍵角176°。
\item 氟化氫根\ce{HF2-}為直線形,鍵角180°。
\eit

在價鍵理論中用二相鄰原子以單鍵鍵結、另一原子較該二原子多一對孤電子對、中間原子為氫時以氫鍵連接兩個部分的兩種共振結構的共振解釋。
\sssc{三中心二電子鍵(3-center 2-electron (3c–2e) bond)}
三個原子各提供一個原子軌域組合成三個分子軌域,一個成鍵軌域離域於所有三個中心,被兩個電子占據,用於解釋某些缺電子分子。如:
\bit
\item 乙硼烷\ce{B2H6}具有兩個由 B-H-B 形成的三中心二電子鍵,形成 [B]1(H)(H)-H-[B](H)(H)-H-1 的結構,B-H-B鍵角83°,3c–2e鍵處的H-B-H鍵角97°,外圍的H-B-H鍵角120°。
\item 三氫正離子\ce{H3+}為正三角形。
\eit

在價鍵理論中用三原子中兩者以單鍵鍵結的三種共振結構的共振解釋。
\ssc{價殼層電子對互斥理論(Valence Shell Electron Pair Repulsion Theory, VSEPR Theory)}
\sssc{主張}
包圍一原子的電子雲都帶負電會互斥,故組成分子或離子團時電子會盡量互相遠離,使原子的電子雲間排斥力降至最低。

相同距離斥力:l.p.–l.p.>l.p.–b.p.>b.p.–b.p.>l.p.–不成對電子>b.p.–不成對孤電子。
\sssc{單中心分子或離子團的鍵角}
\bit
\item 中心原子無孤電子且各周圍原子與鍵結方式均相同,則各鍵角均相同。如:\ce{CH4} 與 \ce{CF4} 鍵角均為 $\arccos\qty(-\frac{1}{3})\approx$109.5°。
\item 同原子上,l.p. 愈多,鍵角愈小。如:\ce{NH3} 鍵角106.8°、\ce{H2O} 鍵角104.5°。
\item 同原子上,多鍵愈多,其對面的鍵角愈小,雙鍵的效應略小於 l.p.。如:丙酮的 C-C-C 鍵角 116.0°、\ce{O3} 鍵角116.8°。
\item 同原子上,不成對孤電子愈多,鍵角愈大,效應遠大於 l.p. 與多鍵。如:\ce{NO2} 鍵角134.3°。
\item 各類電子數相同,中心原子相同,外圍原子不同,則:
\bit
\item 外圍原子半徑:外圍原子半徑愈大,外圍原子自身的非鍵結電子擁擠程度愈大,鍵結電子愈靠近中心原子,鍵角愈大。如:\ce{OCl2}鍵角110.9°、\ce{OF2} 鍵角103°。
\item 外圍原子電負度:外圍原子電負度愈大,鍵結電子愈遠離中心原子,鍵角愈小。如:\ce{H2O} 鍵角104.5°、\ce{OF2}鍵角 103°。
\eit
\item 各類電子數相同,中心原子不同,外圍原子相同,則:
\bit
\item 中心原子電負度:中心原子電負度愈大,鍵結電子愈靠近中心原子,鍵角愈大。如:\ce{H2O} 鍵角104.5°、\ce{H2S} 鍵角92.1°。
\item 中心原子半徑:中心原子半徑愈大,中心原子自身的非鍵結電子擁擠程度愈大,鍵結電子愈遠離中心原子,鍵角愈小。如:\ce{H2O} 鍵角104.5°、\ce{H2S} 鍵角92.1°。
\eit
\eit
無中心原子者不適用此原則,如\ce{P4}的鍵角均為60°。
\sssc{AXE 方法(AXE method)}
令單中心分子總價電子數$x$,其中H以7計,則中心原子 σ 鍵數$m$為:
\[m=\left\lfloor\frac{x}{8}\right\rfloor\]

中心原子孤電子所在軌域數$n$為:
\[n=\left\lceil\frac{x \mod 8}{2}\right\rceil\]

將中心原子的電子配置記作 AX$_m$E$_n$,若$m$為零則可省略X$_m$,若$m$為1則可省略$_m$,若$n$為零則可省略E$_n$,$n$為1則可省略$_n$。如:\ce{CO2}為AX$_2$、\ce{BF3}為AX$_3$、\ce{O3}為AX$_2$E、\ce{H2O}為AX$_2$E$_2$。若將單中心分子或離子團的分子式寫作AX$_m$,其中A表中心原子,X表周圍原子(所有周圍原子均視為X),電荷省略,則恰與 AXE 方法去除E$_n$相同。
\sssc{軌域種類}
中心原子價軌域總數為$m+n$,故可以之推得中心原子價軌域種類:
\begin{longtable}[c]{|c|c|}
\hline
$m+n$ & 軌域\\\hline
\endhead
2 & sp \\\hline
3 & sp$^2$ \\\hline
4 & sp$^3$ 或 dsp$^2$(內軌錯合物) \\\hline
5 & sp$^3$d 或 dsp$^3$(內軌錯合物) \\\hline
6 & sp$^3$d$^2$ 或 d$^2$sp$^3$(內軌錯合物) \\\hline
\end{longtable}\FB
\sssc{分子幾何(Molecular geometry)}
下不考慮內軌錯合物。
\begin{longtable}[c]{|p{0.06\textwidth}|p{0.08\textwidth}|p{0.17\textwidth}|p{0.24\textwidth}|p{0.25\textwidth}|}
\hline
$m+n$ & AXE & 構型 & 鍵角 & 舉例 \\\hline
\endhead
2 & AX$_2$E$_0$ & 直線形(Linear) & 180° & \ce{BeF2}, \ce{HCN}, \ce{CO2}, \ce{CS2} \\\hline
3 & AX$_3$E$_0$ & 平面三角形(Trigonal planar) & 120° & \ce{BF3}, \ce{NO3-}, \ce{SO3}, \ce{CO3^{2-}}, \ce{[CuCl3]^{2-}}, \ce{[HgI3]-} \\\cline{2-5}
& AX$_2$E$_1$ & 角形(Angular)/彎曲形(Bent)/V 形(V-shaped) & $\lessapprox$120° & \ce{O3}, \ce{SO2}, \ce{NO2-} \\\hline
4 & AX$_4$E$_0$ & 四面體(Tetrahedral) & $\arccos\qty(-\frac{1}{3})$$\approx$109.5° & \ce{CH4}, \ce{SO4^{2-}}, \ce{NH4+}, \ce{SiF4}, \ce{BF4-}, \ce{PO4^{3-}}, \ce{ClO4-} \\\cline{2-5}
& AX$_3$E$_1$ & 三角錐(Trigonal pyramidal) & $\approx$107° & \ce{NH3}, \ce{SO3^{2-}}, \ce{PCl3}, \ce{ClO3-}, \ce{H3O+} \\\cline{2-5}
& AX$_2$E$_2$ & 角形 & $\approx$104.5° & \ce{H2O}, \ce{H2S}, \ce{OF2}, \ce{ClO2-} \\\hline
5 & AX$_5$E$_0$ & 三角雙錐(Trigonal bipyramidal) & 赤道三鍵鍵角120°、軸位二鍵鍵角180°、赤道與軸位鍵角90° & \ce{PF5}, \ce{PCl5}, \ce{PF3Cl2}, \ce{AsF5}, \ce{SbCl5} \\\cline{2-5}
& AX$_4$E$_1$ & 翹翹板(Seesaw)/變形四面體/雙蝶形(Disphenoidal) & 赤道二鍵鍵角$\lessapprox$120°、軸位二鍵鍵角$\lessapprox$180°、赤道與軸位鍵角$\lessapprox$90° & \ce{SF4}, \ce{SeF4}, \ce{IF4+} \\\cline{2-5}
& AX$_3$E$_2$ & T形(T-shaped) & $\lessapprox$90° & \ce{BrF3}, \ce{ClF3} \\\cline{2-5}
& AX$_2$E$_3$ & 直線形 & 180° & \ce{XeF2}, \ce{I3-} \\\hline
6 & AX$_6$E$_0$ & 八面體(Octahedral) & 90° & \ce{SF6}, \ce{SiF6^{2-}} \\\cline{2-5}
& AX$_5$E$_1$ & 四角錐(Square pyramidal) & $\lessapprox$90° & \ce{BrF5}, \ce{IF5}, \ce{TeF5-} \\\cline{2-5}
& AX$_4$E$_2$ & 平面方形(Square planar) & 90° & \ce{XeF4}, \ce{ICl4-} \\\hline
7 & AX$_7$E$_0$ & 五角雙錐(Pentagonal bipyramidal) & 赤道五鍵鍵角108°、軸位二鍵鍵角180°、赤道與軸位鍵角90° & \ce{IF7} \\\cline{2-5}
& AX$_6$E$_1$ & 五角錐(Pentagonal pyramidal) & 赤道五鍵鍵角$\lessapprox$108°、赤道與軸位鍵角$\lessapprox$90° & \ce{XeOF5-}, \ce{IOF5^{2-}} \\\cline{2-5}
& AX$_5$E$_2$ & 平面五邊形(Pentagonal planar) & 108° & \ce{XeF5-}, \ce{IF5^{2-}} \\\hline
8 & AX$_8$E$_0$ & 四角反稜柱(Square antiprismatic) & & \ce{XeF8^{2-}}, \ce{ReF8-} \\\hline
\end{longtable}\FloatBarrier
\sssc{幾何異構}
相同化學式的化合物因配體在空間中排列不同而形成的異構物。幾何異構可能改變化合物的氫鍵、極性、熔沸點、溶解度、錯合物的顏色、磁性等。

下 M 為中心原子,A 起各不同大寫字母指不同配體,<A,B> 指 A, B 相對。

直線與四面體形沒有幾何異構。

\tb{平面方形的幾何異構}:
\bit
\item MA$_4$:M <A,A> <A,A> 一種。
\item MA$_3$B:M <A,A> <A,B> 一種。
\item MA$_2$B$_2$:順式 M <A,B> <A,B> 與反式 M <A,A> <B,B> 二種。
\item MA$_2$BC:順式 M <A,B> <A,C> 與反式 M <A,A> <B,C> 二種。
\item MABCD:M <A,B> <C,D>、M <A,C> <B,D>、M <A,D> <B,C> 三種。
\eit

\tb{八面體的幾何異構}:
\begin{itemize}
\item MA$_6$:M <A,A> <A,A> <A,A> 一種。
\item MA$_5$B:M <A,A> <A,A> <A,B> 一種。
\item MA$_4$B$_2$:順式 M <A,A> <A,B> <A,B> 與反式 M <A,A> <A,A> <B,B> 二種。
\item MA$_4$BC:順式 M <A,A> <A,B> <A,C> 與反式 M <A,A> <A,A> <B,C> 二種。
\item MA$_3$B$_3$:順式 M <A,B> <A,B> <A,B> 與反式 M <A,A> <A,B> <B,B> 二種。
\item MA$_3$B$_2$C:M <A,A> <A,B> <B,C>、M <A,B> <A,B> <A,C>、M <A,A> <B,B> <A,C> 三種。
\item MA$_3$BCD:M <A,A> <A,B> <C,D>、M <A,A> <A,C> <B,D>、M <A,A> <A,D> <B,C>、M <A,B> <A,C> <A,D> 四種。
\item MA$_2$B$_2$C$_2$:M <A,A> <B,B> <C,C>、M <A,A> <B,C> <B,C>、M <A,B> <A,B> <C,C>、M <A,B> <A,C> <B,C>、M <A,C> <A,C> <B,B> 五種。
\item MA$_2$B$_2$CD:M <A,A> <B,B> <C,D>、M <A,A> <B,C> <B,D>、M <A,B> <A,B> <C,D>、M <A,B> <A,C> <B,D>、M <A,B> <A,D> <B,C>、M <A,C> <A,D> <B,B> 六種。
\item MA$_2$BCDE:九種。
\item MABCDEF:十五種。
\end{itemize}
\sssc{多中心原子單鍵旋轉}
多中心原子,若有單鍵使得旋轉之可以不在一個平面上,則視為三維結構而非平面結構。


\sct{(配位(Coordination))錯合物/絡合物(Complex)/配位化合物(Coordination compound)/配合物}
\ssc{簡史}
\bit
\item 1704年,Johann Jacob Diesbach 發現普魯士藍/六氰鐵(II)酸鐵(III),是首個發現的錯合物。
\item 1869年,Christian Wilhelm Blomstrand 和約根深(Sophus Mads Jørgensen)提出錯合物的鏈式理論(Chain theory),後被證明為誤。
\item 1893年,維爾納(Alfred Werner)提出錯合物的維爾納配位理論(Werner's coordination theory),主張配體鍵結在配位中心上,配位中心具有一級價和二級價,並通過測量溶液的凝固點下降量、滲透壓、導電度等推得錯鹽結構。
\eit
\ssc{(配位(Coordination))錯合物/絡合物(Complex)/配位化合物(Coordination compound)/配合物}
\sssc{(配位(Coordination))錯合物/絡合物(Complex)/配位化合物(Coordination compound)/配合物}
一個中心原子或離子,稱配位中心(coordination center),以及周圍配置的多個結合分子或離子,稱配位基/配體/配基/配位子/牙基(ligand)/錯合劑/絡合劑(complexing agent),之間形成配位共價鍵(coordinate covalent bond),其中配體提供電子對作為路易斯鹼(Lewis base)/電子對提供者(electron-pair donator),配位中心提供空軌域作為路易斯酸(Lewis acid)/電子對接受者(electron-pair acceptor),所相結合而形成的多原子分子或離子。非錯合物之間相結合形成錯合物的反應稱錯合反應(Complexation reaction)。研究錯合物的化學稱配位化學(Coordination Chemistry)。
\sssc{配位中心(Coordination center)/中心原子(Central atom)/中心金屬(Central metal)}
M,通常是金屬或其陽離子,過渡金屬尤常見,如六氰鐵(III)離子 \ce{[Fe(CN)6]^{3-}} 中的 \ce{Fe^{3+}}、四羰基鎳 \ce{[Ni(CO)4]} 中的鎳。

配位中心除了與配體的配位鍵也可能有金屬間共價鍵,常為多重鍵,如乙酸鉻 \ce{[Cr2(CH3COO)4(H2O)2]} 的兩鉻間為四鍵(Quadruple bond)。

多數過渡金屬離子在水溶液中以以水為配體的錯離子存在,故多具顏色,其中又以六配位數最常見。
\sssc{配位基/配體/配基/配位子/牙基(ligand)/錯合劑/絡合劑(complexing agent)}
L,將孤電子對或 π 電子對配位給配位中心,通常是以碳、氮、磷、氧、硫或鹵素原子提供孤電子對的非金屬陰離子或極性分子(亦有陽離子與非極性者),或以 π 電子參與配位共價鍵的非金屬分子,如六氰鐵(III)離子 \ce{[Fe(CN)6]^{3-}} 中的氰離子 \ce{CN-}、二(η$^5$-環戊二烯陰離子)鐵(II)/二茂鐵(Ferrocene, Fc) \ce{Fe(C5H5)2} 中的環戊二烯陰離子 \ce{C5H5-}。
\sssc{錯鹽(complex salt)}
帶淨電荷的錯合物稱錯離子(complex ion)/配位離子,可再與其他離子以離子鍵鍵結形成錯鹽(complex salt),如氫氧化四氨二水銅(II) \ce{[Cu(NH3)4(H2O)2](OH)2} 中 \ce{Cu^{2+}} 與氨和水間為配位共價鍵、與氫氧根離子間為離子鍵。
\sssc{示性式}
在示性式中,錯合物(包含任何以共價鍵連接於配位中心或配體上的原/離子團)寫在 [ ] 內,若錯離子單獨存在則電荷數標於右上,若錯離子與其他離子(團)形成離子晶體,則帶正電荷的(錯)離子在前、帶負電荷的(錯)離子在後。如六氰鐵(III)離子 \ce{[Fe(CN)6]^{3-}}、六氰鐵(III)酸鉀 \ce{K3[Fe(CN)6]}、六羰基鎢\ce{[W(CO)6]}。

實驗式中省略 [ ] 並常省略配體水或將配體水與水合水同計、常省略溶於水之錯離子的所有配體。
\ssc{牙/齒(數)(Denticity)}
κ,一個配體的牙數指其配位給一個配位中心的電子對數,對於橋接配體對每個配位中心各自計算牙數。一個配體在不同錯合物中可能有不同牙數。
\sssc{單牙(Monodentate or unidentate)配體}
如:鹵素離子、氰離子(碳負離子鍵結者較常見,亦有氮鍵結者)、一氧化碳(碳鍵結,命名中稱羰基)、氨(氮鍵結)、胺基負離子(Azanide)\ce{NH2-}(氮負離子鍵結)、水(氧鍵結)、烯(碳碳 π 鍵鍵結)、卡賓(Carbene)(碳鍵結)、亞硝酸根離子(氮鍵結者較常見,亦有氧負離子鍵結者)、亞硝金翁離子 N$\equiv$O$^+$(氮鍵結)、聯胺金翁離子 \ce{NH2NH3+}(形式電荷為零的氮鍵結)、硫代硫酸根離子(不與氧鍵結的硫鍵結者較常見,亦有氧負離子鍵結者)、硫氰酸根離子(硫負離子鍵結或氮負離子鍵結)\ce{SCN-}、吡啶(py)(氮鍵結)、環戊二烯陰離子(Cyclopentadienyl anions, Cp, Cp$^-$)/茂)\ce{C5H5-}(五碳共軛 π 體系鍵結)、環辛四烯陰離子(Cyclooctatetraenide anion, Cot, Cot$^{2-}$)\ce{C8H8-}(八碳共軛 π 體系鍵結)、氧氣(氧鍵結)。
\sssc{多牙(Polydentate or multidentate)配體}
指非單牙的配體。
\sssc{雙牙(Bidentate or didentate)配體}
如:乙二胺(en)\ce{(NH2CH2)2}(氮鍵結)、草酸根離子(ox$^{2-}$ or ox)(氧負離子鍵結)、碳酸根離子(氧負離子鍵結)、1,2-二(二甲基砷基)苯(diars)\ce{C6H4(As(CH3)2)2}(砷鍵結)、2,2'-聯吡啶(bipy or bpy)(氮鍵結)、1,10-二氮雜菲(phen)(氮鍵結)、4,7-二苯基-1,10-二氮雜菲(氮鍵結)。
\sssc{三牙(Tridentate)配體}
如:2,2';6',2"-三聯吡啶(terpy or tpy)(氮鍵結)。
\sssc{四牙(Quadridentate or tetradentate)配體}
如:N$^1$,N$^1$'-(乙烷-1,2-二基)二(乙烷-1,2-二胺)(trien)(氮鍵結)、Chlorin(氮鍵結)、Porphyrin(氮鍵結)。
\sssc{六牙(Sexidentate or hexadentate)配體}
如:2,2′,2′′,2′′′-(乙烷-1,2-二胺基-N$^1$,N$^1$',N$^2$,N$^2$'-四基)四乙酸根離子(EDTA$^{4-}$ or EDTA)\ce{[CH2N(CH2COO)2]2^{2-}}(氮鍵結與氧負離子鍵結)。
\ssc{螯合物/鉗合物(Chelate complex)}
\sssc{螯合物/鉗合物(Chelate complex)}
一個或多個多齒配體提供多對電子與配位中心形成配位共價鍵的錯合物,該等配體稱螯合劑/鉗合劑(chelants or chelators)/螯合配體(chelating agents or chelating ligands),此鍵結方式稱螯合/鉗合(Chelation),常描述成多齒配體類似於螃蟹的螯夾住配位中心。如三草酸鐵(III)離子 \ce{[Fe(C2O4)3]^{3-}}、EDTA 鐵(III)離子 \ce{[Fe(EDTA)]-}、EDTA 鈣離子 \ce{[Ca(EDTA)]^{2-}}、葉綠素、血基質。
\sssc{應用}
\bit
\item 作為分析金屬離子濃度的試劑。
\item 清潔重金屬離子,如草酸除鏽、EDTA 清潔。
\item 螯合療法,利用螯合劑與重金屬離子形成較安全的螯合物再經腎臟排出體外,清除體內的重金屬離子以解毒,如 EDTA 作為重金屬中毒的解毒劑。
\item 具磁性的金屬螯合物可用作核磁共振造影的試劑,可用於診斷肝纖維化和癌症等。
\eit
\ssc{配體的連配原子數/哈普托數(Hapticity)}
η,指一個配體中有連續的多個原子共同參與配位時,其參與配位的原子數。如二(η$^5$-環戊二烯陰離子)鐵(II)/二茂鐵(Ferrocene, Fc) \ce{Fe(C5H5)2} 中環戊二烯陰離子的哈普托數為五、二(η$^8$-環辛四烯陰離子)鈾(IV)(Uranocene)中環辛四烯陰離子的哈普托數為八、蔡斯鹽/蔡氏鹽(Zeise's salt)/一水合[η$^2$-三氯乙烯鉑(II)酸鉀] \ce{K[PtCl3(C2H4)]$\cdot$H2O} 中三氯乙烯的哈普托數為二。
\ssc{橋接配體/橋連配體(Bridging ligand)與終端配體(Terminal ligand)}
橋接配體指連接多個配位中心的配體;終端配體指僅連接一個配位中心的配體。

有多個孤電子對或 π 電子對的配體都可以作為橋接配體。常見橋接配體有 \ce{CN-} 一碳鍵結一氮鍵結(如六氰鐵(II)酸鐵(III))、\ce{SCN-} 一硫鍵結一氮鍵結、\ce{Cl-}、\ce{Br-}、\ce{I-}、\ce{OH-} 均氧鍵結、\ce{O2^{2-}}(如二鉻酸根)、\ce{S^{2-}}、\ce{CO} 均碳鍵結(如九羰基二鐵)、\ce{NH2-} 均氮鍵結、\ce{N3-} 兩端氮各至多可鍵結二個配位中心。
\ssc{錯合物的系統命名}
\bit
\item 配體數目加配體名稱加配位中心名稱或配體數目加配體名稱加化加配位中心名稱,如四羰基鎳或四羰基化鎳 \ce{[Ni(CO)4]};
\item 有電荷者最後再加上離子,如六氰鐵(II)離子 \ce{[Fe(CN)6]^{4-}};
\item 陽離子為錯離子之錯鹽者稱陰離子名稱化錯離子名稱(均不加離子),如氫氧化四氨二水銅(II) \ce{[Cu(NH3)4(H2O)2](OH)2};
\item 陰離子為錯離子者稱錯離子名稱酸陽離子名稱(均不加離子),如六氰鐵(III)酸鉀 \ce{K3[Fe(CN)6]};
\item 配體有連續的多個原子共同參與配位時配體名稱前綴哈普托數(Hapticity)於 η 右上標再加 -,如 二(η$^5$-環戊二烯陰離子)鐵(II)(俗稱二茂鐵);
\item 橋接配體(Bridging ligand)名稱前綴橋接配體所連接的原子個數於 μ 右下標再加 -,2不標,有多個同情況橋接配體時配體數目後加 - 再綴前述、有多個不同情況橋接配體時依序寫出,若非首字之 μ 前非 - ,則於 μ 前加 -、有非橋接配體時與各自配位中心一同依前規則寫畢後再前綴橋接配體,如 三-μ-羰基-二(三羰基鐵) \ce{Fe2(CO)9}(俗稱九羰基二鐵)。
\item 單牙配體後綴 -參與配位的原子名,含多個相同原子時右上標位置;多牙(Polydentate)配體名稱後綴 -κ ,其右上標牙數(Denticity),再後綴 -參與配位的原子名,含多個相同原子時右上標位置。若配體之牙數與參與配位的原子眾所周知則常省略此。
\eit
\ssc{共價鍵分類方法(Covalent bond classification (CBC) method)/LXZ 符號(LXZ notation)}
\sssc{X 型配體(X-type ligand)}
當把配體視為中性時,其與中心原子間為各提供一個電子的典型共價鍵。如:鹵素離子或擬鹵素離子配體。
\sssc{L 型配體(L-type ligand)}
當把配體視為中性時,其與中心原子間為配體提供電子對、配位中心提供空軌域的配位共價鍵。如:一氧化碳、水、氨與其衍生物、磷化氫與其衍生物、以 π 鍵提供電子對者配體。
\sssc{X 型配體(X-type ligand)}
當把配體視為中性時,其與中心原子間為配位中心提供電子對、配體提供空軌域的配位共價鍵。不常見。
\ssc{配位數(Coordination number, C.N., or ligancy)}
\sssc{配位數(Coordination number, C.N., or ligancy)}
一中心原子的配位數為各個方向上最近鄰的與之有共價鍵、離子鍵、金屬鍵或氫鍵的原子數量總和,但對於配位中心的一個配體中有連續的多個原子共同參與配位時,通常以該配體與該配位中心的配位共價鍵數為該配體貢獻的配位數,即配位數為各個方向上最近鄰的原子數量總和減去(每一個具有非一哈普托數的配體的(哈普托數-1)的和)。配位中心的配位數通常是 2 到 12,6 配位最常見,2 配位和 4 配位次常見。
\sssc{維爾納(Alfred Werner)的一級價/主價(primary valence)/第一價(first valence)與二級價/副價(secondary valence)/第二價(second valence)}
提出錯合物配位中心有一級價,即氧化態,與二級價,即配位數。
\sssc{常見配位中心的常見配位數}
通常金屬陽離子配位中心的配位數是氧化態的兩倍,例外眾多。
\begin{longtable}[c]{|c|c|c|}
\hline
配位中心 & 電子組態 & 常見配位數 \\\hline
\endhead
\ce{Al^{3+}} & p$^6$ & 4, 6 \\\hline
\ce{Cr^{3+}} & d$^4$ & 6 \\\hline
\ce{Mn^{2+}} & d$^5$ & 4, 6 \\\hline
\ce{Fe^{3+}} & d$^5$ & 4, 6 \\\hline
\ce{Fe^{2+}} & d$^6$ & 6 \\\hline
\ce{Co^{3+}} & d$^6$ & 6 \\\hline
\ce{Pd^{4+}} & d$^6$ & 6 \\\hline
\ce{Pt^{4+}} & d$^6$ & 6 \\\hline
\ce{Co^{2+}} & d$7$ & 4, 6 \\\hline
\ce{Au^{3+}} & d$^7$ & 4, 6 \\\hline
\ce{Ni^{2+}} & d$^8$ & 4 \\\hline
\ce{Pd^{2+}} & d$^8$ & 4 \\\hline
\ce{Pt^{2+}} & d$^8$ & 4 \\\hline
\ce{Cu^{2+}} & d$^9$ & 4, 6 \\\hline
\ce{Cu+} & d$^{10}$ & 2, 4 \\\hline
\ce{Zn^{2+}} & d$^{10}$ & 4 \\\hline
\ce{Ag+} & d$^{10}$ & 2 \\\hline
\ce{Hg^{2+}} & d$^{10}$ & 4 \\\hline
\ce{Au+} & d$^{10}$ & 2, 4 \\\hline
\end{longtable}\FB
\ssc{配位域(Coordination sphere)}
\sssc{第一配位域/層(First coordination sphere)/內界}
指配位中心的配體。
\sssc{第二配位域/層(Second coordination sphere)/外界}
指以各種方式(通常是氫鍵或離子鍵)附著在第一個配位域上的分子或離子。如:三水合六氰鐵(II)酸鉀 \ce{K4[Fe(CN6)]$\cdot$3H2O} 中,配位中心 \ce{Fe^{3+}} 的第一配位域有六個氰離子、第二配位域有六個鉀離子和一些結晶水。
\ssc{總表}
\begin{longtable}[c]{|p{0.04\textwidth}|p{0.07\textwidth}|p{0.1\textwidth}|p{0.2\textwidth}|p{0.06\textwidth}|p{0.08\textwidth}|p{0.25\textwidth}|}
\hline
配位數 & 內或外軌/低或高自旋 & 構型 & d 軌域能量分裂 & 混成軌域 & 常見配位中心電子組態 & 舉例 \\\hline\endhead
2 & 外軌/高自旋 & 直線形 & d$_{x^2-y^2}$ = d$_{xy}$ < d$_{yz}$ = d$_{xz}$ < d$_{z^2}$ & sp & d$^{10}$ & \ce{[Ag(NH3)2]+}, \ce{[Ag(CN)2]-}, \ce{[Ag(S2O3)2]^{3-}}, \ce{[Cu(NH3)2]+} \\\hline
3 & 外軌/高自旋 & 平面三角形 & d$_{xz}$ = d$_{yz}$ < d$_{z^2}$ < d$_{xy}$ = d$_{x^2-y^2}$ & sp$^2$ & d$^{10}$ & \ce{[CuCl3]^{2-}}, \ce{[HgI3]-} \\\hline
4 & 外軌/高自旋 & 三角錐 & $e$ (d$_{x^2-y^2}$ = d$_{z^2}$) < $t_2$ (d$_{xy}$ = d$_{xz}$ = d$_{yz}$) & sp$^3$ & d$^5$, d$^7$, d$^8$, d$^{10}$ & \ce{[Zn(NH3)4]^{2+}}, \ce{[Zn(CN)4]^{2-}}, \ce{[Zn(OH)4]^{2-}}, \ce{[Cd(NH3)4]^{2+}}, \ce{[HgI4]^{2-}}, \ce{[CoCl4]^{2-}}, \ce{[NiCl4]^{2-}}, \ce{[Ni(CO)4]}, \ce{[FeCl4]-}, \ce{[AlBr4]-}, \ce{[BeCl4]^{2-}}, \ce{[BeF4]^{2-}} \\\hline
4 & 內軌/低自旋 & 平面方形(Square planar) & d$_{xz}$ = d$_{yz}$ < d$_{xy}$ = d$_{x^2-y^2}$ < d$_{z^2}$ & dsp$^2$ & d$^8$, d$^9$ & \ce{[Pt(NH3)4]^{2+}}, \ce{[Pt(NH3)2Cl2]}, \ce{[Cu(NH3)4]^{2+}}, \ce{[Ni(CN)4]^{2-}}, \ce{[AuCl4]-}, \ce{[PtCl4]^{2-}}, \ce{[PdCl4]^{2-}} \\\hline
5 & 外軌/高自旋 & 三角雙錐 & d$_{xz}$ = d$_{yz}$ < d$_{xy}$ = d$_{x^2-y^2}$ < d$_{z^2}$ & sp$^3$d$^2$ & & \\\hline
5 & 外軌/高自旋 & 方錐(Square pyramidal) & d$_{xz}$ = d$_{yz}$ < d$_{xy}$ = d$_{x^2-y^2}$ < d$_{z^2}$ & sp$^3$d$^2$ & & \ce{[ClF5]}, \ce{[MnCl5]^{2-}}, 血紅素-\ce{Fe^{2+}}五配位數錯合物 \\\hline
6 & 外軌/高自旋 & 八面體 & $t_{2g}$ (d$_{xy}$ = d$_{xz}$ = d$_{yz}$) < $e_g$ (d$_{z^2}$ = d$_{x^2-y^2}$) & sp$^3$d$^2$ & d$^3$, d$^4$, d$^5$, d$^6$, d$^7$, d$^9$ & \ce{[Cr(H2O)6]^{3+}}, \ce{[CoF6]^{3-}}, \ce{[FeCl6]^{3+}}, \ce{[Co(H2O)6]^{2+}}, \ce{[FeF6]^{3-}}, \ce{[Cu(H2O)6]^{2+}}, \ce{[Ca(EDTA)]^{2-}}, \ce{[AlF6]^{3-}}, \ce{[SiF6]^{2-}}, \ce{[SF6]} \\\hline
6 & 內軌/低自旋 & 八面體 & $t_{2g}$ (d$_{xy}$ = d$_{xz}$ = d$_{yz}$) < $e_g$ (d$_{z^2}$ = d$_{x^2-y^2}$) & sp$^3$d$^2$ & d$^5$, d$^6$ & \ce{[Fe(CN)6]^{4-}}, \ce{[Fe(CN)6]^{3-}}, \ce{[Co(NH3)6]^{3+}}, \ce{[Pt(NH3)4Cl2]^{2+}}, \ce{[PtCl6]^{2-}}, \ce{[Fe(C2O4)3]^{3-}}, 血紅素-小分子-\ce{Fe^{2+}}六配位數錯合物 \\\hline
6 & 外軌/高自旋 & 三稜柱(Trigonal prismatic) & $t_{2g}$ (d$_{xy}$ = d$_{xz}$ = d$_{yz}$) < $e_g$ (d$_{z^2}$ = d$_{x^2-y^2}$) & sp$^3$d$^2$ & & \ce{[W(CH3)6]}, \ce{[Mo(CH3)6]}, \ce{[Ta(CH3)6]-}, \ce{[Zr(CH3)6]^{2-}} \\\hline
7 & 外軌/高自旋 & 五角雙錐 & d$_{xz}$ = d$_{yz}$ < d$_{xy}$ = d$_{x^2-y^2}$ < d$_{z^2}$ & sp$^3$d$^3$ & & \ce{[ZrF7]^{3-}}, \ce{[HfF7]^{3-}}, \ce{[Cr(O2)2(NH3)3)]} \\\hline
8 & 外軌/高自旋 & 四角反稜柱 & d$_{z^2}$ < d$_{xy}$ = d$_{x^2-y^2}$ < d$_{xz}$ = d$_{yz}$ & sp$^3$d$^4$ & & \ce{[XeF8]^{2-}}, \ce{[ReF8]-} \\\hline
\end{longtable}\FB
\ssc{錯鹽性質}
\sssc{總論}
錯鹽的陰陽離子間以離子鍵鍵合,錯離子或錯分子內則以共價鍵鍵合,故錯鹽的熔點與溶解度通常遠高於錯分子,又離子鍵溶於水時易解離,而錯離子在水中多數不會解離,因此可以通過測量溶液的凝固點下降量、滲透壓、導電度等得到凡特荷夫因子以分辨錯鹽與複鹽。
\sssc{穩定常數(Stability constant)/形成常數(Formation constant)/結合常數(Binding constant)}
$n$配位數配位中心$M$與配體$L$的第$i$穩定常數為:
\[\beta=\frac{[\ce{[M(H2O)$_{n-i}$L$_i$]}]}{[\ce{[M(H2O)$_{n-i+1}$L$_{i-1}$]}][\ce{L}]}\]
\sssc{氯化鉻與鈷的錯合物為例}
M 為鉻或鈷。
\begin{longtable}[c]{|p{0.15\tw}|p{0.2\tw}|p{0.23\tw}|p{0.12\tw}|p{0.1\tw}|}
\hline
化學式 & 名稱 & 固體顏色 & 固體熔點 & 固體分類 \\\cline{2-5}
& 水中解離 & 水溶液加入過量硝酸銀後沉澱氯化銀莫耳數 & 水溶液凡特荷夫因子 & 溶液導電度 \\\hline\endhead
\ce{MCl3} & 氯化M(III) & 綠 & 高 & 離子 \\\cline{2-5}
& \ce{M^{3+} + 3Cl-} & 3 & 4 & 大 \\\hline
\ce{[M(NH3)3Cl3]} & 三氨三氯M(III) & 順式:紫、反式:綠 & 低 & 分子 \\\cline{2-5}
& 否 & 0 & 1 & 無 \\\hline
\ce{[M(NH3)4Cl2]Cl} & 氯化四氨二氯M(III) & 順式:紫、反式:綠 & 高 & 離子 \\\cline{2-5}
& \ce{[M(NH3)4Cl2]+ + Cl-} & 1 & 2 & 小 \\\hline
\ce{[M(NH3)5Cl]Cl2} & 氯化五氨一氯M(III) & 紫紅 & 高 & 離子 \\\cline{2-5}
& \ce{[M(NH3)5Cl]^{2+} + 2Cl-} & 2 & 3 & 中 \\\hline
\ce{[M(NH3)6]Cl3} & 氯化六氨M(III) & 橙黃 & 高 & 離子 \\\cline{2-5}
& \ce{[M(NH3)6]^{3+} + 3Cl-} & 3 & 4 & 大 \\\hline
\end{longtable}\FB
\ssc{價鍵理論(Valence bond theory, VB theory)}
\sssc{主張}
配位中心形成全空的混成軌域,接受配體的孤對電子形成配位共價鍵,分子幾何遵循 VSEPR 理論。對於4以上的配位數,參與混成的軌域與配體電負性有關,分成內軌形與外軌形。
\sssc{內軌形錯合物(Inner orbital complex)}
配體的電負性較小(即強場配體(strong-field ligand)),將電子強行向內推,配位中心以 $(n-1)$d 和 $n$s、$n$p 進行混成,如 dsp$^2$ 和 d$^2$sp$^3$。
\sssc{外軌形錯合物(Outer orbital complex)}
配體的電負性較大(即弱場配體(weak-field ligand)),配位中心以同一殼層的軌域進行混成,如 sp, sp$^2$, sp$^3$, sp$^3$d 和 sp$^3$d$^2$。
\ssc{晶體場理論(Crystal field theory, CFT)}
\sssc{主張}
以靜電作用解釋配位中心和配體的作用,類似於離子晶體中正負離子間的作用,當配體接近配位中心時,配位中心的 d 軌域受到配體負電荷的靜電作用而能量分裂(energy splitting)。

晶體場理論可以解釋錯合物的磁性、穩定性、顏色、水合能、離子半徑等價鍵理論所無法解釋的性質。
\sssc{d 軌域能量分裂(energy splitting)}
當配體接近金屬離子時,來自配體的電子將更靠近一些 d 軌域並更遠離一些 d 軌域,d 軌域的電子雲和配體的電子雲同性電相斥,因此愈靠近配體的 d 軌域能量愈高,從而導致 d 軌域能量分裂,損失簡併性(degeneracy)。
\sssc{晶體場分裂能(Crystal field splitting energy)}
$\Delta$,d 軌域分裂後,兩組新的簡併能階的能量差。
\sssc{晶體場穩定能(Crystal field stabilization energy, CFSE)}
分裂前的 d 電子總能量與分裂後(錯合物中)的 d 電子總能量差。
\sssc{電子成對能(Electron pairing energy)}
$E_p$,一個軌域已有一個電子時,另一個與之自旋相反的電子填入之所需吸收的能量。
\sssc{強場配體(Strong-field ligand)與低自旋錯合物(Low-spin complex)}
晶體場分裂能大於電子成對能,使得配位共價鍵形成後,配位中心部分價電子不再服從洪德定則,而是先成對地填入分裂後較低能量的軌域,此時的配體稱強場配體,並因為低能量分裂軌域中電子對自旋相反,使得分子的總自旋較低,稱低自旋錯合物。低自旋錯合物即價鍵理論的內軌形錯合物。
\sssc{弱場配體(Weak-field ligand)與高自旋錯合物(High-spin complex)}
晶體場分裂能小於電子成對能,使得配位共價鍵形成後,配位中心的全部價電子仍然服從洪德定則,此時的配體稱弱場配體,並因為電子優先填入空軌域而非半滿軌域,使得分子的總自旋較高,稱高自旋錯合物。高自旋錯合物即價鍵理論的外軌形錯合物。高自旋錯合物常具有鐵磁性或亞鐵磁性、高自旋錯合物的配位中心常具有較大的半徑。
\sssc{晶體場分裂能比較經驗法則}
\bit
\item 同配位中心同配位數不同配體,各配體的晶體場分裂能:\ce{I-} < \ce{Br-} < \ce{S^{2-}} < \ce{SCN-}(硫鍵結) < \ce{Cl-} < \ce{NO3-} < \ce{N3-} < \ce{F-} < \ce{OH-} < \ce{C2O4^{2-}} < \ce{H2O} < \ce{NCS-}(氮鍵結) < \ce{CH3CN} < py < \ce{NH3} < en < bipy < phen < \ce{NO2-} < 三苯基磷 \ce{P(C6H5)3} < \ce{CN-} < \ce{CO}。
\item 同配位中心同配體不同配位數,配位數愈小,晶體場分裂能通常愈小。
\item 同配位數同配體不同配位中心,配位中心的電負性愈高,對配體的吸引力就愈大,晶體場分裂能也愈大。晶體場分裂能通常與配位中心的還原電位、電荷數和主量子數正相關。
\item 四配位數過渡元素配位中心:\ce{Cu^{2+}}, \ce{Pd^{2+}}, \ce{Pt^{2+}}均為低自旋;\ce{Ni^{2+}}晶體場分裂能在\ce{CN-}以上者為低自旋、小於\ce{CN-}者為高自旋;其餘均為高自旋。
\item 六配位數過渡元素配位中心:d$^3$、d$^4$、d$^7$、d$^9$ 均為高自旋;\ce{Pd^{4+}}, \ce{Pt^{4+}}均為低自旋;\ce{Co^{3+}}, \ce{Fe^{3+}}晶體場分裂能在\ce{NH3}以上者為低自旋、小於\ce{NH3}者為高自旋;其餘晶體場分裂能在\ce{CN-}以上者為低自旋、小於\ce{CN-}者為高自旋。
\eit
\sssc{Jahn–Teller 效應(Jahn–Teller effect, JT effect, JTE)}
當簡併能階的軌域上填充有不同個數的電子時,使一些軌域的能階提升、一些軌域的能階降低,消除了簡併性,使錯合物幾何發生畸變,影響錯合物的性質。配位中心若 d 電子分布全滿、半滿或全空時,電子雲是球形對稱的,錯合物幾何不會發生畸變,形成正常的構型;若 d 電子分布不是球形對稱的,則電場的作用不均衡,使錯合物幾何發生畸變,如\ce{Cu^{2+}}的多數六配位數錯合物。

對於 $e_g$ 軌域距全滿或半滿少一個電子的配位中心,所形成的八面體錯合物都存在顯著的 Jahn–Teller 效應與幾何畸變,具有這種結構的包含 d$^4$ 高自旋、d$^7$ 低自旋、d$^9$,如:$\qty(t_{2g}^{\pht{2g}6})\qty(d_{x^2-y^2})^2\qty(d_{z^2})^1$是壓扁的八面體,$\qty(t_{2g}^{\pht{2g}6})\qty(d_{x^2-y^2})^1\qty(d_{z^2})^2$是拉長的八面體。

對於 $t_{2g}$ 軌域距全滿或半滿少一個電子的配位中心,所形成的八面體錯合物 Jahn–Teller 效應與幾何畸變均不顯著,因為 $t_{2g}$ 電子恰與配體方向錯開,與配體間作用相對較小。

對於三角錐構型的錯合物,也可能發生 Jahn–Teller 效應,但因為 d 軌域分裂能較小,形變也極微弱。
\ssc{配位場理論(Ligand field theory, LFT)}
\sssc{主張}
依晶體場理論處理 d 軌域能量分裂後,再依分子軌域理論處理配位中心和配體成鍵的分子軌域電子填充。可以比晶體場理論更好的解釋一些錯合物的穩定性等性質。
\sssc{錯合物的穩定性}
\ben
\item 螯合物一般較非螯合物穩定:二乙二胺銅(II)離子 \ce{[Cu(en)2]^{2+}}>四氨銅(II)離子 \ce{[Cu(NH4)4]^{2+}}。
\item 同配位中心,內軌/低自旋錯合物一般較外軌/高自旋錯合物穩定。
\item 同配體,配位中心電負性愈小,與配體間的鍵能一般愈大,錯合物一般愈穩定,如六氯鐵(III)離子\ce{[FeCl6]^{3-}}>六氯鐵(II)離子\ce{[FeCl6]^{4-}}、六氰鐵(III)離子\ce{[Fe(CN)6]^{3-}}>六氰鐵(II)離子\ce{[Fe(CN)6]^{4-}}。
\item 同配位中心,配體的路易斯鹼性愈強,愈易提供電子對,與配位中心間的鍵能一般愈大,如氯化六氨鉻(III)\ce{[Cr(NH3)6]Cl3}>氯化五氨一氯鉻(III)\ce{[Cr(NH3)5Cl]Cl2}>氯化四氨二氯鉻(III)\ce{[Cr(NH3)4Cl2]Cl}>三氨三氯鉻(III)\ce{[Cr(NH3)3Cl3]}。
\ben
\item 通常布—洛鹼較非布—洛鹼更易提供電子對。
\item 通常較強的布—洛鹼更易提供電子對,如\ce{NH3} > \ce{Cl-}。
\item 均非布—洛鹼時,則通常帶愈多負電荷(即負離子與極性共價鍵的負偶極)者愈易提供電子對,如\ce{CO} > \ce{O2}。
\een
\een
\sssc{σ-π 配鍵/反饋 π 鍵(π backbonding)}
一些錯合物中,配體將電子對配給配位中心形成 σ 配位共價鍵,配位中心將電子對配給配體形成 π 配位共價鍵,稱反饋 π 鍵,配位中心與配體間有 σ 鍵和反饋 π 鍵兩不同方向的配位共價鍵,稱 σ-π 配鍵,兩者相互促進,使得配位中心與配體間的鍵能比共價單鍵大一些。
\bit
\item \tb{參鍵配體}:提供孤對電子給配位中心的原子上具有參鍵的配體(如\ce{CO}, \ce{N2}, \ce{NO+}, \ce{CN-}),配體將孤電子對配給配位中心形成 σ 配位共價鍵,配體參鍵空的 π* 軌域與配位中心的全滿 p 或 d 軌域形成反饋 π 鍵,配體內的該參鍵的鍵能比原先小一些。多數此類錯合物其配位中心最外主殼層有 18 電子,是反磁性的。
\item \tb{不飽和烴配體/Dewar–Chatt–Duncanson model}:不飽和烴配體,碳碳全滿 π 軌域與配位中心的空軌域形成 σ 配位共價鍵,配體空的 π* 軌域與配位中心的全滿 p 或 d 軌域形成反饋 π 鍵,烯配體的兩個碳的幾何從原先的 sp$^2$ 變得更接近於 sp$^3$,炔配體的兩個碳的幾何從原先的 sp 變得更接近於 sp$^2$。
\item \tb{磷配體}:\ce{PR3}(R=H, F, 烷基, etc.)配體,磷的孤對電子與配位中心的空軌域形成 σ 配位共價鍵,磷的 σ* 軌域與配位中心的全滿 p 或 d 軌域形成反饋 π 鍵(π backbonding)。
\eit
\sssc{共軛環配體}
共軛環(如苯、環戊二烯陰離子、環辛四烯陰離子)的離域 π 鍵可以作為一個整體和配位中心形成配位化合物。


\sct{分子間作用力(Intermolecular forces)}
\ssc{極性(Polarity)}
\sssc{極性(Polarity)}
指共價鍵或分子的電荷分布不均勻而在電場中受到力矩的性質,其中帶部分正、負電荷處分別稱正偶極 δ$^+$ 與負偶極 δ$^-$,以化學中的電偶極矩衡量。
\sssc{化學中的(電)偶極矩((Electric) dipole moment)}
與電磁學中定義相同,惟將方向定義為自正電荷指向負電荷(而非電磁學中定義為自負電荷指向正電荷),即,兩點電荷電荷$+q$、$-q$,自電荷$+q$指向電荷$-q$的向量為$\mathbf{d}$,具有電偶極矩$\mu$:
\[\mu=q\mathbf{d}.\]
在化學中的常用單位為德拜(Debye, D),1 D 為 $\frac{10^{-21}}{299792458}$ C m $\approx\scinote{3.33564}{-30}$ C m。

下偶極矩均指此定義。
\sssc{鍵(偶極)矩(向量)(Bond (dipole) moment (vector))}
一共價鍵的偶極矩。
\sssc{離子性(Ionic character)}
離子性定義為一共價鍵的偶極矩測量值除以電子完全轉移至電負度較高者時之理論偶極矩,通常以百分率表示。一般離子性愈強,鍵能愈大。

有時稱 1 減去離子性為共價性(Covalent character),多用於離子鍵。

有時將離子性百分率 50\% 以上者視為離子鍵,50\% 以下者視為共價鍵;有時將兩側電負度差 <0.4 者視為非極性共價鍵(nonpolar covalent bond)、0.4-1.7 者視為極性共價鍵(polar covalent bond)、>1.7 者視為離子鍵。
\sssc{孤電子的偶極矩}
若一原子的一孤電子軌域不旋轉對稱於原子核,會產生一指向其軌域方向的偶極矩。
\sssc{分子的極性}
指所有電子雲貢獻的偶極矩之和。為零者稱非極性分子(Nonpolar molecules),否則稱極性分子(Polar molecules)。
\sssc{非極性分子(Nonpolar molecules)判斷}
非極性分子的充分條件如:
\bit
\item 所有鍵皆鍵偶極矩為零,或
\item 自兩個以上不平行方向觀察皆有二階以上旋轉對稱。
\eit

相同元素組成三原子以上分子,需考慮分子形狀方知極性與否。如\ce{O3}有極性、\ce{I3-}與\ce{S8}無極性。
\sssc{極性分子(Polar molecules)判斷}
極性分子的必要條件如:含有鍵偶極矩不為零的共價鍵或含有孤電子。
\sssc{分子極性比較經驗法則}
\bit
\item 鍵偶極矩夾角愈小,極性愈大,如鄰二氯苯>間二氯苯。
\item 孤電子失去傾向愈大/路易斯鹼性愈大,極性愈大,如\ce{NH3} > \ce{NF3}。
\eit
\ssc{凡得瓦力(Van der Waals Forces or Van der Waals' Forces)}
電偶極子之間的靜電力,與偶極矩之內積除以距離的平方正比。
\sssc{(永久)偶極-(永久)偶極力(Dipole-dipole forces)/取向力}
永久偶極與永久偶極之間(極性分子與極性分子間)的靜電力。最大。
\sssc{(永久)偶極-誘導/誘發/感應偶極力(Dipole-induced dipole forces)/誘導/誘發/感應力}
永久偶極(極性分子)接近非極性分子時,使其電子雲受力而瞬間分布不均形成誘導/誘發/感應偶極(induced dipole),兩者之間的靜電吸引力稱偶極-誘導偶極力。次大,如:\ce{H2O(g)}與\ce{O2(g)}之間的偶極-誘導偶極力小於兩分子\ce{HCl(g)}之間的偶極-偶極力。
\sssc{倫敦(/)分散力(London (/) dispersion forces)}
倫敦(Fritz Wolfgang London)提出。由於電子不停運動,即使是非極性分子也可能瞬間產生電子雲不對稱分布,稱瞬間/瞬時偶極(instantaneous dipole),此偶極再誘導鄰近的分子產生誘導偶極,這些誘導偶極又可在誘導鄰近的分子產生誘導偶極,這些非永久偶極之間的靜電吸引力稱分散力。最小。倫敦力大小與電子雲的大小(分散程度)與相對位置有關,電子雲愈大及分子間距愈小分散力愈大,分散力大小與分子量大致正相關,分子量相近則與接觸面積大致正相關,這解釋了非極性分子分子量愈大沸點愈高的現象。
\sssc{凡得瓦力強度比較經驗法則}
\bit
\item 分子量愈大,凡得瓦力愈強。
\item 極性愈大,凡得瓦力愈強。故均無氫鍵時,順式異構物常較反式異構物沸點更高、鄰位異構物常較間或對位異構物沸點更高。
\item 固體:對稱性(指旋轉對稱)愈大,晶格堆積愈緊密,凡得瓦力愈強,影響大小不一定,一般大於極性而小於分子量。故均無氫鍵時,反式異構物常較順式異構物熔點更高、對位異構物常較鄰或間位異構物熔點更高。
\item 流體:接觸面積愈大,凡得瓦力愈強,影響通常小於極性與分子量。故均無氫鍵時,少支鏈異構物常較多支鏈異構物沸點更高。
\eit
\ssc{氫鍵(Hydrogen bond)}
\sssc{氫鍵(Hydrogen bond)}
氫鍵發生在已經以共價鍵與負偶極原子X結合的正偶極氫原子與另一個負偶極原子Y之間,其中X稱供體(donor),Y稱受體(acceptor),有方向性,記作X-H$\cdots$Y,較凡得瓦力強,較化學鍵弱。
\sssc{氫鍵供體(donor)與受體(acceptor)原子}
典型的氫鍵中,X和Y是電負度強、半徑小的 \ce{F}、\ce{N} 或 \ce{O} 原子,但 \ce{C}、\ce{S}、\ce{Cl}、\ce{P} 甚至 \ce{Br} 和 \ce{I} 原子在某些情況下也能形成氫鍵,惟通常鍵能甚低,常忽略,如:\ce{Cl3CH}$\cdots$\ce{OC(CH3)2},較顯著者如:$\ce{HCN}\cdots\ce{HCN}$。
\sssc{鍵角}
受到能量更高的共價結構剛性不易彎曲的影響,氫鍵鍵角往往不是180°,如苯環上的分子內氫鍵鍵角往往遠小於180°。
\sssc{低勢壘氫鍵(Low-barrier hydrogen bond, LBHB)}
低勢壘氫鍵指 X-H-Y 3c–4e 鍵,H 與 X、Y 距離相近,有效鍵級 0.5,鍵能強於正常氫鍵而與共價鍵相仿,發生於 X-H 與 Y-H 的酸解離常數相近時,通常 X 與 Y 是同元素且具有相似周圍結構,但 X 與 Y 是同元素不一定就是低勢壘氫鍵。

不是低勢壘氫鍵的氫鍵稱正常氫鍵(Normal hydrogen bond)。

[F-H-F]$^-$ 是低勢壘氫鍵,也是常見物質中最強的氫鍵。
\sssc{氫鍵強度比較經驗法則}
\ben
\item LBHB 最大。
\item 若受體與供體相同,它們的電負度愈大愈強,如:F-H$\cdots$F>O-H$\cdots$O>N-H$\cdots$N。
\item 受體相同,供體與氫的偶極矩愈大愈強,如:F-H$\cdots$Y>O-H$\cdots$Y>N-H$\cdots$Y。
\item 供體相同,受體的電負度愈小愈強,如:X-H$\cdots$N>X-H$\cdots$O。
\een
\sssc{分子間氫鍵(Intermolecular hydrogen bond)}
指兩分子之間一者提供供體、一者提供受體形成的氫鍵。常見性質:
\bit
\item 醇、酚、羧酸、胺、醣類等均常有。
\item 常形成多聚體或聚合物的並排結構,尤其在低極性溶液或蒸氣狀態時。例如:乙酸二聚體、氟化氫鏈狀聚合物 [H-(95 pm)F(116°)$\cdots$(155pm)]$_n$、氟化氫環狀六聚體、蛋白質和核酸的 β-褶板。
\item 固態常形成中空結構,降低密度,如冰的密度小於液態水,並可能可在中空處裝入小分子或離子,如甲烷水合物等結晶水合物。
\item 常增加表面張力,因分子間氫鍵形成彈性薄膜,如水。
\item 常增加液體比熱,因受熱時分子間氫鍵吸收能量斷裂,如水。
\item 常增加莫耳汽化熱與沸點,因汽化須破壞氫鍵,如水。
\item 常增加莫耳熔化熱與熔點,但增加量常較沸點小,因熔化僅須打破部分氫鍵以破壞晶格,如水。
\item 互相可形成分子間氫鍵的物質常互相溶解度高甚至混溶,如氨與水。
\item 原先各自均無法形成同種分子間氫鍵的兩種不同分子混合後可形成異種分子間氫鍵者,常造成拉午耳定律較大負偏差,如丙酮和氯仿。
\eit
\sssc{分子內氫鍵(Intramolecular hydrogen bond)}
指一分子同時提供受體和供體形成的氫鍵。常見性質:
\bit
\item 通常形成分子內氫鍵後須圍成五至七員環才是穩定的結構,故分子內氫鍵多發生於順式或鄰位化合物,如:順丁烯二酸、鄰苯二酚、鄰羥基苯甲醛、鄰羥基苯甲酸。
\item 常促進未參與氫鍵的可解離質子解離、抑制參與氫鍵的可解離質子解離,使具未參與氫鍵的可解離質子的多元酸性增強、第一解離常數上升、參與氫鍵的氫的解離常數下降,如:有分子內氫鍵的 2-苯二甲酸 p$Ka_1=2.89$ 遠小於沒有分子內氫鍵的 3-苯二甲酸 p$Ka_1=3.46$ 和 4-苯二甲酸 p$Ka_1=3.54$、有分子內氫鍵的 2-苯二甲酸 p$Ka_2=5.51$ 遠小於沒有分子內氫鍵的 3-苯二甲酸 p$Ka_1=4.46$ 和 4-苯二甲酸 p$Ka_1=4.34$。
\item 常使熔沸點降低,因與其他分子間作用力減小。
\item 加熱時常被破壞。
\item 除參與分子內氫鍵之原子外沒有其他較強電子對供體者,常增加耐酸性,因與酸反應須破壞分子內氫鍵。
\item 除參與分子內氫鍵之原子外沒有其他較強電子對受體者,常增加耐鹼性,因與鹼反應須破壞分子內氫鍵。
\item 常造成一種異構物優於另一種,常為沒有分子內氫鍵的反式異構物優於有分子內氫鍵的順式異構物,如:沒有分子內氫鍵的反丁烯二酸較有分子內氫鍵的順丁烯二酸位能更低而更穩定。
\item 常增加溶解度,因溶質分子間作用力小,在溶劑中易瓦解,如:順式丁烯二酸在水中的溶解度高於反式丁烯二酸。
\item 常增加黏度,如丙三醇>乙二醇>乙醇。
\item 常見供體如羥基與酸性羥基,常見受體如鄰位或同側的羰基氧、硝基氧、羥基氧、氟基、胺基氮,如:順丁烯二酸(順式羧基H-羧基O)、2-羥基苯甲酸(鄰位羥基H-羧基O)、2-羥基苯甲醛(鄰位羥基H-醛基O)、2-氟苯甲酸(鄰位羧基H-氟基)、1,2-苯二酚(鄰位羥基H-羥基O)、2-硝基苯酚(鄰位羥基H-硝基N)、丙二酸(羧基H-羧基O)、2-羥基苯乙酮(鄰位羥基H-酮基O)。
\item 常形成聚合物的螺旋結構,如:蛋白質、核酸、澱粉等的二級螺旋結構。
\eit
\sssc{分子的氫鍵數}
一個分子的氫鍵性質可用分子內氫鍵數(intramolecular hydrogen bond count)、氫鍵受體數(hydrogen bond acceptor count)和氫鍵供體數(hydrogen bond donor count)共同描述。

若分子內氫鍵均形成,每分子最多可參與的同種分子間氫鍵數為氫鍵受體數和氫鍵供體數中較小者減去分子內氫鍵數,此時平均每分子分子間氫鍵數為其二分之一。

兩種分子$a$、$b$勻相混合,不考慮相互形成分子間氫鍵的立體限制,令其分子內氫鍵數分別為$I_a$、$I_b$,氫鍵受體數分別為$A_a$、$A_b$,氫鍵供體數分別為$D_a$、$D_b$,則兩種分子各一間最多可形成的分子間氫鍵數為:
\[\min(D_a-I_A,A_b-I_b)+\min(A_a-I_A,D_b-I_b)\]
\ssc{雙氫鍵(Dihydrogen bond)}
形如 \ce{X}-\ce{H}$\cdots$\ce{H}-\ce{Y},通常 X 是 B$^-$ 或金屬配位中心,Y 是 O 或 N。
\ssc{同類互溶(Like-dissolve-like)規則}
\bit
\item 可互相形成氫鍵的分子互相溶解度高甚至混溶。如:甲醇、乙醇、乙二醇、1-丙醇、2-丙醇、丙三醇、甲醛、乙醛、丙酮、甲酸、乙酸、丙酸、丁酸、2-羥基丙酸混溶於水;對水溶解度乙醇 > 二甲醚 > 乙烷。
\item 高極性分子間通常互溶,低極性分子通常互溶,高極性分子與低極性分子間通常不互溶,同時具有高極性與低極性基團的分子通常可溶於兩者,並可能可以作為界面活性劑,因偶極-偶極力遠大於偶極-誘導偶極力與倫敦分散力。
\item 極性分子若無法與溶劑形成分子間氫鍵,不一定溶於自身可形成強分子間氫鍵的溶劑,因氫鍵遠大於偶極-偶極力。如:氯仿微溶於水。
\item 極性相差較大的分子同處一溶液常造成拉午耳定律正偏差。如:二硫化碳與丙-2-酮。
\item 在水與極性極小之有機物液體分層中,加入與水可形成分子間氫鍵且可與水混溶的小分子,若各物質間均不反應,會溶於水層。
\eit
\ssc{分子物質熔沸點比較規則}
本規則僅為一甚粗略之近似,例外眾多。
\sssc{熔點比較經驗法則}
效應大小略為分子間氫鍵>分子內氫鍵>總電子數或分子量>對稱性>極性。
\bit
\item 分子間氫鍵總鍵能愈大,因斷部分氫鍵,熔點愈高,每分子每多參與一分子間氫鍵沸點約增加10-200°C。
\item 分子內氫鍵愈多,因降低分子間作用力,熔點愈低,每分子每多形成一分子內氫鍵沸點約減少5-50°C。
\item 總電子數愈多或分子量愈大,電子雲分布愈廣,凡得瓦力愈大,熔點愈高。同系物,分子量每乘以二,熔點約增加20-40°C。通常相差在兩個碳以內(氫不計)的分子分子量影響不大。
\item 結構愈對稱(旋轉對稱),晶格堆積愈緊密或晶體愈易於堆積排列,晶格能愈大,熔點愈高,影響大小不一定,其他因素不變,通常沒有旋轉對稱者較有者低約 100°C。
\item 極性愈大,偶極-偶極力愈大,熔點愈高,極性共價鍵兩側原子電負度差每增加0.1,熔點約增加0-20°C。
\eit
\sssc{沸點比較經驗法則}
效應大小略為分子內氫鍵>分子間氫鍵>總電子數或分子量>極性>接觸面積。
\bit
\item 分子內氫鍵愈多,因降低分子間作用力,沸點愈低,每分子每多形成一分子內氫鍵沸點約減少30-250°C。
\item 分子間氫鍵總鍵能愈大,因汽化斷氫鍵,沸點愈高,每分子每多參與一分子間氫鍵沸點約增加10-150°C。
\item 總電子數愈多或分子量愈大,電子雲分布愈廣,凡得瓦力愈大,沸點愈高。同系物,分子量每乘以二,沸點約增加60-100°C。通常相差在兩個碳以內(氫不計)的分子分子量影響不大。
\item 極性愈大,偶極-偶極力愈大,沸點愈高,極性共價鍵兩側原子電負度差每增加0.1,沸點約增加0-20°C。
\item 相鄰分子電子雲接觸面積(通常與截面原子數與原子半徑正相關,與支鏈數量負相關)愈大,分子間作用力愈大,沸點愈高,每一個主鏈原子移到支鏈,沸點約減少0-20°C。
\eit
\ssc{舉例}
數字表溫度(°C),數對表熔點與沸點。
\sssc{一取代基酚}
\begin{longtable}[c]{|c|c|c|c|}
\hline
取代基位置 & 鄰 & 間 & 對\\\hline
苯二酚 & (104, 246) & (110, 277) & (173, 287)\\\hline
氯苯酚 & (9, 175) & (33, 214) & (43, 219)\\\hline
硝基苯酚 & (45, 214) & (95, 277) & (114, 279)\\\hline
\end{longtable}\FB
\bit
\item 對位對稱性最高,鄰、間位對稱性略同,故對位熔點最高。
\item 鄰位有分子內氫鍵故熔沸點最低。
\item 分子間氫鍵鍵能與環上兩取代基距離正相關故對位>間位,故沸點對位>間位。
\item 極性鄰>間>對=0,但影響不如氫鍵與對稱性。
\eit
\sssc{無氫鍵二同取代基苯}
\begin{longtable}[c]{|c|c|c|c|}
\hline
取代基位置 & 鄰 & 間 & 對\\\hline
二甲苯 & (-27, 144) & (-54, 139) & (13, 138)\\\hline
二氯苯 & (-17, 180) & (-24, 172) & (54, 174)\\\hline
\end{longtable}\FB
\bit
\item 對位對稱性最高,鄰、間位對稱性略同,故對位熔點最高。
\item 極性鄰>間>對=0,故熔點鄰>間、沸點鄰>間>對。
\eit
\sssc{無與一取代基苯}
\begin{longtable}[c]{|c|c|}
\hline
苯 & (5.5, 78)\\\hline
甲苯 & (-93, 110.6)\\\hline
乙苯 & (-95, 136)\\\hline
\end{longtable}\FB
對稱性決定熔點;分子量決定沸點。
\sssc{非金屬化氫沸點}
\bit
\item IVA:\ce{PbH4}>\ce{SnH4}>\ce{GeH4}>\ce{SiH4}>\ce{CH4}。依分子量。
\item VA:\ce{BiH3}>\ce{SbH3}>\ce{NH3}>\ce{AsH3}>\ce{PH3}。\ce{NH3}有弱氫鍵故沸點高,其餘依分子量。
\item VIA:\ce{H2O}>\ce{H2Te}>\ce{H2Se}\ce{H2S}。\ce{H2O}有強氫鍵故沸點高,其餘依分子量。
\item VIIA:\ce{HF}>\ce{HI}>\ce{HBr}>\ce{HCl}。\ce{HF}有強氫鍵故沸點高,其餘依分子量。
\item \ce{H2S}>\ce{HCl}。分子量相近且均有極性,分子數決定電子雲密度決定沸點。
\item \ce{GeH4}>\ce{SiH4}>\ce{CH4}。依分子量。
\item \ce{C3H8}>\ce{C2H6}>\ce{CH4}。依分子量。
\eit
\sssc{四氯化碳、乙烷與四氟化碳}
\begin{longtable}[c]{|c|c|c|}
\hline
\ce{CCl4} & (-23, 77) & 分子量最大、電子數最多,電子雲接觸面積最大\\\hline
\ce{C2H6} & (-183, -89) & 原子數最多,雖分子量小於\ce{CF4},但分子較大,電子雲接觸面積次大\\\hline
\ce{CF4} & (-184, -128) & 電子雲最小\\\hline
\end{longtable}\FB
\sssc{水、氟化氫與氨}
\begin{longtable}[c]{|c|c|c|}
\hline
\ce{H2O} & (0, 100) & 二氫鍵每分子,每氫鍵鍵能次強\\\hline
\ce{HF} & (-83, 20) & 一氫鍵每分子,每氫鍵鍵能最強\\\hline
\ce{NH3} & (-78, -33) & 二氫鍵每分子,每氫鍵鍵能最弱,但固體較\ce{HF}穩定\\\hline
\end{longtable}\FB
\sssc{乙醇與乙酸}
\begin{longtable}[c]{|c|c|c|}
\hline
乙醇 & (-114, 78) & 一氫鍵之二聚體,一邊羥基之H供另一邊羥基之O\\\hline
乙酸 & (17, 118) & 二氫鍵之二聚體,兩邊羥基之H均供另一邊羰基之O\\\hline
\end{longtable}\FB
\sssc{第一週期元素}
\begin{longtable}[c]{|c|c|}
\hline
\ce{H2} & (-259, -253)\\\hline
\ce{He} & (-272, -269)\\\hline
\end{longtable}\FB
因電子數相同、分子量相近,故接觸面積決定熔沸點。
\sssc{戊烷}
\begin{longtable}[c]{|c|c|c|c|}
\hline
正戊烷 & (-130, 36)\\\hline
異戊烷 & (-160, 28)\\\hline
新戊烷 & (-18, 10)\\\hline
\end{longtable}\FB
對稱性決定熔點;接觸面積決定沸點。
\sssc{1,2-二氯乙烯}
\begin{longtable}[c]{|c|c|}
\hline
(Z)-1,2-二氯乙烯 & (-80.5, 60.2)\\\hline
(E)-1,2-二氯乙烯 & (-49.4, 48.5)\\\hline
\end{longtable}\FB
對稱性決定熔點;極性決定沸點。
\sssc{乙烴}
\begin{longtable}[c]{|c|c|}
\hline
乙烷 & (-183, -89)\\\hline
乙烯 & (-170, -104)\\\hline
乙炔 & -84°C 昇華\\\hline
\end{longtable}\FB
直線形乙炔對稱性最大且電子雲接觸面積最大;乙烯較乙烷對稱故熔點較高;乙烷較乙烯分子量大故沸點較高。
\sssc{正烷}
\begin{longtable}[c]{|c|c|}
\hline
甲烷 & (-182, -161)\\\hline
乙烷 & (-183, -89)\\\hline
丙烷 & (-187, -42)\\\hline
丁烷 & (-138, -0.5)\\\hline
\end{longtable}\FB
丁烷分子量最大故熔點最高,對稱性決定其餘之熔點;分子量決定沸點。
\sssc{順丁烯二酸與反丁烯二酸}
\begin{longtable}[c]{|c|c|c|c|}
\hline
性質 & 順丁烯二酸 & 反丁烯二酸 & 備註 \\\hline\endhead
每分子分子內氫鍵 & 1 & 0 & 順式羧酸基的羥基供同分子另一羧酸基的羰基 \\\hline
每分子分子間氫鍵 & 1 & 2 & 羧酸基的羥基供另一分子羧酸基的羰基 \\\hline
熔點(°C) & 135 & 287 & 反式對稱性較佳 \\\hline
沸點(°C) & 202 & 356 & 反式分子間氫鍵較多 \\\hline
對水溶解度 & 大 & 小 & 順式有分子內氫鍵在水中易瓦解 \\\hline
第一酸解離常數 & 大 & 小 & 順式釋放未參與分子內氫鍵的質子更穩定 \\\hline
第二酸解離常數 & 小 & 大 & 順式的第二個質子參與分子內氫鍵故不易釋放 \\\hline
位能 & 大 & 小 & 反式較穩定 \\\hline
與鎂帶反應釋放氫氣 & 快 & 慢 & 順式釋放未參與分子內氫鍵的質子更穩定\\\hline
\end{longtable}\FB


\sct{能帶理論(Band theory)}
能帶理論描述固體材料中電子的能量分布和電子在材料中的行為。
\ssc{(電子)能帶結構((Electronic) band structure)}
\sssc{能帶}
多個原子的原子軌域依分子軌域理論產生鍵結、反鍵結等分子軌域,當數目非常巨大(通常數量級$10^{20}$或更多)的原子結合成固體時,分子軌域間的能階能量差將會變的非常小,一群能量相鄰的能階,可以被視為一個連續能帶(Band),分為:
\bit
\item\tb{價帶(Valence band)}:充滿電子的最高能帶。
\item\tb{(傳)導帶(Conduction band)}:空的或部分填充的能帶。其中電子為自由電子(Free electron),可以自由移動形成電流。
\item\tb{能隙(Band gap)}$E_g$:導帶底部(Conduction Band Minimum)的能量減去價帶頂部(Valence Band Maximum)的能量。
\eit
以Li為例,單一原子則電子位於固定能階的2s軌域,Li$_2$分子則兩個2s軌域結合成兩個能量不同的分子軌域,一個較原本高,另一較原本低。若有極多個Li原子時,可視為形成兩個能量幾乎連續的能帶,其中較低能量者為價帶,被電子填滿,較高能量者無電子或部分填充電子為導帶,兩能帶間的能量差即能隙。
\sssc{絕緣體(Insulators)、半導體(Semiconductors)和導體(Conductors)}
\bit
\item 絕緣體(Insulators):具有大的能隙,電子不易躍遷到導帶。
\item 半導體(Semiconductors):具有較小的能隙,電子在適當條件下可以激發到導帶。
\item 導體(Conductors):價帶和導帶重疊或能隙極小,電子可以自由移動到導帶。
\eit
\sssc{費米能階(Fermi level)}
一個電子的假設能階,使得在 0 K 溫度且熱力學平衡時,該能階有50\%的機率在任何給定時間都被占用。令費米能階$E_f$。
\bit
\item 絕緣體中,$E_f$位於能隙上,與導帶和價帶相距甚遠。
\item 金屬或退化半導體中,$E_f$位於導帶上。
\item 無雜質半導體或少量摻雜的半導體中,$E_f$雖位於能隙上,但與導帶和價帶較近。
\eit
\sssc{功函數(Work function)}
令材料的費米能階$E_f$、基本電荷$e$、電子在無限遠真空處的能量$E_{\text{vac}}$,則功函數$W$為:
\[W=E_{\text{vac}}-E_f\]
功函數與光電效應底限頻率關係為:
\[W= h f_0\]
\ssc{電子與電洞(Hole)}
\sssc{電洞/空穴(Hole)/電子電洞(Electron hole)}
h$^+$,是一種缺乏電子的位置或空間,是正電荷載子。電洞可以在晶格中移動,像正電荷一樣在外部電場存在時沿電場方向移動。
\sssc{載子產生(Carrier generation)}
指一個電子從價帶中被激發到導帶,會在價帶中留下一個電洞,產生兩個等值異號的電荷載子。
\sssc{光生伏打效應/光伏效應(Photovoltaic effect)/內光電效應(Internal photoelectric effect)}
指在光的照射下,物質材料中電子的被激發而自價帶躍遷至導帶,稱光生電子,並在價帶留下電洞,稱光生電洞,即產生光生電子電洞對的現象。有時也被視為光電效應的一種。

若入射光頻率大於等於該物質的底限頻率(threshold frequency)$f_0$則立即產生光生電子電洞對,若入射光頻率小於該物質的底限頻率則永不產生光生電子電洞對。類似於光電效應,可用光量子論和能帶理論解釋。令物質能隙$E_g$,則底限頻率為:
\[f_0=\frac{E_g}{h}\]

常用於太陽能電池(Solar cell)、電荷耦合元件(Charge coupled device, CCD)、光敏電阻(Photoresistor)、光控繼電器(Light-controlled relay)。
\sssc{載子複合(Carrier recombination)與電致發光/電激發光(Electroluminescence)}
指電子和電洞複合,會以光子的形式釋放能量,釋放的能量等於能隙$E_g=hf$,故能隙不同的物質放出不同頻率的光。常用於發光二極體(Light emitting diode, LED)。
\sssc{激子(Exciton)}
描述了一對電子與電洞由靜電力相互吸引而構成的束縛態,可視為電中性的准粒子。
\ssc{量子限制(Quantum confinement)}
\sssc{量子井/量子阱(Quantum well)}
指奈米材料的厚度與載子的物質波長具有相同的數量級,材料不呈現連續能帶而是呈現離散的子能帶(energy subbands)的現象,使得能隙增大。
\sssc{量子線(Quantum wire)}
指一維奈米材料的直徑與載子的物質波長具有相同的數量級,載子於其中只能在一個方向上移動的一維量子系統,不遵循電阻等於電阻率乘以長度除以截面積的公式,電導率小於等於基本電荷的平方乘以二除以普朗克常數,其中無內部散射時等於之,與奈米結構極其相關,可以此利用金屬或奈米碳管等製造出導體至絕緣體。
\sssc{量子點(Quantum dot, QD)/半導體奈米晶體(Semiconductor nanocrystal)}
半導體的一種能夠在三維空間中限制載子的零維奈米結構,激子的束縛能取決於量子點的尺寸,可以此調整其吸收與發射光譜。


\sct{半導體(Semiconductor)}
能隙介於導體與絕緣體之間的物質。
\ssc{無機半導體}
1947年,諾基亞貝爾實驗室(Nokia Bell Labs)的巴丁(John Bardeen)、布拉頓(Walter Houser Brattain)和肖克利(William Shockley)以鍺材料發明點接觸電晶體(point-contact transistor),是史上首個半導體電子元件,獲1956年諾貝爾物理獎。

無機半導體材料如矽 \ce{Si}(現最常用)、鍺 \ce{Ge}(最早使用)、砷化鎵 \ce{GaAs}(常用於場效電晶體)、磷化鎵 \ce{GaP}(常用於發光二極體)、磷化鎵鋁銦 \ce{AlGaInP}(常用於發光二極體)、砷化鎵鋁 \ce{AlGaAs}、二氧化鈦 \ce{TiO2}(能隙較大)。
\ssc{有機半導體(Organic semiconductor)}
有機導電聚合物等,如聚乙炔及其衍生物。有機半導體製成的元件通常操作電壓較低、耗電較低、製程簡單、能量密度較高、體積小、重量輕、可撓曲性遠高於類金屬半導體,可用導電塑膠薄膜代替許多類金屬半導體材料,可用於有機發光二極體顯示器、可撓式太陽能電池、可撓式觸控面板、充電電池材料等,以聚苯胺用於充電電池電極的應用最多。但缺點如,有機分子易氧化使生命週期縮短、高分子材料一般導熱性差使廢熱不易導出等,後者可能通過在導電聚合物材料中添加金屬粉末的複合材料解決。
\ssc{溫度效應}
溫度增加時,電子獲得更多能量,更容易躍遷到導帶,故溫度升高會增加半導體的電導率。
\ssc{晶格結構效應}
不同晶格排列可能導致不同的能帶寬度和能隙。
\ssc{本徵半導體(Intrinsic semiconductor)/無摻雜半導體(Undoped semiconductor)/純半導體(Pure semiconductor)}
指沒有摻雜的純晶體半導體,其參與導電的自由電子和電洞濃度相等。
\ssc{非本徵半導體(Extrinsic semiconductor)/摻雜半導體(Doped semiconductor)}
\sssc{摻雜(Doping)}
指透過在半導體晶體結構中添加雜質,可以改變其能隙,亦可增加載子。
\subsubsection{n 型半導體}
引入更多自由電子,費米能階上移(靠近導帶),導電性增加。對於類金屬半導體通常是在高純度半導體中加入少量第15族元素,如 P、As。
\subsubsection{p 型半導體}
引入更多電洞,費米能階下移(靠近價帶),導電性增加。對於類金屬半導體通常是在高純度半導體中加入少量第13族元素,如 B、Al。
\ssc{p–n 接面(p–n junction)}
指 n 型和 p 型半導體的接合面。
\sssc{半導體二極體(Semiconductor diode)}
利用 p–n 接面,順向偏壓時電子從 n 型流向 p 型,逆向偏壓時無電流,可用於整流等。
\sssc{發光二極體(Light emitting diode, LED)}
將順向偏壓通入半導體二極體,使載子複合而電致發光。

早期研發出的紅色和綠色單色發光二極體用途較受限,1992年赤崎勇、天野浩與中村修二以氮化鎵材料發明藍光 LED,始可用光的三原色調配任意所需的顏色,尤其用作照明的白光,較傳統白熾燈泡與日光燈更省電、體積更小、壽命更長,共同獲2014年諾貝爾物理獎。
\sssc{有機發光二極體(Organic light-emitting diode, OLED)}
有機半導體製成的二極體,原料如「三(8-羥基喹啉)鋁/三[8-羥基(苯[b]氮雜環己-1,3,5-三烯)]鋁(tris(8-hydroxyquinolinato) aluminium(III))\ce{Al(C9H6NO)3}」或「聚對苯乙炔(Poly(p-phenylene vinylene), PPV) [-c1ccc(cc1)C=C-]$_n$」。
\subsubsection{光伏電池(Photovoltaic cell)/太陽能電池(Solar cell)}
通常由一個或多個 p–n 接面組成,吸收光子並輸出直流電。種類如:
\begin{itemize}
\item 單結晶矽太陽能電池:高效率但成本較高。
\item 多晶矽太陽能電池:效率較低但成本較低。
\item 薄膜太陽能電池:輕薄柔軟,適用於特定場景。
\end{itemize}
效率提升方法:
\begin{itemize}
\item 減少表面反射:透明導電層和抗反射層減少光子的反射。
\item 改進材料:新材料如有機太陽能電池和混合鈣鈦礦太陽能電池提高效率。
\item 多接面結構:提高光吸收和轉換效率。
\item 雷射細工:提高電池表面光陽極的粗糙度,增加光吸收。
\end{itemize}
\ssc{場效電晶體(Field-effect transistor, FET)}
FET 具有四個端子:源極(source)、閘極(gate)和汲極(drain)、體(body)/基(base)/塊體(bulk)/基板(substrate)。通過在閘極施加電壓,可以改變源極與汲極之間的電導率,從而控制電流,分為需加閘極電壓方能導通源極與汲極的增強模式(enhancement-mode),較常見;與預設導通源極與汲極,加閘極電壓可減弱或關閉它的空乏模式/減少模式(depletion-mode)。體則可以將電晶體調變至運行,在電路設計中少用。

FET 僅使用電子與電洞的其中一種作為電荷載子,前者稱 n-通道(n-channel),後者稱 p-通道(p-channel)。源極與汲極下方的摻雜半導體具有該種載子,在增強模式中預設不相連而斷路,但加閘極電壓可以引入載子使相連而導通;在空乏模式中預設相連而導通,但加閘極電壓可以引入相反的載子使變窄增加電阻,繼而不相連而斷路。



\sct{晶體(Crystal)}
成分粒子(如原子、分子或離子)以高度有序的微觀結構排列的固體,形成向各個方向延伸的晶格。 
\ssc{晶胞(Cell)/晶格(Lattice)}
指構成晶體重複結構的單位。
\sssc{原始晶胞(primitive cell)}
指構成晶體重複結構的最小單位。其表示方法不指一種,但均包含相同數量的晶格點,且對於二維晶體,其面積固定,為四個頂點共有一個粒子的平行四邊形;對於三維晶體,其體積固定,為八個頂點共有一個粒子的平行六面體。
\sssc{配位數}
與一個晶格點最近的晶格點數量。
\sssc{維格納-賽茨晶胞(Wigner–Seitz cell)}
對於二維晶體,一晶格點(lattice point)周圍的維格納-賽茨晶胞被定義為平面上距離該晶格點比距離任何其他晶格點更近的點的軌跡。

對於三維晶體,一晶格點周圍的維格納-賽茨晶胞被定義為空間中距離該晶格點比距離任何其他晶格點更近的點的軌跡。
\sssc{原始平移向量(Primitive translation vectors)}
二維晶體晶格的原始平移向量為$\mathbf {a}$與$\mathbf{b}$表示以晶體中一原子為原點,晶體中的原子均位於$p\mathbf {a} +q\mathbf {b} $,其中$p$、 $q$為整數。

三維晶體晶格的原始平移向量為$\mathbf {a}$、$\mathbf{b}$與$\mathbf {c}$表示以晶體中一原子為原點,晶體中的原子均位於$p\mathbf {a} +q\mathbf {b} +r\mathbf {c} $,其中$p$、 $q$、$r$為整數。
\ssc{晶族(crystal family)、晶系(crystal system)與晶格系(lattice system)}
\sssc{分類}
\begin{longtable}[c]{|p{0.2\tw}|p{0.2\tw}|p{0.2\tw}|p{0.2\tw}|}
    \hline
        晶族 & 晶系(由點群決定) & 點群所需的對稱性 & 晶格系(由布拉維格子決定) \\ \hline\endhead
        三斜(Triclinic) & 三斜(Triclinic) & 無 & 三斜(Triclinic) \\ \hline
        單斜(Monoclinic) & 單斜(Monoclinic) & 1個雙重旋轉軸或1個鏡像平面 & 單斜(Monoclinic) \\ \hline
        斜方/正交(Orthorhombic) & 斜方/正交(Orthorhombic) & 3個雙重旋轉軸或1個雙重旋轉軸和2個鏡像平面 & 斜方/正交(Orthorhombic) \\ \hline
        四方/正方(Tetragonal) & 四方/正方(Tetragonal) & 1個四重旋轉軸 & 四方/正方(Tetragonal) \\ \hline
        六方(Hexagonal) & 三方(Trigonal) & 1個三重旋轉軸 & 菱面(Rhombohedral)或六方(Hexagonal) \\ \hline
        六方(Hexagonal) & 六方(Hexagonal) & 1個六重旋轉軸 & 六方(Hexagonal) \\ \hline
        立方/等軸(Cubic) & 立方/等軸(Cubic) & 4個三重旋轉軸 & 立方/等軸(Cubic) \\ \hline
    \end{longtable}\FB
\sssc{幾何詞彙釋義}
\bit
\item \tb{三維最密堆積/密排(close-packed)}:最密等體積球填充,堆積密度 $\frac{\sqrt{2}\pi}{6}\approx 74\%$。
\item \tb{二維最密堆積/密排(close-packed)}:最密等面積圓填充,堆積密度 $\frac{\sqrt{3}\pi}{6}\approx 91\%$。
\item \tb{截角八面體(Truncated octahedron)}:正八面體的六個頂點截去六個四角錐形成的具有14個面(8個全等正六邊形與6個全等正方形)、36條邊與24個頂點的凸多面體。
\item \tb{菱形十二面體(Rhombic dodecahedron)}:具有12個全等菱形面、24條邊與14個頂點的凸多面體。
\eit
\sssc{單成分粒子晶體}
\begin{longtable}[c]{|p{0.15\textwidth}|p{0.05\textwidth}|p{0.2\textwidth}|p{0.05\textwidth}|p{0.05\textwidth}|p{0.1\textwidth}|p{0.1\textwidth}|p{0.1\textwidth}|}
\hline
堆積方式 & 配位數 & 平行四邊形/平行六面體晶胞描述 & 平行四邊形/平行六面體晶胞粒子數 & 平行四邊形/平行六面體晶胞邊長除以粒子半徑 & 維格納-賽茨晶胞形狀 & 堆積密度 & 舉例 \\\hline\endhead
\tb{簡單立方(Simple cubic)/原始立方(Primitive cubic, cP)} & $6$ & 每個頂點各八分之一個粒子 & $1$ & $2$ & 正方體 & $\frac{\pi}{6}\approx 52\%$ & 釙(唯一已知自然以簡單立方存在之金屬) \\\hline
\tb{面心立方(Face-centered cubic, cF, fcc)/立方密排(cubic closepacked)} & $12$ & 簡單立方每面中心再加半個粒子 & $4$ & $2\sqrt{2}$ & 截角八面體 & $\frac{\sqrt{2}\pi}{6}\approx 74\%$ & IA、鈣、鍶、鋁、鎳、銀、銅、金 \\\hline
\tb{體心立方(Body-centered cubic, cL, bcc)} & $8$ & 簡單立方中心再加一個粒子 & $2$ & $\frac{4\sqrt{3}}{3}$ & 菱形十二面體 & $\frac{\sqrt{3}\pi}{8}\approx 68\%$ & 鋇、鐳、鐵、釩、鈮、鉻 \\\hline
\tb{鑽石立方(diamond cubic, dc)} & $4$ & 面心立方,取四個兩兩為一面對角線的頂點,分別與與該頂點相鄰的三個面的中心做一四面體,在該四個四面體的中心各增加一個粒子 & $8$ & $\frac{8\sqrt{3}}{3}$ & - & $\frac{\sqrt{3}\pi}{16}\approx 34\%$ & 金剛石、矽、碳化矽、α-錫(灰錫) \\\hline
\tb{簡單六方(Simple hexagonal)} & $8$ & 角度60°與120°的菱形底面的菱形柱,底面60°的頂點各 $\frac{1}{12}$ 個粒子、底面120°的頂點各 $\frac{1}{6}$ 個粒子 & $1$ & $2$ & 正六邊形柱 & $\frac{\sqrt{3}\pi}{9}\approx 61\%$ & - \\\hline
\tb{六方密排/六方最密(Hexagonal close-packed, hcp)} & $12$ & 簡單六方,其菱形柱分成兩個正三角柱,其一的中心加一粒子 & $2$ & 菱形底面$2$、高$\frac{8\sqrt{3}}{3}$ & - & $\frac{\sqrt{2}\pi}{6}\approx 74\%$ & 鈹、鎂、鈦、鈷、鋅、鎘、鈧、釔 \\\hline
\end{longtable}\FB
\sssc{一比一雙成分粒子晶體}
\begin{longtable}[c]{|p{0.15\textwidth}|p{0.05\textwidth}|p{0.2\textwidth}|p{0.05\textwidth}|p{0.05\textwidth}|p{0.1\textwidth}|p{0.1\textwidth}|p{0.1\textwidth}|}
\hline
堆積方式 & 配位數 & 平行四邊形/平行六面體晶胞描述 & 平行四邊形/平行六面體晶胞粒子數 & 平行四邊形/平行六面體晶胞邊長除以兩成分粒子半徑和 & 維格納-賽茨晶胞形狀 & 堆積密度 & 舉例 \\\hline\endhead
\tb{平面三角(Trigonal planar)} & $3$ & 角度60°與120°的菱形,60°的頂點各 $\frac{1}{6}$ 個一類粒子、120°的頂點各 $\frac{1}{3}$ 個該類粒子、一60°的頂點與其二相鄰頂點形成的正三角形的中心一個另一類粒子 & $2$ & $\sqrt{3}$ & 正三角形 & $\frac{\sqrt{3}\pi}{9}\approx 61\%$ & - \\\hline
\tb{B1/NaCl/氯化鈉/岩鹽} & $6$ & 一類粒子面心立方、另一類粒子置於邊中點與體心,屬互穿面心立方 & $8$ & $2$ & 正方體 & $\frac{\pi}{6}\approx 52\%$ & \ce{NaCl}, \ce{MgO} \\\hline
\tb{B2/CsCl/氯化銫} & $8$ & 一類粒子簡單立方、另一類粒子置於體心,屬互穿簡單立方 & $2$ & $\frac{2\sqrt{3}}{3}$ & 正八面體 & $\frac{\sqrt{3}\pi}{8}\approx 68\%$ & \ce{CsCl}, \ce{NH4Br} \\\hline
\tb{B3/ZnS/硫化鋅/閃鋅礦} & $4$ & 一類粒子面心立方,取四個兩兩為一面對角線的頂點,分別與與該頂點相鄰的三個面的中心做一四面體,在該四個四面體的中心各增加一個另一類粒子,屬互穿面心立方 & $8$ & $\frac{4\sqrt{3}}{3}$ & 正四面體 & $\frac{\sqrt{3}\pi}{16}\approx 34\%$ & \ce{ZnS}, \ce{CuCl} \\\hline
\end{longtable}\FB
\end{document}