\documentclass[a4paper,12pt]{article}
\setcounter{secnumdepth}{5}
\setcounter{tocdepth}{3}
\newcounter{ZhRenew}
\setcounter{ZhRenew}{1}
\newcounter{SectionLanguage}
\setcounter{SectionLanguage}{1}
\input{/usr/share/latex-toolkit/template.tex}
\begin{document}
\title{基礎化學}
\author{沈威宇}
\date{\temtoday}
\titletocdoc
\section{基礎化學}
\ssc{化學基本學說發展}
\subsubsection{拉瓦節的質量守恆定律(Law of conservation of mass)}
一封閉系統的質量始終不變。(不適用於核反應,僅適用於物理與化學反應)
\subsubsection{普魯斯特的定比/定組成定律(Law of definite proportions)}
一種化合物無論其來源或製備方法如何,其組成元素間有固定的質量比例。
\subsubsection{道耳頓的原子論(Atomic theory)}
\begin{itemize}
\item 一切物質由稱為原子的微小粒子所組成,這種粒子不能產生,不會消滅,也不能再分割。(錯誤)
\item 相同元素的原子,其質量與性質均相同;不同元素的原子,其質量與性質均不同。(同位素質量與性質不同;不同元素可以有相同質量)
\item 化合物是由不同種類的原子以簡單整數比結合而成。(除具有晶格缺陷之晶體外正確)
\item 化學反應中,原子重新排列組合形成新物質,過程中原子不會改變其質量或性質,也不會產生新的原子或消失。(對今定義之化學反應正確,但另有核反應會改變核種)
\end{itemize}
道耳頓的原子論成功解釋質量守恆定律、定比定律和倍比定律。
\subsubsection{道耳頓的倍比定律(Law of multiple proportions)}
若兩種元素可以生成多種化合物,則在這些化合物中,若固定其中一元素的質量,另一元素的質量比為簡單整數比。(除具有晶格缺陷之晶體外正確)
\subsubsection{給呂薩克的氣體化合體積定律(Law of combining volumes)}
在化學反應中,反應物與生成物的氣體物質在同溫同壓下,其體積變化量恆為簡單整數比。(對理想氣體正確)
\subsubsection{亞佛加厥的分子說}
因為道耳頓的原子論無法解釋給呂薩克的氣體化合體積定律而提出。
\begin{itemize}
\item 一切物質由稱為原子的微小粒子所組成,這種粒子不能再分割。(下半句為誤)
\item 相同元素的原子,其質量與性質均相同;不同元素的原子,其質量與性質均不同。(質量部分為誤)
\item 化合物是由不同種類的原子以簡單整數比結合而成。(正確)
\item 化學反應中,原子重新排列組合形成新物質,過程中原子不會改變其質量或性質,也不會產生新的原子或消失。(對今定義之化學反應正確,但另有核反應會改變核種)
\end{itemize}
\subsubsection{亞佛加厥定律(Avogadro's law)}
同溫、同壓時,同體積的任何氣體含有相同數目之分子。(對理想氣體正確)
\ssc{原子(Atom)}
\sssc{符號表示法}
$^{A}_{Z}\text{X}$,其中 X 為該原子所屬元素之符號,$A$為核子數,$Z$為質子數,如$^4_2$He,可在右上加以電荷,標示為電性在印度—阿拉伯數字後,如 Na$^+$、Ca$^{2+}$。
\sssc{原子}
\bit
\item\tb{核子(Nucleon)}:中子(neutron)和質子(proton)的合稱。
\item\tb{原子核(Atomic nucleus)}:由至少一個核子組成的粒子。
\item\tb{原子(Atom)}:由原子核與核外電子組成的粒子。
\item\tb{原子序(Atomic number, Z)}:原子或原子核的質子數。
\item\tb{質量數(Mass number, A)}:原子或原子核的質子數 Z 與中子數 N 之和。
\item\tb{元素(Element)}:一元素 X 為一類原子序 Z 相同的原子,記作 X 或 $_{Z}\text{X}$。
\eit
\sssc{原子量(Atomic mass)}
\begin{itemize}
\item\tb{IUPAC 相對原子質量(Relative atomic mass)定義}:特定來源的元素的相對原子質量是該元素每個原子的平均質量與 $^{12}$C 原子質量的 $\frac{1}{12}$ 之比,無因次。
\item\tb{IUPAC 相對原子質量舊定義}:特定來源的元素的相對原子質量是該元素每個原子的平均質量與 $^1$H 原子質量之比,無因次。
\item\tb{道耳頓(Dalton, Da)/統一原子質量單位(atomic mass unit, amu)/統一原子質量單位(unified atomic mass unit, u)$m_u$/u/Da/amu}:$^{12}$C 原子質量的 $\frac{1}{12}$。
\item\tb{原子量(Atomic mass, $m$)}:特定來源的原子以道耳頓為單位之質量。
\item\tb{平均原子量(Average atomic mass)}:以一元素的各同位素的天然豐度為權重的原子量加權平均。週期表上,天然元素之原子量指平均原子量。
\item\tb{分子量(Molecular mass)}:分子式中各原子之原子量總和。
\item\tb{式量(Formula mass)}:實驗式中各原子之原子量總和。
\item\tb{坎尼札洛法(Cannizzaro's method)}:含某元素的化合物的式量乘以該元素占該化合物的質量比率為該元素的原子量的整數倍。
\item\tb{同位素(Isotope)}:指具有相同質子數與不同中子數的原子。同位素有相似的化學性質與不同的物理性質。可使同位素離子射流通過質譜儀測量之。
\item\tb{天然豐度(Natural abundance, NA)}:一元素的一同位素的天然含量占該元素全部天然含量之比例。
\item\tb{同量素(Isobar)}:原子量相同且原子序不同的原子。
\item\tb{同系物(Homologue)}:同系列(Homologous series)指結構與化學性質相似,相鄰成員的組成相差相同化學單元的一系列化合物,常為有機化合物,如直鏈烷烴系列、直鏈飽和一元伯醇系列。同系列中的化合物互為同系物。
\item\tb{同素異形體(Allotrope)}:指由相同的單一元素組成,而結構形態卻不相同的純物質。
\item\tb{同分異構物(Isomer)}:指擁有相同分子式,但結構式卻不相同的多種分子。
\item\tb{異質同形體(Isomorphism)}:對稱性相同且晶胞(unite cell)參數相似的化合物。
\end{itemize}
\sssc{莫耳(Mole)}
\begin{itemize}
\item\tb{IUPAC 莫耳(Mole)定義}:1 莫耳的物質恰好包含 $\scinote{6.02214076}{23}$ 個基本實體。
\item\tb{IUPAC 莫耳舊定義}:1 莫耳的物質包含的基本實體個數同 $12.0000$ 克的碳-12 所包含的原子數量。
\item\tb{亞佛加厥常數(Avogadro constant)$N_A$}:\scinote{6.02214076}{23} mol$^{-1}$
\item\tb{莫耳質量(Molar mass)}:每莫耳的質量,因次為$\mathsf{M}\mathsf{N}^{-1}$。數值同式量(對於非分子)或分子量(對於分子)。
\item\tb{莫耳分率(Mole fraction)}:混合物中,某物質的莫耳分率等於混合物中該物質之莫耳數除以混合物中各物質之總莫耳數。
\item\tb{莫耳體積(Molar volume)}:每莫耳的體積,因次為長度的立方除以物量。STP 下氣體莫耳體積為 $22.4$ 公升,NTP 下氣體莫耳體積為 $24.5$ 公升。
\item\tb{體積莫耳濃度(Volume molar concentration)}$C_M$/$[\tx{物質}]$:每公升數所包含某物質的莫耳數,單位 M=mol/L。
\end{itemize}
\subsection{物質(Substance)的分類}
\bit
\item\tb{物質(Substance)}:指靜止質量不為零的東西。
\item\tb{純(物)質(Pure substance)/化學物質(Chemical substance)}:具有一定的組成和性質的物質。各元素組成的純物質只要各組分元素比例固定且每個該物質的最小單位均相同即可,不一定要由化學鍵連結所有組分元素,也不一定要有單一的熔沸點。
\item\tb{混合物(Mixture)}:不是純物質的物質。可以藉由物理方法分離出不同的純物質。
\item\tb{均/勻相物質(Homogeneous substance)}:呈現單一相的物質。
\item\tb{非均/勻相物質(Heterogeneous substance)}:不是均相物質的物質。
\item\tb{元素(態)(Element (state))/游離態(Free state)/單質(Elementary substance or simple substance)}:由相同的原子組成的純物質。如紅銅、白金、24K金。
\item\tb{化合物(Compound)}:不是元素的純物質。可以藉由化學方法分離出不同的元素。
\item\tb{溶液(Solution)}:勻相混合物。如碘酒(碘溶於乙醇)、18K金(金銅合金,金占 0.75)、鹽酸(\ce{HCl(aq)})、黃銅(銅鋅合金)、青銅(銅錫合金)、濃硫酸、濃硝酸、酒(乙醇等溶於水)、工業酒精。
\item \tb{晶體(Crystalline)}:粒子有規律地重複排列、形成明確晶格的固體,熔點固定。
\item \tb{無定形(Amorphous)固體}:粒子排列雜亂無序、沒有固定的晶格結構的固體,沒有明顯的熔點,通常是隨加熱而逐漸變軟。
\eit
\subsection{物質的變化與性質}
\sssc{物理變化(Physical change)}
影響化學物質形式的變化,但不影響其化學成分。服從動量守恆、角動量守恆、質量數守恆、電荷守恆、核電荷守恆、質量守恆、能量守恆。例如電子躍遷、相變、溶解(但解離為化學變化)、溫度改變。除電子躍遷、與溫度正比的能量與體積功外,能量級通常小於100kJ/mol,氫原子電子躍遷能量小於等於1312kJ/mol。
\sssc{化學變化(Chemical change)/化學反應(Chemical reaction)}
是導致一組化學物質化學轉化為另一組化學物質的過程。服從動量守恆、角動量守恆、質量數守恆、電荷守恆、核電荷守恆、質量守恆、能量守恆。例如氧化還原、酸鹼中和、鹽類沉澱、解離。能量級通常介於100-1000kJ/mol。
\sssc{核變化(Nuclear change)/核反應(Nuclear reaction)}
是兩個原子核或一個原子核與外部亞原子粒子碰撞產生一種或多種新元素的過程。服從動量守恆、角動量守恆、質量數守恆、電荷守恆、核電荷守恆、質能守恆(靜止質量改變$m$所須的能量$E$服從$E=mc^2$,其中$c$為真空光速),能量級可達$10^8$kJ/mol。
\subsubsection{化學反應的分類}
\begin{itemize}
\item\tb{化合/結合(Synthesis)}:由兩種或兩種以上物質生成另一種物質的反應。
\item\tb{分解(Decomposition)}:由一種反應物生成兩種或兩種以上其他物質的反應。
\item\tb{置換/單取代(Single replacement)}:由元素或離子反應物取代化合物反應物中的一個元素或離子。
\item\tb{複置換/複分解/雙取代(Double replacement)}:兩個化合物交換元素或離子形成不同的化合物,大多發生在水溶液中。
\end{itemize}
\sssc{物理性質(Physical property)}
是物理系統的任何可在不發生物理變化以外的變化下測量的性質,如密度、狀態、顏色、導電度、熔點、沸點、相變焓、溶解度、延性、展性。
\sssc{化學性質(Chemical property)}
是物質僅能在化學反應期間或之後測量的任何性質,如可燃性、助燃性、酸鹼性、氧化還原性、活性。
\subsection{化學式(Chemical formula)}
\subsubsection{實驗式(Empirical formula)/最簡式/簡式}
是用元素符號表示化合物中各元素的原子個數最簡整數比的化學式。其原子量乘上個數的和稱式量。如\ce{NaCl}。
\subsubsection{分子式(Molecular formula)}
用元素符號表示一個分子中各元素的原子個數,僅適用於分子化合物。其原子量乘上個數的和稱分子量。為實驗式的整數倍。如\ce{C6H12O6}。
\subsubsection{路易斯結構式(Lewis structure)/路易斯電子點式(Lewis electron dot formula)}
每個原子在其所在的位置上用不同的元素代號標示,每對鍵結(電子)對/鍵結對電子(bonding pair, b.p.)/共用(電子)對/共用對電子(shared pair)以畫在所屬原子間的一條線或一對點來表示,多鍵者各線平行,每對孤(電子)對/孤對電子(lone pair, l.p.)以畫在所屬原子一側的一對點表示,每個孤不成對電子(Unpaired electron)以畫在所屬原子一側的單個點表示,離子(團)須繪製於方括號 [ ] 中再於其右上標示電荷數。
\subsubsection{結構式(Structural formula)}
在平面上畫出分子結構的拓樸,並以線繪出路易斯電子點式中的鍵結電子對,其餘電子省略。有時將形式電荷為正的原子與為負的原子的鍵結畫作前者指向後者的箭號,如\ce{CO}參鍵其一畫作碳指向氧的箭號;有時在各具有非零形式電荷的原子右上標以形式電荷,如C$^-\equiv$O$^+$。
\subsubsection{縮合結構式(Condensed formula)}
省略大部分的鍵線,將各官能基則以簡寫表示與排列。
\subsubsection{鍵線式/骨架式(Skeletal formula)/線角(結構)式(Line-angle formula)/折線簡式}
將結構式的碳改以鍵線的端點與交點表示,並省略接在碳上的氫與碳-氫鍵。在表示有機化合物的立體結構時尤其常用。
\subsubsection{示性式(Functional group formula)/結構簡式(Condensed formula)}
將縮合結構式的所有鍵線均省略。如\ce{C2H5OH}。用來表示所含的基團。
\subsubsection{立體結構式/結構圖(Structural diagram)}
同結構式但將鍵在立體中的出、入紙面分別以實楔型(wedge)和虛楔型(dashed wedge)表示。
\subsubsection{球棍模型(Ball-and-stick model)/球棒模型}
用球表示原子,用棍表示化學鍵,展示分子的三維結構,球的大小代表不同原子種類的相對體積,棍的長度和角度表示鍵長和鍵角,一鍵中棍的數量表鍵數。
\subsubsection{空間填充模型(Space-filling model)/Calotte model/CPK 模型(CPK model)/Robert Corey - Linus Pauling - Walter Koltun model}
用球表示原子,將原子球緊密相接,展示分子的三維結構,球的大小代表不同原子的相對體積,球相嵌代表共價鍵,相嵌球之中心之距離表示鍵長,兩球與同一球相嵌時兩對相嵌球之中心之連線的夾角表示鍵角。
\subsection{化學計量(Stoichiometry)}
\subsubsection{化學方程式(Chemical equation)}
以符號和化學式形式對化學反應的符號表示。單向反應箭號用\ce{->},雙向可逆反應箭號用\ce{<=>},單純表示該化學方程式已平衡但不涉及反應可能與否箭號用 =;反應物在箭號左,生成物在箭號右,物質之相可以 () 附註於物質後方,如 (g) 指氣相、 (l) 指液相、 (s) 指固相、(aq) 指水溶液、(alc) 指乙醇溶液、(溶劑化學式) 指溶液,正反應與逆反應之催化劑與環境(如有)分別寫在箭號上下,各化學物質可以用各類化學式表示。反應物與生成物之係數使得滿足動量守恆、角動量守恆、質量數守恆、電荷守恆、核電荷守恆、質量守恆者稱該化學方程式是已平衡的(balanced),稱該等係數為平衡係數。
\sssc{(反應/化學)產率((Reaction/chemical) yield)}
\[\frac{\tx{實際產量}}{\tx{理論產量}}\]
\sssc{限量反應物(Limiting reactant)/限量試劑}
若使不斷向正反應方向反應,在反應中耗盡的反應物。
\sssc{原子(使用)效率/原子經濟(Atom economy)}
\[\frac{\tx{欲獲得的主要產物質量}}{\tx{所有產物總質量}}\]
\subsubsection{(化學)當量(Equivalent, eqiv, eq, Eq)}
\bit
\item\tb{當量}:當量 = 試劑式量 / 單位反應所需該試劑提供的反應物的粒子數
\item\tb{當量數}:當量數 = 質量 / 當量
\item\tb{當量濃度}:當量濃度 = 當量數 / 溶液公升數
\eit
\subsection{分析化學(Analytical chemistry)}
\sssc{分離過程(Separation process)}
將樣品轉化為兩種或多種不同混合物或純物質(稱組分(Fraction))的過程(如蒸餾、萃取、層析、再結晶等)。
\sssc{定性分析(Qualitative analysis)}
根據物質的化學或物理性質(例如化學反應性、溶解度、分子量、熔點、放射性(發射、吸收)、質譜、核半衰期等)對物質進行識別或分類的分析。
\sssc{定量分析(Quantitative analysis)}
確定或估算分析物的含量或濃度,並將其以適當單位的數值表示的分析。
\sssc{分子量的測定方法}
蒸氣密度測定法(與理想氣體偏差大者誤差大)、沸點上升法(分子量大者誤差大)、凝固點下降法(分子量大者誤差大)、滲透壓法(分子量大者亦適用)、質譜法等。
\end{document}