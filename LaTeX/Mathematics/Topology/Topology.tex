\documentclass[a4paper,12pt]{article}
\setcounter{secnumdepth}{5}
\setcounter{tocdepth}{3}
\input{/usr/share/latex-toolkit/template.tex}
\begin{document}
\title{Topology}
\author{沈威宇}
\date{\temtoday}
\titletocdoc
\section{Topology (拓樸學)}
\subsection{Topological space (拓樸空間)}
A topological space consists of a set $X$ and a topology $\mathcal{T}$ on it, denoted as \( (X, \mathcal{T}) \). Where the set \( X \) is the set of points in the space, and the topology \( \mathcal{T} \) is the set of subsets of \( X \) that satisfies:
\begin{enumerate}
\item \(\varnothing,X\in\mathcal{T} \)
\item Closed under arbitrary unions: \(\forall\mathcal{S}\subseteq\mathcal{T}:\,\bigcup_{A\in\mathcal{S}}A\in\mathcal{T}\)
\item Closed under finite intersection: \(\forall\mathcal{S}\subseteq\mathcal{T}\text{ s.t. }\abs{\mathcal{S}}<\infty:\,\bigcap_{A\in\mathcal{S}}A\in\mathcal{T}\)
\end{enumerate}
\subsection{Homeomorphism (同胚) and isomorphism (同構)}
Topological spaces ${X,\mathcal{T}_X}$ and ${Y,\mathcal{T}_Y}$ are called homeomorphic if there exists a mapping $f:\,X\rightarrow Y$ between them such that $f$ is bijective and continuous and $f^{-1}$ is continuous, written as \( X \cong Y \), and $f$ is called a homeomorphism between them.

In linear algebra, when $X$ and $Y$ are homeomorphic vector spaces and $f$ is a linear map, $X$ and $Y$ are also called isomorphic, and a homeomorphism between them is also called an isomorphism.
\subsection{Open set (開集)}
In a topological space \( (X, \,\mathcal{T}) \), an open subset \( O \subseteq X \) is defined to be
\[O\in \mathcal{T}.\]
\subsection{Open neighborhood (開鄰域)}
In a topological space $X$, a open neighborhood of a point $P\in X$ is any open subset $O\subseteq X$ such that $P\in O$.
\subsection{Neighborhood (鄰域)}
In a topological space, a subset $U$ is a neighborhood of a point $P$ if and only if there exists an open set $O$ such that $P\in O\subseteq U$.
\subsection{Limit point (極限點), cluster point, or accumulation point (集積點)}
Let $S$ be a subset of a topological space $X$. A point $x$ in $X$ is a limit point, cluster point, or accumulation point of the set $S$ if every neighborhood of $x$ contains at least one point of $S$ different from $x$ itself.
\subsection{Closure (閉包)}
The closure of a subset $S$ of points in a topological space consists of all points in $S$ together with all limit points of $S$.
\subsection{Closed set (閉集)}
A subset $A$ of a topological space $(X,\mathcal{T})$ is called closed if its complement $X\setminus A\in\mathcal{T}$.
\subsection{Borel set (博雷爾集)}
A Borel set $B$ is any set in a topological space $(X,\mathcal{T})$ that can be formed from open sets through the operations of countable union, countable intersection, and relative complement, that is:
\[\begin{aligned}
B\in &\{\bigcup_{O\in\mathcal{S}}O:\,\mathcal{S}\subseteq\mathcal{T},\abs{\mathcal{S}}<\infty\}\\
& \cup \{\bigcap_{O\in\mathcal{S}}O:\,\mathcal{S}\subseteq\mathcal{T},\abs{\mathcal{S}}<\infty\}\\
& \cup \{O\setminus P:\,O,P\in\mathcal{T}\}
\end{aligned}\]
\subsection{Filter}
A filter on a set $X$ is a family $\mathcal{B}$ of subsets of $X$ such that: 
\begin{enumerate}
\item $X\in\mathcal{B}$.
\item $\varnothing\in\mathcal{B}$.
\item $A\in\mathcal{B}\land B\in\mathcal{B}\implies A\cap B\in\mathcal{B}$.
\item $A\subseteq B\subseteq X\land A\in\mathcal{A}\in\mathcal{B}\implies B\in\mathcal{B}$.
\end{enumerate}
\subsection{Base or basis (基)}
Given a topological space $(X,\mathcal{T})$, a base (or basis) for the topology $\mathcal{T}$ (also called a base for $X$ if the topology is understood) is a family $\mathcal{B}\subseteq\mathcal{T}$ of open sets such that every open set of the topology can be represented as the union of some subfamily of $\mathcal{B}$.

The topology generated by a base $\mathcal{B}$, generally denoted by $\tau(\mathcal{B})$ can be defined to be follows: A subset $O\subseteq X$ is to be declared as open, if for all $x\in O$, there exists some $B\in\mathcal{B}$ such that $x\in B\subseteq O$.
\subsection{Prefilter or filter base}
$\mathcal{B}$ is called a prefilter if its upward closure $\uparrow\mathcal{B}$ is a filter.
\subsection{Connected space (連通空間)}
A topological space is disconnected if it can be expressed as the union of two disjoint non-empty open sets. Otherwise, it is connected.
\subsection{Connected set}
A subset $C\subseteq X$ of a topological space $(X,\tau)$ are said to be disconnected if
\[\exists A\,B\in\tau\text{ s.t. }A\cap B=\varnothing\land C\cap A\neq\varnothing\land C\cap B\neq\varnothing\land C\subseteq A\cup B.\]
Otherwise, it is said to be connected.
\subsection{Path-connected space (路徑連通空間)}
A space $X$ is path-connected if for any two points $x,y\in X$, there existes a continuous function $f\colon[0,1]\to X$ such that $f(0)=x$ and $f(y)=1$. Otherwise, it is not path-connected.

Path-connectedness implies connectedness. However, connectedness does not imply path-connectedness, a counterexample is the topologist' sine curve:
\[S = \left\{ (x, \sin(1/x)) \mid x \in (0,1] \right\} \cup \{(0,y) \mid -1 \leq y \leq 1\} \subset \mathbb{R}^2.\]
\subsection{Simply connected space or 1-connected space (單連通空間)}
A space is simply connected if it is path-connected and that for any two points $x,y\in X$ and any two continuous functions $f\colon[0,1]\to X$ and $g\colon[0,1]\to X$ such that $f(0)=g(0)=x$ and $f(1)=g(1)=y$, there exists a continuous function $F\colon[0,1]\times[0,1]\to X$ such that $F(x,0)=f(x)$ and $F(x,1)=g(x)$. Otherwise, it is not simply connected.
\subsection{Compact space (緊緻空間)}
A topological space $X$ is called compact if every open cover of $X$ has a finite subcover. That is, $X$ is compact if for every collection $C$ of open subsets of $X$ such that
\[X=\bigcup_{S\in C}S,\]
there is a finite subcollection $F\subseteq C$ such that
\[X=\bigcup _{S\in F}S.\]
\subsection{Hausdorff space, separated space or T2 space (郝斯多夫空間、分離空間或T2空間)}
Points $x$ and $y$ in a topological space $X$ can be separated by neighborhoods if there exists a neighborhood $U$ of $x$ and a neighborhood $V$ of $y$ such that $U$ and $V$ are disjointed, i.e., $U\cap V=\varnothing$.

$X$ is a Hausdorff space if any two distinct points in $X$ are separated by neighborhoods. This condition is the third separation axiom (after T0 and T1), which is why Hausdorff spaces are also called T2 spaces. The name separated space is also used.
\subsection{Metric space (度量空間或賦距空間)}
Metric space is an ordered pair $(M, d)$ where $M$ is a set and $d$ is a metric on M, i.e., a function $d:\,M\times M\to\mathbb{R}$ satisfying the following axioms for all points $x,y,z\in M$:
\begin{enumerate}
\item The distance from a point to itself is zero: $d(x,x)=0$.
\item (Positivity) The distance between two distinct points is always positive:$x\neq y\implies d(x,y)>0$.
\item (Symmetry) The distance from $x$ to $y$ is always the same as the distance from $y$ to $x$: $d(x,y)=d(y,x)$.
\item The triangle inequality: $d(x,z)\leq d(x,y)+d(y,z)$.
\end{enumerate}
\subsection{Cauchy sequence (柯西序列)}
Given a metric space $(x,d)$, a sequence of elements of $X$:
\[x_1,x_2,x_3,\ldots\]
is Cauchy if for every positive real number $\varepsilon$ there is a positive integer $N$ such that for all positive integers $m,n>N$:
\[d\left(x_m,x_n\right)<\varepsilon.\]
\subsection{Complete metric space (完備度量空間)/Cauchy space (柯西空間)}
A metric space $(x,d)$ is complete if every Cauchy sequence of points in $X$ has a limit that is also in $X$.
\subsection{Open ball (開球)}
In a metric space $(X,d)$, given a point $a$ and radius $r$, the open ball $B(a)_{<r}$ is defined to be:
\[B(a)_{<r}:=\left\{p\in X:\, d(a,p)<r\right\}.\]
\subsection{Closed ball (閉球)}
In a metric space $(X,d)$, given a point $a$ and radius $r$, the closed ball $B(a)_{\leq r}$ is defined to be:
\[B(a)_{\leq r}:=\left\{p\in X:\, d(a,p)\leq r\right\}.\]
\subsection{Topological field (拓樸域)}
A topological field is a topological space, such that addition, multiplication, the maps $a\mapsto -a$, and $a\mapsto a^{-1}$ are continuous maps with respect to the topology of the space.
\subsection{Ordered field (有序域)}
A field $(F,+,\cdot\,)$ together with a total order $\leq$ on $F$ is an ordered field if the order satisfies the following properties for all $a,b,c\in F$:
\begin{enumerate}
\item $a\leq b\implies a+c\leq b+c$.
\item $0\leq a\land 0\leq b\implies 0\leq a\cdot b$.
\end{enumerate}
As usual, we write $a<b$ for $a\leq b$ and $a\neq b$. The notations $b\geq a$ and $b>a$ stand for $a\leq b$ and $a<b$, respectively. Elements $a\in F$ with $a>0$ are called positive.
\end{document}