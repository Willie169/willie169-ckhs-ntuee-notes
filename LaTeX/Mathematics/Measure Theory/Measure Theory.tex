\documentclass[a4paper,12pt]{article}
\setcounter{secnumdepth}{5}
\setcounter{tocdepth}{3}
\input{/usr/share/latex-toolkit/template.tex}
\begin{document}
\title{Measure Theory}
\author{沈威宇}
\date{\temtoday}
\titletocdoc
\section{Measure Theory (測度論)}
\subsection{\text{\textsigma}-algebra}
Let $X$ be some set, and let $P(X)$ represent its power set. Then a subset $\Sigma \subseteq P(X)$ is called a \text{\textsigma}-algebra if and only if it satisfies the following three properties:
\begin{enumerate}
\item $X\in\Sigma$.
\item Closed under complementation: $A\in\Sigma\implies X\setminus A\in\Sigma$.
\item Closed under countable unions: \(\forall\mathcal{S}\subseteq\Sigma\text{ s.t. }\abs{\mathcal{S}}<\infty:\,\bigcup_{A\in\mathcal{S}}A\in\Sigma\).
\end{enumerate}
\subsection{Measurable space (可測空間)}
Consider a set $X$ and a \text{\textsigma}-algebra $\Sigma$ on $X$. Then the tuple $(X,\Sigma)$ is called a measurable space.
\subsection{Measure (測度)}
Let $X$ be a set and $\Sigma$ be a \text{\textsigma}-algebra over $X$. A set function $\mu$ from $\Sigma$ to the extended real number line, defined to be $[-\infty,\infty]$ or $\mathbb{R}\cup\{-\infty,\infty\}$, is called a measure or positive measure on $(X,\Sigma)$ if the following conditions hold:
\begin{enumerate}
\item Non-negativity: \[\forall E\in\Sigma:\,\mu(E)\geq 0\]
\item $\mu (\varnothing )=0$.
\item Countable additivity (or \text{\textsigma}-additivity): For all countable collection $\{E_k\}_{k=1}^\infty$ of pairwise disjoint sets in $\Sigma$ (i.e. $\forall i\neq j:\,E_i\cap E_j=\varnothing$),
\[\mu\left(\bigcup_{k=1}^\infty E_k\right)=\sum_{k=1}^\infty\mu(E_k)\]
\end{enumerate}
If the condition of non-negativity is dropped, then $\mu$ is called a signed measure.
\subsection{\text{\textsigma}-finite measure (\text{\textsigma}有限測度)}
Let $(X,\Sigma)$ be a measurable space and $\mu$ be a positive measure or signed measure on it. $\mu$ is called a \text{\textsigma}-finite measure, if:
\[\exists\{A_n\}_{n\in\mathbb{N}}\subseteq\Sigma\text{ s.t. }\forall n\in\mathbb{N}:\,\mu(A_n)<\infty\land\bigcup_{n\in\mathbb{N}}A_n=X.\]
If $\mu$ is a \text{\textsigma}-finite measure, the measure space $(X,\Sigma,\mu)$ is called a \text{\textsigma}-finite measure space.
\subsection{Measure space (測度空間)}
A measure space is a triple $(X,\Sigma,\mu)$, where:
\begin{enumerate}
\item $X$ is a set.
\item $\Sigma$ is a \text{\textsigma}-algebra on the set $X$. 
\item $\mu$ is a measure on $(X,\Sigma)$.
\end{enumerate}
\subsection{Ring of sets}
We call a family $\mathcal{R}$ of subsets of $\Omega$ a ring of sets if it has the following properties:
\begin{enumerate}
\item $\varnothing\in\mathcal{R}$.
\item Closed under pairwise unions: $\forall A,B\in\mathcal{R}:\,A\cup B\in\mathcal{R}$.
\item Closed under relative complements: $\forall A,B\in\mathcal{R}:\,A\setminus B\in\mathcal{R}$.
\end{enumerate}
\subsection{Pre-measure (前測度)}
Let $\mathcal{R}$ be a ring of subsets of a fixed set $X$. A set function $\mu_0$ from $\mathcal{R}$ to the extended real number line is called a pre-measure on $(X,\mathcal{R})$ if the following conditions hold:
\begin{enumerate}
\item Non-negativity: \[\forall E\in\mathcal{R}:\,\mu_0(E)\geq 0\]
\item $\mu_0(\varnothing )=0$.
\item Countable additivity (or \text{\textsigma}-additivity): For all countable collection $\{E_k\}_{k=1}^\infty$ of pairwise disjoint sets in $\mathcal{R}$ (i.e. $\forall i\neq j:\,E_i\cap E_j=\varnothing$),
\[\mu_0\left(\bigcup_{k=1}^\infty E_k\right)=\sum_{k=1}^\infty\mu_0(E_k)\]
\end{enumerate}
\subsection{Outer measure or exterior measure}
Given a set $X$, let $2^X$ denote the collection of all subsets of $X$, including the empty set $\varnothing$. An outer measure on $X$ is a set function $\mu:\,2^X\to [0,\infty]$ such that
\begin{enumerate}
\item $\mu(\varnothing )=0$.
\item Countably subadditive: 
\[\forall A\subseteq X,\{B_i\subseteq X\}_{i=1}^\infty:\,A\subseteq\bigcup_{i=1}^\infty B_i\implies\mu(A)\leq\sum_{i=1}^\infty\mu(B_i).\]
\end{enumerate}
\subsection{Carathéodory's extension theorem}
Let $\mathcal{R}$ be a ring of sets on $X$, let $\mu:\,\mathcal{R}\to [0,+\infty]$ be a pre-measure on $\mathcal{R}$, and let $\sigma(\mathcal{R})$ be a \text{\textsigma}-algebra generated by $\mathcal{R}$.

The Carathéodory's extension theorem states that there exists a measure $\mu^\prime:\,\sigma(\mathcal{R})\to [0,+\infty]$ such that $\mu^\prime$ is an extension of $\mu$, that is, 
\[\mu^\prime\big\vert_{\mathcal{R}}=\mu.\]
Moreover, if $\mu$ is \text{\textsigma}-finite, then the extension $\mu^\prime$ is unique and also \text{\textsigma}-finite.
\subsection{Lebesgue measure (勒貝格測度)}
For any interval $I=[a,b]$ or $I=(a,b)$ that is a subset of $\mathbb{R}$, let $\ell (I)=b-a$ denote its length. For any subset $E\subseteq\mathbb{R}$, the Lebesgue outer measure $\lambda(E)$ is defined to be
\[\lambda(E)=\inf\left\{\sum_{k=1}^\infty\ell(I_k):\,(I_k)_{k\in\mathbb{N}}\text{ is a sequence of intervals with }E\subseteq\bigcup_{k=1}^\infty I_k\right\}.\]
The above definition can be generalised to higher dimensions as follows. For any $n$-dimensional rectangular cuboid, that is, a cuboid with rectangular faces in which all of its dihedral angles are right angles, $C$, which is a Cartesian product $C=\prod_{i=1}^nI_i$ of intervals, we define its Lebesgue outer measure $\lambda(C)$ to be
\[\lambda(C):=\prod_{i=1}^n\ell(I_i).\]
For any subset $E\subseteq\mathbb{R}^n$, we define its Lebesgue outer measure $\lambda(E)$ to be
\[\lambda(E)=\inf\left\{\sum _{k=1}^\infty\lambda(C_k):\,(C_k)_{k\in\mathbb{N}}\text{ is a sequence of products of intervals with }E\subseteq\bigcup_{k=1}^\infty C_k\right\}.\]
We say a set $E\in\mathbb{R}^n$ satisfies the Carathéodory criterion if 
\[\forall A\subseteq \mathbb {R}:\,\lambda(A)=\lambda(A\cap E)+\lambda(A\cap (\mathbb{R}^n\setminus E)).\]
The sets $E\subseteq\mathbb{R}^n$ that satisfy the Carathéodory criterion are said to be Lebesgue-measurable, with its Lebesgue measure being defined as its Lebesgue outer measure. The set of all such $E$ forms a \text{\textsigma}-algebra.

A set $E\subseteq\mathbb{R}^n$ that does not satisfy the Carathéodory criterion is not Lebesgue-measurable. ZFC proves that such sets do exist.
\subsection{Hausdorff measure (郝斯多夫測度)}
Let $X,p$ be a metric space. For any subset $U\subseteq X$, let $\operatorname {diam} U$ denote its diameter, that is
\[\operatorname {diam} (U):=\sup\{p (x,y):\,x,y\in U\},\quad \operatorname {diam} (\emptyset) :=0.\]
Let $S$ be any subset of $X$, and $\delta >0$ a real number. Define
\[H_\delta^d(S)=\inf\left\{\sum_{i=1}^\infty \left(\operatorname{diam}(U_i)\right)^d:\, S\subseteq\bigcup_{i=1}^\infty U_i\land\operatorname{diam}(U_i)<\delta \right\}.\]
Note that $H_\delta^d(S)$ is monotone nonincreasing in $\delta$ since the larger $\delta$ is, the more collections of sets are permitted, making the infimum not larger. Thus, 
$\lim_{\delta\to 0}H_\delta^d(S)$ exists but may be infinite. Let
\[H^d(S):=\lim_{\delta \to 0}H_\delta^d(S).\]
It can be seen that $H^d(S)$ is an outer measure, or more precisely, a metric outer measure. By Carathéodory's extension theorem, its restriction to the \text{\textsigma}-algebra of Carathéodory-measurable sets is a measure. It is called the $d$-dimensional Hausdorff measure of $S$. Due to the metric outer measure property, all Borel subsets of $X$ are $H^d$ measurable.
\subsection{Radon measure (拉東測度)}
Let $\mu$ be a measure on a \text{\textsigma}-algebra of Borel sets of a Hausdorff topological space $X$.
\begin{itemize}
\item The measure $\mu$ is called inner regular or tight if, for every open set $U$, $\mu(U)$ equals the supremum of $\mu(K)$ over all compact subsets $K$ of $U$.
\item The measure $\mu$ is called outer regular if, for every Borel set $B$, $\mu(B)$ equals the infimum of $\mu(U)$ over all open sets $U$ that contain $B$.
\item The measure $\mu$ is called locally finite if every point of $X$ has a neighborhood $U$ for which $\mu(U)$ is finite.
\end{itemize}
The measure $\mu$ is called a Radon measure if it is inner regular and locally finite. In many situations, such as finite measures on locally compact spaces, this also implies outer regularity.
\end{document}