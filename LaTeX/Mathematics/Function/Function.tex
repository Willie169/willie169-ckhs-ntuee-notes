\documentclass[a4paper,12pt]{article}
\setcounter{secnumdepth}{5}
\setcounter{tocdepth}{3}
\input{/usr/share/latex-toolkit/template.tex}
\begin{document}
\title{Function}
\author{沈威宇}
\date{\temtoday}
\titletocdoc
\section{Function (函數)}
\subsection{Definition and notation}
A function is formed by three sets, the domain (定義域) $X$, the codomain (對應域) $Y$, and the graph $R$ that satisfy the three following conditions:
\[R\subseteq \{(x,y)\mid x\in X,y\in Y\}\]
\[\forall x\in X,\exists y\in Y,\left(x,y\right)\in R \]
\[ (x,y)\in R\land (x,z)\in R\implies y=z\]

A function $f$ that satisfies the above is denoted as:
\[f\colon X\to Y.\]
in which the domain $X$ is also denoted as $D_f$.

The range (值域), denoted as $R_f$ or $f(X)$, is defined as:
\[\{ y \mid \exists x \in X, (x, y) \in R \}.\]

If $x\in X$ and $(x, y) \in R$, we write $y = f(x)$, in which $f(x)$ is called the image of $x$ under $f$, $x$ is called the independent variable (自變數/獨立變數), and $y$ is called the dependent variable (應變數/依賴變數); and the function $f$ is also denoted as:
\[f \colon X \to Y;\, x \mapsto y.\]
\ssc{Properties}
Consider a function $f$:
\[f\colon X\to Y;\,x\mapsto y.\]
\sssc{Injection (單射)/Injective function/One-to-one (一對一) function}
\[\forall a,b\in X\text{\ s.t.\ } f(a)=f(b)\colon a=b\]
\subsection{Many-to-one (多對一) function}
\[\exists a\neq b\in X\colon f(a)=f(b)\]
\sssc{Surjection (滿射/蓋射)/Surjective function/Onto function}
\[f(X)=Y\]
\sssc{Bijection (對射)/Bijective function/One-to-one (一對一) function/One-to-one correspondence (一一對應)}
Injective and surjective function.
\ssc{Increasing and Decreasing}
Consider a function $f$:
\[f\colon X\to Y;\,x\mapsto y,\]
such that $Y$ is a totally ordered set.
\sssc{(Monotone) Increasing ((單調)遞增)/Non-Decreasing (非遞減) function}
$f$ is increasing on $I\subseteq X$ if and only if
\[\forall a,b\in I\colon a<b\implies f(a)\leq f(b).\]
$f$ is increasing if and only if
\[\forall a,b\in X\colon a<b\implies f(a)\leq f(b).\]
\sssc{Strictly increasing (嚴格遞增) function}
$f$ is strictly increasing on $I\subseteq X$ if and only if
\[\forall a,b\in I\colon a<b\implies f(a)<f(b).\]
$f$ is strictly increasing if and only if
\[\forall a,b\in X\colon a<b\implies f(a)<f(b).\]
\sssc{(Monotone) Decreasing ((單調)遞減)/Non-Increasing (非遞增) function}
$f$ is decreasing on $I\subseteq X$ if and only if
\[\forall a,b\in I\colon a<b\implies f(a)\geq f(b).\]
$f$ is decreasing if and only if
\[\forall a,b\in X\colon a<b\implies f(a)\geq f(b).\]
\sssc{Strictly decreasing (嚴格遞減) function}
$f$ is strictly decreasing on $I\subseteq X$ if and only if
\[\forall a,b\in I\colon a<b\implies f(a)>f(b).\]
$f$ is strictly decreasing if and only if
\[\forall a,b\in X\colon a<b\implies f(a)>f(b).\]
\sssc{Monotone (單調) function}
$f$ is monotone on $I\subseteq X$ if and only if it is either monotone increasing or monotone decreasing on $I$.

$f$ is monotone if and only if it is either monotone increasing or monotone decreasing.
\ssc{Transformation}
\sssc{Translation (平移)}
For any function $f\colon\mathbb{R}\to\mathbb{R}$, shifting $y=f(x)$ right by $h$ units and up by $k$ units on the $xy$ coordinate plane yields $y=f(x-h)+k$.
\sssc{Scaling (伸縮/縮放/拉伸)}
For any function $f\colon\mathbb{R}\to\mathbb{R}$, on the $xy$ coordinate plane, expand $y=f(x)$ vertically by $a$ times the original value with the $x$ axis as the reference line, and expand $y=af\qty(\frac{x}{b})$ horizontally by $b$ times the original value with the $y$ axis as the reference line, to obtain $y=af\qty(\frac{x}{b})$.
\ssc{Common ways to define functions}
\sssc{Function composition (函數合成)}
For two functions $f\colon X\to Y$ and $g\colon V\to W$ such that $g(V)\subseteq X$, the composition of them, denoted as $(f\circ g)$, is defined as:
\[(f \circ g)\colon V\to Y;\,x\mapsto = f(g(x))\]
\sssc{Inverse function (反函數)}
For a bijective function $f\colon X\to Y$, the inverse of it, denoted as $f^{-1}$, is defined as:
\[f^{-1}\colon Y\to X;\,f(x)\mapsto x\]
\sssc{Power notation}
For a bijective function $f\colon X\to X$, $f^0$ is defined by:
\[f^0\colon X\to X;\,x\mapsto x,\]
$f^n(x)$ for any $n\in\mathbb{N}$ is defined by:
\[f^n\colon X\to X;\,x\mapsto f\left(f^{n-1}(x)\right),\]
and $f^{-n}(x)$ for any $n\in\mathbb{N}$ is defined by:
\[f^{-n}\colon X\to X;\,x\mapsto f^{-1}\left(f^{-n+1}(x)\right).\]
\sssc{Piecewise function (分段函數)}
A piecewise function is a function defined in the form:
\[f(x) =
\begin{cases}
f_1(x), & \quad x\in A_1, \\
f_2(x), & \quad x\in A_2, \\
\vdots \\
f_n(x), & \quad x \in A_n
\end{cases},\]
where
\[\bigcup_{i=1}^nA_i=D_f\land\forall i\neq j\land i,j\in\mathbb{N}\land i,j\leq n\colon A_i\cap A_j=\varnothing.\]
\end{document}