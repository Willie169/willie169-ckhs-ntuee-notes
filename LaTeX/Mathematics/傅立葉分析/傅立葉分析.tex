\documentclass[a4paper,12pt]{article}
\setcounter{secnumdepth}{5}
\setcounter{tocdepth}{3}
\newcounter{ZhRenew}
\setcounter{ZhRenew}{1}
\newcounter{SectionLanguage}
\setcounter{SectionLanguage}{1}
\input{/usr/share/latex-toolkit/template.tex}
\begin{document}
\title{傅立葉分析}
\author{沈威宇}
\date{\temtoday}
\titletocdoc
\section{傅立葉分析(Fourier Analysis)}
傅立葉分析指將函數分解為多個正弦函數之和的分析。其分解過程稱為傅立葉轉換(或變換)(Fourier transform)。
\subsection{傅立葉級數(Fourier series)}
\[\qty(f(x) = \sum_{n=-\infty}^{\infty} F_n e^{inx})\iff\qty(F_n = \frac{1}{2\pi} \int_{-\pi}^{\pi} f(x) e^{-inx} \, \mathrm{d}x)\]
其中 \( F_n \) 為複振幅。\\
對於實函數可以寫成:
\[ f(x)={\frac {a_{0}}{2}}+\sum _{n=1}^{\infty }\left(a_{n}\cos(nx)+b_{n}\sin(nx)\right)\]
其中$a_n$和$b_n$是實頻率分量的振幅。
\subsection{傅立葉(積分)轉換與逆轉換}
\[ \hat{f}(\mathbf{k}) = \mathcal{F}(f)(\mathbf{k}) = \int_{\mathbb{R}^d} f(\mathbf{x}) e^{-2\pi i \mathbf{k} \cdot \mathbf{x}} \, \mathrm{d}\mathbf{x} \]
其中:\( \mathbf{x} \) 是實空間中的變量,相當空間中的位置向量或時間變量,為一個 $d$-維向量;\( \mathbf{k} \) 是頻率空間中的變量,相當傅立葉級數中於每個正弦波的波數,也是一個 $d$-維向量;$f(\mathbf{x})$符合:
\[ \abs{\int_{\mathbb{R}^d} \qty|f(\mathbf{x})| \, \mathrm{d}\mathbf{x}} < \infty \]
\raggedright\textbf{證明:}\\
\raggedright\textit{Statement.}\\
For all \( f\qty(\mathbf{x}) \) such that \( \abs{\int_{\mathbb{R}^d} f\qty(\mathbf{x})\, \mathrm{d}\mathbf{x}}<\infty \), the following holds true:
\[\qty(F\qty(\mathbf{k}) = \int_{\mathbb{R}^d} f\qty(\mathbf{x}) e^{-i \mathbf{k}\cdot\mathbf{x}} \, \mathrm{d}\mathbf{x})\iff\qty(f(\mathbf{x}) = \frac{1}{2\pi} \int_{\mathbb{R}^d} F(\mathbf{k}) e^{i \mathbf{k} \mathbf{x}} \, \mathrm{d}\mathbf{k})\]
\begin{proof}\mbox{}\\
We begin by assuming that \( F(\mathbf{k}) = \int_{\mathbb{R}^d} f(\mathbf{x}) e^{-i \mathbf{k} \cdot \mathbf{x}} \, \mathrm{d}\mathbf{x} \).
First, we show that the inverse Fourier transform of \( F(\mathbf{k}) \) recovers \( f(\mathbf{x}) \):
\[\begin{aligned}
f(\mathbf{x}) &= \frac{1}{2\pi} \int_{\mathbb{R}^d} F(\mathbf{k}) e^{i \mathbf{k} \cdot \mathbf{x}} \, \mathrm{d}\mathbf{k} \\
&= \frac{1}{2\pi} \int_{\mathbb{R}^d} \left( \int_{\mathbb{R}^d} f(\mathbf{x}') e^{-i \mathbf{k} \cdot \mathbf{x}'} \, \mathrm{d}\mathbf{x}' \right) e^{i \mathbf{k} \cdot \mathbf{x}} \, \mathrm{d}\mathbf{k}.
\end{aligned}\]
By changing the order of integration, we get:
\[\begin{aligned}
f(\mathbf{x}) &= \frac{1}{2\pi} \int_{\mathbb{R}^d} f(\mathbf{x}') \left( \int_{\mathbb{R}^d} e^{-i \mathbf{k} \cdot \mathbf{x}'} e^{i \mathbf{k} \cdot \mathbf{x}} \, \mathrm{d}\mathbf{k} \right) \, \mathrm{d}\mathbf{x}' \\
&= \frac{1}{2\pi} \int_{\mathbb{R}^d} f(\mathbf{x}') \left( \int_{\mathbb{R}^d} e^{i \mathbf{k} \cdot (\mathbf{x} - \mathbf{x}')} \, \mathrm{d}\mathbf{k} \right) \, \mathrm{d}\mathbf{x}'.
\end{aligned}\]
The integral inside is a known result of the Dirac delta function:
\[
\int_{\mathbb{R}^d} e^{i \mathbf{k} \cdot (\mathbf{x} - \mathbf{x}')} \, \mathrm{d}\mathbf{k} = 2\pi \delta(\mathbf{x} - \mathbf{x}').
\]
Therefore,
\[\begin{aligned}
f(\mathbf{x}) &= \frac{1}{2\pi} \int_{\mathbb{R}^d} f(\mathbf{x}') \cdot 2\pi \delta(\mathbf{x} - \mathbf{x}') \, \mathrm{d}\mathbf{x}' \\
&= \int_{\mathbb{R}^d} f(\mathbf{x}') \delta(\mathbf{x} - \mathbf{x}') \, \mathrm{d}\mathbf{x}' \\
&= f(\mathbf{x}).
\end{aligned}\]
This completes the proof that the inverse Fourier transform of \( F(\mathbf{k}) \) gives back \( f(\mathbf{x}) \), thus proving the statement.
\end{proof}
\end{document}