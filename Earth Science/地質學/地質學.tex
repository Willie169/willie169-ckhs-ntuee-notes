\documentclass[a4paper,12pt]{report}
\setcounter{secnumdepth}{5}
\setcounter{tocdepth}{3}
\newcounter{ZhRenew}
\setcounter{ZhRenew}{1}
\newcounter{SectionLanguage}
\setcounter{SectionLanguage}{1}
\input{/usr/share/latex-toolkit/template.tex}
\begin{document}
\title{地質學}
\author{沈威宇}
\date{\temtoday}
\titletocdoc
\chapter{地質學(Geology)}
\section{地質學(Geology)}
\ssc{地質變化理論}
\sssc{均變說(Uniformitarianism)}
十九世紀由地質學之父詹姆斯·赫登(James Hutton)提出,後來由查爾斯·李爾(Charles Lyell)進一步發展。其基本觀點可以概括為:
\begin{itemize}
\item \textbf{現在是過去的鍵鑰}:均變說假設自然界的運行機制在歷史上都是一致的,即地球上的自然過程和定律在漫長的時間尺度上基本不變。因此,今天我們觀察到的自然現象和過程可以用來解釋過去發生在地球上的事件。
\item \textbf{緩慢、逐漸的變化}:均變說反對一些當時流行的災變主義觀點,認為地球的地質變化是由緩慢、逐漸的自然過程引起的,而非突發和劇烈的災難性事件。
\item \textbf{地質過程的穩定性}:均變說強調地球表面的地質過程(如風化、沉積、岩石形成)在不同時間和地點上是持續穩定且相似的,不會有顯著的差異。
\end{itemize}
\sssc{災變說(Catastrophism)}
認為地球的地質特徵主要是由突發的劇變事件所造成的,例如大規模的火山爆發、地震和隕石撞擊,造成生物的大規模滅絕與地質結構的改變等。這一理論的早期版本由法國地質學家喬治·居維葉(Georges Cuvier)提出。
\sssc{現代地質學}
現代地質學綜合了均變說和災變說的觀點,認為地球的地質歷史是由長期的漸進變化和偶爾的劇變事件共同塑造的。例如,板塊構造學說解釋了地殼運動的長期漸進過程,而隕石撞擊學說解釋了恐龍滅絕等劇變事件。
\subsection{相對地質年代(Relative age)}
將地質事件、岩石層次等安排在時間上順序的方法,但並不給出具體的數字或年代。
\subsubsection{岩層相對關係判斷相對地質年代的定律}
\begin{itemize}
\item \textbf{超位定律(Principle of Superposition)/疊置定律}:年代較老的岩層位於年代較晚的岩層之下。 
\item \textbf{原始水平定律(Principle of Original Horizontality)}:岩層在形成時是水平或接近水平的。 
\item \textbf{橫向連續性定律(Principle of Lateral Continuity)}:岩層在形成時會在一定範圍內保持連續性。 
\item \textbf{切割關係定律(Principle of Cross-Cutting Relationships)/截切定律}:侵入或切割其他岩石的地質體是比被侵入或切割的岩石還要年輕。 
\item \textbf{包裹體定律(Principle of Inclusions)}:包含在另一岩石中的碎片或礦物,比含有它的岩石還要年老。
\end{itemize}
\subsubsection{標準化石(Index Fossil)}
指示沉積年代的化石。18世紀提出。地質學家使用標準化石來識別和對比不同地層的相對年代,這是基於假設在同一地理區域內,相同標準化石的存在代表相同的地質時期。它們通常具有以下特徵:
\begin{itemize}
\item \textbf{易於辨認}:標準化石的形態或特徵通常易於辨認,並且在不同的地區有相似的外觀。 
\item \textbf{廣泛分布}:它們在廣泛的地理區域內有穩定和廣泛的分布。 
\item \textbf{生存期限短}:它們存在的時間範圍相對短暫,通常是幾百萬年到數千萬年。
\end{itemize}
不同地質年代的標準化石:
\begin{itemize}
\item \textbf{古生代}:三葉蟲、腕足動物、牙形刺。 
\item \textbf{中生代}:頭足綱菊石亞綱、恐龍。 
\item \textbf{新生代}:有孔蟲的鈣質超微化石、哺乳動物。
\end{itemize}
\subsubsection{指相化石(Guide Fossil)}
指示沉積環境的化石。化石除了具有關聯特定年代地層的功能外,有些生物對於生存環境的條件較為敏感也可以利用其生存條件的侷限性,推知當時環境的特性。例如珊瑚通常生長在溫暖淺海的乾淨海域,就是一種良好的指相化石,可以得知該岩層沉積時的環境應該符合珊瑚生長的條件。
\subsection{絕對地質年代(Absolute age)}
用具體的時間單位來測量地質事件、岩石層次等的年齡。
\subsubsection{放射性元素定年法(Radiative dating)}
岩石形成時放射性元素濃度與環境相同,而後衰變;生物生存時因與環境交換故與環境相同,而後衰變,故可以測定母元素與子元素的比例定年,但受成分置換或風化作用可能使元素含量偏差。

19世紀末發現放射性元素,20世紀初開始以放射性元素衰變作為定年的工具,可以用來測量標本的絕對地質年代。

常用於定年的放射性元素及適用地質材料:
\bctf\begin{tabular}{|c|c|c|c|c|c|}
\hline
母元素 & 子元素 & 衰變形式(不計 γ 衰變) & 半衰期(年) & 適用定年範圍(年) & 適用地質材料\\\hline
鈾-238 & 鉛-206 & 經多個元素的衰變鏈 & \scinote{4.5}{9} & >\scinote{1.0}{7} & 酸性火成岩,如鋯石 \\\hline
鉀-40 & 氬-40 & β$^+$ & \scinote{1.25}{9} & >\scinote{5.0}{5} & 酸或中性火成岩,如鉀長石、雲母 \\\hline
銣-87 & 鍶-87 & β$^-$ & \scinote{4.97}{10} & >\scinote{6.5}{7} & 酸或中性火成岩,如雲母、角閃石 \\\hline
碳-14 & 氮-14 & β$^-$ & 5730 & 100-60000 & 沉積岩,如枯木、骨頭、貝殼 \\\hline
\etbr\efct
\subsubsection{地質年代表(Geologic time scale, GTS)}
依國際地層委員會(ICS)以宙/元(eon)、代(era)、紀(period)、世(epoch)、期(age)表示,其中顯生宙/元前通稱前寒武紀或隱生宙/元。
以下年代以遠到近排列,層級以大到小排列,單位百萬年前:
\begin{itemize}
\item \textbf{冥古宙}:4600(地球形成)-4000(後期隕石轟炸結束,最古老的遺留至今的沉積岩形成可知海洋形成)
\item \textbf{太古宙}:4000-2500(冰室開始)
\item \textbf{元古宙}:2500-541(溫室開始,寒武紀大爆發)
\item \textbf{顯生宙}:541-0
\begin{itemize}
\item \textbf{古生代}:541-252
\begin{itemize}
\item \textbf{寒武紀}:541-485
\item \textbf{奧陶紀}:485-444(第一次大滅絕,臭氧層遭超新星爆炸摧毀,冰室開始)
\item \textbf{志留紀}:444-419(溫室結束)
\item \textbf{泥盆紀}:419-359(第二次大滅絕,主要為海洋生物滅絕,冰室開始)
\item \textbf{石炭紀}:359-299(溫室開始)
\item \textbf{二疊紀}:299-252(第三次大滅絕,盤古大陸形成)
\end{itemize}
\item \textbf{中生代}:252-66
\begin{itemize}
\item \textbf{三疊紀}:252-201(第四次大滅絕)
\item \textbf{侏羅紀}:201-145
\item \textbf{白堊紀}:145-66(第五次大滅絕,隕石撞擊)
\end{itemize}
\item \textbf{新生代}:66-0
\begin{itemize}
\item \textbf{古近紀}:66-23
\begin{itemize}
\item \textbf{古新世}:66-56
\item \textbf{始新世}:56-33.9(冰室開始)
\item \textbf{漸新世}:33.9-23
\end{itemize}
\item \textbf{新近紀}:23-0
\begin{itemize}
\item \textbf{中新世}:23-5.3
\item \textbf{上新世}:5.3-2.6(巧人出現,第四紀冰河時期開始)
\end{itemize}
\item \textbf{第四紀}:2.6-0
\begin{itemize}
\item \textbf{更新世}:2.6-0.012(新仙女木期結束)
\item \textbf{全新世}:0.012-0
\end{itemize}
\end{itemize}
\end{itemize}
\end{itemize}
\section{地震(Earthquake)}
\subsection{彈性回跳理論(Elastic-rebound theory)}
20世紀初,科學家根據1906年舊金山地震的地表變形狀況提出,用於解釋地震成因。指出地震是由於地殼中的岩層受力、變形,應變能量積累到一定程度後,岩石突然斷裂並釋放能量,從而引起地震波的產生。這一過程就像折壓塑膠尺彎曲超過某一限度突然斷裂並回彈。
\subsection{地震波}
分為實體波,即能在固體地球內部傳遞的波,與表面波,即不能在固體地球內部傳遞的波。
\subsubsection{P波(Primary Wave)}
P波屬於體波,是縱波(疏密波),又稱初達波,是地震波中速度最快的波,首先到達地表。P波傳播時,地震波的振動方向與波的前進方向一致,類似於聲波在空氣中傳播。

物理上,在一等向性介質中,P波傳播速度的公式為:
$$\displaystyle v_{p}={\sqrt {\frac {K+{\frac {4}{3}}G}{\rho }}}={\sqrt {\frac {(1-v)E}{(1+v)(1-2v)\rho }}}$$
其中$\displaystyle K$是體積模量,$\displaystyle G$是剪下模量,$\displaystyle \rho $是介質密度,$\displaystyle E$是楊氏係數,$\displaystyle v$是蒲松比。
\subsubsection{S波(Secondary Wave)}
S波屬於體波,是橫波(簡切波),傳播速度較P波慢,P波波速約為其1.73倍。S波的振動方向與波的前進方向垂直,因此只能在固體中傳播,無法在液體或氣體中傳播。

物理上,在一等向性介質中,S波傳播速度的公式為:
$$\displaystyle v_{s}={\sqrt {\frac {G}{\rho }}}={\sqrt {\frac {E}{2(1+v)\rho }}}$$
\subsubsection{表面波}
表面波能量集中在地表或近地表區域,有低頻率、高振幅、頻散(Dispersion)特性,衰減較慢,對建物損害較大。
\begin{itemize}
\item 雷利波(Rayleigh Wave)/地滾波:雷利波的波形像海浪一樣,地表的質點在垂直面內沿著橢圓軌道運動,是P波和S波干涉而成。
\item 洛夫波(Love Wave):洛夫波的振動方向與波的前進方向垂直,且在地表平行於地表運動,是S波之間干涉而成。
\end{itemize}
\subsubsection{地震波轉性}
由地球的內部並不是均質,因此當介質有所差異時,地震波的性質和行徑就有可能改變,稱為轉性。例如,P波在穿越液-固界面時可能會轉變為S波。
\subsection{地震觀測}
\subsubsection{地震儀}
地震儀的基本構造由重錘、紀錄設備和彈簧三者組成。重錘在地震儀中扮演的角色就是不會搖晃的參考點,這是因為連接在彈簧上的重錘很重,在地震中會因為慣性作用而保持幾乎不動。記錄設備可以是電子感應線圈或是傳統的滾紙 ,固定在地震儀的支架上,隨大地搖晃,並隨時紀錄搖晃過程中與重錘的相對位置。彈簧用於懸吊或支撐重錘,確保他能夠不受搖晃地面的干擾。一組地震儀通常可以偵測三個方向的震動情形。
\subsubsection{發生位置}
\begin{itemize}
\item 震源(Hypocenter):地下岩層斷裂錯動的發生地震之處,宏觀上是地震能量大量釋放之處,微觀上是首先發生地震波的地方,以地表位置與地震深度表示。
\item 震央(Epicenter):中國大陸稱震中,指地震發生時,地震震源向上垂直投影到地面的位置。
\item 震源深度:指震源到地面的鉛直距離。可根據震源深度不同分類地震:
\begin{itemize}
\item 極淺源地震:震源深度在0-30公里的地震
\item 淺源地震:震源深度在30-70公里的地震。發生頻率最多,對人類影響最大,在地球陸地的地震,95\%以上都是淺源地震。
\item 中源地震:震源深度在70-300公里的地震。90\%以上的中源地震發生在環太平洋地震帶上,一般不會造成災害。
\item 深源地震:震源深度超過300公里的地震。深源地震多發生在太平洋一帶的深海溝附近,一般也不會造成災害。
\end{itemize}
\end{itemize}
\subsubsection{主震與餘震}
主震是地震序列中規模最大的地震,若最大者有多個,則先發生者稱為主震。 餘震(Aftershock)是主震後發生的一連串地震。
\subsubsection{地震震度與規模}
\begin{itemize}
\item 震度(Intensity):描述地震在特定地點的影響程度和破壞力,同一地震在不同地點可能測得不同的震度。
\item 地震規模(Magnitude):衡量的是地震本身釋放出的能量大小,與震源深度和地震波振幅有關。規模是一個定量的測量,不隨地點變化而變化。無單位。
\end{itemize}
\sssc{中華民國中央氣象局地震震度分級(2020年起)}
單位:級。

範圍欄對於4級以下為最大地動加速度 PGA,單位 cm/s\textsuperscript{2};對於5級以上為最大地動速度 PGV,單位 cm/s。

屋內情形係以低樓層為例。
\begin{center}
\begin{table}[H]
\centering
\begin{tabular}{|p{0.06\textwidth}|p{0.04\textwidth}|p{0.21\textwidth}|p{0.27\textwidth}|p{0.3\textwidth}|}
\hline
\textbf{震度分級} & \textbf{範圍} & \textbf{人的感受} & \textbf{屋內情形} & \textbf{屋外情形} \\
\hline
0級,無感 & 0-0.8 & 人無感覺。 & 無。 & 無。 \\
\hline
1級,微震 & 0.8-2.5 & 人靜止或位於高樓層時可感覺微小搖晃。 & 無。 & 無。 \\
\hline
2級,輕震 & 2.5-8 & 大多數的人可感到搖晃,睡眠中的人有部分會醒來。 & 電燈等懸掛物有小搖晃。 & 靜止的汽車輕輕搖晃,類似卡車經過,但歷時很短。 \\
\hline
3級,弱震 & 8-25 & 幾乎所有的人都感覺搖晃,有的人會有恐懼感。 & 房屋震動,碗盤門窗發出聲音,懸掛物搖擺。 & 靜止的汽車明顯搖動,電線略有搖晃。 \\
\hline
4級,中震 & 25-80 & 有相當程度的恐懼感,部分的人會尋求躲避的地方,睡眠中的人幾乎都會驚醒。 & 房屋搖動甚烈,少數未固定物品可能傾倒掉落,少數傢俱移動,可能有輕微災害。 & 電線明顯搖晃,少數建築物牆磚可能剝落,小範圍山區可能發生落石,極少數地區電力或自來水可能中斷。 \\
\hline
5弱,強震 & 15-30 & 大多數人會感到驚嚇恐慌,難以走動。 & 部分未固定物品傾倒掉落,少數傢俱可能移動或翻倒,少數門窗可能變形,部分牆壁產生裂痕。 & 部分建築物牆磚剝落,部分山區可能發生落石,少數地區電力、自來水、瓦斯或通訊可能中斷。 \\
\hline
5強,強震 & 30-50 & 幾乎所有的人會感到驚嚇恐慌,難以走動。 & 大量未固定物品傾倒掉落,傢俱移動或翻倒,部分門窗變形,部分牆壁產生裂痕,極少數耐震較差房屋可能損壞或崩塌。 & 部分建築物牆磚剝落,部分山區發生落石,鬆軟土層可能出現噴沙噴泥現象,部分地區電力、自來水、瓦斯或通訊中斷,少數耐震較差磚牆可能損壞或崩塌。 \\
\hline
6弱,烈震 & 50-80 & 搖晃劇烈以致站立困難。 & 大量傢俱大幅移動或翻倒,門窗扭曲變形,部分耐震能力較差房屋可能損壞或倒塌。 & 部分地面出現裂痕,部分山區可能發生山崩,鬆軟土層出現噴沙噴泥現象,部分地區電力、自來水、瓦斯或通訊中斷。 \\
\hline
6強,烈震 & 80-140 & 搖晃劇烈以致無法站穩。 & 大量傢俱大幅移動或翻倒,門窗扭曲變形,部分耐震能力較差房屋可能損壞或倒塌,耐震能力較強房屋亦可能受損。 & 部分地面出現裂痕,山區可能發生山崩,鬆軟土層出現噴沙噴泥現象,可能大範圍地區電力、自來水、瓦斯或通訊中斷。 \\
\hline
7級,劇震 & >140 & 搖晃劇烈以致無法依意志行動。 & 幾乎所有傢俱都大幅移動或翻倒,部分耐震較強建築物可能損壞或倒塌。 & 山崩地裂,地形地貌亦可能改變,多處鬆軟土層出現噴沙噴泥現象,大範圍地區電力、自來水、瓦斯或通訊中斷,鐵軌彎曲。 \\
\hline
\end{tabular}
\end{table}\end{center}
\sssc{芮氏地震規模(Richter magnitude scale)級/近震規模}
中國大陸稱里氏震,港澳稱黎克特制地震震級。由美國芮克特(Charles Francis Richter)和德國古騰堡(Beno Gutenberg)制定。記作$M_L$。定義為「距離震央100公里處之觀測點的伍德-安德森扭力式地震儀記錄到的最大水平位移(微米)的常用對數」,若觀測點不同則換算振幅減弱等因素求得之。一般取自小數點後1位。

預期能量($J$)與$M_L$的關係式:
$$\displaystyle E=10^{4.8+1.5M_L}$$
缺點:芮氏規模大於約6.8或觀測點與震央距離超過約600公里便不準確,對於大地震評估規模過小。
\sssc{地震矩規模(Moment magnitude scale,MMS)}
中國大陸、香港、澳門稱矩震級。1977年由金森博雄制定。記作$M_W$。一般取自小數點後1位。在較低規模時與芮氏規模數值相近。定義為:
$$\displaystyle M_{W}=\frac{2}{3}\log_{10}M_{0}-10.73$$
其中,$\displaystyle M_{0}$為地震矩。

地震矩定義為:
$$\displaystyle M_{0}=\mu \cdot \bar {D}\cdot S$$
其中,$\displaystyle \mu $為斷層物質之剛性係數(Rigidity/Shear modulus),一般為 30$\times 10^9 Pa$;$\displaystyle \bar {D}$為斷層之平均滑動量(位移);$displaystyle S$為斷層面積。

缺點:測量需要較長時間。
\subsubsection{場址效應(Site effects)}
一種影響地震震度的因素,又稱盆地效應。當震源的地震波傳到地表時,地表淺層的介質的軟硬程度會影響此地的震度大小,較鬆軟的沖積層使波速降低故振幅較大,地震時間較長。
\ssc{地震紀錄}
\subsubsection{臺灣地震紀錄}
\begin{itemize}
\item 1906/3/17 梅山地震
\item 1935/4/21 新竹-臺中地震
\item 1964/1/18 白河地震
\item 1999/9/21 集集地震
\item 2016/2/6 高雄美濃地震
\item 2018/2/6 花蓮地震
\item 根據中華民國中央氣象局,2012-2017年臺灣平均每年發生超過三萬次地震,平均每年發生約627次有感地震,1999年發生49928次地震,超過三千次有感地震。
\item 臺灣東部為板塊交界帶,地震活動頻率最高;而歷年重大災害地震則多發生在西部人口稠密區。
\end{itemize}
\sssc{世界地震紀錄}
\begin{itemize}
\item 規模最大:1960/5/22 智利,規模 9.5
\item 傷亡最多:2004/12/26 蘇門答臘外海,規模 9.1
\end{itemize}
\subsection{地震災害}
\sssc{地震災害}
\begin{itemize}
\item 直接災害
\begin{itemize}
\item 建物毀損
\item 地表破裂
\item 山崩
\item 海嘯:大規模擾動海水造成,通常成因有:
\begin{itemize}
\item 大規模海底淺源地震使海床快速產生垂直位移。
\item 大規模海底火山爆發或海底山崩。
\item 陸地大體積物質如山崩土石落入排水不易之海灣。
\end{itemize}
\item 土壤液化(Soil liquefaction):指土壤因地震的壓密作用,造成原本在深層土壤的水份被擠壓到表層,土壤顆粒間的有效應力下降為零,土壤失去剪應力強度,呈現如砂質顆粒浮在水中的現象。將使砂質土壤失去承載建築物重量的力量,造成建築物下陷或傾斜。常發生於地層較薄的年輕細沙或排列整齊的泥土伴隨地下水位較高之處遭遇震度5級以上地震時,如填水造陸或將廢水注入地下的區域。當地表承受不住地下水的壓力時就會破裂,水壓可能帶著沙土噴出地面,即噴沙湧水,常發生於地面較薄處。
\end{itemize}
\item 間接災害
\begin{itemize}
\item 火災
\item 水災
\end{itemize}
\end{itemize}
\subsubsection{地震防災}
根據中華民國內政部消防署,地震時的桌下避難三步驟為趴下(Drop)、掩護(Cover)、穩住,抓住桌腳(Hold on)。
\subsubsection{地震預警}
透過P波傳遞較造成災害的S波傳遞更快,令P波速率$v_p$、S波速率$v_s$、距離$S$、P波和S波到他時間差$\Delta T_{ps}$:
\[\Delta T_{ps}=\frac{S}{v_s}-\frac{S}{v_p}\]
可以在震央附近至少三個測站收到P波時計算震央、規模、各地預期震度等資訊並廣播之,達到地震早期預警。

預警盲區:地震預警訊息來不及在S波到來前傳達之區域,地震儀密度愈高,預警盲區愈小。

災防告警細胞廣播訊息服務(Public Warning Cell Broadcast Message Service)/災防告警系統(Public Warning System, PWS):是中華民國政府利用細胞廣播技術,在短時間內經由臺灣4G業者的行動寬頻系統,大量傳送地震速報、土石流警戒、公路封閉等防災警示訊息到手機,即時通知民眾的系統。此系統可以在數秒內傳送訊息到特定區域內的數十萬到百萬臺手機,手機在接收到訊息時,會發出特殊的警告聲響與振動。
\section{火成岩(Igneous rock)}
\subsection{火成岩列表}
\begin{itemize}
\item 長英質矽酸岩:白雲母、石英、長石、高嶺石。
\item 鐵鎂質矽酸岩:橄欖岩、角閃石、輝石、黑雲母。
\item 花崗岩質岩漿火山岩:流紋岩(冷卻較慢)、浮石(冷卻較快,多孔洞,玻璃質)。
\item 安山岩質岩漿火山岩:安山岩(冷卻較慢)、黑曜石(冷卻較快)。
\item 安山岩質岩漿深成岩:閃長岩。
\item 花崗岩質岩漿深成岩:花崗岩。
\item 玄武岩質岩漿深成岩:輝長石。
\item 橄欖岩質岩漿深成岩:橄欖岩。
\item 玄武岩質岩漿火山岩:玄武岩。
\end{itemize}
\subsection{火成岩質比較}
\begin{center}
\begin{table}[H]
\centering
\begin{tabular}{|p{0.1\textwidth}|p{0.2\textwidth}|p{0.2\textwidth}|p{0.2\textwidth}|p{0.2\textwidth}|}
\hline
項目 & 玄武岩(Basalt) & 安山岩(Andesite) & 花崗岩(Granite) & 橄欖岩(Peridotite)\\
\hline
二氧化矽含量 & 約45-52\%(低) & 約57-63\%(中) & 約63-75\%(高) & 低於45\%(極低) \\
\hline
金屬氧化物含量 & 高含量,如鎂、鐵、鋁的氧化物 & 中含量,如鈉、鉀的氧化物 & 低含量,主要為鈉和鉀的氧化物 & 最高,富含鎂、鐵氧化物 \\
\hline
形成位置 & 主要在淺層,海洋地殼表面形成 & 中等深度,火山附近的地殼形成 & 深層,地殼內部緩慢冷卻形成 & 深成岩,來自於地函較深處 \\
\hline
岩漿來源位置 & 來自於地函淺層 & 來自於隱沒帶中等深度 & 來自於大陸地殼深部 & 來自於地函深層 \\
\hline
熔點 & 約1000-1200°C(高) & 約800-1000°C(中) & 約600-800°C(低) & 約1200-1400°C(極高) \\
\hline
酸鹼(基)性 & 鹼性 & 中性 & 酸性 & 強鹼性 \\
\hline
密度 & 約2.8-3.3 g/cm\textsuperscript{3}(高) & 約2.6-2.8 g/cm\textsuperscript{3}(中) & 約2.65-2.75 g/cm\textsuperscript{3}(低) & 約3.2-3.4 g/cm\textsuperscript{3}(極高) \\
\hline
莫氏硬度 & 約5-6(低) & 約6(中) & 約6-7(高) & 約6-7(高) \\
\hline
顏色 & 深色(黑色、深灰色) & 中等顏色(灰色) & 淺色(淡灰色、粉紅色) & 深色(黑色、深綠色) \\
\hline
成分 & 主要成分是鹼性礦物,即二氧化矽、矽鋁酸鈉、矽鋁酸鉀與金屬氧化物(主要是鋁氧化物),如橄欖石、輝石及斜長石。 & 主要成分是中性礦物,鹼金屬成分不超過6\%,如斜長石、角閃石及少量石英。 & 主要成分是酸性礦物,即二氧化矽與金屬氧化物(主要為鹼金屬與鹼土金屬氧化物),如石英、長石(鈉長石和鉀長石)和雲母。 & 主要成分為橄欖石和輝石,極低的二氧化矽含量,富含鎂和鐵。 \\
\hline
成因與環境 & 寧靜式噴發火山。主要存在於海洋地殼,從中洋脊火山與熱點火山爆發中快速冷卻結晶形成,岩漿來自地函 & 猛烈式噴發火山。主要存在於俯衝帶地區,岩漿來自隱沒帶下熔化的地殼,如火山島弧。 & 侵入式火山。主要存在於大陸地殼,在地底緩慢冷卻結晶形成深成岩。 & 主要存在於地函,為地球上最常見的深成岩之一,為大陸板塊運動提供材料。 \\
\hline
\end{tabular}
\end{table}
\end{center}
\section{變質岩(Metamorphic rocks)}
由原有岩石(火成岩、沉積岩或其他變質岩)在地殼內部經受高溫、高壓或化學流體的作用而形成的。這些作用導致岩石的礦物組成、結構或質地發生變化,但不會經歷完全的熔融。
\sssc{按成因分類}
\begin{itemize}
\item 接觸變質岩:由岩漿侵入體周圍的高溫作用形成,主要受熱影響,壓力作用較小。通常形成細粒、緻密的變質岩。例如:
\begin{itemize}
\item 角岩(Hornfels):具有致密且細粒的結構,常由頁岩或火山灰岩變質而成。
\item 矽卡岩(Skarn):富含鈣矽酸鹽礦物,常由石灰岩與岩漿接觸變質形成。
\end{itemize}
\item 區域變質岩:由板塊碰撞或地殼運動導致的高壓、高溫作用形成,通常發生在大範圍內。具有明顯的片理或片狀結構。例如:
\begin{itemize}
\item 片岩(Schist):以片理結構為特徵,常富含雲母。
\item 片麻岩(Gneiss):具有條帶狀結構,礦物分布不均。
\item 千枚岩(Slate):由頁岩經低度變質形成,具有板狀裂解。
\end{itemize}
\item 動力變質岩:由斷層帶內的剪切壓力和摩擦熱作用形成,通常伴隨強烈變形。結構被拉伸、破碎或重新結晶。例如
\begin{itemize}
\item 糜棱岩(Mylonite):因強烈剪切作用而細粒化。
\item 角礫岩(Breccia):由碎裂的岩石重新膠結而成。
\end{itemize}
\end{itemize}
\sssc{按變質程度分類}
\begin{itemize}
\item 低度變質岩:僅經歷輕微的礦物改變,原始岩石特性部分保留。例如:千枚岩、板岩。
\item 中度變質岩:出現較大的礦物改變,形成明顯的片理或條帶。例如:片岩。
\item 高度變質岩:礦物完全重結晶,原始岩石特徵消失。例如:片麻岩。
\end{itemize}
\sssc{按岩石組成分類}
\begin{itemize}
\item  矽酸鹽變質岩:由富含矽酸鹽的岩石變質而成,主要成分為石英和長石。例如:片麻岩、片岩。
\item 碳酸鹽變質岩:由石灰岩或白雲岩變質而成,主要成分為方解石或白雲石。例如:大理岩(Marble)由石灰岩變質,常用於建材。
\item 超基性變質岩:由超基性岩(如橄欖岩)變質而成,富含鎂和鐵。例如:蛇紋岩(Serpentinite)由橄欖石變質,質地光滑。
\end{itemize}
\section{固體地球(Solid Earth)}
\ssc{固體地球內部分層}
\begin{itemize}
\item \textbf{地殼(Crust)}:矽酸鹽與矽氧化物岩質。熔點大於溫度。
\begin{itemize}
\item \textbf{海洋地殼}:0至7\sim 10 km 深。矽鎂質玄武岩構成,偏向鹼性,密度較大,平均密度約3.0\,$\text{g/cm}^3$。岩石圈一部分。固態。
\item \textbf{大陸地殼}:0至25\sim 70(通常35\sim 40) km 深。矽鋁質岩石如花崗岩、變質岩構成,偏向酸性,密度較小,平均密度約2.7\,$\text{g/cm}^3$。岩石圈一部分。固態。
\end{itemize}
\item \textbf{莫荷不連續面}:地殼與地函交界,其下之波速與密度大於其上。1909年克羅埃西亞科學家莫荷(Andrija Mohorovicic)發現沿該界面傳遞的體波(折射波)速度較快,較路徑較短的直達波更快抵達。
\item \textbf{地函/地幔(Mantle)}:橄欖岩質,證據為捕獲岩,即地函岩漿上噴帶至地表的地函岩石碎塊。熔點、溫度與壓力均隨深度遞增。
\begin{itemize}
\item \textbf{上部地函}:莫荷不連續面至660 km 深。密度隨深度遞增。
\begin{itemize}
\item \textbf{岩石圈除了地殼的部分}:莫荷不連續面至65-150 km(視地殼深度,平均約70-100 km)深。固態,即熔點大於溫度。
\item \textbf{軟流圈(震波低速帶)}:65-150至190-250 km 深。固態,少部分(約3\%)熔融,即熔點平均大於溫度但少部分低於溫度。
\end{itemize}
\item \textbf{660 km}:其下之波速與密度大於其上。
\item \textbf{下部地函}:660至2890 km 深。結晶方式與上部地函不同。波速與密度均隨深度遞增。熔點大於溫度。
\end{itemize}
\item \textbf{古氏不連續面}:2890 km 深。其下之密度大於其上很多,其下之P波波速小於其上很多,S波在其下之外核不存在,其下之熔點遠低於其上,使外核為液態。1913年德國科學家古騰堡發現穿透界面的P波因為介質不同,導致波速改變,造成折射角變化,明顯偏離至距震央較遠處,導致地表上103°至143°與-103°至-143°(令地心為極點,地心與震央的連線為極軸)的地方接收不到自震央直達的P波,稱P波陰影帶(Shadow Zone),後進一步發現地表上103°經180°的地方接收不到自震央直達的S波,稱S波陰影帶,推測其下應為S波無法傳遞的液態介質。
\bct\bfH\ctr\icg[width=0.8\textwidth]{S.png}\ef\FB\ect
\item \textbf{地核(Core)}:鐵、鎳質,證據為外來隕石成分推測早期地球成分。
\begin{itemize}
\item \textbf{外核}:2890至5150 km 深。液態。溫度、壓力、密度與波速均隨深度遞增,不存在S波。
\item \textbf{雷嫚不連續面}:5150 km 深。其下之P波波速與密度大於其上,S波在其下之內核可存在,熔點恰等於溫度,其上熔點小於溫度故為液態,其下熔點大於溫度故為固態。1936年丹麥科學家雷嫚(Inge Lehmann)發現。
\item \textbf{內核}:5150至6370 km 深。固態。密度與波速均隨深度略呈遞增。
\end{itemize}
\end{itemize}
\ssc{固體地球內度壓力、溫度與熔點變化}
固體地球內部壓力與溫度均為愈深愈高。固體地球內部熔點於地殼(海洋地殼大於大陸地殼)至地函愈來愈高且高於該處溫度,於地函與地核交界迅速下降至低於該處溫度,而後愈深愈高且上升率高於溫度上升率,至外核與內核交界處上升至與該處溫度相同,而後在內核高於該處溫度。
\ssc{固體地球內部 P 波與 S 波波速變化}
\bct\bfH\ctr\icg[width=0.8\textwidth]{V.png}\caption{Karla Panchuk, modified after Steven Earle (2016)}\ef\FB\ect
\section{板塊}
\subsection{學說}
\subsubsection{大陸漂移學說(Continental drift)}
大陸可能漂移的推測最早由亞伯拉罕·奧特利烏斯 (Abraham Ortelius)於1596年提出。

德國氣象學家阿爾弗雷德·韋格納(Alfred Wegener)於1912年提出大陸漂移學說(Continental Drift Theory),指出,現今的各大陸曾經是一個超大陸,稱為盤古大陸(Pangaea),這個超大陸在2億多年前開始分裂,並逐漸漂移到今天的位置。

證據:
\begin{itemize}
\item 大陸邊緣的形狀相吻合,如南美洲東側和非洲西側的海岸線。
\item 無法跨洋繁殖之物種的相似化石分布在不同大陸上出現,如水龍和中龍在南美洲南部、非洲南部的連續分布、犬頭龍在非洲南部、印度南部、南極洲的連續分布、舌羊齒在南美洲南部、非洲南部、印度南部、南極洲和澳洲南部的連續分布。
\item 不同大陸之間的地質構造和岩石層吻合,如沙漠層積和煤層在盤古大陸上呈現依據緯度變化的連續分布,其中低緯度的煤層包含南美洲北部地區、北美洲東部、歐亞大陸從歐洲西部到中南半島的連續分布,北半球馬緯度的沙漠沉積包含北美洲大平原和歐洲大平原的連續分布,南半球馬緯度的沙漠沉積包含從巴西的西北非的連續分布。
\item 冰川生成與向外移動的痕跡在不同大陸上連續分布,如古生代晚期地層中可發現冰川移動的刮蝕痕跡,拼合後恰為以非洲南部與南極洲為向澳洲南部、南美洲東南部、非洲中南部、印度南部移動。
\end{itemize}
關於推動大陸漂移的作用力,韋格納提出離心力或潮汐力等,經計算證明不足以帶動大陸,且其預測的大陸漂移速度較實際速度快很多,且其非地質學家,故其學說直到他於格陵蘭極地氣象站死亡仍未受廣泛接受。

英國地質學家阿瑟·霍姆斯(Arthur Holmes)擁護該理論,並於1931年提出,地函對流說,指出地函中含有對流細胞,這些對流細胞可以消散放射性衰變產生的熱量,並移動地表的地殼,1944出版《自然地質學原理》。
\subsubsection{海底擴張學說(Seafloor spreading)}
背景:
\begin{itemize}
\item 二戰時,雙方均推進海下地形與環境的探測技術。
\item 1950年代首張海底地形圖問世,海底不僅有高低起伏,還有綿延全球各大洋、長達8萬公里、寬達1000公里、高度落差達數公里的中洋脊地形。
\item 1950年代後,對於大西洋的海下研究指出海底地形以中洋脊為對稱軸線對稱,且該處的沉積物最薄、年代最新。
\item 地熱測量指出中洋脊附近的海洋地殼熱流值較高,認為與地函對流活動有關。
\item 中洋脊兩側的海洋地殼具有條帶狀地磁異常紀錄,可看出地磁反轉次序在中洋脊兩側對稱分布,與陸地火成岩非對稱不同。
\end{itemize}
哈里·哈蒙德·海斯(Harry Hammond Hess)於1962年提出了海底擴張學說(Seafloor Spreading Theory),指出,新的海洋地殼在中洋脊處形成,然後向兩側擴張,年老的海洋地殼則在海溝附近隱沒回到地函。海洋地殼維持變動、不停生滅的結果,可以解釋海洋地殼年齡小於2億年、愈遠離中洋脊的海洋地殼沉積物厚度愈厚,與中洋脊兩側條帶狀地磁異常紀錄對稱等現象。

條帶狀地磁異常紀錄:地函岩漿從中洋脊湧出,溫度下降到約450-500°C時,新生玄武岩所含的磁性礦物受到當時地球磁場磁化,順著地磁方向排列,冷卻後保有生成時地球磁場的紀錄。當地球磁場發生極性反轉時,新生玄武岩受到影響,已冷卻凝固的海洋地殼則否,故形成正、反向(正向指與現今相同)磁場交替的紀錄,稱條帶狀地磁異常紀錄。地磁可能成因為外核液態鐵鎳物質的流動,其反轉無固定週期。
\subsubsection{板塊構造學說(Plate tectonics)}
加拿大地質與地球物理學家威爾遜(John Tuzo Wilson)在1965年指出中洋脊之間因擴張而出現水平錯動的轉形斷層。

麥肯齊(Dan McKenzie)、摩根(W. Jason Morgan)等各國多位科學家在1967年將全球地表中洋脊、轉形斷層、海溝、火山島弧與褶皺山脈等構造對照地震學研究成果,如震央位置的帶狀分布、岩石圈與軟流圈的性質等,於次年正式提出板塊構造學說(Plate Tectonics Theory),整合了韋格納的大陸漂移學說和海斯的海底擴張學說證據。指出,堅硬的岩石圈受到地函熱對流帶動,因受力不均破裂為若干塊區塊,稱為板塊,這些板塊漂浮在部分熔融的軟流圈之上,並且彼此之間存在著相互作用。這些板塊的運動和相互作用形成了地震、火山活動、山脈形成等地質現象。
\subsection{地球表面的板塊}
\bct\bfH\ctr\icg[width=0.9\textwidth]{plate.png}\caption{Eric Gaba, 2006, updated 2015}\ef\FB\ect
\begin{itemize}
\item 七大板塊
\begin{itemize}
\item 歐亞板塊
\item 非洲板塊
\item 印澳板塊
\item 太平洋板塊
\item 北美洲板塊
\item 南美洲板塊
\item 南極洲板塊
\end{itemize}
\item 小板塊
\begin{itemize}
\item 加勒比海板塊(加勒比海與中美洲)
\item 寇克斯板塊(中美洲西邊,太平洋板塊東邊)
\item 菲律賓海板塊(菲律賓東邊、臺灣東邊、日本東南邊、馬里亞納海溝西邊)
\item 皇安德福卡板塊(加州聖安得烈斯斷層北邊的喀斯開俯衝帶西邊)
\item 那茲卡板塊(南美洲板塊西邊,寇克斯板塊南邊,太平洋板塊東邊)
\item 斯科夏板塊(南美洲板塊南邊)
\item 阿拉伯板塊(阿拉伯半島)
\end{itemize}
\end{itemize}
\subsection{板塊邊界分類}
\begin{itemize}
\item 張裂型板塊邊界:兩側板塊相互分離,岩石圈厚度較薄,地函熱對流上升,物質溫度較高、密度較小,地形突出隆起形成中洋脊,岩漿冷凝形成玄武岩質火成岩,構成新生海洋地殼的主體。地震以淺源居多,規模通常較聚合型與錯動性板塊邊界小。洋脊附近的海水也可以穿過裂隙接觸高溫的岩石圈,藉由循環及散熱達成調節地球內部熱能、控制海水化學性質的功能,並造就深海熱泉生態系統。
\begin{itemize}
\item 大西洋中洋脊:擴張較慢,山脊高且窄,且較崎嶇。
\item 東太平洋中洋脊:擴張較快,山脊寬廣。
\item 東非大裂谷(地塹):索馬利亞板剛要自非洲板塊分離的狀態。
\end{itemize}
\item 聚合型板塊邊界:兩側的板塊相互靠近,其一隱沒至另一之下,形成隱沒帶(俯衝帶)。具有地殼隱沒的聚合型板塊邊界,若火山來自海洋地殼隱沒產生的熔融岩漿,其火成岩多以安山岩為主;因承受極大擠壓力,常造成頻繁地震活動,當堅硬的岩石圈向下隱沒並發生破裂,地震的震源位置分布就會呈現一向下傾斜的帶狀,由淺源延伸到深源,稱為班尼奧夫帶,為板塊隱沒方向與深度的直接證據。有隱沒帶者通常淺、中、深源地震皆有,兩側都是大陸地殼的聚合型板塊邊界則以淺、中源地震為主。
\begin{itemize}
\item 一側是海洋地殼,一側是大陸地殼:密度較大的海洋地殼隱沒到大陸地殼下方,隱沒處常形成海溝,已隱沒的海洋地殼上方常形成島弧或造山帶等地貌,如那茲卡板塊隱沒到南美洲板塊下形成的秘魯-智利海溝與安地斯山脈。
\item 兩側都是海洋地殼:較老、較厚、密度較大的板塊往往隱沒到較年輕、較薄、密度較小的板塊之下,隱沒處常形成極深邃的海溝,已隱沒的海洋地殼上方常形成島弧,如太平洋板塊隱沒到菲律賓海板塊下形成的馬里亞納海溝與馬里亞納火山島弧。
\item 兩側都是大陸地殼:兩側都是密度較小、較厚、不易隱沒的大陸地殼,則相互擠壓,造成地殼變形增厚而抬升,形成山脈如印澳板塊隱沒到歐亞板塊下形成的喜馬拉雅山與西藏高原有世界屋脊之稱。
\end{itemize}
\item 錯動型板塊邊界:兩側的板塊相互水平移動,最常見者為介於兩段中洋脊之間的轉形斷層。這種邊界不會有物質新生或消滅,地貌呈現水平錯動特徵,少有火山或岩漿活動,地震以淺源居多。例如加州舊金山與洛杉磯之間的聖安得烈斯斷層、紐西蘭南、北島之間的阿爾卑斯斷層。
\end{itemize}
\subsection{斷層}
\sssc{構造}
斷層有斷層面、斷層線和斷層磐三要素。斷層面是指岩層受力後發生相對位移的破裂面,通常成呈面狀或弧形展布,具有一定的走向、傾向、傾角。斷層線是指斷層面與地面的交線。斷層磐是指斷層面兩側的斷塊,原始位置較上者稱上磐,下者稱下磐。
\sssc{分類}
\begin{itemize}
\item 傾向斷層
\begin{itemize}
\item 正斷層:是岩層因為受張力或是重力的作用而產生拉裂的斷層,上盤相對下降,下盤相對上升。
\item 逆斷層:是岩層因為受到擠壓的力量而產生的斷層。上盤相對上升,下盤相對下降。
\end{itemize}
\item 平移(滑移、走滑)斷層:是岩層受到水平方向的剪切力而產生左右位移的斷層。
\begin{itemize}
\item 左移斷層:站在斷層一側,向對面看過去,發現對面岩層向左移動的斷層。
\item 右移斷層:站在斷層一側,向對面看過去,發現對面岩層向右移動的斷層。
\end{itemize}
\item 轉移斷層:是一種特殊的平移斷層,指在具有水平相對運動的板塊邊界上,鄰近岩層受力產生平移斷層的破裂,在張裂型、聚合型與兩者之間都能找到,常發生於銜接兩段中洋脊、銜接兩段海溝或銜接中洋脊與海溝。其破裂帶有時可延伸至板塊邊界線兩端一段距離,但斷層與地震通常僅在板塊邊界本身。中洋脊上的轉形斷層成因之一是由於緯度愈低的區域地球自轉速度四愈快,使軟流圈上層隨緯度相對平移。
\end{itemize}
\subsubsection{活動斷層}
依中華民國中央地質調查所的定義,活動斷層指過去十萬年內曾有活動證據且未來可能再次錯動的斷層。
\begin{itemize}
\item 第一類活動斷層:指過去十萬年內曾有活動證據的活動斷層。
\item 第二類活動斷層:指不是第一類活動斷層的活動斷層。
\end{itemize}
\subsection{熱點火山}
熱點火山又稱島鏈火山,指非出現在板塊邊界上的火山,例如夏威夷與黃石公園。無法以板塊構造學說解釋,但可作為板塊移動的證據之一。目前研究認為該處岩漿來自較深處的地函物質湧出,以熱柱方式貫穿地表而成。熱柱與上方板塊相對運動造成地表上形成一連串的熱點火山島鏈,具有年代先後的順序性,例如夏威夷群島。岩漿來源較板塊擠壓造成的島弧火山深,故鎂鐵含量較高,以玄武岩質為主。
\section{南海與臺灣地區地質歷史}
\bctf\icg[height=0.7\th]{earthquake_fig02.jpg}\cpt{國立臺灣大學。台灣的地震活動。\href{https://homepage.ntu.edu.tw/~tengls/geo-info\_earthquake.htm}{https://homepage.ntu.edu.tw/~tengls/geo-info\_earthquake.htm}.}\efct
\begin{itemize}
\item 距今1億4000萬年-2000萬年前:古太平洋板塊向歐亞板塊隱沒,在亞洲東部如中國東南和日本一帶形成大規模的火山噴發,歐亞板塊邊緣大陸棚上堆積的沉積物被推擠隆起。
\item 距今8000萬年-6500萬年前:古太平洋板塊減慢與停止向西隱沒到歐亞板塊,歐亞板塊不再受擠壓,形成張裂的環境,華南區域火山活動減少。
\item 距今6500萬年-3000萬年前:臺灣及鄰近地區開始沉陷成沉積盆地,南海盆地裂谷系統開始發育,堆積來自中國華南地區侵蝕下來的大量沉積物,即今日雪山山脈、西部麓山帶及中央山脈西翼的巨厚沉積岩層。由於印澳板塊和歐亞板塊的碰撞,裂谷在距今5000萬年前左右加劇。
\item 距今3000萬年-1500萬年:南中國海開始海底擴張,造成歐亞板塊裂開,並產生新的海洋地殼,稱南中國海板塊。印澳板塊與歐亞板塊的碰撞引起了婆羅洲的旋轉和南海南部邊界的閉合。南中國海岩圈向東隱沒到臺灣東側的菲律賓海板塊之下,並且在菲律賓海板塊上形成一長串的安山岩質火山島鏈,稱呂宋島弧。這串島鏈北端的火山島就是現今的海岸山脈、綠島與蘭嶼。
\item 距今2000萬年-600萬年前:板塊伸張趨於緩和,大陸棚陷落盆堆積厚層沉積物,溢流式的玄武岩質岩漿時有噴發,1000萬年前一次較大規模的火山噴發,形成今澎湖群島。
\item 距今600萬年-300萬年前:菲律賓海板塊西緣的呂宋島弧北端與今臺灣島附近的歐亞大陸邊緣碰撞,稱弧陸碰撞,使位在臺灣附近大陸棚上的沉積岩層及火山島弧相互推擠、隆起,進行造山運動,從南到北逐漸露出海面形成今日的臺灣島,稱蓬萊運動。呂宋海峽隨著臺灣的抬升而開通。
\item 300萬年-20萬年前:臺灣島已經大致具有現今地貌雛形,分屬菲律賓海板塊與歐亞板塊的東部海岸山脈的火山島弧和中央山脈逐漸拼接在一起,現以狹長的花東縱谷為界。西北方移動的菲律賓海板塊一部分隱沒到北臺灣島之下,自280萬年前開始在臺灣北部及外海地區如基隆嶼、棉花嶼、彭佳嶼與釣魚臺形成火山噴發。主要的弧陸碰撞擠壓點逐漸南移,加上菲律賓海板塊向西北隱沒,歐亞板塊在臺灣北部地殼由原先的擠壓作用逐漸轉變為逆時針旋開的張裂作用,造成臺北盆地陷落,如今臺灣北部的山腳斷層為第二類正斷層、三峽一帶存在玄武岩。距今100萬年前,金瓜石一帶基隆火山群岩漿噴發或侵入沉積岩層中,並形成豐富的金銅礦產。距今80-20萬年前,是安山岩質的大屯火山群主要的噴發期。在菲律賓海板塊向西北隱沒形成琉球島弧的同時,也造成歐亞板塊邊緣張裂,形成1000-2000公尺深的弧後沖繩海槽及蘭陽盆地,隱沒之處形成琉球海溝,隱沒帶之上形成琉球火山島弧,至今宜蘭外海與琉球群島等地的隱沒及張裂活動還不斷在進行。
\item 現今:北緯24度東北方宜蘭外海存在一個向北往琉球海溝傾斜的班尼奧夫帶向東延伸於琉球群島轉向東北,為菲律賓海板塊向西北往歐亞板塊下隱沒的隱沒帶。北緯23度臺東南方存在一個向東傾斜的班尼奧夫帶向南延呂宋火山島弧延伸至菲律賓北方,是歐亞板塊向東往菲律賓海板塊下隱沒的隱沒帶,係接續自南中國海板塊隱沒到菲律賓海板塊之下。臺灣東部約以每年向西北方8公分相對於澎湖白沙測站S01R移動。臺灣本島除臺北盆地一帶因歐亞大陸板塊逆時針旋開形成的正斷層,如山腳斷層為第二類正斷層外,大型斷層多為逆斷層。臺灣本島上,中央山脈與雪山山脈以變質岩為主,多片麻岩、大理岩、片岩等;大屯山與海岸山脈則以火成岩為主,其中大屯山主要為安山岩,海岸山脈以安山岩為主亦有他種火成岩與少數變質岩;西部麓山帶與西部海岸平原則以沉積岩為主。
\end{itemize}
\end{document}