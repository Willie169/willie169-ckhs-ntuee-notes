\documentclass[a4paper,12pt]{report}
\setcounter{secnumdepth}{5}
\setcounter{tocdepth}{3}
\newcounter{ZhRenew}
\setcounter{ZhRenew}{1}
\newcounter{SectionLanguage}
\setcounter{SectionLanguage}{1}
\input{/usr/share/latex-toolkit/template.tex}
\begin{document}
\title{古氣候學}
\author{沈威宇}
\date{\temtoday}
\titletocdoc
\ch{古氣候學(UK: Palaeoclimatology/US: Paleoclimatology)}
\sct{詞彙釋義}
\ssc{古氣候學(UK: Palaeoclimatology/US: Paleoclimatology)}
是研究地球歷史各時期氣候、分析其驅動原因與建立預測模型的一門學科。影響古氣候變化的主要因素被認為是地球軌道變化、太陽常數的長期變化、大氣氣體含量改變以及板塊動力學引起的海洋和造山變化等,大氣中二氧化碳濃度、氧同位素紀錄等均是其重要指標。
\ssc{古氣候(Paleoclimate)紀錄}
由於人類儀器測量大氣狀態的紀錄僅約400年,歷史文獻亦僅數千年,若想獲知更久遠之前的氣候狀態,就需借助大自然中的替代數據/代用指標(proxy)。因為地質材料受到地質作用改變其原始紀錄,大致而言,新生代以前的氣候紀錄只能概略看出冷暖變化,新生代以來的氣候變動較能被詳細地重建。
\ssc{冰河學(Glaciology)}
是研究地球表面各種自然冰體的一門學科。自然冰體的範圍包括山嶽冰河、大陸冰蓋、海冰以及河冰、湖冰、地下水、季節性結冰等。
\ssc{天氣(Weather)與氣候(Climate)}
\begin{itemize}
\item 天氣:指短時間內的大氣狀態。
\item 氣候:指長時間(30年以上)大氣、海洋及等陸地系統的氣象要素統計平均結果。
\item 氣候標準平均值:一段時間內某種氣象要素的平均值,被選為該項要素在某區域內的標準值。
\item 極端天氣(Extreme weather)事件:指特定地區發生在其統計分布一成以下的罕見天氣事件。
\end{itemize}
\sct{影響古氣候的因素}
\ssc{板塊運動}
板塊運動速率每年僅數公分,萬年內不易看出明顯海陸變化,但百萬年到億年以上可見板塊動力學引起的海洋和造山變化,並影響地球的氣候型態。
\bct\bfH\ctr\icg[width=0.9\textwidth]{Plate425Myr.png}\caption{Martha Papadopoulou, Tiffany L. Barry, Batulzii Dash, Alison M. Halton, Sarah C. Sherlock, Alison C. Hunt. 2024. Evidence for long-lived (>100 Myr) continental intraplate volcanism: Mongolia since the last ocean closure, Gondwana Research, Volume 133, 30-59. \href{https://doi.org/10.1016/j.gr.2024.04.009}{https://doi.org/10.1016/j.gr.2024.04.009}.}\ef\FB\ect\bct\bfH\ctr\icg[width=0.9\textwidth]{Plate.jpg}\ef\FB\ect
\sssc{特提斯洋(Tethys Ocean)封閉與喜瑪拉雅山脈隆起}
60百萬年前,歐亞大陸印度之間存在古地中海/特提斯洋,從50百萬年前開始,印度陸塊撞上歐亞陸塊使之逐漸封閉,同時造成喜瑪拉雅山脈的隆起,改變大氣環流,將水氣侷限於其南側,使原先水草豐美的亞洲內陸變乾,形成塔克拉瑪干沙漠。
\sssc{南美洲大陸與南極洲大陸分離}
34百萬年前,南美洲大陸與南極洲大陸分離,德雷克(Drake)海峽打開,繞南極環流/南極繞極環流(Antarctic Circumpolar Current)出現(亦與海洋鹽度改變阻隔北方暖水南下有關),使得南極大陸孤懸於極圈內,南極冰原開始擴張,地球向冰室氣候發展。
\sssc{大陸冰層退卻影響板塊張裂}
335到到175百萬年前,全球大陸幾乎都聚合成一塊超級大陸,即盤古大陸。因部分位於南極區,有厚重的大陸冰層壓在大陸地殼上,使大陸地殼下沉,冰層溶解後則會上浮。大陸地殼在有冰層覆蓋與沒有冰層覆蓋的邊界出現應力差異,導致地殼變形與斷層發育。290到280百萬年前開始(二疊紀中期),由於該地區處於張力狀態,使冰層邊界的地層弱化淺薄,淺層岩漿上湧形成地函熱柱,一系列的玄武岩質火山沿著冰河邊緣噴發。經過數百萬年的噴發後,大陸地殼分裂,海洋地殼形成,新的海洋誕生。這些因冰層溶解而將大陸地殼分開的區域約在今土耳其、伊朗、阿富汗、西藏、中南半島、馬來西亞一線,綿延13千餘公里。230百萬年前(三疊紀中期),盤古大陸西岸開始分裂形成現今的墨西哥灣和加勒比海。175百萬年前(侏羅紀早期),裂縫加大開始形成北大西洋南部,盤古大陸重新分裂成僅有一小塊陸橋連接的勞亞大陸(今北美洲和亞洲不含阿拉伯半島與印度半島)和岡瓦納大陸(今南極洲、澳洲、印度半島、阿拉伯半島、非洲、南美洲)。
\ssc{火山運動}
火山作用通常在短時間因火山灰、微粒與氣溶膠體蔽空,造成地球降溫,但長時間下其排放之二氧化碳、二氧化硫等溫室氣體,造成地球升溫。
\sssc{中生代火山運動}
侏羅紀末至白堊紀之時,火山運動頻繁,空氣中二氧化碳濃度約為現在的三到五倍,處於溫室氣候中較暖的時期。中白堊紀時,甚至連高緯度地區都沒有冰原覆蓋。
\ssc{氣溶膠體}
空氣中固態或液態顆粒的聚集體,通常在十奈米到十微米,能在對流層中駐留數小時以上。
\begin{itemize}
\item 其散射和吸收輻射的直接影響,如煤煙吸收短波輻射,使對流層變暖,但地表吸收的輻射量卻減少。約可抵銷30\%二氧化碳暖化作用。
\item 其可在雲的形成過程中扮演凝結核或冰核,或改變雲的光學性質,可增加小雲滴之數量與雲之反照率,使大氣及地表吸收的輻射變少。約可抵銷42\%二氧化碳暖化作用。
\end{itemize}
\ssc{風化作用}
岩石通常由矽酸鹽組成,以\ce{CaSiO3}為例,其風化通常伴隨二氧化碳溶於水的過程,即:\ce{CaSiO3 + 2CO2 + H2O -> Ca^{2+} + 2HCO3 + SiO2}。當水中的碳酸氫根和鈣離子濃度就會發生碳酸鈣沉澱,形成石灰岩,即:\ce{Ca^{2+} + 2HCO3- -> CaCO3}$\downarrow$ + \ce{CO2 + H2O};兩者結合即:\ce{CO2 + CaSiO3 -> CaCO3}$\downarrow$ + \ce{SiO2}。因此:
\begin{itemize}
\item 風化作用反應本身會大氣消耗二氧化碳。
\item 風化作用會釋出陽離子到海洋,消耗碳酸根形成碳酸鹽沉澱,使更多二氧化碳自大氣移出而溶入海洋中,亦降低大氣二氧化碳濃度。
\item 風化作用會釋出營養鹽,促使浮游光和作用生物增加,亦降低大氣二氧化碳濃度。
\item 當造山運動促使岩石接觸空氣面積增加,會導致二氧化碳減少,使地球變冷。
\item 當冰川覆蓋地球時促使岩石無法接觸空氣行風化作用,無法消耗大氣中的二氧化碳,待火山作用等放出的溫室氣體逐漸累積,就會使地球變暖。
\item 當板塊運動造成熱帶、亞熱帶暖溼區域地形抬升,有機會加劇該處岩石風化作用,如喜馬拉雅山造山運動使山脈南面風化侵蝕劇烈。
\end{itemize}
\ssc{植被}
植被愈少,光合作用愈弱,二氧化碳含量增加趨勢愈大。例如夏季大氣二氧化碳濃度低於冬季的循環。
\ssc{反照率/反射率(Albedo)與冰雪覆蓋的正回饋效應}
反照率指物體表面反射太陽輻射與接收太陽輻射的比值。
\begin{itemize}
\item 反照率愈高,地表吸收的太陽輻射愈少。太陽輻射約有一半能達到地表並被吸收,其餘被大氣吸收或被雲、懸浮微粒與地表反射。
\item 地球反照率約31\%,金星反照率約90\%。
\item 冰雪反照率較高,可達九成,故冰雪覆蓋愈多,地球吸收的太陽輻射愈少,融冰因此變少,使冰雪繼續累積,是為冰雪覆蓋的正回饋效應。
\item 森林、海洋的反照率較低,僅一成。
\end{itemize}
\ssc{地球活動/米蘭科維奇循環(Milankovitch Cycles)}
塞爾維亞工程師暨科學家米蘭科維奇提出的地球軌道參數循環,是冰期-間冰期循環的主要成因。
\bct\bfH\ctr\icg[width=0.9\textwidth]{MCycle.png}\caption{Incredio, 2009.}\ef\FB\ect
\sssc{離/偏心率循環(Eccentricity Cycle)}
地球公轉軌道的離心率變化,約100千年一個週期,離心率愈大時近日點與遠日點接收的太陽熱能差異愈大。波峰最大約在0.05,最小約在0.02;波谷最大約在0.02,最小接近0;現在約0.0167。是冰期-間冰期循環的主要成因。
\sssc{地軸傾斜循環(Obliquity Cycle)}
地球自轉軸傾斜角度的變化,即章動(Nutation),約41千年一個週期。地軸傾角愈大,高緯度地區夏季和冬季接收太陽熱能差異愈大,愈不易將冰跨年累積,故愈不容易進入冰期,低緯度太陽輻射接收量變少,高緯度太陽輻射接收量變大,四季均變明顯;地軸傾角愈小,高緯度地區夏季接收的熱能愈少,愈不易溶冰,故愈易進入冰期,低緯度太陽輻射接收量變大,高緯度太陽輻射接收量變小,四季均變不明顯。波峰最大24.5度,波谷最小22.1度,現在23.5度。
\sssc{歲差/進動循環(Precession Cycle)}
地軸(地球自轉軸)指向的變化,北極上空觀察為逆時針向,即每年通過近日點的時間不斷提前,稱地軸歲差/進動(Axial procession),約26千年一個週期。現地軸指向北極星(勾陳一/小熊座α),13千年後將指向織女星。
\begin{itemize}
\item 赤道歲差的影響主要表現為春分點以每年約51角秒的速率連續向西運動。這種影響能夠通過回歸年與恆星年之間的差異,以及北天極和北極星相對位置的變化被直接觀察到。
\item 黃道歲差的影響則表現為春分點以每年約0.1角秒的速率向東移動,以及黃赤交角的緩慢變化。
\item 赤道歲差與黃道歲差又被統稱為總歲差(General precession)。
\item 近百萬年,北緯65度夏季日照強度(瓦每平方米)與偏心率乘以近日點黃經的正弦值大致呈正比。
\item 現北半球大陸面積較大,故當北半球夏季愈涼,即愈接近遠日點時,愈容易進入冰期。
\end{itemize}
\ssc{太陽活動}
\bct\bfH\ctr\icg[width=0.9\textwidth]{Sunspot.png}\caption{Robert A. Rohde.}\ef\FB\ect
\sssc{長期太陽輻射強度}
理論推測46億年前太陽光度約僅現在的四分之三。
\sssc{太陽黑子循環(Sunspot cycle)或太陽循環(Solar cycle)}
太陽黑子的數量、太陽風與太陽輻射強度的變化,大約11年一個週期。
\sssc{Gleissberg 循環(Gleissberg/Gleißberg cycle)}
太陽黑子循環的振幅變化,大約70到100年一個週期,或相當於七到八個太陽黑子週期。
\sssc{Suess 循環(Suess cycle/de Vries cycle)}
碳-14 代理資料中的太陽活動強弱變化週期,大約210年一個週期。
\ssc{海洋循環}
海洋循環(尤指溫鹽環流)受到鹽度分布、海陸分布等影響。
\sssc{大西洋經向翻轉環流(Atlantic Meridional Overturning Circulation, AMOC)}
AMOC 強度變化的週期大約60-80年,與極地翹翹板效應、二氧化碳濃度等有關,對氣候影響顯著,尤其是北大西洋地區。
\sssc{太平洋十年振盪(Pacific Decadal Oscillation, PDO)}
大約20-30年一個週期。特徵為太平洋北緯20度以北區域表層海水溫度的變化,在太平洋十年振盪暖相位/正相位期間西太平洋偏冷而東太平洋偏暖,在冷相位/負相位期間西太平洋偏暖而東太平洋偏冷。
\sct{代用指標(Proxy)}
\ssc{試用時間範圍}
\begin{itemize}
\item 海洋沉積:海洋形成至今。
\item 冰層、珊瑚:約10$^5$年尺度。
\item 湖泊沉積:約10$^4$年尺度。
\item 化石:生物出現至今。
\item 樹輪(樹木年輪):約10$^3$年尺度。
\end{itemize}
\ssc{冰芯}
\sssc{冰芯中封存的氣泡}
格陵蘭和兩極冰蓋與冰川中鑽探出來的冰芯中常封存過去的氣泡,可得當時氣體含量比例。
\sssc{冰芯碳、氧同位素比例}
可推知古氣溫。
\ssc{岩石}
\sssc{岩鹽、石膏}
代表乾燥。
\sssc{冰磧石}
代表冰川。
\sssc{鐘乳石與海洋沉積物碳、氧同位素比例}
可推知古氣溫或海溫。
\ssc{生物與其化石}
\sssc{植物化石}
沉積物與沉積岩中可得化石等生物代用指標,孢子和花粉等植物化石可得當時植被。
\sssc{珊瑚}
\begin{itemize}
\item 珊瑚骨骼微量元素比例可推知當時水溫。
\item 從珊瑚分布可推知當時溫暖清澈的低緯度海域的分布位置。
\item 生長紋愈接近水面愈新,可推知海溫變化。
\item 其高度可重建海平面高度改變。
\end{itemize}
\sssc{樹輪}
愈外愈新,愈暖溼的環境樹輪愈寬、顏色愈淺,一寬一窄為一年。
\sssc{仙女木(Dryas)}
是一種在北極高山地區定居的指示物種。各次仙女木事件地層的特徵是花粉和仙女木其他殘留物的豐度。
\ssc{孑遺(Relict)}
孑遺,又稱殘存、殘遺、殘留,指某一地質年代很普遍或非常多樣化的生物群或類群,後來幾乎滅絕,現僅殘留個別物種的現象。\\
廣義上,孑遺的種稱為孑遺種,孑遺的屬稱為孑遺屬,孑遺的生物稱孑遺生物。\\
狹義上,孑遺種指的是因新生代第四紀發生了多次冰川活動,地球北方物種擴張領域至現代溫暖地帶,當冰川活動結束或後續地殼變動,其分布縮減至孤立地區形成孤立族群,大多數或全部其他親緣類群或族群卻滅絕,只剩這些少數孤立族群繼續生存演化,而保存至今的古老物種。由於長時期演化的結果,現今的孑遺種與其遠古時期原初物種可能有不同程度的差異。
\ssc{氧-18相對於氧-16的比例(δ$^{18}$O)}
O-16的質量比O-18輕且較常見,蒸發O-18形成的\ce{H2O}需要較多能量,故氣溫愈低,冰愈多,海平面愈低,能抵達極區成為積雪而非在此前便凝結成雨或雪降下的O-18愈少,因此冰芯中的δ$^{18}$O愈低,深海岩芯的δ$^{18}$O愈高。從冰芯與深海岩芯內的δ$^{18}$O變化便可大約知道地球過去的氣溫情況。
\sssc{冰芯與岩芯氧同位素紀錄}
最直接的氧同位素定年方法是數算O-18與O-16比例的年際變化(冰芯夏季高於冬季、深海岩芯冬季高於夏季),亦可從其他氣候記錄中辨認出一些特別的事件(例如火山爆發),以校正氧同位素記錄的時間尺度,從而重建出過去數萬年的氣溫時間序列。
\sssc{海洋同位素階段(Marine isotope stages, MIS)/海洋氧同位素階段(Marine oxygen-isotope stages)/氧同位素階段(Oxygen isotope stages, OIS)}
根據深海岩芯樣本的氧同位素數據推導出來的地球古氣候的暖冷交替時期。從現在(MIS 1)向以前推算,偶數階段的O-18含量較高,代表冰期;奇數階段的O-18含量較低,代表間冰期。這些數據來自鑽探的海洋沉積物岩心、腐泥中的花粉、有孔蟲遺骸等代理。
\ssc{冰雪覆蓋範圍擴大的效應}
\begin{itemize}
\item 反照率增加的正回饋效應。
\item 降溫使水循環變慢而使氣候變乾、沙漠擴大、森林縮小。
\item 陸上冰雪累積會使海水變少,海平面下降,陸塊重量之改變並可能影響板塊運動。
\item 降溫造成海水密度變大,海平面下降。
\end{itemize}
\ssc{氣溫變高的效應}
\begin{itemize}
\item 森林增加,二氧化碳濃度減少,導致溫室效應減弱,但反照率下降使地表吸收太陽輻射變多。
\item 融冰入海與海水升溫膨脹使海平面上升。
\item 蒸發量上升與對流旺盛,利於雲的產生,增加地表輻射反射回地表而減少其散失到太空,增加氣溫,但氣溶膠增加又造成氣溫下降效應。
\item 水循環加快,造成一些地區的旱澇加劇。
\end{itemize}
\sct{地球古氣候的分期}
\ssc{溫室地球(Greenhouse Earth)與冰室地球(Icehouse Earth)}
\begin{itemize}
\item 縱觀地球古氣候,其氣候在兩個主要狀態之間波動:溫室地球和冰室地球,兩種氣候狀態都會持續數百萬年。
\item 溫室地球指地球上任何地方都不存在大陸冰川的時期,其餘時期為冰室地球。
\item 地球歷史大部分的時期都處於溫室地球時期。
\item 溫室期和冰室期透過直接和間接地迫使生物在不同時間尺度上進行適應和更替,在地球生命的演化中發揮了關鍵作用。
\end{itemize}
\subsubsection{冰河時期(Ice Age)}
冰河時期又稱大冰期、冰川期、冰河期,是指地球大氣和地表長期低溫導致極地和山地冰蓋大幅擴展甚至覆蓋整個大陸的時期,與冰室期(Icehouse period)/冰室地球(Icehouse Earth)的時期大致相同。冰河時期內部又分為幾次冰期(Glacial period、Glacials或Glaciations)與間冰期(Interglacials)。地球氣候史上有五個已知的冰河時期:休倫冰河時期、成冰紀冰河時期、安第斯-撒哈拉冰河時期、卡魯冰河時期、晚新生代冰河時期,其餘均為溫室地球。有時亦將冰河時期中的冰期稱為冰河時期,此不用。
\subsection{千年尺度週期之事件}
\subsubsection{邦德事件(Bond events)}
指全新世(Holocene)發生的北大西洋地區的冰筏沉積事件(Ice rafting events)。冰筏沉積指冰山崩解和融化時,將冰中攜帶的岩石碎片(冰筏沉積物)帶到海洋。大約1000到1500年發生一次。
\subsubsection{丹斯高-厄施格事件/D-O 事件(
(Dansgaard–Oeschger events, D-O events)}
指在末次冰河期(Last Glacial Period, LGP)發生了 25 次的快速氣候波動。一些科學家表示,這些事件是準週期性發生,週期約為 1470 年。D-O事件的紀錄主要基於格陵蘭和南極冰芯中的氧同位素等數據。D-O 事件的特徵為:
\begin{itemize}
\item 北半球快速暖化:格陵蘭地區的氣溫在數十年內上升約8攝氏度。北半球和南半球溫度變化可能不同步的現象稱極地翹翹板(Polar seesaw)/兩極翹翹板(Bipolar seesaw)。
\item 溶氧暖化:溶氧事件導致的暖化進一步向南延伸,全球氣溫快速上升,進入間冰期。
\item 隨後的緩慢降溫:在暖化之後,氣溫會在幾百年到幾千年的時間內緩慢下降,回到較冷的冰期條件。
\item 部分 D-O 事件伴隨著 Heinrich 事件,但後者的週期似乎較長。
\end{itemize}
D-O事件的成因可能為:
\begin{itemize}
\item 大西洋經向翻轉環流變化可能與溶氧暖化事件有關。
\item D-O事件可能與太陽活動的週期循環有關。
\end{itemize}
\sssc{Heinrich 事件(Heinrich event)}
大群冰山從勞倫泰德冰蓋上脫落並穿過哈德遜海峽進入北大西洋。\\
冰蓋的狂暴-清除循環(Binge-purge cycle)可能是Heinrich事件的成因。
\subsection{冰河時期中的冰期(Glacial period)-間冰期(Interglacial period)循環}
冰期-間冰期循環略與米蘭科維奇地球公轉軌道偏心率循環重疊。
\sssc{冰期}
指在一個冰河時期中,一段持續的全球低溫、大陸冰蓋大幅度向赤道延伸的時期。\\
冰期時氣溫寒冷,其中同一半球緯度愈高氣溫下降愈多,高低緯度溫差擴大,使大氣與海洋環流加強,大氣飽和水氣壓較低,天氣乾燥少雨,極地冰原擴大,大量的水以冰雪形式留存在陸地上,導致平均海平面下降。冬季海陸溫差較大。\\
海冰融化對海平面高度無顯著影響,大陸冰層融化會造成海平面高度上升。
\sssc{間冰期}
指兩次冰期之間,全球溫度較高,大陸冰蓋大幅度消融退縮的時期。\\
間冰期時氣溫溫暖,大氣飽和水氣壓較高,天氣潮溼多雨,冰原消融,導致平均海平面上升。高低緯度溫差變小,影響環流系統。
\sssc{冰期-間冰期週期}
冰期-間冰期循環大約10萬年一個週期,與米蘭科維奇循環的離心率循環大致重疊。
\sssc{冰段(Stadial)/副冰期/冰階}
指間冰期內部的一段低溫時期,其時長或溫度不足以被認定為冰期。如老仙女木期,新仙女木期以及從16世紀持續到19世紀的小冰期。
\sssc{間冰段(Interstadial)}
指冰期內部的一段溫暖時期,其時長或溫度不足以被認定為間冰期。如 Bølling–Allerød 間冰段。
\sct{地球古氣候史}
\ssc{早期地球演化階段}
\sssc{熔岩海洋(magma ocean)}
當地球在約46億年前形成時,此時太陽光度約僅現在的四分之三,由於重力的作用,物質在中心聚集,形成高溫高壓的熔融狀態。其熱源包含重力收縮的位能變成熱能、隕石撞擊的動能變成熱能與內部放射性元素核分裂的質量變成熱能。
\sssc{最早大氣(earliest atmosphere)/第一大氣(first atmosphere)}
地球形成初期,重力聚集了原始大氣,主要包括氫、氦、甲烷和氨等。由於地球重力無法有效束縛氫、氦,隕石撞擊也使甲烷和氨得以獲得足夠能量脫離地球引力,使此原始大氣無法保留。
\sssc{冷卻(cooling)}
重力收縮逐漸穩定,隕石撞擊因為太陽系行星逐漸形成、殘餘物質逐漸吹散而減少,表面散熱,使得地球開始冷卻。但內部核分裂熱源仍然存在至今。
\sssc{分層(differentiation)}
隨著地球逐漸冷卻,密度較大的物質(如鐵和鎳)向中心沉降,形成地核。較輕的物質浮在表面,形成矽酸鹽為主的地函和矽氧化物為主的地殼。這一過程導致了地球內部的分層結構。
\sssc{第二大氣(second atmosphere)/前生物大氣(prebiotic atmosphere)}
主要由火山活動釋放出的氣體組成,水蒸氣最多,二氧化碳次多,其餘如二氧化硫、氮氣等。當時溫度使水主要以水氣存在而非液態。
\sssc{海洋(oceans)}
隨著地球表面逐漸冷卻,地球表面冷卻到水的凝結點以下,大量的水蒸氣凝結形成液態水,積聚在地表低窪處,形成原始海洋。這一過程可能持續了數百萬年。40億年前,最古老的遺留至今的沉積岩形成,可知海洋形成。大量源自於火山噴發的二氧化碳、二氧化硫等自大氣溶解於海洋中(大氣圈 $\rightarrow$ 水圈),提供Cl$^-$、SO4$^{2-}$、CO3$^{2-}$等陰離子。降水沖刷岩石造成風化,使Na$^+$、Ca$^{2+}$、Mg$^{2+}$等陽離子溶入原始海洋。該等離子形成了海洋中的鹽類。除碳作用包含二氧化碳自大氣溶解於水中(大氣圈 $\rightarrow$ 水圈)、碳酸鹽類沉積為沉積岩(水圈 $\rightarrow$ 岩石圈),如石灰岩、生物形成化石與生物礁石石灰岩(水圈 $\rightarrow$ 生物圈 $\rightarrow$ 岩石圈)。地球初期的除碳作用使地球上最大的碳儲存庫為岩石圈。由於氮氣化性不活潑,難溶於水,當原始海洋的形成使大氣中的水蒸氣、二氧化碳和二氧化硫濃度逐漸降低,即造成以,氮氣之比例逐漸提高,並遺留至今,使現今大氣中氮氣約占百分之七十八。而最初的氧氣可能源自於紫外線照射水氣而成。
\sssc{第三大氣(Third atmosphere)}
大約35億年前,藍綠菌(Cyanobacteria)/藍藻開始出現在地球海洋中。這些早期的光合作用生物能夠通過光合作用利用陽光將二氧化碳和水轉化為有機物,釋放出氧氣。疊層石為藍綠菌與泥沙混合形成之化石,澳洲西部有之。初期在淺海生成的氧氣被海中鐵離子等物質消耗掉,而後大氣中的氧氣濃度逐漸增加,最終形成了接近地球現代的大氣組成,這一過程被稱為大氧化事件,約開始於大約24億年前。紅褐色的帶狀鐵礦(縞狀鐵礦)大約形成於24億年前到18億年前之間的地質時期,成分為三氧化二鐵,係鐵在水中氧化混以硫化物、碳酸鹽類等形成的沉積層,可作為淺海中已有充足溶氧的證據,直到18億年前海水中大部分鐵離子幾乎氧化耗盡不再累積,而大氣中氧的比例也在約20億年前開始累積並顯著上升。帶狀鐵礦也是今天鐵礦的重要來源。
\sssc{陸地生物}
初期的生物只能從生存在水裡,因上岸會遭受紫外線襲擊,當氧氣進入大氣,受短波輻射照射分解為氧原子,又與氧氣結合成臭氧。4億年前的地層中首見陸地生物的化石,代表生物已能脫離海水的保護,其中重要原因是臭氧層的保護使陸地免於紫外線傷害。陸地生物的光和作用與呼吸作用使大氣氧氣比例逐漸達成動態平衡,與氮氣達成長期穩定的比例。
\subsection{溫室與冰室期}
\bct\bfH\ctr\icg[width=0.9\textwidth]{542Myr.png}\caption{Dragons flight, 2005.}\ef\FB\ect\bct\bfH\ctr\icg[width=0.9\textwidth]{65Myr.png}\caption{Robert A. Rohde, 2005.}\ef\FB\ect
\subsubsection{新太古代冰河時期/休倫冰河時期(Huronian Ice Age)}
出現於24億年前到21億年前。這可能是地球上最嚴重最漫長的寒冷期。其成因可能是大氧化事件,大氣層中急劇增加的氧氣破壞了原始大氣中的主要溫室氣體甲烷與二氧化碳所致。
\subsubsection{前寒武紀冰河時期/成冰紀冰河時期(Cryogenian Ice Age)}
出現於元古宙新元古代成冰紀,從8.5億年前到6.3億年前。極地冰蓋擴展到赤道,海洋也完全凍結,全球冰封,形成雪球地球。火山噴發的二氧化碳因地球生物不能光合作用而逐步累積,最終形成的溫室效應使得地球走出冰封。隨後是埃迪卡拉生物群(Ediacaran Biota)標誌著多細胞生物的出現,以及寒武紀生命大爆發出現了現有所有的門的生物。
\sssc{雪球/團地球(Snowball Earth)假說}
前寒武紀冰河時期冰漬殘留地點廣布,全球遍布冰川痕跡,但是在赤道地區除了高山地區外找不到冰川痕跡。該理論認為地球曾經一度被全部冰凍起來。由於當時超大陸集中在赤道附近,陸地表面冰川很難被全部覆蓋,日積月累當風化過程慢慢消耗完溫室氣體後,冰川開始蔓延,直到全球凍結。整個地球表面全部被冰川覆蓋,地球的日照輻射反射率上升。由於水氣蒸騰作用被大幅減少,大氣變得較乾燥,其餘溫室氣體也較少。至於其消融,則是由於冰川的延伸,導致部分陸地被冰面覆蓋,從而延緩了岩石的風化過程(岩石風化會吸收二氧化碳),導致溫室氣體(例如火山活動產生)累積,溫室效應增強,因此冰川到一定時候時停止蔓延。
\subsubsection{早古生代冰河時期/安第斯-撒哈拉冰河時期(Andean-Saharan Ice Age)}
出現於古生代晚奧陶紀與志留紀,從4.6億年前到4.3億年前。
\subsubsection{晚古生代冰河時期(Late Paleozoic Ice Age)/卡魯冰河時期(Karoo Ice Age)}
出現於古生代末期的石炭紀與二疊紀,從3.6億年前到2.6億年前。可能的原因是在此前的泥盆紀陸生植物大量繁育,導致地球大氣中氧含量的增加、二氧化碳的大幅減少所致。
\subsubsection{晚新生代冰河時期(Late Cenozoic Ice Age)/南極冰河時期(Antarctic Glaciation)}
\begin{itemize}
\item 晚新生代冰河時期開始於339萬年前(另一說為約2000萬年前南極大陸出現永久冰蓋之時開始),延續至今。其中開258萬年前至今又稱第四紀冰河時期(Quaternary glaciation)/更新世冰河時期(Pleistocene glaciation)。當前冰河時期(Current ice age)/末次冰河時期(Last ice age)指晚新生代冰河時期或第四紀冰河時期。
\item 此次冰河時期之產生與喜馬拉雅山造山運動改變大氣環流與亞洲氣候、海洋鹽度改變阻隔北方暖水形成南極繞極環流、南極冰原擴張產生冰雪覆蓋的正回饋效應等有關。
\item 此次冰河時期中,地球處於冰期與間冰期交替出現。在此期間,南極大陸與格陵蘭島形成了永久性的冰蓋,歐亞大陸北部與北美的北部也出現了面積廣大的大陸冰蓋,並在世界各處高緯度或高海拔地區廣泛出現了山嶽冰川。受到南極與南美洲間海峽打開產生南極繞極環流、大陸環繞北冰洋、南北美大陸匯合改變洋流等因素影響。
\end{itemize}
\subsection{倒數第二次冰期至今氣候史}
\bct\bfH\ctr\icg[width=0.9\textwidth]{Holocene.png}\caption{Robert A. Rohde.}\ef\FB\ect\bct\bfH\ctr\icg[width=0.9\textwidth]{MIS.jpg}\caption{Tobias Lauer \& Marcel Weiss, 2018. Timing of the Saalian- and Elsterian glacial cycles and the implications for Middle – Pleistocene hominin presence in central Europe. \href{https://www.nature.com/articles/s41598-018-23541-w}{https://www.nature.com/articles/s41598-018-23541-w}.}\ef\FB\ect\bct\bfH\ctr\icg[width=0.9\textwidth]{LGM.jpg}\caption{Daniel E. Platt, Marc Haber, Magda Bou Dagher-Kharrat, Bouchra Douaihy, Georges Khazen, Maziar Ashrafian Bonab, Angélique Salloum, Francis Mouzaya, Donata Luiselli, Chris Tyler-Smith, Colin Renfrew, Elizabeth Matisoo-Smith \& Pierre A. Zalloua, 2009.}\ef\FB\ect\bct\bfH\ctr\icg[width=0.9\textwidth]{2kyr.png}\caption{RCraig09, 2020.}\ef\FB\ect
\subsubsection{倒數第二次冰期(Penultimate Glacial Period, PGP)/海洋同位素階段 6(Marine Isotope Stage 6, MIS6)}
開始於約194至191千年前,結束於約135千年前。
\subsubsection{末次間冰期(Last Interglacial Period, Eemian)}
開始於約130至125千年前,結束於約117千至115千年前。
\subsubsection{海洋同位素階段 5(Marine Isotope Stage 5, MIS5)}
開始於約135至130千年前,結束於約71千年前。
\subsubsection{末次冰期(Last Glacial Period, LGP, Last glacial cycle)}
開始於約117千至115千年前,結束於約11.7千年前新仙女木事件結束時。其中最冷的時期為末次冰盛期。
\subsubsection{海洋同位素階段 4(Marine Isotope Stage 4, MIS4)}
開始於約71千年前,結束於約57千年前。
\subsubsection{海洋同位素階段 3(Marine Isotope Stage 3, MIS3)}
開始於約57千年前,結束於約29千年前末次冰盛期開始時。
\subsubsection{末次冰盛期(Last Glacial Maximum, LGM)}
\begin{itemize}
\item 開始於約30-27千年前,結束於約18千年前。是末次冰期中最冷的時期。冰蓋在大約26千年前達到頂峰,冰川消融開始於20千至19千年前。
\item 全球平均氣溫較現今低約6度。
\item 臺灣平地氣溫類似於溫帶氣候,冬夏溫差可達9攝氏度(現在約5攝氏度),中高緯度生物生長範圍延伸至此,並在冰期結束後往中高海拔遷徙或滅絕,未滅絕者成為孑遺種,如山毛櫸、櫻花鉤吻鮭等,臺灣高山有冰河,今可見冰斗、U型谷、角峰、冰磧物等冰河地形為證,如雪山圈谷。
\item 北極冰原擴張到今美國紐約處(北緯37度)。
\item 靈長類動物逐漸演化,出現智人,並從發源地非洲向世界各地遷徙。
\item 平均海平面較現在低約120公尺,亞洲大陸附近水深不到100公 尺的部分區域露出海面,陸生生物可經由這些通道遷徙至臺灣、日本、 婆羅洲、蘇門答臘島、爪哇島等,臺南左鎮和澎湖水道犀牛、猛獁象等大型哺乳動物化石可為證。
\item 白令海峽露出海面成為陸地通道,亞洲北部人類遷徙至美洲可為證。
\item 冰川擴張到約緯度45度。
\end{itemize}
\subsubsection{最老仙女木(Oldest Dryas)事件}
開始於約18.5至17千年前,結束於約15-14千年前。溫度下降。
\subsubsection{Bølling–Allerød 間冰段(Bølling–Allerød Interstadial)與老仙女木(Old Dryas)事件}
Bølling–Allerød 間冰期,開始於約14.7千年前,結束於約12.9千年前,北半球快速變暖。其中約14千年前發生老仙女木事件,溫度短暫降低 ,但僅在部分地區找到證據。
\subsubsection{海洋同位素階段 2(Marine Isotope Stage 2, MIS2 }
開始於約29千年前,一般認為結束於約11.7千年前新仙女木事件結束時,亦有認為結束於約14千年前老仙女木事件結束時。
\subsubsection{新仙女木(Younger Dryas, YD)事件}
\begin{itemize}
\item 開始於約12.9千年前,結束於約11.7千年前,約在12千年前接近溫度谷值。
\item 北半球急劇降溫,北美年均氣溫下降了約3攝氏度,歐洲下降了約2-6攝氏度,格陵蘭降溫約10攝氏度,且在幾十年內發生 ,其中格陵蘭降溫尤迅速,可能在3年內完成,北極冰原再次覆蓋歐洲和北美的中高緯度地區。
\item 南半球變暖。
\item 全球淨降溫約0.6攝氏度。
\item 主要成因為北美洲冰原快速消融,將淡水注入大西洋,導致表層海水鹽度大幅降低、密度減小,無法產生下層的海流,阻礙低緯度暖水北上 (北大西洋暖流),導致AMOC與溫鹽環流減弱與中止,使北美和歐洲中高緯度大陸缺乏熱量迅速冷凍,冰原快速擴張,發生冰雪覆蓋的正回饋效應。
\end{itemize}
\subsubsection{海洋同位素階段 1(Marine Isotope Stage 1, MIS1)}
一般認為開始於約11.7千年前新仙女木事件結束時,亦有認為開始於約14 千年前老仙女木事件結束時,尚未結束。
\subsubsection{全新世暖期(The Holocene warm period)}
\begin{itemize}
\item 開始於約11.7千年前,尚未結束。
\item 約11.7千年前開始,北大西洋鹽度恢復正常,AMOC變化不再能抵銷二氧化碳的增加,溫鹽環流再次流動,新仙女木事件結束,經歷劇烈變暖,將地球從更新世末次冰盛期過渡到全新世暖期(間冰期)。
\item 冰原快速消融,海平面快速回升,許多冰期露出的低窪陸地再次沒入海中,如臺灣海峽。
\item 臺灣海岸線位置往山區移動,草原和沼澤沒入海中,原先遷居臺灣的大型陸生生物,例如早坂犀牛,被迫遷往丘陵地帶 ,最終因棲地消失而滅絕。
\item 過去一萬年間,海平面變化較小,臺灣因山高水急,風化、侵蝕作用強烈,大量沉積物堆積在河口,使得海岸線往外移動,例如一萬年前彰化至臺南的海岸線逼近丘陵地帶,如今距離丘陵已有幾十公里, 沖積平原增加,蘭陽平原亦增加。
\item 今日臺灣高山地區夏季氣溫高於零攝氏度,冬季降雪無法越年累積形成冰河。
\end{itemize}
\sssc{全新世氣候最適宜期(Holocene climatic optimum)/全新世大暖期}
開始於約9千年前,結束於約5千年前,而後氣溫逐漸下降,直到約2千年前。北極地區氣溫升高約4°C,北半球變暖,南半球可能與現在相當。可能是由於9年前地軸傾角達到了24°,在北半球的夏季更朝向於太陽。
\subsubsection{中世紀溫暖時期(Medieval Warm Period, MWP)/中世紀氣候異常期(Medieval Climatic Anomaly)/中世紀氣候最佳期(Medieval Climate Optimum)/中世紀暖期}
約在公元950至1250年。是一個全球範圍內的相對溫暖的時期,特別是北半球,如歐洲和北美,使北歐地區農業擴展、歐洲人口增長、維京人在格陵蘭島上建立定居點。
\subsubsection{小冰期(Little Ice Age, LIA)}
約在公元1400年到1850年。是一個氣溫顯著下降的時期,特別是在北半球,歐洲、北美和亞洲冬天變冷,歐洲的河流在冬季會結成厚冰,冰川擴展,格陵蘭島的維京人定居點被迫遷移甚至消失,臺灣有雞籠積雪的紀錄。推測與當時太陽輻射能量偏弱有關,觀測記錄顯示當時太陽黑子數量處於極小值。
\sssc{人類活動與近期氣候變化}
\bct\bfH\ctr\icg[width=0.9\textwidth]{CO2.png}\ef\FB\ect\bct\bfH\ctr\icg[width=0.9\textwidth]{greenhouse.png}\ef\FB\ect\bct\bfH\ctr\icg[width=0.9\textwidth]{TAnomaly2023.png}\caption{C3S/ECMWF, 2023.}\ef\FB\ect\bct\bfH\ctr\icg[width=0.9\textwidth]{SnowCover.jpg}\ef\FB\ect\bct\bfH\ctr\icg[width=0.9\textwidth]{SeaLevel.jpg}\ef\FB\ect\bct\bfH\ctr\icg[width=0.9\textwidth]{Polar.jpg}\caption{Hugo Ahlenius, UNEP/GRID-Arendal, 2016. Maps of average sea ice extent in the Arctic summer (September) and winter (March), and in the Antarctic summer (February) and winter (September). \href{https://www.grida.no/resources/5239}{https://www.grida.no/resources/5239}.}\ef\FB\ect
\begin{itemize}
\item 根據地質紀錄、文獻、儀器觀測資料等重建近1500年來的氣溫紀錄,發現自18世紀以來雖經多次冷暖交替,但整體上升約1攝氏度,較過去1500年來各時期都要顯著。
\item 過去一百年全球均溫上升約0.8攝氏度,過去三十年全球均溫上升約0.5攝氏度。
\item 以1961至1990年平均溫度為基準,2000年溫度有正離均差約0.5攝氏度。
\item 根據聯合國政府間氣候變遷專門委員會(Intergovernmental Panel on Climate Change, IPCC)於2013年公布 的第五次氣候評估報告指出,近百年來各地地表溫度普遍上升,部分地區上升達2攝氏度,陸地冰雪與海冰覆蓋面積的減少使全球平均海平面高度上升。
\item 全球平均溫度 9 年移動平均線自1903年至1940年及自1971年至2015均呈向上為主,自1880至2015年平均每年上升0.007攝氏度 ,自1986年至2015年每年平均上升0.016攝氏度,2015年氣溫較1901至2000年平均值13.9攝氏度高出0.9攝氏度,為1880年以來最暖,近30年增溫約0.4度。
\item 臺灣13站平均平地溫度自1898年至2015年每年平均上升0.012攝氏度,自1986年至2015年每年平均上升0.020攝氏度,1998年氣溫、2015年分別較1901至2000年平均值23.0攝氏度高出1.35攝氏度、1.27攝氏度 ,近30年增溫約0.6度。
\item 自1760年代工業革命起,人類活動如使用化石燃料、農牧等開始排放溫室氣體,二氧化碳、甲烷、一氧化二氮等濃度已較18世紀顯著增加。工業革命前二氧化碳濃度約280ppm,一氧化二氮濃度約260ppb,甲烷濃度約790ppb;公元2019年時二氧化碳濃度約400ppm,一氧化二氮濃度約330ppb,甲烷濃度約1870ppb。
\item 根據氣候模擬,近50年來的增溫趨勢除了自然因素外,人為因素也造成影響。
\item 2018年全球平均海平面高度較1993年上升約9公分。
\item 近年,熱浪、暴雨、強颱、乾旱等極端天氣發生頻率增加,病媒蚊棲息地往溫帶擴張。
\end{itemize}
\end{document}