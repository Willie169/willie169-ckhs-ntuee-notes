\documentclass[a4paper,12pt]{article}
\setcounter{secnumdepth}{5}
\setcounter{tocdepth}{3}
\input{/usr/share/LaTeX-ToolKit/template.tex}
\begin{document}
\title{The Immune System}
\author{沈威宇}
\date{\temtoday}
\titletocdoc
\section{The Immune System}
\ssc{Introduction}
\bit
\item\Tb{Pathogens}: Agents that cause disease, such as bacteria, funguses, or viruses.
\item\Tb{Immune system}: Dedicated cells that interact with and destroy foreign molecules or cells. Note that a foreign molecule or cell doesn’t have to be pathogenic to elicit an immune response, but we’ll focus on the immune system’s role in defending against pathogens.
\item\Tb{Innate immunity}: All animals have innate immunity, a defense activated immediately upon infection.
\item\Tb{Innate recognition}: Each of a small set of receptors recognizes a molecule absent from animals, but common to a type of pathogen, such as dsRNA (nucleic acid in genome of viruses), flagellin (protein in flagella of bacteria), mannan (oligosaccharide in cell wall of fungus).
\item\Tb{Adaptive immunity}: Vertebrates have adaptive immunity, a defense activated after the innate response and develops more slowly.
\item\Tb{Adaptive recognition}: Each of a vast number of receptors is specific for a particular part of a protein in one pathogen, such as a amino acid sequences of surface protein of the influenza (flu) virus.
\eit
\ssc{Innate Immunity}
\sssc{Barrier defense}
