\documentclass[a4paper,12pt]{report}
\setcounter{secnumdepth}{5}
\setcounter{tocdepth}{3}
\newcounter{ZhRenew}
\setcounter{ZhRenew}{1}
\newcounter{SectionLanguage}
\setcounter{SectionLanguage}{1}
\input{/usr/share/latex-toolkit/template.tex}
\begin{document}
\title{遺傳學}
\author{沈威宇}
\date{\temtoday}
\titletocdoc
\chapter{遺傳學(Genetics)}
\section{詞彙釋義}
\begin{itemize}
\item 性狀(Trait)/表現性狀(Phenotypic trait)/特徵(Character):指生物體的特徵。例如人眼睛的顏色。
\item 表徵:同一性狀所表現的不同。例如人眼睛有藍色、黑色、棕色。
\item 基因型(Genotype):生物體內操縱某一性狀的基因的不同。
\item 表現型(Phenotype):/表型:基因型所表現的表徵,是基因型的函數。
\item 遺傳(Heredity):性狀從父母傳遞給後代的過程。 
\item 遺傳(Inheritance)描述了性狀及其從一代到另一代的傳遞的途徑。
\item 遺傳學:研究生物體的遺傳和變異的科學。
\item 顯性(Dominance)和隱性(Recessive):是一個基因座中一對等位基因之間的關係,其中一個等位基因的表型會表現出來,掩蓋了同一基因座中另一個等位基因的表型,則前者稱顯性基因,一般記作$P$或其他大寫字母,後者稱隱性,一般記作$p$或其他小寫字母。
\item 遺傳因子(Genetic factor):即今基因。
\item 基因(Gene):編碼功能產品訊息的 DNA 序列片段,無論該產品為蛋白質或 RNA 分子。一些認為應包含非編碼 DNA 序列片段。
\item 基因的長度一般以鹼基對的數目呈現,如人類角蛋白基因約有3800個鹼基對。
\item 等位基因(Allele):指同源染色體上對位的基因。
\item 基因座(Gene locus):指某個基因或具有調控作用的遺傳標記在染色體上所處的特定位置。
\item 基因圖譜(Genetic map):所有基因座在基因組中的排列位置。
\item 同型(Homozygous)/純合子與異型(Heterozygous)/雜/異合子:在同一基因座上有相同的等位基因稱同型合子,否則稱異型合子。
\item 環境因子對性狀的影響:環境因子會影響性狀的表現。例如繡球花花色受土壤酸鹼性影響、喜瑪拉雅兔毛色受氣溫影響。
\end{itemize}
\section{孟德爾遺傳(Mendelian inheritance)}
\subsection{孟德爾豌豆實驗}
孟德爾嘗試豌豆、山柳菊、玉米等,以豌豆最成功。因豌豆採用自花授粉(豌豆雄蕊與雌蕊包被於花瓣中,使雌蕊僅能接受同朵花雄蕊的花粉),且未開花前已完成授粉,可避免外來花粉干擾。便於人工異花授粉(以人工方式在雄蕊可以授粉前移除,再以筆刷沾取其他豌豆花雄蕊的花粉施於此花的雌蕊),以進行雜交。此外,豌豆生命週期短(播種後約3週即可開花結果)、容易大量栽植、子代數目多(利於統計分析)、有多種性狀具差異明顯的表徵可供觀察。\\
孟德爾選出以下七項豌豆之性狀:
\begin{center}
\begin{tabular}{|c|c|c|c|c|c|c|c|}
\hline
性狀 & 種皮顏色 & 成熟種子形狀 & 未成熟豆莢顏色 & 成熟豆莢形狀 & 花色 & 花位置 & 莖高度 \\ \hline 
顯性 & 黃色 & 圓滑 & 綠色 & 飽滿 & 紫色 & 腋生 & 高 \\ \hline 
隱色 & 綠色 & 皺皮 & 黃色 & 扁皺 & 白色 & 頂生 & 矮 \\ \hline
\end{tabular}
\end{center}
\begin{itemize}
\item 自交:自花授粉。
\item 純品系:指經過多代自交或互交產生的後代表現型均相同,可知為同型合子。
\item 單性狀雜交實驗:以兩種表現型的純品系豌豆作為親代P,第一子代F$_1$全為基因型一顯性一隱性、表現型顯性,第二子代F$_2$理論表現型顯性:隱性為3:1、基因型二顯性:一顯性一隱性:二顯性為1:2:1。觀察第一子代之表現型即可知何者為顯性。
\item 試交:將表現型為顯性表徵者與隱性純品系交配,若子代有隱性表現型者可知該顯性親代基因型為一顯性一隱性,否則為二顯性。孟德爾由單性狀雜交實驗推論其中第一子代為一顯性一隱性的基因型,即以試交驗證。
\item 互交:將原先實驗的雄性與雌性互換。如原先雄性紫花與雌性白花雜交,則互交為雄性白花與雌性紫花雜交。互交實驗表示雌雄親代對子代遺傳表徵的貢獻相同。
\item 雙性狀雜交實驗:取在兩種不同性狀皆為對比表徵的植株雜交,發現兩性狀的遺傳因子遺傳至後代係獨立而互不干擾的。
\end{itemize}
\subsection{孟德爾定律成立前提}
\begin{itemize}
\item 性狀是單調(Unitary)的,即離散的,例如白花相對於紫花,而不可以有中間顏色的花。
\item 性狀有多種基因型。
\item 兩個同源染色體上的等位基分別遺傳自父母。
\item 一個等位基因比另一個等位基因占主導地位,即表型反映了顯性等位基因。
\item 配子是透過隨機分離產生的。異型合子產生配子時,配子中含有兩個等位基因頻率相同。
\item 基因彼此是不相關的。
\end{itemize}
\subsection{棋盤方格法(Punnett Squares)}
Reginald Punnett 創建,以解釋孟德爾實驗與分析子代基因型和表現型。 例如:若親代的遭傳因子組合為 Pp,其中P為顯性、p為隱性 ,產生配子時,此二個遺傳因子會分配到不同配子中,即產生P配子與p配子的機會各為0.5;同理,若另一親代的遭傳因子組合亦為 Pp,產生P、p兩種配子的機會亦各為0.5。則來自親代雙方的P配子結合的機會為0.5 * 0.5 = 0.25,依此類推,即可得到子代有 0.75的機會為顯性表徵,0.25的機會為隱性表徵。
\begin{center}
\begin{tabular}{|c|c|c|}
\hline
& P & p \\ \hline
P & PP & Pp \\ \hline
p & Pp & pp \\ \hline
\end{tabular}
\end{center}
\subsection{孟德爾定律(Mendel's laws)}
\subsubsection{孟德爾第一定律}
\begin{itemize}
\item 顯性和統一律(Law of dominance and uniformity):簡稱顯性律。決定生物體性狀表徵的遺傳因子有兩種,一種為顯性遺傳因子,另一種為隱性遺傳因子。每種性狀都是由一對遺傳因子(即今等位基因)決定,若決定表徵的一對遺傳因子分別為隱性與顯性,則表現顯性特徵。
\item 分離律(Law of segregation):生物體在形成配子時,原來成對的遺傳因子會分離,因而配子僅含其中一個遺傳因子。當雌雄配子結合時,遺傳因子又恢復成對。
\end{itemize}
\subsubsection{孟德爾第二定律}
獨立分配律(Law of independent assortment):形成配子時,控制不同性狀之遺傳因子的分離時獨立的,不受其他種遺傳因子的影響,可隨機組合至同一配子中。
\subsection{非孟德爾遺傳}
孟德爾提出。指許多生物體性狀的遺傳方式與孟德爾所描述的原理不同。例如細胞質遺傳。
\section{包法利—薩登遺傳的染色體學說(Boveri–Sutton chromosome theory of inheritance)}
\begin{itemize}
\item 子代由精卵結合而來,故推知遺傳因子位於親代的精卵中。根據孟德爾實驗結果,父母對子代的貢獻相等,觀察減數分裂,發現精子細胞質遠少於卵子,但細胞核大小相同,故推知染色體為遺傳因子所在,並提出染色體存在於所有分裂細胞中,從一代傳到下一代,是所有遺傳的基礎。
\item 包法利發現,染色體數目和結構的異常會造成胚胎的發育失常。例如其在海膽實驗中發現海膽必須具有所有的染色體其胚胎才能完成正常發育。
\item 染色體在間期保持其個體性。
\item 染色體是線性結構,孟德爾所述之遺傳因子位於沿染色體上的特定位點(今基因座)。
\item 薩登發現,染色體數目遠小於性狀數目,故推知一對染色體上有多對基因。
\item 薩登發現等位基因分別位於成對的同源染色體上,位於同一染色體上的許多基因,減數分裂時,不服從獨立分配率。
\item 薩登發現,聯會前生殖細胞的染色體由兩個等價的染色體系列構成,即那些大小相同的染色體(今同源染色體),其中一個來自父親,另一個來自母親;聯會過程包括兩個同源染色體成對聯合;聯會後的第一次分裂不會產生姊妹染色分體分離;聯會後的第二次分裂會導致姊妹染色分體分離,並分配到不同的生殖細胞中。
\end{itemize}
\section{性染色體(Sex chromosome)}
\subsection{性別決定系統(Sex-determination system)}
在某些物種中,性別決定是遺傳的:雄性和雌性具有不同的等位基因,甚至不同的基因座來指定其性形態。在動物中,這通常伴隨著染色體差異,通常是透過XY、ZW、XO、ZO染色體或單倍二倍體(Haplodiploidy)的組合。性別分化通常由一個主要的性別基因座觸發。
\subsubsection{XY性別決定系統}
個體的性別是由一對性染色體決定的:雌性有兩條相同的性染色體(XX),稱為同配性別;雄性有兩種不同的性染色體(XY),稱為異配性別。雌配子總是含有X染色體,因此子代的性別取決於雄配子中存在X或Y染色體。包括人類、果蠅、銀杏等。人類的Y染色體上有SRY基因,使人類發展處男性性徵,故具有Y染色體者即為生理男性,例如XY、XXY等。
\subsubsection{ZW性別決定系統}
個體的性別是由一對性染色體決定的:雄性有兩條相同的性染色體(ZZ),稱為同配性別;雌性有兩種不同的性染色體(ZW),稱為異配性別。雄配子總是含有Z染色體,因此子代的性別取決於雌配子中存在Z或W染色體。包括鳥類、部分蝴蝶和飛蛾等。
\subsubsection{XO性別決定系統}
個體的性別是由一種性染色體的數量決定的,稱為X,字母 O(或作零)表示沒有Y染色體:雌性有兩條相同的性染色體(XX);雄性有一條性染色體(XO)。雌配子總是含有X染色體,因此子代的性別取決於雄配子中是否存在性染色體。包括蜘蛛、蜻蜓、蟑螂、蚱蜢和部分蝙蝠等。
\subsubsection{ZO性別決定系統}
個體的性別是由一種性染色體的數量決定的,稱為Z,字母 O(有時是零)表示沒有W染色體:雄性有兩條相同的性染色體(ZZ);雌性有一條性染色體(ZO)。雄配子總是含有Z染色體,因此子代的性別取決於雌配子中是否存在性染色體。包括部分飛蛾等。
\subsubsection{單倍二倍體性別決定系統}
雄性由未受精卵發育而來,為單倍體;雌性由受精卵發育而來,為二倍體。
\subsection{性染色體與常染色體}
\begin{itemize}
\item 性染色體:上述性別決定系統中的X、Y、Z、W染色體。
\item 常染色體(Autosome):不是性染色體的染色體。
\item 聯會:聯會過程中,常染色體由聯會複合體整個連接在一起,而性染色體則不參與聯會或只在偽常染色體區域(Pseudoautosomal region)發生。
\end{itemize}
\section{基因遺傳的種類}
\subsection{控制一性狀的基因數量}
\subsubsection{單基因遺傳(Monogenic inheritance)}
由一個基因決定一種性狀的表現型。可為孟德爾遺傳。
\subsubsection{寡基因遺傳(Oligogenic inheritance)}
由少數幾個基因決定一種性狀的表現型。如果表型無法預測到單一強相關基因座,但已知另一個基因座的基因型可以增加其相關性,則這是該性狀寡基因遺傳的證據。為非孟德爾遺傳。
\subsubsection{多基因移傳(Polygenic inheritance)/數量(Quantitative)遺傳}
由多個基因決定一種性狀的表現型。通常性狀的表現型呈現連續性的變化,且通常其表現型之分布接近常態分布。為非孟德爾遺傳。
\subsection{等位基因之間顯隱性關係}
\subsubsection{完全顯性(Complete dominance)}
以兩個不同種等位基因的純品系為親代,所有子代所表現的性狀和親代之一完全一樣者。可為孟德爾遺傳。
\subsubsection{不完全顯性(Incomplete dominance)遺傳/中間型遺傳}
決定一種性狀的兩種等位基因在異型合子個體中表現出的性狀是兩個同型合子性狀的中間形態。如紫茉莉花色的紅與白的中間型為粉紅色。為非孟德爾遺傳,但仍可以棋盤方格法推測。
\subsubsection{共顯性(Codominance)遺傳/等顯性遺傳}
決定一種性狀的兩種等位基因在異型合子個體中同時表現,沒有一個等位基因被另一個等位基因掩蓋。如人類ABO的AB型表現型的基因型為$I^{\tx{A}}I^{\tx{B}}$。為非孟德爾遺傳,但仍可以棋盤方格法推測。
\subsection{等位基因種類數}
\subsubsection{單等位基因遺傳}
決定一種性狀的等位基因有二種以下。可能為孟德爾遺傳。
\subsubsection{複等位基因遺傳(Multiple alleles inheritance)}
決定一種性狀的等位基因有三種以上。如人類ABO血型基因座上有$I^{\tx{A}}$、$I^{\tx{B}}$和$i$。可能為孟德爾遺傳。
\subsection{基因位置}
\subsubsection{常染色體遺傳}
決定性狀的等位基因位於常染色體上。可能為孟德爾遺傳。
\subsubsection{遺傳連鎖(Genetic linkage)}
當基因位於同一染色體上並且在染色體分離成配子時沒有發生交叉,遺傳性狀將連帶遺傳,故位於同一染色體上的許多基因,減數分裂時,不服從獨立分配率。為非孟德爾遺傳。
\subsubsection{性聯(Sex linkage)遺傳}
決定性狀的等位基因位於性染色體上。為遺傳連鎖,使雄性和雌性子代個體擁有相同表現型的機率可能不相等。為非孟德爾遺傳,但仍可以棋盤方格法推測。
\subsection{摩根果蠅眼色遺傳實驗}
決定眼色的基因座位於X染色體,紅眼為顯性記作Χ$^R$、白眼為隱性記作X$^r$。
\subsubsection{純品系紅眼雌性與白眼雄性實驗}
\begin{enumerate}
\item 親代P:純品系紅眼雌性X$^R$X$^R$與白眼雄性X$^r$Y。
\item 第一子代F$_1$:0.5機率為基因型X$^R$X$^r$紅眼雌性,0.5機率為基因型X$^R$Y紅眼雄性。
\begin{center}
\begin{tabular}{|c|c|}
\hline
& X$^R$ \\ \hline
X$^r$ & X$^R$X$^r$ \\ \hline
Y & X$^R$Y \\ \hline
\end{tabular}
\end{center}
\item 第二子代F$_1$:0.25機率為基因型X$^R$X$^R$紅眼雌性,0.25機率為基因型X$^R$X$^r$紅眼雌性,0.25機率為基因型X$^R$Y紅眼雄性,0.25機率為基因型X$^r$Y白眼雄性。
\begin{center}
\begin{tabular}{|c|c|c|}
\hline
& X$^R$ & X$^r$ \\ \hline
X$^R$ & X$^R$X$^R$ & X$^R$X$^r$ \\ \hline
Y & X$^R$Y & X$^r$Y \\ \hline
\end{tabular}
\end{center}
\end{enumerate}
\subsubsection{互交實驗}
\begin{enumerate}
\item 親代P:純品系白眼雌性X$^r$X$^r$與紅眼雄性X$^R$Y。
\item 第一子代F$_1$:0.5機率為基因型X$^R$X$^r$紅眼雌性,0.5機率為基因型X$^r$Y白眼雄性。
\begin{center}
\begin{tabular}{|c|c|}
\hline
& X$^r$ \\ \hline
X$^R$ & X$^R$X$^r$ \\ \hline
Y & X$^r$Y \\ \hline
\end{tabular}
\end{center}
\item 第二子代F$_1$:0.25機率為基因型X$^R$X$^r$紅眼雌性,0.25機率為基因型X$^r$X$^r$白眼雌性,0.25機率為基因型X$^R$Y紅眼雄性,0.25機率為基因型X$^r$Y白眼雄性。
\begin{center}
\begin{tabular}{|c|c|c|}
\hline
& X$^R$ & X$^r$ \\ \hline
X$^r$ & X$^R$X$^r$ & X$^r$X$^r$ \\ \hline
Y & X$^R$Y & X$^r$Y \\ \hline
\end{tabular}
\end{center}
\end{enumerate}
\subsection{人類ABO血型}
該基因座上有複等位基因$I^{\tx{A}}$、$I^{\tx{B}}$和$i$,其中前二者為共顯性,前二者中任一者相對於$i$為完全顯性。有$I^{\tx{A}}$者擁有抗原A,有$I^{\tx{B}}$者擁有抗原B,無$I^{\tx{A}}$者擁有抗體A,無$I^{\tx{B}}$者擁有抗體B。$I^{\tx{A}}I^{\tx{A}}$、$I^{\tx{A}}i$為A型,有抗原A和抗體B;$I^{\tx{B}}I^{\tx{B}}$、$I^{\tx{B}}i$為B型,有抗原B和抗體A,$I^{\tx{A}}I^{\tx{B}}$為AB型,有抗原A和抗原B;$ii$為O型,有抗體A和抗體B。抗原A遇到抗體A會凝集,因此擁有抗原A者不可接受擁有抗體A者之輸血,抗原B同理。
\section{遺傳變異(Genetic variation)}
\begin{itemize}
\item 定義:族群內不同個體個體之間表徵的差異,或同一物種中不同族群之間表徵的差異。
\item 來源:基因重組或突變。
\item 基因重組:舊有基因種類進行排列組和。
\item 突變:遺傳物質發生改變。
\item 基因突變:基因發生改變。例如鐮刀型血球貧血症。
\item 輻射可能促進突變,如 \tx{\textgamma} 射線會造成 DNA 斷裂、蛋白質結構改變,但不會分解核苷酸與胺基酸等小分子。
\end{itemize}
遺傳疾病:
\begin{center}
\begin{tabular}{|p{0.25\tw}|p{0.25\tw}|p{0.3\tw}|}
\hline
遺傳方式 & 遺傳疾病舉例 & 特點 \\ \hline
體染色體單基因隱性遺傳 & 白化症、鐮刀型血球貧血症 & 患者為隱性同型合子 \\ \hline
體染色體單基因顯性遺傳 & 多指症 & 正常人為隱性同型合子 \\ \hline
X染色體單基因隱性遺傳 & 紅綠色盲、蠶豆症、血友病、杜顯氏肌肉萎縮症 & 女性患者為隱性同型合子、男性患者為一個隱性,故男性患病率高於女性 \\ \hline
X染色體單基因顯性遺傳 & 低磷酸鹽佝僂症 & 男性正常人為隱性同型合子、男性正常人為一個隱性,故男性患病率低於女性 \\ \hline
\end{tabular}
\end{center}
\section{遺傳工程(Genetic engineering)/基因工程(Gene engineering)}
遺傳工程:指將新的遺傳信息插入現有細胞以便修改特定生物體以改變其特徵的過程。例如:以基因改造細菌製造胰島素取代從牛、豬胰臟提取降低成本、將生長激素基因轉殖到大西洋鮭魚上使生長快速、臺灣培育帶人類凝血因子基因的山羊以自其羊奶純化凝血因子治療血友病、基改木瓜以抗蟲害、轉殖胡蘿蔔素基因至稻米製成更營養的黃金米、基改酵母菌分解纖維素生產酒精、將巴西堅果基因轉殖到大豆以改善蛋白質但造成部分食用者過敏、轉殖入分解塑膠基因的細菌。
\begin{itemize}
\item 基因改造生物(Genetically modified organisms, GMO)/轉基因生物(Transgenic organism):指通過基因工程實現基因改變的生物。
\item 細菌質體重組DNA:先使用相同限制酶剪斷目標基因(例如人類細胞的胰島素合成基因)兩側,再用相同限制酶剪斷細菌的質體,再將目標基因與切斷成線性的質體以連接酶重新連接成環狀。
\item 限制酶(Restriction enzyme)/限制性內切核酸酶:是一種能將雙股DNA切開的酶,其切割方法是將醣類分子與磷酸之間的鍵結切斷,進而於兩條DNA鏈上各產生一個切口,且不破壞核苷酸與鹼基。切口可以是具有突出單股DNA的黏狀末端或末端平整無凸起的平滑末端。例如限制酶$Eco$RI可識別並切割GAATTC序列的DNA,形成突出單股的末端。
\item 目標基因:欲植入生物細胞的基因。
\item 基因轉殖:將重組DNA分子送入目標細胞中。
\item 載體(Vector)/運載體:為用於將一段DNA送入生物細胞內。細菌質體與病毒DNA為常用的載體,以將重組DNA送入宿主細胞。
\item 質體(Plasmid):細菌體內的小段環狀DNA,獨立於染色體DNA外,可自行進行複製,細胞分裂產生的子細胞皆有之。進行遺傳工程操作時較不易斷裂,可作為載體,且易於使用抗生素篩選質體上具有特定抗抗生素基因的個體。
\item 轉染(Transfection):指非病毒媒介地刻意將遺傳物質植入真核細胞(或專指動物)。約在1980年代成功。
\item 轉化(Transformation):指非病毒媒介地通過攝取外源遺傳物質將遺傳物質植入細菌或植物細胞。
\item 轉導(Transduction):通過病毒媒介使一個DNA片段植入到另一生物細胞中的過程。
\item 接合(Conjugation):指兩個細菌細胞直接接合或者通過類似於橋一樣的通道接合,並且發生遺傳物質交換現象,是細菌有性生殖的一個重要階段。
\item 直接法:將重組DNA直接送入欲表現的生物體內,如基因槍(Gene gun)、顯微注射。
\item 間接法:透過病毒、細菌等生物性媒介將重組DNA送入欲表現的生物體內。
\item 篩選與培養:透過抗生素等方法篩選具有重組DNA分子者並培養之。
\end{itemize}
\end{document}